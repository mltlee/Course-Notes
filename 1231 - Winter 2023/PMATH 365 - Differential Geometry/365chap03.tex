\section{Surfaces in $\R^3$}\label{sec:3}
By a surface, we always mean a smooth $2$-dimensional submanifold of $\R^3$. 
In particular, all the charts we use will be smooth. 

Let $S$ be a surface, and let $\sigma : U \subset \R^2 \to V = \sigma(U) 
\subset S \subset \R^3$ be a smooth coordinate charts with components 
\[ (u, v) \mapsto (x(u, v), y(u, v), z(u, v)). \] 
Then we know that $\sigma$ is a homeomorphism onto its image $V$, 
$\sigma$ is smooth, and $D\sigma$ has rank $2$ on $U$. This implies that 
$\sigma^{-1}$ is smooth as well. If we denote $\sigma_u 
:= \frac{\partial\sigma}{\partial u}$ and $\sigma_v := 
\frac{\partial\sigma}{\partial v}$, then the derivative matrix of $\sigma$ is 
\[ D\sigma = \left[ \begin{array}{c|c}
    \!\!\!\sigma_u & \sigma_v\!\!\!
\end{array} \right]. \] 
Since $D\sigma$ has rank $2$, we see that $\{\sigma_u, \sigma_v\}$ is 
linearly independent on $U$. Equivalently, we have that $\sigma_u \times 
\sigma_v \neq \mathbf 0$ on $U$. Therefore, if $p_0 = \sigma(u_0, v_0)$ 
for some $(u_0, v_0) \in U$, then 
\[ T_{p_0}S = \Span_{\R}\{\sigma_u(u_0, v_0), \sigma_v(u_0, v_0)\}. \] 
Moreover, we have $\sigma_u(u_0, v_0) \times \sigma_v(u_0, v_0) \perp T_{p_0}S$. 

\begin{defn}{defn:3.1}
    For all $p_0 = \sigma(u_0, v_0) \in V = \sigma(U)$, we define 
    the {\bf standard unit normal} to be 
    \[ N_\sigma(u_0, v_0) := \frac{\sigma_u(u_0, v_0) \times \sigma_v(u_0, v_0)}
    {\|\sigma_u(u_0, v_0) \times \sigma_v(u_0, v_0)\|}. \] 
\end{defn}\vspace{-0.25cm}

Since $S \subset \R^3$ and $T_{p_0}S$ is a $2$-dimensional affine subspace of 
$T_{p_0}\R^3 \simeq \R^3$, there are only two possible \emph{unit} normal 
directions to $T_{p_0}S$ at $p_0$. Therefore, $N_\sigma(u_0, v_0)$ is 
uniquely determined by $S$ up to a sign. 

For example, consider the paraboloid $S = \{(x, y, z) \in \R^3 : z = x^2 + y^2\}
\subset \R^3$. This is the graph of the smooth function $f(x, y) = x^2 + y^2$, so 
we have a smooth chart 
\begin{align*}
    \sigma : \R^2 &\to S \subset \R^3 \\ 
    (u, v) &\mapsto (u, v, u^2 + v^2). 
\end{align*}
Then $\sigma_u = (1, 0, 2u)$ and $\sigma_v = (0, 1, 2v)$, which gives 
\[ D\sigma = \begin{bmatrix}
    1 & 0 \\ 
    0 & 1 \\ 
    2u & 2v 
\end{bmatrix} \] 
and $\sigma_u \times \sigma_v = (-2u, -2v, 1) \neq \mathbf 0$. At the point
$p_0 = \sigma(u_0, v_0) = (u_0, v_0, u_0^2+v_0^2)$, we have 
\[ T_{p_0}S = \Span_{\R}\{(1, 0, 2u_0), (0, 1, 2v_0)\}. \] 
For example, with $p_0 = \mathbf 0 = (0, 0, 0) = \sigma(0, 0)$, we have 
\[ T_{\mathbf 0} S = \Span_{\R}\{(1, 0, 0), (0, 1, 0)\}, \] 
which is the $xy$-plane. Moreover, since 
\[ \|\sigma_u \times \sigma_v\| = \sqrt{4u^2 + 4v^2 + 1} \neq 0, \] 
we see that the standard unit normal is 
\[ N_\sigma = \frac{1}{\sqrt{4u^2 + 4v^2 + 1}} (-2u, -2v, 1). \] 
We have that $N_\sigma(0, 0) = (0, 0, 1)$, which is perpendicular 
to $(1, 0, 0)$ and $(0, 1, 0)$ as expected. \newpage 

What happens when we change the coordinate chart? Suppose that 
$\tilde\sigma : \tilde U \to \tilde V$ is another coordinate chart with 
$V \cap \tilde V \neq \varnothing$. Then we can write 
\[ \tilde \sigma = \sigma \circ (\sigma^{-1} \circ \tilde\sigma) = \sigma \circ \Phi, \] 
where $\Phi := \sigma^{-1} \circ \tilde\sigma : \sigma^{-1}(V \cap \tilde V) 
\subset \R^2 \to \sigma^{-1}(V \cap \tilde V) \subset \R^2$ is a smooth
diffeomorphism with components 
\[ (\tilde u, \tilde v) \mapsto (u(\tilde u, \tilde v), v(\tilde u, \tilde v)). \] 
By the chain rule, we obtain $D\tilde\sigma = D\sigma D\Phi$, or more explicitly
\[ \left[ \begin{array}{c|c}
    \!\!\!\tilde\sigma_{\tilde u} & \tilde\sigma_{\tilde v}\!\!\!
\end{array} \right] = \left[ \begin{array}{c|c}
    \!\!\!\sigma_u & \sigma_v\!\!\!
\end{array} \right] \! D\Phi. \] 
Here, $D\Phi$ is the change of basis matrix of $T_pS$. Note that 
\[ D\Phi = \begin{bmatrix}
    \partial u/\partial\tilde u & \partial u/\partial\tilde v \\
    \partial v/\partial\tilde u & \partial v/\partial\tilde v
\end{bmatrix}. \] 
Moreover, we have 
\begin{align*} 
    \tilde\sigma_{\tilde u} &= \sigma_u \cdot \frac{\partial u}{\partial\tilde u} 
    + \sigma_v \cdot \frac{\partial v}{\partial\tilde u}, \\ 
    \tilde\sigma_{\tilde v} &= \sigma_u \cdot \frac{\partial u}{\partial\tilde v} 
    + \sigma_v \cdot \frac{\partial v}{\partial\tilde v},
\end{align*} 
which implies that 
\begin{align*}
    \tilde\sigma_{\tilde u} \times \tilde\sigma_{\tilde v}
    &= \left( \sigma_u \cdot \frac{\partial u}{\partial\tilde u} 
    + \sigma_v \cdot \frac{\partial v}{\partial\tilde u} \right) 
    \times \left( \sigma_u \cdot \frac{\partial u}{\partial\tilde v} 
    + \sigma_v \cdot \frac{\partial v}{\partial\tilde v} \right) \\ 
    &= (\sigma_u \times \sigma_u) \left( \frac{\partial u}{\partial\tilde u}
    \frac{\partial u}{\partial\tilde v} \right) + (\sigma_u \times \sigma_v) 
    \left( \frac{\partial u}{\partial\tilde u} \frac{\partial v}{\partial\tilde v}
    - \frac{\partial v}{\partial\tilde u} \frac{\partial u}{\partial\tilde v} \right) 
    + (\sigma_v \times \sigma_v) \left( \frac{\partial v}{\partial\tilde u}
    \frac{\partial v}{\partial\tilde v} \right) \\ 
    &= \left( \frac{\partial u}{\partial\tilde u} \frac{\partial v}{\partial\tilde v}
    - \frac{\partial v}{\partial\tilde u} \frac{\partial u}{\partial\tilde v} \right) 
    \sigma_u \times \sigma_v = (\det D\Phi) \sigma_u \times \sigma_v,
\end{align*}
where $\sigma_u \times \sigma_u = \sigma_v \times \sigma_v = \mathbf 0$. 
It follows that 
\[ N_{\tilde\sigma} = \frac{\tilde\sigma_{\tilde u} \times \tilde\sigma_{\tilde v}}
{\|\tilde\sigma_{\tilde u} \times \tilde\sigma_{\tilde v}\|}
= \frac{\det D\Phi}{\lvert\det D\Phi\rvert} \cdot \frac{\sigma_u \times \sigma_v}
{\|\sigma_u \times \sigma_v\|} = \pm N_\sigma, \] 
where $\pm$ is given by the sign of $\det D\Phi$. Therefore, changing the 
coordinate chart only changes the standard unit normal up to a sign. 

\subsection{First fundamental form} \label{subsec:3.1}
Let $p_0 \in \sigma(U) \subset S$. If $p_0 = \sigma(u_0, v_0)$, then 
\[ T_{p_0}S = \Span_{\R}\{\sigma_u(u_0, v_0), \sigma_v(u_0, v_0)\} \] 
is a $2$-dimensional subspace of $T_{p_0}\R^3 \simeq \R^3$. Suppose that 
the inner product on $T_{p_0}\R^3$ is the usual dot product. That is, 
for $X, Y \in T_{p_0}\R^3 \simeq \R^3$, we have 
\[ X \cdot Y := X^T Y = X^TIY, \] 
where the latter is the matrix representation with respect to the standard basis.

What if we restrict the dot product to $T_{p_0}S$? Since $T_{p_0}S$ is 
$2$-dimensional, the restriction of the dot product can be represented 
by a $2\times 2$ matrix with respect to a basis of $T_{p_0}S$. 

Using the basis ${\cal B} = \{\sigma_u(u_0, v_0), \sigma_v(u_0, v_0)\}$, 
then for all $X, Y \in T_{p_0}S$, we have 
\begin{align*}
    X &= a\sigma_u(u_0, v_0) + b\sigma_v(u_0, v_0), \\ 
    Y &= c\sigma_u(u_0, v_0) + d\sigma_v(u_0, v_0)
\end{align*}
for some $a, b, c, d \in \R$. More concisely, we have 
\begin{align*}
    X &= D\sigma(u_0, v_0) \begin{bmatrix}
        a \\ b 
    \end{bmatrix}, \\
    Y &= D\sigma(u_0, v_0) \begin{bmatrix}
        c \\ d  
    \end{bmatrix},
\end{align*}
which gives us 
\begin{align*}
    X \cdot Y = X^T Y 
    &= \left( D\sigma(u_0, v_0) \begin{bmatrix}
        a \\ b 
    \end{bmatrix} \right)^{\!T} \left( D\sigma(u_0, v_0) \begin{bmatrix}
        c \\ d  
    \end{bmatrix} \right) \\ 
    &= \begin{bmatrix}
        a & b 
    \end{bmatrix} D\sigma(u_0, v_0)^T D\sigma(u_0, v_0) \begin{bmatrix}
        c \\ d
    \end{bmatrix} \\ 
    &= \begin{bmatrix}
        a & b 
    \end{bmatrix} \begin{bmatrix} 
        \sigma_u \cdot \sigma_u & \sigma_u \cdot \sigma_v \\ 
        \sigma_v \cdot \sigma_u & \sigma_v \cdot \sigma_v 
    \end{bmatrix} \begin{bmatrix}
        c \\ d
    \end{bmatrix}.
\end{align*}
The matrix $D\sigma(u_0, v_0)^T D\sigma(u_0, v_0)$ above is our matrix representation 
of the dot product with respect to the basis ${\cal B} = \{\sigma_u(u_0, v_0), 
\sigma_v(u_0, v_0)\}$. 

\begin{defn}{defn:3.2}
    The {\bf first fundamental form of $\sigma$} is defined to be 
    \[ \FFF = D\sigma(u_0, v_0)^T D\sigma(u_0, v_0) 
    = \begin{bmatrix} 
        \sigma_u \cdot \sigma_u & \sigma_u \cdot \sigma_v \\ 
        \sigma_v \cdot \sigma_u & \sigma_v \cdot \sigma_v 
    \end{bmatrix}. \] 
\end{defn}\vspace{-0.25cm}

Note that $\FFF$ is symmetric since $\sigma_u \cdot \sigma_v = \sigma_v 
\cdot \sigma_u$. Moreover, if $\tilde\sigma = \sigma \circ \Phi$ is another 
chart, then 
\[ D\tilde\sigma = D\sigma D\Phi \] 
where $D\Phi$ is a change of basis matrix, with 
\begin{align*}
    \tilde{\FFF} &= (D\tilde\sigma)^T D\tilde\sigma \\ 
    &= (D\sigma D\Phi)^T (D\sigma D\Phi) \\ 
    &= D\Phi^T (D\sigma^T D\sigma) D\Phi \\ 
    &= D\Phi^T \FFF D\Phi.
\end{align*}
Therefore, $\tilde{\FFF}$ is similar to $\FFF$, and the first fundamental 
form depends on the coordinate chart $\sigma$. 

Let's compute some examples. 
\begin{enumerate}[(1)]
    \item {\bf (Plane.)} Let $\sigma(u, v) = p_0 + uw_1 + vw_2$ for $u, v \in \R$, 
    where $p_0$ is a point and $w_1$ and $w_2$ are direction vectors. We can choose 
    $w_1$ and $w_2$ such that $\{w_1, w_2\}$ is orthonormal. Then 
    $\sigma_u = w_1$ and $\sigma_v = w_2$, so 
    \[ \FFF = \begin{bmatrix}
        w_1 \cdot w_1 & w_1 \cdot w_2 \\ 
        w_2 \cdot w_1 & w_2 \cdot w_2 
    \end{bmatrix} = \begin{bmatrix}
        1 & 0 \\ 0 & 1
    \end{bmatrix}. \] 

    \item {\bf (Cylinder.)} Consider the cylinder $x^2 + y^2 = 1$ with 
    coordinate chart $\tilde\sigma(u, v) = (\cos u, \sin u, v)$ where 
    $(u, v) \in (0, 2\pi) \times \R$. We have $\tilde\sigma_u 
    = (-\sin u, \cos u, 0)$ and $\tilde\sigma_v = (0, 0, 1)$, which yields 
    $\sigma_u \cdot \sigma_u = \sigma_v \cdot \sigma_v = 1$ 
    and $\sigma_u \cdot \sigma_v = 0$. In particular, we have that 
    \[ \tilde{\FFF} = \begin{bmatrix}
        1 & 0 \\ 0 & 1
    \end{bmatrix}. \] 
    This is the same first fundamental form as the plane! Here, 
    we see that the first fundamental form alone cannot distinguish 
    between different shapes. 

    {\bf Remark.} This is not too surprising since the cylinder can be obtained 
    from the plane by folding it smoothly (without stretching or shrinking) 
    so that length and distances are preserved. The first fundamental 
    form is an inner product that is the restriction of the dot product, 
    which is used to measure lengths and distances. 

    \item {\bf (Sphere.)} Consider the sphere $x^2 + y^2 + z^2 = a^2$ 
    for some $a > 0$ and the coordinate chart 
    \[ \sigma(\theta, \varphi) = (a\cos\theta\sin\varphi, a\sin\theta\sin\varphi, a\cos\varphi) \] 
    for $(\theta, \varphi) \in (0, 2\pi) \times (0, \pi)$. We have 
    \begin{align*}
        \sigma_\theta &= (-a\sin\theta\sin\varphi, a\cos\theta\sin\varphi, 0), \\ 
        \sigma_\varphi &= (a\cos\theta\cos\varphi, a\sin\theta\cos\varphi, -a\sin\varphi),
    \end{align*}
    which gives us 
    \begin{align*}
        \sigma_\theta \cdot \sigma_\theta &= a^2\sin^2\varphi, \\ 
        \sigma_\theta \cdot \sigma_\varphi &= 0, \\ 
        \sigma_\varphi \cdot \sigma_\varphi &= a^2.
    \end{align*}
    Therefore, the first fundamental form of $\sigma$ is 
    \[ \FFF = \begin{bmatrix}
        a^2\sin^2\varphi & 0 \\ 0 & a^2
    \end{bmatrix}, \] 
    which is different from the first fundamental form of the plane. 

    {\bf Remark.} The sphere can be obtained from the plane and adding a 
    point at $\infty$. 
\end{enumerate}

\subsection{Second fundamental form} \label{subsec:3.2}
In the above examples, we saw that the first fundamental form $\FFF$ 
is not enough to distinguish surfaces and capture curvature since 
seemingly different surfaces such as planes and cylinders can 
have the same first fundamental form. Therefore, we will introduce a 
``second fundamental form''.

Let $w_1, w_2 \in \R^3$. Recall that the {\bf scalar projection of 
$w_1$ onto $w_2$} is 
\[ \ell = \frac{w_1 \cdot w_2}{\|w_2\|} \] 
because if $\theta$ is the angle between $w_1$ and $w_2$, then 
$\cos\theta = \ell/\|w_1\|$ so that 
\[ \ell = \cos\theta \|w_1\| = \frac{w_1 \cdot w_2}{\|w_1\|\|w_2\|}\|w_1\| 
= \frac{w_1 \cdot w_2}{\|w_2\|}. \] 
In particular, if $\|w_2\| = 1$, then $\ell = w_1 \cdot w_2$. 

Define the vector $\Delta \sigma = \sigma(u_0 + \Delta u, v_0 + \Delta v) 
- \sigma(u_0, v_0) \in \R^3$. Then $\Delta\sigma \cdot N_\sigma$ is 
the scalar projection of $\Delta\sigma$ onto $N_\sigma$ (since 
$N_\sigma$ is unit), and we call it the {\bf deviation}. This 
measures how much $S$ moves away from the tangent plane $T_{p_0}S$ 
to $S$ at $p_0 = \sigma(u_0, v_0)$. 

By Taylor's Theorem, we have 
\begin{align*}
    \Delta\sigma 
    &= \sigma(u_0 + \Delta u, v_0 + \Delta v) \\
    &= \sigma_u \Delta u + \sigma_v \Delta v 
    + \frac12 (\sigma_{uu} \Delta u^2 + 2\sigma_{uv} \Delta u \Delta v 
    + \sigma_{vv} \Delta v^2) + \text{remainder}, 
\end{align*}
where we used the fact that $\sigma_{uv} = \sigma_{vu}$ since $\sigma$ is smooth.
Since $T_pS \perp N_\sigma$ and $T_pS = \Span_{\R}\{\sigma_u, \sigma_v\}$, we have 
$\sigma_u \cdot N_\sigma = \sigma_v \cdot N_\sigma = 0$. This yields 
\begin{align*}
    \Delta\sigma \cdot N_\sigma 
    &= (\sigma_u \cdot N_\sigma)\Delta u + (\sigma_v \cdot N_\sigma) \Delta v 
    + \frac12 [(\sigma_{uu} \cdot N_\sigma) \Delta u^2 + 2(\sigma_{uv} \cdot N_\sigma) 
    \Delta u \Delta v + (\sigma_{vv} \cdot N_\sigma)\Delta v^2] + \cdots
\end{align*}
and so the deviation is well approximated with 
\begin{align*} 
    \Delta\sigma \cdot N_\sigma 
    &\approx \frac12 [(\sigma_{uu} \cdot N_\sigma) 
    \Delta u^2 + 2(\sigma_{uv} \cdot N_\sigma) \Delta u \Delta v + (\sigma_{vv} 
    \cdot N_\sigma)\Delta v^2] \\
    &= \frac12\begin{bmatrix} \Delta u & \Delta v \end{bmatrix} 
    \begin{bmatrix}
        \sigma_{uu} \cdot N_\sigma & \sigma_{uv} \cdot N_\sigma \\ 
        \sigma_{vu} \cdot N_\sigma & \sigma_{vv} \cdot N_\sigma
    \end{bmatrix} \begin{bmatrix}
        \Delta u \\ \Delta v
    \end{bmatrix}.
\end{align*} 
This matrix is reminiscent of the Hessian from multivariable calculus!

\begin{defn}{defn:3.3}
    We define the {\bf second fundamental form of $\sigma$} to be 
    \[ \SFF := \begin{bmatrix}
        \sigma_{uu} \cdot N_\sigma & \sigma_{uv} \cdot N_\sigma \\ 
        \sigma_{vu} \cdot N_\sigma & \sigma_{vv} \cdot N_\sigma
    \end{bmatrix}. \]
\end{defn}\vspace{-0.25cm}

Note that $\sigma$ is smooth, so $\sigma_{uv} = \sigma_{vu}$ and 
$\SFF$ is symmetric. We look at some examples. 

\begin{enumerate}[(1)]
    \item {\bf (Plane.)} As before, let $\sigma(u, v) = p_0 + uw_1 + vw_2$ 
    for $u, v \in \R$ where $p_0$ is a point and $\{w_1, w_2\}$ is orthonormal.
    We have $\sigma_u = w_1$ and $\sigma_v = w_2$. Then 
    $\sigma_{uu} = \sigma_{uv} = \sigma_{vv} = 0$ so that 
    \[ \SFF = \begin{bmatrix}
        0 & 0 \\ 0 & 0
    \end{bmatrix}. \] 

    \item {\bf (Cylinder.)} Let $\tilde\sigma(u, v) = (\cos u, \sin u, v)$ for 
    $u, v \in \R$, and note that $\tilde\sigma_u = (-\sin u, \cos u, 0)$ 
    and $\tilde\sigma_v = (0, 0, 1)$. Then $\tilde\sigma_u \times 
    \tilde\sigma_v = (\cos u, \sin u, 0)$ with $\|\tilde\sigma_u \times 
    \tilde\sigma_v\| = 1$, so we have $N_{\tilde\sigma} = (\cos u, \sin u, 0)$.
    Moreover, we have $\tilde\sigma_{uu} = (-\cos u, -\sin u, 0)$ and 
    $\tilde\sigma_{uv} = \tilde\sigma_{vv} = 0$, which implies that 
    $\sigma_{uu} \cdot N_\sigma = -\cos^2 u - \sin^2 u = -1$ and the second 
    fundamental form of $\tilde\sigma$ is 
    \[ \tilde{\SFF} = \begin{bmatrix}
        -1 & 0 \\ 0 & 0
    \end{bmatrix}. \] 
    We see that this is different from the second fundamental form of the plane.

    \item {\bf (Sphere.)} Consider the sphere $x^2 + y^2 + z^2 = a^2$ where $a > 0$ 
    with coordinate chart 
    \[ \sigma(\theta, \varphi) = (a\cos\theta\sin\varphi, a\sin\theta\sin\varphi, a\cos\varphi) \] 
    for $(\theta, \varphi) \in (0, 2\pi) \times (0, \pi)$. A direct 
    computation shows that 
    \[ \SFF = \begin{bmatrix}
        a\sin^2\varphi & 0 \\ 0 & a
    \end{bmatrix}. \] 

    \item {\bf (Graph of a smooth function.)} Let $f : U \subset \R^2 \to \R$ 
    be smooth where $U \subset \R^2$ is open. The graph 
    \[ S = \{(x, y, z) \in \R^3 : z = f(x, y)\} \subset \R^3 \] 
    of $f$ is a surface with smooth coordinate chart 
    \begin{align*}
        \sigma : U \subset \R^2 &\to S \subset \R^3 \\ 
        (u, v) &\mapsto (u, v, f(u, v)). 
    \end{align*}
    We have $\sigma_u = (1, 0, f_u)$ and $\sigma_v = (0, 1, f_v)$ 
    where $f_u = \frac{\partial f}{\partial u}$ and $f_v = 
    \frac{\partial f}{\partial v}$, which implies that 
    \[ \FFF = \begin{bmatrix}
        1 + (f_u)^2 & f_u f_v \\ 
        f_v f_u & 1 + (f_v)^2
    \end{bmatrix}. \] 
    Next, we have $\sigma_u \times \sigma_v = (-f_u, -f_v, 1)$, so 
    \[ N_\sigma = \frac{1}{\sqrt{1+(f_u)^2 + (f_v)^2}}(-f_u, -f_v, 1). \] 
    Moreover, we know that $\sigma_{uu} = (0, 0, f_{uu})$, 
    $\sigma_{uv} = (0, 0, f_{uv})$, and $\sigma_{vv} = (0, 0, f_{vv})$, so 
    \[ \SFF = \frac{1}{\sqrt{1+(f_u)^2+(f_v)^2}} \begin{bmatrix}
        f_{uu} & f_{uv} \\ f_{vu} & f_{vv}
    \end{bmatrix}. \]
    In particular, the above matrix is exactly the Hessian $H(f)$ of $f$, 
    which appears in the second derivative test. At a critical point of 
    $f(x, y)$, we have $f_u = f_v = 0$ so that $\FFF = I_{2\times 2}$ 
    and $\SFF = H(f)$. 
\end{enumerate}

\subsection{The shape operator} \label{subsec:3.3}
We begin with the following observation. 

\begin{lemma}{lemma:3.4}
    We have $\det\FFF = \|\sigma_u \times \sigma_v\|^2$.
\end{lemma}\vspace{-0.25cm} 
\begin{pf}[Lemma~\ref{lemma:3.4}]
    This follows directly from the cross product identity 
    \[ (a \times b) \cdot (c \times d) = (a \cdot c)(b \cdot d) - 
    (a \cdot d)(b \cdot c) \] 
    for all $a, b, c, d \in \R^3$. We have that 
    \begin{align*}
        \|\sigma_u \cdot \sigma_v\|^2 
        &= (\sigma_u \times \sigma_v) \cdot (\sigma_u \times \sigma_v) \\ 
        &= (\sigma_u \cdot \sigma_u)(\sigma_v \cdot \sigma_v) 
        - (\sigma_u \cdot \sigma_v)(\sigma_v \cdot \sigma_u) \\ 
        &= \det\FFF. \tag*{\qed} 
    \end{align*}
\end{pf}\vspace{-0.25cm}

Since $\sigma_u \times \sigma_v \neq \mathbf 0$ everywhere for any 
coordinate chart $\sigma$, this means that $\FFF$ is always invertible. 

\begin{defn}{defn:3.5}
    The {\bf shape operator} (or {\bf Weingarten matrix}) is defined to be 
    \[ {\cal W} := {\cal F}_{\text{\RN{1}}}^{-1} \SFF. \] 
    The {\bf Gaussian curvature} is $K := \det {\cal W}$, and the {\bf mean curvature} 
    is $H := \frac12 \tr {\cal W}$. 

    Let $p_0 = \sigma(u_0, v_0)$ for some $(u_0, v_0) \in U$. Then: 
    \begin{enumerate}[(i)]
        \item $p_0$ is {\bf elliptic} if $K > 0$. 
        \item $p_0$ is {\bf hyperbolic} if $K < 0$. 
        \item $p_0$ is {\bf parabolic} if $K = 0$ and $H \neq 0$.
        \item $p_0$ is {\bf planar} if $K = H = 0$. 
    \end{enumerate}
\end{defn}\vspace{-0.25cm}

We compute some examples of the shape operator. 
\begin{enumerate}[(1)]
    \item {\bf (Plane.)} Let $\sigma(u, v) = p_0 + uw_1 + vw_2$ where 
    $\{w_1, w_2\}$ is orthonormal and $p_0$ is a point. We found that 
    \begin{align*}
        \FFF &= \begin{bmatrix}
            1 & 0 \\ 0 & 1 
        \end{bmatrix}, \qquad \SFF = \begin{bmatrix}
            0 & 0 \\ 0 & 0 
        \end{bmatrix},
    \end{align*}
    which implies that the shape operator is 
    \[ {\cal W} = {\cal F}_{\text{\RN{1}}}^{-1} \SFF = \begin{bmatrix}
        0 & 0 \\ 0 & 0 
    \end{bmatrix}. \] 
    We have that $K = \det{\cal W} = 0$ and $H = \frac12\tr{\cal W} = 0$ 
    everywhere, so every point on the plane is planar!

    \item {\bf (Cylinder.)} For the cylinder $x^2 + y^2 = 1$ with 
    smooth coordinate chart $\sigma(u, v) = (\cos u, \sin u, v)$ 
    where $(u, v) \in (0, 2\pi) \times \R$, we saw that 
    \begin{align*}
        \FFF &= \begin{bmatrix}
            1 & 0 \\ 0 & 1 
        \end{bmatrix}, \qquad \SFF = \begin{bmatrix}
            -1 & 0 \\ 0 & 0 
        \end{bmatrix}.
    \end{align*}
    Then the shape operator is given by 
    \[ {\cal W} = {\cal F}_{\text{\RN{1}}}^{-1} \SFF = \begin{bmatrix}
        -1 & 0 \\ 0 & 0 
    \end{bmatrix}, \] 
    so $K = \det{\cal W} = 0$ and $H = \frac12\tr{\cal W} = -\frac12 \neq 0$. 
    Thus, every point on the cylinder is parabolic. 

    The missing points on the cylinder are covered by the 
    chart $\sigma(u, v) = (\cos u, \sin u, v)$ over $(u, v) \in (-\pi, \pi)
    \times \R$, which has the same first and second fundamental forms 
    and thus the same shape operator. 

    {\bf Remark.} Cylinders locally look like parabolic cylinders. For example, 
    the set 
    \[ S = \{(x, y, z) \in \R^3 : z = x^2\} \subset \R^3 \] 
    is a parabolic cylinder. Since $S$ is the graph of the smooth 
    scalar function $f : \R^2 \to \R$ defined by $(x, y) \mapsto x^2$, 
    we see that $S$ is a surface which can be covered by the coordinate 
    chart $\sigma(u, v) = (u, v, u^2)$ for $(u, v) \in \R^2$. Then 
    $\sigma_u = (1, 0, 2u)$ and $\sigma_v = (0, 1, 0)$ so that 
    \[ \FFF = \begin{bmatrix}
        1+4u^2 & 0 \\ 0 & 1
    \end{bmatrix}. \]
    Moreover, we have $\sigma_{uu} = (0, 0, 2)$, $\sigma_{uv} = \sigma_{vv} 
    = (0, 0, 0)$ and 
    \[ N_\sigma = \frac{\sigma_u \times \sigma_v}{\|\sigma_u \times \sigma_v\|} 
    = \frac{1}{\sqrt{1+4u^2}}(-2u, 0, 1), \] 
    so the second fundamental form is 
    \[ \SFF = \begin{bmatrix}
        2/\sqrt{1+4u^2} & 0 \\ 0 & 0 
    \end{bmatrix}. \] 
    It follows that the shape operator is 
    \[ {\cal W} = {\cal F}_{\text{\RN{1}}}^{-1} \SFF 
    = \begin{bmatrix}
        1/(1+4u^2) & 0 \\ 0 & 0 
    \end{bmatrix} \begin{bmatrix}
        2/\sqrt{1+4u^2} & 0 \\ 0 & 0 
    \end{bmatrix} = \begin{bmatrix}
        2/(1+4u^2)^{3/2} & 0 \\ 0 & 0 
    \end{bmatrix}. \] 
    We obtain $K = \det{\cal W} = 0$ and 
    $H = \frac12\tr{\cal W} = 1/(1+4u^2)^{3/2} \neq 0$, so every point on $S$ 
    is parabolic.

    \item {\bf (Sphere.)} Consider the sphere $x^2 + y^2 + z^2 = a^2$ where $a > 0$ with 
    smooth coordinate chart 
    \[ \sigma(\theta, \varphi) = (a\cos\theta\sin\varphi, a\sin\theta\sin\varphi, a\cos\varphi) \] 
    for $(\theta, \varphi) \in (0, 2\pi) \times (-\pi, \pi)$. We have seen that 
    \begin{align*}
        \FFF &= \begin{bmatrix}
            a^2\sin^2\varphi & 0 \\ 0 & a^2
        \end{bmatrix}, \qquad \SFF = \begin{bmatrix}
            a\sin^2\varphi & 0 \\ 0 & a 
        \end{bmatrix}.
    \end{align*}
    Then the shape operator is given by 
    \[ {\cal W} = {\cal F}_{\text{\RN{1}}}^{-1} \SFF = \begin{bmatrix}
        1/a & 0 \\ 0 & 1/a
    \end{bmatrix}, \]
    so $K = \det{\cal W} = 1/a^2$ and $H = \frac12\tr{\cal W} = 1/a$ 
    everywhere. Thus, every point on the sphere is elliptic.

    {\bf Remark.} By using different spherical coordinates, we can cover the 
    entire sphere. In each case, we have that $K = 1/a^2$ and $H = \pm1/a$. 

    \item {\bf (Saddle surface.)} Consider the surface $z = x^2 - y^2$ where 
    $x, y \in \R$. This is the graph of the smooth function $f(x, y) = x^2 - y^2$, 
    so we obtain the smooth coordinate chart $\sigma(u, v) = (u, v, u^2 - v^2)$ 
    where $(u, v) \in \R^2$. A direct computation shows that 
    \begin{align*}
        \FFF &= \begin{bmatrix}
            1+4u^2 & -4uv \\ -4uv & 1+4v^2
        \end{bmatrix}, \qquad \SFF = \frac{1}{\sqrt{1+4u^2+4v^2}} \begin{bmatrix}
            2 & 0 \\ 0 & -2
        \end{bmatrix}.
    \end{align*}
    Note that to find the Gaussian curvature $K = \det{\cal W}$, we need not compute ${\cal W}$ 
    explicitly. Recalling that the determinant is multiplicative and 
    $\det(A^{-1}) = (\det A)^{-1}$ for a matrix $A$, we have 
    \[ K = \det{\cal W} = \det({\cal F}_{\text{\RN{1}}}^{-1} \SFF) 
    = \det({\cal F}_{\text{\RN{1}}}^{-1}) \det\SFF = (\det \FFF)^{-1} \det\SFF. \]
    We see that $\det\FFF = (1+4u^2)(1+4v^2) - (-4uv)^2 = 1 + 4u^2 + 4v^2$ and 
    \[ \det\SFF = \frac{1}{(\sqrt{1+4u^2+4v^2})^2} (2(-2) - 0^2) = 
    -\frac{4}{1+4u^2+4v^2}, \] 
    which implies that the Gaussian curvature is 
    \[ K = -\frac{4}{(1+4u^2+4v^2)^2} < 0 \] 
    for all $(u, v) \in \R^2$. Therefore, every point on the saddle surface is hyperbolic.
    
\end{enumerate}

Let's describe a few more properties of the shape operator. First off, 
note that although $\FFF$ and $\SFF$ are always symmetric, it is possible 
that ${\cal W}$ is not symmetric. 

For example, consider the saddle surface $z = xy$ parametrized by 
$\sigma(u, v) = (u, v, uv)$ where $u, v \in \R$. A direct computation then gives 
\begin{align*}
    \FFF &= \begin{bmatrix}
        1+v^2 & uv \\ uv & 1+u^2
    \end{bmatrix}, \qquad \SFF = \frac{1}{(1+u^2+v^2)^{3/2}} \begin{bmatrix}
        0 & 1 \\ 1 & 0 
    \end{bmatrix}
\end{align*}
so that the shape operator is 
\[ {\cal W} = {\cal F}_{\text{\RN{1}}}^{-1} \SFF = \frac{1}{(1+u^2+v^2)^{3/2}} \begin{bmatrix}
    uv & 1+v^2 \\ 1+u^2 & uv 
\end{bmatrix}. \] 
In particular, we have $1+v^2 \neq 1+u^2$ away from points where $u = \pm v$, 
so ${\cal W}$ is not symmetric.

Nonetheless, we have the following result. 

\begin{prop}{prop:3.6}
    The shape operator ${\cal W}$ is a diagonalizable matrix with real eigenvalues.
\end{prop}\vspace{-0.25cm}
\begin{pf}[Proposition~\ref{prop:3.6}]
    Let $w_1$ and $w_2$ be two orthogonal unit vectors in $T_pS$ so that 
    $w_1 \cdot w_1 = w_2 \cdot w_2 = 1$ and $w_1 \cdot w_2 = 0$. Since 
    $w_1, w_2 \in T_pS = \Span_{\R}\{\sigma_u, \sigma_v\}$, we can write 
    \begin{align*}
        w_1 &= a_1\sigma_u + b_1\sigma_v, \\ 
        w_2 &= a_2\sigma_u + b_2\sigma_v 
    \end{align*}
    for some $a_1, a_2, b_1, b_2 \in \R$. For $i, j \in \{1, 2\}$, observe that 
    \[ w_i \cdot w_j = \begin{bmatrix}
        a_i & b_i
    \end{bmatrix} \FFF \begin{bmatrix}
        a_j \\ b_j 
    \end{bmatrix} = \delta_{ij}, \] 
    where $\delta_{ij}$ is the Kronecker delta. Setting 
    \[ C := \left[ \begin{array}{c|c} \!\!\! w_1 & w_2 \!\!\! \end{array} \right] 
    = \begin{bmatrix}
        a_1 & a_2 \\ b_1 & b_2 
    \end{bmatrix}, \] 
    we have that 
    \[ C^T \FFF C = \begin{bmatrix}
        a_1 & b_1 \\ a_2 & b_2 
    \end{bmatrix} \FFF \begin{bmatrix}
        a_1 & a_2 \\ b_1 & b_2 
    \end{bmatrix} = \begin{bmatrix}
        w_1 \cdot w_1 & w_1 \cdot w_2 \\ 
        w_2 \cdot w_1 & w_2 \cdot w_2 
    \end{bmatrix} = \begin{bmatrix}
        1 & 0 \\ 0 & 1 
    \end{bmatrix} = I_{2\times 2}. \] 
    Note that the columns of $C$ are linearly independent, so $C$ is invertible. 
    Therefore, we obtain 
    \[ C^{-1} {\cal F}_{\text{\RN{1}}}^{-1} (C^T)^{-1} = (C^T \FFF C)^{-1} = (I_{2\times 2})^{-1} = 
    I_{2\times 2}, \] 
    which implies that 
    \begin{align*}
        C^{-1}{\cal W}C &= C^{-1}({\cal F}_{\text{\RN{1}}}^{-1} \SFF)C \\
        &= C^{-1}{\cal F}_{\text{\RN{1}}}^{-1} (C^T)^{-1} C^T \SFF C \\ 
        &= C^T \SFF C =: B. 
    \end{align*}
    This means that ${\cal W}$ is similar to $B = C^T \SFF C$. But $B$ 
    is symmetric since 
    \[ B^T = (C^T \SFF C)^T = C^T {\cal F}_{\text{\RN{2}}}^T (C^T)^T = C^T \SFF C = B. \] 
    Then $B$ is diagonalizable with real eigenvalues. Since ${\cal W}$ is 
    similar to $B$, it follows that ${\cal W}$ is also diagonalizable with 
    real eigenvalues. \qed 
\end{pf}\vspace{-0.25cm}

Due to Proposition~\ref{prop:3.6}, it makes sense to define the following. 

\begin{defn}{defn:3.7}
    The eigenvalues $\kappa_1$ and $\kappa_2$ of ${\cal W}$ 
    are called the {\bf principal curvatures} of the coordinate chart $\sigma$. 
\end{defn}\vspace{-0.25cm} 

Observe that we have $K = \det{\cal W} = \kappa_1\kappa_2$ and 
$H = \frac12\tr{\cal W} = \frac12(\kappa_1 + \kappa_2)$. 

Therefore, given the principal curvatures $\kappa_1$ and $\kappa_2$ 
and a point $p_0 = \sigma(u_0, v_0)$, we see that: 
\begin{enumerate}[(i)]
    \item $p_0$ is elliptic when $K > 0$, which happens 
    if and only if $\kappa_1$ and $\kappa_2$ are nonzero with the same sign. 
    \item $p_0$ is hyperbolic when $K < 0$, which happens if and only if 
    $\kappa_1$ and $\kappa_2$ are nonzero with opposite signs. 
    \item $p_0$ is parabolic when $K = 0$ and $H \neq 0$, which 
    happens if and only if $\kappa_1 = 0$ and $\kappa_2 \neq 0$ (or vice versa). 
    \item $p_0$ is planar when $K = H = 0$, which happens if and only if 
    $\kappa_1 = \kappa_2 = 0$. 
\end{enumerate}

{\bf Remark.} We know that $S$ is locally the graph of a function. After 
possibly rotating and translating $S$, we can assume that it is the graph 
of a function $f(x, y)$ near $p_0$ and that $p_0$ corresponds to a critical 
point of $f(x, y)$. We have seen that at $p_0$, we have 
$\FFF = I_{2\times 2}$ and $\SFF = H(f)$ so that ${\cal W} = H(f)$. 
We can now apply the second derivative test. 
\begin{itemize}
    \item If $\det H(f) > 0$ and $f_{xx} > 0$, then $p_0$ is a local minimum. 
    \item If $\det H(f) > 0$ and $f_{xx} < 0$, then $p_0$ is a local maximum. 
    \item If $\det H(f) < 0$, then $p_0$ is a saddle point. 
    \item If $\det H(f) = 0$, then the test is inconclusive. 
\end{itemize}
The first two situations correspond to elliptic points. The third 
corresponds to hyperbolic points, and the fourth corresponds to parabolic or 
planar points.

\subsection{Normal and geodesic curvatures} \label{subsec:3.4}
We now study in more detail the curvature of regular curves 
or surfaces. Since any regular curve can be reparametrized using arclength 
to be unit speed, we will assume throughout that the curves are unit speed. 

Let $\gamma : (\alpha, \beta) \to \sigma(U) = V \subset S$ be a unit speed 
curve on $S$ included in the coordinate patch $\sigma : U \subset \R^2 
\to V \subset S \subset \R^3$. Let $p_0 = \gamma(t_0)$ be a point on the 
curve. Since $\gamma$ is unit speed, we know by Proposition~\ref{prop:2.2} 
that $\gamma'(t_0) \cdot \gamma''(t_0) = 0$. Under the assumption 
$\gamma''(t_0) \neq \mathbf 0$, this means that 
\[ \gamma'(t_0) \perp \gamma''(t_0). \]
Therefore, we will assume that $\gamma''(t_0) \neq \mathbf 0$. 
Since $\gamma'(t_0) \in T_{p_0}S$, we have that 
\[ \gamma'(t_0) \perp N_\sigma \] 
since $N_\sigma \perp T_{p_0}S$, and by the definition of cross product, we obtain 
\[ \gamma'(t_0) \perp N_\sigma \times \gamma'(t_0). \] 
In particular, we see that $\gamma''(t_0) \in \Span_{\R}\{N_\sigma, N_\sigma 
\times \gamma'(t_0)\}$. We can write 
\[ \gamma''(t_0) = \kappa_n N_\sigma + \kappa_g (N_\sigma \times \gamma'(t_0)) \] 
for some $\kappa_n, \kappa_g \in \R$. But $\|\gamma'(t_0)\| = \|N_\sigma\| = 1$,
so $\|N_\sigma \times \gamma'(t_0)\| = 1$. Since 
$N_\sigma \cdot (N_\sigma \times \gamma'(t_0)) = 0$,
we have 
\begin{align*}
    \gamma''(t_0) \cdot N_\sigma 
    &= (\kappa_n N_\sigma + \kappa_g (N_\sigma \times \gamma'(t_0))) \cdot N_\sigma \\ 
    &= \kappa_n N_\sigma \cdot N_\sigma + \kappa_g (N_\sigma \times \gamma'(t_0)) \cdot N_\sigma \\ 
    &= \kappa_n \cdot 1 + \kappa_g \cdot 0 = \kappa_n. 
\end{align*}
By a similar computation, we have that 
\[ \gamma''(t_0) \cdot (N_\sigma \times \gamma''(t_0)) = \kappa_g. \] 
Since $\|N_\sigma\| = \|N_\sigma \times \gamma'(t_0)\| = 1$, recall from 
Section~\ref{subsec:3.2} that $\gamma''(t_0) \cdot N_\sigma$ is the scalar 
projection of $\gamma''(t_0)$ onto $N_\sigma$ and $\gamma''(t_0) \cdot 
(N_\sigma \times \gamma'(t_0))$ is the scalar projection of $\gamma''(t_0)$ 
onto $(N_\sigma \times \gamma'(t_0))$. 

\begin{defn}{defn:3.8}
    The {\bf normal curvature} is defined to be 
    \[ \kappa_n := \gamma''(t_0) \cdot N_\sigma. \] 
    The {\bf geodesic curvature} is defined to be 
    \[ \kappa_g := \gamma''(t_0) \cdot (N_\sigma \times \gamma'(t_0)). \] 
    We call $\kappa_n N_\sigma$ the {\bf normal component of $\gamma''(t_0)$}.
\end{defn}\vspace{-0.25cm}

Note that we can have $\kappa_n = 0$ or $\kappa_g = 0$, but we cannot 
have $\kappa_n = \kappa_g = 0$ since $\gamma''(t_0) \neq \mathbf 0$. 
Moreover, depending on where $\gamma''(t_0)$ lies in the plane 
$\Span_{\R}\{N_\sigma, N_\sigma \times \gamma'(t_0)\}$, we may have 
$\kappa_n < 0$ or $\kappa_g < 0$. 

Since $\gamma''(t_0) \neq \mathbf 0$, we see that 
$\kappa = \|\gamma''(t_0)\| > 0$,
where $\kappa$ is the usual curvature of $\gamma$. Note that 
\begin{align*}
    \kappa^2 = \|\gamma''(t_0)\|^2 
    &= \gamma''(t_0) \cdot \gamma''(t_0) \\
    &= (\kappa_n N_\sigma + \kappa_g(N_\sigma \times \gamma'(t_0))) 
    \cdot (\kappa_n N_\sigma + \kappa_g(N_\sigma \times \gamma'(t_0))) \\
    &= \kappa_n^2 N_\sigma \cdot N_\sigma + 2\kappa_n \kappa_g 
    (N_\sigma \cdot (N_\sigma \times \gamma'(t_0))) + \kappa_g^2 
    (N_\sigma \times \gamma'(t_0)) \cdot (N_\sigma \times \gamma'(t_0)) \\ 
    &= \kappa_n^2 + \kappa_g^2, 
\end{align*}
which implies that $\kappa = \sqrt{\kappa_n^2 + \kappa_g^2}$. 
In particular, if $\kappa_n = 0$, then $\kappa = |\kappa_g|$, and if 
$\kappa_g = 0$, then $\kappa = |\kappa_n|$. 

\begin{defn}{defn:3.9}
    We call $\gamma$ a {\bf geodesic} if $\kappa_g = 0$ everywhere. 
\end{defn}\vspace{-0.25cm}
