\section{Integration on manifolds}\label{sec:4}

\subsection{Integration of scalar functions} \label{subsec:4.1}
Let $M \subset \R^n$ be a smooth $k$-dimensional submanifold and let 
$\alpha : U \subset \R^k \to V \subset M$ be a smooth coordinate chart. 
If $p_0 \in V$, then $p_0 = \alpha(u_0)$ for some $u_0 \in U$ and 
\[ T_{p_0}M = \Span_{\R}\left\{ \frac{\partial\alpha}{\partial u_1}(u_0), 
\dots, \frac{\partial\alpha}{\partial u_k}(u_0) \right\} \subset T_{p_0}\R^n \simeq \R^n. \] 
Moreover, recall that we can write the derivative matrix as 
\[ D\alpha(u_0) = \left[ \begin{array}{c|c|c} 
    \!\!\!\dfrac{\partial\alpha}{\partial u_1}(u_0) & \cdots & 
    \dfrac{\partial\alpha}{\partial u_k}(u_0)\!\!\!
\end{array} \right]. \]
Note that $D\alpha(u_0)^T D\alpha(u_0)$ is a $k \times k$ matrix whose 
$(i, j)$ entry is 
\[ a_{ij} = \frac{\partial\alpha}{\partial u_i}(u_0) \cdot 
\frac{\partial\alpha}{\partial u_j}(u_0). \] 
In other words, we have that 
\[ D\alpha(u_0)^T D\alpha(u_0) = \left[ \frac{\partial\alpha}{\partial u_i}(u_0) \cdot 
\frac{\partial\alpha}{\partial u_j}(u_0) \right]_{1\leq i,j\leq k} \] 
and as with surfaces, we will define this matrix to be the first fundamental 
form of $\alpha$. 

\begin{defn}{defn:4.1}
    The matrix $\FFF = D\alpha^T D\alpha$ is called the {\bf first fundamental 
    form of $\alpha$}.
\end{defn}\vspace{-0.25cm}

Recall from linear algebra that if $A$ is an $n \times k$ (with $n \geq k$) 
of maximal rank $k$, then $A^T A$ is a $k \times k$ matrix of maximal rank $k$. 
Since $D\alpha$ has maximal rank $k$ everywhere on $U$, it follows that 
$\FFF$ has maximal rank $k$ everywhere on $U$. In other words, $\FFF$ 
is an invertible matrix, so $\det \FFF(u_0) \neq 0$ for all $u_0 \in U$. 
Moreover, for any two tangent vectors $X, Y \in T_{p_0}M$, we can write 
\begin{align*}
    X &= a_1 \frac{\partial\alpha}{\partial u_1}(u_0) + \cdots 
    + a_k \frac{\partial\alpha}{\partial u_k}(u_0), \\ 
    Y &= b_1 \frac{\partial\alpha}{\partial u_1}(u_0) + \cdots 
    + b_k \frac{\partial\alpha}{\partial u_k}(u_0),
\end{align*}
for some $a_1, \dots, a_k, b_1, \dots, b_k \in \R$. Observe that 
\begin{align*}
    X \cdot Y &= \left( \frac{\partial\alpha}{\partial u_1}(u_0) + \cdots 
    + a_k \frac{\partial\alpha}{\partial u_k}(u_0) \right) 
    \cdot \left( b_1 \frac{\partial\alpha}{\partial u_1}(u_0) + \cdots 
    + b_k \frac{\partial\alpha}{\partial u_k}(u_0) \right) \\
    &= \sum_{i=1}^k \sum_{j=1}^k a_i b_j \left( \frac{\partial\alpha}{\partial u_i}(u_0) 
    \cdot \frac{\partial\alpha}{\partial u_j}(u_0) \right) \\ 
    &= \begin{bmatrix}
        a_1 & \cdots & a_k 
    \end{bmatrix} D\alpha(u_0)^T D\alpha(u_0) \begin{bmatrix}
        b_1 \\ \vdots \\ b_k 
    \end{bmatrix} 
    = \begin{bmatrix}
        a_1 & \cdots & a_k 
    \end{bmatrix} \FFF \begin{bmatrix}
        b_1 \\ \vdots \\ b_k 
    \end{bmatrix}. 
\end{align*}
Since $(a_1, \dots, a_k)$ and $(b_1, \dots, b_k)$ are the components of 
$X$ and $Y$ with respect to the basis 
\[ {\cal B} = \left\{ \frac{\partial\alpha}{\partial u_1}(u_0), 
\dots, \frac{\partial\alpha}{\partial u_k}(u_0) \right\} \] 
of $T_{p_0}M$, this means that $\FFF$ is the matrix representation 
of the dot product $\cdot$ in $T_{p_0}\R^n \simeq \R^n$ to $T_{p_0}M$ 
with respect to the basis ${\cal B}$. Moreover, for any 
$a = (a_1, \dots, a_k) \in \R^k$, we have 
\[ a^T \FFF a = \begin{bmatrix}
    a_1 & \cdots & a_k 
\end{bmatrix} \FFF \begin{bmatrix}
    a_1 \\ \vdots \\ a_k 
\end{bmatrix} = X \cdot X \geq 0 \] 
where $X = a_1 \frac{\partial\alpha}{\partial u_1}(u_0) + \cdots 
+ a_k \frac{\partial\alpha}{\partial u_k}(u_0)$ and $a^T \FFF a = X \cdot X = 0$ 
if and only if $X = \mathbf 0$ (i.e. $a = \mathbf 0$) by positive 
definiteness of the dot product. Consequently, $\FFF$ is positive 
definite and $\det\FFF > 0$. In particular, this means that 
$\sqrt{\det\FFF}$ is well-defined. To interpret $\sqrt{\det\FFF}$, we 
start by looking at two familiar cases. 
\begin{enumerate}[(1)]
    \item If $k = 1$, then $M$ is a regular curve in $\R^n$. Suppose that 
    it is parametrized by $\gamma : (a, b) \to \R^n$ (so that $\gamma$ 
    plays the role of $\alpha$ in the above formulae). Then we have 
    $D\gamma(t) = \gamma'(t)$ and 
    \[ \FFF = D\gamma(t)^T D\gamma(t) = \gamma'(t)^T \gamma'(t) 
    = \gamma'(t) \cdot \gamma'(t) = \|\gamma'(t)\|^2. \] 
    In this case, $\FFF$ is a $1 \times 1$ matrix so that 
    $\det\FFF = \FFF$ and 
    \[ \sqrt{\det\FFF} = \|\gamma'(t)\|. \] 
    
    \item If $k = 2$ and $n = 3$, then $M$ is a smooth surface $S$ in $\R^3$. 
    Let $\sigma : U \subset \R^2 \to V \subset S \subset \R^3$ be a smooth 
    coordinate chart on $S$ (so that $\sigma$ plays the role of $\alpha$ 
    in the above formulae). Then 
    \[ \FFF = D\sigma^T D\sigma \] 
    is the usual first fundamental form. We know from Lemma~\ref{lemma:3.4} that 
    \[ \sqrt{\det\FFF} = \|\sigma_u \times \sigma_v\|, \] 
    which is the area of the parallelogram spanned by $\sigma_u$ and $\sigma_v$. 
\end{enumerate}

More generally, one can show that if $X_1, \dots, X_k$ are $k$ linearly 
independent vectors in $\R^n$ and $B$ is the $k \times k$ matrix whose 
$(i, j)$-entry is $b_{ij} = X_i \cdot X_j$, then 
$\sqrt{\det B}$ is the volume of the parallelepiped spanned by $X_1, \dots, X_k$. 
Therefore, in our situation, $\sqrt{\det\FFF}$ is the volume of the 
parallelepiped spanned by the elements 
\[ {\cal B} = \left\{ \frac{\partial\alpha}{\partial u_1}(u_0), 
\dots, \frac{\partial\alpha}{\partial u_k}(u_0) \right\}. \] 
Given this, we can now define the integral of a scalar function of $V \subset M$. 

\begin{defn}{defn:4.2}
    Suppose that $V \subset M$ is bounded and let $f : V \to \R$ is a continuous 
    scalar function. We define the {\bf integral of $f$ on $V$} to be 
    \[ \int_V f\dVol := \int_U f(\alpha(u))\sqrt{\det\FFF}\dd u_1 \cdots \dd u_k. \] 
\end{defn}\vspace{-0.25cm}

Here, we can think of $\sqrt{\det\FFF}$ as some sort of correcting term. 

Note that the definition of the integral of $f$ on $V$ is independent of the 
choice of coordinate chart $\alpha$. Indeed, if $\tilde\alpha : 
\tilde U \subset \R^k \to V \subset M$ is another coordinate chart 
parametrizing $V$, then $\tilde\alpha = \alpha \circ \Phi$ where 
\[ \Phi := \alpha^{-1} \circ \tilde\alpha : \tilde U \subset \R^k 
\to U \subset \R^k \] 
is a diffeomorphism. By the chain rule, we have $D\tilde\alpha 
= D\alpha D\Phi$ so that 
\[ \tilde\FFF = D\tilde\alpha^T D\tilde\alpha = (D\alpha D\Phi)^T 
(D\alpha D\Phi) = D\Phi^T D\alpha^T D\alpha D\Phi = D\Phi^T \FFF D\Phi \] 
and hence $\det\tilde\FFF = \det(D\Phi)^2 \det\FFF$ since determinant is 
multiplicative and $\det D\Phi = \det D\Phi^T$.
This gives
\[ \sqrt{\det\tilde\FFF} = \lvert\det D\Phi\rvert \sqrt{\det \FFF}, \] 
which implies that 
\begin{align*}
    \int_{\tilde U} f(\tilde\alpha(\tilde u)) \sqrt{\det\tilde\FFF} \dd\tilde u_1 \cdots \dd\tilde u_k  
    &= \int_{\tilde U} f(\tilde\alpha(\tilde u)) \sqrt{\det\FFF} \,\lvert\det D\Phi\rvert \dd\tilde u \cdots \dd\tilde u_k \\ 
    &= \int_U f(\alpha(u)) \sqrt{\det\FFF} \dd u_1 \cdots \dd u_k 
\end{align*} 
since $\lvert\det D\Phi\rvert \dd\tilde u_1 \cdots \dd\tilde u_k 
= \dd u_1 \cdots \dd u_k$ by the change of variable formula. 

We now see what the integral formula looks like in three special cases. 
\begin{enumerate}[(1)]
    \item If $k = 1$ so that $M$ is a regular curve $\gamma : (a, b) \to \R^n$, 
    we denote the integral by $\int_\gamma f\dd s$ and 
    \[ \int_\gamma f\dd s = \int_a^b f(\gamma(t))\|\gamma'(t)\|\dd t. \] 
    In particular, if $f \equiv 1$, then we obtain 
    \[ \int_\gamma \dd s = \int_a^b \|\gamma'(t)\| \dd t, \] 
    which is the arclength of $\gamma$ between $\gamma(a)$ and $\gamma(b)$.

    \item If $k = 2$ and $n = 3$ so that $M$ is a smooth surface $S$ in $\R^3$ 
    and $\sigma : U \subset \R^2 \to V \subset S \subset \R^3$ is a smooth 
    coordinate chart on $S$, then we denote the integral by $\int_V f\dd A$ 
    and 
    \[ \int_V f\dd A = \int_U f(\sigma(u, v)) \|\sigma_u \times \sigma_v\|\dd u \dd v. \] 
    If $f \equiv 1$, then we obtain 
    \[ \int_V \dd A = \int_U \|\sigma_u \times \sigma_v\|\dd u \dd v, \] 
    which is the surface area of $V \subset S$. 
    
    \item If $M$ is a bounded open subset of $\R^n$, then we can pick 
    $\alpha = \Id_M$ as the smooth coordinate chart. We have 
    $D\alpha = I_{n\times n}$ so that $\FFF = I_{n\times n}$ and 
    $\det\FFF = \sqrt{\det\FFF} = 1$. In this case, we simply have 
    \[ \int_M f\dVol = \int_M f(x_1, \dots, x_n) \dd x_1 \cdots \dd x_n, \] 
    which is the usual integral of $f$ on the region $M \subset \R^n$. 
    Moreover, if $f \equiv 1$, then 
    \[ \int_M \dVol = \int_M \dd x_1 \cdots \dd x_n \] 
    is the volume of $M$. 
    
\end{enumerate}

We illustrate this with a few examples. 
\begin{enumerate}[(1)]
    \item {\bf (Sphere.)} Let $S$ be the sphere $x^2 + y^2 + z^2 = a^2$ where $a > 0$, 
    which we can parametrize using
    \[ \sigma(\theta, \varphi) = (a\cos\theta\sin\varphi, a\sin\theta\sin\varphi, a\cos\varphi) \] 
    where $(\theta, \varphi) \in (0, 2\pi) \times (0, \pi)$. We have
    $\sqrt{\det\FFF} = \|\sigma_\theta \times \sigma_\varphi\| = a^2\sin\varphi > 0$. 
    By taking $f \equiv 1$, we see that the surface area of a sphere of radius $a$ is 
    \begin{align*}
        \int_S \dd A &= \int_U \|\sigma_\theta \times \sigma_\varphi\|\dd\theta \dd\varphi \\
        &= \int_0^\pi \int_0^{2\pi} a^2\sin\varphi \dd\theta \dd\varphi \\ 
        &= 2\pi a^2 \int_0^\pi \sin\varphi\dd\varphi \\ 
        &= 2\pi a^2(-\cos(\pi) + \cos(0)) = 4\pi a^2. 
    \end{align*}
    We have also seen that $K = 1/a^2$ (see example (3) on page 40) so that 
    \[ \int_S K\dd A = \int_S \frac{1}{a^2} \dd A = \frac{1}{a^2} \int_S \dd A 
    = \frac{1}{a^2} (4\pi a^2) = 4\pi = 4\pi(1-g), \] 
    where $g = 0$ since the sphere has no holes. Therefore, Gauss-Bonnet 
    (which we stated at the end of Section~\ref{subsec:3.5}, page 50) holds 
    for spheres!

    \item {\bf (Helix.)} Let $\gamma(t) = (a\cos t, a\sin t, bt)$ where 
    $t \in [0, \pi/3]$ be a part of the helix in $\R^3$, with $a, b > 0$. 
    Consider the scalar function $f(x, y, z) = x + y$ on $\R^3$. Since 
    we are only taking the parameter $t$ in the bounded interval $[0, \pi/3]$, 
    the piece of the helix is itself bounded. Moreover, $f$ is continuous 
    everywhere, so we can integrate $f$ over $\gamma$. Since $\gamma$ is a 
    regular curve, we know that 
    \[ \int_\gamma f\dd s = \int_0^{\pi/3} f(\gamma(t))\|\gamma'(t)\|\dd t. \] 
    In this case, we have $\gamma'(t) = (-a\sin t, a\cos t, b)$ so that 
    $\|\gamma'(t)\| = \sqrt{a^2 + b^2}$ and 
    \begin{align*}
        \int_\gamma f\dd s &= \int_0^{\pi/3} f(\gamma(t)) \|\gamma'(t)\|\dd t \\ 
        &= \int_0^{\pi/3} (a\cos t + a\sin t) \sqrt{a^2 + b^2}\dd t \\ 
        &= a\sqrt{a^2 + b^2} \int_0^{\pi/3} (\cos t + \sin t)\dd t \\ 
        &= a\sqrt{a^2 + b^2} \left[\left(\sin\left(\frac{\pi}{3}\right) - 
        \cos\left(\frac{\pi}{3}\right)\right) - (\sin 0 - \cos 0)\right] \\
        &= \frac{a\sqrt{a^2 + b^2}}{2} (1 + \sqrt{3}). 
    \end{align*}

    \item {\bf (Cylinder.)} Let $C$ be the part of the cylinder in $\R^3$ where 
    $x^2 + y^2 = 1$ and $2 \leq z \leq 4$. Since we are fixing the values 
    of $z$ to be between $2$ and $4$, we see that $C$ is bounded.  
    It can be parametrized by 
    \[ \sigma(u, v) = (\cos u, \sin u, v) \] 
    where $(u, v) \in U = (0, 2\pi) \times (2, 4)$. We have previously seen that 
    $\FFF = I_{2\times 2}$ for this coordinate chart $\sigma$ 
    (see example (2) on page 35) so that $\sqrt{\det\FFF} = 1$. Since 
    $C$ is a smooth surface, we can compute its surface area with 
    \[ \int_C \dd A = \int_U \sqrt{\det\FFF} \dd u \dd v = 
    \int_U \dd u \dd v = \int_2^4 \int_0^{2\pi} \dd u \dd v = 
    \int_2^4 2\pi \dd v = 4\pi. \] 

    Let us also integrate $f(x, y, z) = x^2 y + z$ over $C$. We have 
    \begin{align*}
        \int_C f\dd A &= \int_U f(\sigma(u))\sqrt{\det \FFF}\dd u \dd v \\ 
        &= \int_2^4 \int_0^{2\pi} ((\cos u)^2 \sin u + v) \dd u \dd v \\ 
        &= \int_2^4 \left[ -\frac13(\cos u)^3 + uv \right]_0^{2\pi} \dd v \\ 
        &= \int_2^4 2\pi v\dd v = \pi(4^2 - 2^2) = 12\pi. 
    \end{align*}

    \item {\bf (Torus.)} We now consider the $2$-dimensional torus 
    $\mathbb{T}^2$ in $\R^3$ described in Problem 3 of Assignment 4. It can be 
    parametrized with 
    \[ \sigma(\theta, \alpha) = ((b + a\cos\alpha)\cos\theta, 
    (b + a\cos\alpha)\sin\theta, a\sin\alpha) \] 
    where $(\theta, \alpha) \in U = (0, 2\pi) \times (0, 2\pi)$. A direct 
    computation gives $\|\sigma_\theta \times \sigma_\alpha\| = a(b + a\cos\theta)$ 
    and 
    \[ K = \frac{\cos\alpha}{a(b + a\cos\alpha)}. \] 
    Then the surface area of $\mathbb T^2$ is 
    \begin{align*} 
        \int_{\mathbb T^2} \dd A 
        &= \int_0^{2\pi} \int_0^{2\pi} a(b + a\cos\alpha)\dd\theta \dd\alpha \\ 
        &= 2\pi a \int_0^{2\pi} (b + a\cos\alpha)\dd \alpha \\ 
        &= 2\pi a \Big[ b\alpha + a\sin\alpha \Big]_0^{2\pi} = 4\pi^2 ab. 
    \end{align*}
    Moreover, if we integrate over the Gaussian curvature $K$ on $\mathbb T^2$, we obtain 
    \begin{align*}
        \int_{\mathbb T^2} K\dd A 
        &= \int_0^{2\pi} \int_0^{2\pi} \frac{\cos\alpha}{a(b+a\cos\alpha)} 
        a(b+a\cos\alpha) \dd\theta \dd \alpha \\ 
        &= 2\pi \int_0^{2\pi} \cos\alpha\dd \alpha 
        = 2\pi(\sin(2\pi) - \sin 0) = 0 = 4\pi(1-g), 
    \end{align*}
    where $g = 1$ is the genus of the torus since it has a single hole. 
    Therefore, Gauss-Bonnet also holds here!

\end{enumerate}

\subsection{Integrating vector fields over submanifolds} \label{subsec:4.2}
Let $M \subset \R^n$ be a smooth $k$-dimensional submanifold. By a 
vector field on $M$, we mean the following: 

\begin{defn}{defn:4.3}
    A {\bf vector field on $M$} is a vector-valued function $F : M \to \R^n$. 
\end{defn}\vspace{-0.25cm}

A vector field is a function that attaches to every point on $M$ 
a vector in the ambient space $\R^n$. Therefore, it is important 
to note in the definition that if $M$ is a submanifold of $\R^n$, then the 
vector field $F$ also takes values in $\R^n$. 

Let's consider some examples of vector fields. 
\begin{enumerate}[(1)]
    \item The function $F(x, y, z, w) = (x^2y - z, y\cos w, 4e^x + 1, z - 3)$ 
    is a vector field on $\R^4$. 
    \item If $\gamma : (a, b) \to \R^n$ is a parametrized curve in $\R^n$, 
    then the velocity $\gamma'(t)$ and the acceleration $\gamma''(t)$ are 
    vector fields on $\gamma$. 
    \item If a vector field $F : M \to \R^n$ is such that $F(p) \in T_pM$ 
    for all $p \in M$, then $F$ is called a {\bf tangent vector field}. 
    For example, the velocity $\gamma'(t)$ is a tangent vector field 
    on the parametrized curve $\gamma : (a, b) \to \R^n$.
    \item If a vector field $F : M \to \R^n$ is such that $F(p) \perp T_pM$ for 
    all $p \in M$, then $F$ is called a {\bf normal vector field}. 
    For example, the principal unit normal $N(t)$ and binormal $B(t)$ 
    of a regular curve $\gamma : (a, b) \to \R^n$ are normal vector fields 
    (where they are defined, namely where the curvature is strictly positive).
    \item If $\sigma : U \subset \R^2 \to V \subset S \subset \R^3$ is a smooth 
    coordinate chart of a surface $S$ in $\R^3$, then $N_\sigma$ is a unit 
    normal vector field on $V$. 
    \item If $M$ is the zero set of a smooth scalar function $G : \R^n \to \R$ 
    with $DG = \nabla G$ of maximal rank everywhere, then we know that $M$ is a smooth 
    $(n-1)$-dimensional submanifold of $\R^n$. The gradient $\nabla G$ 
    is a normal vector field on $M$ since we have 
    \[ T_pM \simeq \{v \in \R^n : \nabla G(p) \cdot v = 0\} \] 
    for all $p \in M$ (see the end of Section~\ref{subsec:1.5}, page 21 
    for the discussion of this fact).
    \begin{enumerate}[(a)]
        \item Recall that the unit $n$-sphere $\mathbb S^n : x_1^2 + 
        \cdots + x_{n+1}^2 = 1$ in $\R^{n+1}$ is the zero set of the function 
        $G(x_1, \dots, x_{n+1}) = x_1^2 + \cdots + x_{n+1}^2 - 1$. Its 
        gradient is $\nabla G(x_1, \dots, x_{n+1}) = (2x_1, \dots, 2x_{n+1})$, 
        which is nonzero everywhere on $\mathbb S^n$. Consequently, the gradient 
        $\nabla G$ is a nowhere vanishing normal vector field on $\mathbb S^n$. 
        In particular, this means we can normalize it to see that 
        \[ N(x_1, \dots, x_{n+1}) = \frac{\nabla G(x_1, \dots, x_{n+1})}
        {\|\nabla G(x_1, \dots, x_{n+1})\|} = \frac{1}{\sqrt{x_1^2 + \cdots 
        + x_{n+1}^2}}(x_1, \dots, x_{n+1}) \] 
        is a unit normal vector field on $\mathbb S^n$. 

        \item The paraboloid $S : z = x^2 + y^2$ in $\R^3$ is the zero 
        set of the smooth scalar function $G(x, y, z) = x^2 + y^2 - z$ 
        whose gradient $\nabla G(x, y, z) = (2x, 2y, -1)$. After normalizing, 
        we see that 
        \[ N(x, y, z) = \frac{\nabla G(x, y, z)}{\|\nabla G(x, y, z)\|} 
        = \frac{1}{4x^2 + 4y^2 + 1} (2x, 2y, -1) \] 
        is a unit normal vector field on $S$. 
    \end{enumerate}
\end{enumerate}

\subsubsection{Line integrals of vector fields} \label{subsubsec:4.2.1}
We start by considering integrals of vector fields on curves. Let 
$\gamma : [a, b] \to \R^n$ be a smooth regular curve in $\R^n$. Its 
velocity is a smooth nowhere vanishing tangent vector field on $\gamma$.
If $F$ is a vector field defined on an open subset $U$ of $\R^n$ containing 
$\gamma$, then the restriction of $F$ to $\gamma$ is $F(\gamma(t))$ for 
$t \in (a, b)$. Moreover, the dot product of $F$ with $\gamma'$ at points on 
$\gamma$ is the scalar function 
\[ f(t) = F(\gamma(t)) \cdot \gamma'(t) \] 
with $t \in (a, b)$. If $F$ is smooth, then the scalar function $f$ is 
also smooth and therefore integrable on $(a, b)$. We then define 
the integral of $F$ on $\gamma$ to be $\int_a^b f(t)\dd t$. More precisely, 
we have the following definition: 

\begin{defn}{defn:4.4}
    Let $\gamma : [a, b] \to \R^n$ be a smooth regular curve in $\R^n$, 
    and let $F : U \to \R^n$ be a smooth vector field defined on an 
    open set $U \subset \R^n$ containing $\gamma$. We define 
    the {\bf line integral of $F$ on $\gamma$} to be 
    \[ \int_\gamma F\cdot {\rm d}r := \int_a^b F(\gamma(t)) \cdot \gamma'(t) \dd t. \] 
\end{defn}\vspace{-0.25cm}

{\bf Example.} Let $\gamma(t) = (t, t^2, t^3)$ where $0 \leq t \leq 1$ 
be (part of) the twisted cubic in $\R^3$ and let $F(x, y, z) = (8x^2yz, 5z, -4xy)$ 
be a smooth vector field on $\R^3$. Then we have 
\[ F(\gamma(t)) = (8(t)^2(t^2)(t^3), 5t^3, -4(t)(t^2)) = (8t^7, 5t^3, -4t^3) \] 
and $\gamma'(t) = (1, 2t, 3t^2)$, which gives 
\[ F(\gamma(t)) \cdot \gamma'(t) = (8t^7, 5t^3, -4t^3) \cdot 
(1, 2t, 3t^2) = 8t^7 - 12t^5 + 10t^4. \] 
Therefore, the line integral of $F$ on $\gamma$ is 
\[ \int_\gamma F\cdot{\rm d}r = \int_0^1 F(\gamma(t)) \cdot \gamma'(t)\dd t 
= \int_0^1 (8t^7 - 12t^5 + 10t^4)\dd t = \Big[ t^8 - 2t^6 + 2t^5 \Big]_0^1 
= 1. \] 

{\bf Remark.} Since $\gamma$ is regular, it has a well-defined unit tangent 
vector field $T(t) = \gamma'(t)/\|\gamma'(t)\|$. Then 
\[ F(\gamma(t)) \cdot \gamma'(t) = (F(\gamma(t)) \cdot T(t))\,\|\gamma'(t)\| \] 
so that we have 
\[ \int_\gamma F\cdot{\rm d}r := \int_a^b F(\gamma(t)) \cdot \gamma'(t)\dd t 
= \int_a^b (F(\gamma(t)) \cdot T(t))\,\|\gamma'(t)\|\dd t = 
\int_\gamma (F \cdot T)\dd s, \] 
where the last integral is the line integral of the scalar function 
$F \cdot T$ on $\gamma$ which we defined in Section~\ref{subsec:4.1}
(see page 53). Some authors define the line integral of a vector field 
on a regular curve as $\int_\gamma (F \cdot T)\dd s$. 

By thinking of the definition in terms of the unit tangent vector, we see that 
line integrals of vector fields are uniquely defined by the curve only 
up to orientation. Indeed, at any point $\gamma(t)$ on $\gamma$, there are only 
two possible \emph{unit} tangent directions, namely $T(t)$ and $-T(t)$. 
If we reparametrize the curve as $\bar\gamma(u) = \gamma(a+b-u)$ where 
$u \in [a, b]$, we obtain a parametrization that traces the curve in the 
opposite direction, starting at $\gamma(b) = \bar\gamma(a)$ and ending 
at $\gamma(a) = \bar\gamma(b)$. Moreover, we have $\bar\gamma'(u) = 
-\gamma'(a+b-u)$ so that $\bar T(u) = -T(a+b-u)$ at the point 
$p = \bar\gamma(u) = \gamma(a+b-u)$ on the curve. This implies that 
\[ \int_{\bar\gamma} F\cdot{\rm d}r = -\int_\gamma F\cdot{\rm d}r. \] 
We denote the reparametrization $\bar\gamma$ by $-\gamma$ and write 
\[ \int_{-\gamma} F\cdot{\rm d}r = -\int_\gamma F\cdot{\rm d}r. \] 
So far, we have only defined line integrals of vector fields on smooth 
regular curves. They can also be defined over piecewise smooth regular curves. 

\begin{defn}{defn:4.5}
    A parametrized curve $\gamma : [a, b] \to \R^n$ is called a {\bf 
    piecewise smooth regular curve} if $\gamma$ is a piecewise smooth 
    vector-valued function 
    \[ \gamma(t) = \begin{cases}
        \gamma_1(t), & \text{if } a = a_0 \leq t \leq a_1, \\ 
        \gamma_2(t), & \text{if } a_1 \leq t \leq a_2, \\ 
        \quad\vdots & \qquad\qquad \vdots \\ 
        \gamma_\ell(t), & \text{if } a_{\ell-1} \leq t \leq a_\ell = b 
    \end{cases} \] 
    such that $\gamma'_i(t) \neq \mathbf 0$ for all $t \in [a_{i-1}, a_i]$ 
    and $i = 1, \dots, \ell$. 
\end{defn}\vspace{-0.25cm}

The line integral of the vector field $F$ on the piecewise smooth regular 
curve $\gamma$ is then defined to be 
\[ \int_\gamma F\cdot{\rm d}r := \sum_{i=1}^\ell \int_{\gamma_i} F\cdot{\rm d}r. \] 

{\bf Example.} Let $\gamma : [-1, 3] \to \R^2$ be the piecewise smooth curve 
in $\R^2$ given by 
\[ \gamma(t) = \begin{cases}
    \gamma_1(t) = (t, t^2), & \text{if } -1 \leq t \leq 1, \\ 
    \gamma_2(t) = (2-t, 1), & \text{if } 1 \leq t \leq 3. 
\end{cases} \] 
Consider the smooth vector field $F(x, y, z) = (x-1, 4xy)$ on $\R^2$. Then 
we have 
\begin{align*} 
    \int_\gamma F\cdot{\rm d}r 
    &= \int_{\gamma_1} F\cdot{\rm d}r + \int_{\gamma_2} F\cdot{\rm d}r \\ 
    &= \int_{-1}^1 F(\gamma_1(t)) \cdot \gamma'_1(t)\dd t 
    + \int_1^3 F(\gamma_2(t)) \cdot \gamma'_2(t) \dd t. 
\end{align*} 
Note that $F(\gamma_1(t)) = (t-1, 4t^3)$ and $F(\gamma_2(t)) = 
(1-t, 4(2-t))$ with $\gamma'_1(t) = (1, 2t)$ and $\gamma'_2(t) = 
(-1, 0)$. Therefore, we can compute 
\begin{align*}
    F(\gamma_1(t)) \cdot \gamma'_1(t) &= (t-1, 4t^3) \cdot (1, 2t) = 
    8t^4 + t - 1, \\ 
    F(\gamma_2(t)) \cdot \gamma'_2(t) &= (1-t, 4(2-t)) \cdot (-1, 0) = 
    t-1, 
\end{align*}
and the line integral of $F$ over $\gamma$ is 
\begin{align*} 
    \int_\gamma F\cdot{\rm d}r 
    &= \int_{-1}^1 (8t^4 + t - 1)\dd t + \int_1^3 (t-1)\dd t \\ 
    &= \left[ \frac85t^5 + \frac12t^2 - t\right]_{-1}^1 + 
    \left[ \frac12 t^2 - t \right]_1^3 \\ 
    &= \frac65 + 2 = \frac{16}{5}. 
\end{align*} 

Recall that the Fundamental Theorem of Calculus states that if a 
scalar one-variable function $g : [a, b] \to \R$ has an antiderivative 
$G : [a, b] \to \R$ (i.e. $G(x) = g(x)$ for all $x \in [a, b]$), then 
\[ \int_a^b g(x)\dd x = G(b) - G(a). \] 
We have a very similar result for line integrals of vector fields, which we 
call the fundamental theorem of line integrals of vector fields. 

\begin{theo}[Fundamental Theorem of Line Integrals of Vector Fields]{theo:4.6}
    Let $F : U \to \R^n$ be a vector field on an open subset $U$ of $\R^n$. 
    If there exists a scalar function $f : U \to \R$ such that $F = \nabla f$, 
    then we have 
    \[ \int_\gamma F\cdot{\rm d}r = f(\gamma(b)) - f(\gamma(a)) \] 
    for any piecewise smooth regular curve $\gamma$ in $U$. In 
    particular, the line integral is \emph{independent of path}. 
\end{theo}\vspace{-0.25cm}\newpage
\begin{pf}[Theorem~\ref{theo:4.6}]
    It is enough to prove the theorem for smooth regular curves. Suppose that 
    $F = \nabla f = (f_{x_1}, \dots, f_{x_n})$. Then for any smooth regular 
    curve $\gamma(t) = (x_1(t), \dots, x_n(t))$ with $t \in [a, b]$, we have 
    \[ F(\gamma(t)) \cdot \gamma'(t) = (f_{x_1}(\gamma(t)), \dots, 
    f_{x_n}(\gamma(t))) \cdot (x'_1(t), \dots, x'_n(t)) = 
    \sum_{i=1}^n f_{x_i}(\gamma(t))x'_i(t) = \frac{\rm d}{{\rm d}t} f(\gamma(t)), \] 
    where the last equality follows from the chain rule. By the Fundamental 
    Theorem of Calculus, we have 
    \[ \int_\gamma F\cdot{\rm d}r = \int_a^b F(\gamma(t)) \cdot \gamma'(t)\dd t 
     = \int_a^b \frac{\rm d}{{\rm d}t} f(\gamma(t))\dd t = f(\gamma(b)) - 
     f(\gamma(a)). \tag*{\qed} \] 
\end{pf}\vspace{-0.25cm}

{\bf Remark.} If there exists a scalar function $f : U \to \R$ such that 
$F = \nabla f$, then Theorem~\ref{theo:4.6} tells us that 
\[ \int_\gamma F \cdot {\rm d}r = 0 \] 
for any closed curve $\gamma$ 
(i.e. $\gamma(a) = \gamma(b)$) since $f(\gamma(b)) - f(\gamma(a)) = 0$. 
In particular, if we can find a closed 
curve $\gamma$ such that $\int_\gamma F \cdot {\rm d}r \neq 0$, then 
this implies that $F$ cannot be the gradient of a function. 

{\bf Example.} Consider the vector field 
\[ F(x, y) = \left( -\frac{y}{x^2+y^2}, \frac{x}{x^2+y^2} \right) \] 
on the open set $U = \R^2 \setminus \{(0, 0)\}$. If $\gamma : [0, 2\pi] 
\to \R^2$ given by $t \mapsto (\cos t, \sin t)$ is the unit circle, then 
\[ F(\gamma(t)) = \left( -\frac{\sin t}{(\cos t)^2 + (\sin t)^2}, 
\frac{\cos t}{(\cos t)^2 + (\sin t)^2} \right) = (-\sin t, \cos t) = \gamma'(t) \] 
so that $F(\gamma(t)) \cdot \gamma'(t) = \gamma'(t) \cdot \gamma'(t) = 1$ and 
\[ \int_\gamma F\cdot {\rm d}r = \int_0^{2\pi} 1\dd t = 2\pi \neq 0. \] 
It follows that $F$ is not the gradient of a scalar function on $U$. 

\subsubsection{Integrals of vector fields over hypersurfaces} \label{subsubsec:4.2.2}
Let us now assume that $M$ is an $(n-1)$-dimensional smooth submanifold of $\R^3$. 
When $n = 3$, then $M$ is just a surface in $\R^3$. If $n > 3$, such an $M$ 
is called a {\bf hypersurface}. We have seen that $M$ is locally the zero 
set of a smooth scalar function $G : V \subset \R^n \to \R$ with $DG = 
\nabla G$ of maximal rank everywhere on an open subset $V$ of $\R^n$, and 
that the gradient is a normal vector field on $M \cap V = \{G \equiv 0\}$. 
Since $\nabla G$ is nowhere vanishing on $M \cap V$, we can normalize it to 
obtain a \emph{unit} normal vector field 
\[ N(p) := \frac{\nabla G(p)}{\|\nabla G(p)\|}, \] 
where $p \in M \cap V$. If $F : V \subset \R^n \to \R^n$ is a vector field on $V$, 
we can define the integral of $F$ on a bounded subset $B$ of $M \cap V$ as follows: 

\begin{defn}{defn:4.7}
    The integral of the vector field $F : V \subset \R^n \to \R^n$ on the 
    bounded subset $B$ of $M \cap V$ with unit normal vector $N$ is defined as 
    \[ \int_B F \cdot \dVol := \int_B (F \cdot N) \dVol, \] 
    where the second integral is the integral of the scalar function 
    $F \cdot N$ on $B$ (as in Definition~\ref{defn:4.2}).
\end{defn}\vspace{-0.25cm}

We compute a few examples. 
\begin{enumerate}[(1)]
    \item {\bf (Sphere.)} Let $S$ be the sphere $x^2 + y^2 + z^2 = a^2$ in $\R^3$, where $a > 0$. 
    This is the zero set of the function $G(x, y, z) = x^2 + y^2 + z^2 - a^2$ on 
    $\R^3$ whose gradient $\nabla G(x, y, z) = (2x, 2y, 2z)$ is nowhere vanishing 
    on $S$, giving us the unit normal vector field 
    \[ N(x, y, z) = \frac{\nabla G(x, y, z)}{\|\nabla G(x, y, z)\|} 
    = \frac{1}{\sqrt{x^2+y^2+z^2}}(x, y, z) = \frac1a(x, y, z) \] 
    since $x^2 + y^2 + z^2 = a^2$ on $S$. Consider the smooth vector field 
    $F(x, y, z) = (-y, x, 1)$ on $\R^3$. We will compute $\int_S F\cdot\dVol$. 
    In this case, we have 
    \[ F \cdot N = (-y, x, 1) \cdot \frac1a(x, y, z) = \frac1a(-yx + xy + z) 
    = \frac1az. \] 
    Therefore, we need to compute 
    \[ \int_S F \cdot \dVol = \int_S (F \cdot N)\dd A = \int_S \frac1az \dd A. \] 
    Now, we need to parametrize the sphere. Using the spherical coordinates 
    \[ \sigma(\theta, \varphi) = (a\cos\theta\sin\varphi, a\sin\theta\sin\varphi, a\cos\varphi) \] 
    with $(\theta, \varphi) \in U = (0, 2\pi) \times (0, \pi)$, we obtain 
    $\|\sigma_\theta \times \sigma_\varphi\| = a^2 \sin\varphi > 0$, so 
    \begin{align*}
        \int_S \frac1az\dd A 
        &= \int_U \frac1a(a\cos\varphi)\|\sigma_\theta \times \sigma_\varphi\|\dd\theta\dd\varphi \\ 
        &= \int_0^\pi \int_0^{2\pi} a^2 \cos\varphi \sin\varphi \dd\theta\dd\varphi \\ 
        &= 2\pi a^2 \int_0^\pi \sin\varphi \cos\varphi \dd\varphi \\ 
        &= 2\pi a^2 \left[ \frac12 \sin^2\varphi \right]_0^\pi = \pi a^2(0 - 0) = 0. 
    \end{align*}

    \item Suppose that $S$ is a surface in $\R^3$ and $\sigma : U 
    \subset \R^2 \to V \subset S$ is a smooth coordinate chart. The 
    standard unit normal $N_\sigma$ is then a smooth unit normal vector 
    field on $V$. If $F$ is any smooth vector field on $V$, its integral 
    on a bounded subset $B$ of $V$ with the unit vector field $N_\sigma$ is 
    \begin{align*}
        \int_B F \cdot \dVol 
        &= \int_B (F \cdot N_\sigma)\dd A \\ 
        &= \int_{\sigma^{-1}(B)} F(\sigma(u, v)) \cdot 
        \left( \frac{\sigma_u \times \sigma_v}{\|\sigma_u \times \sigma_v\|} 
        \right) \|\sigma_u \times \sigma_v\|\dd u \dd v \\ 
        &= \int_{\sigma^{-1}(B)} F(\sigma(u, v)) \cdot (\sigma_u \times \sigma_v)\dd u \dd v.
    \end{align*}
    For example, consider the half-cylinder $C = \{(x, y, z) \in \R^3 : 
    x^2+y^2=1,\,y\geq 0,\,2\leq z \leq 4\}$ in $\R^3$. Then $C$ is bounded 
    and can be parametrized by 
    \[ \sigma(u, v) = (\cos u, \sin u, v) \] 
    where $(u, v) \in U = [0, \pi] \times [2, 4]$ and $\sigma_u \times \sigma_v 
    = (\cos u, \sin u, 0)$. Let $F(x, y, z) = (xy, 2, z-1)$. Then 
    \[ F(\sigma(u, v)) \cdot (\sigma_u \times \sigma_v) = 
    (\cos u \sin u, 2, v-1) \cdot (\cos u, \sin u, 0) = (\cos u)^2 \sin^2 + 2\sin u. \] 
    The integral of $F$ on $C$ with normal vector $N_\sigma$ is 
    \begin{align*}
        \int_C F \cdot \dVol 
        &= \int_U F(\sigma(u, v)) \cdot (\sigma_u \times \sigma_v)\dd u\dd v \\ 
        &= \int_2^4 \int_0^\pi [(\cos u)^2 \sin u + 2\sin u] \dd u \dd v \\ 
        &= \int_2^4 \left[ -\frac{(\cos u)^3}{3} - 2\cos u \right]_0^\pi \dd v \\ 
        &= \int_2^4 \left[ \left( \frac13 + 2 \right) - \left(-\frac13 - 2\right) \right] \dd v \\ 
        &= \int_2^4 \frac{14}{3} \dd v = \frac{14}{3}(4-2) = \frac{28}{3}. 
    \end{align*}

\end{enumerate}

On a hypersurface $M$ in $\R^n$, the tangent space $T_pM$ to $M$ at any point 
$p \in M$ is an $(n-1)$-dimensional subspace of $T_p\R^n \simeq \R^n$. 
Hence, there is only a $1$-dimensional subspace of normal directions to $T_pM$ 
at $p$ in $T_p\R^n$ so that there are only two \emph{unit} normal vectors 
to $M$ at $p$; one is minus the other. To be precise, if $N$ is a unit 
normal vector to $M$ at $p$, then $\pm N$ are the only possible unit 
normal vectors to $M$ at $p$. 

An {\bf orientation} of the tangent space $T_pM$ is a choice of unit 
normal vector (either $N$ or $-N$). Moreover, an {\bf orientation} on the 
hypersurface $M$ is a choice of the smooth unit normal vector field $N$ on $M$. 
For example, if $M$ is the zero set of a smooth scalar function $G : 
V \subset \R^n \to \R$ with $\nabla G$ of maximal rank everywhere on an open 
subset $V$ of $\R^n$, it admits two orientations, namely 
\[ N(p) := \pm \frac{\nabla G(p)}{\|\nabla G(p)\|}, \] 
where $p \in M$. Similarly, if $S$ is a surface in $\R^3$ and $\sigma : 
U \subset \R^2 \to V \subset S$ is a smooth coordinate chart, then the 
open set $V \subset S$ admits two possible orientations, namely $\pm N_\sigma$. 

\begin{defn}{defn:4.8}
    A hypersurface $M$ in $\R^n$ is called {\bf orientable} if it admits a 
    smooth unit normal vector $N$. A choice of a smooth unit normal vector 
    field $N$ on $M$ is called an {\bf orientation} of $M$. 
\end{defn}\vspace{-0.25cm}

We give some examples of orientable hypersurfaces. 
\begin{enumerate}[(1)]
    \item The zero set $M$ of a smooth scalar function $G : V \subset \R^n 
    \to \R$ with $\nabla G$ of maximal rank everywhere on an open 
    set $V$ of $\R^n$ is orientable. Its two orientations are 
    $\pm\nabla G/\|\nabla G\|$. Since any hypersurface is locally 
    the zero set of such a function, this means that hypersurfaces 
    are always locally orientable. 

    \item If $S$ is a surface in $\R^3$ and $\sigma : U \subset \R^2 \to V \subset S$ 
    is a smooth coordinate chart, then the open set $V \subset \R^3$ is 
    orientable and its orientations are $\pm N_\sigma$. Hence, any 
    surface in $\R^3$ is orientable. 

    \item Although every hypersurface is locally orientable, it might not be 
    globally orientable. For example, the M\"obius strip is \emph{not} 
    orientable because it does not admit a smooth unit normal vector field 
    (see Assignment 5).
\end{enumerate}

{\bf Remark.} Integrals of vector fields on orientable hypersurfaces 
are uniquely determined by the hypersurface only up to orientation. 
Indeed, changing the orientation changes the sign of the integral, with 
\[ \int_M (F \cdot (-N))\dVol = -\int_M (F \cdot N)\dVol\!. \] 
Therefore, it is important to specify an orientation when computing the 
integral of a vector field. 

\subsection{Stokes' Theorem} \label{subsec:4.3}
We end our presentation of integration on submanifolds with Stokes' Theorem, 
which is a generalization of the Fundamental Theorem of Calculus. 
First, we need the following definition. 

\begin{defn}{defn:4.9}
    Let $F : U \subset \R^n \to \R^n$ be a smooth vector field on an 
    open subset $U$ of $\R^n$. The {\bf divergence of $F$} is the 
    scalar function on $U$ defined by 
    \[ \Div F := \frac{\partial F_1}{\partial x_1} + \cdots + \frac{\partial F_n}{\partial x_n}. \] 
\end{defn}\vspace{-0.25cm}

We now state Stokes' Theorem. 

\begin{theo}[Stokes' Theorem]{theo:4.10}
    Suppose that $\Omega \subset \R^n$ is a bounded $n$-dimensional submanifold 
    of $\R^n$ and that its boundary $\partial\Omega$ is an orientable 
    $(n-1)$-dimensional submanifold of $\R^n$ whose orientation is a 
    smooth unit vector field $N$ on $\partial\Omega$ pointing outward. 
    Moreover, let $F$ be a smooth vector field on $\Omega$. Then we have 
    \[ \int_\Omega \Div F \dVol = \int_{\partial\Omega} F\cdot\dVol. \] 
\end{theo}\vspace{-0.25cm}

In other words, the integral of the scalar function $\Div F$ on $\Omega$ 
is equal to the integral of the vector field $F$ on $\partial\Omega$ 
with unit normal vector field pointing outward. Note that since $\Omega$ 
is an $n$-dimensional submanifold of $\R^n$, it must be an open set 
so that the integral 
\[ \int_\Omega \Div F \dVol \] 
is just standard integration on a region in $\R^n$. 

{\bf Example.} Let $\Omega = \{(x, y, z) \in \R^3 : x^2 + y^2 + z^2 \leq 1\}$ be the 
unit ball in $\R^3$. Its boundary is the unit sphere $\mathbb S^2 : x^2 + y^2 + z^2 = 1$, 
which is orientable with possible orientations $N = \pm(x, y, z)$, where 
$(x, y, z) \in \mathbb S^2$. To verify Stokes' Theorem on $\Omega$, we need 
to pick the unit normal vector field that is pointing outwards from the 
unit ball, namely $N = (x, y, z)$. Consider the vector field 
$F(x, y, z) = (-y, x, 1 + 2z)$ on $\R^3$ with 
\[ \Div F = \frac{\partial F_1}{\partial x} + \frac{\partial F_2}{\partial y} 
+ \frac{\partial F_3}{\partial z} = 0 + 0 + 2 = 2. \] 
For the left-hand side of Stokes' Theorem, we first see that
\[ \int_\Omega \Div F \dVol = \int_\Omega 2\dVol = 2 \int_\Omega \dd x \dd y \dd z \] 
since $\Omega$ is an open subset of $\R^3$. This integral corresponds to the volume of the unit ball, which is 
\[ \int_\Omega \dd x \dd y \dd z = \frac{4\pi}3, \] 
Therefore, we find that $\int_\Omega \Div F = 8\pi/3$. Now, 
to compute the right-hand side, we observe that 
\[ F \cdot N = (-y, x, 1+2z) \cdot (x, y, z) = -yx + xy + (1+2z)z = z + 2z^2. \] 
Using the spherical coordinates 
\[ \sigma(\theta, \varphi) = (\cos\theta\sin\varphi, \sin\theta\sin\varphi, \cos\varphi) \] 
where $(\theta, \varphi) \in U = (0, 2\pi) \times (0, \pi)$, we have 
\[ (F \cdot N)(\sigma(\theta, \varphi)) = \cos \varphi + 2(\cos \varphi)^2 \] 
and $\|\sigma_\theta \times \sigma_\varphi\| = \sin \varphi > 0$. 
Therefore, we obtain 
\begin{align*}
    \int_{\mathbb S^2} (F \cdot N)\dd A 
    &= \int_U (\cos\varphi + 2(\cos\varphi)^2)\|\sigma_\theta \times \sigma_\varphi\|\dd\theta\dd\varphi \\ 
    &= \int_0^\pi \int_0^{2\pi} (\cos\varphi + 2(\cos\varphi)^2) \sin\varphi \dd\theta \dd \varphi \\ 
    &= 2\pi \int_0^\pi (\cos\varphi + 2(\cos\varphi)^2) \sin\varphi\dd\varphi \\ 
    &= 2\pi \left[ -\frac12(\cos\varphi)^2 - \frac23(\cos\varphi)^3 \right]_0^\pi \\ 
    &= 2\pi \left[ \left( -\frac12 + \frac23 \right) - \left( -\frac12 - \frac23 \right) \right]  
    = \frac{8\pi}{3}.
\end{align*}
This is precisely equal to $\int_\Omega \Div F$, as we expected! 

\subsubsection{Applications} \label{subsubsec:4.3.1}
An important application of Stokes' Theorem is to simplify the computation 
of integrals (when possible). 

Integrating a vector field over a hypersurface that is the union of several 
hypersurfaces can be quite complicated and time consuming. However, 
if this hypersurface bounds a region in $\R^n$ that is easy to describe, it 
may be better to use Stokes' Theorem to convert the integral of the 
vector field into a scalar integral of its divergence. We give a few concrete 
examples of this. 
\begin{enumerate}[(1)]
    \item Suppose we want to integrate the vector field $F(x, y, z) = 
    (x^2y^2, 3z-1, e^x\cos y)$ for $(x, y, z) \in \R^3$ over the boundary 
    $\partial\Omega$ of the rectangular box $\Omega = [-1, 2] \times [4, 5] 
    \times [0, 6]$, with outward unit normal vector field $N$. While the 
    vector field $F$ is smooth and has component functions that are 
    easy to integrate over the boundary $\partial\Omega$ of $\Omega$ 
    which is made up of $2$ horizontal faces and $4$ vertical faces, 
    we would need to compute $6$ integrals in this way! Instead, we can use 
    Stokes' Theorem, which tells us that 
    \[ \int_{\partial\Omega} F\cdot\dVol = \int_\Omega \Div F\dVol 
    = \int_\Omega \Div F \dd x \dd y \dd z \] 
    since $N$ is pointing outward. By setting $F(x, y, z) = 
    (A(x, y, z), B(x, y, z), C(x, y, z))$, we see that 
    \[ \Div F = \frac{\partial A}{\partial x} + \frac{\partial B}{\partial y} 
    + \frac{\partial C}{\partial z} = 2xy^2 + 0 + 0 = 2xy^2. \] 
    It follows that 
    \begin{align*}
        \int_\Omega \Div F \dVol 
        &= \int_0^6 \int_4^5 \int_{-1}^2 2xy^2 \dd x \dd y \dd z \\ 
        &= \int_0^6 \int_4^5 \Big[ x^2 y^2 \Big]_{-1}^2 \dd y \dd z \\ 
        &= 3 \int_0^6 \int_4^5 y^2 \dd y \dd z \\ 
        &= 3 \int_0^6 \left[ \frac{y^3}{3} \right]_4^5 \dd z \\ 
        &= \int_0^6 (5^3 - 4^3)\dd z = 6 \cdot 61 = 366,
    \end{align*}
    and therefore $\int_{\partial\Omega} F \cdot \dVol = 366$. 

    \item Consider the vector field 
    \[ F(x, y, z) = \left( \frac{x^3}{3} + y, \frac{y^3}{3} - \sin(xz), z - x - y \right) \] 
    for $(x, y, z) \in \R^3$ over the boundary $\partial\Omega$ of the solid cylinder 
    \[ \Omega = \{(x, y, z) \in \R^3 : x^2 + y^2 \leq 1,\, 0 \leq z \leq 2\} \] 
    with outward unit normal vector field $N$. The boundary of $\Omega$ 
    is made up of two discs of radius $1$ in the horizontal planes $z = 0$ and 
    $z = 2$, as well as the part of the cylinder $x^2 + y^2 = 1$ for which 
    $0 \leq z \leq 2$. Therefore, computing the integral of $F$ over 
    $\partial\Omega$ would require $3$ parametrizations. Moreover, the integral 
    of $F$ would be quite messy to compute given the $\sin(xz)$ in the second 
    component. Instead, we use Stokes' Theorem, which tells us that 
    \[ \int_{\partial\Omega} F \cdot \dVol = \int_\Omega \Div F \dVol
    = \int_\Omega \Div F \dd x \dd y \dd z \]
    since $N$ is pointing outward. Again, we set $F(x, y, z) = 
    (A(x, y, z), B(x, y, z), C(x, y, z))$ to get 
    \[ \Div F = \frac{\partial A}{\partial x} + \frac{\partial B}{\partial y} 
    + \frac{\partial C}{\partial z} = x^2 + y^2 + 1. \]
    Since $\Omega$ is a solid cylinder, we will use cylindrical coordinates 
    so that $x = r\cos\theta$, $y = r\sin\theta$, and $z = z$ for 
    $0 \leq r \leq 1$, $0 \leq \theta \leq 2\pi$, and $0 \leq z \leq 2$. 
    This gives $x^2 + y^2 + 1 = r^2 + 1$ and ${\rm d}x\dd y \dd z = 
    r\dd r \dd\theta \dd z$, so 
    \begin{align*}
        \int_\Omega \Div F \dVol 
        &= \int_0^2 \int_0^{2\pi} \int_0^1 (r^2 + 1)r \dd r \dd \theta \dd z \\ 
        &= \int_0^2 \int_0^{2\pi} \left[ \frac14r^4 + \frac12r^2 \right]_0^1 \dd\theta \dd z \\ 
        &= \frac34 \int_0^2 \int_0^{2\pi} \dd\theta \dd z = \frac34 \cdot 4\pi = 3\pi. 
    \end{align*}
    We conclude that $\int_{\partial\Omega} F \cdot \dVol = 3\pi$. 
\end{enumerate}

Stokes' Theorem can also be used to relate an integral over a region to 
an integral over its boundary. This is the case in the proof of the 
Gauss-Bonnet Theorem.