\documentclass[10pt]{article}
\usepackage[T1]{fontenc}
\usepackage{amsmath,amssymb,amsthm}
\usepackage{mathtools}
\usepackage[shortlabels]{enumitem}
\usepackage[english]{babel}
\usepackage[utf8]{inputenc}
\usepackage{fancyhdr}
\usepackage{bold-extra}
\usepackage{color}   
\usepackage{tocloft}
\usepackage{graphicx}
\usepackage{lipsum}
\usepackage{wrapfig}
\usepackage{cutwin}
\usepackage{hyperref}
\usepackage{lastpage}
\usepackage{multicol}
\usepackage{tikz}
\usepackage{xcolor}
\usepackage{microtype}
\usepackage[framemethod=TikZ]{mdframed}
\usepackage{mleftright}

% some useful math commands
\newcommand{\eps}{\varepsilon}
\newcommand{\R}{\mathbb{R}}
\newcommand{\C}{\mathbb{C}}
\newcommand{\N}{\mathbb{N}}
\newcommand{\Z}{\mathbb{Z}}
\newcommand{\Q}{\mathbb{Q}}
\newcommand{\K}{\mathbb{K}}
\newcommand{\F}{\mathbb{F}}
\newcommand{\T}{\mathbb{T}}

\numberwithin{equation}{section}

\newcommand{\dd}{\,\mathrm{d}}
\newcommand{\ddz}{\frac{\rm d}{{\rm d}z}}
\newcommand{\pv}{\text{p.v.}}

\renewcommand{\Re}{{\rm Re}}

\DeclareMathOperator{\GL}{GL}
\DeclareMathOperator{\id}{id}
\DeclareMathOperator{\Arg}{Arg}
\DeclareMathOperator{\Log}{Log}
\DeclareMathOperator{\PV}{PV}
\DeclareMathOperator{\sech}{sech}
\DeclareMathOperator{\csch}{csch}
\DeclareMathOperator{\Res}{Res}
\DeclareMathOperator{\Li}{Li}
\DeclareMathOperator{\QR}{QR}
\DeclareMathOperator{\NR}{NR}
\DeclareMathOperator{\lcm}{lcm}
\DeclareMathOperator{\divergence}{div}
\DeclareMathOperator*{\esssup}{ess\,sup}
\DeclareMathOperator{\Span}{span}
\DeclareMathOperator{\Pol}{Pol}
\DeclareMathOperator{\dist}{dist}
\DeclareMathOperator{\Id}{Id}

\DeclarePairedDelimiter\ceil{\lceil}{\rceil}
\DeclarePairedDelimiter\floor{\lfloor}{\rfloor}

\newcommand{\suchthat}{\;\ifnum\currentgrouptype=16 \;\middle|\;\else\mid\fi\;}

% title formatting
\newcommand{\newtitle}[4]{
  \begin{center}
	\huge{\textbf{\textsc{#1 Course Notes}}}
    
	\large{\sc #2}
    
	{\sc #3 \textbullet\, #4 \textbullet\, University of Waterloo}
	\normalsize\vspace{1cm}\hrule
  \end{center}
}

\newcounter{theo}[section]\setcounter{theo}{0}
\renewcommand{\thetheo}{\arabic{section}.\arabic{theo}}
\newenvironment{theo}[2][]{%
\refstepcounter{theo}%
\ifstrempty{#1}%
{\mdfsetup{%
frametitle={%
\tikz[baseline=(current bounding box.east),outer sep=0pt]
\node[anchor=east,rectangle,fill=blue!20]
{\strut {\sc Theorem~\thetheo}};}}
}%
{\mdfsetup{%
frametitle={%
\tikz[baseline=(current bounding box.east),outer sep=0pt]
\node[anchor=east,rectangle,fill=blue!20]
{\strut {\sc Theorem~\thetheo:~#1}};}}%
}%
\mdfsetup{innertopmargin=10pt,linecolor=blue!20,%
linewidth=2pt,topline=true,%
frametitleaboveskip=\dimexpr-\ht\strutbox\relax
}
\begin{mdframed}[nobreak=false]\relax%
\label{#2}}{\end{mdframed}}

%%%%%%%%%%%%%%%%%%%%%%%%%%%%%%
%Definition
\newenvironment{defn}[2][]{%
\refstepcounter{theo}%
\ifstrempty{#1}%
{\mdfsetup{%
frametitle={%
\tikz[baseline=(current bounding box.east),outer sep=0pt]
\node[anchor=east,rectangle,fill=yellow!20]
{\strut {\sc Definition~\thetheo}};}}
}%
{\mdfsetup{%
frametitle={%
\tikz[baseline=(current bounding box.east),outer sep=0pt]
\node[anchor=east,rectangle,fill=yellow!20]
{\strut {\sc Definition~\thetheo:~#1}};}}%
}%
\mdfsetup{innertopmargin=10pt,linecolor=yellow!20,%
linewidth=2pt,topline=true,%
frametitleaboveskip=\dimexpr-\ht\strutbox\relax
}
\begin{mdframed}[nobreak=true]\relax%
\label{#2}}{\end{mdframed}}

%%%%%%%%%%%%%%%%%%%%%%%%%%%%%%
%Example
\newenvironment{exmp}[2][]{%
\refstepcounter{theo}%
\ifstrempty{#1}%
{\mdfsetup{%
frametitle={%
\tikz[baseline=(current bounding box.east),outer sep=0pt]
\node[anchor=east,rectangle,fill=cyan!20]
{\strut {\sc Example~\thetheo}};}}
}%
{\mdfsetup{%
frametitle={%
\tikz[baseline=(current bounding box.east),outer sep=0pt]
\node[anchor=east,rectangle,fill=cyan!20]
{\strut {\sc Example~\thetheo:~#1}};}}%
}%
\mdfsetup{innertopmargin=10pt,linecolor=cyan!20,%
linewidth=2pt,topline=true,%
frametitleaboveskip=\dimexpr-\ht\strutbox\relax
}
\begin{mdframed}[nobreak=false]\relax%
\label{#2}}{\end{mdframed}}

%%%%%%%%%%%%%%%%%%%%%%%%%%%%%%
%Corollary
\newenvironment{cor}[2][]{%
\refstepcounter{theo}%
\ifstrempty{#1}%
{\mdfsetup{%
frametitle={%
\tikz[baseline=(current bounding box.east),outer sep=0pt]
\node[anchor=east,rectangle,fill=lime!20]
{\strut {\sc Corollary~\thetheo}};}}
}%
{\mdfsetup{%
frametitle={%
\tikz[baseline=(current bounding box.east),outer sep=0pt]
\node[anchor=east,rectangle,fill=lime!20]
{\strut {\sc Corollary~\thetheo:~#1}};}}%
}%
\mdfsetup{innertopmargin=10pt,linecolor=lime!20,%
linewidth=2pt,topline=true,%
frametitleaboveskip=\dimexpr-\ht\strutbox\relax
}
\begin{mdframed}[nobreak=true]\relax%
\label{#2}}{\end{mdframed}}

%%%%%%%%%%%%%%%%%%%%%%%%%%%%%%
%Remark
\newenvironment{remark}[2][]{%
\refstepcounter{theo}%
\ifstrempty{#1}%
{\mdfsetup{%
frametitle={%
\tikz[baseline=(current bounding box.east),outer sep=0pt]
\node[anchor=east,rectangle,fill=orange!20]
{\strut {\sc Remark~\thetheo}};}}
}%
{\mdfsetup{%
frametitle={%
\tikz[baseline=(current bounding box.east),outer sep=0pt]
\node[anchor=east,rectangle,fill=orange!20]
{\strut {\sc Remark~\thetheo:~#1}};}}%
}%
\mdfsetup{innertopmargin=10pt,linecolor=orange!20,%
linewidth=2pt,topline=true,%
frametitleaboveskip=\dimexpr-\ht\strutbox\relax
}
\begin{mdframed}[nobreak=true]\relax%
\label{#2}}{\end{mdframed}}

%%%%%%%%%%%%%%%%%%%%%%%%%%%%%%
%Exercise
\newenvironment{exercise}[2][]{%
\refstepcounter{theo}%
\ifstrempty{#1}%
{\mdfsetup{%
frametitle={%
\tikz[baseline=(current bounding box.east),outer sep=0pt]
\node[anchor=east,rectangle,fill=pink!20]
{\strut {\sc Exercise~\thetheo}};}}
}%
{\mdfsetup{%
frametitle={%
\tikz[baseline=(current bounding box.east),outer sep=0pt]
\node[anchor=east,rectangle,fill=pink!20]
{\strut {\sc Exercise~\thetheo:~#1}};}}%
}%
\mdfsetup{innertopmargin=10pt,linecolor=pink!20,%
linewidth=2pt,topline=true,%
frametitleaboveskip=\dimexpr-\ht\strutbox\relax
}
\begin{mdframed}[nobreak=true]\relax%
\label{#2}}{\end{mdframed}}

%%%%%%%%%%%%%%%%%%%%%%%%%%%%%%
%Lemma
\newenvironment{lemma}[2][]{%
\refstepcounter{theo}%
\ifstrempty{#1}%
{\mdfsetup{%
frametitle={%
\tikz[baseline=(current bounding box.east),outer sep=0pt]
\node[anchor=east,rectangle,fill=green!20]
{\strut {\sc Lemma~\thetheo}};}}
}%
{\mdfsetup{%
frametitle={%
\tikz[baseline=(current bounding box.east),outer sep=0pt]
\node[anchor=east,rectangle,fill=green!20]
{\strut {\sc Lemma~\thetheo:~#1}};}}%
}%
\mdfsetup{innertopmargin=10pt,linecolor=green!20,%
linewidth=2pt,topline=true,%
frametitleaboveskip=\dimexpr-\ht\strutbox\relax
}
\begin{mdframed}[nobreak=true]\relax%
\label{#2}}{\end{mdframed}}

%%%%%%%%%%%%%%%%%%%%%%%%%%%%%%
%Proposition
\newenvironment{prop}[2][]{%
\refstepcounter{theo}%
\ifstrempty{#1}%
{\mdfsetup{%
frametitle={%
\tikz[baseline=(current bounding box.east),outer sep=0pt]
\node[anchor=east,rectangle,fill=purple!20]
{\strut {\sc Proposition~\thetheo}};}}
}%
{\mdfsetup{%
frametitle={%
\tikz[baseline=(current bounding box.east),outer sep=0pt]
\node[anchor=east,rectangle,fill=purple!20]
{\strut {\sc Proposition~\thetheo:~#1}};}}%
}%
\mdfsetup{innertopmargin=10pt,linecolor=purple!20,%
linewidth=2pt,topline=true,%
frametitleaboveskip=\dimexpr-\ht\strutbox\relax
}
\begin{mdframed}[nobreak=true]\relax%
\label{#2}}{\end{mdframed}}

%%%%%%%%%%%%%%%%%%%%%%%%%%%%%%
%Fact
\newenvironment{fact}[2][]{%
\refstepcounter{theo}%
\ifstrempty{#1}%
{\mdfsetup{%
frametitle={%
\tikz[baseline=(current bounding box.east),outer sep=0pt]
\node[anchor=east,rectangle,fill=gray!20]
{\strut {\sc Fact~\thetheo}};}}
}%
{\mdfsetup{%
frametitle={%
\tikz[baseline=(current bounding box.east),outer sep=0pt]
\node[anchor=east,rectangle,fill=gray!20]
{\strut {\sc Fact~\thetheo:~#1}};}}%
}%
\mdfsetup{innertopmargin=10pt,linecolor=gray!20,%
linewidth=2pt,topline=true,%
frametitleaboveskip=\dimexpr-\ht\strutbox\relax
}
\begin{mdframed}[nobreak=true]\relax%
\label{#2}}{\end{mdframed}}

% new proof environment
\newenvironment{pf}[1][\proofname]
  {\par\noindent\normalfont\textbf{Proof of #1.}\par\nopagebreak%
  \begin{mdframed}[
     linewidth=1pt,
     linecolor=black,
     bottomline=false,topline=false,rightline=false,
     innerrightmargin=0pt,innertopmargin=0pt,innerbottommargin=0pt,
     innerleftmargin=1em,% Distance between vertical rule & proof content
     skipabove=0.75\baselineskip
   ]}
  {\end{mdframed}}

% 1-inch margins
\topmargin 0pt
\advance \topmargin by -\headheight
\advance \topmargin by -\headsep
\textheight 8.9in
\oddsidemargin 0pt
\evensidemargin \oddsidemargin
\marginparwidth 0.5in
\textwidth 6.5in

\parindent 0in
\parskip 1.5ex

\setlist[itemize]{topsep=0pt}
\setlist[enumerate]{topsep=0pt}

\newcommand{\pushright}[1]{\ifmeasuring@#1\else\omit\hfill$\displaystyle#1$\fi\ignorespaces}

% hyperlinks
\hypersetup{
  colorlinks=true, 
  linktoc=all,     % table of contents is clickable  
  allcolors=red    % all hyperlink colours
}

% table of contents
\addto\captionsenglish{
  \renewcommand{\contentsname}%
    {Table of Contents}%
}
\renewcommand{\cftsecfont}{\normalfont}
\renewcommand{\cftsecpagefont}{\normalfont}
\cftsetindents{section}{0em}{2em}

\fancypagestyle{plain}{%
\fancyhf{} % clear all header and footer fields
\lhead{PMATH 365: Winter 2023}
\fancyhead[R]{Table of Contents}
%\headrule
\fancyfoot[R]{{\small Page \thepage\ of \pageref*{LastPage}}}
}

% headers and footers
\pagestyle{fancy}
\renewcommand{\sectionmark}[1]{\markboth{#1}{#1}}
\lhead{PMATH 365: Winter 2023}
\cfoot{}
\setlength\headheight{14pt}

%\setcounter{section}{-1}

\begin{document}

\pagestyle{fancy}
\newtitle{PMATH 365}{Differential Geometry}{Ruxandra Moraru}{Winter 2023}
\rhead{Table of Contents}
\rfoot{{\small Page \thepage\ of \pageref*{LastPage}}}

\tableofcontents
\vspace{1cm}\hrule
\fancyhead[R]{\nouppercase\rightmark}
\newpage 
\fancyhead[R]{Section \thesection: \nouppercase\leftmark}

\section{Submanifolds of $\R^n$}\label{sec:1}

\subsection{Preliminaries}\label{subsec:1.1}
To begin, we'll recall some facts about the topology of $\R^n$ and 
vector-valued functions.

In this course, we'll be working with the metric topology with respect 
to the Euclidean norm (or metric). Let $x = (x_1, \dots, x_n),\,
y = (y_1, \dots, y_n) \in \R^n$.
The {\bf Euclidean norm} is defined to be 
\[ \|x\| = \sqrt{x_1^2 + \cdots + x_n^2}, \] 
and {\bf Euclidean distance} is given by 
\[ \dist(x, y) = \|y - x\| = \sqrt{(y_1 - x_1)^2 + 
\cdots + (y_n - x_n)^2}. \] 
We define the {\bf open ball} of radius $r > 0$ centered at $x \in \R^n$ by 
\[ B_r(x) := \{y \in \R^n : \dist(x, y) < r\} \subset \R^n. \] 
A {\bf topology} on $\R^n$ is a collection $\mathcal{U} = \{U_\alpha\}_{\alpha \in A}$ 
of subsets $U_\alpha \subset \R^n$ that satisfy the following properties.
\begin{enumerate}[(i)]
    \item $\varnothing$ and $\R^n$ are in ${\cal U}$. 
    \item For any subcollection $\mathcal{V} = \{U_\beta\}_{\beta \in B}$ 
    with $U_\beta \in {\cal U}$ for all $\beta \in B$, we have 
    $\bigcup_{\beta \in B} U_\beta \in {\cal U}$. 
    \item For any \emph{finite} subcollection $\{U_{\alpha_1}, \dots, 
    U_{\alpha_m}\} \subset {\cal U}$, we have $\bigcap_{i=1}^m 
    U_{\alpha_i} \in {\cal U}$. 
\end{enumerate}
The sets $U_\alpha \in {\cal U}$ are called the {\bf open sets} of the topology; 
their complements $F_\alpha = \R^n \setminus U_\alpha$ are called the 
{\bf closed sets}. 

Note that the sets $\varnothing$ and $\R^n$ are both open and closed. 
Moreover, the notion of a topology can be extended to more general sets 
$X$, not just $\R^n$. A topology can also be defined starting with closed sets,
but we prefer to work with open sets because many nice properties, such as 
differentiability, are better described with them.

Under the metric topology, we say that a set $A \subset \R^n$ is {\bf open} 
if $A = \varnothing$ or if for all $p \in A$, there exists $r > 0$ 
such that $B_r(p) \subset A$. Moreover, $A$ is {\bf closed} if its 
complement $A^c = \R^n \setminus A$ is open. (We leave it as an 
exercise to show that this is indeed a topology.) 

For example, the open balls $B_r(x)$ are open sets for all 
$x \in \R^n$ and $r > 0$. Indeed, for any point $p \in 
B_r(x)$, one sees that by picking $r' = (r - \|p - x\|)/2$, 
we have $B_{r'}(p) \subset B_r(x)$. 

In general, open sets are described with strict inequalities, while closed 
sets are described using equality or inclusive inequalities. However, note 
that most sets are neither open nor closed, such as the half-open interval 
$U = (-1, 1]$ over $\R$. 

The metric topology is not the only topology on $\R^n$; one example is 
the one consisting of only the sets ${\cal U} = \{\varnothing, \R^n\}$.
However, we generally want more open sets to work with since we might want to 
know the behaviour of functions around a point $p \in \R^n$. If the 
only non-empty open set we had was $\R^n$, then this would apply to all points 
in $\R^n$, which does not yield a lot of information. 

Let $p \in \R^n$. The previous paragraph leads us to the definition of an 
{\bf open neighbourhood} of $p$, which is just an open set $U \subset \R^n$ 
such that $p \in U$. 

We now turn our discussion to vector-valued functions. Let $U \subset \R^n$ 
and consider the vector-valued function 
\begin{align*}
    F : U \subset \R^n &\to B \subset \R^m \\ 
    x = (x_1, \dots, x_n) &\mapsto (F_1(x), \dots, F_m(x)). 
\end{align*}
Then $F$ is continuous if and only if the component functions 
$F_i : U \to \R$ are continuous for all $i = 1, \dots, m$. 

We say that $F$ is a {\bf homeomorphism} if it is a continuous bijection 
whose inverse 
\[ F^{-1} : B \subset \R^m \to U \subset \R^n \] 
is also continuous. For example, the identity map $\Id_{\R^n} : \R^n \to \R^n$ 
and the function $f : \R \to \R$ defined by $f(x) = x^3$ are both homeomorphisms. 

It is a known fact that homeomorphisms map open sets to open sets and closed 
sets to closed sets. This follows from the topological characterization 
of continuity, which states that $F$ is continuous if and only if for 
every open (respectively closed) set $V \subset \R^m$, we have that 
$F^{-1}(V)$ is open (respectively closed). In fact, homeomorphisms preserve 
much more structure than this, as we'll see later. 

\subsection{Topological submanifolds of $\R^n$}\label{subsec:1.2}
We now define the main object we'll be working with in this course.

\begin{defn}{defn:1.1}
    A {\bf $k$-dimensional topological submanifold} (or 
    {\bf topological $k$-submanifold}) of $\R^n$ is a subset $M \subset \R^n$ 
    such that for every $p \in M$, there exists an open neighbourhood $V$ 
    of $p$ in $\R^n$, an open set $U \subset \R^k$, and a homeomorphism 
    \[ \alpha : U \subset \R^k \to V \cap M \subset \R^n. \] 
    The homeomorphism $\alpha$ is called a {\bf coordinate chart} 
    (or {\bf patch}) on $M$.
\end{defn}\vspace{-0.25cm}

Note that the open neighbourhood $V \subset \R^n$ of $p$, the 
open set $U \subset \R^k$, and the map $\alpha$ do not need to be unique.
But we'll see later that the dimension $k$ must be unique and is completely 
determined by $M$.

For example, $\R^n$ is a topological $n$-submanifold of $\R^n$ 
by taking $U = V = \R^n$ and $\alpha = \Id_{\R^n}$. Any open set 
$W \subset \R^n$ is a topological $n$-submanifold of $\R^n$ by 
taking $U = V = W$ and $\alpha = \Id_W$. 

Let's now consider some non-trivial examples. Consider 
\[ M = \{(x, y) \in \R^2 : y = x^2\} \subset \R^2, \] 
which is the graph of the parabola $f(x) = x^2$. Then $M$ is a 
topological $1$-submanifold of $\R^2$ by considering the map 
$\alpha : \R^1 \to M \subset \R^2,\, t \mapsto (t, t^2)$.
The inverse $\alpha^{-1} : M \to \R^1$ is just the projection of the 
first coordinate, which is continuous. 

More generally, let $U \subset \R^k$ be an open set. Consider the graph of a 
continuous function 
\[ F : U \subset \R^k \to \R^{n-k},\, x \mapsto (F_1(x), \dots, F_{n-k}(x)). \]
In other words, we are looking at the set 
\[ G = \{(x, y) \in \R^k \times \R^{n-k} : y = F(x),\, x \in U\} \subset \R^n. \] 
We claim that $G$ is a $k$-dimensional topological submanifold of $\R^n$. To 
see this, define $\alpha : U \subset \R^k \to G \subset \R^n$ by 
$x \mapsto (x, F(x))$. Then $\alpha$ is continuous since $F$ is continuous, 
and it is a bijection since we are restricted to $G$. Moreover, 
it has continuous inverse $\alpha^{-1} : G \subset \R^n \to 
U \subset \R^k,\, (x, y) \mapsto x$. 

Here are two more examples of this in action. 
\begin{enumerate}[(1)]
    \item Let $M = \{(x, y, z) \in \R^3 : z = x^2 + y^2\} \subset \R^3$. 
    Then $M$ is the graph of the continuous function $f(x, y) = x^2 + y^2$, 
    so it is a $2$-dimensional topological submanifold of $\R^3$. 
    \item Observe that $M = \{(x, y, z) \in \R^3 : y = x^2,\, z = x^3\} 
    \subset \R^3$ is the graph of the continuous function $F(t) = (t^2, t^3)$, 
    so it is a $1$-dimensional topological submanifold of $\R^3$.
\end{enumerate}
In all the examples above, we only needed one coordinate chart which 
worked for all points. However, this is not always the case! Consider 
the unit circle 
\[ \mathbb{S}^1 := \{(x, y) \in \R^2 : x^2 + y^2 = 1 \} \subset \R^2. \] 
Note that $\mathbb{S}^1$ is compact. Therefore, by Heine-Borel, 
it is closed and bounded. Recall that homeomorphisms preserve closed sets, 
so it is impossible to find a unique chart $\alpha$. Indeed, if we had 
such a homeomorphism $\alpha : U \subset \R^1 \to \mathbb{S}^1 \subset \R^2$ 
for some open set $U$, then $U = \alpha^{-1}(\mathbb{S}^1)$ would be 
a compact subset of $\R^1$. But the only open and compact subset of 
$\R^n$ is $\varnothing$, which is a contradiction! 

Nonetheless, two coordinate charts are enough to cover all points on 
$\mathbb{S}^1$. Define 
\begin{align*}
    V_1 = \R^2 \setminus \{(x, 0) \in \R^2 : x \leq 0\}, \\ 
    V_2 = \R^2 \setminus \{(x, 0) \in \R^2 : x \geq 0\},
\end{align*}
which are both open sets. Then the homeomorphism 
\[ \alpha_1 : U_1 = (-\pi, \pi) \to \mathbb{S}^1 \cap V_1,\,
 t \mapsto (\cos t, \sin t) \]
covers all points on $\mathbb{S}^1$ except for $(-1, 0)$, while 
\[ \alpha_2 : U_2 = (0, 2\pi) \to \mathbb{S}^1 \cap V_2,\,
 t \mapsto (\cos t, \sin t) \]
covers all points on $\mathbb{S}^1$ except for $(1, 0)$. 

\subsection{More preliminaries} \label{subsec:1.3}
We now introduce another definition from topology.

\begin{defn}{defn:1.2}
    Let $A \subset \R^n$. A subset $U \subset A$ is {\bf relatively open} 
    if it is of the form $U = A \cap U'$ for some open set $U' \subset \R^n$. 
    Similarly, we say that $F \subset A$ is {\bf relatively closed} 
    if $F = A \cap F'$ for some closed set $F' \subset \R^n$. 
\end{defn}\vspace{-0.25cm}

For example, consider $\R$ equipped with the metric topology so that 
the open (respectively closed) sets are the unions of open intervals 
(respectively finite intersections of closed intervals). Let 
$A = [-1, 2) \subset \R$, which is neither open nor closed in $\R$. 
Take $U = [-1, 1) \subset \R$, which is again neither open nor closed in $\R$. 
But $U$ is relatively open in $A$ since $U = A \cap (-3, 1)$. Similarly, 
$F = [-1, 1] = A \cap [-1, 1]$ is relatively closed in $A$. 

Using the language of relatively open and closed sets, a lot of statements 
can be made simpler.
\begin{enumerate}[(1)]
    \item We define an {\bf open neighbourhood of $p$ in $A$} to be a relatively 
    open set $U$ containing $p$.
    \item The relatively open sets form a topology on $A$, called the 
    {\bf relative topology} (verify this as an exercise).
    \item We now have a more concise definition of a topological submanifold. 
    Let $M \subseteq \R^n$. Then $M$ is a {\bf $k$-dimensional topological 
    submanifold} of $\R^n$ if for every $p \in M$, there exists an 
    open neighbourhood $V$ of $p$ in $M$, an open set $U \subset \R^k$, 
    and a homeomorphism $\alpha : U \subset \R^k \to V \subset \R^n$. 
\end{enumerate}

\begin{defn}{defn:1.3}
    Let $A \subset \R^n$. Then $A$ is {\bf connected} if it cannot be written 
    in the form $A = U \cup V$ where $U, V \neq \varnothing$ are relatively 
    open in $A$ and $U \cap V = \varnothing$. Otherwise, we say that $A$ 
    is {\bf disconnected}; we call $U$ and $V$ {\bf disconnecting sets} for $A$.
\end{defn}\vspace{-0.25cm}

Let's go over a few example of connected sets. 
It can be shown that an open set in $\R^n$ is connected if and only if 
it is path connected; that is, there is a path between any two points in the set.
This result can help us build some intuition for what a connected set should 
look like. 
\begin{enumerate}[(1)]
    \item $\R^n$ is connected.
    \item Let $\alpha < \beta \in \R$. Then $(\alpha, \beta)$, 
    $(\alpha, \beta]$, $[\alpha, \beta)$, and $[\alpha, \beta]$ are 
    all connected. 
    \item Observe that $A = (-1, 0] \cup [1, 2]$ is a disconnected set because 
    $(-1, 0] = A \cap (-1.3, 0.3)$ and $[1, 2] = A \cap (0.9, 2.1)$ are 
    both relatively open in $A$ and disjoint. 
    \item The open ball $B_r(p)$ is connected for all $p \in \R^n$ and $r > 0$. 
\end{enumerate}
An important property of connected sets is that the continuous image of a 
connected set is connected! This can be used to prove that a subset 
$M \subset \R^n$ is not a submanifold. 

\begin{enumerate}[(1)]
    \item Consider the {\bf $\alpha$-curve} $C := \{(x, y) \in \R^2 : y^2 = x^2(x+1)\} \subset \R^2$.
    This can be parametrized by the map 
    \begin{align*}
        \alpha : \R &\to C \subset \R^2 \\
        t &\mapsto (t^2 - 1, t(t^2 - 1)).
    \end{align*}
    Note that $\alpha$ is not injective since $\alpha(-1) = \alpha(1) = (0, 0)$.
    That is, $\alpha$ is not a homeomorphism on $\R$, but it becomes one 
    if we remove the points $t = \pm1$, whose inverse is 
    \begin{align*}
        \alpha^{-1} : C \setminus \{(0, 0)\} &\to \R \setminus \{\pm1\} \subset \R \\ 
        (x, y) &\mapsto 1/x. 
    \end{align*} 
    Thus, $C$ is a $1$-dimensional submanifold away from the point $(0, 0)$. 
    
    Our goal now is to show that the whole of $C$ is not a topological submanifold 
    of $\R^2$. By contradiction, suppose that it were. By our above discussion, 
    it must have dimension $1$ because it has dimension $1$ away from $(0, 0)$. 
    Since $(0, 0) \in C$, it follows from the definition that there exists an 
    open neighbourhood $V$ of $(0, 0)$ in $C$, 
    an open set $U \subset \R^1$, and a homeomorphism 
    \[ \alpha : U \subset \R^2 \to V \subset C. \] 
    There must be a unique point $t_0 \in U$ such that $\alpha(t_0) = (0, 0)$ 
    since $\alpha$ is a bijection. Since $U$ is open and $t_0 \in U$, there 
    exists $\eps > 0$ such that $B_\eps(t_0) \subset U$. But $U \subset \R^1$, 
    so $B_\eps(t_0) = (t_0 - \eps, t_0 + \eps) =: U'$. Then $\alpha|_{U'}$ 
    is also a homeomorphism. Let $V' = \alpha(U')$. Observe that 
    $V' \setminus \{(0, 0)\}$ has three or four pieces depending on how large 
    the open set $U'$ is: one on the top right quadrant, one on the bottom right 
    quadrant, and one or two on the left of the $y$-axis. On the other hand, 
    $U' \setminus \{t_0\}$ has only two components, contradicting the fact that 
    homeomorphisms preserve the number of connected components. 

    \item Consider the {\bf double cone} $M = \{(x, y, z) \in \R^3 : 
    z^2 = x^2 + y^2\} \subset \R^3$. Away from $(0, 0, 0)$, every point in $M$ 
    lies on the graph of one of the continuous functions $f_1(x, y) =
    \sqrt{x^2+y^2}$ or $f_2(x, y) = -\sqrt{x^2 + y^2}$. Therefore, 
    $M \setminus \{(0, 0, 0)\}$ is a $2$-dimensional topological submanifold 
    of $\R^3$ since $f_1$ and $f_2$ are both functions of two variables. 

    However, there is a problem at the point $(0, 0, 0)$ since it 
    lies on the graph of both $f_1$ and $f_2$. Suppose that $M$ 
    is a topological submanifold of $\R^3$. Then $M$ must necessarily be of 
    dimension $2$ because $M \setminus \{(0, 0, 0)\}$ is of dimension $2$. 
    Then by definition, there exists an open neighbourhood $V$ of $(0, 0, 0)$, an 
    open set $U \subset \R^2$, and a homeomorphism 
    \[ \alpha : U \subset \R^2 \to V \subset M. \] 
    Since $\alpha$ is a bijection, there exists a unique point $(x_0, y_0) \in U$ 
    such that $\alpha(x_0, y_0) = (0, 0, 0)$. After shrinking $U$ (by the 
    same argument as above), we may take $U' = B_{\eps}((x_0, y_0))$ to 
    ensure that we have a connected set. Consider now the restriction 
    $\alpha|_{U'} : U' \to V = \alpha(U')$. Then $U' \setminus \{(x_0, y_0)\}$ 
    has one component, whereas $V' \setminus \{(0, 0, 0)\}$ has two components
    (that is, it is disconnected), which is a contradiction. 
\end{enumerate}

\newpage 
Now, we want to prove the invariance of dimension.
\begin{theo}[Invariance of Dimension]{theo:1.4}
    $\R^m$ is homeomorphic to $\R^n$ if and only if $m = n$. 
\end{theo}\vspace{-0.25cm}
If $m = n$, then $\R^m = \R^n$, so there is nothing to prove. The other 
implication is much harder, and we'll need the following result. 

\begin{theo}[Brouwer Invariance of Domain]{theo:1.5}
    Let $U \subset \R^n$ be open, and let $f : U \subset \R^n \to \R^n$ 
    be an injective continuous map. Then $f(U) \subset \R^n$ is open. 
    In particular, $f$ is a homeomorphism onto its image.
\end{theo}\vspace{-0.25cm}

An elementary proof can be found on 
\href{https://terrytao.wordpress.com/2011/06/13/brouwers-fixed-point-and-invariance-of-domain-theorems-and-hilberts-fifth-problem/}{Terry Tao's blog} 
where he uses the Brouwer Fixed Point Theorem to prove it. Nowadays, the 
standard proof uses algebraic topology. 

It is important that both the domain and codomain involve the same dimension $n$. 
For example, consider the injective continuous map $f : \R \to \R^2$ defined by 
$x \mapsto (x, 0)$. Observe that $f(U)$ is the $x$-axis, which is not open in $\R^2$. 

\begin{pf}[Theorem~\ref{theo:1.4}]
    We proceed by contradiction. Suppose that there is a homeomorphism 
    $f : \R^m \to \R^n$ and that $m > n$. Consider the inclusion 
    \begin{align*}
        \iota : \R^n &\to \R^m \\ 
        (x_1, \dots, x_n) &\mapsto (x_1, \dots, x_m, 0, \dots 0),
    \end{align*} 
    which is an injective continuous map. Then $\iota \circ f : \R^m 
    \to \R^m$ is also an injective continuous map since it is the composition of 
    two injective continuous maps. By Theorem~\ref{theo:1.5}, we have that 
    $\iota \circ f(\R^m)$ is an open set in $\R^m$. But this is impossible 
    because if $(x_1, \dots, x_n, 0, \dots, 0) \in \iota \circ f(\R^m)$, 
    then 
    \[ (x_1, \dots, x_n, \eps/2, 0, \dots, 0) \notin \iota \circ f(\R^m) \]  
    for all $\eps > 0$. Then $B_\eps((x_1, \dots, x_n, 0, \dots, 0)) 
    \not\subset \iota \circ f(\R^m)$ for all $\eps > 0$, implying that 
    $\iota \circ f(\R^m)$ is not open in $\R^m$. Therefore, we must have 
    $m \leq n$. If $n < m$, then we can repeat the same argument with 
    $f^{-1} : \R^n \to \R^m$, which again leads to a contradiction. 
    We conclude that $n = m$. \qed 
\end{pf}\vspace{-0.25cm}

Note that we actually proved something stronger: if $m > n$ and $U$ is a 
nonempty open subset of $\R^m$, then there is no continuous mapping from 
$U$ to $\R^n$. As a consequence, we get the following. 
\begin{prop}{prop:1.6}
    If $M \subset \R^n$ is a $k$-dimensional topological submanifold of 
    $\R^n$, then $k \leq n$. 
\end{prop}\vspace{-0.25cm}
\begin{pf}[Proposition~\ref{prop:1.6}]
    If $M \subset \R^n$ is a $k$-dimensional topological submanifold of $\R^n$, 
    then for all $p \in M$, there exists an open set $U \subset \R^k$, 
    an open neighbourhood $V \subset M$ of $p$, and a homeomorphism
    \[ \alpha : U \subset \R^k \to V \subset M \subset \R^n. \] 
    Since $\alpha$ is an injective continuous map, this forces $k \leq n$ 
    by the above discussion. \qed
\end{pf}\vspace{-0.25cm}
Finally, we must have the same $k$ for any chart $\alpha : U \subset \R^k 
\to V \subset M$. Indeed, let $p \in M$, and suppose that we have two 
different charts, say $\alpha : U \subset \R^k \to V \subset M$ and 
$\beta : U' \subset \R^{k'} \to V' \subset M$ where $p \in V \cap V'$. 
Then $V \cap V' \neq \varnothing$, so we can consider the restrictions 
\begin{align*}
    \alpha|_{\alpha^{-1}(V \cap V')} &: \alpha^{-1}(V \cap V') \to V \cap V', \\
    \beta|_{\beta^{-1}(V \cap V')} &: \beta^{-1}(V \cap V') \to V \cap V'.
\end{align*} 
Then $\beta^{-1} \circ \alpha : \alpha^{-1}(V \cap V') \subset \R^k 
\to \beta^{-1}(V \cap V') \subset \R^{k'}$ is a homeomorphism. Hence, 
$\alpha^{-1}(V \cap V')$ is a $k$-dimensional topological submanifold
of $\R^{k'}$. By Proposition~\ref{prop:1.6}, we have $k \leq k'$. 
Similarly, $\alpha^{-1} \circ \beta : \beta^{-1}(V \cap V') \subset 
\R^{k'} \to \alpha^{-1}(V \cap V') \subset \R^k$ is a homeomorphism, so 
$\beta^{-1}(V \cap V')$ is a $k'$-dimensional submanifold of $\R^k$. 
It follows that $k' \leq k$ and so $k' = k$. 

\subsection{Submanifolds of $\R^n$ of class $C^r$} \label{subsec:1.4}
Let $U \subset \R^n$ be an open set and consider the vector-valued function 
\begin{align*}
    F : U \subset \R^n &\to \R^m \\ 
    x = (x_1, \dots, x_n) &\mapsto (F_1(x), \dots, F_m(x)). 
\end{align*}
Recall that $F$ is of {\bf class $C^r$} for $r \geq 1$ if each component 
function $F_i : U \subset \R^n \to \R$ is of class $C^r$. That is, 
the partial derivatives of $F_i$ exist and are continuous up to order $r$. 
Also, we say that $F$ is of class $C^\infty$ or {\bf smooth} if each 
$F_i$ is smooth (the partial derivatives exist up to any order).
\begin{enumerate}[(1)]
    \item All polynomials are smooth. 
    \item The function $f(x) = x^{4/3}$ is of class $C^1$. Its derivative 
    $f'(x) = \frac43 x^{1/3}$ is continuous, but the second derivative 
    $f''(x) = \frac49 x^{-2/3}$ is not defined at $x = 0$. 
    \item The vector-valued function $F(x, y) = (2\cos x, xy - 1, e^{2\sin y+x})$ 
    is smooth on $\R^2$ because each component function is smooth. 
\end{enumerate}
The {\bf partial derivative} of $F$ with respect to the variable $x_j$ is 
\[ \frac{\partial F}{\partial x_j} := \left( \frac{\partial F_1}{\partial x_j}, 
\dots, \frac{\partial F_m}{\partial x_j} \right). \] 
If we fix a component function $F_i$, its {\bf gradient} is 
\[ \nabla F_i := \left( \frac{\partial F_i}{\partial x_1}, 
\dots, \frac{\partial F_i}{\partial x_n} \right). \] 
The {\bf derivative matrix} or {\bf Jacobian matrix} of $F$ is the $m \times n$ 
matrix 
\[ DF := \begin{bmatrix} 
    \partial F_1/\partial x_1 & \cdots & \partial F_1/\partial x_n \\ 
    \vdots & \ddots & \vdots \\ 
    \partial F_m/\partial x_1 & \cdots & \partial F_m/\partial x_n
\end{bmatrix}. \] 
That is, the rows correspond to the component functions $F_i$, and the 
columns correspond to the variables $x_j$. We can also think of the 
rows as the gradients and the columns as the partial derivatives; that is, 
we have 
\[ DF = \mleft[ \begin{array}{c|c|c}
    \dfrac{\partial F}{\partial x_1} & \cdots & \dfrac{\partial F}{\partial x_n}
\end{array} \mright] = \mleft[ \begin{array}{c}
    \nabla F_1 \\ \hline 
    \vdots \\[2pt] \hline 
    \nabla F_m \\
\end{array} \mright] \] 
In general, we want to work with some differentiability. 
This leads to the following definition. 

\begin{defn}{defn:1.7}
    Let $M \subset \R^n$. Suppose that for every $p \in M$, there exists 
    an open neighbourhood $V$ of $p$ in $M$, an open subset $U \subset \R^k$, 
    and a homeomorphism $\alpha : U \subset \R^k \to V \subset M$ such that 
    \begin{enumerate}[(1)]
        \item $\alpha$ is of class $C^r$ for some $r \geq 1$; 
        \item $D\alpha(x)$ has rank $k$ for all $x \in U$. 
    \end{enumerate}
    Then $M$ is called a {\bf $k$-dimensional submanifold of $\R^n$ of 
    class $C^r$}. We call $\alpha$ a {\bf coordinate chart} (or 
    {\bf coordinate patch}) about $p$.
\end{defn}\vspace{-0.25cm}

Note that every submanifold of class $C^r$ is a topological submanifold. 
We are only imposing the extra conditions (1) and (2) on the coordinate charts. 
We will see that condition (2) will allow us to define tangent spaces to 
the submanifolds at every point. A submanifold of class $C^\infty$ is 
called a {\bf smooth submanifold}. 

As usual, let's go over some examples. 
\begin{enumerate}[(1)]
    \item Let $U \subset \R^n$ be open. Then $\alpha : U \subset \R^n \to 
    V = U \subset \R^n$ sending $x$ to itself is smooth. Since the 
    component functions are $F_i(x) = x_i$ for all $i = 1, \dots, n$, we have 
    \[ D\alpha(x) = \left[ \frac{\partial F_i}{\partial x_j} \right] 
    = \left[ \frac{\partial x_i}{\partial x_j} \right] = [\delta_{ij}], \] 
    where $\delta_{ij}$ is the Kronecker delta. In other words, $D\alpha(x)$ 
    is the $n \times n$ identity matrix and has rank $n$ for all $x \in U$, 
    so $U \subset \R^n$ is a smooth $n$-dimensional submanifold of $\R^n$. 

    \item {\bf Graphs of functions of class $C^r$.} Let $U \subset \R^k$ 
    be an open set and consider a function 
    \begin{align*}
        F : U \subset \R^k &\to \R^{n-k} \\ 
        (x_1, \dots, x_k) &\mapsto (F_1(x), \dots, F_{n-k}(x)) 
    \end{align*} 
    of class $C^r$ (so each $F_i$ is of class $C^r$). Let 
    \[ M = \{(x, F(x)) \in \R^k \times \R^{n-k} : x \in U\} \subset \R^n \] 
    be the graph of $F$. We have already seen that $M$ is a $k$-dimensional 
    submanifold of $\R^n$ by taking $V = M$ and the homeomorphism 
    $\alpha : U \subset \R^k \to V \subset \R^n$ defined by 
    \[ \alpha(x) = (x, F(x)) = (x_1, \dots, x_k, F_1(x), \dots, F_{n-k}(x)). \]
    In particular, $\alpha$ is of class $C^r$ since the identity components 
    are smooth and the $F_i$ are of class $C^r$. Let's look at the 
    derivative matrix in terms of the columns of partial derivatives. We have 
    \[ \frac{\partial\alpha}{\partial x_j} = 
    \left( 0, \dots, 0, 1, 0, \dots, 0, \frac{\partial F_1}{\partial x_j}, 
    \dots, \frac{\partial F_{n-k}}{\partial x_j} \right) \] 
    where the $1$ corresponds to the $j$-th component, and hence 
    \[ D\alpha(x) = \mleft[ \begin{array}{cccc}
        1 & 0 & \cdots & 0 \\ 
        0 & 1 & \cdots & 0 \\ 
        \vdots & \vdots & \ddots & \vdots \\ 
        0 & 0 & \cdots & 1 \\ \hline 
        \partial F_1/\partial x_1 & \partial F_1/\partial x_2 & \cdots & \partial F_1/\partial x_k \\ 
        \vdots & \vdots & \ddots & \vdots \\ 
        \partial F_{n-k}/\partial x_1 & \partial F_{n-k}/\partial x_2 & \cdots & \partial F_{n-k}/\partial x_k \\
    \end{array} \mright] = \mleft[ \begin{array}{c} I_{k\times k} \\\hline DF(x) \end{array} \mright]. \]
    This matrix has rank $k$ for all $x \in U$, so $M$ is a $k$-dimensional 
    submanifold of class $C^r$. 

    \item We saw that the circle $\mathbb{S}^1 = \{(x, y) \in \R^2 : x^2 + y^2 = 1\} 
    \subset \R^2$ was a $1$-dimensional topological submanifold using
    the charts $\alpha_1 : U_1 = (-\pi, \pi) \subset \R^1 \to 
    \mathbb{S}^1 \setminus \{(-1, 0)\} \subset \R^2$ and 
    $\alpha_2 : U_2 = (0, 2\pi) \subset \R^1 \to \mathbb{S}^1 \setminus 
    \{(1, 0)\} \subset \R^2$, both defined by $t \mapsto (\cos t, \sin t)$. 
    Note that both $\alpha_i$ are smooth functions with derivative matrix 
    \[ D\alpha_i = \left[ \frac{\textrm{d}\alpha_i}{\textrm{d}t} \right] = 
    \begin{bmatrix} -\sin t \\ \cos t \end{bmatrix}, \] 
    which has rank $1$ because $\sin t$ and $\cos t$ don't have the same zeroes, 
    and hence $D\alpha_i$ is never the zero vector. Thus, $\mathbb{S}^1$ 
    is a smooth $1$-dimensional submanifold of $\R^2$. 
\end{enumerate}
Not every topological submanifold of $\R^n$ is of class $C^r$ 
for some $r \geq 1$. For example, consider the graph 
\[ M = \{(x, |x|) : x \in \R\} \] 
of the function $f(x) = |x|$ on $\R$, which fails to be differentiable 
at $x = 0$. Since $f$ is continuous, we know that 
$M$ is a $1$-dimensional topological submanifold of $\R^2$. Note that $f$ is 
smooth away from $x = 0$, so $M \setminus \{(0, 0)\}$ is a smooth $1$-dimensional
submanifold of $\R^2$. 

However, we claim that it cannot be a submanifold of 
class $C^r$ on any neighbourhood of the point $(0, 0)$. 
Suppose otherwise, so there exists an open set $U \subset \R^1$, 
an open neighbourhood $V \subset M$ of $(0, 0)$, and a homeomorphism 
$\alpha : U \subset \R^1 \to V \subset M \subset \R^2$ of class $C^r$ 
for some $r \geq 1$. Moreover, assume that $D\alpha(t)$ is of rank $1$ 
for all $t \in U$. Since $k = 1$, we have 
\[ D\alpha(t) = \frac{{\rm d}\alpha}{{\rm d}t} = \alpha'(t) \neq 0 \] 
for all $t \in U$ using the rank $1$ assumption. But $\alpha'(t)$ 
is tangent to $M$ at $\alpha(t)$. There are two possibilities: 
\begin{itemize}
    \item If $\alpha(t)$ is on the line $y = x$, then $\alpha'(t)$ 
    is a direction vector of $y = x$, so for some $c : I 
    \subset \R \to \R$, we have 
    \[ \alpha'(t) = c(t) \begin{pmatrix} 1 \\ 1 \end{pmatrix}. \tag{1.4.1} \label{eq:1.4.1}\] 
    Since $\alpha(t)$ is of class $C^r$ for some $r \geq 1$, we know that 
    $\alpha'(t)$ is continuous, so $c : I \to \R$ is also continuous.
    \item If $\alpha(t)$ is on $y = -x$, then $\alpha'(t)$ is a 
    direction vector of $y = -x$, so for some $d : I' \subset \R \to \R$, 
    we have 
    \[ \alpha'(t) = d(t) \begin{pmatrix} 1 \\ -1 \end{pmatrix}. \tag{1.4.2} \label{eq:1.4.2} \] 
    The same argument as above shows that $d : I' \to \R$ is continuous.
\end{itemize}
However, since $\alpha$ is a bijection, we have $(0, 0) = \alpha(t_0)$ 
for some $t_0 \in U$. By the continuity of $\alpha'(t)$, we obtain 
\[ \lim_{t\to t_0^-} \alpha'(t) = \lim_{t\to t_0^+} \alpha'(t). \] 
Without loss of generality, assume that $\alpha(t)$ is moving along $M$ 
from left to right. (Otherwise, we can simply parametrize in the other 
direction.) Then if $t < t_0$, equation $\eqref{eq:1.4.2}$ holds, whereas if 
$t > t_0$, equation $\eqref{eq:1.4.1}$ holds. This means that 
\[ \lim_{t\to t_0^-} \alpha'(t) = \lim_{t\to t_0^-} d(t) \begin{pmatrix} 1 \\ -1 \end{pmatrix} 
= \lim_{t\to t_0^+} c(t) \begin{pmatrix} 1 \\ 1 \end{pmatrix} = \lim_{t\to t_0^+} \alpha'(t). \] 
But the above vectors are not parallel to each other, so the only way that 
these limits are equal is if $\lim_{t\to t_0^-} d(t) = 
\lim_{t\to t_0^+} c(t) = 0$. This implies that $\alpha'(t_0)$ is the zero 
vector, which is a contradiction to the fact that $D\alpha(t) = \alpha'(t)$ 
has rank $1$ for all $t \in U$! This concludes the example that not 
every topological submanifold is a submanifold of class $C^r$ for some $r \geq 1$. 

In the definition of a submanifold of class $C^r$, it is important that 
$\alpha$ is a homeomorphism and not just a function of class $C^r$ 
with $D\alpha(x)$ of rank $k$ for all $x \in U$. These conditions alone
don't even ensure that $\alpha$ is injective! For example, take the 
$\alpha$-curve $C = \{(x, y) \in \R^2 : y^2 = x^2(x+1)\} \subset \R^2$ 
which we introduced back in Section~\ref{subsec:1.3}, which could be 
parametrized using $\alpha(t) = (t^2 - 1, t(t^2-1))$ for $t \in \R$. 
We saw that this map was not injective, but it is smooth with derivative matrix 
\[ D\alpha(t) = \frac{{\rm d}\alpha}{{\rm d}t} = \begin{pmatrix} 2t \\ 3t^2 - 1 \end{pmatrix} 
\neq \begin{pmatrix} 0 \\ 0 \end{pmatrix} \] 
for all $t \in \R$. A map satisfying these two conditions is said to be 
an {\bf immersion}, and a topological submanifold whose maps 
are immersions is called an {\bf immersed manifold}. We record this in 
the definition below. 

\begin{defn}{defn:1.8}
    Let $U \subset \R^k$ be open with $k \leq n$. A map $\alpha : U 
    \subset \R^k \to \R^n$ is an {\bf immersion} (of class $C^r$) if 
    \begin{enumerate}[(1)]
        \item $\alpha$ is of class $C^r$; and 
        \item $D\alpha(x)$ has rank $k$ for all $x \in U$. 
    \end{enumerate}
\end{defn}\vspace{-0.25cm}
We give some examples of immersions. 
\begin{enumerate}[(1)]
    \item {\bf Canonical immersion.} The inclusion map $\iota : \R^k \to \R^n$ defined by 
    $(x_1, \dots, x_k) \mapsto (x_1, \dots, x_k, 0, \dots, 0)$
    is an immersion of class $C^\infty$.
    Indeed, we see that $\iota$ is smooth and its derivative matrix is
    \[ D\iota = \mleft[ \begin{array}{ccc}
        1 & \cdots & 0 \\ 
        \vdots & \ddots & \vdots \\ 
        0 & \cdots & 1 \\ \hline 
        0 & \cdots & 0 \\ 
        \vdots & \ddots & \vdots \\ 
        0 & \cdots & 0 \\ 
    \end{array} \mright] \]
    which has rank $k$ because it contains the $k \times k$ identity matrix.

    \item The parametrization $\alpha(t) = (t^2 - 1, t(t^2 - 1))$ of the 
    $\alpha$-curve is an immersion of class $C^\infty$. 
    
    \item The charts $\alpha : U \to V$ of class $C^r$ of a submanifold 
    $M \subset \R^n$ of class $C^r$ are immersions of class $C^r$. 
\end{enumerate}
Recall that a {\bf diffeomorphism} is a differentiable bijection 
(of class $C^r$) whose inverse is also differentiable (of class $C^r$). The 
following proposition tells us that up to a diffeomorphism (by composition), 
every immersion is locally the canonical immersion.

\begin{prop}{prop:1.9}
    Let $U \subset \R^k$ be open and $\alpha : \R^k \to \R^n$ be an immersion 
    of class $C^r$. Then up to a local diffeomorphism, $\alpha$ is the 
    canonical immersion $\iota : \R^k \to \R^n$ defined by 
    $\iota(x_1, \dots, x_k) = (x_1, \dots, x_k, 0, \dots, 0)$. 
\end{prop}\vspace{-0.25cm}
In order to prove this, we require the Inverse Function Theorem. 

\begin{theo}[Inverse Function Theorem]{theo:1.10}
    Let $U \subset \R^\ell$ be open and let $F : U \subset \R^\ell \to \R^\ell$ 
    be of class $C^r$. Suppose that for some $x_0 \in U$, the $\ell \times \ell$ 
    derivative matrix $DF(x_0)$ is invertible (that is, $\det(DF(x_0)) \neq 0$). 
    Then $F$ is invertible in an open neighbourhood $U_0 \subset U$ of $x_0$ and 
    $F^{-1} : V_0 = F(U_0) \subset \R^\ell \to U_0 \subset \R^\ell$ 
    is also of class $C^r$.
\end{theo}\vspace{-0.25cm}

Note that when we take $\ell = 1$ in the Inverse Function Theorem 
and we have a function $f : U \subset \R^1 \to \R^1$, then $Df(x) = f'(x)$. 
If $f'(x_0) \neq 0$, then either $f'(x) > 0$ around $x_0$ or $f'(x) < 0$. 
In particular, $f$ is increasing or decreasing around $x_0$, which implies 
that it is strictly monotone and thus invertible around $x_0$. 

\begin{pf}[Proposition~\ref{prop:1.9}]
    Suppose that $\alpha : U \subset \R^k \to \R^n$ is an immersion given by 
    \[ x = (x_1, \dots, x_k) \mapsto (f_1(x), \dots, f_n(x)). \] 
    The derivative matrix of $\alpha$ is the $n \times k$ matrix 
    \[ D\alpha(x) = \begin{bmatrix}
        \nabla f_1(x) \\ \vdots \\ \nabla f_n(x) 
    \end{bmatrix} \] 
    and $D\alpha(x)$ has rank $k$ for all $x \in U$ by definition. Then 
    $k$ of the rows of $D\alpha(x)$ are linearly independent. Without loss of 
    generality, we can assume that these are the first $k$ rows after possibly 
    permuting the variables in $\R^n$. We now divide $\alpha$ into 
    the parts 
    \[ \alpha(x) = (\alpha_1(x), \alpha_2(x)), \] 
    where $\alpha_1 : U \subset \R^k \to \R^k$ corresponds to the first 
    $k$ component functions of $\alpha$, and $\alpha_2 : U \subset \R^k \to \R^{n-k}$ 
    corresponds to the remaining $n-k$ component functions. Then we can write 
    \[ D\alpha(x) = \mleft[ \begin{array}{c} D\alpha_1(x) \\ \hline D\alpha_2(x) \end{array} \mright] \]
    where $D\alpha_1(x)$ is a $k \times k$ matrix and $D\alpha_2(x)$ is an 
    $(n-k) \times k$ matrix. Notice that $D\alpha_1(x)$ has rank $k$, which 
    implies that $D\alpha_1(x)$ is invertible. By the Inverse Function Theorem 
    (Theorem~\ref{theo:1.10}), there exist open neighbourhoods $U_0 \subset U$ 
    of $x_0$ and $V_0 \subset \R^k$ of $\alpha_1(x_0)$ such that 
    \[ \alpha_1|_{U_0} : U_0 \subset U \subset \R^k \to V_0 \subset \R^k \] 
    is invertible with inverse 
    $(\alpha_1|_{U_0})^{-1} : V_0 \subset \R^k \to U_0 \subset U \subset \R^k$ 
    of class $C^r$. Hence, $\alpha_1|_{U_0}$ is a diffeomorphism of class $C^r$. 
    Now, consider the composition $\alpha \circ \alpha_1^{-1} : 
    V_0 \subset \R^k \to \R^n$ which yields 
    \[ x \mapsto \alpha(\alpha_1^{-1}(x)) = (\alpha_1(\alpha_1^{-1}(x)), 
    \alpha_2(\alpha_1^{-1}(x))) =: (x, f(x)) \]  
    where $f : V_0 \subset \R^k \to \R^{n-k}$. So up to the diffeomorphism, 
    $\alpha$ is the parametrization of the graph of a function of class $C^r$ 
    (because the composition is of class $C^r$). 

    If $f(x) = 0$ for all $x \in V_0$, then we are already done. Otherwise, 
    compose the above function with $h : V_0 \times \R^{n-k} \subset \R^n 
    \to \R^n$ defined by 
    \[ (x_1, \dots, x_k, x_{k+1}, \dots, x_n) \mapsto 
    (x_1, \dots, x_k, x_{k+1} - f_{k+1} \circ \alpha^{-1}(x), \dots, 
    x_n - f_n \circ \alpha^{-1}(x)), \] 
    which is a diffeomorphism of class $C^r$. For $x \in V_0$, we obtain 
    $h \circ \alpha \circ \alpha^{-1}(x) = (x, 0) = \iota(x)$ as desired. \qed 
\end{pf}\vspace{-0.25cm}
As a consequence, we have the following result. 

\begin{cor}{cor:1.11}
    The image of an immersion of class $C^r$ is locally the graph of a 
    function of class $C^r$ (up to diffeomorphism). In particular, 
    any submanifold of $\R^n$ of class $C^r$ is locally the graph of a 
    function of class $C^r$.
\end{cor}\vspace{-0.25cm} \newpage
\begin{pf}[Corollary~\ref{cor:1.11}]
    From the proof of Proposition~\ref{prop:1.9}, we had a map
    $\alpha \circ \alpha_1^{-1} : V_0 \subset \R^k \to \R^n$ which sent 
    $x \mapsto (x, f(x))$ by taking $f = \alpha_2 \circ \alpha_1^{-1}$ which 
    was of class $C^r$. Then $\alpha(V_0)$ is the graph of $f$ in $\R^n$. 
    
    If $M \subset \R^n$ is a $k$-dimensional submanifold of class $C^r$, 
    then for all $p \in M$, there exists a coordinate chart 
    \[ \alpha : U \subset \R^k \to V \subset M \] 
    with $p \in V$. Set $x_0 = \alpha^{-1}(p)$. Using the above notation, 
    there exist open sets $U_0, V_0 \subset \R^k$ with $x_0 \in U_0$ 
    and a diffeomorphism $\alpha_1 : U_0 \subset \R^k \to V_0 \subset \R^k$ 
    such that 
    \begin{align*} 
        \tilde\alpha := \alpha \circ \alpha_1^{-1} : V_0 \subset \R^k 
        &\to \alpha(U_0) \subset M \\ 
        x &\mapsto (x, f(x)) 
    \end{align*}
    is of class $C^r$ since $\alpha$ and $\alpha_1^{-1}$ are of 
    class $C^r$. (Note that $\alpha(U_0)$ is open in $M$ since $\alpha$ 
    is a homeomorphism.) This implies that $f$ is of class $C^r$, so $M$ 
    is locally the graph of the $C^r$ function $f : V_0 \subset \R^k \to \R^{n-k}$. \qed 
\end{pf}\vspace{-0.25cm}

We now introduce the notion of an embedding.
\begin{defn}{defn:1.12}
    Let $U \subset \R^k$ be open and $\alpha : U \subset \R^k \to \R^n$ 
    be a map. Then $\alpha$ is an {\bf embedding} of class $C^r$ if 
    \begin{enumerate}[(1)]
        \item $\alpha$ is a homeomorphism onto its image; 
        \item $\alpha$ is of class $C^r$; 
        \item $D\alpha(x)$ has rank $k$ for all $x \in U$. 
    \end{enumerate}
\end{defn}\vspace{-0.25cm}
In other words, an embedding is just an immersion that is homeomorphic onto 
its image. In particular, submanifolds of $\R^n$ of class $C^r$ are subsets 
of $\R^n$ that are locally the image of an embedding of class $C^r$. 

The following proposition tells us that embeddings are in fact 
diffeomorphisms onto their images. 
\begin{prop}{prop:1.13}
    Let $U \subset \R^k$ be open, and let $\alpha : U \subset \R^k \to 
    \R^n$ be an embedding of class $C^r$. Then 
    \[ \alpha^{-1} : \alpha(U) \subset \R^n \to U \subset \R^k \] 
    is also of class $C^r$. 
\end{prop}\vspace{-0.25cm}
\begin{pf}[Proposition~\ref{prop:1.13}]
    Note that $\alpha$ is an immersion of class $C^r$, so we can write it as 
    $\alpha(x) = (\alpha_1(x), \alpha_2(x))$,
    where $\alpha_1 : U \subset \R^k \to \R^k$ is locally invertible. 
    Letting $\pi : \R^n \to \R^k$ be the projection $(y_1, \dots, y_n) 
    \mapsto (y_1, \dots, y_k)$, we have 
    \begin{align*} 
        \alpha^{-1} : \alpha(U) \subset \R^n &\to U \subset \R^k \\ 
        (\alpha_1(x), \alpha_2(x)) &\mapsto x = \alpha_1^{-1} \circ 
        \pi(x_1, \dots, x_n).
    \end{align*}
    But $\alpha_1^{-1}$ and $\pi$ are both of class $C^r$, so 
    $\alpha^{-1}$ is also of class $C^r$. \qed 
\end{pf}\vspace{-0.25cm}
We now introduce atlases, which we could've done a while ago when we 
defined topological submanifolds at the beginning. However, we can now 
talk about atlases of class $C^r$.

\begin{defn}{defn:1.14}
    Let $M \subset \R^n$ be a $k$-dimensional topological submanifold. 
    An {\bf atlas} of $M$ is a collection of charts 
    \[ \{\alpha_a : U_a \subset \R^k \to V_a \subset M\}_{a \in A} \] 
    such that $\bigcup_{a \in A} V_a = M$, where each $U_a \subset \R^k$ 
    and $V_a \subset M$ is open. Moreover, if all the charts in the atlas 
    are of class $C^r$, then it is called an 
    {\bf atlas of class $C^r$}, or a {\bf smooth atlas} if it is of 
    class $C^\infty$. 
\end{defn}\vspace{-0.25cm}
We look at some examples of atlases. 
\begin{enumerate}[(1)]
    \item Let $U \subset \R^n$ be open. Then $\alpha = \Id_U$ is a smooth 
    chart for $U$ such that every point in $U$ is contained in $\alpha(U) = U$. 
    Therefore, $\{\alpha = \Id_U : U \to U\}$ is a smooth atlas for $U$. 
    \item Consider the graph $M$ of a function $F : U \subset \R^k \to \R^{n-k}$ 
    of class $C^r$, where $U \subset \R^k$ is open. Then 
    $M = \{(x, F(x)) : x \in U\}$ is a $k$-dimensional 
    submanifold of class $C^r$. It admits the chart $\alpha : U \subset \R^k 
    \to M \subset \R^n$ defined by $x \mapsto (x, F(x))$ of class $C^r$, 
    so $\{\alpha : U \to M\}$ is an atlas of class $C^r$ for $M$. 
    \item Recall that the points on the circle $\mathbb{S}^1 = 
    \{(x, y) : x^2 + y^2 = 1\} \subset \R^2$ can be described using 
    the charts $\alpha_1 : (-\pi, \pi) \to \mathbb{S}^1 \setminus \{(-1, 0)\}$ 
    and $\alpha_2 : (0, 2\pi) \to \mathbb{S}^1 \setminus \{(1, 0)\}$ 
    via $t \mapsto (\cos t, \sin t)$. Then $\{\alpha_1, \alpha_2\}$ 
    is a smooth atlas for $\mathbb{S}^1$. 
\end{enumerate}
Note that if $M$ is compact (that is, closed and bounded), then any atlas 
of $M$ must contain at least two charts. Moreover, atlases are not unique in general! 
For example, consider $\{\alpha : \R \to M,\, t \mapsto (t, 0)\}$ 
and $\{\beta : \R \to M,\, t \mapsto (-t, 0)\}$, which are both smooth 
atlases for the $x$-axis $M$ in $\R^2$. 

We now give an alternate definition of a $k$-dimensional submanifold 
of $\R^n$ of class $C^r$. We will soon prove that this is equivalent to the 
original definition.

\begin{defn}{defn:1.15}
    Let $M \subset \R^n$ be such that $M$ is locally given by the zero set 
    $\{F \equiv 0\}$ of a $C^r$ map $F : V \subset \R^n \to \R^{n-k}$ 
    with maximal rank. That is, 
    $DF(x)$ has rank $n - k$ for all $x \in V \cap M$, where $V \cap M 
    = F^{-1}(0)$ holds for an appropriately chosen neighbourhood $V$ 
    of every point in $M$. Then $M$ is called a 
    {\bf $k$-dimensional submanifold of $\R^n$ of class $C^r$}.
\end{defn}\vspace{-0.25cm} 
This alternate definition is useful, because it is generally easier to show 
that a given subset is locally the zero set of a function than to 
explicitly exhibit charts covering the space. 
\begin{enumerate}[(1)]
    \item Consider the circle $\mathbb{S}^1 = \{(x, y) \in \R^2 : x^2 + y^2 = 1\}$, 
    which is given by the equation $x^2 + y^2 = 1$, which we can rearrange as 
    $x^2 + y^2 - 1 = 0$.
    Define $F : \R^2 \to \R$ via $F(x, y) = x^2 + y^2 - 1$. Then $\mathbb{S}^1 
    = F^{-1}(0)$, so we can take $V = \R^2$ in the definition. Also, we have 
    \[ DF(x, y) = \begin{bmatrix}
        2x & 2y 
    \end{bmatrix}, \] 
    which has rank $1$ for every point since $(0, 0) \notin \mathbb{S}^1$. 
    Thus, under the alternate definition, $\mathbb{S}^1$ is a smooth $1$-dimensional
    submanifold of $\R^2$. 

    \item More generally, the $n$-sphere $\mathbb{S}^n = \{(x_1, \dots, 
    x_{n+1}) \in \R^{n+1} : x_1^2 + \cdots + x_{n+1}^2 = 1\} \subset \R^{n+1}$ 
    can be viewed as the zero set of the smooth function $F : \R^{n+1} \to \R$ 
    given by $F(x_1, \dots, x_{n+1}) = x_1^2 + \cdots + x_{n+1}^2 - 1$. 
    Here, we can take $V = \R^{n+1}$ in the definition. Since 
    \[ DF(x) = \begin{bmatrix} 2x_1 & \cdots & 2x_{n+1} \end{bmatrix} \] 
    has rank $1$ for all $x \in \mathbb{S}^n$, we see that $\mathbb{S}^n$ 
    is a smooth $n$-dimensional submanifold of $\R^{n+1}$. 

    \item Consider the twisted cubic $M = \{(x, y, z) : y = x^2,\, z = x^3\} \subset \R^3$.
    We have seen that this is a smooth $1$-dimensional submanifold of $\R^3$. 
    For all $(x, y, z) \in M$, we have $y = x^2$ and $z = x^3$. Rearranging 
    gives $y - x^2 = 0$ and $z - x^3 = 0$, so we can view $M$ has the zero 
    set of the smooth function $F(x, y, z) = (y-x^2, z-x^3)$ defined for all 
    $(x, y, z) \in \R^3$. This time, we have $V = \R^3$, and $F : 
    \R^3 \to \R^2$ so that $n = 3$ and $k = 1$. The derivative matrix of $F$ is 
    \[ DF(x, y, z) = \begin{bmatrix} 
        -2x & 1 & 0 \\ 
        -3x^2 & 0 & 1 
    \end{bmatrix} \] 
    which is rank $2$ for all $(x, y, z) \in M$ because the last two columns 
    are the $2 \times 2$ identity matrix. Hence, $M$ is a smooth $1$-dimensional 
    submanifold of $\R^3$.     

    {\bf Remark.} Not all zero sets of functions of class $C^r$ are submanifolds 
    of class $C^r$. Take the $\alpha$-curve $C = \{(x, y) \in \R^2 : y^2 = 
    x^2(x+1)\}$, which is the zero set of the function 
    $F : \R^2 \to \R$ defined by $(x, y) \mapsto y^2 - x^2(x+1)$. 
    The derivative matrix of $F$ is 
    \[ DF(x, y) = \begin{bmatrix} -3x^2 - 2x & 2y \end{bmatrix}, \] 
    which is equal to the zero vector if and only if $(x, y) \in 
    \{(0, 0), (-2/3, 0)\}$. We see that $(0, 0) \in C$, and this 
    is the problematic point. 

    \item {\bf (Graph of a function of class $C^r$.)} Let $f : U \subset \R^k 
    \to \R^{n-k}$ be function of class $C^r$, where $U \subset \R^k$ is open. 
    Let $M = \{(x, f(x)) \in \R^k \times \R^{n-k} : x \in U\} \subset \R^n$,
    which we have already seen is a $k$-dimensional submanifold of class $C^r$. 
    We now view this using the zero set characterization. 

    For all $(x, y) \in M \subset \R^k \times \R^{n-k}$, we have 
    $y = f(x)$ if and only if $F(x, y) := f(x) - y = 0 \in \R^{n-k}$.
    Note that $U \times \R^{n-k}$ is an open subset of $\R^k \times \R^{n-k}$. 
    Then $M$ is the zero set of the $C^r$ function $F : U \times \R^{n-k} 
    \subset \R^k \times \R^{n-k} \to \R^{n-k}$ given by $(x, y) \mapsto f(x) - y$.

    It remains to check the rank condition. We have that 
    \[ DF(x, y) = \left[ \begin{array}{c|c} 
        \dfrac{\partial F}{\partial x} & \dfrac{\partial F}{\partial y} 
    \end{array} \right] \]
    which has $n-k$ rows. Since $F(x, y) = f(x) - y$, we see that 
    $\partial F/\partial x = Df(x)$ and $\partial F/\partial y = -I_{n-k}$, 
    so $DF(x, y)$ has rank $n-k$ since $I_{n-k}$ does. 
\end{enumerate}

\begin{theo}{theo:1.16}
    Let $M \subset \R^n$. The following are equivalent: 
    \begin{enumerate}[(i)]
        \item $M$ is a $k$-dimensional submanifold of class $C^r$ (using Definition~\ref{defn:1.7}).
        \item $M$ is locally the graph of a function $f : U \subset \R^k \to \R^{n-k}$ 
        of class $C^r$, where $U \subset \R^k$ is open. 
        \item $M$ is locally the zero set of a $C^r$ function $F : V \subset \R^n 
        \to \R^{n-k}$ of maximal rank, where $V \subset \R^n$ is open 
        (using Definition~\ref{defn:1.15}).
    \end{enumerate}
\end{theo}\vspace{-0.25cm}

Due to Corollary~\ref{cor:1.11}, we already know that (i) and (ii) are 
equivalent. Example (4) above shows that (ii) implies (iii). Therefore, 
it suffices to prove that (iii) implies (ii), and we will see that this is a 
direct consequence of the Implicit Function Theorem. Let's recall what this says.

Suppose that $\R^n = \R^{k+m} = \R^k \times \R^m$ has coordinates $(x, y) 
= (x_1, \dots, x_k, y_1, \dots, y_m)$. Let $U \subset \R^n$ be an open subset 
and $F : U \subset \R^n \to \R^m$ be a function of class $C^r$. Then the 
derivative matrix of $F$ is 
\[ DF(x, y) = \left[ \begin{array}{c|c} 
    \dfrac{\partial F}{\partial x} & \dfrac{\partial F}{\partial y} 
\end{array} \right]. \]
where if $F_1, \dots, F_m$ are the component functions of $F$, 
then we have $\partial F/\partial x = (\partial F_i/\partial x_j)_{1\leq i\leq m, 
1\leq j\leq n}$ and $\partial F/\partial y = 
(\partial F_i/\partial y_j)_{1\leq i, j\leq m}$. In particular, 
$\partial F/\partial y$ is an $m \times m$ matrix.

\begin{theo}[Implicit Function Theorem]{theo:1.17}
    Let $(x_0, y_0) \in U$ be such that $F(x_0, y_0) = 0$. Suppose that 
    \[ \det\left( \frac{\partial F}{\partial y}(x_0, y_0) \right) \neq 0. \] 
    Then there exists an open neighbourhood $V_0 \subset \R^k$ of $x_0$ 
    and a unique function $g : V_0 \to \R^m$ of class $C^r$ such that 
    $g(x_0) = y_0$ and $F(x, g(x)) = 0$ for all $x \in V_0$. 
\end{theo}\vspace{-0.25cm}

In other words, the Implicit Function Theorem tells us that if 
$\det(\partial F(x_0, y_0)/\partial y) \neq 0$, then 
for all points $(x, y) \in \{F \equiv 0\}$ in an open neighbourhood of 
$(x_0, y_0)$, we have $y = g(x)$ for some function of class $C^r$. Thus, 
we can express the variables $(y_1, \dots, y_m)$ as functions of 
$(x_1, \dots, x_k)$ of class $C^r$ near $(x_0, y_0)$. 

Before proving the theorem, we illustrate what the result tells us 
with a simple example. Let $F(x, y) = x^2 + y^2 - 1$ for all $(x, y) \in \R^2$. 
Then $F : \R^2 \to \R$ is smooth and has derivative matrix 
\[ DF(x, y) = \begin{bmatrix}
    2x & 2y 
\end{bmatrix}. \] 
In this case, we have $m = 1$ and $m + k = 2$ so that $k = 1$. We are writing 
$\R^2 = \R^1 \times \R^1$ where $x$ corresponds to the first copy of $\R^1$ 
and $y$ corresponds to the second copy. Then $\partial F/\partial y = 2y$, 
which is nonzero if and only if $y \neq 0$. By the Implicit Function 
Theorem, the points on 
\[ \{F \equiv 0\} = \{(x, y) : x^2 + y^2 - 1 = 0\} \] 
have a $y$-coordinate that can be expressed locally as a function of $x$. 
This is indeed true since $x^2 + y^2 - 1 = 0$ if and only if 
$y = \pm\sqrt{1-x^2}$, which is smooth for $x \notin \{\pm1\}$ and hence 
for $y \neq 0$ on $\{F \equiv 0\}$. These parametrize $\mathbb{S}^1 
\setminus \{(\pm1, 0)\}$. Similarly, note that 
$\partial F/\partial x = 2x \neq 0$ if and only if $x \neq 0$, so this 
holds for the points $(x, y) \neq (0, \pm1)$ on $\mathbb{S}^1$. We see that 
$\mathbb{S}^1$ can be expressed as $x = \pm\sqrt{1-y^2}$ away from 
$(0, \pm1)$.

\begin{pf}[Implicit Function Theorem (Theorem~\ref{theo:1.17})]
    This follows from the Inverse Function Theorem (Theorem~\ref{theo:1.10}).
    Define $H : U \subset \R^n \to \R^n = \R^k \times \R^m$ by $(x, y) 
    \mapsto (x, F(x, y))$. The derivative matrix is
    \[ DH = \left[ \begin{array}{c|c}
        I_{k\times k} & 0 \\ \hline
        \partial F/\partial x & \partial F/\partial y
    \end{array} \right], \]
    which is an $n \times n$ matrix. 
    Since $I_{k\times k}$ has rank $k$ and $\partial F/\partial y$ has 
    rank $m$ at $(x_0, y_0)$ using the fact that it is invertible there, 
    it follows that $DH(x_0, y_0)$ has rank $k + m = n$. That is, 
    $\det(DH(x_0, y_0)) \neq 0$ and $H$ is locally invertible with 
    some inverse $G$ by the Inverse Function Theorem.

    Write $G(u, v) = (G_1(u, v), G_2(u, v))$ where $u \in \R^k$ and
    $v \in \R^m$, and we separate $G$ into the components $G_1(u, v) \in \R^k$ and 
    $G_2(u, v) \in \R^m$. Then we have 
    \begin{align*}
        (u, v) &= H \circ G(u, v) \\ 
        &= H(G_1(u, v), G_2(u, v)) \\ 
        &= (G_1(u, v), F(G_1(u, v), G_2(u, v))). 
    \end{align*}
    This implies that $u = G_1(u, v)$, so $G(u, v) = (u, G_2(u, v))$ for some 
    function $G : V_0 \subset \R^n \to \R^m$ of class $C^r$ (where $V_0 
    \subset \R^n$ is open). Moreover, for all $(x, y)$ with $F(x, y) = 0$, 
    we have $H(x, y) = (x, F(x, y)) = (x, 0)$,
    and hence $(x, y) = G \circ H(x, y) = G(x, 0) = (x, G_2(x, 0))$. Then 
    $y = G_2(x, 0)$ for all $(x, y) \in \{F \equiv 0\}$. By setting 
    $g(x) := G_2(x, 0)$, we have $y = g(x)$ for all $(x, y) \in \{F \equiv 0\}$
    near $(x_0, y_0)$, and $g$ is of class $C^r$. We see that 
    $F(x, g(x)) = 0$.
    For the proof of uniqueness of $g$ and more details, we refer to 
    \emph{Topology} by Munkres, Theorem 9.2 on page 74. \qed 
\end{pf}\vspace{-0.25cm} 

Finally, we prove that the definitions are equivalent. Recall from 
our earlier discussion that it suffices to show that (iii) implies (ii). 

\begin{pf}[Theorem~\ref{theo:1.16}]
    Suppose that $M$ is locally the zero set of a $C^r$ function $F : 
    V \subset \R^n \to \R^{n-k}$ of maximal rank on $M \cap V$. Then 
    for all $(x_0, y_0) \in M \cap V \subset \R^k \times \R^{n-k} = \R^n$, 
    the derivative matrix 
    \[ DF(x_0, y_0) = \left[ \begin{array}{c|c}
        \!\!\dfrac{\partial F}{\partial x}(x_0, y_0) & \dfrac{\partial F}{\partial y}(x_0, y_0)\!\!
    \end{array} \right] \] 
    has rank $n-k$, where $\frac{\partial F}{\partial y}(x_0, y_0)$ is 
    an $(n-k) \times (n-k)$ matrix.

    After possibly permuting the variables $(x_1, \dots, x_k, y_1, \dots, 
    y_{n-k})$ and therefore the columns of $DF(x_0, y_0)$, we may assume that 
    $\partial F(x_0, y_0)/\partial y$ has rank $n-k$. This implies that 
    $\det(\partial F(x_0, y_0)/\partial y) \neq 0$. By the Implicit 
    Function Theorem (Theorem~\ref{theo:1.17}), we see that $y$ 
    is a $C^r$ function of $x$ on $M \cap V$ near $(x_0, y_0)$. 
    Write $y = g(x)$ for some $C^r$ function $g : V_0 \subset \R^k 
    \to \R^m$, where $V_0$ is an open neighbourhood of $x_0$. 
    Then $M$ is locally of the form $\{(x, g(x)) : x \in V_0\}$,
    which implies that $M$ is locally the graph of $g$. \qed 
\end{pf}\vspace{-0.25cm}

To end our discussion, we give one more example of how we can use the 
zero set characterization to find charts for a submanifold $M$ 
using the Implicit Function Theorem. Consider the special linear group 
\[ \SL(2, \R) = \{A \in M_{2\times 2}(\R) : \det A = 1\}. \] 
In a natural way, we can identify the matrix 
\[ A = \begin{bmatrix}
    a_1 & a_2 \\ a_3 & a_4 
\end{bmatrix} \in M_{2\times 2}(\R) \] 
with the point $(a_1, a_2, a_3, a_4) \in \R^4$, and we have 
\[ \det A - 1 = a_1 a_4 - a_2 a_3 - 1 =: F(a_1, a_2, a_3, a_4) \] 
so that $\SL(2, \R) = \{F \equiv 0\}$ where $F : \R^4 \to \R$ is smooth. 
The derivative matrix of $F$ is 
\[ DF(a_1, a_2, a_3, a_4) = \begin{bmatrix}
    a_4 & -a_3 & -a_2 & a_1
\end{bmatrix}, \] 
which can have rank at most $1$ since it is a $1 \times 4$ matrix.
In particular, $DF(a_1, a_2, a_3, a_4)$ has rank $1$ if and only if it is 
not the zero vector, which occurs if at least one of the $a_i$ is nonzero. But 
for all $(a_1, a_2, a_3, a_4) \in \SL(2, \R)$, we have $a_1a_4 - a_2a_3 = 1$ 
so that at least one of the $a_i$ must be nonzero. 

Suppose that $a_1 \neq 0$. Then we have $\partial F/\partial a_4 = a_1 \neq 0$, 
so the Implicit Function Theorem tells us that $a_4$ can be expressed as a 
smooth function of the remaining three variables on $\SL(2, \R)$. Indeed, if 
$(a_1, a_2, a_3, a_4) \in \SL(2, \R)$ with $a_1 \neq 0$, then 
$a_1a_4 - a_2a_3 = 1$ can be rearranged to obtain 
\[ a_4 = \frac{a_2a_3 + 1}{a_1}. \] 
A similar analysis can be done when the other variables are nonzero. 
Since $DF$ has rank $1$ everywhere, it follows that $\SL(2, \R)$ is a 
smooth submanifold of $\R^4$ of dimension $4-1 = 3$. 

\subsection{Tangent vectors and tangent vector fields} \label{subsec:1.5}
The simplest example of a tangent vector is the velocity vector of a curve. 

\begin{defn}{defn:1.18}
    Let $\gamma : (a, b) \subset \R \to \R^n$ be a map of class $C^r$. 
    We define the {\bf velocity vector} of $\gamma$ at $\gamma(t)$ to be 
    \[ \gamma'(t) := D\gamma(t). \] 
\end{defn}\vspace{-0.25cm}

Note that if $\gamma(t) = (\gamma_1(t), \dots, \gamma_n(t))$, then 
$\gamma'(t) = (\gamma'_1(t), \dots, \gamma'_n(t))$. Moreover, we have 
\[ \gamma'(t_0) = \lim_{t\to t_0} \frac{\gamma(t) - \gamma(t_0)}{t - t_0}. \] 
Observe that $(\gamma(t) - \gamma(t_0))/(t - t_0)$ is the velocity vector 
of the secant $L$ passing through $\gamma(t)$ and $\gamma(t_0)$. 
Taking the limit as $t \to t_0$, it follows that $\gamma'(t_0)$ is the tangent 
vector to the curve in $\R^n$ given by $\gamma(t)$, under the assumption
that $\gamma'(t_0) \neq 0$. Let's look at some examples. 

\begin{enumerate}[(1)]
    \item Let $x \in \R^n$ and $0 \neq v \in \R^n$. Set $\gamma(t) = x + tv$ 
    for $t \in \R$, which parametrizes the line $L$ in $\R^n$ passing through 
    $x$ with direction vector $v$. We have that $\gamma'(t) = v$ for all 
    $t \in \R$, so at every point on the line, the velocity vector coincides 
    with the direction vector. 

    \item Let $\gamma : \R \to \R^2$ be defined by $t \mapsto (\cos t, \sin t)$, 
    which parametrizes $\mathbb{S}^1$. Note that $\gamma'(t) = 
    (-\sin t, \cos t) \neq (0, 0)$ for all $t \in \R$. 

    \item Let $\gamma : \R \to \R^2$ be given by $t \mapsto (t^2, t^3)$. 
    This parametrizes the cusp curve $y^2 = x^3$. We have $\gamma'(t) = (2t, 3t^2)$. 
    Observe that $\gamma'(0) = (0, 0)$ so that the velocity vector 
    is zero at $\gamma(0) = (0, 0)$. 

    \item Recall that the $\alpha$-curve can be parametrized with 
    $\gamma : \R \to \R^2$ defined by $t \mapsto (t^2-1, t(t^2-1))$. 
    We have $\gamma'(t) = (2t, 3t^2-1) \neq (0, 0)$ for all $t \in \R$. 
\end{enumerate}

Note that if $\gamma$ is a homeomorphism onto its image 
\[ C = \{x \in \R^n : x = \gamma(t) \text{ for some } t \in (a, b)\} \] 
and $D\gamma(t)$ has rank $1$ for all $t \in (a, b)$, then $C$ is a 
$1$-dimensional submanifold of $\R^n$ of class $C^r$. In fact, we see that
$D\gamma(t)$ has rank $1$ if and only if $\gamma'(t) \neq 0$, so $\gamma'(t)$ 
is a direction vector for the tangent line to $\gamma$ at $\gamma(t)$. 
In particular, the $1$-dimensional submanifold of $\R^n$ determined by $\gamma$ 
has a well-defined tangent line $L$ at every point. We call 
$L := \{\gamma(t_0) + s\gamma'(t_0) : s \in \R\}$
the {\bf tangent line to $C$ at $\gamma(t_0)$}. 

From example (1) above, we know that a line $\gamma(t) = x + tv$ coincides 
with its tangent line at every point.

\begin{defn}{defn:1.19}
    Let $x \in \R^n$. A {\bf tangent vector to $\R^n$ at $x$} is defined as 
    a pair $(x; v)$ where $v \in \R^n$. We call 
    \[ T_x(\R^n) := \{(x; v) : v \in \R^n\} \] 
    the {\bf tangent space to $\R^n$ at $x$}. This is the set of all 
    tangent vectors to $\R^n$ at $x$. 
\end{defn}\vspace{-0.25cm}

We can give $T_x(\R^n)$ a vector space structure with the operations 
$(x; v) + (x; w) = (x; v+w)$ and $c(x; v) = (x; cv)$ for all 
$c \in \R$ and $v, w \in \R^n$. Note that $T_x(\R^n) \simeq \R^n$ as vector 
spaces, where the isomorphism $T_x(\R^n) \to \R^n$ is given by $(x; v) \mapsto v$. 

\begin{lemma}{lemma:1.20}
    For all $x \in \R^n$, we have 
    \[ T_x(\R^n) = \{(x; v) : v \text{ is a velocity vector of some 
    curve $\gamma(t)$ passing through $x$}\}. \] 
\end{lemma}\vspace{-0.25cm}
\begin{pf}[Lemma~\ref{lemma:1.20}]
    Let $(x; v) \in T_x(\R^n)$. Set $\gamma(t) = x + tv$ for $t \in \R$, 
    which satisfies $\gamma'(t) = v$ for all $t \in \R$. In particular, we 
    have $\gamma(0) = x$ and $\gamma'(0) = v$. Note that if $v \neq 0$, then 
    $\gamma(t)$ parametrizes the line through $x$ with velocity vector $v$. \qed 
\end{pf}\vspace{-0.25cm}

Can we get something similar for submanifolds $M \subset \R^n$? 
Towards this direction, we make a new definition. 

\begin{defn}{defn:1.21}
    Let $U \subset \R^k$ be open and let $\alpha : U \to \R^n$ be of class $C^r$. 
    Also, let $x \in U$ and set $p = \alpha(x)$. The map 
    \begin{align*}
        \alpha_* : T_x(\R^k) &\to T_p(\R^n) \\ 
        (x; v) &\mapsto (p; D\alpha(x) v)
    \end{align*}
    is called the {\bf pushforward of $\alpha$ at $x$}. 
\end{defn}\vspace{-0.25cm}

Note that $\alpha_* : T_x(\R^k) \to T_p(\R^n)$ is a vector space homomorphism since it is linear; 
in particular, it is given by multiplication by the $n\times k$ derivative matrix. 

For all $(x; v) \in T_x(\R^k)$, consider the map $\gamma : (-\eps, \eps) \to \R^n$
defined by $t \mapsto \alpha(x+tv)$
with $\eps > 0$ such that $x + tv \in U$ for all $t \in (-\eps, \eps)$. 
That is, the line segment $L = \{x + tv : t \in (-\eps, \eps)\}$
is included in the open set $U$. Then for all $t_0 \in (-\eps, \eps)$, 
the chain rule gives 
\begin{align*}
    \gamma'(t_0) &= D\alpha(x + t_0 v) D(x + tv)(t_0) 
    = D\alpha(x + t_0 v) v 
\end{align*}
since $D(x + tv)(t_0) = v$ for all $t_0 \in \R$. In particular, we obtain 
\begin{align*}
    (\gamma(0); \gamma'(0)) &= (\alpha(x); D\alpha(x)v) \\ 
    &= (p, D\alpha(x)v) \\ 
    &= \alpha_*(x; v). 
\end{align*}
This tells us that $\alpha_*(x; v)$ is the velocity vector of 
$\gamma(t) := \alpha(x + tv)$ at $p = \gamma(0)$.

\begin{lemma}{lemma:1.22}
    If $\alpha : U \subset \R^k \to \R^n$ and $\beta : V \subset \R^\ell 
    \to \R^k$ are functions of class $C^r$ where $U \subset \R^k$ and $V 
    \subset \R^\ell$ are open with $\beta(V) \subset U$, then 
    $(\alpha \circ \beta)_* = \alpha_* \circ \beta_*$ on $V$. 
\end{lemma}\vspace{-0.25cm} 
\begin{pf}[Lemma~\ref{lemma:1.22}]
    Note that $\alpha \circ \beta : V \subset \R^\ell \to \R^n$ is 
    of class $C^r$. Let $x \in V$. Then for all $v \in \R^\ell$, we have 
    \begin{align*}
        (\alpha \circ \beta)_*(x; v) 
        &= (\alpha \circ \beta(x); D(\alpha \circ \beta)(x)v) \\ 
        &= (\alpha(\beta(x)); D\alpha(\beta(x))D\beta(x)v) \\ 
        &= \alpha_*(\beta(x); D\beta(x)v) \\ 
        &= \alpha_*(\beta_*(x; v)), 
    \end{align*} 
    where the second equality follows from the chain rule. \qed 
\end{pf}\vspace{-0.25cm}

From Lemma~\ref{lemma:1.22}, it now makes sense to define the following. 

\begin{defn}{defn:1.23}
    Let $M \subset \R^n$ be a $k$-dimensional submanifold of $\R^n$ 
    of class $C^r$ for some $r \geq 1$. Let $p \in M$ and let $\alpha : 
    U \subset \R^k \to V \subset M$ be a coordinate chart of $M$ 
    about $p$ (i.e. $p \in V$) of class $C^r$. We define the 
    {\bf tangent space of $M$ at $p$} to be 
    \[ T_p(M) := \alpha_*(T_{x_0}(\R^k)) \subset T_p(\R^n), \] 
    where $x_0 \in U$ is the unique point in $U$ such that $\alpha(x_0) = p$.  
\end{defn}\vspace{-0.25cm}\newpage
\section{Curves in $\R^n$}\label{sec:2}

Tangent spaces: Let $M \subset \R^n$ with class $C^r$ of dimension $k$. For a chart 
$\alpha : U \subset \R^k \to V \subset M \subset \R^n$, we have for all $p \in V$ that 
\[ T_p(M) := \alpha_*(T_{x_0}(\R^k)), \] 
where $p = \alpha(x_0)$. We showed that 
\[ T_p(M) = {\rm span}_{\R} \left\{ \frac{\partial\alpha}{\partial x_1}(x_0), \dots, 
\frac{\partial\alpha}{\partial x_k}(x_0) \right\} \subset T_p(\R^n), \] 
which is a $k$-dimensional subspace of $T_p(\R^n)$. 

Examples: 
\begin{enumerate}[(1)]
    \item Let $U \subset \R^n$ be an open set. Then $\alpha : U \subset \R^n \to V = U \subset \R^n$ 
    given by $x \mapsto x$ yields $D\alpha(x) = I_{n\times n}$ for all $x \in U$, so 
    \[ T_x(U) = \alpha_*(T_x(\R^n)) \] 
    by definition. Then for all $(x; v) \in T_x(\R^n)$, we get 
    \[ \alpha_*(x; v) = (\alpha(x); D\alpha(x)v) = (x; v). \] 
    That is, we have $T_x(U) = T_x(\R^n)$. 

    \item We saw two different ways of seeing if something is a submanifold. If $f : U \subset \R^k 
    \to \R^{n-k}$ is a function of a class $C^r$ and 
    \[ M = \{(x, f(x)) \in \R^n : x \in U\} \subset \R^n \] 
    its graph, then $M$ is a $k$-dimensional submanifold of $\R^n$ of class $C^r$. We can parameterize 
    all points in $M$ with the map $\alpha : U \subset \R^k \to M \subset \R^n$ defined by 
    $\alpha(x) = (x, f(x))$. The derivative matrix is 
    \[ D\alpha(x) = \begin{bmatrix} 
        I_{k\times k} \\\hline Df(x)
    \end{bmatrix} \]
    for all $x \in U$. Let $p \in M$ so that $p = (x_0, f(x_0))$ for some $x_0 \in U$. Then 
    \begin{align*} 
        T_p(M) &= \alpha_*(T_{x_0}(\R^k)) \\
        &= \{\alpha_*(x_0; v) : v \in T_{x_0}(\R^k)\} \\
        &= \{(\alpha(x_0); D\alpha(x_0)v) : v \in \R^k\} \\ 
        &= \{(p; w) : w = (v, Df(x_0)v),\, v \in \R^k\}, 
    \end{align*} 
    since $\alpha(x_0) = p$ and $D\alpha(x)$ is the block matrix with $I_{k\times k}$ 
    upstairs and $Df(x)$ downstairs. Also, 
    \[ T_p(M) = {\rm span}_{\R} \left\{ \frac{\partial\alpha}{\partial x_1}(x_0), \dots, 
    \frac{\partial\alpha}{\partial x_k}(x_0) \right\} = {\rm span}_{\R} 
    \left\{ \begin{bmatrix} e_i \\\hline \partial f(x_0)/\partial x_i \end{bmatrix} : 
    i = 1, \dots, k \right\}. \] 

    \item Let $U \subset \R^n$ be open. If $F : U \subset \R^n \to \R^{n-k}$ is a function of 
    class $C^r$ with $DF(p)$ having rank $n-k$ for all $p \in U$, then 
    \[ M = \{x \in U : F(x) = 0\} \] 
    is a $k$-dimensional submanifold of $\R^n$ of class $C^r$. In this case, we leave it as an 
    exercise to show that 
    \[ T_p(M) = \ker(DF(p)). \] 
    In particular, if $k = n-1$, then $F : U \subset \R^n \to \R$ is a scalar function and 
    \[ T_p(M) = \ker(\nabla F(p)). \] 
    Then $\nabla F(p)$ is the normal vector of $T_p(M)$. 

    For example, take the $n$-sphere 
    \[ S^n = \{x \in \R^{n+1} : \|x\|^2 = 1\}, \] 
    which is the zero set of the function $F : \R^{n+1} \to \R$ defined by $F(x) = \|x\|^2 - 1 
    = x_1^2 + \cdots + x_{n+1}^2 - 1$. The derivative matrix is just the gradient; that is, 
    \[ DF(x) = \nabla F(x) = \begin{bmatrix} 2x_1 & \cdots & 2x_{n+1} \end{bmatrix} = 2x. \] 
\end{enumerate}

\subsection{}\label{subsec:2.1}
What is a curve? Intuitively, it is a $1$-dimensional subset of $\R^n$. 

The level sets $f(x, y) = k$ of a two variable function in $\R^2$ are curves. 
For example, for $f(x, y) = x^2 + y^2$, we see that $C \colon x^2 + y^2 = k$ for $k > 0$ 
is a circle centered at $(0, 0)$ of radius $k^{1/2}$. 

The intersection of two surfaces in $\R^3$ is also a curve. For example, 
take $z = x^2 + y^2$ and $z = 2$. Their intersection is the circle 
$C \colon x^2 + y^2 = 2$ in the plane $z = 2$.

For our purposes, we'll work with parametrized curves $\gamma : I = (\alpha, \beta) \subset 
\R \to \R^n$ of class $C^r$. (In practice, we need $r \geq n$ when working over $\R^n$.)

Example: {\bf Circular helix.} Let $\gamma(t) = (a\cos t, a\sin t, bt)$ for $t \in \R$ and $a, b > 0$. 
Note that $x^2 + y^2 = a^2$, so $\gamma(t)$ lies above the circle $x^2 + y^2 = a^2$ in the $xy$-plane. 
We have $\gamma(0) = (a, 0, 0)$ and $\gamma(\pi/2) = (0, a, b\pi/2)$. (The circular helix 
looks like a spiral along the cylinder.)

\begin{defn}{defn:2.1}
    A parameterized curve $\gamma : (\alpha, \beta) \to \R^n$ of class $C^r$ is called 
    {\bf regular} if for all $t \in (\alpha, \beta)$, we have 
    \[ \gamma'(t) = \frac{{\rm d}\gamma}{{\rm d}t}(t) \neq 0. \] 
    We call $\|\gamma'(t)\|$ the {\bf speed of $\gamma$ 
    at $\gamma(t)$} and we say that $\gamma$ is {\bf unit speed} if $\|\gamma'(t)\| = 1$ for all 
    $t \in (\alpha, \beta)$. 
\end{defn}\vspace{-0.25cm}
Note that unit speed implies regular because $\|x\| = 0$ if and only if $x = 0$. 

Example: Let $a > 0$ and take $\gamma(t) = (a\cos t, a\sin t)$ for $t \in \R$, which is a 
parametrization of the circle of radius $a$. Then $\gamma'(t) = (-a\sin t, a\cos t) \neq (0, 0)$ 
for all $t \in \R$ and $\|\gamma'(t)\| = a$, so $\gamma$ is unit speed if and only if $a = 1$.  \newpage

\end{document}
