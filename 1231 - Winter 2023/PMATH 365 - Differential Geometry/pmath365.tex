\documentclass[10pt]{article}
\usepackage[T1]{fontenc}
\usepackage{amsmath,amssymb,amsthm}
\usepackage{mathtools}
\usepackage[shortlabels]{enumitem}
\usepackage[english]{babel}
\usepackage[utf8]{inputenc}
\usepackage{fancyhdr}
\usepackage{bold-extra}
\usepackage{color}   
\usepackage{tocloft}
\usepackage{graphicx}
\usepackage{lipsum}
\usepackage{wrapfig}
\usepackage{cutwin}
\usepackage{hyperref}
\usepackage{lastpage}
\usepackage{multicol}
\usepackage{tikz}
\usepackage{xcolor}
\usepackage{microtype}
\usepackage[framemethod=TikZ]{mdframed}
\usepackage{mleftright}
\usepackage{biblatex}

% some useful math commands
\newcommand{\eps}{\varepsilon}
\newcommand{\R}{\mathbb{R}}
\newcommand{\C}{\mathbb{C}}
\newcommand{\N}{\mathbb{N}}
\newcommand{\Z}{\mathbb{Z}}
\newcommand{\Q}{\mathbb{Q}}
\newcommand{\K}{\mathbb{K}}
\newcommand{\F}{\mathbb{F}}
\newcommand{\T}{\mathbb{T}}

\numberwithin{equation}{section}

\newcommand{\dd}{\,\mathrm{d}}
\newcommand{\ddz}{\frac{\rm d}{{\rm d}z}}
\newcommand{\pv}{\text{p.v.}}

\renewcommand{\Re}{{\rm Re}}

\DeclareMathOperator{\GL}{GL}
\DeclareMathOperator{\id}{id}
\DeclareMathOperator{\Arg}{Arg}
\DeclareMathOperator{\Log}{Log}
\DeclareMathOperator{\PV}{PV}
\DeclareMathOperator{\sech}{sech}
\DeclareMathOperator{\csch}{csch}
\DeclareMathOperator{\Res}{Res}
\DeclareMathOperator{\Li}{Li}
\DeclareMathOperator{\QR}{QR}
\DeclareMathOperator{\NR}{NR}
\DeclareMathOperator{\lcm}{lcm}
\DeclareMathOperator{\divergence}{div}
\DeclareMathOperator*{\esssup}{ess\,sup}
\DeclareMathOperator{\Span}{span}
\DeclareMathOperator{\Pol}{Pol}
\DeclareMathOperator{\dist}{dist}
\DeclareMathOperator{\Id}{Id}
\DeclareMathOperator{\SL}{SL}
\DeclareMathOperator{\FFF}{{\cal F}_{\RN{1}}}
\DeclareMathOperator{\SFF}{{\cal F}_{\RN{2}}}
\DeclareMathOperator{\tr}{tr}
\DeclareMathOperator{\dVol}{dVol}
\DeclareMathOperator{\Div}{Div}

\DeclarePairedDelimiter\ceil{\lceil}{\rceil}
\DeclarePairedDelimiter\floor{\lfloor}{\rfloor}

\newcommand{\suchthat}{\;\ifnum\currentgrouptype=16 \;\middle|\;\else\mid\fi\;}

% title formatting
\newcommand{\newtitle}[4]{
  \begin{center}
	\huge{\textbf{\textsc{#1 Course Notes}}}
    
	\large{\sc #2}
    
	{\sc #3 \textbullet\, #4 \textbullet\, University of Waterloo}
	\normalsize\vspace{1cm}\hrule
  \end{center}
}

\newcounter{theo}[section]\setcounter{theo}{0}
\renewcommand{\thetheo}{\arabic{section}.\arabic{theo}}
\newenvironment{theo}[2][]{%
\refstepcounter{theo}%
\ifstrempty{#1}%
{\mdfsetup{%
frametitle={%
\tikz[baseline=(current bounding box.east),outer sep=0pt]
\node[anchor=east,rectangle,fill=blue!20]
{\strut {\sc Theorem~\thetheo}};}}
}%
{\mdfsetup{%
frametitle={%
\tikz[baseline=(current bounding box.east),outer sep=0pt]
\node[anchor=east,rectangle,fill=blue!20]
{\strut {\sc Theorem~\thetheo:~#1}};}}%
}%
\mdfsetup{innertopmargin=10pt,linecolor=blue!20,%
linewidth=2pt,topline=true,%
frametitleaboveskip=\dimexpr-\ht\strutbox\relax
}
\begin{mdframed}[nobreak=false]\relax%
\label{#2}}{\end{mdframed}}

%%%%%%%%%%%%%%%%%%%%%%%%%%%%%%
%Definition
\newenvironment{defn}[2][]{%
\refstepcounter{theo}%
\ifstrempty{#1}%
{\mdfsetup{%
frametitle={%
\tikz[baseline=(current bounding box.east),outer sep=0pt]
\node[anchor=east,rectangle,fill=yellow!20]
{\strut {\sc Definition~\thetheo}};}}
}%
{\mdfsetup{%
frametitle={%
\tikz[baseline=(current bounding box.east),outer sep=0pt]
\node[anchor=east,rectangle,fill=yellow!20]
{\strut {\sc Definition~\thetheo:~#1}};}}%
}%
\mdfsetup{innertopmargin=10pt,linecolor=yellow!20,%
linewidth=2pt,topline=true,%
frametitleaboveskip=\dimexpr-\ht\strutbox\relax
}
\begin{mdframed}[nobreak=true]\relax%
\label{#2}}{\end{mdframed}}

%%%%%%%%%%%%%%%%%%%%%%%%%%%%%%
%Example
\newenvironment{exmp}[2][]{%
\refstepcounter{theo}%
\ifstrempty{#1}%
{\mdfsetup{%
frametitle={%
\tikz[baseline=(current bounding box.east),outer sep=0pt]
\node[anchor=east,rectangle,fill=cyan!20]
{\strut {\sc Example~\thetheo}};}}
}%
{\mdfsetup{%
frametitle={%
\tikz[baseline=(current bounding box.east),outer sep=0pt]
\node[anchor=east,rectangle,fill=cyan!20]
{\strut {\sc Example~\thetheo:~#1}};}}%
}%
\mdfsetup{innertopmargin=10pt,linecolor=cyan!20,%
linewidth=2pt,topline=true,%
frametitleaboveskip=\dimexpr-\ht\strutbox\relax
}
\begin{mdframed}[nobreak=false]\relax%
\label{#2}}{\end{mdframed}}

%%%%%%%%%%%%%%%%%%%%%%%%%%%%%%
%Corollary
\newenvironment{cor}[2][]{%
\refstepcounter{theo}%
\ifstrempty{#1}%
{\mdfsetup{%
frametitle={%
\tikz[baseline=(current bounding box.east),outer sep=0pt]
\node[anchor=east,rectangle,fill=lime!20]
{\strut {\sc Corollary~\thetheo}};}}
}%
{\mdfsetup{%
frametitle={%
\tikz[baseline=(current bounding box.east),outer sep=0pt]
\node[anchor=east,rectangle,fill=lime!20]
{\strut {\sc Corollary~\thetheo:~#1}};}}%
}%
\mdfsetup{innertopmargin=10pt,linecolor=lime!20,%
linewidth=2pt,topline=true,%
frametitleaboveskip=\dimexpr-\ht\strutbox\relax
}
\begin{mdframed}[nobreak=true]\relax%
\label{#2}}{\end{mdframed}}

%%%%%%%%%%%%%%%%%%%%%%%%%%%%%%
%Remark
\newenvironment{remark}[2][]{%
\refstepcounter{theo}%
\ifstrempty{#1}%
{\mdfsetup{%
frametitle={%
\tikz[baseline=(current bounding box.east),outer sep=0pt]
\node[anchor=east,rectangle,fill=orange!20]
{\strut {\sc Remark~\thetheo}};}}
}%
{\mdfsetup{%
frametitle={%
\tikz[baseline=(current bounding box.east),outer sep=0pt]
\node[anchor=east,rectangle,fill=orange!20]
{\strut {\sc Remark~\thetheo:~#1}};}}%
}%
\mdfsetup{innertopmargin=10pt,linecolor=orange!20,%
linewidth=2pt,topline=true,%
frametitleaboveskip=\dimexpr-\ht\strutbox\relax
}
\begin{mdframed}[nobreak=true]\relax%
\label{#2}}{\end{mdframed}}

%%%%%%%%%%%%%%%%%%%%%%%%%%%%%%
%Exercise
\newenvironment{exercise}[2][]{%
\refstepcounter{theo}%
\ifstrempty{#1}%
{\mdfsetup{%
frametitle={%
\tikz[baseline=(current bounding box.east),outer sep=0pt]
\node[anchor=east,rectangle,fill=pink!20]
{\strut {\sc Exercise~\thetheo}};}}
}%
{\mdfsetup{%
frametitle={%
\tikz[baseline=(current bounding box.east),outer sep=0pt]
\node[anchor=east,rectangle,fill=pink!20]
{\strut {\sc Exercise~\thetheo:~#1}};}}%
}%
\mdfsetup{innertopmargin=10pt,linecolor=pink!20,%
linewidth=2pt,topline=true,%
frametitleaboveskip=\dimexpr-\ht\strutbox\relax
}
\begin{mdframed}[nobreak=true]\relax%
\label{#2}}{\end{mdframed}}

%%%%%%%%%%%%%%%%%%%%%%%%%%%%%%
%Lemma
\newenvironment{lemma}[2][]{%
\refstepcounter{theo}%
\ifstrempty{#1}%
{\mdfsetup{%
frametitle={%
\tikz[baseline=(current bounding box.east),outer sep=0pt]
\node[anchor=east,rectangle,fill=green!20]
{\strut {\sc Lemma~\thetheo}};}}
}%
{\mdfsetup{%
frametitle={%
\tikz[baseline=(current bounding box.east),outer sep=0pt]
\node[anchor=east,rectangle,fill=green!20]
{\strut {\sc Lemma~\thetheo:~#1}};}}%
}%
\mdfsetup{innertopmargin=10pt,linecolor=green!20,%
linewidth=2pt,topline=true,%
frametitleaboveskip=\dimexpr-\ht\strutbox\relax
}
\begin{mdframed}[nobreak=true]\relax%
\label{#2}}{\end{mdframed}}

%%%%%%%%%%%%%%%%%%%%%%%%%%%%%%
%Proposition
\newenvironment{prop}[2][]{%
\refstepcounter{theo}%
\ifstrempty{#1}%
{\mdfsetup{%
frametitle={%
\tikz[baseline=(current bounding box.east),outer sep=0pt]
\node[anchor=east,rectangle,fill=purple!20]
{\strut {\sc Proposition~\thetheo}};}}
}%
{\mdfsetup{%
frametitle={%
\tikz[baseline=(current bounding box.east),outer sep=0pt]
\node[anchor=east,rectangle,fill=purple!20]
{\strut {\sc Proposition~\thetheo:~#1}};}}%
}%
\mdfsetup{innertopmargin=10pt,linecolor=purple!20,%
linewidth=2pt,topline=true,%
frametitleaboveskip=\dimexpr-\ht\strutbox\relax
}
\begin{mdframed}[nobreak=true]\relax%
\label{#2}}{\end{mdframed}}

%%%%%%%%%%%%%%%%%%%%%%%%%%%%%%
%Fact
\newenvironment{fact}[2][]{%
\refstepcounter{theo}%
\ifstrempty{#1}%
{\mdfsetup{%
frametitle={%
\tikz[baseline=(current bounding box.east),outer sep=0pt]
\node[anchor=east,rectangle,fill=gray!20]
{\strut {\sc Fact~\thetheo}};}}
}%
{\mdfsetup{%
frametitle={%
\tikz[baseline=(current bounding box.east),outer sep=0pt]
\node[anchor=east,rectangle,fill=gray!20]
{\strut {\sc Fact~\thetheo:~#1}};}}%
}%
\mdfsetup{innertopmargin=10pt,linecolor=gray!20,%
linewidth=2pt,topline=true,%
frametitleaboveskip=\dimexpr-\ht\strutbox\relax
}
\begin{mdframed}[nobreak=true]\relax%
\label{#2}}{\end{mdframed}}

% new proof environment
\newenvironment{pf}[1][\proofname]
  {\par\noindent\normalfont\textbf{Proof of #1.}\par\nopagebreak%
  \begin{mdframed}[
     linewidth=1pt,
     linecolor=black,
     bottomline=false,topline=false,rightline=false,
     innerrightmargin=0pt,innertopmargin=0pt,innerbottommargin=0pt,
     innerleftmargin=1em,% Distance between vertical rule & proof content
     skipabove=0.75\baselineskip
   ]}
  {\end{mdframed}}

% 1-inch margins
\topmargin 0pt
\advance \topmargin by -\headheight
\advance \topmargin by -\headsep
\textheight 8.9in
\oddsidemargin 0pt
\evensidemargin \oddsidemargin
\marginparwidth 0.5in
\textwidth 6.5in

\parindent 0in
\parskip 1.5ex

\setlist[itemize]{topsep=0pt}
\setlist[enumerate]{topsep=0pt}

\newcommand{\pushright}[1]{\ifmeasuring@#1\else\omit\hfill$\displaystyle#1$\fi\ignorespaces}

% hyperlinks
\hypersetup{
  colorlinks=true, 
  linktoc=all,     % table of contents is clickable  
  allcolors=red    % all hyperlink colours
}

% table of contents
\addto\captionsenglish{
  \renewcommand{\contentsname}%
    {Table of Contents}%
}
\renewcommand{\cftsecfont}{\normalfont}
\renewcommand{\cftsecpagefont}{\normalfont}
\cftsetindents{section}{0em}{2em}

\fancypagestyle{plain}{%
\fancyhf{} % clear all header and footer fields
\lhead{PMATH 365: Winter 2023}
\fancyhead[R]{Table of Contents}
%\headrule
\fancyfoot[R]{{\small Page \thepage\ of \pageref*{LastPage}}}
}

% headers and footers
\pagestyle{fancy}
\renewcommand{\sectionmark}[1]{\markboth{#1}{#1}}
\lhead{PMATH 365: Winter 2023}
\cfoot{}
\setlength\headheight{14pt}

%\setcounter{section}{-1}

\begin{document}

\pagestyle{fancy}
\newtitle{PMATH 365}{Differential Geometry}{Ruxandra Moraru}{Winter 2023}
\rhead{Table of Contents}
\rfoot{{\small Page \thepage\ of \pageref*{LastPage}}}

\tableofcontents
\vspace{1cm}\hrule
\fancyhead[R]{\nouppercase\rightmark}
\newpage 
\fancyhead[R]{Section \thesection: \nouppercase\leftmark}

\section{Submanifolds of $\R^n$}\label{sec:1}

\subsection{Preliminaries}\label{subsec:1.1}
To begin, we'll recall some facts about the topology of $\R^n$ and 
vector-valued functions.

In this course, we'll be working with the metric topology with respect 
to the Euclidean norm (or metric). Let $x = (x_1, \dots, x_n),\,
y = (y_1, \dots, y_n) \in \R^n$.
The {\bf Euclidean norm} is defined to be 
\[ \|x\| = \sqrt{x_1^2 + \cdots + x_n^2}, \] 
and {\bf Euclidean distance} is given by 
\[ \dist(x, y) = \|y - x\| = \sqrt{(y_1 - x_1)^2 + 
\cdots + (y_n - x_n)^2}. \] 
We define the {\bf open ball} of radius $r > 0$ centered at $x \in \R^n$ by 
\[ B_r(x) := \{y \in \R^n : \dist(x, y) < r\} \subset \R^n. \] 
A {\bf topology} on $\R^n$ is a collection $\mathcal{U} = \{U_\alpha\}_{\alpha \in A}$ 
of subsets $U_\alpha \subset \R^n$ that satisfy the following properties.
\begin{enumerate}[(i)]
    \item $\varnothing$ and $\R^n$ are in ${\cal U}$. 
    \item For any subcollection $\mathcal{V} = \{U_\beta\}_{\beta \in B}$ 
    with $U_\beta \in {\cal U}$ for all $\beta \in B$, we have 
    $\bigcup_{\beta \in B} U_\beta \in {\cal U}$. 
    \item For any \emph{finite} subcollection $\{U_{\alpha_1}, \dots, 
    U_{\alpha_m}\} \subset {\cal U}$, we have $\bigcap_{i=1}^m 
    U_{\alpha_i} \in {\cal U}$. 
\end{enumerate}
The sets $U_\alpha \in {\cal U}$ are called the {\bf open sets} of the topology; 
their complements $F_\alpha = \R^n \setminus U_\alpha$ are called the 
{\bf closed sets}. 

Note that the sets $\varnothing$ and $\R^n$ are both open and closed. 
Moreover, the notion of a topology can be extended to more general sets 
$X$, not just $\R^n$. A topology can also be defined starting with closed sets,
but we prefer to work with open sets because many nice properties, such as 
differentiability, are better described with them.

Under the metric topology, we say that a set $A \subset \R^n$ is {\bf open} 
if $A = \varnothing$ or if for all $p \in A$, there exists $r > 0$ 
such that $B_r(p) \subset A$. Moreover, $A$ is {\bf closed} if its 
complement $A^c = \R^n \setminus A$ is open. (We leave it as an 
exercise to show that this is indeed a topology.) 

For example, the open balls $B_r(x)$ are open sets for all 
$x \in \R^n$ and $r > 0$. Indeed, for any point $p \in 
B_r(x)$, one sees that by picking $r' = (r - \|p - x\|)/2$, 
we have $B_{r'}(p) \subset B_r(x)$. 

In general, open sets are described with strict inequalities, while closed 
sets are described using equality or inclusive inequalities. However, note 
that most sets are neither open nor closed, such as the half-open interval 
$U = (-1, 1]$ over $\R$. 

The metric topology is not the only topology on $\R^n$; one example is 
the one consisting of only the sets ${\cal U} = \{\varnothing, \R^n\}$.
However, we generally want more open sets to work with since we might want to 
know the behaviour of functions around a point $p \in \R^n$. If the 
only non-empty open set we had was $\R^n$, then this would apply to all points 
in $\R^n$, which does not yield a lot of information. 

Let $p \in \R^n$. The previous paragraph leads us to the definition of an 
{\bf open neighbourhood} of $p$, which is just an open set $U \subset \R^n$ 
such that $p \in U$. 

We now turn our discussion to vector-valued functions. Let $U \subset \R^n$ 
and consider the vector-valued function 
\begin{align*}
    F : U \subset \R^n &\to B \subset \R^m \\ 
    x = (x_1, \dots, x_n) &\mapsto (F_1(x), \dots, F_m(x)). 
\end{align*}
Then $F$ is continuous if and only if the component functions 
$F_i : U \to \R$ are continuous for all $i = 1, \dots, m$. 

We say that $F$ is a {\bf homeomorphism} if it is a continuous bijection 
whose inverse 
\[ F^{-1} : B \subset \R^m \to U \subset \R^n \] 
is also continuous. For example, the identity map $\Id_{\R^n} : \R^n \to \R^n$ 
and the function $f : \R \to \R$ defined by $f(x) = x^3$ are both homeomorphisms. 

It is a known fact that homeomorphisms map open sets to open sets and closed 
sets to closed sets. This follows from the topological characterization 
of continuity, which states that $F$ is continuous if and only if for 
every open (respectively closed) set $V \subset \R^m$, we have that 
$F^{-1}(V)$ is open (respectively closed). In fact, homeomorphisms preserve 
much more structure than this, as we'll see later. 

\subsection{Topological submanifolds of $\R^n$}\label{subsec:1.2}
We now define the main object we'll be working with in this course.

\begin{defn}{defn:1.1}
    A {\bf $k$-dimensional topological submanifold} (or 
    {\bf topological $k$-submanifold}) of $\R^n$ is a subset $M \subset \R^n$ 
    such that for every $p \in M$, there exists an open neighbourhood $V$ 
    of $p$ in $\R^n$, an open set $U \subset \R^k$, and a homeomorphism 
    \[ \alpha : U \subset \R^k \to V \cap M \subset \R^n. \] 
    The homeomorphism $\alpha$ is called a {\bf coordinate chart} 
    (or {\bf patch}) on $M$.
\end{defn}\vspace{-0.25cm}

Note that the open neighbourhood $V \subset \R^n$ of $p$, the 
open set $U \subset \R^k$, and the map $\alpha$ do not need to be unique.
But we'll see later that the dimension $k$ must be unique and is completely 
determined by $M$.

For example, $\R^n$ is a topological $n$-submanifold of $\R^n$ 
by taking $U = V = \R^n$ and $\alpha = \Id_{\R^n}$. Any open set 
$W \subset \R^n$ is a topological $n$-submanifold of $\R^n$ by 
taking $U = V = W$ and $\alpha = \Id_W$. 

Let's now consider some non-trivial examples. Consider 
\[ M = \{(x, y) \in \R^2 : y = x^2\} \subset \R^2, \] 
which is the graph of the parabola $f(x) = x^2$. Then $M$ is a 
topological $1$-submanifold of $\R^2$ by considering the map 
$\alpha : \R^1 \to M \subset \R^2,\, t \mapsto (t, t^2)$.
The inverse $\alpha^{-1} : M \to \R^1$ is just the projection of the 
first coordinate, which is continuous. 

More generally, let $U \subset \R^k$ be an open set. Consider the graph of a 
continuous function 
\[ F : U \subset \R^k \to \R^{n-k},\, x \mapsto (F_1(x), \dots, F_{n-k}(x)). \]
In other words, we are looking at the set 
\[ G = \{(x, y) \in \R^k \times \R^{n-k} : y = F(x),\, x \in U\} \subset \R^n. \] 
We claim that $G$ is a $k$-dimensional topological submanifold of $\R^n$. To 
see this, define $\alpha : U \subset \R^k \to G \subset \R^n$ by 
$x \mapsto (x, F(x))$. Then $\alpha$ is continuous since $F$ is continuous, 
and it is a bijection since we are restricted to $G$. Moreover, 
it has continuous inverse $\alpha^{-1} : G \subset \R^n \to 
U \subset \R^k,\, (x, y) \mapsto x$. 

Here are two more examples of this in action. 
\begin{enumerate}[(1)]
    \item Let $M = \{(x, y, z) \in \R^3 : z = x^2 + y^2\} \subset \R^3$. 
    Then $M$ is the graph of the continuous function $f(x, y) = x^2 + y^2$, 
    so it is a $2$-dimensional topological submanifold of $\R^3$. 
    \item Observe that $M = \{(x, y, z) \in \R^3 : y = x^2,\, z = x^3\} 
    \subset \R^3$ is the graph of the continuous function $F(t) = (t^2, t^3)$, 
    so it is a $1$-dimensional topological submanifold of $\R^3$.
\end{enumerate}
In all the examples above, we only needed one coordinate chart which 
worked for all points. However, this is not always the case! Consider 
the unit circle 
\[ \mathbb{S}^1 := \{(x, y) \in \R^2 : x^2 + y^2 = 1 \} \subset \R^2. \] 
Note that $\mathbb{S}^1$ is compact. Therefore, by Heine-Borel, 
it is closed and bounded. Recall that homeomorphisms preserve closed sets, 
so it is impossible to find a unique chart $\alpha$. Indeed, if we had 
such a homeomorphism $\alpha : U \subset \R^1 \to \mathbb{S}^1 \subset \R^2$ 
for some open set $U$, then $U = \alpha^{-1}(\mathbb{S}^1)$ would be 
a compact subset of $\R^1$. But the only open and compact subset of 
$\R^n$ is $\varnothing$, which is a contradiction! 

Nonetheless, two coordinate charts are enough to cover all points on 
$\mathbb{S}^1$. Define 
\begin{align*}
    V_1 = \R^2 \setminus \{(x, 0) \in \R^2 : x \leq 0\}, \\ 
    V_2 = \R^2 \setminus \{(x, 0) \in \R^2 : x \geq 0\},
\end{align*}
which are both open sets. Then the homeomorphism 
\[ \alpha_1 : U_1 = (-\pi, \pi) \to \mathbb{S}^1 \cap V_1,\,
 t \mapsto (\cos t, \sin t) \]
covers all points on $\mathbb{S}^1$ except for $(-1, 0)$, while 
\[ \alpha_2 : U_2 = (0, 2\pi) \to \mathbb{S}^1 \cap V_2,\,
 t \mapsto (\cos t, \sin t) \]
covers all points on $\mathbb{S}^1$ except for $(1, 0)$. 

\subsection{More preliminaries} \label{subsec:1.3}
We now introduce another definition from topology.

\begin{defn}{defn:1.2}
    Let $A \subset \R^n$. A subset $U \subset A$ is {\bf relatively open} 
    if it is of the form $U = A \cap U'$ for some open set $U' \subset \R^n$. 
    Similarly, we say that $F \subset A$ is {\bf relatively closed} 
    if $F = A \cap F'$ for some closed set $F' \subset \R^n$. 
\end{defn}\vspace{-0.25cm}

For example, consider $\R$ equipped with the metric topology so that 
the open (respectively closed) sets are the unions of open intervals 
(respectively finite intersections of closed intervals). Let 
$A = [-1, 2) \subset \R$, which is neither open nor closed in $\R$. 
Take $U = [-1, 1) \subset \R$, which is again neither open nor closed in $\R$. 
But $U$ is relatively open in $A$ since $U = A \cap (-3, 1)$. Similarly, 
$F = [-1, 1] = A \cap [-1, 1]$ is relatively closed in $A$. 

Using the language of relatively open and closed sets, a lot of statements 
can be made simpler.
\begin{enumerate}[(1)]
    \item We define an {\bf open neighbourhood of $p$ in $A$} to be a relatively 
    open set $U$ containing $p$.
    \item The relatively open sets form a topology on $A$, called the 
    {\bf relative topology} (verify this as an exercise).
    \item We now have a more concise definition of a topological submanifold. 
    Let $M \subseteq \R^n$. Then $M$ is a {\bf $k$-dimensional topological 
    submanifold} of $\R^n$ if for every $p \in M$, there exists an 
    open neighbourhood $V$ of $p$ in $M$, an open set $U \subset \R^k$, 
    and a homeomorphism $\alpha : U \subset \R^k \to V \subset \R^n$. 
\end{enumerate}

\begin{defn}{defn:1.3}
    Let $A \subset \R^n$. Then $A$ is {\bf connected} if it cannot be written 
    in the form $A = U \cup V$ where $U, V \neq \varnothing$ are relatively 
    open in $A$ and $U \cap V = \varnothing$. Otherwise, we say that $A$ 
    is {\bf disconnected}; we call $U$ and $V$ {\bf disconnecting sets} for $A$.
\end{defn}\vspace{-0.25cm}

Let's go over a few example of connected sets. 
It can be shown that an open set in $\R^n$ is connected if and only if 
it is path connected; that is, there is a path between any two points in the set.
This result can help us build some intuition for what a connected set should 
look like. 
\begin{enumerate}[(1)]
    \item $\R^n$ is connected.
    \item Let $\alpha < \beta \in \R$. Then $(\alpha, \beta)$, 
    $(\alpha, \beta]$, $[\alpha, \beta)$, and $[\alpha, \beta]$ are 
    all connected. 
    \item Observe that $A = (-1, 0] \cup [1, 2]$ is a disconnected set because 
    $(-1, 0] = A \cap (-1.3, 0.3)$ and $[1, 2] = A \cap (0.9, 2.1)$ are 
    both relatively open in $A$ and disjoint. 
    \item The open ball $B_r(p)$ is connected for all $p \in \R^n$ and $r > 0$. 
\end{enumerate}
An important property of connected sets is that the continuous image of a 
connected set is connected! This can be used to prove that a subset 
$M \subset \R^n$ is not a submanifold. 

\begin{enumerate}[(1)]
    \item Consider the {\bf $\alpha$-curve} $C := \{(x, y) \in \R^2 : y^2 = x^2(x+1)\} \subset \R^2$.
    This can be parametrized by the map 
    \begin{align*}
        \alpha : \R &\to C \subset \R^2 \\
        t &\mapsto (t^2 - 1, t(t^2 - 1)).
    \end{align*}
    Note that $\alpha$ is not injective since $\alpha(-1) = \alpha(1) = (0, 0)$.
    That is, $\alpha$ is not a homeomorphism on $\R$, but it becomes one 
    if we remove the points $t = \pm1$, whose inverse is 
    \begin{align*}
        \alpha^{-1} : C \setminus \{(0, 0)\} &\to \R \setminus \{\pm1\} \subset \R \\ 
        (x, y) &\mapsto 1/x. 
    \end{align*} 
    Thus, $C$ is a $1$-dimensional submanifold away from the point $(0, 0)$. 
    
    Our goal now is to show that the whole of $C$ is not a topological submanifold 
    of $\R^2$. By contradiction, suppose that it were. By our above discussion, 
    it must have dimension $1$ because it has dimension $1$ away from $(0, 0)$. 
    Since $(0, 0) \in C$, it follows from the definition that there exists an 
    open neighbourhood $V$ of $(0, 0)$ in $C$, 
    an open set $U \subset \R^1$, and a homeomorphism 
    \[ \alpha : U \subset \R^2 \to V \subset C. \] 
    There must be a unique point $t_0 \in U$ such that $\alpha(t_0) = (0, 0)$ 
    since $\alpha$ is a bijection. Since $U$ is open and $t_0 \in U$, there 
    exists $\eps > 0$ such that $B_\eps(t_0) \subset U$. But $U \subset \R^1$, 
    so $B_\eps(t_0) = (t_0 - \eps, t_0 + \eps) =: U'$. Then $\alpha|_{U'}$ 
    is also a homeomorphism. Let $V' = \alpha(U')$. Observe that 
    $V' \setminus \{(0, 0)\}$ has three or four pieces depending on how large 
    the open set $U'$ is: one on the top right quadrant, one on the bottom right 
    quadrant, and one or two on the left of the $y$-axis. On the other hand, 
    $U' \setminus \{t_0\}$ has only two components, contradicting the fact that 
    homeomorphisms preserve the number of connected components. 

    \item Consider the {\bf double cone} $M = \{(x, y, z) \in \R^3 : 
    z^2 = x^2 + y^2\} \subset \R^3$. Away from $(0, 0, 0)$, every point in $M$ 
    lies on the graph of one of the continuous functions $f_1(x, y) =
    \sqrt{x^2+y^2}$ or $f_2(x, y) = -\sqrt{x^2 + y^2}$. Therefore, 
    $M \setminus \{(0, 0, 0)\}$ is a $2$-dimensional topological submanifold 
    of $\R^3$ since $f_1$ and $f_2$ are both functions of two variables. 

    However, there is a problem at the point $(0, 0, 0)$ since it 
    lies on the graph of both $f_1$ and $f_2$. Suppose that $M$ 
    is a topological submanifold of $\R^3$. Then $M$ must necessarily be of 
    dimension $2$ because $M \setminus \{(0, 0, 0)\}$ is of dimension $2$. 
    Then by definition, there exists an open neighbourhood $V$ of $(0, 0, 0)$, an 
    open set $U \subset \R^2$, and a homeomorphism 
    \[ \alpha : U \subset \R^2 \to V \subset M. \] 
    Since $\alpha$ is a bijection, there exists a unique point $(x_0, y_0) \in U$ 
    such that $\alpha(x_0, y_0) = (0, 0, 0)$. After shrinking $U$ (by the 
    same argument as above), we may take $U' = B_{\eps}((x_0, y_0))$ to 
    ensure that we have a connected set. Consider now the restriction 
    $\alpha|_{U'} : U' \to V = \alpha(U')$. Then $U' \setminus \{(x_0, y_0)\}$ 
    has one component, whereas $V' \setminus \{(0, 0, 0)\}$ has two components
    (that is, it is disconnected), which is a contradiction. 
\end{enumerate}

\newpage 
Now, we want to prove the invariance of dimension.
\begin{theo}[Invariance of Dimension]{theo:1.4}
    $\R^m$ is homeomorphic to $\R^n$ if and only if $m = n$. 
\end{theo}\vspace{-0.25cm}
If $m = n$, then $\R^m = \R^n$, so there is nothing to prove. The other 
implication is much harder, and we'll need the following result. 

\begin{theo}[Brouwer Invariance of Domain]{theo:1.5}
    Let $U \subset \R^n$ be open, and let $f : U \subset \R^n \to \R^n$ 
    be an injective continuous map. Then $f(U) \subset \R^n$ is open. 
    In particular, $f$ is a homeomorphism onto its image.
\end{theo}\vspace{-0.25cm}

An elementary proof can be found on 
\href{https://terrytao.wordpress.com/2011/06/13/brouwers-fixed-point-and-invariance-of-domain-theorems-and-hilberts-fifth-problem/}{Terry Tao's blog} 
where he uses the Brouwer Fixed Point Theorem to prove it. Nowadays, the 
standard proof uses algebraic topology. 

It is important that both the domain and codomain involve the same dimension $n$. 
For example, consider the injective continuous map $f : \R \to \R^2$ defined by 
$x \mapsto (x, 0)$. Observe that $f(U)$ is the $x$-axis, which is not open in $\R^2$. 

\begin{pf}[Theorem~\ref{theo:1.4}]
    We proceed by contradiction. Suppose that there is a homeomorphism 
    $f : \R^m \to \R^n$ and that $m > n$. Consider the inclusion 
    \begin{align*}
        \iota : \R^n &\to \R^m \\ 
        (x_1, \dots, x_n) &\mapsto (x_1, \dots, x_m, 0, \dots 0),
    \end{align*} 
    which is an injective continuous map. Then $\iota \circ f : \R^m 
    \to \R^m$ is also an injective continuous map since it is the composition of 
    two injective continuous maps. By Theorem~\ref{theo:1.5}, we have that 
    $\iota \circ f(\R^m)$ is an open set in $\R^m$. But this is impossible 
    because if $(x_1, \dots, x_n, 0, \dots, 0) \in \iota \circ f(\R^m)$, 
    then 
    \[ (x_1, \dots, x_n, \eps/2, 0, \dots, 0) \notin \iota \circ f(\R^m) \]  
    for all $\eps > 0$. Then $B_\eps((x_1, \dots, x_n, 0, \dots, 0)) 
    \not\subset \iota \circ f(\R^m)$ for all $\eps > 0$, implying that 
    $\iota \circ f(\R^m)$ is not open in $\R^m$. Therefore, we must have 
    $m \leq n$. If $n < m$, then we can repeat the same argument with 
    $f^{-1} : \R^n \to \R^m$, which again leads to a contradiction. 
    We conclude that $n = m$. \qed 
\end{pf}\vspace{-0.25cm}

Note that we actually proved something stronger: if $m > n$ and $U$ is a 
nonempty open subset of $\R^m$, then there is no continuous mapping from 
$U$ to $\R^n$. As a consequence, we get the following. 
\begin{prop}{prop:1.6}
    If $M \subset \R^n$ is a $k$-dimensional topological submanifold of 
    $\R^n$, then $k \leq n$. 
\end{prop}\vspace{-0.25cm}
\begin{pf}[Proposition~\ref{prop:1.6}]
    If $M \subset \R^n$ is a $k$-dimensional topological submanifold of $\R^n$, 
    then for all $p \in M$, there exists an open set $U \subset \R^k$, 
    an open neighbourhood $V \subset M$ of $p$, and a homeomorphism
    \[ \alpha : U \subset \R^k \to V \subset M \subset \R^n. \] 
    Since $\alpha$ is an injective continuous map, this forces $k \leq n$ 
    by the above discussion. \qed
\end{pf}\vspace{-0.25cm}
Finally, we must have the same $k$ for any chart $\alpha : U \subset \R^k 
\to V \subset M$. Indeed, let $p \in M$, and suppose that we have two 
different charts, say $\alpha : U \subset \R^k \to V \subset M$ and 
$\beta : U' \subset \R^{k'} \to V' \subset M$ where $p \in V \cap V'$. 
Then $V \cap V' \neq \varnothing$, so we can consider the restrictions 
\begin{align*}
    \alpha|_{\alpha^{-1}(V \cap V')} &: \alpha^{-1}(V \cap V') \to V \cap V', \\
    \beta|_{\beta^{-1}(V \cap V')} &: \beta^{-1}(V \cap V') \to V \cap V'.
\end{align*} 
Then $\beta^{-1} \circ \alpha : \alpha^{-1}(V \cap V') \subset \R^k 
\to \beta^{-1}(V \cap V') \subset \R^{k'}$ is a homeomorphism. Hence, 
$\alpha^{-1}(V \cap V')$ is a $k$-dimensional topological submanifold
of $\R^{k'}$. By Proposition~\ref{prop:1.6}, we have $k \leq k'$. 
Similarly, $\alpha^{-1} \circ \beta : \beta^{-1}(V \cap V') \subset 
\R^{k'} \to \alpha^{-1}(V \cap V') \subset \R^k$ is a homeomorphism, so 
$\beta^{-1}(V \cap V')$ is a $k'$-dimensional submanifold of $\R^k$. 
It follows that $k' \leq k$ and so $k' = k$. 

\subsection{Submanifolds of $\R^n$ of class $C^r$} \label{subsec:1.4}
Let $U \subset \R^n$ be an open set and consider the vector-valued function 
\begin{align*}
    F : U \subset \R^n &\to \R^m \\ 
    x = (x_1, \dots, x_n) &\mapsto (F_1(x), \dots, F_m(x)). 
\end{align*}
Recall that $F$ is of {\bf class $C^r$} for $r \geq 1$ if each component 
function $F_i : U \subset \R^n \to \R$ is of class $C^r$. That is, 
the partial derivatives of $F_i$ exist and are continuous up to order $r$. 
Also, we say that $F$ is of class $C^\infty$ or {\bf smooth} if each 
$F_i$ is smooth (the partial derivatives exist up to any order).
\begin{enumerate}[(1)]
    \item All polynomials are smooth. 
    \item The function $f(x) = x^{4/3}$ is of class $C^1$. Its derivative 
    $f'(x) = \frac43 x^{1/3}$ is continuous, but the second derivative 
    $f''(x) = \frac49 x^{-2/3}$ is not defined at $x = 0$. 
    \item The vector-valued function $F(x, y) = (2\cos x, xy - 1, e^{2\sin y+x})$ 
    is smooth on $\R^2$ because each component function is smooth. 
\end{enumerate}
The {\bf partial derivative} of $F$ with respect to the variable $x_j$ is 
\[ \frac{\partial F}{\partial x_j} := \left( \frac{\partial F_1}{\partial x_j}, 
\dots, \frac{\partial F_m}{\partial x_j} \right). \] 
If we fix a component function $F_i$, its {\bf gradient} is 
\[ \nabla F_i := \left( \frac{\partial F_i}{\partial x_1}, 
\dots, \frac{\partial F_i}{\partial x_n} \right). \] 
The {\bf derivative matrix} or {\bf Jacobian matrix} of $F$ is the $m \times n$ 
matrix 
\[ DF := \begin{bmatrix} 
    \partial F_1/\partial x_1 & \cdots & \partial F_1/\partial x_n \\ 
    \vdots & \ddots & \vdots \\ 
    \partial F_m/\partial x_1 & \cdots & \partial F_m/\partial x_n
\end{bmatrix}. \] 
That is, the rows correspond to the component functions $F_i$, and the 
columns correspond to the variables $x_j$. We can also think of the 
rows as the gradients and the columns as the partial derivatives; that is, 
we have 
\[ DF = \mleft[ \begin{array}{c|c|c}
    \dfrac{\partial F}{\partial x_1} & \cdots & \dfrac{\partial F}{\partial x_n}
\end{array} \mright] = \mleft[ \begin{array}{c}
    \nabla F_1 \\ \hline 
    \vdots \\[2pt] \hline 
    \nabla F_m \\
\end{array} \mright] \] 
In general, we want to work with some differentiability. 
This leads to the following definition. 

\begin{defn}{defn:1.7}
    Let $M \subset \R^n$. Suppose that for every $p \in M$, there exists 
    an open neighbourhood $V$ of $p$ in $M$, an open subset $U \subset \R^k$, 
    and a homeomorphism $\alpha : U \subset \R^k \to V \subset M$ such that 
    \begin{enumerate}[(1)]
        \item $\alpha$ is of class $C^r$ for some $r \geq 1$; 
        \item $D\alpha(x)$ has rank $k$ for all $x \in U$. 
    \end{enumerate}
    Then $M$ is called a {\bf $k$-dimensional submanifold of $\R^n$ of 
    class $C^r$}. We call $\alpha$ a {\bf coordinate chart} (or 
    {\bf coordinate patch}) about $p$.
\end{defn}\vspace{-0.25cm}

Note that every submanifold of class $C^r$ is a topological submanifold. 
We are only imposing the extra conditions (1) and (2) on the coordinate charts. 
We will see that condition (2) will allow us to define tangent spaces to 
the submanifolds at every point. A submanifold of class $C^\infty$ is 
called a {\bf smooth submanifold}. 

As usual, let's go over some examples. 
\begin{enumerate}[(1)]
    \item Let $U \subset \R^n$ be open. Then $\alpha : U \subset \R^n \to 
    V = U \subset \R^n$ sending $x$ to itself is smooth. Since the 
    component functions are $F_i(x) = x_i$ for all $i = 1, \dots, n$, we have 
    \[ D\alpha(x) = \left[ \frac{\partial F_i}{\partial x_j} \right] 
    = \left[ \frac{\partial x_i}{\partial x_j} \right] = [\delta_{ij}], \] 
    where $\delta_{ij}$ is the Kronecker delta. In other words, $D\alpha(x)$ 
    is the $n \times n$ identity matrix and has rank $n$ for all $x \in U$, 
    so $U \subset \R^n$ is a smooth $n$-dimensional submanifold of $\R^n$. 

    \item {\bf Graphs of functions of class $C^r$.} Let $U \subset \R^k$ 
    be an open set and consider a function 
    \begin{align*}
        F : U \subset \R^k &\to \R^{n-k} \\ 
        (x_1, \dots, x_k) &\mapsto (F_1(x), \dots, F_{n-k}(x)) 
    \end{align*} 
    of class $C^r$ (so each $F_i$ is of class $C^r$). Let 
    \[ M = \{(x, F(x)) \in \R^k \times \R^{n-k} : x \in U\} \subset \R^n \] 
    be the graph of $F$. We have already seen that $M$ is a $k$-dimensional 
    submanifold of $\R^n$ by taking $V = M$ and the homeomorphism 
    $\alpha : U \subset \R^k \to V \subset \R^n$ defined by 
    \[ \alpha(x) = (x, F(x)) = (x_1, \dots, x_k, F_1(x), \dots, F_{n-k}(x)). \]
    In particular, $\alpha$ is of class $C^r$ since the identity components 
    are smooth and the $F_i$ are of class $C^r$. Let's look at the 
    derivative matrix in terms of the columns of partial derivatives. We have 
    \[ \frac{\partial\alpha}{\partial x_j} = 
    \left( 0, \dots, 0, 1, 0, \dots, 0, \frac{\partial F_1}{\partial x_j}, 
    \dots, \frac{\partial F_{n-k}}{\partial x_j} \right) \] 
    where the $1$ corresponds to the $j$-th component, and hence 
    \[ D\alpha(x) = \mleft[ \begin{array}{cccc}
        1 & 0 & \cdots & 0 \\ 
        0 & 1 & \cdots & 0 \\ 
        \vdots & \vdots & \ddots & \vdots \\ 
        0 & 0 & \cdots & 1 \\ \hline 
        \partial F_1/\partial x_1 & \partial F_1/\partial x_2 & \cdots & \partial F_1/\partial x_k \\ 
        \vdots & \vdots & \ddots & \vdots \\ 
        \partial F_{n-k}/\partial x_1 & \partial F_{n-k}/\partial x_2 & \cdots & \partial F_{n-k}/\partial x_k \\
    \end{array} \mright] = \mleft[ \begin{array}{c} I_{k\times k} \\\hline DF(x) \end{array} \mright]. \]
    This matrix has rank $k$ for all $x \in U$, so $M$ is a $k$-dimensional 
    submanifold of class $C^r$. 

    \item We saw that the circle $\mathbb{S}^1 = \{(x, y) \in \R^2 : x^2 + y^2 = 1\} 
    \subset \R^2$ was a $1$-dimensional topological submanifold using
    the charts $\alpha_1 : U_1 = (-\pi, \pi) \subset \R^1 \to 
    \mathbb{S}^1 \setminus \{(-1, 0)\} \subset \R^2$ and 
    $\alpha_2 : U_2 = (0, 2\pi) \subset \R^1 \to \mathbb{S}^1 \setminus 
    \{(1, 0)\} \subset \R^2$, both defined by $t \mapsto (\cos t, \sin t)$. 
    Note that both $\alpha_i$ are smooth functions with derivative matrix 
    \[ D\alpha_i = \left[ \frac{\textrm{d}\alpha_i}{\textrm{d}t} \right] = 
    \begin{bmatrix} -\sin t \\ \cos t \end{bmatrix}, \] 
    which has rank $1$ because $\sin t$ and $\cos t$ don't have the same zeroes, 
    and hence $D\alpha_i$ is never the zero vector. Thus, $\mathbb{S}^1$ 
    is a smooth $1$-dimensional submanifold of $\R^2$. 
\end{enumerate}
Not every topological submanifold of $\R^n$ is of class $C^r$ 
for some $r \geq 1$. For example, consider the graph 
\[ M = \{(x, |x|) : x \in \R\} \] 
of the function $f(x) = |x|$ on $\R$, which fails to be differentiable 
at $x = 0$. Since $f$ is continuous, we know that 
$M$ is a $1$-dimensional topological submanifold of $\R^2$. Note that $f$ is 
smooth away from $x = 0$, so $M \setminus \{(0, 0)\}$ is a smooth $1$-dimensional
submanifold of $\R^2$. 

However, we claim that it cannot be a submanifold of 
class $C^r$ on any neighbourhood of the point $(0, 0)$. 
Suppose otherwise, so there exists an open set $U \subset \R^1$, 
an open neighbourhood $V \subset M$ of $(0, 0)$, and a homeomorphism 
$\alpha : U \subset \R^1 \to V \subset M \subset \R^2$ of class $C^r$ 
for some $r \geq 1$. Moreover, assume that $D\alpha(t)$ is of rank $1$ 
for all $t \in U$. Since $k = 1$, we have 
\[ D\alpha(t) = \frac{{\rm d}\alpha}{{\rm d}t} = \alpha'(t) \neq 0 \] 
for all $t \in U$ using the rank $1$ assumption. But $\alpha'(t)$ 
is tangent to $M$ at $\alpha(t)$. There are two possibilities: 
\begin{itemize}
    \item If $\alpha(t)$ is on the line $y = x$, then $\alpha'(t)$ 
    is a direction vector of $y = x$, so for some $c : I 
    \subset \R \to \R$, we have 
    \[ \alpha'(t) = c(t) \begin{pmatrix} 1 \\ 1 \end{pmatrix}. \tag{1.4.1} \label{eq:1.4.1}\] 
    Since $\alpha(t)$ is of class $C^r$ for some $r \geq 1$, we know that 
    $\alpha'(t)$ is continuous, so $c : I \to \R$ is also continuous.
    \item If $\alpha(t)$ is on $y = -x$, then $\alpha'(t)$ is a 
    direction vector of $y = -x$, so for some $d : I' \subset \R \to \R$, 
    we have 
    \[ \alpha'(t) = d(t) \begin{pmatrix} 1 \\ -1 \end{pmatrix}. \tag{1.4.2} \label{eq:1.4.2} \] 
    The same argument as above shows that $d : I' \to \R$ is continuous.
\end{itemize}
However, since $\alpha$ is a bijection, we have $(0, 0) = \alpha(t_0)$ 
for some $t_0 \in U$. By the continuity of $\alpha'(t)$, we obtain 
\[ \lim_{t\to t_0^-} \alpha'(t) = \lim_{t\to t_0^+} \alpha'(t). \] 
Without loss of generality, assume that $\alpha(t)$ is moving along $M$ 
from left to right. (Otherwise, we can simply parametrize in the other 
direction.) Then if $t < t_0$, equation $\eqref{eq:1.4.2}$ holds, whereas if 
$t > t_0$, equation $\eqref{eq:1.4.1}$ holds. This means that 
\[ \lim_{t\to t_0^-} \alpha'(t) = \lim_{t\to t_0^-} d(t) \begin{pmatrix} 1 \\ -1 \end{pmatrix} 
= \lim_{t\to t_0^+} c(t) \begin{pmatrix} 1 \\ 1 \end{pmatrix} = \lim_{t\to t_0^+} \alpha'(t). \] 
But the above vectors are not parallel to each other, so the only way that 
these limits are equal is if $\lim_{t\to t_0^-} d(t) = 
\lim_{t\to t_0^+} c(t) = 0$. This implies that $\alpha'(t_0)$ is the zero 
vector, which is a contradiction to the fact that $D\alpha(t) = \alpha'(t)$ 
has rank $1$ for all $t \in U$! This concludes the example that not 
every topological submanifold is a submanifold of class $C^r$ for some $r \geq 1$. 

In the definition of a submanifold of class $C^r$, it is important that 
$\alpha$ is a homeomorphism and not just a function of class $C^r$ 
with $D\alpha(x)$ of rank $k$ for all $x \in U$. These conditions alone
don't even ensure that $\alpha$ is injective! For example, take the 
$\alpha$-curve $C = \{(x, y) \in \R^2 : y^2 = x^2(x+1)\} \subset \R^2$ 
which we introduced back in Section~\ref{subsec:1.3}, which could be 
parametrized using $\alpha(t) = (t^2 - 1, t(t^2-1))$ for $t \in \R$. 
We saw that this map was not injective, but it is smooth with derivative matrix 
\[ D\alpha(t) = \frac{{\rm d}\alpha}{{\rm d}t} = \begin{pmatrix} 2t \\ 3t^2 - 1 \end{pmatrix} 
\neq \begin{pmatrix} 0 \\ 0 \end{pmatrix} \] 
for all $t \in \R$. A map satisfying these two conditions is said to be 
an {\bf immersion}, and a topological submanifold whose maps 
are immersions is called an {\bf immersed manifold}. We record this in 
the definition below. 

\begin{defn}{defn:1.8}
    Let $U \subset \R^k$ be open with $k \leq n$. A map $\alpha : U 
    \subset \R^k \to \R^n$ is an {\bf immersion} (of class $C^r$) if 
    \begin{enumerate}[(1)]
        \item $\alpha$ is of class $C^r$; and 
        \item $D\alpha(x)$ has rank $k$ for all $x \in U$. 
    \end{enumerate}
\end{defn}\vspace{-0.25cm}
We give some examples of immersions. 
\begin{enumerate}[(1)]
    \item {\bf Canonical immersion.} The inclusion map $\iota : \R^k \to \R^n$ defined by 
    $(x_1, \dots, x_k) \mapsto (x_1, \dots, x_k, 0, \dots, 0)$
    is an immersion of class $C^\infty$.
    Indeed, we see that $\iota$ is smooth and its derivative matrix is
    \[ D\iota = \mleft[ \begin{array}{ccc}
        1 & \cdots & 0 \\ 
        \vdots & \ddots & \vdots \\ 
        0 & \cdots & 1 \\ \hline 
        0 & \cdots & 0 \\ 
        \vdots & \ddots & \vdots \\ 
        0 & \cdots & 0 \\ 
    \end{array} \mright] \]
    which has rank $k$ because it contains the $k \times k$ identity matrix.

    \item The parametrization $\alpha(t) = (t^2 - 1, t(t^2 - 1))$ of the 
    $\alpha$-curve is an immersion of class $C^\infty$. 
    
    \item The charts $\alpha : U \to V$ of class $C^r$ of a submanifold 
    $M \subset \R^n$ of class $C^r$ are immersions of class $C^r$. 
\end{enumerate}
Recall that a {\bf diffeomorphism} is a differentiable bijection 
(of class $C^r$) whose inverse is also differentiable (of class $C^r$). The 
following proposition tells us that up to a diffeomorphism (by composition), 
every immersion is locally the canonical immersion.

\begin{prop}{prop:1.9}
    Let $U \subset \R^k$ be open and $\alpha : \R^k \to \R^n$ be an immersion 
    of class $C^r$. Then up to a local diffeomorphism, $\alpha$ is the 
    canonical immersion $\iota : \R^k \to \R^n$ defined by 
    $\iota(x_1, \dots, x_k) = (x_1, \dots, x_k, 0, \dots, 0)$. 
\end{prop}\vspace{-0.25cm}
In order to prove this, we require the Inverse Function Theorem. 

\begin{theo}[Inverse Function Theorem]{theo:1.10}
    Let $U \subset \R^\ell$ be open and let $F : U \subset \R^\ell \to \R^\ell$ 
    be of class $C^r$. Suppose that for some $x_0 \in U$, the $\ell \times \ell$ 
    derivative matrix $DF(x_0)$ is invertible (that is, $\det(DF(x_0)) \neq 0$). 
    Then $F$ is invertible in an open neighbourhood $U_0 \subset U$ of $x_0$ and 
    $F^{-1} : V_0 = F(U_0) \subset \R^\ell \to U_0 \subset \R^\ell$ 
    is also of class $C^r$.
\end{theo}\vspace{-0.25cm}

Note that when we take $\ell = 1$ in the Inverse Function Theorem 
and we have a function $f : U \subset \R^1 \to \R^1$, then $Df(x) = f'(x)$. 
If $f'(x_0) \neq 0$, then either $f'(x) > 0$ around $x_0$ or $f'(x) < 0$. 
In particular, $f$ is increasing or decreasing around $x_0$, which implies 
that it is strictly monotone and thus invertible around $x_0$. 

\begin{pf}[Proposition~\ref{prop:1.9}]
    Suppose that $\alpha : U \subset \R^k \to \R^n$ is an immersion given by 
    \[ x = (x_1, \dots, x_k) \mapsto (f_1(x), \dots, f_n(x)). \] 
    The derivative matrix of $\alpha$ is the $n \times k$ matrix 
    \[ D\alpha(x) = \begin{bmatrix}
        \nabla f_1(x) \\ \vdots \\ \nabla f_n(x) 
    \end{bmatrix} \] 
    and $D\alpha(x)$ has rank $k$ for all $x \in U$ by definition. Then 
    $k$ of the rows of $D\alpha(x)$ are linearly independent. Without loss of 
    generality, we can assume that these are the first $k$ rows after possibly 
    permuting the variables in $\R^n$. We now divide $\alpha$ into 
    the parts 
    \[ \alpha(x) = (\alpha_1(x), \alpha_2(x)), \] 
    where $\alpha_1 : U \subset \R^k \to \R^k$ corresponds to the first 
    $k$ component functions of $\alpha$, and $\alpha_2 : U \subset \R^k \to \R^{n-k}$ 
    corresponds to the remaining $n-k$ component functions. Then we can write 
    \[ D\alpha(x) = \mleft[ \begin{array}{c} D\alpha_1(x) \\ \hline D\alpha_2(x) \end{array} \mright] \]
    where $D\alpha_1(x)$ is a $k \times k$ matrix and $D\alpha_2(x)$ is an 
    $(n-k) \times k$ matrix. Notice that $D\alpha_1(x)$ has rank $k$, which 
    implies that $D\alpha_1(x)$ is invertible. By the Inverse Function Theorem 
    (Theorem~\ref{theo:1.10}), there exist open neighbourhoods $U_0 \subset U$ 
    of $x_0$ and $V_0 \subset \R^k$ of $\alpha_1(x_0)$ such that 
    \[ \alpha_1|_{U_0} : U_0 \subset U \subset \R^k \to V_0 \subset \R^k \] 
    is invertible with inverse 
    $(\alpha_1|_{U_0})^{-1} : V_0 \subset \R^k \to U_0 \subset U \subset \R^k$ 
    of class $C^r$. Hence, $\alpha_1|_{U_0}$ is a diffeomorphism of class $C^r$. 
    Now, consider the composition $\alpha \circ \alpha_1^{-1} : 
    V_0 \subset \R^k \to \R^n$ which yields 
    \[ x \mapsto \alpha(\alpha_1^{-1}(x)) = (\alpha_1(\alpha_1^{-1}(x)), 
    \alpha_2(\alpha_1^{-1}(x))) =: (x, f(x)) \]  
    where $f : V_0 \subset \R^k \to \R^{n-k}$. So up to the diffeomorphism, 
    $\alpha$ is the parametrization of the graph of a function of class $C^r$ 
    (because the composition is of class $C^r$). 

    If $f(x) = 0$ for all $x \in V_0$, then we are already done. Otherwise, 
    compose the above function with $h : V_0 \times \R^{n-k} \subset \R^n 
    \to \R^n$ defined by 
    \[ (x_1, \dots, x_k, x_{k+1}, \dots, x_n) \mapsto 
    (x_1, \dots, x_k, x_{k+1} - f_{k+1} \circ \alpha^{-1}(x), \dots, 
    x_n - f_n \circ \alpha^{-1}(x)), \] 
    which is a diffeomorphism of class $C^r$. For $x \in V_0$, we obtain 
    $h \circ \alpha \circ \alpha^{-1}(x) = (x, 0) = \iota(x)$ as desired. \qed 
\end{pf}\vspace{-0.25cm}
As a consequence, we have the following result. 

\begin{cor}{cor:1.11}
    The image of an immersion of class $C^r$ is locally the graph of a 
    function of class $C^r$ (up to diffeomorphism). In particular, 
    any submanifold of $\R^n$ of class $C^r$ is locally the graph of a 
    function of class $C^r$.
\end{cor}\vspace{-0.25cm} \newpage
\begin{pf}[Corollary~\ref{cor:1.11}]
    From the proof of Proposition~\ref{prop:1.9}, we had a map
    $\alpha \circ \alpha_1^{-1} : V_0 \subset \R^k \to \R^n$ which sent 
    $x \mapsto (x, f(x))$ by taking $f = \alpha_2 \circ \alpha_1^{-1}$ which 
    was of class $C^r$. Then $\alpha(V_0)$ is the graph of $f$ in $\R^n$. 
    
    If $M \subset \R^n$ is a $k$-dimensional submanifold of class $C^r$, 
    then for all $p \in M$, there exists a coordinate chart 
    \[ \alpha : U \subset \R^k \to V \subset M \] 
    with $p \in V$. Set $x_0 = \alpha^{-1}(p)$. Using the above notation, 
    there exist open sets $U_0, V_0 \subset \R^k$ with $x_0 \in U_0$ 
    and a diffeomorphism $\alpha_1 : U_0 \subset \R^k \to V_0 \subset \R^k$ 
    such that 
    \begin{align*} 
        \tilde\alpha := \alpha \circ \alpha_1^{-1} : V_0 \subset \R^k 
        &\to \alpha(U_0) \subset M \\ 
        x &\mapsto (x, f(x)) 
    \end{align*}
    is of class $C^r$ since $\alpha$ and $\alpha_1^{-1}$ are of 
    class $C^r$. (Note that $\alpha(U_0)$ is open in $M$ since $\alpha$ 
    is a homeomorphism.) This implies that $f$ is of class $C^r$, so $M$ 
    is locally the graph of the $C^r$ function $f : V_0 \subset \R^k \to \R^{n-k}$. \qed 
\end{pf}\vspace{-0.25cm}

We now introduce the notion of an embedding.
\begin{defn}{defn:1.12}
    Let $U \subset \R^k$ be open and $\alpha : U \subset \R^k \to \R^n$ 
    be a map. Then $\alpha$ is an {\bf embedding} of class $C^r$ if 
    \begin{enumerate}[(1)]
        \item $\alpha$ is a homeomorphism onto its image; 
        \item $\alpha$ is of class $C^r$; 
        \item $D\alpha(x)$ has rank $k$ for all $x \in U$. 
    \end{enumerate}
\end{defn}\vspace{-0.25cm}
In other words, an embedding is just an immersion that is homeomorphic onto 
its image. In particular, submanifolds of $\R^n$ of class $C^r$ are subsets 
of $\R^n$ that are locally the image of an embedding of class $C^r$. 

The following proposition tells us that embeddings are in fact 
diffeomorphisms onto their images. 
\begin{prop}{prop:1.13}
    Let $U \subset \R^k$ be open, and let $\alpha : U \subset \R^k \to 
    \R^n$ be an embedding of class $C^r$. Then 
    \[ \alpha^{-1} : \alpha(U) \subset \R^n \to U \subset \R^k \] 
    is also of class $C^r$. 
\end{prop}\vspace{-0.25cm}
\begin{pf}[Proposition~\ref{prop:1.13}]
    Note that $\alpha$ is an immersion of class $C^r$, so we can write it as 
    $\alpha(x) = (\alpha_1(x), \alpha_2(x))$,
    where $\alpha_1 : U \subset \R^k \to \R^k$ is locally invertible. 
    Letting $\pi : \R^n \to \R^k$ be the projection $(y_1, \dots, y_n) 
    \mapsto (y_1, \dots, y_k)$, we have 
    \begin{align*} 
        \alpha^{-1} : \alpha(U) \subset \R^n &\to U \subset \R^k \\ 
        (\alpha_1(x), \alpha_2(x)) &\mapsto x = \alpha_1^{-1} \circ 
        \pi(x_1, \dots, x_n).
    \end{align*}
    But $\alpha_1^{-1}$ and $\pi$ are both of class $C^r$, so 
    $\alpha^{-1}$ is also of class $C^r$. \qed 
\end{pf}\vspace{-0.25cm}
We now introduce atlases, which we could've done a while ago when we 
defined topological submanifolds at the beginning. However, we can now 
talk about atlases of class $C^r$.

\begin{defn}{defn:1.14}
    Let $M \subset \R^n$ be a $k$-dimensional topological submanifold. 
    An {\bf atlas} of $M$ is a collection of charts 
    \[ \{\alpha_a : U_a \subset \R^k \to V_a \subset M\}_{a \in A} \] 
    such that $\bigcup_{a \in A} V_a = M$, where each $U_a \subset \R^k$ 
    and $V_a \subset M$ is open. Moreover, if all the charts in the atlas 
    are of class $C^r$, then it is called an 
    {\bf atlas of class $C^r$}, or a {\bf smooth atlas} if it is of 
    class $C^\infty$. 
\end{defn}\vspace{-0.25cm}
We look at some examples of atlases. 
\begin{enumerate}[(1)]
    \item Let $U \subset \R^n$ be open. Then $\alpha = \Id_U$ is a smooth 
    chart for $U$ such that every point in $U$ is contained in $\alpha(U) = U$. 
    Therefore, $\{\alpha = \Id_U : U \to U\}$ is a smooth atlas for $U$. 
    \item Consider the graph $M$ of a function $F : U \subset \R^k \to \R^{n-k}$ 
    of class $C^r$, where $U \subset \R^k$ is open. Then 
    $M = \{(x, F(x)) : x \in U\}$ is a $k$-dimensional 
    submanifold of class $C^r$. It admits the chart $\alpha : U \subset \R^k 
    \to M \subset \R^n$ defined by $x \mapsto (x, F(x))$ of class $C^r$, 
    so $\{\alpha : U \to M\}$ is an atlas of class $C^r$ for $M$. 
    \item Recall that the points on the circle $\mathbb{S}^1 = 
    \{(x, y) : x^2 + y^2 = 1\} \subset \R^2$ can be described using 
    the charts $\alpha_1 : (-\pi, \pi) \to \mathbb{S}^1 \setminus \{(-1, 0)\}$ 
    and $\alpha_2 : (0, 2\pi) \to \mathbb{S}^1 \setminus \{(1, 0)\}$ 
    via $t \mapsto (\cos t, \sin t)$. Then $\{\alpha_1, \alpha_2\}$ 
    is a smooth atlas for $\mathbb{S}^1$. 
\end{enumerate}
Note that if $M$ is compact (that is, closed and bounded), then any atlas 
of $M$ must contain at least two charts. Moreover, atlases are not unique in general! 
For example, consider $\{\alpha : \R \to M,\, t \mapsto (t, 0)\}$ 
and $\{\beta : \R \to M,\, t \mapsto (-t, 0)\}$, which are both smooth 
atlases for the $x$-axis $M$ in $\R^2$. 

We now give an alternate definition of a $k$-dimensional submanifold 
of $\R^n$ of class $C^r$. We will soon prove that this is equivalent to the 
original definition.

\begin{defn}{defn:1.15}
    Let $M \subset \R^n$ be such that $M$ is locally given by the zero set 
    $\{F \equiv 0\}$ of a $C^r$ map $F : V \subset \R^n \to \R^{n-k}$ 
    with maximal rank. That is, 
    $DF(x)$ has rank $n - k$ for all $x \in V \cap M$, where $V \cap M 
    = F^{-1}(0)$ holds for an appropriately chosen neighbourhood $V$ 
    of every point in $M$. Then $M$ is called a 
    {\bf $k$-dimensional submanifold of $\R^n$ of class $C^r$}.
\end{defn}\vspace{-0.25cm} 
This alternate definition is useful, because it is generally easier to show 
that a given subset is locally the zero set of a function than to 
explicitly exhibit charts covering the space. 
\begin{enumerate}[(1)]
    \item Consider the circle $\mathbb{S}^1 = \{(x, y) \in \R^2 : x^2 + y^2 = 1\}$, 
    which is given by the equation $x^2 + y^2 = 1$, which we can rearrange as 
    $x^2 + y^2 - 1 = 0$.
    Define $F : \R^2 \to \R$ via $F(x, y) = x^2 + y^2 - 1$. Then $\mathbb{S}^1 
    = F^{-1}(0)$, so we can take $V = \R^2$ in the definition. Also, we have 
    \[ DF(x, y) = \begin{bmatrix}
        2x & 2y 
    \end{bmatrix}, \] 
    which has rank $1$ for every point since $(0, 0) \notin \mathbb{S}^1$. 
    Thus, under the alternate definition, $\mathbb{S}^1$ is a smooth $1$-dimensional
    submanifold of $\R^2$. 

    \item More generally, the $n$-sphere $\mathbb{S}^n = \{(x_1, \dots, 
    x_{n+1}) \in \R^{n+1} : x_1^2 + \cdots + x_{n+1}^2 = 1\} \subset \R^{n+1}$ 
    can be viewed as the zero set of the smooth function $F : \R^{n+1} \to \R$ 
    given by $F(x_1, \dots, x_{n+1}) = x_1^2 + \cdots + x_{n+1}^2 - 1$. 
    Here, we can take $V = \R^{n+1}$ in the definition. Since 
    \[ DF(x) = \begin{bmatrix} 2x_1 & \cdots & 2x_{n+1} \end{bmatrix} \] 
    has rank $1$ for all $x \in \mathbb{S}^n$, we see that $\mathbb{S}^n$ 
    is a smooth $n$-dimensional submanifold of $\R^{n+1}$. 

    \item Consider the twisted cubic $M = \{(x, y, z) : y = x^2,\, z = x^3\} \subset \R^3$.
    We have seen that this is a smooth $1$-dimensional submanifold of $\R^3$. 
    For all $(x, y, z) \in M$, we have $y = x^2$ and $z = x^3$. Rearranging 
    gives $y - x^2 = 0$ and $z - x^3 = 0$, so we can view $M$ has the zero 
    set of the smooth function $F(x, y, z) = (y-x^2, z-x^3)$ defined for all 
    $(x, y, z) \in \R^3$. This time, we have $V = \R^3$, and $F : 
    \R^3 \to \R^2$ so that $n = 3$ and $k = 1$. The derivative matrix of $F$ is 
    \[ DF(x, y, z) = \begin{bmatrix} 
        -2x & 1 & 0 \\ 
        -3x^2 & 0 & 1 
    \end{bmatrix} \] 
    which is rank $2$ for all $(x, y, z) \in M$ because the last two columns 
    are the $2 \times 2$ identity matrix. Hence, $M$ is a smooth $1$-dimensional 
    submanifold of $\R^3$.     

    {\bf Remark.} Not all zero sets of functions of class $C^r$ are submanifolds 
    of class $C^r$. Take the $\alpha$-curve $C = \{(x, y) \in \R^2 : y^2 = 
    x^2(x+1)\}$, which is the zero set of the function 
    $F : \R^2 \to \R$ defined by $(x, y) \mapsto y^2 - x^2(x+1)$. 
    The derivative matrix of $F$ is 
    \[ DF(x, y) = \begin{bmatrix} -3x^2 - 2x & 2y \end{bmatrix}, \] 
    which is equal to the zero vector if and only if $(x, y) \in 
    \{(0, 0), (-2/3, 0)\}$. We see that $(0, 0) \in C$, and this 
    is the problematic point. 

    \item {\bf (Graph of a function of class $C^r$.)} Let $f : U \subset \R^k 
    \to \R^{n-k}$ be function of class $C^r$, where $U \subset \R^k$ is open. 
    Let $M = \{(x, f(x)) \in \R^k \times \R^{n-k} : x \in U\} \subset \R^n$,
    which we have already seen is a $k$-dimensional submanifold of class $C^r$. 
    We now view this using the zero set characterization. 

    For all $(x, y) \in M \subset \R^k \times \R^{n-k}$, we have 
    $y = f(x)$ if and only if $F(x, y) := f(x) - y = 0 \in \R^{n-k}$.
    Note that $U \times \R^{n-k}$ is an open subset of $\R^k \times \R^{n-k}$. 
    Then $M$ is the zero set of the $C^r$ function $F : U \times \R^{n-k} 
    \subset \R^k \times \R^{n-k} \to \R^{n-k}$ given by $(x, y) \mapsto f(x) - y$.

    It remains to check the rank condition. We have that 
    \[ DF(x, y) = \left[ \begin{array}{c|c} 
        \dfrac{\partial F}{\partial x} & \dfrac{\partial F}{\partial y} 
    \end{array} \right] \]
    which has $n-k$ rows. Since $F(x, y) = f(x) - y$, we see that 
    $\partial F/\partial x = Df(x)$ and $\partial F/\partial y = -I_{n-k}$, 
    so $DF(x, y)$ has rank $n-k$ since $I_{n-k}$ does. 
\end{enumerate}

\begin{theo}{theo:1.16}
    Let $M \subset \R^n$. The following are equivalent: 
    \begin{enumerate}[(i)]
        \item $M$ is a $k$-dimensional submanifold of class $C^r$ (using Definition~\ref{defn:1.7}).
        \item $M$ is locally the graph of a function $f : U \subset \R^k \to \R^{n-k}$ 
        of class $C^r$, where $U \subset \R^k$ is open. 
        \item $M$ is locally the zero set of a $C^r$ function $F : V \subset \R^n 
        \to \R^{n-k}$ of maximal rank, where $V \subset \R^n$ is open 
        (using Definition~\ref{defn:1.15}).
    \end{enumerate}
\end{theo}\vspace{-0.25cm}

Due to Corollary~\ref{cor:1.11}, we already know that (i) and (ii) are 
equivalent. Example (4) above shows that (ii) implies (iii). Therefore, 
it suffices to prove that (iii) implies (ii), and we will see that this is a 
direct consequence of the Implicit Function Theorem. Let's recall what this says.

Suppose that $\R^n = \R^{k+m} = \R^k \times \R^m$ has coordinates $(x, y) 
= (x_1, \dots, x_k, y_1, \dots, y_m)$. Let $U \subset \R^n$ be an open subset 
and $F : U \subset \R^n \to \R^m$ be a function of class $C^r$. Then the 
derivative matrix of $F$ is 
\[ DF(x, y) = \left[ \begin{array}{c|c} 
    \dfrac{\partial F}{\partial x} & \dfrac{\partial F}{\partial y} 
\end{array} \right]. \]
where if $F_1, \dots, F_m$ are the component functions of $F$, 
then we have $\partial F/\partial x = (\partial F_i/\partial x_j)_{1\leq i\leq m, 
1\leq j\leq n}$ and $\partial F/\partial y = 
(\partial F_i/\partial y_j)_{1\leq i, j\leq m}$. In particular, 
$\partial F/\partial y$ is an $m \times m$ matrix.

\begin{theo}[Implicit Function Theorem]{theo:1.17}
    Let $(x_0, y_0) \in U$ be such that $F(x_0, y_0) = 0$. Suppose that 
    \[ \det\left( \frac{\partial F}{\partial y}(x_0, y_0) \right) \neq 0. \] 
    Then there exists an open neighbourhood $V_0 \subset \R^k$ of $x_0$ 
    and a unique function $g : V_0 \to \R^m$ of class $C^r$ such that 
    $g(x_0) = y_0$ and $F(x, g(x)) = 0$ for all $x \in V_0$. 
\end{theo}\vspace{-0.25cm}

In other words, the Implicit Function Theorem tells us that if 
$\det(\partial F(x_0, y_0)/\partial y) \neq 0$, then 
for all points $(x, y) \in \{F \equiv 0\}$ in an open neighbourhood of 
$(x_0, y_0)$, we have $y = g(x)$ for some function of class $C^r$. Thus, 
we can express the variables $(y_1, \dots, y_m)$ as functions of 
$(x_1, \dots, x_k)$ of class $C^r$ near $(x_0, y_0)$. 

Before proving the theorem, we illustrate what the result tells us 
with a simple example. Let $F(x, y) = x^2 + y^2 - 1$ for all $(x, y) \in \R^2$. 
Then $F : \R^2 \to \R$ is smooth and has derivative matrix 
\[ DF(x, y) = \begin{bmatrix}
    2x & 2y 
\end{bmatrix}. \] 
In this case, we have $m = 1$ and $m + k = 2$ so that $k = 1$. We are writing 
$\R^2 = \R^1 \times \R^1$ where $x$ corresponds to the first copy of $\R^1$ 
and $y$ corresponds to the second copy. Then $\partial F/\partial y = 2y$, 
which is nonzero if and only if $y \neq 0$. By the Implicit Function 
Theorem, the points on 
\[ \{F \equiv 0\} = \{(x, y) : x^2 + y^2 - 1 = 0\} \] 
have a $y$-coordinate that can be expressed locally as a function of $x$. 
This is indeed true since $x^2 + y^2 - 1 = 0$ if and only if 
$y = \pm\sqrt{1-x^2}$, which is smooth for $x \notin \{\pm1\}$ and hence 
for $y \neq 0$ on $\{F \equiv 0\}$. These parametrize $\mathbb{S}^1 
\setminus \{(\pm1, 0)\}$. Similarly, note that 
$\partial F/\partial x = 2x \neq 0$ if and only if $x \neq 0$, so this 
holds for the points $(x, y) \neq (0, \pm1)$ on $\mathbb{S}^1$. We see that 
$\mathbb{S}^1$ can be expressed as $x = \pm\sqrt{1-y^2}$ away from 
$(0, \pm1)$.

\begin{pf}[Implicit Function Theorem (Theorem~\ref{theo:1.17})]
    This follows from the Inverse Function Theorem (Theorem~\ref{theo:1.10}).
    Define $H : U \subset \R^n \to \R^n = \R^k \times \R^m$ by $(x, y) 
    \mapsto (x, F(x, y))$. The derivative matrix is
    \[ DH = \left[ \begin{array}{c|c}
        I_{k\times k} & 0 \\ \hline
        \partial F/\partial x & \partial F/\partial y
    \end{array} \right], \]
    which is an $n \times n$ matrix. 
    Since $I_{k\times k}$ has rank $k$ and $\partial F/\partial y$ has 
    rank $m$ at $(x_0, y_0)$ using the fact that it is invertible there, 
    it follows that $DH(x_0, y_0)$ has rank $k + m = n$. That is, 
    $\det(DH(x_0, y_0)) \neq 0$ and $H$ is locally invertible with 
    some inverse $G$ by the Inverse Function Theorem.

    Write $G(u, v) = (G_1(u, v), G_2(u, v))$ where $u \in \R^k$ and
    $v \in \R^m$, and we separate $G$ into the components $G_1(u, v) \in \R^k$ and 
    $G_2(u, v) \in \R^m$. Then we have 
    \begin{align*}
        (u, v) &= H \circ G(u, v) \\ 
        &= H(G_1(u, v), G_2(u, v)) \\ 
        &= (G_1(u, v), F(G_1(u, v), G_2(u, v))). 
    \end{align*}
    This implies that $u = G_1(u, v)$, so $G(u, v) = (u, G_2(u, v))$ for some 
    function $G : V_0 \subset \R^n \to \R^m$ of class $C^r$ (where $V_0 
    \subset \R^n$ is open). Moreover, for all $(x, y)$ with $F(x, y) = 0$, 
    we have $H(x, y) = (x, F(x, y)) = (x, 0)$,
    and hence $(x, y) = G \circ H(x, y) = G(x, 0) = (x, G_2(x, 0))$. Then 
    $y = G_2(x, 0)$ for all $(x, y) \in \{F \equiv 0\}$. By setting 
    $g(x) := G_2(x, 0)$, we have $y = g(x)$ for all $(x, y) \in \{F \equiv 0\}$
    near $(x_0, y_0)$, and $g$ is of class $C^r$. We see that 
    $F(x, g(x)) = 0$.
    For the proof of uniqueness of $g$ and more details, we refer to 
    \emph{Topology} by Munkres, Theorem 9.2 on page 74. \qed 
\end{pf}\vspace{-0.25cm} 

Finally, we prove that the definitions are equivalent. Recall from 
our earlier discussion that it suffices to show that (iii) implies (ii). 

\begin{pf}[Theorem~\ref{theo:1.16}]
    Suppose that $M$ is locally the zero set of a $C^r$ function $F : 
    V \subset \R^n \to \R^{n-k}$ of maximal rank on $M \cap V$. Then 
    for all $(x_0, y_0) \in M \cap V \subset \R^k \times \R^{n-k} = \R^n$, 
    the derivative matrix 
    \[ DF(x_0, y_0) = \left[ \begin{array}{c|c}
        \!\!\dfrac{\partial F}{\partial x}(x_0, y_0) & \dfrac{\partial F}{\partial y}(x_0, y_0)\!\!
    \end{array} \right] \] 
    has rank $n-k$, where $\frac{\partial F}{\partial y}(x_0, y_0)$ is 
    an $(n-k) \times (n-k)$ matrix.

    After possibly permuting the variables $(x_1, \dots, x_k, y_1, \dots, 
    y_{n-k})$ and therefore the columns of $DF(x_0, y_0)$, we may assume that 
    $\partial F(x_0, y_0)/\partial y$ has rank $n-k$. This implies that 
    $\det(\partial F(x_0, y_0)/\partial y) \neq 0$. By the Implicit 
    Function Theorem (Theorem~\ref{theo:1.17}), we see that $y$ 
    is a $C^r$ function of $x$ on $M \cap V$ near $(x_0, y_0)$. 
    Write $y = g(x)$ for some $C^r$ function $g : V_0 \subset \R^k 
    \to \R^m$, where $V_0$ is an open neighbourhood of $x_0$. 
    Then $M$ is locally of the form $\{(x, g(x)) : x \in V_0\}$,
    which implies that $M$ is locally the graph of $g$. \qed 
\end{pf}\vspace{-0.25cm}

To end our discussion, we give one more example of how we can use the 
zero set characterization to find charts for a submanifold $M$ 
using the Implicit Function Theorem. Consider the special linear group 
\[ \SL(2, \R) = \{A \in M_{2\times 2}(\R) : \det A = 1\}. \] 
In a natural way, we can identify the matrix 
\[ A = \begin{bmatrix}
    a_1 & a_2 \\ a_3 & a_4 
\end{bmatrix} \in M_{2\times 2}(\R) \] 
with the point $(a_1, a_2, a_3, a_4) \in \R^4$, and we have 
\[ \det A - 1 = a_1 a_4 - a_2 a_3 - 1 =: F(a_1, a_2, a_3, a_4) \] 
so that $\SL(2, \R) = \{F \equiv 0\}$ where $F : \R^4 \to \R$ is smooth. 
The derivative matrix of $F$ is 
\[ DF(a_1, a_2, a_3, a_4) = \begin{bmatrix}
    a_4 & -a_3 & -a_2 & a_1
\end{bmatrix}, \] 
which can have rank at most $1$ since it is a $1 \times 4$ matrix.
In particular, $DF(a_1, a_2, a_3, a_4)$ has rank $1$ if and only if it is 
not the zero vector, which occurs if at least one of the $a_i$ is nonzero. But 
for all $(a_1, a_2, a_3, a_4) \in \SL(2, \R)$, we have $a_1a_4 - a_2a_3 = 1$ 
so that at least one of the $a_i$ must be nonzero. 

Suppose that $a_1 \neq 0$. Then we have $\partial F/\partial a_4 = a_1 \neq 0$, 
so the Implicit Function Theorem tells us that $a_4$ can be expressed as a 
smooth function of the remaining three variables on $\SL(2, \R)$. Indeed, if 
$(a_1, a_2, a_3, a_4) \in \SL(2, \R)$ with $a_1 \neq 0$, then 
$a_1a_4 - a_2a_3 = 1$ can be rearranged to obtain 
\[ a_4 = \frac{a_2a_3 + 1}{a_1}. \] 
A similar analysis can be done when the other variables are nonzero. 
Since $DF$ has rank $1$ everywhere, it follows that $\SL(2, \R)$ is a 
smooth submanifold of $\R^4$ of dimension $4-1 = 3$. 

\subsection{Tangent vectors and tangent vector fields} \label{subsec:1.5}
The simplest example of a tangent vector is the velocity vector of a curve. 

\begin{defn}{defn:1.18}
    Let $\gamma : (a, b) \subset \R \to \R^n$ be a map of class $C^r$. 
    We define the {\bf velocity vector} of $\gamma$ at $\gamma(t)$ to be 
    \[ \gamma'(t) := D\gamma(t). \] 
\end{defn}\vspace{-0.25cm}

Note that if $\gamma(t) = (\gamma_1(t), \dots, \gamma_n(t))$, then 
$\gamma'(t) = (\gamma'_1(t), \dots, \gamma'_n(t))$. Moreover, we have 
\[ \gamma'(t_0) = \lim_{t\to t_0} \frac{\gamma(t) - \gamma(t_0)}{t - t_0}. \] 
Observe that $(\gamma(t) - \gamma(t_0))/(t - t_0)$ is the velocity vector 
of the secant $L$ passing through $\gamma(t)$ and $\gamma(t_0)$. 
Taking the limit as $t \to t_0$, it follows that $\gamma'(t_0)$ is the tangent 
vector to the curve in $\R^n$ given by $\gamma(t)$, under the assumption
that $\gamma'(t_0) \neq 0$. Let's look at some examples. 

\begin{enumerate}[(1)]
    \item Let $x \in \R^n$ and $0 \neq v \in \R^n$. Set $\gamma(t) = x + tv$ 
    for $t \in \R$, which parametrizes the line $L$ in $\R^n$ passing through 
    $x$ with direction vector $v$. We have that $\gamma'(t) = v$ for all 
    $t \in \R$, so at every point on the line, the velocity vector coincides 
    with the direction vector. 

    \item Let $\gamma : \R \to \R^2$ be defined by $t \mapsto (\cos t, \sin t)$, 
    which parametrizes $\mathbb{S}^1$. Note that $\gamma'(t) = 
    (-\sin t, \cos t) \neq (0, 0)$ for all $t \in \R$. 

    \item Let $\gamma : \R \to \R^2$ be given by $t \mapsto (t^2, t^3)$. 
    This parametrizes the cusp curve $y^2 = x^3$. We have $\gamma'(t) = (2t, 3t^2)$. 
    Observe that $\gamma'(0) = (0, 0)$ so that the velocity vector 
    is zero at $\gamma(0) = (0, 0)$. 

    \item Recall that the $\alpha$-curve can be parametrized with 
    $\gamma : \R \to \R^2$ defined by $t \mapsto (t^2-1, t(t^2-1))$. 
    We have $\gamma'(t) = (2t, 3t^2-1) \neq (0, 0)$ for all $t \in \R$. 
\end{enumerate}

Note that if $\gamma$ is a homeomorphism onto its image 
\[ C = \{x \in \R^n : x = \gamma(t) \text{ for some } t \in (a, b)\} \] 
and $D\gamma(t)$ has rank $1$ for all $t \in (a, b)$, then $C$ is a 
$1$-dimensional submanifold of $\R^n$ of class $C^r$. In fact, we see that
$D\gamma(t)$ has rank $1$ if and only if $\gamma'(t) \neq 0$, so $\gamma'(t)$ 
is a direction vector for the tangent line to $\gamma$ at $\gamma(t)$. 
In particular, the $1$-dimensional submanifold of $\R^n$ determined by $\gamma$ 
has a well-defined tangent line $L$ at every point. We call 
$L := \{\gamma(t_0) + s\gamma'(t_0) : s \in \R\}$
the {\bf tangent line to $C$ at $\gamma(t_0)$}. 

From example (1) above, we know that a line $\gamma(t) = x + tv$ coincides 
with its tangent line at every point.

\begin{defn}{defn:1.19}
    Let $x \in \R^n$. A {\bf tangent vector to $\R^n$ at $x$} is defined as 
    a pair $(x; v)$ where $v \in \R^n$. We call 
    \[ T_x(\R^n) := \{(x; v) : v \in \R^n\} \] 
    the {\bf tangent space to $\R^n$ at $x$}. This is the set of all 
    tangent vectors to $\R^n$ at $x$. 
\end{defn}\vspace{-0.25cm}

We can give $T_x(\R^n)$ a vector space structure with the operations 
$(x; v) + (x; w) = (x; v+w)$ and $c(x; v) = (x; cv)$ for all 
$c \in \R$ and $v, w \in \R^n$. Note that $T_x(\R^n) \simeq \R^n$ as vector 
spaces, where the isomorphism $T_x(\R^n) \to \R^n$ is given by $(x; v) \mapsto v$. 

\begin{lemma}{lemma:1.20}
    For all $x \in \R^n$, we have 
    \[ T_x(\R^n) = \{(x; v) : v \text{ is a velocity vector of some 
    curve $\gamma(t)$ passing through $x$}\}. \] 
\end{lemma}\vspace{-0.25cm}
\begin{pf}[Lemma~\ref{lemma:1.20}]
    Let $(x; v) \in T_x(\R^n)$. Set $\gamma(t) = x + tv$ for $t \in \R$, 
    which satisfies $\gamma'(t) = v$ for all $t \in \R$. In particular, we 
    have $\gamma(0) = x$ and $\gamma'(0) = v$. Note that if $v \neq 0$, then 
    $\gamma(t)$ parametrizes the line through $x$ with velocity vector $v$. \qed 
\end{pf}\vspace{-0.25cm}

Can we get something similar for submanifolds $M \subset \R^n$? 
Towards this direction, we make a new definition. 

\begin{defn}{defn:1.21}
    Let $U \subset \R^k$ be open and let $\alpha : U \to \R^n$ be of class $C^r$. 
    Also, let $x \in U$ and set $p = \alpha(x)$. The map 
    \begin{align*}
        \alpha_* : T_x(\R^k) &\to T_p(\R^n) \\ 
        (x; v) &\mapsto (p; D\alpha(x) v)
    \end{align*}
    is called the {\bf pushforward of $\alpha$ at $x$}. 
\end{defn}\vspace{-0.25cm}

Note that $\alpha_* : T_x(\R^k) \to T_p(\R^n)$ is a vector space homomorphism since it is linear; 
in particular, it is given by multiplication by the $n\times k$ derivative matrix. 

For all $(x; v) \in T_x(\R^k)$, consider the map $\gamma : (-\eps, \eps) \to \R^n$
defined by $t \mapsto \alpha(x+tv)$
with $\eps > 0$ such that $x + tv \in U$ for all $t \in (-\eps, \eps)$. 
That is, the line segment $L = \{x + tv : t \in (-\eps, \eps)\}$
is included in the open set $U$. Then for all $t_0 \in (-\eps, \eps)$, 
the chain rule gives 
\begin{align*}
    \gamma'(t_0) &= D\alpha(x + t_0 v) D(x + tv)(t_0) 
    = D\alpha(x + t_0 v) v 
\end{align*}
since $D(x + tv)(t_0) = v$ for all $t_0 \in \R$. In particular, we obtain 
\begin{align*}
    (\gamma(0); \gamma'(0)) &= (\alpha(x); D\alpha(x)v) \\ 
    &= (p, D\alpha(x)v) \\ 
    &= \alpha_*(x; v). 
\end{align*}
This tells us that $\alpha_*(x; v)$ is the velocity vector of 
$\gamma(t) := \alpha(x + tv)$ at $p = \gamma(0)$.

\begin{lemma}{lemma:1.22}
    If $\alpha : U \subset \R^k \to \R^n$ and $\beta : V \subset \R^\ell 
    \to \R^k$ are functions of class $C^r$ where $U \subset \R^k$ and $V 
    \subset \R^\ell$ are open with $\beta(V) \subset U$, then 
    $(\alpha \circ \beta)_* = \alpha_* \circ \beta_*$ on $V$. 
\end{lemma}\vspace{-0.25cm} 
\begin{pf}[Lemma~\ref{lemma:1.22}]
    Note that $\alpha \circ \beta : V \subset \R^\ell \to \R^n$ is 
    of class $C^r$. Let $x \in V$. Then for all $v \in \R^\ell$, we have 
    \begin{align*}
        (\alpha \circ \beta)_*(x; v) 
        &= (\alpha \circ \beta(x); D(\alpha \circ \beta)(x)v) \\ 
        &= (\alpha(\beta(x)); D\alpha(\beta(x))D\beta(x)v) \\ 
        &= \alpha_*(\beta(x); D\beta(x)v) \\ 
        &= \alpha_*(\beta_*(x; v)), 
    \end{align*} 
    where the second equality follows from the chain rule. \qed 
\end{pf}\vspace{-0.25cm}

From Lemma~\ref{lemma:1.22}, it now makes sense to define the following. 

\begin{defn}{defn:1.23}
    Let $M \subset \R^n$ be a $k$-dimensional submanifold of $\R^n$ 
    of class $C^r$ for some $r \geq 1$. Let $p \in M$ and let $\alpha : 
    U \subset \R^k \to V \subset M$ be a coordinate chart of $M$ 
    about $p$ (i.e. $p \in V$) of class $C^r$. We define the 
    {\bf tangent space of $M$ at $p$} to be 
    \[ T_p(M) := \alpha_*(T_{x_0}(\R^k)) \subset T_p(\R^n), \] 
    where $x_0 \in U$ is the unique point in $U$ such that $\alpha(x_0) = p$.  
\end{defn}\vspace{-0.25cm}\newpage
\section{Curves in $\R^n$}\label{sec:2}

Tangent spaces: Let $M \subset \R^n$ with class $C^r$ of dimension $k$. For a chart 
$\alpha : U \subset \R^k \to V \subset M \subset \R^n$, we have for all $p \in V$ that 
\[ T_p(M) := \alpha_*(T_{x_0}(\R^k)), \] 
where $p = \alpha(x_0)$. We showed that 
\[ T_p(M) = {\rm span}_{\R} \left\{ \frac{\partial\alpha}{\partial x_1}(x_0), \dots, 
\frac{\partial\alpha}{\partial x_k}(x_0) \right\} \subset T_p(\R^n), \] 
which is a $k$-dimensional subspace of $T_p(\R^n)$. 

Examples: 
\begin{enumerate}[(1)]
    \item Let $U \subset \R^n$ be an open set. Then $\alpha : U \subset \R^n \to V = U \subset \R^n$ 
    given by $x \mapsto x$ yields $D\alpha(x) = I_{n\times n}$ for all $x \in U$, so 
    \[ T_x(U) = \alpha_*(T_x(\R^n)) \] 
    by definition. Then for all $(x; v) \in T_x(\R^n)$, we get 
    \[ \alpha_*(x; v) = (\alpha(x); D\alpha(x)v) = (x; v). \] 
    That is, we have $T_x(U) = T_x(\R^n)$. 

    \item We saw two different ways of seeing if something is a submanifold. If $f : U \subset \R^k 
    \to \R^{n-k}$ is a function of a class $C^r$ and 
    \[ M = \{(x, f(x)) \in \R^n : x \in U\} \subset \R^n \] 
    its graph, then $M$ is a $k$-dimensional submanifold of $\R^n$ of class $C^r$. We can parameterize 
    all points in $M$ with the map $\alpha : U \subset \R^k \to M \subset \R^n$ defined by 
    $\alpha(x) = (x, f(x))$. The derivative matrix is 
    \[ D\alpha(x) = \begin{bmatrix} 
        I_{k\times k} \\\hline Df(x)
    \end{bmatrix} \]
    for all $x \in U$. Let $p \in M$ so that $p = (x_0, f(x_0))$ for some $x_0 \in U$. Then 
    \begin{align*} 
        T_p(M) &= \alpha_*(T_{x_0}(\R^k)) \\
        &= \{\alpha_*(x_0; v) : v \in T_{x_0}(\R^k)\} \\
        &= \{(\alpha(x_0); D\alpha(x_0)v) : v \in \R^k\} \\ 
        &= \{(p; w) : w = (v, Df(x_0)v),\, v \in \R^k\}, 
    \end{align*} 
    since $\alpha(x_0) = p$ and $D\alpha(x)$ is the block matrix with $I_{k\times k}$ 
    upstairs and $Df(x)$ downstairs. Also, 
    \[ T_p(M) = {\rm span}_{\R} \left\{ \frac{\partial\alpha}{\partial x_1}(x_0), \dots, 
    \frac{\partial\alpha}{\partial x_k}(x_0) \right\} = {\rm span}_{\R} 
    \left\{ \begin{bmatrix} e_i \\\hline \partial f(x_0)/\partial x_i \end{bmatrix} : 
    i = 1, \dots, k \right\}. \] 

    \item Let $U \subset \R^n$ be open. If $F : U \subset \R^n \to \R^{n-k}$ is a function of 
    class $C^r$ with $DF(p)$ having rank $n-k$ for all $p \in U$, then 
    \[ M = \{x \in U : F(x) = 0\} \] 
    is a $k$-dimensional submanifold of $\R^n$ of class $C^r$. In this case, we leave it as an 
    exercise to show that 
    \[ T_p(M) = \ker(DF(p)). \] 
    In particular, if $k = n-1$, then $F : U \subset \R^n \to \R$ is a scalar function and 
    \[ T_p(M) = \ker(\nabla F(p)). \] 
    Then $\nabla F(p)$ is the normal vector of $T_p(M)$. 

    For example, take the $n$-sphere 
    \[ S^n = \{x \in \R^{n+1} : \|x\|^2 = 1\}, \] 
    which is the zero set of the function $F : \R^{n+1} \to \R$ defined by $F(x) = \|x\|^2 - 1 
    = x_1^2 + \cdots + x_{n+1}^2 - 1$. The derivative matrix is just the gradient; that is, 
    \[ DF(x) = \nabla F(x) = \begin{bmatrix} 2x_1 & \cdots & 2x_{n+1} \end{bmatrix} = 2x. \] 
\end{enumerate}

\subsection{}\label{subsec:2.1}
What is a curve? Intuitively, it is a $1$-dimensional subset of $\R^n$. 

The level sets $f(x, y) = k$ of a two variable function in $\R^2$ are curves. 
For example, for $f(x, y) = x^2 + y^2$, we see that $C \colon x^2 + y^2 = k$ for $k > 0$ 
is a circle centered at $(0, 0)$ of radius $k^{1/2}$. 

The intersection of two surfaces in $\R^3$ is also a curve. For example, 
take $z = x^2 + y^2$ and $z = 2$. Their intersection is the circle 
$C \colon x^2 + y^2 = 2$ in the plane $z = 2$.

For our purposes, we'll work with parametrized curves $\gamma : I = (\alpha, \beta) \subset 
\R \to \R^n$ of class $C^r$. (In practice, we need $r \geq n$ when working over $\R^n$.)

Example: {\bf Circular helix.} Let $\gamma(t) = (a\cos t, a\sin t, bt)$ for $t \in \R$ and $a, b > 0$. 
Note that $x^2 + y^2 = a^2$, so $\gamma(t)$ lies above the circle $x^2 + y^2 = a^2$ in the $xy$-plane. 
We have $\gamma(0) = (a, 0, 0)$ and $\gamma(\pi/2) = (0, a, b\pi/2)$. (The circular helix 
looks like a spiral along the cylinder.)

\begin{defn}{defn:2.1}
    A parameterized curve $\gamma : (\alpha, \beta) \to \R^n$ of class $C^r$ is called 
    {\bf regular} if for all $t \in (\alpha, \beta)$, we have 
    \[ \gamma'(t) = \frac{{\rm d}\gamma}{{\rm d}t}(t) \neq 0. \] 
    We call $\|\gamma'(t)\|$ the {\bf speed of $\gamma$ 
    at $\gamma(t)$} and we say that $\gamma$ is {\bf unit speed} if $\|\gamma'(t)\| = 1$ for all 
    $t \in (\alpha, \beta)$. 
\end{defn}\vspace{-0.25cm}
Note that unit speed implies regular because $\|x\| = 0$ if and only if $x = 0$. 

Example: Let $a > 0$ and take $\gamma(t) = (a\cos t, a\sin t)$ for $t \in \R$, which is a 
parametrization of the circle of radius $a$. Then $\gamma'(t) = (-a\sin t, a\cos t) \neq (0, 0)$ 
for all $t \in \R$ and $\|\gamma'(t)\| = a$, so $\gamma$ is unit speed if and only if $a = 1$.  \newpage
\section{Surfaces in $\R^3$}\label{sec:3}
By a surface, we always mean a smooth $2$-dimensional submanifold of $\R^3$. 
In particular, all the charts we use will be smooth. 

Let $S$ be a surface, and let $\sigma : U \subset \R^2 \to V = \sigma(U) 
\subset S \subset \R^3$ be a smooth coordinate charts with components 
\[ (u, v) \mapsto (x(u, v), y(u, v), z(u, v)). \] 
Then we know that $\sigma$ is a homeomorphism onto its image $V$, 
$\sigma$ is smooth, and $D\sigma$ has rank $2$ on $U$. This implies that 
$\sigma^{-1}$ is smooth as well. If we denote $\sigma_u 
:= \frac{\partial\sigma}{\partial u}$ and $\sigma_v := 
\frac{\partial\sigma}{\partial v}$, then the derivative matrix of $\sigma$ is 
\[ D\sigma = \left[ \begin{array}{c|c}
    \!\!\!\sigma_u & \sigma_v\!\!\!
\end{array} \right]. \] 
Since $D\sigma$ has rank $2$, we see that $\{\sigma_u, \sigma_v\}$ is 
linearly independent on $U$. Equivalently, we have that $\sigma_u \times 
\sigma_v \neq \mathbf 0$ on $U$. Therefore, if $p_0 = \sigma(u_0, v_0)$ 
for some $(u_0, v_0) \in U$, then 
\[ T_{p_0}S = \Span_{\R}\{\sigma_u(u_0, v_0), \sigma_v(u_0, v_0)\}. \] 
Moreover, we have $\sigma_u(u_0, v_0) \times \sigma_v(u_0, v_0) \perp T_{p_0}S$. 

\begin{defn}{defn:3.1}
    For all $p_0 = \sigma(u_0, v_0) \in V = \sigma(U)$, we define 
    the {\bf standard unit normal} to be 
    \[ N_\sigma(u_0, v_0) := \frac{\sigma_u(u_0, v_0) \times \sigma_v(u_0, v_0)}
    {\|\sigma_u(u_0, v_0) \times \sigma_v(u_0, v_0)\|}. \] 
\end{defn}\vspace{-0.25cm}

Since $S \subset \R^3$ and $T_{p_0}S$ is a $2$-dimensional affine subspace of 
$T_{p_0}\R^3 \simeq \R^3$, there are only two possible \emph{unit} normal 
directions to $T_{p_0}S$ at $p_0$. Therefore, $N_\sigma(u_0, v_0)$ is 
uniquely determined by $S$ up to a sign. 

For example, consider the paraboloid $S = \{(x, y, z) \in \R^3 : z = x^2 + y^2\}
\subset \R^3$. This is the graph of the smooth function $f(x, y) = x^2 + y^2$, so 
we have a smooth chart 
\begin{align*}
    \sigma : \R^2 &\to S \subset \R^3 \\ 
    (u, v) &\mapsto (u, v, u^2 + v^2). 
\end{align*}
Then $\sigma_u = (1, 0, 2u)$ and $\sigma_v = (0, 1, 2v)$, which gives 
\[ D\sigma = \begin{bmatrix}
    1 & 0 \\ 
    0 & 1 \\ 
    2u & 2v 
\end{bmatrix} \] 
and $\sigma_u \times \sigma_v = (-2u, -2v, 1) \neq \mathbf 0$. At the point
$p_0 = \sigma(u_0, v_0) = (u_0, v_0, u_0^2+v_0^2)$, we have 
\[ T_{p_0}S = \Span_{\R}\{(1, 0, 2u_0), (0, 1, 2v_0)\}. \] 
For example, with $p_0 = \mathbf 0 = (0, 0, 0) = \sigma(0, 0)$, we have 
\[ T_{\mathbf 0} S = \Span_{\R}\{(1, 0, 0), (0, 1, 0)\}, \] 
which is the $xy$-plane. Moreover, since 
\[ \|\sigma_u \times \sigma_v\| = \sqrt{4u^2 + 4v^2 + 1} \neq 0, \] 
we see that the standard unit normal is 
\[ N_\sigma = \frac{1}{\sqrt{4u^2 + 4v^2 + 1}} (-2u, -2v, 1). \] 
We have that $N_\sigma(0, 0) = (0, 0, 1)$, which is perpendicular 
to $(1, 0, 0)$ and $(0, 1, 0)$ as expected. \newpage 

What happens when we change the coordinate chart? Suppose that 
$\tilde\sigma : \tilde U \to \tilde V$ is another coordinate chart with 
$V \cap \tilde V \neq \varnothing$. Then we can write 
\[ \tilde \sigma = \sigma \circ (\sigma^{-1} \circ \tilde\sigma) = \sigma \circ \Phi, \] 
where $\Phi := \sigma^{-1} \circ \tilde\sigma : \sigma^{-1}(V \cap \tilde V) 
\subset \R^2 \to \sigma^{-1}(V \cap \tilde V) \subset \R^2$ is a smooth
diffeomorphism with components 
\[ (\tilde u, \tilde v) \mapsto (u(\tilde u, \tilde v), v(\tilde u, \tilde v)). \] 
By the chain rule, we obtain $D\tilde\sigma = D\sigma D\Phi$, or more explicitly
\[ \left[ \begin{array}{c|c}
    \!\!\!\tilde\sigma_{\tilde u} & \tilde\sigma_{\tilde v}\!\!\!
\end{array} \right] = \left[ \begin{array}{c|c}
    \!\!\!\sigma_u & \sigma_v\!\!\!
\end{array} \right] \! D\Phi. \] 
Here, $D\Phi$ is the change of basis matrix of $T_pS$. Note that 
\[ D\Phi = \begin{bmatrix}
    \partial u/\partial\tilde u & \partial u/\partial\tilde v \\
    \partial v/\partial\tilde u & \partial v/\partial\tilde v
\end{bmatrix}. \] 
Moreover, we have 
\begin{align*} 
    \tilde\sigma_{\tilde u} &= \sigma_u \cdot \frac{\partial u}{\partial\tilde u} 
    + \sigma_v \cdot \frac{\partial v}{\partial\tilde u}, \\ 
    \tilde\sigma_{\tilde v} &= \sigma_u \cdot \frac{\partial u}{\partial\tilde v} 
    + \sigma_v \cdot \frac{\partial v}{\partial\tilde v},
\end{align*} 
which implies that 
\begin{align*}
    \tilde\sigma_{\tilde u} \times \tilde\sigma_{\tilde v}
    &= \left( \sigma_u \cdot \frac{\partial u}{\partial\tilde u} 
    + \sigma_v \cdot \frac{\partial v}{\partial\tilde u} \right) 
    \times \left( \sigma_u \cdot \frac{\partial u}{\partial\tilde v} 
    + \sigma_v \cdot \frac{\partial v}{\partial\tilde v} \right) \\ 
    &= (\sigma_u \times \sigma_u) \left( \frac{\partial u}{\partial\tilde u}
    \frac{\partial u}{\partial\tilde v} \right) + (\sigma_u \times \sigma_v) 
    \left( \frac{\partial u}{\partial\tilde u} \frac{\partial v}{\partial\tilde v}
    - \frac{\partial v}{\partial\tilde u} \frac{\partial u}{\partial\tilde v} \right) 
    + (\sigma_v \times \sigma_v) \left( \frac{\partial v}{\partial\tilde u}
    \frac{\partial v}{\partial\tilde v} \right) \\ 
    &= \left( \frac{\partial u}{\partial\tilde u} \frac{\partial v}{\partial\tilde v}
    - \frac{\partial v}{\partial\tilde u} \frac{\partial u}{\partial\tilde v} \right) 
    \sigma_u \times \sigma_v = (\det D\Phi) \sigma_u \times \sigma_v,
\end{align*}
where $\sigma_u \times \sigma_u = \sigma_v \times \sigma_v = \mathbf 0$. 
It follows that 
\[ N_{\tilde\sigma} = \frac{\tilde\sigma_{\tilde u} \times \tilde\sigma_{\tilde v}}
{\|\tilde\sigma_{\tilde u} \times \tilde\sigma_{\tilde v}\|}
= \frac{\det D\Phi}{\lvert\det D\Phi\rvert} \cdot \frac{\sigma_u \times \sigma_v}
{\|\sigma_u \times \sigma_v\|} = \pm N_\sigma, \] 
where $\pm$ is given by the sign of $\det D\Phi$. Therefore, changing the 
coordinate chart only changes the standard unit normal up to a sign. 

\subsection{First fundamental form} \label{subsec:3.1}
Let $p_0 \in \sigma(U) \subset S$. If $p_0 = \sigma(u_0, v_0)$, then 
\[ T_{p_0}S = \Span_{\R}\{\sigma_u(u_0, v_0), \sigma_v(u_0, v_0)\} \] 
is a $2$-dimensional subspace of $T_{p_0}\R^3 \simeq \R^3$. Suppose that 
the inner product on $T_{p_0}\R^3$ is the usual dot product. That is, 
for $X, Y \in T_{p_0}\R^3 \simeq \R^3$, we have 
\[ X \cdot Y := X^T Y = X^TIY, \] 
where the latter is the matrix representation with respect to the standard basis.

What if we restrict the dot product to $T_{p_0}S$? Since $T_{p_0}S$ is 
$2$-dimensional, the restriction of the dot product can be represented 
by a $2\times 2$ matrix with respect to a basis of $T_{p_0}S$. 

Using the basis ${\cal B} = \{\sigma_u(u_0, v_0), \sigma_v(u_0, v_0)\}$, 
then for all $X, Y \in T_{p_0}S$, we have 
\begin{align*}
    X &= a\sigma_u(u_0, v_0) + b\sigma_v(u_0, v_0), \\ 
    Y &= c\sigma_u(u_0, v_0) + d\sigma_v(u_0, v_0)
\end{align*}
for some $a, b, c, d \in \R$. More concisely, we have 
\begin{align*}
    X &= D\sigma(u_0, v_0) \begin{bmatrix}
        a \\ b 
    \end{bmatrix}, \\
    Y &= D\sigma(u_0, v_0) \begin{bmatrix}
        c \\ d  
    \end{bmatrix},
\end{align*}
which gives us 
\begin{align*}
    X \cdot Y = X^T Y 
    &= \left( D\sigma(u_0, v_0) \begin{bmatrix}
        a \\ b 
    \end{bmatrix} \right)^{\!T} \left( D\sigma(u_0, v_0) \begin{bmatrix}
        c \\ d  
    \end{bmatrix} \right) \\ 
    &= \begin{bmatrix}
        a & b 
    \end{bmatrix} D\sigma(u_0, v_0)^T D\sigma(u_0, v_0) \begin{bmatrix}
        c \\ d
    \end{bmatrix} \\ 
    &= \begin{bmatrix}
        a & b 
    \end{bmatrix} \begin{bmatrix} 
        \sigma_u \cdot \sigma_u & \sigma_u \cdot \sigma_v \\ 
        \sigma_v \cdot \sigma_u & \sigma_v \cdot \sigma_v 
    \end{bmatrix} \begin{bmatrix}
        c \\ d
    \end{bmatrix}.
\end{align*}
The matrix $D\sigma(u_0, v_0)^T D\sigma(u_0, v_0)$ above is our matrix representation 
of the dot product with respect to the basis ${\cal B} = \{\sigma_u(u_0, v_0), 
\sigma_v(u_0, v_0)\}$. 

\begin{defn}{defn:3.2}
    The {\bf first fundamental form of $\sigma$} is defined to be 
    \[ \FFF = D\sigma(u_0, v_0)^T D\sigma(u_0, v_0) 
    = \begin{bmatrix} 
        \sigma_u \cdot \sigma_u & \sigma_u \cdot \sigma_v \\ 
        \sigma_v \cdot \sigma_u & \sigma_v \cdot \sigma_v 
    \end{bmatrix}. \] 
\end{defn}\vspace{-0.25cm}

Note that $\FFF$ is symmetric since $\sigma_u \cdot \sigma_v = \sigma_v 
\cdot \sigma_u$. Moreover, if $\tilde\sigma = \sigma \circ \Phi$ is another 
chart, then 
\[ D\tilde\sigma = D\sigma D\Phi \] 
where $D\Phi$ is a change of basis matrix, with 
\begin{align*}
    \tilde{\FFF} &= (D\tilde\sigma)^T D\tilde\sigma \\ 
    &= (D\sigma D\Phi)^T (D\sigma D\Phi) \\ 
    &= D\Phi^T (D\sigma^T D\sigma) D\Phi \\ 
    &= D\Phi^T \FFF D\Phi.
\end{align*}
Therefore, $\tilde{\FFF}$ is similar to $\FFF$, and the first fundamental 
form depends on the coordinate chart $\sigma$. 

Let's compute some examples. 
\begin{enumerate}[(1)]
    \item {\bf (Plane.)} Let $\sigma(u, v) = p_0 + uw_1 + vw_2$ for $u, v \in \R$, 
    where $p_0$ is a point and $w_1$ and $w_2$ are direction vectors. We can choose 
    $w_1$ and $w_2$ such that $\{w_1, w_2\}$ is orthonormal. Then 
    $\sigma_u = w_1$ and $\sigma_v = w_2$, so 
    \[ \FFF = \begin{bmatrix}
        w_1 \cdot w_1 & w_1 \cdot w_2 \\ 
        w_2 \cdot w_1 & w_2 \cdot w_2 
    \end{bmatrix} = \begin{bmatrix}
        1 & 0 \\ 0 & 1
    \end{bmatrix}. \] 

    \item {\bf (Cylinder.)} Consider the cylinder $x^2 + y^2 = 1$ with 
    coordinate chart $\tilde\sigma(u, v) = (\cos u, \sin u, v)$ where 
    $(u, v) \in (0, 2\pi) \times \R$. We have $\tilde\sigma_u 
    = (-\sin u, \cos u, 0)$ and $\tilde\sigma_v = (0, 0, 1)$, which yields 
    $\sigma_u \cdot \sigma_u = \sigma_v \cdot \sigma_v = 1$ 
    and $\sigma_u \cdot \sigma_v = 0$. In particular, we have that 
    \[ \tilde{\FFF} = \begin{bmatrix}
        1 & 0 \\ 0 & 1
    \end{bmatrix}. \] 
    This is the same first fundamental form as the plane! Here, 
    we see that the first fundamental form alone cannot distinguish 
    between different shapes. 

    {\bf Remark.} This is not too surprising since the cylinder can be obtained 
    from the plane by folding it smoothly (without stretching or shrinking) 
    so that length and distances are preserved. The first fundamental 
    form is an inner product that is the restriction of the dot product, 
    which is used to measure lengths and distances. 

    \item {\bf (Sphere.)} Consider the sphere $x^2 + y^2 + z^2 = a^2$ 
    for some $a > 0$ and the coordinate chart 
    \[ \sigma(\theta, \varphi) = (a\cos\theta\sin\varphi, a\sin\theta\sin\varphi, a\cos\varphi) \] 
    for $(\theta, \varphi) \in (0, 2\pi) \times (0, \pi)$. We have 
    \begin{align*}
        \sigma_\theta &= (-a\sin\theta\sin\varphi, a\cos\theta\sin\varphi, 0), \\ 
        \sigma_\varphi &= (a\cos\theta\cos\varphi, a\sin\theta\cos\varphi, -a\sin\varphi),
    \end{align*}
    which gives us 
    \begin{align*}
        \sigma_\theta \cdot \sigma_\theta &= a^2\sin^2\varphi, \\ 
        \sigma_\theta \cdot \sigma_\varphi &= 0, \\ 
        \sigma_\varphi \cdot \sigma_\varphi &= a^2.
    \end{align*}
    Therefore, the first fundamental form of $\sigma$ is 
    \[ \FFF = \begin{bmatrix}
        a^2\sin^2\varphi & 0 \\ 0 & a^2
    \end{bmatrix}, \] 
    which is different from the first fundamental form of the plane. 

    {\bf Remark.} The sphere can be obtained from the plane and adding a 
    point at $\infty$. 
\end{enumerate}

\subsection{Second fundamental form} \label{subsec:3.2}
In the above examples, we saw that the first fundamental form $\FFF$ 
is not enough to distinguish surfaces and capture curvature since 
seemingly different surfaces such as planes and cylinders can 
have the same first fundamental form. Therefore, we will introduce a 
``second fundamental form''.

Let $w_1, w_2 \in \R^3$. Recall that the {\bf scalar projection of 
$w_1$ onto $w_2$} is 
\[ \ell = \frac{w_1 \cdot w_2}{\|w_2\|} \] 
because if $\theta$ is the angle between $w_1$ and $w_2$, then 
$\cos\theta = \ell/\|w_1\|$ so that 
\[ \ell = \cos\theta \|w_1\| = \frac{w_1 \cdot w_2}{\|w_1\|\|w_2\|}\|w_1\| 
= \frac{w_1 \cdot w_2}{\|w_2\|}. \] 
In particular, if $\|w_2\| = 1$, then $\ell = w_1 \cdot w_2$. 

Define the vector $\Delta \sigma = \sigma(u_0 + \Delta u, v_0 + \Delta v) 
- \sigma(u_0, v_0) \in \R^3$. Then $\Delta\sigma \cdot N_\sigma$ is 
the scalar projection of $\Delta\sigma$ onto $N_\sigma$ (since 
$N_\sigma$ is unit), and we call it the {\bf deviation}. This 
measures how much $S$ moves away from the tangent plane $T_{p_0}S$ 
to $S$ at $p_0 = \sigma(u_0, v_0)$. 

By Taylor's Theorem, we have 
\begin{align*}
    \Delta\sigma 
    &= \sigma(u_0 + \Delta u, v_0 + \Delta v) \\
    &= \sigma_u \Delta u + \sigma_v \Delta v 
    + \frac12 (\sigma_{uu} \Delta u^2 + 2\sigma_{uv} \Delta u \Delta v 
    + \sigma_{vv} \Delta v^2) + \text{remainder}, 
\end{align*}
where we used the fact that $\sigma_{uv} = \sigma_{vu}$ since $\sigma$ is smooth.
Since $T_pS \perp N_\sigma$ and $T_pS = \Span_{\R}\{\sigma_u, \sigma_v\}$, we have 
$\sigma_u \cdot N_\sigma = \sigma_v \cdot N_\sigma = 0$. This yields 
\begin{align*}
    \Delta\sigma \cdot N_\sigma 
    &= (\sigma_u \cdot N_\sigma)\Delta u + (\sigma_v \cdot N_\sigma) \Delta v 
    + \frac12 [(\sigma_{uu} \cdot N_\sigma) \Delta u^2 + 2(\sigma_{uv} \cdot N_\sigma) 
    \Delta u \Delta v + (\sigma_{vv} \cdot N_\sigma)\Delta v^2] + \cdots
\end{align*}
and so the deviation is well approximated with 
\begin{align*} 
    \Delta\sigma \cdot N_\sigma 
    &\approx \frac12 [(\sigma_{uu} \cdot N_\sigma) 
    \Delta u^2 + 2(\sigma_{uv} \cdot N_\sigma) \Delta u \Delta v + (\sigma_{vv} 
    \cdot N_\sigma)\Delta v^2] \\
    &= \frac12\begin{bmatrix} \Delta u & \Delta v \end{bmatrix} 
    \begin{bmatrix}
        \sigma_{uu} \cdot N_\sigma & \sigma_{uv} \cdot N_\sigma \\ 
        \sigma_{vu} \cdot N_\sigma & \sigma_{vv} \cdot N_\sigma
    \end{bmatrix} \begin{bmatrix}
        \Delta u \\ \Delta v
    \end{bmatrix}.
\end{align*} 
This matrix is reminiscent of the Hessian from multivariable calculus!

\begin{defn}{defn:3.3}
    We define the {\bf second fundamental form of $\sigma$} to be 
    \[ \SFF := \begin{bmatrix}
        \sigma_{uu} \cdot N_\sigma & \sigma_{uv} \cdot N_\sigma \\ 
        \sigma_{vu} \cdot N_\sigma & \sigma_{vv} \cdot N_\sigma
    \end{bmatrix}. \]
\end{defn}\vspace{-0.25cm}

Note that $\sigma$ is smooth, so $\sigma_{uv} = \sigma_{vu}$ and 
$\SFF$ is symmetric. We look at some examples. 

\begin{enumerate}[(1)]
    \item {\bf (Plane.)} As before, let $\sigma(u, v) = p_0 + uw_1 + vw_2$ 
    for $u, v \in \R$ where $p_0$ is a point and $\{w_1, w_2\}$ is orthonormal.
    We have $\sigma_u = w_1$ and $\sigma_v = w_2$. Then 
    $\sigma_{uu} = \sigma_{uv} = \sigma_{vv} = 0$ so that 
    \[ \SFF = \begin{bmatrix}
        0 & 0 \\ 0 & 0
    \end{bmatrix}. \] 

    \item {\bf (Cylinder.)} Let $\tilde\sigma(u, v) = (\cos u, \sin u, v)$ for 
    $u, v \in \R$, and note that $\tilde\sigma_u = (-\sin u, \cos u, 0)$ 
    and $\tilde\sigma_v = (0, 0, 1)$. Then $\tilde\sigma_u \times 
    \tilde\sigma_v = (\cos u, \sin u, 0)$ with $\|\tilde\sigma_u \times 
    \tilde\sigma_v\| = 1$, so we have $N_{\tilde\sigma} = (\cos u, \sin u, 0)$.
    Moreover, we have $\tilde\sigma_{uu} = (-\cos u, -\sin u, 0)$ and 
    $\tilde\sigma_{uv} = \tilde\sigma_{vv} = 0$, which implies that 
    $\sigma_{uu} \cdot N_\sigma = -\cos^2 u - \sin^2 u = -1$ and the second 
    fundamental form of $\tilde\sigma$ is 
    \[ \tilde{\SFF} = \begin{bmatrix}
        -1 & 0 \\ 0 & 0
    \end{bmatrix}. \] 
    We see that this is different from the second fundamental form of the plane.

    \item {\bf (Sphere.)} Consider the sphere $x^2 + y^2 + z^2 = a^2$ where $a > 0$ 
    with coordinate chart 
    \[ \sigma(\theta, \varphi) = (a\cos\theta\sin\varphi, a\sin\theta\sin\varphi, a\cos\varphi) \] 
    for $(\theta, \varphi) \in (0, 2\pi) \times (0, \pi)$. A direct 
    computation shows that 
    \[ \SFF = \begin{bmatrix}
        a\sin^2\varphi & 0 \\ 0 & a
    \end{bmatrix}. \] 

    \item {\bf (Graph of a smooth function.)} Let $f : U \subset \R^2 \to \R$ 
    be smooth where $U \subset \R^2$ is open. The graph 
    \[ S = \{(x, y, z) \in \R^3 : z = f(x, y)\} \subset \R^3 \] 
    of $f$ is a surface with smooth coordinate chart 
    \begin{align*}
        \sigma : U \subset \R^2 &\to S \subset \R^3 \\ 
        (u, v) &\mapsto (u, v, f(u, v)). 
    \end{align*}
    We have $\sigma_u = (1, 0, f_u)$ and $\sigma_v = (0, 1, f_v)$ 
    where $f_u = \frac{\partial f}{\partial u}$ and $f_v = 
    \frac{\partial f}{\partial v}$, which implies that 
    \[ \FFF = \begin{bmatrix}
        1 + (f_u)^2 & f_u f_v \\ 
        f_v f_u & 1 + (f_v)^2
    \end{bmatrix}. \] 
    Next, we have $\sigma_u \times \sigma_v = (-f_u, -f_v, 1)$, so 
    \[ N_\sigma = \frac{1}{\sqrt{1+(f_u)^2 + (f_v)^2}}(-f_u, -f_v, 1). \] 
    Moreover, we know that $\sigma_{uu} = (0, 0, f_{uu})$, 
    $\sigma_{uv} = (0, 0, f_{uv})$, and $\sigma_{vv} = (0, 0, f_{vv})$, so 
    \[ \SFF = \frac{1}{\sqrt{1+(f_u)^2+(f_v)^2}} \begin{bmatrix}
        f_{uu} & f_{uv} \\ f_{vu} & f_{vv}
    \end{bmatrix}. \]
    In particular, the above matrix is exactly the Hessian $H(f)$ of $f$, 
    which appears in the second derivative test. At a critical point of 
    $f(x, y)$, we have $f_u = f_v = 0$ so that $\FFF = I_{2\times 2}$ 
    and $\SFF = H(f)$. 
\end{enumerate}

\subsection{The shape operator} \label{subsec:3.3}
We begin with the following observation. 

\begin{lemma}{lemma:3.4}
    We have $\det\FFF = \|\sigma_u \times \sigma_v\|^2$.
\end{lemma}\vspace{-0.25cm} 
\begin{pf}[Lemma~\ref{lemma:3.4}]
    This follows directly from the cross product identity 
    \[ (a \times b) \cdot (c \times d) = (a \cdot c)(b \cdot d) - 
    (a \cdot d)(b \cdot c) \] 
    for all $a, b, c, d \in \R^3$. We have that 
    \begin{align*}
        \|\sigma_u \cdot \sigma_v\|^2 
        &= (\sigma_u \times \sigma_v) \cdot (\sigma_u \times \sigma_v) \\ 
        &= (\sigma_u \cdot \sigma_u)(\sigma_v \cdot \sigma_v) 
        - (\sigma_u \cdot \sigma_v)(\sigma_v \cdot \sigma_u) \\ 
        &= \det\FFF. \tag*{\qed} 
    \end{align*}
\end{pf}\vspace{-0.25cm}

Since $\sigma_u \times \sigma_v \neq \mathbf 0$ everywhere for any 
coordinate chart $\sigma$, this means that $\FFF$ is always invertible. 

\begin{defn}{defn:3.5}
    The {\bf shape operator} (or {\bf Weingarten matrix}) is defined to be 
    \[ {\cal W} := {\cal F}_{\text{\RN{1}}}^{-1} \SFF. \] 
    The {\bf Gaussian curvature} is $K := \det {\cal W}$, and the {\bf mean curvature} 
    is $H := \frac12 \tr {\cal W}$. 

    Let $p_0 = \sigma(u_0, v_0)$ for some $(u_0, v_0) \in U$. Then: 
    \begin{enumerate}[(i)]
        \item $p_0$ is {\bf elliptic} if $K > 0$. 
        \item $p_0$ is {\bf hyperbolic} if $K < 0$. 
        \item $p_0$ is {\bf parabolic} if $K = 0$ and $H \neq 0$.
        \item $p_0$ is {\bf planar} if $K = H = 0$. 
    \end{enumerate}
\end{defn}\vspace{-0.25cm}

We compute some examples of the shape operator. 
\begin{enumerate}[(1)]
    \item {\bf (Plane.)} Let $\sigma(u, v) = p_0 + uw_1 + vw_2$ where 
    $\{w_1, w_2\}$ is orthonormal and $p_0$ is a point. We found that 
    \begin{align*}
        \FFF &= \begin{bmatrix}
            1 & 0 \\ 0 & 1 
        \end{bmatrix}, \qquad \SFF = \begin{bmatrix}
            0 & 0 \\ 0 & 0 
        \end{bmatrix},
    \end{align*}
    which implies that the shape operator is 
    \[ {\cal W} = {\cal F}_{\text{\RN{1}}}^{-1} \SFF = \begin{bmatrix}
        0 & 0 \\ 0 & 0 
    \end{bmatrix}. \] 
    We have that $K = \det{\cal W} = 0$ and $H = \frac12\tr{\cal W} = 0$ 
    everywhere, so every point on the plane is planar!

    \item {\bf (Cylinder.)} For the cylinder $x^2 + y^2 = 1$ with 
    smooth coordinate chart $\sigma(u, v) = (\cos u, \sin u, v)$ 
    where $(u, v) \in (0, 2\pi) \times \R$, we saw that 
    \begin{align*}
        \FFF &= \begin{bmatrix}
            1 & 0 \\ 0 & 1 
        \end{bmatrix}, \qquad \SFF = \begin{bmatrix}
            -1 & 0 \\ 0 & 0 
        \end{bmatrix}.
    \end{align*}
    Then the shape operator is given by 
    \[ {\cal W} = {\cal F}_{\text{\RN{1}}}^{-1} \SFF = \begin{bmatrix}
        -1 & 0 \\ 0 & 0 
    \end{bmatrix}, \] 
    so $K = \det{\cal W} = 0$ and $H = \frac12\tr{\cal W} = -\frac12 \neq 0$. 
    Thus, every point on the cylinder is parabolic. 

    The missing points on the cylinder are covered by the 
    chart $\sigma(u, v) = (\cos u, \sin u, v)$ over $(u, v) \in (-\pi, \pi)
    \times \R$, which has the same first and second fundamental forms 
    and thus the same shape operator. 

    {\bf Remark.} Cylinders locally look like parabolic cylinders. For example, 
    the set 
    \[ S = \{(x, y, z) \in \R^3 : z = x^2\} \subset \R^3 \] 
    is a parabolic cylinder. Since $S$ is the graph of the smooth 
    scalar function $f : \R^2 \to \R$ defined by $(x, y) \mapsto x^2$, 
    we see that $S$ is a surface which can be covered by the coordinate 
    chart $\sigma(u, v) = (u, v, u^2)$ for $(u, v) \in \R^2$. Then 
    $\sigma_u = (1, 0, 2u)$ and $\sigma_v = (0, 1, 0)$ so that 
    \[ \FFF = \begin{bmatrix}
        1+4u^2 & 0 \\ 0 & 1
    \end{bmatrix}. \]
    Moreover, we have $\sigma_{uu} = (0, 0, 2)$, $\sigma_{uv} = \sigma_{vv} 
    = (0, 0, 0)$ and 
    \[ N_\sigma = \frac{\sigma_u \times \sigma_v}{\|\sigma_u \times \sigma_v\|} 
    = \frac{1}{\sqrt{1+4u^2}}(-2u, 0, 1), \] 
    so the second fundamental form is 
    \[ \SFF = \begin{bmatrix}
        2/\sqrt{1+4u^2} & 0 \\ 0 & 0 
    \end{bmatrix}. \] 
    It follows that the shape operator is 
    \[ {\cal W} = {\cal F}_{\text{\RN{1}}}^{-1} \SFF 
    = \begin{bmatrix}
        1/(1+4u^2) & 0 \\ 0 & 0 
    \end{bmatrix} \begin{bmatrix}
        2/\sqrt{1+4u^2} & 0 \\ 0 & 0 
    \end{bmatrix} = \begin{bmatrix}
        2/(1+4u^2)^{3/2} & 0 \\ 0 & 0 
    \end{bmatrix}. \] 
    We obtain $K = \det{\cal W} = 0$ and 
    $H = \frac12\tr{\cal W} = 1/(1+4u^2)^{3/2} \neq 0$, so every point on $S$ 
    is parabolic.

    \item {\bf (Sphere.)} Consider the sphere $x^2 + y^2 + z^2 = a^2$ where $a > 0$ with 
    smooth coordinate chart 
    \[ \sigma(\theta, \varphi) = (a\cos\theta\sin\varphi, a\sin\theta\sin\varphi, a\cos\varphi) \] 
    for $(\theta, \varphi) \in (0, 2\pi) \times (-\pi, \pi)$. We have seen that 
    \begin{align*}
        \FFF &= \begin{bmatrix}
            a^2\sin^2\varphi & 0 \\ 0 & a^2
        \end{bmatrix}, \qquad \SFF = \begin{bmatrix}
            a\sin^2\varphi & 0 \\ 0 & a 
        \end{bmatrix}.
    \end{align*}
    Then the shape operator is given by 
    \[ {\cal W} = {\cal F}_{\text{\RN{1}}}^{-1} \SFF = \begin{bmatrix}
        1/a & 0 \\ 0 & 1/a
    \end{bmatrix}, \]
    so $K = \det{\cal W} = 1/a^2$ and $H = \frac12\tr{\cal W} = 1/a$ 
    everywhere. Thus, every point on the sphere is elliptic.

    {\bf Remark.} By using different spherical coordinates, we can cover the 
    entire sphere. In each case, we have that $K = 1/a^2$ and $H = \pm1/a$. 

    \item {\bf (Saddle surface.)} Consider the surface $z = x^2 - y^2$ where 
    $x, y \in \R$. This is the graph of the smooth function $f(x, y) = x^2 - y^2$, 
    so we obtain the smooth coordinate chart $\sigma(u, v) = (u, v, u^2 - v^2)$ 
    where $(u, v) \in \R^2$. A direct computation shows that 
    \begin{align*}
        \FFF &= \begin{bmatrix}
            1+4u^2 & -4uv \\ -4uv & 1+4v^2
        \end{bmatrix}, \qquad \SFF = \frac{1}{\sqrt{1+4u^2+4v^2}} \begin{bmatrix}
            2 & 0 \\ 0 & -2
        \end{bmatrix}.
    \end{align*}
    Note that to find the Gaussian curvature $K = \det{\cal W}$, we need not compute ${\cal W}$ 
    explicitly. Recalling that the determinant is multiplicative and 
    $\det(A^{-1}) = (\det A)^{-1}$ for a matrix $A$, we have 
    \[ K = \det{\cal W} = \det({\cal F}_{\text{\RN{1}}}^{-1} \SFF) 
    = \det({\cal F}_{\text{\RN{1}}}^{-1}) \det\SFF = (\det \FFF)^{-1} \det\SFF. \]
    We see that $\det\FFF = (1+4u^2)(1+4v^2) - (-4uv)^2 = 1 + 4u^2 + 4v^2$ and 
    \[ \det\SFF = \frac{1}{(\sqrt{1+4u^2+4v^2})^2} (2(-2) - 0^2) = 
    -\frac{4}{1+4u^2+4v^2}, \] 
    which implies that the Gaussian curvature is 
    \[ K = -\frac{4}{(1+4u^2+4v^2)^2} < 0 \] 
    for all $(u, v) \in \R^2$. Therefore, every point on the saddle surface is hyperbolic.
    
\end{enumerate}

Let's describe a few more properties of the shape operator. First off, 
note that although $\FFF$ and $\SFF$ are always symmetric, it is possible 
that ${\cal W}$ is not symmetric. 

For example, consider the saddle surface $z = xy$ parametrized by 
$\sigma(u, v) = (u, v, uv)$ where $u, v \in \R$. A direct computation then gives 
\begin{align*}
    \FFF &= \begin{bmatrix}
        1+v^2 & uv \\ uv & 1+u^2
    \end{bmatrix}, \qquad \SFF = \frac{1}{(1+u^2+v^2)^{3/2}} \begin{bmatrix}
        0 & 1 \\ 1 & 0 
    \end{bmatrix}
\end{align*}
so that the shape operator is 
\[ {\cal W} = {\cal F}_{\text{\RN{1}}}^{-1} \SFF = \frac{1}{(1+u^2+v^2)^{3/2}} \begin{bmatrix}
    uv & 1+v^2 \\ 1+u^2 & uv 
\end{bmatrix}. \] 
In particular, we have $1+v^2 \neq 1+u^2$ away from points where $u = \pm v$, 
so ${\cal W}$ is not symmetric.

Nonetheless, we have the following result. 

\begin{prop}{prop:3.6}
    The shape operator ${\cal W}$ is a diagonalizable matrix with real eigenvalues.
\end{prop}\vspace{-0.25cm}
\begin{pf}[Proposition~\ref{prop:3.6}]
    Let $w_1$ and $w_2$ be two orthogonal unit vectors in $T_pS$ so that 
    $w_1 \cdot w_1 = w_2 \cdot w_2 = 1$ and $w_1 \cdot w_2 = 0$. Since 
    $w_1, w_2 \in T_pS = \Span_{\R}\{\sigma_u, \sigma_v\}$, we can write 
    \begin{align*}
        w_1 &= a_1\sigma_u + b_1\sigma_v, \\ 
        w_2 &= a_2\sigma_u + b_2\sigma_v 
    \end{align*}
    for some $a_1, a_2, b_1, b_2 \in \R$. For $i, j \in \{1, 2\}$, observe that 
    \[ w_i \cdot w_j = \begin{bmatrix}
        a_i & b_i
    \end{bmatrix} \FFF \begin{bmatrix}
        a_j \\ b_j 
    \end{bmatrix} = \delta_{ij}, \] 
    where $\delta_{ij}$ is the Kronecker delta. Setting 
    \[ C := \left[ \begin{array}{c|c} \!\!\! w_1 & w_2 \!\!\! \end{array} \right] 
    = \begin{bmatrix}
        a_1 & a_2 \\ b_1 & b_2 
    \end{bmatrix}, \] 
    we have that 
    \[ C^T \FFF C = \begin{bmatrix}
        a_1 & b_1 \\ a_2 & b_2 
    \end{bmatrix} \FFF \begin{bmatrix}
        a_1 & a_2 \\ b_1 & b_2 
    \end{bmatrix} = \begin{bmatrix}
        w_1 \cdot w_1 & w_1 \cdot w_2 \\ 
        w_2 \cdot w_1 & w_2 \cdot w_2 
    \end{bmatrix} = \begin{bmatrix}
        1 & 0 \\ 0 & 1 
    \end{bmatrix} = I_{2\times 2}. \] 
    Note that the columns of $C$ are linearly independent, so $C$ is invertible. 
    Therefore, we obtain 
    \[ C^{-1} {\cal F}_{\text{\RN{1}}}^{-1} (C^T)^{-1} = (C^T \FFF C)^{-1} = (I_{2\times 2})^{-1} = 
    I_{2\times 2}, \] 
    which implies that 
    \begin{align*}
        C^{-1}{\cal W}C &= C^{-1}({\cal F}_{\text{\RN{1}}}^{-1} \SFF)C \\
        &= C^{-1}{\cal F}_{\text{\RN{1}}}^{-1} (C^T)^{-1} C^T \SFF C \\ 
        &= C^T \SFF C =: B. 
    \end{align*}
    This means that ${\cal W}$ is similar to $B = C^T \SFF C$. But $B$ 
    is symmetric since 
    \[ B^T = (C^T \SFF C)^T = C^T {\cal F}_{\text{\RN{2}}}^T (C^T)^T = C^T \SFF C = B. \] 
    Then $B$ is diagonalizable with real eigenvalues. Since ${\cal W}$ is 
    similar to $B$, it follows that ${\cal W}$ is also diagonalizable with 
    real eigenvalues. \qed 
\end{pf}\vspace{-0.25cm}

Due to Proposition~\ref{prop:3.6}, it makes sense to define the following. 

\begin{defn}{defn:3.7}
    The eigenvalues $\kappa_1$ and $\kappa_2$ of ${\cal W}$ 
    are called the {\bf principal curvatures} of the coordinate chart $\sigma$. 
\end{defn}\vspace{-0.25cm} 

Observe that we have $K = \det{\cal W} = \kappa_1\kappa_2$ and 
$H = \frac12\tr{\cal W} = \frac12(\kappa_1 + \kappa_2)$. 

Therefore, given the principal curvatures $\kappa_1$ and $\kappa_2$ 
and a point $p_0 = \sigma(u_0, v_0)$, we see that: 
\begin{enumerate}[(i)]
    \item $p_0$ is elliptic when $K > 0$, which happens 
    if and only if $\kappa_1$ and $\kappa_2$ are nonzero with the same sign. 
    \item $p_0$ is hyperbolic when $K < 0$, which happens if and only if 
    $\kappa_1$ and $\kappa_2$ are nonzero with opposite signs. 
    \item $p_0$ is parabolic when $K = 0$ and $H \neq 0$, which 
    happens if and only if $\kappa_1 = 0$ and $\kappa_2 \neq 0$ (or vice versa). 
    \item $p_0$ is planar when $K = H = 0$, which happens if and only if 
    $\kappa_1 = \kappa_2 = 0$. 
\end{enumerate}

{\bf Remark.} We know that $S$ is locally the graph of a function. After 
possibly rotating and translating $S$, we can assume that it is the graph 
of a function $f(x, y)$ near $p_0$ and that $p_0$ corresponds to a critical 
point of $f(x, y)$. We have seen that at $p_0$, we have 
$\FFF = I_{2\times 2}$ and $\SFF = H(f)$ so that ${\cal W} = H(f)$. 
We can now apply the second derivative test. 
\begin{itemize}
    \item If $\det H(f) > 0$ and $f_{xx} > 0$, then $p_0$ is a local minimum. 
    \item If $\det H(f) > 0$ and $f_{xx} < 0$, then $p_0$ is a local maximum. 
    \item If $\det H(f) < 0$, then $p_0$ is a saddle point. 
    \item If $\det H(f) = 0$, then the test is inconclusive. 
\end{itemize}
The first two situations correspond to elliptic points. The third 
corresponds to hyperbolic points, and the fourth corresponds to parabolic or 
planar points.

\subsection{Normal and geodesic curvatures} \label{subsec:3.4}
We now study in more detail the curvature of regular curves 
or surfaces. Since any regular curve can be reparametrized using arclength 
to be unit speed, we will assume throughout that the curves are unit speed. 

Let $\gamma : (\alpha, \beta) \to \sigma(U) = V \subset S$ be a unit speed 
curve on $S$ included in the coordinate patch $\sigma : U \subset \R^2 
\to V \subset S \subset \R^3$. Let $p_0 = \gamma(t_0)$ be a point on the 
curve. Since $\gamma$ is unit speed, we know by Proposition~\ref{prop:2.2} 
that $\gamma'(t_0) \cdot \gamma''(t_0) = 0$. Under the assumption 
$\gamma''(t_0) \neq \mathbf 0$, this means that 
\[ \gamma'(t_0) \perp \gamma''(t_0). \]
Therefore, we will assume that $\gamma''(t_0) \neq \mathbf 0$. 
Since $\gamma'(t_0) \in T_{p_0}S$, we have that 
\[ \gamma'(t_0) \perp N_\sigma \] 
since $N_\sigma \perp T_{p_0}S$, and by the definition of cross product, we obtain 
\[ \gamma'(t_0) \perp N_\sigma \times \gamma'(t_0). \] 
In particular, we see that $\gamma''(t_0) \in \Span_{\R}\{N_\sigma, N_\sigma 
\times \gamma'(t_0)\}$. We can write 
\[ \gamma''(t_0) = \kappa_n N_\sigma + \kappa_g (N_\sigma \times \gamma'(t_0)) \] 
for some $\kappa_n, \kappa_g \in \R$. But $\|\gamma'(t_0)\| = \|N_\sigma\| = 1$,
so $\|N_\sigma \times \gamma'(t_0)\| = 1$. Since 
$N_\sigma \cdot (N_\sigma \times \gamma'(t_0)) = 0$,
we have 
\begin{align*}
    \gamma''(t_0) \cdot N_\sigma 
    &= (\kappa_n N_\sigma + \kappa_g (N_\sigma \times \gamma'(t_0))) \cdot N_\sigma \\ 
    &= \kappa_n N_\sigma \cdot N_\sigma + \kappa_g (N_\sigma \times \gamma'(t_0)) \cdot N_\sigma \\ 
    &= \kappa_n \cdot 1 + \kappa_g \cdot 0 = \kappa_n. 
\end{align*}
By a similar computation, we have that 
\[ \gamma''(t_0) \cdot (N_\sigma \times \gamma''(t_0)) = \kappa_g. \] 
Since $\|N_\sigma\| = \|N_\sigma \times \gamma'(t_0)\| = 1$, recall from 
Section~\ref{subsec:3.2} that $\gamma''(t_0) \cdot N_\sigma$ is the scalar 
projection of $\gamma''(t_0)$ onto $N_\sigma$ and $\gamma''(t_0) \cdot 
(N_\sigma \times \gamma'(t_0))$ is the scalar projection of $\gamma''(t_0)$ 
onto $(N_\sigma \times \gamma'(t_0))$. 

\begin{defn}{defn:3.8}
    The {\bf normal curvature} is defined to be 
    \[ \kappa_n := \gamma''(t_0) \cdot N_\sigma. \] 
    The {\bf geodesic curvature} is defined to be 
    \[ \kappa_g := \gamma''(t_0) \cdot (N_\sigma \times \gamma'(t_0)). \] 
    We call $\kappa_n N_\sigma$ the {\bf normal component of $\gamma''(t_0)$}.
\end{defn}\vspace{-0.25cm}

Note that we can have $\kappa_n = 0$ or $\kappa_g = 0$, but we cannot 
have $\kappa_n = \kappa_g = 0$ since $\gamma''(t_0) \neq \mathbf 0$. 
Moreover, depending on where $\gamma''(t_0)$ lies in the plane 
$\Span_{\R}\{N_\sigma, N_\sigma \times \gamma'(t_0)\}$, we may have 
$\kappa_n < 0$ or $\kappa_g < 0$. 

Since $\gamma''(t_0) \neq \mathbf 0$, we see that 
$\kappa = \|\gamma''(t_0)\| > 0$,
where $\kappa$ is the usual curvature of $\gamma$. Note that 
\begin{align*}
    \kappa^2 = \|\gamma''(t_0)\|^2 
    &= \gamma''(t_0) \cdot \gamma''(t_0) \\
    &= (\kappa_n N_\sigma + \kappa_g(N_\sigma \times \gamma'(t_0))) 
    \cdot (\kappa_n N_\sigma + \kappa_g(N_\sigma \times \gamma'(t_0))) \\
    &= \kappa_n^2 N_\sigma \cdot N_\sigma + 2\kappa_n \kappa_g 
    (N_\sigma \cdot (N_\sigma \times \gamma'(t_0))) + \kappa_g^2 
    (N_\sigma \times \gamma'(t_0)) \cdot (N_\sigma \times \gamma'(t_0)) \\ 
    &= \kappa_n^2 + \kappa_g^2, 
\end{align*}
which implies that $\kappa = \sqrt{\kappa_n^2 + \kappa_g^2}$. 
In particular, if $\kappa_n = 0$, then $\kappa = |\kappa_g|$, and if 
$\kappa_g = 0$, then $\kappa = |\kappa_n|$. 

\begin{defn}{defn:3.9}
    We call $\gamma$ a {\bf geodesic} if $\kappa_g = 0$ everywhere. 
\end{defn}\vspace{-0.25cm}

Now, we show that the normal curvature $\kappa_n$ of a smooth unit 
speed curve $\gamma : (\alpha, \beta) \to V \subset S$ only depends on 
$\gamma'$ and $\SFF$. 

First, note that since the coordinate chart $\sigma : U \subset \R^2 
\to V = \sigma(U) \subset S$ is a diffeomorphism, we know that 
$\sigma^{-1} : V \subset S \to U \subset \R^2$ is also smooth. We set 
\[ \bar\gamma := \sigma^{-1} \circ \gamma : (\alpha, \beta) \to U \subset \R^2. \] 
Then $\bar\gamma$ is a smooth curve in $U \subset \R^2$ with component functions 
$u, v : (\alpha, \beta) \to \R$ such that 
\[ \bar\gamma(t) = (u(t), v(t)) \] 
for all $t \in (\alpha, \beta)$. Therefore, we can write 
\[ \gamma(t) = \sigma \circ \bar\gamma(t) = \sigma(u(t), v(t)). \] 
In other words, any curve $\gamma$ in the coordinate chart $\sigma : 
U \subset \R^2 \to V \subset S$ is of the form $\gamma(t) = \sigma(u(t), v(t))$ 
for smooth functions $u, v : (\alpha, \beta) \to \R$. By the chain rule, we have 
\[ \gamma'(t) = \sigma_u(u(t), v(t)) u'(t) + \sigma_v(u(t), v(t)) v'(t) \] 
with $u'(t), v'(t) \in \R$, so the components of $\gamma'(t)$ are 
$(u'(t), v'(t))$ with respect to the basis $\{\sigma_u, \sigma_v\}$ of 
$T_{\gamma(t)}S$. Furthermore, the product rule gives 
\begin{align*}
    \gamma''(t) 
    &= \frac{{\rm d}}{{\rm d}t} (\sigma_u(u(t), v(t))) u'(t) + \sigma_u u''(t) 
    + \frac{{\rm d}}{{\rm d}t}(\sigma_v(u(t), v(t))) v'(t) + \sigma_v v''(t) \\ 
    &= [\sigma_{uu} u'(t) + \sigma_{uv} v'(t)] u'(t) + \sigma_u u''(t) 
    + [\sigma_{vu} u'(t) + \sigma_{vv} v'(t)] v'(t) + \sigma_v v''(t) \\ 
    &= [\sigma_{uu} (u'(t))^2 + 2\sigma_{uv} u'(t) v'(t) + \sigma_{vv} (v'(t))^2]
    + \sigma_u u''(t) + \sigma_v v''(t). 
\end{align*}

\begin{prop}{prop:3.10}
    If $\gamma(t) = \sigma(u(t), v(t))$, then 
    \[ \kappa_n = \begin{bmatrix}
        u' & v'
    \end{bmatrix} \SFF \begin{bmatrix}
        u' \\ v'
    \end{bmatrix}. \]
\end{prop}\vspace{-0.25cm}
\begin{pf}[Proposition~\ref{prop:3.10}]
    We have by definition and our above computation for $\gamma''(t)$ that 
    \begin{align*}
        \kappa_n &= \gamma''(t) \cdot N_\sigma \\ 
        &= (\sigma_{uu} \cdot N_\sigma)(u'(t))^2 + 2(\sigma_{uv} \cdot N_\sigma)
        u'(t)v'(t) + (\sigma_{vv} \cdot N_\sigma) (v'(t))^2 + 
        (\sigma_u \cdot N_\sigma) u''(t) + (\sigma_v \cdot N_\sigma) v''(t) \\ 
        &= (\sigma_{uu} \cdot N_\sigma)(u'(t))^2 + 2(\sigma_{uv} \cdot N_\sigma)
        u'(t)v'(t) + (\sigma_{vv} \cdot N_\sigma) (v'(t))^2,
    \end{align*}
    where we used the fact that $\sigma_{u} \cdot N_\sigma = \sigma_{v} \cdot N_\sigma 
    = 0$ since $N_\sigma \perp T_{\gamma(t)}S = \Span_{\R}\{\sigma_u, \sigma_v\}$. 
    If we set 
    \begin{align*}
        L := \sigma_{uu} \cdot N_\sigma, \\ 
        M := \sigma_{uv} \cdot N_\sigma, \\ 
        N := \sigma_{vv} \cdot N_\sigma, 
    \end{align*}
    then we see that the second fundamental form is 
    \[ \SFF = \begin{bmatrix}
        L & M \\ M & N
    \end{bmatrix} \] 
    and we obtain 
    \[ \kappa_n = L(u'(t))^2 + 2Mu'(t)v'(t) + N(v'(t))^2 = 
    \begin{bmatrix} u' & v' \end{bmatrix} 
    \begin{bmatrix} L & M \\ M & N \end{bmatrix} 
    \begin{bmatrix} u' \\ v' \end{bmatrix} 
    = \begin{bmatrix} u' & v' \end{bmatrix} \SFF
    \begin{bmatrix} u' \\ v' \end{bmatrix}. \tag*{\qed} \] 
\end{pf}\vspace{-0.25cm}

As a corollary, we get the following classical result. 

\begin{theo}[Meusnier]{theo:3.11}
    Let $\gamma$ be a unit speed curve on $S$. Then the normal component 
    $\kappa_n N_\sigma$ of $\gamma''$ only depends on the unit tangent 
    vector $\gamma'$ and not on $\gamma$. (That is, $\kappa_n$ 
    only depends on $\gamma'$ and not in $\gamma$.)
\end{theo}\vspace{-0.25cm}

Using this, we can prove the following result. 

\begin{prop}{prop:3.12}
    The principal curvatures $\kappa_1$ and $\kappa_2$ (the eigenvalues of 
    ${\cal W} = {\cal F}_{\text{\RN{1}}}^{-1} \SFF$)
    are the maximum and minimum values of $\kappa_n$ for any unit speed curves 
    on $S$ passing through $p_0$. 
\end{prop}\vspace{-0.25cm} 
\begin{pf}[Proposition~\ref{prop:3.12}]
    We prove this using the Lagrange multiplier method. By Meusnier's theorem 
    (Theorem~\ref{theo:3.11}), we know that $\kappa_n$ is completely determined by 
    $\gamma'(t) \in \Span_{\R}\{\sigma_u, \sigma_v\}$. Suppose that 
    $\gamma'(t) = a\sigma_u + b\sigma_v$
    where $a, b \in \R$. Since we are considering unit speed curves, we have 
    $\|\gamma'(t)\| = 1$ and hence 
    \[ 1 = \|\gamma'(t)\|^2 = \gamma'(t) \cdot \gamma'(t) 
    = \begin{bmatrix}
        a & b 
    \end{bmatrix} \FFF \begin{bmatrix}
        a \\ b
    \end{bmatrix}. \] 
    By Proposition~\ref{prop:3.10}, we also know that 
    \[ \kappa_n = \begin{bmatrix}
        a & b 
    \end{bmatrix} \SFF \begin{bmatrix}
        a \\ b 
    \end{bmatrix}. \] 
    For all $a, b \in \R$, define the functions 
    \begin{align*}
        f(a, b) &= \begin{bmatrix}
            a & b 
        \end{bmatrix} \SFF \begin{bmatrix}
            a \\ b 
        \end{bmatrix}, & 
        g(a, b) &= \begin{bmatrix}
            a & b 
        \end{bmatrix} \FFF \begin{bmatrix}
            a \\ b 
        \end{bmatrix}.
    \end{align*}
    Our goal is to find the maximum and minimum values of $f(a, b)$ 
    subject to the constraint that $g(a, b) = 1$. 
    For ease of notation, we set 
    \begin{align*}
        \FFF &= \begin{bmatrix} 
            \sigma_u \cdot \sigma_u & \sigma_u \cdot \sigma_v \\ 
            \sigma_v \cdot \sigma_u & \sigma_v \cdot \sigma_v 
        \end{bmatrix} = \begin{bmatrix}
            E & F \\ F & G 
        \end{bmatrix}, & \SFF &= \begin{bmatrix} 
            \sigma_{uu} \cdot N_\sigma & \sigma_{uv} \cdot N_\sigma \\ 
            \sigma_{vu} \cdot N_\sigma & \sigma_{vv} \cdot N_\sigma 
        \end{bmatrix} = \begin{bmatrix}
            L & M \\ M & N
        \end{bmatrix}.
    \end{align*}
    Then we obtain 
    \begin{align*}
        f(a, b) &= \begin{bmatrix}
            a & b 
        \end{bmatrix} \begin{bmatrix}
            L & M \\ M & N
        \end{bmatrix} \begin{bmatrix}
            a \\ b 
        \end{bmatrix} = a^2 L + 2ab M + b^2 N, \\ 
        g(a, b) &= \begin{bmatrix}
            a & b 
        \end{bmatrix} \begin{bmatrix}
            E & F \\ F & G
        \end{bmatrix} \begin{bmatrix}
            a \\ b 
        \end{bmatrix} = a^2 E + 2ab F + b^2 G.
    \end{align*}
    We see that 
    \begin{align*}
        \nabla f(a, b) &= \left( \frac{\partial f}{\partial a}, 
        \frac{\partial f}{\partial b} \right) = (2aL + 2bM, 2aM + 2bN) 
        = 2\SFF \begin{bmatrix} a \\ b \end{bmatrix}, \\ 
        \nabla g(a, b) &= \left( \frac{\partial g}{\partial a}, 
        \frac{\partial g}{\partial b} \right) = (2aE + 2bF, 2aF + 2bG) 
        = 2\FFF \begin{bmatrix} a \\ b \end{bmatrix}.
    \end{align*}
    By the Lagrange multiplier method, the maximum and minimum of $f(a, b)$ 
    subject to $g(a, b) = 1$ occur at points $(a, b) \in \R^2$ that are 
    solutions of the system 
    \[ \begin{cases}
        \nabla f(a, b) = \lambda \nabla g(a, b) \\ 
        g(a, b) = 1
    \end{cases} \]
    for some $\lambda \in \R$. We see that $\nabla f(a, b) = \lambda \nabla 
    g(a, b)$ if and only if 
    \[ \SFF \begin{bmatrix} a \\ b \end{bmatrix} = \lambda \FFF 
    \begin{bmatrix} a \\ b \end{bmatrix}, \] 
    and since $\FFF$ is invertible, this gives 
    \[ {\cal W} \begin{bmatrix} a \\ b \end{bmatrix} = 
    {\cal F}_{\text{\RN{1}}}^{-1} \SFF \begin{bmatrix} a \\ b \end{bmatrix} 
    = \lambda \begin{bmatrix} a \\ b \end{bmatrix}. \] 
    In particular, we have that $(a, b)^T$ is an eigenvector of ${\cal W} 
    = {\cal F}_{\text{\RN{1}}}^{-1} \SFF$, and $\lambda$ is an 
    eigenvalue of ${\cal W}$ so that $\lambda \in \{\kappa_1, \kappa_2\}$. 
    Using the constraint $g(a, b) = 1$, we find that 
    \begin{align*}
        \kappa_n &= \begin{bmatrix} a & b \end{bmatrix} \SFF 
        \begin{bmatrix} a \\ b \end{bmatrix} 
        = \begin{bmatrix} a & b \end{bmatrix} \FFF 
        \left( {\cal F}_{\text{\RN{1}}}^{-1} \SFF 
        \begin{bmatrix} a \\ b \end{bmatrix} \right) 
        = \begin{bmatrix} a & b \end{bmatrix} \FFF \lambda  
        \begin{bmatrix} a \\ b \end{bmatrix} = \lambda g(a, b) = \lambda. 
    \end{align*}
    Therefore, the maximum and minimum values of $\kappa_n = f(a, b)$ 
    subject to $g(a, b) = 1$ are the eigenvalues of ${\cal W}$, namely 
    the principal curvatures $\kappa_1$ and $\kappa_2$. \qed 
\end{pf}\vspace{-0.25cm}\newpage

As a corollary, we obtain the following. 

\begin{cor}{cor:3.13}
    For $i \in \{1, 2\}$, let $(a_i, b_i)^T$ be an eigenvector of 
    $\kappa_i$ so that ${\cal W}t_i = \kappa_i t_i$. If $\kappa_1 \neq \kappa_2$, 
    then the vectors $X_i = a_i \sigma_u + b_i \sigma_v \in T_{p_0}S$ 
    are perpendicular. 
\end{cor}\vspace{-0.25cm}
\begin{pf}[Corollary~\ref{cor:3.13}]
    We wish to show that $X_1 \cdot X_2 = 0$. Note that 
    \[ X_1 \cdot X_2 = \begin{bmatrix}
        a_1 & b_1 
    \end{bmatrix} \FFF \begin{bmatrix}
        a_2 \\ b_2 
    \end{bmatrix}, \] 
    which implies that 
    \[ \kappa_2(X_1 \cdot X_2) = \kappa_2 \begin{bmatrix}
        a_1 & b_1 
    \end{bmatrix} \FFF \begin{bmatrix}
        a_2 \\ b_2 
    \end{bmatrix} = \begin{bmatrix}
        a_1 & b_1 
    \end{bmatrix} \left( \kappa_2 \FFF \begin{bmatrix}
        a_2 \\ b_2 
    \end{bmatrix} \right). \] 
    But $(a_i, b_i)^T$ is a $\kappa_i$-eigenvector of ${\cal W} = 
    {\cal F}_{\text{\RN{1}}}^{-1} \SFF$, so 
    ${\cal F}_{\text{\RN{1}}}^{-1} \SFF (a_i, b_i)^T 
    = \kappa_i (a_i, b_i)^T$ and hence 
    \[ \SFF \begin{bmatrix}
        a_i \\ b_i 
    \end{bmatrix} = \kappa_i \FFF \begin{bmatrix}
        a_i \\ b_i 
    \end{bmatrix} \] 
    by left multiplying $\FFF$. Setting $t_i = (a_i, b_i)^T$ 
    for $i \in \{1, 2\}$, this gives us
    \begin{align*}
        \kappa_2(X_1 \cdot X_2) 
        &= t_1^T \SFF t_2  
        = (t_1^T \SFF t_2)^T \\
        &= t_2^T \SFF t_1 
        = t_2^T (\kappa_1 \FFF t_1) \\ 
        &= \kappa_1 t_2 \FFF t_1^T 
        = \kappa_1 (X_2 \cdot X_1) 
        = \kappa_1 (X_1 \cdot X_2),
    \end{align*}
    where the second equality follows since $t_1^T \SFF t_2$ is 
    a real number and taking the transpose has no effect. 
    This implies that $(\kappa_1 - \kappa_2)(t_1 \cdot t_2) = 0$, and 
    $\kappa_1 \neq \kappa_2$ implies that $t_1 \cdot t_2 = 0$. \qed 
\end{pf}\vspace{-0.25cm}

As a consequence of Corollary~\ref{cor:3.13}, we can always pick an orthonormal 
basis $\{t_1, t_2\}$ of $T_{p_0}S$ such that the components of $t_i$ are 
$\kappa_i$-eigenvectors. 

Note that $\gamma'(t_0) \in T_{p_0}S$ with $\|\gamma'(t_0)\| = 1$, so we 
can write 
\[ \gamma'(t_0) = \cos(\theta) t_1 + \sin(\theta) t_2 \] 
whee $\theta \in [0, 2\pi)$ is the angle between $\gamma'(t_0)$ and $t_1$. 
Moreover, if we have $\gamma'(t_0) = a\sigma_u + b\sigma_v$, then a similar 
computation as before implies that 
\[ \kappa_n = \begin{bmatrix} a & b \end{bmatrix} \SFF \begin{bmatrix}
    a \\ b 
\end{bmatrix} = \cos^2(\theta) \kappa_1 + \sin^2(\theta) \kappa_2 =: f(\theta). \] 
It follows that the average value of the $\kappa_n$'s is 
\begin{align*} 
    \frac{1}{2\pi} \int_0^{2\pi} f(\theta)\dd\theta 
    &= \frac{1}{2\pi} \int_0^{2\pi} \kappa_1 \cos^2\theta + \kappa_2 \sin^2\theta \dd\theta \\ 
    &= \frac12 (\kappa_1 + \kappa_2) = H
\end{align*}
where the above integral can be solved by using the double angle formulae 
$\cos^2\theta = \frac12(1+\cos2\theta)$ and $\sin^2\theta = \frac12(1-\cos2\theta)$. 
From this, it makes sense why we call $H$ the mean curvature. 

\subsection{Geodesics} \label{subsec:3.5}
We will work with the same setting as before, where $S$ is a surface, 
$\sigma : U \subset \R^2 \to V \subset S \subset \R^3$ is a smooth 
coordinate chart, and $\gamma : (\alpha, \beta) \to V$ is a unit speed 
curve on $S$. Recall from Definition~\ref{defn:3.9} that we called $\gamma$ a 
geodesic if $\kappa_g = 0$ everywhere. An equivalent definition is as follows: 

\begin{defn}{defn:3.14}
    A unit speed curve $\gamma$ is a {\bf geodesic} if $\gamma''(t) = \mathbf 0$ 
    or $\gamma''(t) \parallel N_\sigma$ (i.e. they are parallel) 
    for all $t \in (\alpha, \beta)$.
\end{defn}\vspace{-0.25cm} 

To see that these are indeed equivalent, let's start with this new definition. 
Then $\gamma$ is a geodesic if and only if $\sigma_u \cdot \gamma''(t) = 
\sigma_v \cdot \gamma''(t) = 0$ for all $t \in (\alpha, \beta)$. Since 
$\gamma''(t) \in \Span_{\R}\{N_\sigma, N_\sigma \times \gamma'(t)\}$ 
and 
\[ \gamma''(t) = \kappa_n N_\sigma + \kappa_g (N_\sigma \times \gamma'(t)) \] 
with $\kappa_n = \gamma''(t_0) \cdot N_\sigma$ and 
$\kappa_g = \gamma''(t_0) \cdot (N_\sigma \times \gamma'(t_0))$, 
this shows that our new definition is equivalent to having $\kappa_g = 0$ 
for all $t \in (\alpha, \beta)$.

We look at some examples of geodesics. 
\begin{enumerate}[(1)]
    \item {\bf (Lines.)} Let $\gamma(t) = p_0 + tw$ for $t \in \R$, 
    where $p_0$ is a point and $w$ is a vector with $\|w\| = 1$. 
    Then $\gamma''(t_0) = \mathbf 0$ for all $t \in \R$, so $\gamma$ is a geodesic.
    \item {\bf (Great circles on a sphere.)} Consider the sphere 
    $x^2 + y^2 + z^2 = a^2$ for some $a > 0$. A {\bf great circle} 
    $C$ on $S$ is given by the intersection of $S$ with a plane in $\R^3$ 
    passing through the origin (the center of $S$). 

    Let us check that great circles are geodesics by looking at intersections 
    with vertical planes. (The other cases are proven similarly 
    after first rotating the sphere so that the plane of intersection 
    becomes vertical.) Recall that the sphere can be parametrized by 
    \[ \sigma(\theta, \varphi) = (a\cos\theta\sin(\varphi/a), a\sin\theta
    \sin(\varphi/a), a\cos(\varphi/a)), \] 
    where $(\theta, \varphi) \in (0, 2\pi) \times (0, a\pi)$. We 
    introduced an extra factor of $1/a$ in the 
    angle $\varphi$ to obtain a unit speed curve.
    A direct computation shows that the standard unit normal is 
    \[ N_\sigma(\theta, \varphi) = -(\cos\theta \sin(\varphi/a), \sin\theta\sin(\varphi/a), 
    a\cos(\varphi/a)). \] 
    A vertical plane is obtained by fixing $\theta = \theta_0$. 
    The corresponding great circle is the curve 
    \[ C : \gamma(\varphi) = (a\cos\theta_0\sin(\varphi/a), a\sin\theta_0\sin(\varphi/a),
    a\cos(\varphi/a)), \] 
    which is unit speed with $\gamma''(\varphi) = 
    -\frac1a(\cos\theta_0\sin(\varphi/a), \sin\theta_0\sin(\varphi/a), 
    \cos(\varphi/a)) = \frac1a N_\sigma(\theta_0, \varphi) \neq \mathbf 0$.
    We see that $\gamma''(\varphi) \parallel N_\sigma(\theta_0, \varphi)$ 
    everywhere, so $\gamma$ is a geodesic. 
\end{enumerate}

{\bf Geodesic equations.} As we discussed earlier, the geodesics on $S$ are the solutions of the equations 
\[ \begin{cases}
    \gamma''(t) \cdot \sigma_u(\gamma(t)) = 0 \\ 
    \gamma''(t) \cdot \sigma_v(\gamma(t)) = 0.
\end{cases} \] 
This is a second order ODE, which can be solved. Write 
\[ \FFF = \begin{bmatrix}
    \sigma_u \cdot \sigma_u & \sigma_u \cdot \sigma_v \\ 
    \sigma_v \cdot \sigma_u & \sigma_v \cdot \sigma_v 
\end{bmatrix} = \begin{bmatrix}
    E & F \\ F & G
\end{bmatrix} \] 
and suppose that $\gamma(t) = \sigma(u(t), v(t))$ where $u, v : (\alpha, \beta) 
\to \R$ are scalar functions. (We can always write $\gamma$ in this way; 
see the discussion immediately after Definition~\ref{defn:3.9}.) The above 
equations can be rewritten as 
\begin{align*}
    \gamma''(t) \cdot \sigma_u(\gamma(t)) = 0 
    &\iff \frac{\rm d}{{\rm d}t} (Eu' + Fv') = \frac12 \begin{bmatrix}
        u' & v' 
    \end{bmatrix} \begin{bmatrix}
        E_u & F_u \\ F_u & G_u 
    \end{bmatrix} \begin{bmatrix}
        u' \\ v' 
    \end{bmatrix}, \\ 
    \gamma''(t) \cdot \sigma_v(\gamma(t)) = 0 
    &\iff \frac{\rm d}{{\rm d}t} (Fu' + Gv') = \frac12 \begin{bmatrix}
        u' & v' 
    \end{bmatrix} \begin{bmatrix}
        E_v & F_v \\ F_v & G_v 
    \end{bmatrix} \begin{bmatrix}
        u' \\ v' 
    \end{bmatrix},
\end{align*}
and we call these the {\bf geodesic equations}. \newpage

\begin{pf}[the geodesic equations] 
    Note that $\gamma(t) = \sigma(u(t), v(t))$, which gives 
    $\gamma'(t) = \sigma_u u'(t) + \sigma_v v'(t)$ and 
    \[ \gamma''(t) \cdot \sigma_u = \frac{\rm d}{{\rm d}t} (\gamma'(t)) \cdot \sigma_u 
    = \frac{\rm d}{{\rm d}t} (\gamma'(t) \cdot \sigma_u) - \gamma'(t) \cdot 
    \frac{\rm d}{{\rm d}t} (\sigma_u). \] 
    This means that $\gamma''(t) \cdot \sigma_u = 0$ if and only if 
    \[ \frac{\rm d}{{\rm d}t} (\gamma'(t) \cdot \sigma_u) = \gamma'(t) \cdot 
    \frac{\rm d}{{\rm d}t} (\sigma_u). \] 
    For the left-hand side, we see that 
    \[ \gamma'(t) \cdot \sigma_u = (\sigma_u u'(t) + \sigma_v v'(t))
    \cdot \sigma_u = (\sigma_u \cdot \sigma_u) u'(t) + (\sigma_v \cdot \sigma_u)v'(t) 
    = Eu'(t) + Fv'(t). \] 
    On the other hand, note that $\frac{{\rm d}}{{\rm d}t} [\sigma_u(u(t), v(t))]
    = \sigma_{uu} u'(t) + \sigma_{uv} v'(t)$ and hence 
    \begin{align*}
        \gamma'(t) \cdot \frac{\rm d}{{\rm d}t}(\sigma_u)
        &= (\sigma_u u'(t) + \sigma_v v'(t)) \cdot (\sigma_{uu} u'(t) + \sigma_{uv} v'(t)) \\ 
        &= (\sigma_u \cdot \sigma_{uu}) (u'(t))^2 + (\sigma_u \cdot \sigma_{uv} 
        + \sigma_v \cdot \sigma_{uu}) u'(t) v'(t) + (\sigma_v \cdot \sigma_{uv}) (v'(t))^2.
    \end{align*}
    But we notice that 
    \begin{align*}
        E_u &= (\sigma_u \cdot \sigma_u)_u = 2\sigma_u \cdot \sigma_{uu}, \\ 
        F_u &= (\sigma_u \cdot \sigma_v)_u = \sigma_u \cdot \sigma_{uv} + \sigma_u \cdot \sigma_{vv}, \\ 
        G_u &= (\sigma_v \cdot \sigma_v)_u = 2\sigma_v \cdot \sigma_{uv}, 
    \end{align*}
    which gives us  
    \[ \gamma'(t) \cdot \frac{\rm d}{{\rm d}t} (\sigma_u) = \frac12 E_u (u'(t))^2 
    + F_u u'(t)v'(t) + \frac12 G_u (v'(t))^2 = 
    \frac12 \begin{bmatrix}
        u' & v' 
    \end{bmatrix} \begin{bmatrix}
        E_u & F_u \\ F_u & G_u 
    \end{bmatrix} \begin{bmatrix}
        u' \\ v' 
    \end{bmatrix}. \] 
    This proves the first geodesic equation, and the second is proved similarly. \qed 
\end{pf}\vspace{-0.25cm}

The geodesic equations tell us the following: 
\begin{itemize}
    \item Geodesics \emph{always} exist because they are solutions to 
    nice differential equations. 
    \item Geodesics are completely determined by $\FFF$. That is, if 
    two surfaces have coordinate patches with the same first fundamental 
    form, then they have the ``same'' geodesics.
\end{itemize}

\begin{defn}{defn:3.15}
    A diffeomorphism $\Phi : S_1 \to S_2$ between two surfaces $S_1$ and $S_2$ 
    is called an {\bf isometry} if for any smooth coordinate chart 
    $\sigma : U \subset \R^2 \to V \subset S_1 \subset \R^3$ of $S_1$ 
    and corresponding smooth coordinate chart 
    \[ \tilde\sigma := \Phi \circ \sigma : U \subset \R^2 \to 
    \Phi(V) \subset S_2 \subset \R^3 \] 
    of $S_2$, we have that $\FFF = \tilde\FFF$.  
\end{defn}\vspace{-0.25cm}

Note that translations and rotations are isometries, whereas dilations 
and contractions are not. For another example, let $S_1 = 
\{(x, y, z) \in \R^3 : z = 0,\, 0 < x < 2\pi\}$ be a subset of the $xy$-plane and let $S_2 = 
\{(x, y, z) \in \R^3 : x^2 + y^2 = 1,\, x \neq 1\}$ be the upright cylinder of radius $1$ minus the line $x=1$. 
Consider the diffeomorphism
\begin{align*}
    \Phi : S_1 &\to S_2 \\ 
    (x, y, 0) &\mapsto (\cos x, \sin x, y). 
\end{align*}
Let $\sigma : \R^2 \to S_1$ be defined by $(u, v) \mapsto (u, v, 0)$ and let 
\begin{align*} 
    \tilde\sigma = \Phi \circ \sigma : \R^2 &\to S_2 \\ 
    (u, v) &\mapsto (\cos u, \sin u, v). 
\end{align*} 
Then $\tilde\sigma$ is the usual parametrization of the cylinder, so 
we have that $\FFF = \tilde\FFF = I_{2\times 2}$ and $\Phi$ 
is an isometry.

We can use isometries to find geodesics because the image of a geodesic 
under an isometry is again a geodesic (using the fact that 
the geodesic equations depend only on $\FFF$).

\begin{enumerate}[(1)]
    \item {\bf (Geodesics on a plane.)} For all $(x, y) \in \R^2$, set 
    $\sigma(u, v) = p_0 + uw_1 + vw_2$ where $\{w_1, w_2\}$ is an orthonormal 
    set of direction vectors for the plane. Then $\sigma$ is a 
    coordinate chart for the plane such that 
    \[ \FFF = \begin{bmatrix}
        1 & 0 \\ 0 & 1 
    \end{bmatrix} = \begin{bmatrix}
        E & F \\ F & G
    \end{bmatrix}, \] 
    which implies that 
    \[ \begin{bmatrix}
        E_u & F_u \\ F_u & G_u
    \end{bmatrix} = \begin{bmatrix}
        E_v & F_v \\ F_v & G_v 
    \end{bmatrix} = \begin{bmatrix}
        0 & 0 \\ 0 & 0
    \end{bmatrix}. \] 
    Then the geodesic equations become 
    \begin{align*}
        \frac{{\rm d}}{{\rm d}t} (u') = \frac{{\rm d}}{{\rm d}t} (v') = 0, 
    \end{align*}
    which means that $u'' = v'' = 0$. In particular, we have 
    $u(t) = at+b$ and $v(t) = ct+d$ for some $a, b, c, d \in \R$. Then 
    all geodesics are of the form 
    \[ \gamma(t) = \sigma(at+b, ct+d) \] 
    where $\sigma(u, v) = p_0 + uw_1 + vw_2$, so $\gamma$ is a line! 

    \item {\bf (Geodesics on the sphere.)} Solving the geodesic equations 
    on the sphere $x^2 + y^2 + z^2 = a^2$ shows that the great circles 
    are the only geodesics (see Pressley, Exercise 8.4 on page 144). 

    \item {\bf (Geodesics on the cylinder.)} Consider the cylinder 
    $x^2 + y^2 = 1$ in $\R^3$. As with our previous example, let 
    $S_1 = \{(x, y, z) \in \R^3 : z=0,\, 0 < x < 2\pi\}$ and 
    $S_2 = \{(x, y, z) \in \R^3 : x^2 + y^2 = 1,\, x \neq 1\}$. 
    We saw that the map $\Phi : S_1 \to S_2$ defined by $(x, y, z) \mapsto
    (\cos x, \sin x, y)$ is an isometry. This gives us a one-to-one correspondence 
    between geodesics on $S_1$ and $S_2$. Therefore, the geodesics on the 
    cylinder are the images of lines on $S_1$ under $\Phi$. 
    In particular, they are given by 
    \begin{align*}
        \gamma(t) 
        = \Phi(at+b, ct+d) = (\cos(at+b), \sin(at+b), ct+d). 
    \end{align*}
    This is a line if $a = 0$, a circle if $c = 0$, and a helix 
    if $a$ and $c$ are both nonzero. 
\end{enumerate}

{\bf Geodesics as shortest paths.} Let $S$ be a surface. 
Let $\sigma : U \subset \R^2 \to V \subset S \subset \R^3$ be a smooth 
coordinate patch, and for simplicity, let $\gamma : [0, 1] \to V \subset S$ 
be a unit speed curve on $S$. We can write $\gamma(t) = \sigma(u(t), v(t))$
for some scalar functions $u, v : [0, 1] \to \R$.

Suppose that $p = \gamma(0)$ and $q = \gamma(1)$. We want to find 
necessary conditions for $\gamma$ to be the shortest path on $S$ between $p$ 
and $q$. That is, we want $\gamma$ to have the smallest arclength 
of any family of curves on $S$ starting at $p$ and ending at $q$. 
We will find that if $\gamma$ minimizes distance between $p$ and 
$q$, then $\gamma$ is a geodesic. 

We start by constructing such a family. Let $\delta > 0$. For any $\eps \in (-\delta, \delta)$, we set 
\[ \gamma_\eps(t) := \sigma(u(t) + \eps\alpha(t), v(t) + \eps\beta(t)) \] 
where $\alpha, \beta : [0, 1] \to \R$ are smooth functions such that 
$\alpha(0) = \beta(0) = 0$ and $\alpha(1) = \beta(1) = 0$. Then we have 
that $\gamma_\eps(0) = \sigma(u(0), v(0)) = \gamma(0) = p$ and 
$\gamma_\eps(1) = \sigma(u(1), v(1)) = \gamma(1) = q$ for all 
$\eps \in (-\delta, \delta)$, so each of these curves passes through $p$ and $q$.
Note that $\gamma(t) = \gamma_0(t)$. 

The arclength of $\gamma_\eps(t)$ between $p$ and $q$ is 
\[ L(\eps) := \int_0^1 \|\gamma'_\eps(t)\|\dd t. \] 
If $\gamma = \gamma_0$ minimizes arclength, then it must be a global 
minimum of $L(\eps)$; in particular, it is a critical point of $L(\eps)$, 
which implies that for any choice of 
$\alpha, \beta : [0, 1] \to \R$, we have 
\[ \frac{\rm d}{{\rm d}\eps} (L(\eps)) \Big|_{\eps=0} = 0. \]

\begin{prop}{prop:3.16}
    We have $\frac{\rm d}{{\rm d}\eps} L(\eps)|_{\eps=0} = 0$ if and only 
    if $\gamma$ is a geodesic. 
\end{prop}\vspace{-0.25cm} 
\begin{pf}[Proposition~\ref{prop:3.16}]
    By a direct computation (in the notes), it can be shown that 
    \[ \frac{\rm d}{{\rm d}\eps} (L(\eps)) \Big|_{\eps=0} 
    = -\int_0^1 [\alpha(t) (\sigma_u \cdot \gamma''(t)) + 
    \beta(t) (\sigma_v \cdot \gamma''(t))]\dd t. \] 
    We know that $\gamma$ is a geodesic if and only if 
    $\sigma_u \cdot \gamma''(t) = \sigma_v \cdot \gamma''(t) = 0$, 
    which implies that 
    \[ \frac{\rm d}{{\rm d}\eps} (L(\eps)) \Big|_{\eps=0} = 0. \] 
    Conversely, suppose that $\frac{\rm d}{{\rm d}\eps} L(\eps)|_{\eps=0} = 0$
    but $\gamma$ is not a geodesic. This means that 
    $\sigma_u \cdot \gamma''(t_0) \neq 0$ or $\sigma_v \cdot \gamma''(t_0) \neq 0$ 
    for some $t_0 \in [0, 1]$. Without loss of generality, 
    suppose that $\sigma_u \cdot \gamma''(t_0) > 0$. 
    By continuity, we have that $\sigma_u \cdot \gamma''(t) > 0$ 
    on $[t_0 - c, t_0 + c]$ for some $c > 0$. Set $\beta \equiv 0$ 
    and let $\alpha$ be a bump function on $[0, 1]$ such that 
    $\alpha$ is $1$ on $[t_0 - c/2, t_0 + c/2]$ and $0$ otherwise. This 
    gives us 
    \[ \frac{\rm d}{{\rm d}\eps} L(\eps)|_{\eps=0} = -\int_0^1 
    \alpha(t) (\sigma_u \cdot \gamma''(t))\dd t < 0 \] 
    since $\alpha(t) (\sigma_u \cdot \gamma''(t))$ is nonnegative 
    on $[0, 1]$ and strictly positive on $[t_0 - c/2, t_0 + c/2]$. 
    This gives us a contradiction, so $\gamma$ must be a geodesic. \qed 
\end{pf}\vspace{-0.25cm}

This tells us that any curve that minimizes distance is a geodesic. 
However, it is important to note that the converse is false. Although 
geodesics minimize arclength locally (i.e. around a point), they 
may not do so globally. 
For example, consider the unit sphere $\mathbb{S}^2$. Let $p$ and 
$q$ be two points on $\mathbb{S}^2$ that are not antipodal (i.e. 
on opposite ends of the sphere). The geodesic passing through 
$p$ and $q$ is a great circle, but only one of the circle arcs 
minimizes the distance between $p$ and $q$. 

Another reason why we are interested in geodesics is due to the Gauss-Bonnet 
theorem.
\begin{enumerate}[(1)]
    \item Draw a ``geodesic triangle'' and let $\alpha$, $\beta$, and $\gamma$ 
    be the angles between them. Then we will have 
    \[ \alpha + \beta + \gamma = \pi + \int_T K\dd A. \] 
    We will define what this integral means later. 

    \item If $S$ is a compact ``oriented'' surface, then Gauss-Bonnet 
    tells us that 
    \[ \frac{1}{2\pi} \int K\dd A = 2(g-1). \] 
    This is called the {\bf Euler characteristic}. The
    number $g$ is called the {\bf genus}, which is the number of 
    holes in the surface. 
\end{enumerate}\newpage
\section{Integration on manifolds}\label{sec:4}

\subsection{Integration of scalar functions} \label{subsec:4.1}
Let $M \subset \R^n$ be a smooth $k$-dimensional submanifold and let 
$\alpha : U \subset \R^k \to V \subset M$ be a smooth coordinate chart. 
If $p_0 \in V$, then $p_0 = \alpha(u_0)$ for some $u_0 \in U$ and 
\[ T_{p_0}M = \Span_{\R}\left\{ \frac{\partial\alpha}{\partial u_1}(u_0), 
\dots, \frac{\partial\alpha}{\partial u_k}(u_0) \right\} \subset T_{p_0}\R^n \simeq \R^n. \] 
Moreover, recall that we can write the derivative matrix as 
\[ D\alpha(u_0) = \left[ \begin{array}{c|c|c} 
    \!\!\!\dfrac{\partial\alpha}{\partial u_1}(u_0) & \cdots & 
    \dfrac{\partial\alpha}{\partial u_k}(u_0)\!\!\!
\end{array} \right]. \]
Note that $D\alpha(u_0)^T D\alpha(u_0)$ is a $k \times k$ matrix whose 
$(i, j)$ entry is 
\[ a_{ij} = \frac{\partial\alpha}{\partial u_i}(u_0) \cdot 
\frac{\partial\alpha}{\partial u_j}(u_0). \] 
In other words, we have that 
\[ D\alpha(u_0)^T D\alpha(u_0) = \left[ \frac{\partial\alpha}{\partial u_i}(u_0) \cdot 
\frac{\partial\alpha}{\partial u_j}(u_0) \right]_{1\leq i,j\leq k} \] 
and as with surfaces, we will define this matrix to be the first fundamental 
form of $\alpha$. 

\begin{defn}{defn:4.1}
    The matrix $\FFF = D\alpha^T D\alpha$ is called the {\bf first fundamental 
    form of $\alpha$}.
\end{defn}\vspace{-0.25cm}

Recall from linear algebra that if $A$ is an $n \times k$ (with $n \geq k$) 
of maximal rank $k$, then $A^T A$ is a $k \times k$ matrix of maximal rank $k$. 
Since $D\alpha$ has maximal rank $k$ everywhere on $U$, it follows that 
$\FFF$ has maximal rank $k$ everywhere on $U$. In other words, $\FFF$ 
is an invertible matrix, so $\det \FFF(u_0) \neq 0$ for all $u_0 \in U$. 
Moreover, for any two tangent vectors $X, Y \in T_{p_0}M$, we can write 
\begin{align*}
    X &= a_1 \frac{\partial\alpha}{\partial u_1}(u_0) + \cdots 
    + a_k \frac{\partial\alpha}{\partial u_k}(u_0), \\ 
    Y &= b_1 \frac{\partial\alpha}{\partial u_1}(u_0) + \cdots 
    + b_k \frac{\partial\alpha}{\partial u_k}(u_0),
\end{align*}
for some $a_1, \dots, a_k, b_1, \dots, b_k \in \R$. Observe that 
\begin{align*}
    X \cdot Y &= \left( \frac{\partial\alpha}{\partial u_1}(u_0) + \cdots 
    + a_k \frac{\partial\alpha}{\partial u_k}(u_0) \right) 
    \cdot \left( b_1 \frac{\partial\alpha}{\partial u_1}(u_0) + \cdots 
    + b_k \frac{\partial\alpha}{\partial u_k}(u_0) \right) \\
    &= \sum_{i=1}^k \sum_{j=1}^k a_i b_j \left( \frac{\partial\alpha}{\partial u_i}(u_0) 
    \cdot \frac{\partial\alpha}{\partial u_j}(u_0) \right) \\ 
    &= \begin{bmatrix}
        a_1 & \cdots & a_k 
    \end{bmatrix} D\alpha(u_0)^T D\alpha(u_0) \begin{bmatrix}
        b_1 \\ \vdots \\ b_k 
    \end{bmatrix} 
    = \begin{bmatrix}
        a_1 & \cdots & a_k 
    \end{bmatrix} \FFF \begin{bmatrix}
        b_1 \\ \vdots \\ b_k 
    \end{bmatrix}. 
\end{align*}
Since $(a_1, \dots, a_k)$ and $(b_1, \dots, b_k)$ are the components of 
$X$ and $Y$ with respect to the basis 
\[ {\cal B} = \left\{ \frac{\partial\alpha}{\partial u_1}(u_0), 
\dots, \frac{\partial\alpha}{\partial u_k}(u_0) \right\} \] 
of $T_{p_0}M$, this means that $\FFF$ is the matrix representation 
of the dot product $\cdot$ in $T_{p_0}\R^n \simeq \R^n$ to $T_{p_0}M$ 
with respect to the basis ${\cal B}$. Moreover, for any 
$a = (a_1, \dots, a_k) \in \R^k$, we have 
\[ a^T \FFF a = \begin{bmatrix}
    a_1 & \cdots & a_k 
\end{bmatrix} \FFF \begin{bmatrix}
    a_1 \\ \vdots \\ a_k 
\end{bmatrix} = X \cdot X \geq 0 \] 
where $X = a_1 \frac{\partial\alpha}{\partial u_1}(u_0) + \cdots 
+ a_k \frac{\partial\alpha}{\partial u_k}(u_0)$ and $a^T \FFF a = X \cdot X = 0$ 
if and only if $X = \mathbf 0$ (i.e. $a = \mathbf 0$) by positive 
definiteness of the dot product. Consequently, $\FFF$ is positive 
definite and $\det\FFF > 0$. In particular, this means that 
$\sqrt{\det\FFF}$ is well-defined. To interpret $\sqrt{\det\FFF}$, we 
start by looking at two familiar cases. 
\begin{enumerate}[(1)]
    \item If $k = 1$, then $M$ is a regular curve in $\R^n$. Suppose that 
    it is parametrized by $\gamma : (a, b) \to \R^n$ (so that $\gamma$ 
    plays the role of $\alpha$ in the above formulae). Then we have 
    $D\gamma(t) = \gamma'(t)$ and 
    \[ \FFF = D\gamma(t)^T D\gamma(t) = \gamma'(t)^T \gamma'(t) 
    = \gamma'(t) \cdot \gamma'(t) = \|\gamma'(t)\|^2. \] 
    In this case, $\FFF$ is a $1 \times 1$ matrix so that 
    $\det\FFF = \FFF$ and 
    \[ \sqrt{\det\FFF} = \|\gamma'(t)\|. \] 
    
    \item If $k = 2$ and $n = 3$, then $M$ is a smooth surface $S$ in $\R^3$. 
    Let $\sigma : U \subset \R^2 \to V \subset S \subset \R^3$ be a smooth 
    coordinate chart on $S$ (so that $\sigma$ plays the role of $\alpha$ 
    in the above formulae). Then 
    \[ \FFF = D\sigma^T D\sigma \] 
    is the usual first fundamental form. We know from Lemma~\ref{lemma:3.4} that 
    \[ \sqrt{\det\FFF} = \|\sigma_u \times \sigma_v\|, \] 
    which is the area of the parallelogram spanned by $\sigma_u$ and $\sigma_v$. 
\end{enumerate}

More generally, one can show that if $X_1, \dots, X_k$ are $k$ linearly 
independent vectors in $\R^n$ and $B$ is the $k \times k$ matrix whose 
$(i, j)$-entry is $b_{ij} = X_i \cdot X_j$, then 
$\sqrt{\det B}$ is the volume of the parallelepiped spanned by $X_1, \dots, X_k$. 
Therefore, in our situation, $\sqrt{\det\FFF}$ is the volume of the 
parallelepiped spanned by the elements 
\[ {\cal B} = \left\{ \frac{\partial\alpha}{\partial u_1}(u_0), 
\dots, \frac{\partial\alpha}{\partial u_k}(u_0) \right\}. \] 
Given this, we can now define the integral of a scalar function of $V \subset M$. 

\begin{defn}{defn:4.2}
    Suppose that $V \subset M$ is bounded and let $f : V \to \R$ is a continuous 
    scalar function. We define the {\bf integral of $f$ on $V$} to be 
    \[ \int_V f\dVol := \int_U f(\alpha(u))\sqrt{\det\FFF}\dd u_1 \cdots \dd u_k. \] 
\end{defn}\vspace{-0.25cm}

Here, we can think of $\sqrt{\det\FFF}$ as some sort of correcting term. 

Note that the definition of the integral of $f$ on $V$ is independent of the 
choice of coordinate chart $\alpha$. Indeed, if $\tilde\alpha : 
\tilde U \subset \R^k \to V \subset M$ is another coordinate chart 
parametrizing $V$, then $\tilde\alpha = \alpha \circ \Phi$ where 
\[ \Phi := \alpha^{-1} \circ \tilde\alpha : \tilde U \subset \R^k 
\to U \subset \R^k \] 
is a diffeomorphism. By the chain rule, we have $D\tilde\alpha 
= D\alpha D\Phi$ so that 
\[ \tilde\FFF = D\tilde\alpha^T D\tilde\alpha = (D\alpha D\Phi)^T 
(D\alpha D\Phi) = D\Phi^T D\alpha^T D\alpha D\Phi = D\Phi^T \FFF D\Phi \] 
and hence $\det\tilde\FFF = \det(D\Phi)^2 \det\FFF$ since determinant is 
multiplicative and $\det D\Phi = \det D\Phi^T$.
This gives
\[ \sqrt{\det\tilde\FFF} = \lvert\det D\Phi\rvert \sqrt{\det \FFF}, \] 
which implies that 
\begin{align*}
    \int_{\tilde U} f(\tilde\alpha(\tilde u)) \sqrt{\det\tilde\FFF} \dd\tilde u_1 \cdots \dd\tilde u_k  
    &= \int_{\tilde U} f(\tilde\alpha(\tilde u)) \sqrt{\det\FFF} \,\lvert\det D\Phi\rvert \dd\tilde u \cdots \dd\tilde u_k \\ 
    &= \int_U f(\alpha(u)) \sqrt{\det\FFF} \dd u_1 \cdots \dd u_k 
\end{align*} 
since $\lvert\det D\Phi\rvert \dd\tilde u_1 \cdots \dd\tilde u_k 
= \dd u_1 \cdots \dd u_k$ by the change of variable formula. 

We now see what the integral formula looks like in three special cases. 
\begin{enumerate}[(1)]
    \item If $k = 1$ so that $M$ is a regular curve $\gamma : (a, b) \to \R^n$, 
    we denote the integral by $\int_\gamma f\dd s$ and 
    \[ \int_\gamma f\dd s = \int_a^b f(\gamma(t))\|\gamma'(t)\|\dd t. \] 
    In particular, if $f \equiv 1$, then we obtain 
    \[ \int_\gamma \dd s = \int_a^b \|\gamma'(t)\| \dd t, \] 
    which is the arclength of $\gamma$ between $\gamma(a)$ and $\gamma(b)$.

    \item If $k = 2$ and $n = 3$ so that $M$ is a smooth surface $S$ in $\R^3$ 
    and $\sigma : U \subset \R^2 \to V \subset S \subset \R^3$ is a smooth 
    coordinate chart on $S$, then we denote the integral by $\int_V f\dd A$ 
    and 
    \[ \int_V f\dd A = \int_U f(\sigma(u, v)) \|\sigma_u \times \sigma_v\|\dd u \dd v. \] 
    If $f \equiv 1$, then we obtain 
    \[ \int_V \dd A = \int_U \|\sigma_u \times \sigma_v\|\dd u \dd v, \] 
    which is the surface area of $V \subset S$. 
    
    \item If $M$ is a bounded open subset of $\R^n$, then we can pick 
    $\alpha = \Id_M$ as the smooth coordinate chart. We have 
    $D\alpha = I_{n\times n}$ so that $\FFF = I_{n\times n}$ and 
    $\det\FFF = \sqrt{\det\FFF} = 1$. In this case, we simply have 
    \[ \int_M f\dVol = \int_M f(x_1, \dots, x_n) \dd x_1 \cdots \dd x_n, \] 
    which is the usual integral of $f$ on the region $M \subset \R^n$. 
    Moreover, if $f \equiv 1$, then 
    \[ \int_M \dVol = \int_M \dd x_1 \cdots \dd x_n \] 
    is the volume of $M$. 
    
\end{enumerate}

We illustrate this with a few examples. 
\begin{enumerate}[(1)]
    \item {\bf (Sphere.)} Let $S$ be the sphere $x^2 + y^2 + z^2 = a^2$ where $a > 0$, 
    which we can parametrize using
    \[ \sigma(\theta, \varphi) = (a\cos\theta\sin\varphi, a\sin\theta\sin\varphi, a\cos\varphi) \] 
    where $(\theta, \varphi) \in (0, 2\pi) \times (0, \pi)$. We have
    $\sqrt{\det\FFF} = \|\sigma_\theta \times \sigma_\varphi\| = a^2\sin\varphi > 0$. 
    By taking $f \equiv 1$, we see that the surface area of a sphere of radius $a$ is 
    \begin{align*}
        \int_S \dd A &= \int_U \|\sigma_\theta \times \sigma_\varphi\|\dd\theta \dd\varphi \\
        &= \int_0^\pi \int_0^{2\pi} a^2\sin\varphi \dd\theta \dd\varphi \\ 
        &= 2\pi a^2 \int_0^\pi \sin\varphi\dd\varphi \\ 
        &= 2\pi a^2(-\cos(\pi) + \cos(0)) = 4\pi a^2. 
    \end{align*}
    We have also seen that $K = 1/a^2$ (see example (3) on page 40) so that 
    \[ \int_S K\dd A = \int_S \frac{1}{a^2} \dd A = \frac{1}{a^2} \int_S \dd A 
    = \frac{1}{a^2} (4\pi a^2) = 4\pi = 4\pi(1-g), \] 
    where $g = 0$ since the sphere has no holes. Therefore, Gauss-Bonnet 
    (which we stated at the end of Section~\ref{subsec:3.5}, page 50) holds 
    for spheres!

    \item {\bf (Helix.)} Let $\gamma(t) = (a\cos t, a\sin t, bt)$ where 
    $t \in [0, \pi/3]$ be a part of the helix in $\R^3$, with $a, b > 0$. 
    Consider the scalar function $f(x, y, z) = x + y$ on $\R^3$. Since 
    we are only taking the parameter $t$ in the bounded interval $[0, \pi/3]$, 
    the piece of the helix is itself bounded. Moreover, $f$ is continuous 
    everywhere, so we can integrate $f$ over $\gamma$. Since $\gamma$ is a 
    regular curve, we know that 
    \[ \int_\gamma f\dd s = \int_0^{\pi/3} f(\gamma(t))\|\gamma'(t)\|\dd t. \] 
    In this case, we have $\gamma'(t) = (-a\sin t, a\cos t, b)$ so that 
    $\|\gamma'(t)\| = \sqrt{a^2 + b^2}$ and 
    \begin{align*}
        \int_\gamma f\dd s &= \int_0^{\pi/3} f(\gamma(t)) \|\gamma'(t)\|\dd t \\ 
        &= \int_0^{\pi/3} (a\cos t + a\sin t) \sqrt{a^2 + b^2}\dd t \\ 
        &= a\sqrt{a^2 + b^2} \int_0^{\pi/3} (\cos t + \sin t)\dd t \\ 
        &= a\sqrt{a^2 + b^2} \left[\left(\sin\left(\frac{\pi}{3}\right) - 
        \cos\left(\frac{\pi}{3}\right)\right) - (\sin 0 - \cos 0)\right] \\
        &= \frac{a\sqrt{a^2 + b^2}}{2} (1 + \sqrt{3}). 
    \end{align*}

    \item {\bf (Cylinder.)} Let $C$ be the part of the cylinder in $\R^3$ where 
    $x^2 + y^2 = 1$ and $2 \leq z \leq 4$. Since we are fixing the values 
    of $z$ to be between $2$ and $4$, we see that $C$ is bounded.  
    It can be parametrized by 
    \[ \sigma(u, v) = (\cos u, \sin u, v) \] 
    where $(u, v) \in U = (0, 2\pi) \times (2, 4)$. We have previously seen that 
    $\FFF = I_{2\times 2}$ for this coordinate chart $\sigma$ 
    (see example (2) on page 35) so that $\sqrt{\det\FFF} = 1$. Since 
    $C$ is a smooth surface, we can compute its surface area with 
    \[ \int_C \dd A = \int_U \sqrt{\det\FFF} \dd u \dd v = 
    \int_U \dd u \dd v = \int_2^4 \int_0^{2\pi} \dd u \dd v = 
    \int_2^4 2\pi \dd v = 4\pi. \] 

    Let us also integrate $f(x, y, z) = x^2 y + z$ over $C$. We have 
    \begin{align*}
        \int_C f\dd A &= \int_U f(\sigma(u))\sqrt{\det \FFF}\dd u \dd v \\ 
        &= \int_2^4 \int_0^{2\pi} ((\cos u)^2 \sin u + v) \dd u \dd v \\ 
        &= \int_2^4 \left[ -\frac13(\cos u)^3 + uv \right]_0^{2\pi} \dd v \\ 
        &= \int_2^4 2\pi v\dd v = \pi(4^2 - 2^2) = 12\pi. 
    \end{align*}

    \item {\bf (Torus.)} We now consider the $2$-dimensional torus 
    $\mathbb{T}^2$ in $\R^3$ described in Problem 3 of Assignment 4. It can be 
    parametrized with 
    \[ \sigma(\theta, \alpha) = ((b + a\cos\alpha)\cos\theta, 
    (b + a\cos\alpha)\sin\theta, a\sin\alpha) \] 
    where $(\theta, \alpha) \in U = (0, 2\pi) \times (0, 2\pi)$. A direct 
    computation gives $\|\sigma_\theta \times \sigma_\alpha\| = a(b + a\cos\theta)$ 
    and 
    \[ K = \frac{\cos\alpha}{a(b + a\cos\alpha)}. \] 
    Then the surface area of $\mathbb T^2$ is 
    \begin{align*} 
        \int_{\mathbb T^2} \dd A 
        &= \int_0^{2\pi} \int_0^{2\pi} a(b + a\cos\alpha)\dd\theta \dd\alpha \\ 
        &= 2\pi a \int_0^{2\pi} (b + a\cos\alpha)\dd \alpha \\ 
        &= 2\pi a \Big[ b\alpha + a\sin\alpha \Big]_0^{2\pi} = 4\pi^2 ab. 
    \end{align*}
    Moreover, if we integrate over the Gaussian curvature $K$ on $\mathbb T^2$, we obtain 
    \begin{align*}
        \int_{\mathbb T^2} K\dd A 
        &= \int_0^{2\pi} \int_0^{2\pi} \frac{\cos\alpha}{a(b+a\cos\alpha)} 
        a(b+a\cos\alpha) \dd\theta \dd \alpha \\ 
        &= 2\pi \int_0^{2\pi} \cos\alpha\dd \alpha 
        = 2\pi(\sin(2\pi) - \sin 0) = 0 = 4\pi(1-g), 
    \end{align*}
    where $g = 1$ is the genus of the torus since it has a single hole. 
    Therefore, Gauss-Bonnet also holds here!

\end{enumerate}

\subsection{Integrating vector fields over submanifolds} \label{subsec:4.2}
Let $M \subset \R^n$ be a smooth $k$-dimensional submanifold. By a 
vector field on $M$, we mean the following: 

\begin{defn}{defn:4.3}
    A {\bf vector field on $M$} is a vector-valued function $F : M \to \R^n$. 
\end{defn}\vspace{-0.25cm}

A vector field is a function that attaches to every point on $M$ 
a vector in the ambient space $\R^n$. Therefore, it is important 
to note in the definition that if $M$ is a submanifold of $\R^n$, then the 
vector field $F$ also takes values in $\R^n$. 

Let's consider some examples of vector fields. 
\begin{enumerate}[(1)]
    \item The function $F(x, y, z, w) = (x^2y - z, y\cos w, 4e^x + 1, z - 3)$ 
    is a vector field on $\R^4$. 
    \item If $\gamma : (a, b) \to \R^n$ is a parametrized curve in $\R^n$, 
    then the velocity $\gamma'(t)$ and the acceleration $\gamma''(t)$ are 
    vector fields on $\gamma$. 
    \item If a vector field $F : M \to \R^n$ is such that $F(p) \in T_pM$ 
    for all $p \in M$, then $F$ is called a {\bf tangent vector field}. 
    For example, the velocity $\gamma'(t)$ is a tangent vector field 
    on the parametrized curve $\gamma : (a, b) \to \R^n$.
    \item If a vector field $F : M \to \R^n$ is such that $F(p) \perp T_pM$ for 
    all $p \in M$, then $F$ is called a {\bf normal vector field}. 
    For example, the principal unit normal $N(t)$ and binormal $B(t)$ 
    of a regular curve $\gamma : (a, b) \to \R^n$ are normal vector fields 
    (where they are defined, namely where the curvature is strictly positive).
    \item If $\sigma : U \subset \R^2 \to V \subset S \subset \R^3$ is a smooth 
    coordinate chart of a surface $S$ in $\R^3$, then $N_\sigma$ is a unit 
    normal vector field on $V$. 
    \item If $M$ is the zero set of a smooth scalar function $G : \R^n \to \R$ 
    with $DG = \nabla G$ of maximal rank everywhere, then we know that $M$ is a smooth 
    $(n-1)$-dimensional submanifold of $\R^n$. The gradient $\nabla G$ 
    is a normal vector field on $M$ since we have 
    \[ T_pM \simeq \{v \in \R^n : \nabla G(p) \cdot v = 0\} \] 
    for all $p \in M$ (see the end of Section~\ref{subsec:1.5}, page 21 
    for the discussion of this fact).
    \begin{enumerate}[(a)]
        \item Recall that the unit $n$-sphere $\mathbb S^n : x_1^2 + 
        \cdots + x_{n+1}^2 = 1$ in $\R^{n+1}$ is the zero set of the function 
        $G(x_1, \dots, x_{n+1}) = x_1^2 + \cdots + x_{n+1}^2 - 1$. Its 
        gradient is $\nabla G(x_1, \dots, x_{n+1}) = (2x_1, \dots, 2x_{n+1})$, 
        which is nonzero everywhere on $\mathbb S^n$. Consequently, the gradient 
        $\nabla G$ is a nowhere vanishing normal vector field on $\mathbb S^n$. 
        In particular, this means we can normalize it to see that 
        \[ N(x_1, \dots, x_{n+1}) = \frac{\nabla G(x_1, \dots, x_{n+1})}
        {\|\nabla G(x_1, \dots, x_{n+1})\|} = \frac{1}{\sqrt{x_1^2 + \cdots 
        + x_{n+1}^2}}(x_1, \dots, x_{n+1}) \] 
        is a unit normal vector field on $\mathbb S^n$. 

        \item The paraboloid $S : z = x^2 + y^2$ in $\R^3$ is the zero 
        set of the smooth scalar function $G(x, y, z) = x^2 + y^2 - z$ 
        whose gradient $\nabla G(x, y, z) = (2x, 2y, -1)$. After normalizing, 
        we see that 
        \[ N(x, y, z) = \frac{\nabla G(x, y, z)}{\|\nabla G(x, y, z)\|} 
        = \frac{1}{4x^2 + 4y^2 + 1} (2x, 2y, -1) \] 
        is a unit normal vector field on $S$. 
    \end{enumerate}
\end{enumerate}

\subsubsection{Line integrals of vector fields} \label{subsubsec:4.2.1}
We start by considering integrals of vector fields on curves. Let 
$\gamma : [a, b] \to \R^n$ be a smooth regular curve in $\R^n$. Its 
velocity is a smooth nowhere vanishing tangent vector field on $\gamma$.
If $F$ is a vector field defined on an open subset $U$ of $\R^n$ containing 
$\gamma$, then the restriction of $F$ to $\gamma$ is $F(\gamma(t))$ for 
$t \in (a, b)$. Moreover, the dot product of $F$ with $\gamma'$ at points on 
$\gamma$ is the scalar function 
\[ f(t) = F(\gamma(t)) \cdot \gamma'(t) \] 
with $t \in (a, b)$. If $F$ is smooth, then the scalar function $f$ is 
also smooth and therefore integrable on $(a, b)$. We then define 
the integral of $F$ on $\gamma$ to be $\int_a^b f(t)\dd t$. More precisely, 
we have the following definition: 

\begin{defn}{defn:4.4}
    Let $\gamma : [a, b] \to \R^n$ be a smooth regular curve in $\R^n$, 
    and let $F : U \to \R^n$ be a smooth vector field defined on an 
    open set $U \subset \R^n$ containing $\gamma$. We define 
    the {\bf line integral of $F$ on $\gamma$} to be 
    \[ \int_\gamma F\cdot {\rm d}r := \int_a^b F(\gamma(t)) \cdot \gamma'(t) \dd t. \] 
\end{defn}\vspace{-0.25cm}

{\bf Example.} Let $\gamma(t) = (t, t^2, t^3)$ where $0 \leq t \leq 1$ 
be (part of) the twisted cubic in $\R^3$ and let $F(x, y, z) = (8x^2yz, 5z, -4xy)$ 
be a smooth vector field on $\R^3$. Then we have 
\[ F(\gamma(t)) = (8(t)^2(t^2)(t^3), 5t^3, -4(t)(t^2)) = (8t^7, 5t^3, -4t^3) \] 
and $\gamma'(t) = (1, 2t, 3t^2)$, which gives 
\[ F(\gamma(t)) \cdot \gamma'(t) = (8t^7, 5t^3, -4t^3) \cdot 
(1, 2t, 3t^2) = 8t^7 - 12t^5 + 10t^4. \] 
Therefore, the line integral of $F$ on $\gamma$ is 
\[ \int_\gamma F\cdot{\rm d}r = \int_0^1 F(\gamma(t)) \cdot \gamma'(t)\dd t 
= \int_0^1 (8t^7 - 12t^5 + 10t^4)\dd t = \Big[ t^8 - 2t^6 + 2t^5 \Big]_0^1 
= 1. \] 

{\bf Remark.} Since $\gamma$ is regular, it has a well-defined unit tangent 
vector field $T(t) = \gamma'(t)/\|\gamma'(t)\|$. Then 
\[ F(\gamma(t)) \cdot \gamma'(t) = (F(\gamma(t)) \cdot T(t))\,\|\gamma'(t)\| \] 
so that we have 
\[ \int_\gamma F\cdot{\rm d}r := \int_a^b F(\gamma(t)) \cdot \gamma'(t)\dd t 
= \int_a^b (F(\gamma(t)) \cdot T(t))\,\|\gamma'(t)\|\dd t = 
\int_\gamma (F \cdot T)\dd s, \] 
where the last integral is the line integral of the scalar function 
$F \cdot T$ on $\gamma$ which we defined in Section~\ref{subsec:4.1}
(see page 53). Some authors define the line integral of a vector field 
on a regular curve as $\int_\gamma (F \cdot T)\dd s$. 

By thinking of the definition in terms of the unit tangent vector, we see that 
line integrals of vector fields are uniquely defined by the curve only 
up to orientation. Indeed, at any point $\gamma(t)$ on $\gamma$, there are only 
two possible \emph{unit} tangent directions, namely $T(t)$ and $-T(t)$. 
If we reparametrize the curve as $\bar\gamma(u) = \gamma(a+b-u)$ where 
$u \in [a, b]$, we obtain a parametrization that traces the curve in the 
opposite direction, starting at $\gamma(b) = \bar\gamma(a)$ and ending 
at $\gamma(a) = \bar\gamma(b)$. Moreover, we have $\bar\gamma'(u) = 
-\gamma'(a+b-u)$ so that $\bar T(u) = -T(a+b-u)$ at the point 
$p = \gamma(u) = \gamma(a+b-u)$ on the curve. This implies that 
\[ \int_{\bar\gamma} F\cdot{\rm d}r = -\int_\gamma F\cdot{\rm d}r. \] 
We denote the reparametrization $\bar\gamma$ by $-\gamma$ and write 
\[ \int_{-\gamma} F\cdot{\rm d}r = -\int_\gamma F\cdot{\rm d}r. \] 
So far, we have only defined line integrals of vector fields on smooth 
regular curves. They can also be defined over piecewise smooth regular curves. 

\begin{defn}{defn:4.5}
    A parametrized curve $\gamma : [a, b] \to \R^n$ is called a {\bf 
    piecewise smooth regular curve} if $\gamma$ is a piecewise smooth 
    vector-valued function 
    \[ \gamma(t) = \begin{cases}
        \gamma_1(t), & \text{if } a = a_0 \leq t \leq a_1, \\ 
        \gamma_2(t), & \text{if } a_1 \leq t \leq a_2, \\ 
        \quad\vdots & \qquad\qquad \vdots \\ 
        \gamma_\ell(t), & \text{if } a_{\ell-1} \leq t \leq a_\ell = b 
    \end{cases} \] 
    such that $\gamma'_i(t) \neq \mathbf 0$ for all $t \in [a_{i-1}, a_i]$ 
    and $i = 1, \dots, \ell$. 
\end{defn}\vspace{-0.25cm}

The line integral of the vector field $F$ on the piecewise smooth regular 
curve $\gamma$ is then defined to be 
\[ \int_\gamma F\cdot{\rm d}r := \sum_{i=1}^\ell \int_{\gamma_i} F\cdot{\rm d}r. \] 

{\bf Example.} Let $\gamma : [-1, 3] \to \R^2$ be the piecewise smooth curve 
in $\R^2$ given by 
\[ \gamma(t) = \begin{cases}
    \gamma_1(t) = (t, t^2), & \text{if } -1 \leq t \leq 1, \\ 
    \gamma_2(t) = (2-t, 1), & \text{if } 1 \leq t \leq 3. 
\end{cases} \] 
Consider the smooth vector field $F(x, y, z) = (x-1, 4xy)$ on $\R^2$. Then 
we have 
\begin{align*} 
    \int_\gamma F\cdot{\rm d}r 
    &= \int_{\gamma_1} F\cdot{\rm d}r + \int_{\gamma_2} F\cdot{\rm d}r \\ 
    &= \int_{-1}^1 F(\gamma_1(t)) \cdot \gamma'_1(t)\dd t 
    + \int_1^3 F(\gamma_2(t)) \cdot \gamma'_2(t) \dd t. 
\end{align*} 
Note that $F(\gamma_1(t)) = (t-1, 4t^3)$ and $F(\gamma_2(t)) = 
(1-t, 4(2-t))$ with $\gamma'_1(t) = (1, 2t)$ and $\gamma'_2(t) = 
(-1, 0)$. Therefore, we can compute 
\begin{align*}
    F(\gamma_1(t)) \cdot \gamma'_1(t) &= (t-1, 4t^3) \cdot (1, 2t) = 
    8t^4 + t - 1, \\ 
    F(\gamma_2(t)) \cdot \gamma'_2(t) &= (1-t, 4(2-t)) \cdot (-1, 0) = 
    t-1, 
\end{align*}
and the line integral of $F$ over $\gamma$ is 
\begin{align*} 
    \int_\gamma F\cdot{\rm d}r 
    &= \int_{-1}^1 (8t^4 + t - 1)\dd t + \int_1^3 (t-1)\dd t \\ 
    &= \left[ \frac85t^5 + \frac12t^2 - t\right]_{-1}^1 + 
    \left[ \frac12 t^2 - t \right]_1^3 \\ 
    &= \frac65 + 2 = \frac{16}{5}. 
\end{align*} 

Recall that the Fundamental Theorem of Calculus states that if a 
scalar one-variable function $g : [a, b] \to \R$ has an antiderivative 
$G : [a, b] \to \R$ (i.e. $G(x) = g(x)$ for all $x \in [a, b]$), then 
\[ \int_a^b g(x)\dd x = G(b) - G(a). \] 
We have a very similar result for line integrals of vector fields, which we 
call the fundamental theorem of line integrals of vector fields. 

\begin{theo}[Fundamental Theorem of Line Integrals of Vector Fields]{theo:4.6}
    Let $F : U \to \R^n$ be a vector field on an open subset $U$ of $\R^n$. 
    If there exists a scalar function $f : U \to \R$ such that $F = \nabla f$, 
    then we have 
    \[ \int_\gamma F\cdot{\rm d}r = f(\gamma(b)) - f(\gamma(a)) \] 
    for any piecewise smooth regular curve $\gamma$ in $U$. In 
    particular, the line integral is \emph{independent of path}. 
\end{theo}\vspace{-0.25cm}\newpage
\begin{pf}[Theorem~\ref{theo:4.6}]
    It is enough to prove the theorem for smooth regular curves. Suppose that 
    $F = \nabla f = (f_{x_1}, \dots, f_{x_n})$. Then for any smooth regular 
    curve $\gamma(t) = (x_1(t), \dots, x_n(t))$ with $t \in [a, b]$, we have 
    \[ F(\gamma(t)) \cdot \gamma'(t) = (f_{x_1}(\gamma(t)), \dots, 
    f_{x_n}(\gamma(t))) \cdot (x'_1(t), \dots, x'_n(t)) = 
    \sum_{i=1}^n f_{x_i}(\gamma(t))x'_i(t) = \frac{\rm d}{{\rm d}t} f(\gamma(t)), \] 
    where the last equality follows from the chain rule. By the Fundamental 
    Theorem of Calculus, we have 
    \[ \int_\gamma F\cdot{\rm d}r = \int_a^b F(\gamma(t)) \cdot \gamma'(t)\dd t 
     = \int_a^b \frac{\rm d}{{\rm d}t} f(\gamma(t))\dd t = f(\gamma(b)) - 
     f(\gamma(a)). \tag*{\qed} \] 
\end{pf}\vspace{-0.25cm}

{\bf Remark.} If there exists a scalar function $f : U \to \R$ such that 
$F = \nabla f$, then Theorem~\ref{theo:4.6} tells us that 
\[ \int_\gamma F \cdot {\rm d}r = 0 \] 
for any closed curve $\gamma$ 
(i.e. $\gamma(a) = \gamma(b)$) since $f(\gamma(b)) - f(\gamma(a)) = 0$. 
In particular, if we can find a closed 
curve $\gamma$ such that $\int_\gamma F \cdot {\rm d}r \neq 0$, then 
this implies that $F$ cannot be the gradient of a function. 

{\bf Example.} Consider the vector field 
\[ F(x, y) = \left( -\frac{y}{x^2+y^2}, \frac{x}{x^2+y^2} \right) \] 
on the open set $U = \R^2 \setminus \{(0, 0)\}$. If $\gamma : [0, 2\pi] 
\to \R^2$ given by $t \mapsto (\cos t, \sin t)$ is the unit circle, then 
\[ F(\gamma(t)) = \left( -\frac{\sin t}{(\cos t)^2 + (\sin t)^2}, 
\frac{\cos t}{(\cos t)^2 + (\sin t)^2} \right) = (-\sin t, \cos t) = \gamma'(t) \] 
so that $F(\gamma(t)) \cdot \gamma'(t) = \gamma'(t) \cdot \gamma'(t) = 1$ and 
\[ \int_\gamma F\cdot {\rm d}r = \int_0^{2\pi} 1\dd t = 2\pi \neq 0. \] 
It follows that $F$ is not the gradient of a scalar function on $U$. 

\subsubsection{Integrals of vector fields over hypersurfaces} \label{subsubsec:4.2.2}
Let us now assume that $M$ is an $(n-1)$-dimensional smooth submanifold of $\R^3$. 
When $n = 3$, then $M$ is just a surface in $\R^3$. If $n > 3$, such an $M$ 
is called a {\bf hypersurface}. We have seen that $M$ is locally the zero 
set of a smooth scalar function $G : V \subset \R^n \to \R$ with $DG = 
\nabla G$ of maximal rank everywhere on an open subset $V$ of $\R^n$, and 
that the gradient is a normal vector field on $M \cap V = \{G \equiv 0\}$. 
Since $\nabla G$ is nowhere vanishing on $M \cap V$, we can normalize it to 
obtain a \emph{unit} normal vector field 
\[ N(p) := \frac{\nabla G(p)}{\|\nabla G(p)\|}, \] 
where $p \in M \cap V$. If $F : V \subset \R^n \to \R^n$ is a vector field on $V$, 
we can define the integral of $F$ on a bounded subset $B$ of $M \cap V$ as follows: 

\begin{defn}{defn:4.7}
    The integral of the vector field $F : V \subset \R^n \to \R$ on the 
    bounded subset $B$ of $M \cap V$ with unit normal vector $N$ is defined as 
    \[ \int_B F \cdot \dVol := \int_B (F \cdot N) \dVol, \] 
    where the second integral is the integral of the scalar function 
    $F \cdot N$ on $B$ (as in Definition~\ref{defn:4.2}).
\end{defn}\vspace{-0.25cm}

We compute a few examples. 
\begin{enumerate}[(1)]
    \item {\bf (Sphere.)} Let $S$ be the sphere $x^2 + y^2 + z^2 = a^2$ in $\R^3$, where $a > 0$. 
    This is the zero set of the function $G(x, y, z) = x^2 + y^2 + z^2 - a^2$ on 
    $\R^3$ whose gradient $\nabla G(x, y, z) = (2x, 2y, 2z)$ is nowhere vanishing 
    on $S$, giving us the unit normal vector field 
    \[ N(x, y, z) = \frac{\nabla G(x, y, z)}{\|\nabla G(x, y, z)\|} 
    = \frac{1}{\sqrt{x^2+y^2+z^2}}(x, y, z) = \frac1a(x, y, z) \] 
    since $x^2 + y^2 + z^2 = a^2$ on $S$. Consider the smooth vector field 
    $F(x, y, z) = (-y, x, 1)$ on $\R^3$. We will compute $\int_S F\cdot\dVol$. 
    In this case, we have 
    \[ F \cdot N = (-y, x, 1) \cdot \frac1a(x, y, z) = \frac1a(-yx + xy + z) 
    = \frac1az. \] 
    Therefore, we need to compute 
    \[ \int_S F \cdot \dVol = \int_S (F \cdot N)\dd A = \int_S \frac1az \dd A. \] 
    Now, we need to parametrize the sphere. Using the spherical coordinates 
    \[ \sigma(\theta, \varphi) = (a\cos\theta\sin\varphi, a\sin\theta\sin\varphi, a\cos\varphi) \] 
    with $(\theta, \varphi) \in U = (0, 2\pi) \times (0, \pi)$, we obtain 
    $\|\sigma_\theta \times \sigma_\varphi\| = a^2 \sin\varphi > 0$, so 
    \begin{align*}
        \int_S \frac1az\dd A 
        &= \int_U \frac1a(a\cos\varphi)\|\sigma_\theta \times \sigma_\varphi\|\dd\theta\dd\varphi \\ 
        &= \int_0^\pi \int_0^{2\pi} a^2 \cos\varphi \sin\varphi \dd\theta\dd\varphi \\ 
        &= 2\pi a^2 \int_0^\pi \sin\varphi \cos\varphi \dd\varphi \\ 
        &= 2\pi a^2 \left[ \frac12 \sin^2\varphi \right]_0^\pi = \pi a^2(0 - 0) = 0. 
    \end{align*}

    \item Suppose that $S$ is a surface in $\R^3$ and $\sigma : U 
    \subset \R^2 \to V \subset S$ is a smooth coordinate chart. The 
    standard unit normal $N_\sigma$ is then a smooth unit normal vector 
    field on $V$. If $F$ is any smooth vector field on $V$, its integral 
    on a bounded subset $B$ of $V$ with the unit vector field $N_\sigma$ is 
    \begin{align*}
        \int_B F \cdot \dVol 
        &= \int_B (F \cdot N_\sigma)\dd A \\ 
        &= \int_{\sigma^{-1}(B)} F(\sigma(u, v)) \cdot 
        \left( \frac{\sigma_u \times \sigma_v}{\|\sigma_u \times \sigma_v\|} 
        \right) \|\sigma_u \times \sigma_v\|\dd u \dd v \\ 
        &= \int_{\sigma^{-1}(B)} F(\sigma(u, v)) \cdot (\sigma_u \times \sigma_v)\dd u \dd v.
    \end{align*}
    For example, consider the half-cylinder $C = \{(x, y, z) \in \R^3 : 
    x^2+y^2=1,\,y\geq 0,\,2\leq z \leq 4\}$ in $\R^3$. Then $C$ is bounded 
    and can be parametrized by 
    \[ \sigma(u, v) = (\cos u, \sin u, v) \] 
    where $(u, v) \in U = [0, \pi] \times [2, 4]$ and $\sigma_u \times \sigma_v 
    = (\cos u, \sin u, 0)$. Let $F(x, y, z) = (xy, 2, z-1)$. Then 
    \[ F(\sigma(u, v)) \cdot (\sigma_u \times \sigma_v) = 
    (\cos u \sin u, 2, v-1) \cdot (\cos u, \sin u, 0) = (\cos u)^2 \sin^2 + 2\sin u. \] 
    The integral of $F$ on $C$ with normal vector $N_\sigma$ is 
    \begin{align*}
        \int_C F \cdot \dVol 
        &= \int_U F(\sigma(u, v)) \cdot (\sigma_u \times \sigma_v)\dd u\dd v \\ 
        &= \int_2^4 \int_0^\pi [(\cos u)^2 \sin u + 2\sin u] \dd u \dd v \\ 
        &= \int_2^4 \left[ -\frac{(\cos u)^3}{3} - 2\cos u \right]_0^\pi \dd v \\ 
        &= \int_2^4 \left[ \left( \frac13 + 2 \right) - \left(-\frac13 - 2\right) \right] \dd v \\ 
        &= \int_2^4 \frac{14}{3} \dd v = \frac{14}{3}(4-2) = \frac{28}{3}. 
    \end{align*}

\end{enumerate}

On a hypersurface $M$ in $\R^n$, the tangent space $T_pM$ to $M$ at any point 
$p \in M$ is an $(n-1)$-dimensional subspace of $T_p\R^n \simeq \R^n$. 
Hence, there is only a $1$-dimensional subspace of normal directions to $T_pM$ 
at $p$ in $T_p\R^n$ so that there are only two \emph{unit} normal vectors 
to $M$ at $p$; one is minus the other. To be precise, if $N$ is a unit 
normal vector to $M$ at $p$, then $\pm N$ are the only possible unit 
normal vectors to $M$ at $p$. 

An {\bf orientation} of the tangent space $T_pM$ is a choice of unit 
normal vector (either $N$ or $-N$). Moreover, an {\bf orientation} on the 
hypersurface $M$ is a choice of the smooth unit normal vector field $N$ on $M$. 
For example, if $M$ is the zero set of a smooth scalar function $G : 
V \subset \R^n \to \R$ with $\nabla G$ of maximal rank everywhere on an open 
subset $V$ of $\R^n$, it admits two orientations, namely 
\[ N(p) := \pm \frac{\nabla G(p)}{\|\nabla G(p)\|}, \] 
where $p \in M$. Similarly, if $S$ is a surface in $\R^3$ and $\sigma : 
U \subset \R^2 \to V \subset S$ is a smooth coordinate chart, then the 
open set $V \subset S$ admits two possible orientations, namely $\pm N_\sigma$. 

\begin{defn}{defn:4.8}
    A hypersurface $M$ in $\R^n$ is called {\bf orientable} if it admits a 
    smooth unit normal vector $N$. A choice of a smooth unit normal vector 
    field $N$ on $M$ is called an {\bf orientation} of $M$. 
\end{defn}\vspace{-0.25cm}

We give some examples of orientable hypersurfaces. 
\begin{enumerate}[(1)]
    \item The zero set $M$ of a smooth scalar function $G : V \subset \R^n 
    \to \R$ with $\nabla G$ of maximal rank everywhere on an open 
    set $V$ of $\R^n$ is orientable. Its two orientations are 
    $\pm\nabla G/\|\nabla G\|$. Since any hypersurface is locally 
    the zero set of such a function, this means that hypersurfaces 
    are always locally orientable. 

    \item If $S$ is a surface in $\R^3$ and $\sigma : U \subset \R^2 \to V \subset S$ 
    is a smooth coordinate chart, then the open set $V \subset \R^3$ is 
    orientable and its orientations are $\pm N_\sigma$. Hence, any 
    surface in $\R^3$ is orientable. 

    \item Although every hypersurface is locally orientable, it might not be 
    globally orientable. For example, the M\"obius strip is \emph{not} 
    orientable because it does not admit a smooth unit normal vector field 
    (see Assignment 5).
\end{enumerate}

{\bf Remark.} Integrals of vector fields on orientable hypersurfaces 
are uniquely determined by the hypersurface only up to orientation. 
Indeed, changing the orientation changes the sign of the integral, with 
\[ \int_M (F \cdot (-N))\dVol = -\int_M (F \cdot N)\dVol\!. \] 
Therefore, it is important to specify an orientation when computing the 
integral of a vector field. 

\subsection{Stokes' Theorem} \label{subsec:4.3}
We end our presentation of integration on submanifolds with Stokes' Theorem, 
which is a generalization of the Fundamental Theorem of Calculus. 
First, we need the following definition. 

\begin{defn}{defn:4.9}
    Let $F : U \subset \R^n \to \R^n$ be a smooth vector field on an 
    open subset $U$ of $\R^n$. The {\bf divergence of $F$} is the 
    scalar function on $U$ defined by 
    \[ \Div F := \frac{\partial F_1}{\partial x_1} + \cdots + \frac{\partial F_n}{\partial x_n}. \] 
\end{defn}\vspace{-0.25cm}

We now state Stokes' Theorem. 

\begin{theo}[Stokes' Theorem]{theo:4.10}
    Suppose that $\Omega \subset \R^n$ is a bounded $n$-dimensional submanifold 
    of $\R^n$ and that its boundary $\partial\Omega$ is an orientable 
    $(n-1)$-dimensional submanifold of $\R^n$ whose orientation is a 
    smooth unit vector field $N$ on $\partial\Omega$ pointing outward. 
    Moreover, let $F$ be a smooth vector field on $\Omega$. Then we have 
    \[ \int_\Omega \Div F \dVol = \int_{\partial\Omega} F\cdot\dVol. \] 
\end{theo}\vspace{-0.25cm}

In other words, the integral of the scalar function $\Div F$ on $\Omega$ 
is equal to the integral of the vector field $F$ on $\partial\Omega$ 
with unit normal vector field pointing outward. Note that since $\Omega$ 
is an $n$-dimensional submanifold of $\R^n$, it must be an open set 
so that the integral 
\[ \int_\Omega \Div F \dVol \] 
is just standard integration on a region in $\R^n$. 

{\bf Example.} Let $\Omega = \{(x, y, z) \in \R^3 : x^2 + y^2 + z^2 \leq 1\}$ be the 
unit ball in $\R^3$. Its boundary is the unit sphere $\mathbb S^2 : x^2 + y^2 + z^2 = 1$, 
which is orientable with possible orientations $N = \pm(x, y, z)$, where 
$(x, y, z) \in \mathbb S^2$. To verify Stokes' Theorem on $\Omega$, we need 
to pick the unit normal vector field that is pointing outwards from the 
unit ball, namely $N = (x, y, z)$. Consider the vector field 
$F(x, y, z) = (-y, x, 1 + 2z)$ on $\R^3$ with 
\[ \Div F = \frac{\partial F_1}{\partial x} + \frac{\partial F_2}{\partial y} 
+ \frac{\partial F_3}{\partial z} = 0 + 0 + 2 = 2. \] 
For the left-hand side of Stokes' Theorem, we first see that
\[ \int_\Omega \Div F \dVol = \int_\Omega 2\dVol = 2 \int_\Omega \dd x \dd y \dd z \] 
since $\Omega$ is an open subset of $\R^3$. This integral corresponds to the volume of the unit ball, which is 
\[ \int_\Omega \dd x \dd y \dd z = \frac{4\pi}3, \] 
Therefore, we find that $\int_\Omega \Div F = 8\pi/3$. Now, 
to compute the right-hand side, we observe that 
\[ F \cdot N = (-y, x, 1+2z) \cdot (x, y, z) = -yx + xy + (1+2z)z = z + 2z^2. \] 
Using the spherical coordinates 
\[ \sigma(\theta, \varphi) = (\cos\theta\sin\varphi, \sin\theta\sin\varphi, \cos\varphi) \] 
where $(\theta, \varphi) \in U = (0, 2\pi) \times (0, \pi)$, we have 
\[ (F \cdot N)(\sigma(\theta, \varphi)) = \cos \varphi + 2(\cos \varphi)^2 \] 
and $\|\sigma_\theta \times \sigma_\varphi\| = \sin \varphi > 0$. 
Therefore, we obtain 
\begin{align*}
    \int_{\mathbb S^2} (F \cdot N)\dd A 
    &= \int_U (\cos\varphi + 2(\cos\varphi)^2)\|\sigma_\theta \times \sigma_\varphi\|\dd\theta\dd\varphi \\ 
    &= \int_0^\pi \int_0^{2\pi} (\cos\varphi + 2(\cos\varphi)^2) \sin\varphi \dd\theta \dd \varphi \\ 
    &= 2\pi \int_0^\pi (\cos\varphi + 2(\cos\varphi)^2) \sin\varphi\dd\varphi \\ 
    &= 2\pi \left[ -\frac12(\cos\varphi)^2 - \frac23(\cos\varphi)^3 \right]_0^\pi \\ 
    &= 2\pi \left[ \left( -\frac12 + \frac23 \right) - \left( -\frac12 - \frac23 \right) \right]  
    = \frac{8\pi}{3}.
\end{align*}
This is precisely equal to $\int_\Omega \Div F$, as we expected! 

\subsubsection{Applications} \label{subsubsec:4.3.1}
An important application of Stokes' Theorem is to simplify the computation 
of integrals (when possible). 

Integrating a vector field over a hypersurface that is the union of several 
hypersurfaces can be quite complicated and time consuming. However, 
if this hypersurface bounds a region in $\R^n$ that is easy to describe, it 
may be better to use Stokes' Theorem to convert the integral of the 
vector field into a scalar integral of its divergence. We give a few concrete 
examples of this. 
\begin{enumerate}[(1)]
    \item Suppose we want to integrate the vector field $F(x, y, z) = 
    (x^2y^2, 3z-1, e^x\cos y)$ for $(x, y, z) \in \R^3$ over the boundary 
    $\partial\Omega$ of the rectangular box $\Omega = [-1, 2] \times [4, 5] 
    \times [0, 6]$, with outward unit normal vector field $N$. While the 
    vector field $F$ is smooth and has component functions that are 
    easy to integrate over the boundary $\partial\Omega$ of $\Omega$ 
    which is made up of $2$ horizontal faces and $4$ vertical faces, 
    we would need to compute $6$ integrals in this way! Instead, we can use 
    Stokes' Theorem, which tells us that 
    \[ \int_{\partial\Omega} F\cdot\dVol = \int_\Omega \Div F\dVol 
    = \int_\Omega \Div F \dd x \dd y \dd z \] 
    since $N$ is pointing outward. By setting $F(x, y, z) = 
    (A(x, y, z), B(x, y, z), C(x, y, z))$, we see that 
    \[ \Div F = \frac{\partial A}{\partial x} + \frac{\partial B}{\partial y} 
    + \frac{\partial C}{\partial z} = 2xy^2 + 0 + 0 = 2xy^2. \] 
    It follows that 
    \begin{align*}
        \int_\Omega \Div F \dVol 
        &= \int_0^6 \int_4^5 \int_{-1}^2 2xy^2 \dd x \dd y \dd z \\ 
        &= \int_0^6 \int_4^5 \Big[ x^2 y^2 ]_{-1}^2 \dd y \dd z \\ 
        &= 3 \int_0^6 \int_4^5 y^2 \dd y \dd z \\ 
        &= 3 \int_0^6 \left[ \frac{y^3}{3} \right]_4^5 \dd z \\ 
        &= \int_0^6 (5^3 - 4^3)\dd z = 6 \cdot 61 = 366,
    \end{align*}
    and therefore $\int_{\partial\Omega} F \cdot \dVol = 366$. 

    \item Consider the vector field 
    \[ F(x, y, z) = \left( \frac{x^3}{3} + y, \frac{y^3}{3} - \sin(xz), z - x - y \right) \] 
    for $(x, y, z) \in \R^3$ over the boundary $\partial\Omega$ of the solid cylinder 
    \[ \Omega = \{(x, y, z) \in \R^3 : x^2 + y^2 \leq 1,\, 0 \leq z \leq 2\} \] 
    with outward unit normal vector field $N$. The boundary of $\Omega$ 
    is made up of two discs of radius $1$ in the horizontal planes $z = 0$ and 
    $z = 2$, as well as the part of the cylinder $x^2 + y^2 = 1$ for which 
    $0 \leq z \leq 2$. Therefore, computing the integral of $F$ over 
    $\partial\Omega$ would require $3$ parametrizations. Moreover, the integral 
    of $F$ would be quite messy to compute given the $\sin(xz)$ in the second 
    component. Instead, we use Stokes' Theorem, which tells us that 
    \[ \int_{\partial\Omega} F \cdot \dVol = \int_\Omega \Div F \dVol
    = \int_\Omega \Div F \dd x \dd y \dd z \]
    since $N$ is pointing outward. Again, we set $F(x, y, z) = 
    (A(x, y, z), B(x, y, z), C(x, y, z))$ to get 
    \[ \Div F = \frac{\partial A}{\partial x} + \frac{\partial B}{\partial y} 
    + \frac{\partial C}{\partial z} = x^2 + y^2 + 1. \]
    Since $\Omega$ is a solid cylinder, we will use cylindrical coordinates 
    so that $x = r\cos\theta$, $y = r\sin\theta$, and $z = z$ for 
    $0 \leq r \leq 1$, $0 \leq \theta \leq 2\pi$, and $0 \leq z \leq 2$. 
    This gives $x^2 + y^2 + 1 = r^2 + 1$ and ${\rm d}x\dd y \dd z = 
    r\dd r \dd\theta \dd z$, so 
    \begin{align*}
        \int_\Omega \Div F \dVol 
        &= \int_0^2 \int_0^{2\pi} \int_0^1 (r^2 + 1)r \dd r \dd \theta \dd z \\ 
        &= \int_0^2 \int_0^{2\pi} \left[ \frac14r^4 + \frac12r^2 \right]_0^1 \dd\theta \dd z \\ 
        &= \frac34 \int_0^2 \int_0^{2\pi} \dd\theta \dd z = \frac34 \cdot 4\pi = 3\pi. 
    \end{align*}
    We conclude that $\int_{\partial\Omega} F \cdot \dVol = 3\pi$. 
\end{enumerate}

Stokes' Theorem can also be used to relate an integral over a region to 
an integral over its boundary. This is the case in the proof of the 
Gauss-Bonnet Theorem.\newpage

\end{document}
