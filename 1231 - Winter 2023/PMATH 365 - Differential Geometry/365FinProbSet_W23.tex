\documentclass{article}
\usepackage{amssymb, amsmath, hyperref, amsthm}
\usepackage{graphicx}
\usepackage{anysize}
\usepackage{xcolor}
\usepackage{enumerate}
\marginsize{1in}{1in}{1in}{1in}

\DeclareMathOperator{\R}{\mathbb{R}}
\DeclareMathOperator{\tr}{tr}
\DeclareMathOperator{\supp}{supp}
\DeclareMathOperator{\sgn}{sgn}
\DeclareMathOperator{\spanning}{span}
\DeclareMathOperator{\Jac}{Jac}

\newenvironment{solution}{%
	\color{blue} 
	\setlength\parindent{0pt}\par\medskip\textbf{Solution.}}
	%{%
	%\hfill\tiny$\blacksquare$\par\medskip}

\begin{document}
	\thispagestyle{empty}
	\begin{center}
		{\bf PMATH 365: Suggested problems for the final}
	\end{center}
	
\noindent
To prepare for the final exam, I recommend reviewing Quizzes 1-10. I also recommend going through the following list of questions.

\bigskip

\noindent
{\bf A. Differentiable maps, immersions, embeddings and submanifolds.}

\begin{enumerate}	

\item 
Define the following:
\begin{enumerate}
\item
A $k$-dimensional topological submanifold of $\R^n$. 

\begin{solution}
	A set $M \subset \R^n$ such that for all $p \in M$, there exists 
	an open neighbourhood $V$ of $p$, an open set $U \subset \R^k$, and a 
	homeomorphism $\alpha : U \subset \R^k \to V \subset M \subset \R^n$. 
\end{solution}

\item
An immersion of class $C^r$, $r \geq 1$.

\begin{solution}
	A map $\alpha : U \subset \R^k \to \R^n$ such that $\alpha$ is of 
	class $C^r$ and $D\alpha$ has maximal rank $k$ everywhere on $U$.
\end{solution}

\item
An embedding of class $C^r$, $r \geq 1$.

\begin{solution}
	An immersion that is homeomorphic onto its image. 
\end{solution}

\item
A $k$-dimensional submanifold of $\R^n$ of class $C^r$, $r \geq 1$. 

\begin{solution}
	A set $M \subset \R^n$ such that for all $p \in M$, there exists an 
	open neighbourhood $V$ of $p$, an open set $U \subset \R^k$, and 
	a homeomorphism $\alpha : U \subset \R^k \to V \subset M \subset \R^n$ 
	satisfying the following properties:
	\begin{enumerate}[(i)]
		\item $\alpha$ is of class $C^r$; 
		\item $D\alpha(x)$ has rank $k$ for all $x \in U$. 
	\end{enumerate}
\end{solution}

\item 
A coordinate chart (of class $C^r$) of a $k$-dimensional submanifold of $\R^n$ (of class $C^r$, $r \geq 1$).

\begin{solution}
	This is a homeomorphism as in (d).
\end{solution}

\item
An atlas (of class $C^r$) of a $k$-dimensional submanifold of $\R^n$ (of class $C^r$, $r \geq 1$).

\begin{solution}
	A set of coordinate charts $\{\alpha_i : U_i \subset \R^k \to V_i \subset M 
	\subset \R^n\}_{i\in I}$ of class $C^r$ for $M$ (where $I$ is some index set) such that 
	$\bigcup_{i\in I} V_i = M$. 
\end{solution} 

\end{enumerate}


\item
What is the difference between an immersion and an embedding? Give an example of both. In particular, give an example of an immersion that is {\em not} an embedding.

\begin{solution}
	An embedding has the requirement that it is homeomorphic onto 
	its image. The canonical immersion is the inclusion map $\iota 
	: \R^k \to \R^n$ defined by $(x_1, \dots, x_k) \mapsto 
	(x_1, \dots, x_k, 0, \dots, 0)$. This is an immersion of class $C^\infty$
	since $\iota$ is smooth with 
	\[ D\iota = \left[ \begin{array}{ccc}
		1 & \cdots & 0 \\ 
		\vdots & \ddots & \vdots \\ 
		0 & \cdots & 1 \\ \hline 
		0 & \cdots & 0 \\ 
		\vdots & \ddots & \vdots \\ 
		0 & \cdots & 0 
	\end{array} \right] \] 
	so that $D\iota$ has rank $k$ everywhere. It is also an embedding 
	as it is homeomorphic onto its image with inverse 
	$\iota^{-1} : \iota(\R^k) \subset \R^n \to \R^k$ given by 
	$(x_1, \dots, x_k, 0, \dots, 0) \mapsto (x_1, \dots, x_k)$. 

	For an example of an immersion that is not an embedding, consider the 
	parametrization $\alpha(t) = (t^2-1, t(t^2-1))$ of the $\alpha$-curve, 
	where $t \in \R$. We see that $\alpha$ is smooth with derivative matrix 
	$D\alpha(t) = (2t, 3t^2-1) \neq (0, 0)$ for all $t \in \R$, so 
	$\alpha$ is an immersion of class $C^\infty$. However, it is not 
	homeomorphic onto its image since it is not injective. Indeed, we have 
	$\alpha(1) = \alpha(-1) = (0, 0)$. 
\end{solution}

\item
Give an example of a subset of $\R^n$ that is {\em not} a topological submanifold of $\R^n$. Explain!

\begin{solution}
	We again consider the $\alpha$-curve $C = \{(x, y) \in \R^2 : y^2 = x^2(x+1)\} \subset \R^2$, 
	which can be parametrized by the map $\alpha : \R \to C \subset \R^2$ 
	given by $t \mapsto (t^2-1, t(t^2-1))$. Note that if we remove the 
	points $t = \pm1$, we see that $\alpha$ has inverse 
	\begin{align*} 
		\alpha^{-1} : C \setminus \{(0, 0)\} &\to \R \setminus \{\pm1\} \\ 
		(x, y) &\mapsto 1/x.
	\end{align*}
	That is, $C$ is a $1$-dimensional topological submanifold of $\R^2$ 
	away from the point $(0, 0)$. 

	Now, suppose that $C$ were a topological submanifold of $\R^2$. 
	Our discussion above means that $C$ must be $1$-dimensional. Since 
	$(0, 0) \in C$, there exists an open neighbourhood $V$ of $(0, 0)$, 
	an open set $U \subset \R^1$, and a homeomorphism $\alpha : 
	U \subset \R^1 \to V \subset C \subset \R^2$. Since $\alpha$ is a homeomorphism, 
	there exists a unique point $t_0 \in U$ such that $\alpha(t_0) = (0, 0)$. 
	Since $U$ is open, we can pick $\varepsilon > 0$ such that $B_\varepsilon(t_0) = 
	(t_0-\varepsilon, t_0+\varepsilon) \subset U$. Write $U' = B_\varepsilon(t_0)$ 
	and let $V' = \alpha(U')$ so that $\alpha|_{U'} : U' \to V'$ is also a 
	homeomorphism. 
	
	Observe that $V' \setminus \{(0, 0)\}$ consists of 
	three or four connected components depending on how large $V'$ is: one on the bottom 
	right quadrant, one on the top right quadrant, and one or two on the left 
	of the $y$-axis. On the other hand, $U' \setminus \{t_0\}$ has two components, 
	which is a contradiction since homeomorphisms preserve the number of 
	connected components. Therefore, $C$ is not a topological submanifold of $\R^2$.
\end{solution}

\item
Give an example of a subset of $\R^n$ that is a topological submanifold but {\em not} a submanifold of class $C^r$ for any $r \geq 1$. Explain!

\begin{solution}
	Let $M = \{(x, |x|) : x \in \R\} \subset \R^2$, which is the 
	graph of $f(x) = |x|$ over $x \in \R$. In particular, since $f(x) 
	= |x|$ is continuous, it follows that $M$ is a $1$-dimensional topological 
	submanifold of $\R^2$. Note that $f$ is smooth away from $x = 0$, 
	so $M \setminus \{(0, 0)\}$ is a $1$-dimensional submanifold of class $C^\infty$. 

	However, we claim that $M$ is not a submanifold of class $C^r$ for some $r \geq 1$. 
	Suppose otherwise, so at the point $(0, 0) \in M$, there exists 
	an open neighbourhood $V$ of $(0, 0)$, an open set $U \subset \R$, 
	and a homeomorphism $\alpha : U \subset \R \to V \subset M \subset \R^2$ 
	such that $\alpha$ is of class $C^r$ and $D\alpha$ has maximal 
	rank $1$ everywhere on $U$. In particular, this means that 
	$D\alpha(t) = \alpha'(t) \neq (0, 0)$ for all $t \in U$. But 
	$\alpha'(t)$ is tangent to $M$ at $\alpha(t)$, which gives two possibilities: 
	\begin{itemize}
		\item If $\alpha(t)$ is on the line $y = x$, then $\alpha'(t)$ is a 
		direction vector of $y = x$. This means that for some $c : I 
		\subset \R \to \R$, we have 
		\[ \alpha'(t) = c(t) \begin{pmatrix}
			1 \\ 1 
		\end{pmatrix}. \] 
		Since $\alpha$ is of class $C^r$ where $r \geq 1$, this means  
		$\alpha'(t)$ is continuous, so $c : I \to \R$ is continuous.
		\item If $\alpha(t)$ is on the line $y = -x$, then $\alpha'(t)$ 
		is a direction vector of $y = -x$. Then for some $d : I' \subset \R \to \R$, 
		we have 
		\[ \alpha'(t) = d(t) \begin{pmatrix}
			1 \\ - 1 
		\end{pmatrix}. \] 
		The same argument shows that $d : I' \to \R$ is continuous.
	\end{itemize}
	But $\alpha$ is a bijection, so $(0, 0) = \alpha(t_0)$ for some $t_0 \in U$. 
	By the continuity of $\alpha'(t)$, we obtain 
	\[ \lim_{t\to t_0^-} \alpha'(t) = \lim_{t\to t_0^+} \alpha'(t). \] 
	Without loss of generality, suppose that $\alpha(t)$ is moving along $M$ 
	from left to right. (We can parametrize in the other direction otherwise.) 
	Then if $t < t_0$, then we are in the second case above, whereas if 
	$t > t_0$, then we are in the first case. This implies that 
	\[ \lim_{t\to t_0^-} \alpha'(t) = \lim_{t\to t_0^-} d(t) \begin{pmatrix}
		1 \\ -1 
	\end{pmatrix} = \lim_{t\to t_0^+} c(t) \begin{pmatrix}
		1 \\ 1 
	\end{pmatrix} = \lim_{t\to t_0^+} \alpha'(t). \] 
	But these vectors are not parallel to each other, so 
	$\lim_{t\to t_0^-} d(t) = \lim_{t\to t_0^+} c(t) = 0$. This implies that 
	$\alpha'(t_0) = (0, 0)$, which contradicts the fact that $D\alpha(t) 
	= \alpha'(t)$ has rank $1$ for all $t \in U$. 
\end{solution}

\item
True of false? 
\begin{enumerate}
\item Every topological submanifold of $\R^n$ is also a submanifold of $\R^n$ of class $C^r$ for some $r \geq 1$.

\begin{solution}
	False, take the graph of $f(x) = |x|$.
\end{solution}

\item Every submanifold of $\R^n$ of class $C^r$ for some $r \geq 1$ is also a topological submanifold of $\R^n$.

\begin{solution}
	True. 
\end{solution}

\item A subset of $\R^n$ that is not a topological submanifold may be a submanifold of class $C^r$ for some $r \geq 1$. 

\begin{solution}
	False; consider the contrapositive of (b). 
\end{solution}

\item Every topological submanifold of $\R^n$ admits an atlas of class $C^r$ for some $r \geq 1$.

\begin{solution}
	False. If a topological submanifold of $\R^n$ admits an atlas of class 
	$C^r$ for some $r \geq 1$, then it is a submanifold of $\R^n$ of 
	class $C^r$, but this is not true of all topological submanifolds by (a). 
\end{solution}

\item Every coordinate chart of class $C^r$, $r \geq 1$, of a submanifold is a homeomorphism.

\begin{solution}
	True. 
\end{solution}

\item Every coordinate chart of class $C^r$, $r \geq 1$, of a submanifold is an embedding.

\begin{solution}
	True. 
\end{solution}

\item Let $\alpha: U \subset \R^k \rightarrow V \subset M \subset \R^n$ be a coordinate chart of class $C^r$, $r \geq 1$, of a $k$-dimensional submanifold $M$ of $\R^n$. Then, $\alpha^{-1}: V \subset \R^n \rightarrow U \subset \R^k$ is also of class $C^r$.

\begin{solution}
	True. 
\end{solution}

\item Every immersion is a homeomorphism.

\begin{solution}
	False; take the parametrization $\alpha(t) = (t^2-1, t(t^2-1))$ of the $\alpha$-curve. 
\end{solution}

\item Every embedding is a homeomorphism. 

\begin{solution}
	True. 
\end{solution}

\item Every submanifold of $\R^n$ is also an immersed submanifold of $\R^n$ of class $C^r$. 

\begin{solution}
	True; the coordinate charts are embeddings and therefore immersions.
\end{solution}

\item The $n$-sphere admits an atlas that consists of a single chart.

\begin{solution}
	False. The $n$-sphere is compact, and hence closed and bounded by Heine-Borel. 
	The domain of a coordinate chart is an open set. 
	Since homeomorphisms send closed sets to closed sets and the only 
	open and closed set in $\R^{n+1}$ is $\varnothing$, it is impossible 
	to use a single chart. 
\end{solution}

\item Every $k$-dimensional submanifold of $\R^n$  of class $C^r$ , $r \geq 1$, is locally the graph of a function $f: U \subset \R^r \rightarrow \R^{n-k}$ of class $C^r$.

\begin{solution}
	True.
\end{solution}

\item Every submanifold of $\R^n$  of class $C^r$ , $r \geq 1$, is locally the zero set a function $F: U \subset \R^n \rightarrow \R^{n-k}$ of class $C^r$ of maximal rank (that is, $DF$ has maximal rank $n-k$ on $F^{-1}(0)$).

\begin{solution}
	True. 
\end{solution}

\end{enumerate}

\item 
Determine whether the following sets are topological submanifolds of $\mathbb{R}^n$ or submanifolds of $\R^n$ of class $C^r$ for some $r \geq 1$. Justify your answers!

\begin{enumerate}
\item $\R^n$.

\begin{solution}
	$\R^n$ is a submanifold of $\R^n$ of class $C^\infty$ since we can 
	choose the identity map on $\R^n$ as the coordinate chart.  
\end{solution}

\item An open subset $U$ of $\R^n$.

\begin{solution}
	This is a submanifold of $\R^n$ of class $C^\infty$; take the identity 
	map on $U$ as the coordinate chart. 
\end{solution}

\item The graph of a continuous function $f: U \subset \R^k \rightarrow \R^{n-k}$.

\begin{solution}
	This is a $k$-dimensional topological submanifold of $\R^n$. If 
	$M = \{(x, f(x)) \in \R^k \times \R^{n-k} : x \in U\}$ is the graph, 
	then $\alpha : U \subset \R^k \to M \subset \R^n$ given by 
	$x \mapsto (x, f(x))$ is a homeomorphism with inverse 
	$(x_1, \dots, x_n) \mapsto (x_1, \dots, x_k)$. 
\end{solution}

\item The graph of a function $f: U \subset \R^k \rightarrow \R^{n-k}$ of class $C^r$ for some $r \geq 1$.

\begin{solution}
	This is a $k$-dimensional submanifold of class $C^r$. Take the same 
	homeomorphism $\alpha : U \subset \R^k \to M \subset \R^n$ 
	defined by $x \mapsto (x, f(x))$ as above. This is of class 
	$C^r$ since $f$ is of class $C^r$ and $D\alpha(x)$ has 
	rank $k$ for all $x \in U$ since the upper part of $D\alpha(x)$ 
	is the $k \times k$ identity matrix.
\end{solution}

\item The zero set of a continuous function $F: U \subset \R^n \rightarrow \R^{n-k}$.

\begin{solution}
	This is not a topological submanifold of $\R^n$ in general. 
	Consider the $\alpha$-curve, which is the zero set of 
	$f(x, y) = y^2 - x^2(x+1)$ but is not a topological submanifold of $\R^2$.  
\end{solution}

\item The zero set of a function $F: U \subset \R^n \rightarrow \R^{n-k}$ of class $C^r$, $r \geq 1$, of maximal rank.

\begin{solution}
	This is a $k$-dimensional submanifold of class $C^r$; in fact, this is 
	an equivalent characterization.
\end{solution}

\item
$\{ (x,y) \in \mathbb{R}^2: y=x^2, -1 \leq x \leq 1 \} \cup \{ (x,y) \in \mathbb{R}^2: y=1, -1 \leq x \leq 1 \} \subset \mathbb{R}^2$.

\begin{solution}
	This is a $1$-dimensional topological submanifold of $\R^2$, but 
	not a submanifold of $\R^2$ of class $C^r$ by considering the 
	points $(\pm1, 1)$, which lie on both curves. 
\end{solution}

\item
The $\alpha$-curve $\{ (x,y) \in \mathbb{R}^2: y^2 = x^2(x+1) \} \subset \mathbb{R}^2$. For a picture, \href{http://maabooks.blogspot.com/2012/03/algebraic-curves-questions-and.html}{see} (second picture).

\begin{solution}
	Not a topological submanifold of $\R^2$.
\end{solution}

\item
$\{ (x,y) \in \mathbb{R}^2: xy = 1 \} \subset \mathbb{R}^2$.

\begin{solution}
	This is a smooth submanifold of $\R^2$ since it is the graph 
	of the smooth function $f(x) = 1/x$ on $x \in \R \setminus \{0\}$. 
\end{solution}

\item
The cylinder $\{ (x,y,z) \in \mathbb{R}^3: x^2+y^2 = 1 \} \subset \mathbb{R}^3$.

\begin{solution}
	Smooth submanifold of $\R^3$ since it is the graph of the smooth 
	function $f(x) = x^2+y^2-1$ on $x \in \R$. 
\end{solution}

\item
The cone $\{ (x,y,z) \in \mathbb{R}^3:  z = \sqrt{x^2+y^2} \} \subset \mathbb{R}^3$.

\begin{solution}
	Topological submanifold of $\R^3$ since it is the graph of the continuous 
	function $f(x, y) = \sqrt{x^2+y^2}$, but not a submanifold of $\R^3$ 
	of class $C^r$ for some $r \geq 1$ by considering the point $(0, 0, 0)$ 
	in the cone. 
\end{solution}

\item
The double cone $\{ (x,y,z) \in \mathbb{R}^3:  z^2 = x^2+y^2 \} \subset \mathbb{R}^3$.

\begin{solution}
	Not a topological submanifold of $\R^3$ by considering $(0, 0, 0)$.
\end{solution}

\item
The twisted cubic $\{ (x,y,z) \in \mathbb{R}^3: y = x^2, z = x^3 \} \subset \mathbb{R}^3$.

\begin{solution}
	Smooth submanifold of $\R^3$, since it is the graph of $f(x) = (x^2, x^3)$.
\end{solution}

\item
$\{ (x,y,z) \in \mathbb{R}^3: y=-1 \} \cup \{ (x,y,z) \in \mathbb{R}^3: x=y=0 \} \subset \mathbb{R}^3$.

\begin{solution}
	Neither; one of the sets is $2$-dimensional while the other is $1$-dimensional.
\end{solution}

\item
The $n$-sphere $S^n = \{ (x_1,\dots,x_{n+1}) \in \mathbb{R}^{n+1}:  x_1^2+\cdots+x_{n+1}^2 = 1 \} \subset \mathbb{R}^{n+1}$.

\begin{solution}
	Smooth submanifold of $\R^{n+1}$ since it is the zero set of the function
	$f(x_1, \dots, x_{n+1}) = x_1^2 + \cdots + x_{n+1}^2 - 1$. 
\end{solution}

\item
The general linear group $GL(n,\R) = \{ A \in M_{n \times n}(\R) : \det A \neq 0 \} \subset M_{n \times n} = \R^{n^2}$. 

\begin{solution}
	Note that $\det$ is a smooth function as it is a polynomial 
	of the matrix entries. Then $\R \setminus \{0\}$ is open, 
	and therefore $\text{GL}(n, \R)$ being the preimage of 
	$\det$ under $\R \setminus \{0\}$ is also open. Therefore, 
	$\text{GL}(n, \R)$ is a smooth submanifold of $M_{n\times n}(\R)$.
\end{solution}

\item
The special linear group $SL(2,\R) = \{ A \in M_{2 \times 2}(\R) : \det A = 1 \} \subset M_{2 \times 2} = \R^4$. 

\begin{solution}
	Note that $\det A = a_1 a_4 - a_2 a_3$ and so $\text{SL}(2, \R)$ 
	is the zero set of the function $F(a_1, a_2, a_3, a_4) = 
	a_1 a_4 - a_2 a_3 - 1$. Then $\text{SL}(2, \R)$ is a smooth 
	submanifold of $\R^4$: we see that $F$ is smooth and 
	for any $A \in \text{SL}(2, \R)$, we have 
	\[ DF(A) = (a_4, -a_3, -a_2, a_1). \] 
	Note that $A$ is not the zero matrix since $\det A = 1$, so 
	$DF(A)$ has rank $1$ for all $A \in \text{SL}(2, \R)$.
\end{solution}

\end{enumerate}


\item
Construct a smooth atlas for the following smooth submanifolds.
\begin{enumerate}
\item $\R^n$.

\begin{solution}
	The identity map on $\R^n$. 
\end{solution}

\item An open subset $U$ of $\R^n$.

\begin{solution}
	The identity map on $U$. 
\end{solution}

\item
The general linear group $GL(n,\R) = \{ A \in M_{n \times n}(\R) : \det A \neq 0 \} \subset M_{n \times n} = \R^{n^2}$. 

\begin{solution}
	The identity map on $\text{GL}(n, \R)$ since it is open.
\end{solution}

\item The graph of a function $F: U \subset \R^k \rightarrow \R^{n-k}$ of class $C^r$ for some $r \geq 1$.

\begin{solution}
	The map $U \subset \R^k \to \R^n$ defined by $x \mapsto (x, F(x))$.
\end{solution}

\item The cubic $\{ (x,y) \in \R^2 : x = y^3 \} \subset \R^2$.

\begin{solution}
	The map $\R \to \R^2$ defined by $x \mapsto (x, x^{1/3})$.
\end{solution}

\item The hyperbola $\{ (x,y) \in \R^2 : xy = 1 \} \subset \R^2$.

\begin{solution}
	The map $\R \setminus \{0\} \to \R^2$ defined by $x \mapsto 
	(x, 1/x)$. 
\end{solution}

\item The unit circle $S^1 = \{ (x,y) \in \R^2: x^2+y^2=1 \} \subset \R^2$.

\begin{solution}
	The maps $\alpha_1 : (0, 2\pi) \subset \R \to S^1 \setminus \{(1, 0)\}$ 
	and $\alpha_2 : (-\pi, \pi) \subset \R \to S^1 \setminus \{(-1, 0)\}$, 
	both defined by $t \mapsto (\cos t, \sin t)$. 
\end{solution}

\item
The twisted cubic $\{ (x,y,z) \in \mathbb{R}^3: y = x^2, z = x^3 \} \subset \mathbb{R}^3$.

\begin{solution}
	The map $t \mapsto (t, t^2, t^3)$.
\end{solution}

\item The plane $\{ (x,y,z) \in \R^3: x+y+z = 1 \} \subset \R^3$.

\begin{solution}
	This is the graph of the function $f(x, y) = 1-x-y$, so we can take 
	the chart $(x, y) \mapsto (x, y, 1-x-y)$. 
\end{solution}

\item The paraboloid $\{ (x,y,z) \in \R^3: y = x^2+z^2+1 \} \subset \R^3$.

\begin{solution}
	This is the graph of the function $f(x, z) = x^2 + z^2 + 1$, 
	so we can take the chart $(x, z) \mapsto (x, x^2 + z^2 + 1, z)$. 
\end{solution}

\item The open unit ball $\{ (x,y,z) \in \R^3: x^2+y^2+z^2 < 1 \} \subset \R^3$.

\begin{solution}
	Note that $B_1((0, 0, 0)) \subset \R^3$ is open, so we can take the identity map.
\end{solution}
\end{enumerate}

\item
One version of the {\em Invariance of Dimension Theorem} states that if $m>n$ and $U$ is an open subset of $\R^m$, then there exists no continuous injective mapping from $U$ to $\R^n$. Use this to prove the following:
\begin{enumerate}
\item $\R^m$ is homeomorphic to $\R^n$ if and only if $m=n$.

\begin{solution}
	We see that $m = n$ implies $\R^m = \R^n$, so there is 
	nothing to prove. Conversely, suppose that 
	$\R^m$ is homeomorphic to $\R^n$. Then there is a continuous injective mapping 
	$f : \R^m \to \R^n$, so $m \leq n$ by the Invariance of Dimension Theorem. 
	Similarly, we see that 
	$f^{-1} : \R^n \to \R^m$ is a continuous injective mapping, so 
	$n \leq m$ and hence $m = n$. 
\end{solution}

\item If $M \subset \R^n$ is a $k$-dimensional submanifold of $\R^n$, then $k \leq n$.

\begin{solution}
	Let $M \subset \R^n$ be a $k$-dimensional submanifold of $\R^n$. 
	Then for all $p \in M$, there exists a neighbourhood $V$ of $p$, 
	an open set $U \subset \R^k$, and a homeomorphism 
	$\alpha : U \subset \R^k \to V \subset M \subset \R^n$. In particular, 
	we have that $\alpha$ is a continuous injective mapping from 
	$U$ to $\R^n$, so we must have $k \leq n$ by the Invariance of Dimension Theorem.
\end{solution}

\end{enumerate}
\end{enumerate}

\newpage 
\noindent
{\bf B. Velocity vectors, pushforwards, and tangents spaces.}

\begin{enumerate}
\item
Define the following:
\begin{enumerate}
\item
The velocity vector of a parametrised curve $\gamma: (\alpha,\beta) \rightarrow \R^n$ of class $C^r$, $r \geq 1$. 

\begin{solution}
	The velocity vector is $\gamma'(t)$.
\end{solution}

\item
The tangent space $T_{\vec{x}}(\R^n)$ to $\R^n$ at $\vec{x} \in \R^n$.

\begin{solution}
	The tangent space $T_{\vec x}(\R^n)$ to $\R^n$ at $\vec{x} \in \R^n$ is 
	\[ T_{\vec{x}}(\R^n) = \{(\vec x; \vec v) : \vec v \in \R^n\}. \] 
\end{solution}
\item
The pushforward $\alpha_*$ of a map $\alpha: U \rightarrow \R^n$ of class $C^r$, $r \geq 1$, where $U$ is an open subset of $\R^k$.

\begin{solution}
	The pushward $\alpha_*$ of the map $\alpha : U \to \R^n$ is defined by
	\begin{align*}
		\alpha_* : T_{\vec x}\R^k &\to T_p\R^n \\ 
		(\vec x; \vec v) &\mapsto (p; D\alpha(\vec x)\vec v), 
	\end{align*} 
	where $\vec x \in U$ and $p = \alpha(\vec x)$. 
\end{solution}

\item
The tangent space of $T_pM$ to a $k$-submanifold of $\R^n$ of class $C^r$ at $p \in M$.

\begin{solution}
	Let $\alpha : U \subset \R^k \to V \subset M \subset \R^n$ be a coordinate chart about $p$. The tangent space of $M$ at $p$ is defined as 
	\[ T_pM := \alpha_*(T_{x_0}\R^k) \subset T_p\R^n, \] 
	where $x_0 \in U$ is the unique point such that $p = \alpha(x_0)$. 
\end{solution}
\end{enumerate}

\item
Prove that, for all $\vec{x} \in \R^n$,
\[ T_{\vec{x}} \R^n = \{ (\vec{x};\vec{v}) : \mbox{ $\vec{v}$ is the velocity vector at $\vec{x}$ of a parametrised curve $\gamma$ passing through $\vec{x}$} \}.\]

\begin{solution}
	Let $\vec x \in \R^n$. For all $\vec v \in \R^n$, consider the 
	parametrized curve $\gamma(t) = \vec x + t\vec v$, where $t \in \R$. 
	Observe that $\gamma'(t) = \vec v$. In particular, at $t = 0$, we have 
	$\gamma(0) = \vec x$ and $\gamma'(0) = \vec v$, so $\vec v$ is the 
	velocity vector at $\vec x$.
\end{solution}

\item
Let $M$ be a $k$-dimensional submanifold of $\R^n$ of class $C^r$, $r \geq 1$. Show that $\dim_{\R} T_pM = k$ for all $p \in M$.

\begin{solution}
	Let $p \in M$. Note that $T_pM = \alpha_*(T_{x_0}\R^k)$ where $x_0 \in U$ is the 
	unique point such that $\alpha(x_0) = p$. We see that 
	$D\alpha$ has rank $k$ everywhere, so the image of the map
	\[ \alpha_*(x_0; \vec v) = (p; D\alpha(x_0)\vec v), \]
	namely $T_pM$, must be of dimension $k$. 
\end{solution}

\item Compute the tangent space at the given point.
\begin{enumerate}
\item 
$T_pU$ for any point $p$ in the open set $U \subset \R^n$.

\begin{solution}
	The identity map $\alpha : U \to U$ has $D\alpha(x) = I_{n\times n}$ 
	for all $x \in U$. By definition, we have $T_pU = \alpha_*(T_p\R^n)$ 
	for all $p \in U$ since $p$ is the unique point for which 
	$\alpha(p) = p$. Then for all $(p; v) \in T_p\R^n$, we see that  
	\[ \alpha_*(p; \vec v) = (p; D\alpha(p)\vec v) = (p; \vec v), \] 
	which implies that $T_pU = T_p\R^n$. 
\end{solution}
\item The graph of a function $f : U \subset \R^k \rightarrow \R^{n-k}$ of class $C^r$, where $U$ is an open subset of $\R^k$. Take any $p \in M$.

\begin{solution}
	Consider the $C^r$ map $\alpha : U \subset \R^k \to \R^{n-k}$ given by 
	$\vec x \mapsto (\vec x, f(\vec x))$. We have that 
	\[ D\alpha(\vec x) = \left[ \begin{array}{c}
		I_{k\times k} \\ \hline 
		Df(\vec x)
	\end{array} \right], \] 
	where $Df(\vec x)$ is an $(n-k) \times k$ matrix for all $\vec x \in U$. Now, let $p \in M$ 
	so that $p = \alpha(\vec x_0) = (\vec x_0, f(\vec x_0))$ for some $\vec x_0 \in U$. 
	Then we find that 
	\[ T_pM = \text{span}_{\R} \left\{ \left[ \begin{array}{c} 
		\vec e_i \\ \hline \partial f(\vec x_0)/\partial x_i \end{array} \right] 
		: i = 1, \dots, k \right\}, \] 
	where $\vec e_1, \dots, \vec e_k$ is the standard basis for $\R^k$. 
	In particular, these are the columns of $D\alpha(\vec x_0)$. 
\end{solution}

\item The zero set $M$ of a function $F : U \subset \R^n \rightarrow \R^{n-k}$ of class $C^r$, where $U$ is an open set in $\R^n$, such that $DF(p)$ has rank $n-k$ for all $p \in M$. Take any $p \in M$.

\begin{solution}
	Here, we use Problem 1 of Assignment 3. Let $M$ be a 
	$k$-dimensional submanifold of $\R^n$ of class $C^r$ with $r \geq 1$ 
	and let $p \in M$. Then $(p; \vec v)$ is a tangent vector to $M$ at $p$ 
	if and only if there exists a parametrized curve $\gamma : (-\varepsilon, 
	\varepsilon) \to \R^n$ of class $C^r$ whose image lies in $M$ 
	and is such that $(p; \vec v) = (\gamma(0); \gamma'(0))$. 

	We now show that $T_pM = \{(p; \vec v) \in M \times \R^n : DF(p)\vec v = 
	\vec 0\} = \ker(DF(p))$. Let $(p; \vec v) \in T_pM$. Then 
	there exists a curve $\gamma : (-\varepsilon, \varepsilon) \to \R^n$ 
	of class $C^r$ on $M$ such that $(p; \vec v) = (\gamma(0); \gamma'(0))$. 
	We have $F(\gamma(t)) = \vec 0$ for all $t \in (-\varepsilon, \varepsilon)$, 
	so by the chain rule, we have 
	\[ DF(p)\vec v = DF(\gamma(0))\gamma'(0) = \frac{\rm d}{{\rm d}t} F(\gamma(t)) 
	\Big|_{t=0} = 0. \] 
	Then $T_pM \subset \ker(DF(p))$. But $T_pM$ has dimension $k$. 
	Since $DF(p)$ has rank $n-k$ for all $p \in M$, we see that 
	$\ker(DF(p))$ is a $k$-dimensional subspace of $\R^n$. Since 
	$T_pM$ is a subspace of $\ker(DF(p))$ with both of dimension $k$, 
	this implies that they are equal, as desired. 
\end{solution}

\item The sphere $S^n$ in $\R^{n+1}$ at any point in $S^n$.

\begin{solution}
	Note that $S^n$ is the zero set of the function 
	$F(x_1, \dots, x_{n+1}) = x_1^2 + \cdots + x_{n+1}^2 - 1$ 
	with $DF(x_1, \dots, x_{n+1}) = (2x_1, \dots, 2x_{n+1})$. 
	Let $\vec x = (x_1, \dots, x_{n+1}) \in S^1$. Then we have 
	\begin{align*}
		T_{\vec{x}}S^n &= \ker(DF(\vec x)) \\ 
		&= \{\vec v \in \R^{n+1} : (2x_1, \dots, 2x_{n+1}) \cdot \vec v = 0\} \\ 
		&= \{\vec v \in \R^{n+1} : x_1v_1 + \cdots + x_{n+1}v_{n+1} = 0\}. 
	\end{align*}
\end{solution}

\item $T_I GL(n, \R)$ where $I$ is the identity $n \times n$ matrix.

\begin{solution}
	We recall that $\text{GL}(n, \R)$ is open in $M_{n\times n}(\R)$, 
	so using (a), we have that 
	\[ T_I \text{GL}(n, \R) = T_I M_{n\times n}(\R) \simeq M_{n\times n}(\R). \] 
\end{solution}

\item $T_ISL(2, \R)$ where $I$ is the identity $2 \times 2$ matrix.

\begin{solution}
	We see that $\text{SL}(2, \R)$ is the zero set of the map 
	$F(a_1, a_2, a_3, a_4) = a_1a_4 - a_2a_3 - 1$. Then $DF(a_1, 
	a_2, a_3, a_4) = (a_4, -a_3, -a_2, a_1)$
	with $DF(I) = (1, 0, 0, 1)$. 
	Applying (c) yields 
	\begin{align*}
		T_I \text{SL}(2, \R) 
		&= \ker(DF(I)) \\ 
		&= \{B \in M_{2\times 2}(\R) : (1, 0, 0, 1) \cdot (b_1, b_2, 
		b_3, b_4) = 0\} \\ 
		&= \{B \in M_{2\times 2}(\R) : b_1 + b_4 = 0\} \\ 
		&= \{B \in M_{2\times 2}(\R) : \tr B = 0\}. 
	\end{align*}
\end{solution}
\end{enumerate}

\end{enumerate}

\newpage 
\noindent
{\bf C. Curves.}

\begin{enumerate}
\item Let $\gamma: (\alpha,\beta) \rightarrow \R$ be a parametrised curve of class $C^r$, $r \geq 3$. Define the following:
\begin{enumerate}
\item The velocity, speed and acceleration of $\gamma$ at $t \in (\alpha,\beta)$.

\begin{solution}
	The velocity of $\gamma$ at $t \in (\alpha, \beta)$ is $\gamma'(t)$. 
	The speed is $\|\gamma'(t)\|$ and the acceleration is $\gamma''(t)$. 
\end{solution}

\item The arclength of $\gamma$ starting that the point $\gamma(t_0)$ for any $t_0 \in (\alpha,\beta)$.

\begin{solution}
	The arclength of $\gamma$ starting at the point $\gamma(t_0)$ is 
	\[ s(t) := \int_{t_0}^t \|\gamma'(t)\|\,{\rm d}t. \] 
\end{solution}

\item The curve $\gamma$ is regular.

\begin{solution}
	A curve $\gamma$ is regular if $\gamma'(t) \neq \vec0$ for all $t \in (\alpha, \beta)$.
\end{solution}

\item The curve $\gamma$ is unit-speed.

\begin{solution}
	A curve $\gamma$ is unit-speed if $\|\gamma'(t)\| = 1$ for all $t \in (\alpha, \beta)$.
\end{solution}

\item The curvature $\kappa(t)$ of $\gamma$ at $\gamma(t)$ for all $t \in (\alpha,\beta)$ (whether the parametrisation is unit-speed or just regular).

\begin{solution}
	When $\gamma$ is unit-speed, we have $\kappa(t) = \|\gamma''(t)\|$. 
	If $\gamma$ is regular, we have 
	\[ \kappa(t) = \frac{\|\gamma'(t) \times \gamma''(t)\|}{\|\gamma'(t)\|^3}. \] 
\end{solution}

\item The torsion $\tau(t)$ at points where $\kappa(t) > 0$ (whether the parametrisation is unit-speed or just regular).

\begin{solution}
	When $\gamma$ is unit-speed, we define $\tau : (\alpha, \beta) 
	\to \R$ to be the scalar function such that $\frac{{\rm d}}{{\rm d}t} 
	B(t) = -\tau(t)N(t)$. In the general case where $\gamma$ is regular, we have 
	\[ \tau(t) = \frac{(\gamma'(t) \times \gamma''(t)) \cdot \gamma'''(t)}
	{\|\gamma'(t) \times \gamma''(t)\|^2}. \] 
\end{solution}

\item The Frenet frame of $\gamma$ (whether the parametrisation is unit-speed or just regular).

\begin{solution}
	We define the Frenet frame of $\gamma$ to be $\{T, N, B\}$, where 
	$T$ is the unit tangent vector, $N$ is the principal unit normal, 
	and $B$ is the unit binormal. 
	When $\gamma$ is unit-speed, we have $T(t) = \gamma'(t)$, 
	$N(t) = \gamma''(t)/\kappa(t) = \gamma''(t)/\|\gamma''(t)\|$, 
	and $B(t) = T(t) \times N(t)$. When $\gamma$ is regular, we have 
	\begin{align*}
		T(t) &= \frac{\gamma'(t)}{\|\gamma'(t)\|}, \\ 
		B(t) &= \frac{\gamma'(t) \times \gamma''(t)}{\|\gamma'(t) \times \gamma''(t)\|}, \\ 
		N(t) &= B(t) \times T(t). 
	\end{align*}
\end{solution}

\item Frenet-Serret equations for unit-speed curves.

\begin{solution}
	The Frenet-Serret equations are $\frac{{\rm d}T}{{\rm d}s} 
	= \kappa T$, $\frac{{\rm d}N}{{\rm d}s} = -\kappa T + \tau B$, and 
	$\frac{{\rm d}B}{{\rm d}s} = -\tau N$. More compactly, they 
	can be written as the matrix equation 
	\[ \begin{bmatrix}
		\frac{{\rm d}T}{{\rm d}s} \\[5pt] 
		\frac{{\rm d}N}{{\rm d}s} \\[5pt]
		\frac{{\rm d}B}{{\rm d}s}
	\end{bmatrix} = \begin{bmatrix}
		0 & \kappa & 0 \\ 
		-\kappa & 0 & \tau \\ 
		0 & -\tau & 0  
	\end{bmatrix} \begin{bmatrix}
		T \\ N \\ B 
	\end{bmatrix}. \]
\end{solution}
 \end{enumerate}

\item Let $u: (\alpha,\beta) \rightarrow \R^n, t \mapsto u(t)$, be a vector-valued function of class $C^r$, $r \geq 1$, such that $\| u(t) \| = c$ for all $t \in (\alpha,\beta)$, for some fixed $c \in \R^{>0}$. Show that $u(t) \cdot \dot{u}(t) = 0$ for all $t \in (\alpha,\beta)$. In particular, if $\gamma$ is a unit-speed curve, then $\gamma''(t)$ is either zero or perpendicular to $\gamma'(t)$ for all $t \in (\alpha,\beta)$. 

\begin{solution}
	Note that $u(t) \cdot u(t) = \|u(t)\|^2 = c^2$ for all $t \in (\alpha, \beta)$. 
	Then 
	\begin{align*}
		0 = \frac{\rm d}{{\rm d}t} (u(t) \cdot u(t)) 
		= u'(t) \cdot u(t) + u(t) \cdot u'(t) = 2u(t) \cdot u'(t), 
	\end{align*}
	and therefore $u(t) \cdot u'(t) = 0$ for all $t \in (\alpha, \beta)$. 
	For a unit-speed curve $\gamma$, we can set $u(t) = \gamma'(t)$ so that 
	$\gamma'(t) \cdot \gamma''(t) = 0$ for all $t \in (\alpha, \beta)$. 
	This implies that $\gamma''(t)$ is either zero or perpendicular to 
	$\gamma'(t)$ for all $t \in (\alpha, \beta)$. 
\end{solution}

\item Prove that any regular curve $\gamma: (\alpha,\beta) \rightarrow \R$ of class $C^r$, $r \geq 1$, can be reparametrised in terms of arclength.

\begin{solution}
	Note that the arclength function $s$ of $\gamma$ is smooth with 
	$\frac{{\rm d}s}{{\rm d}t} = \|\gamma'(t)\| > 0$ for all $t 
	\in (\alpha, \beta)$ since $\gamma$ is regular. Then $s$ is strictly 
	increasing on $(\alpha, \beta)$ and its inverse is smooth by the 
	Inverse Function Theorem. Then $s = s(t)$ and $t = t(s)$ (where 
	$t = s^{-1}$) are both smooth on their domains, and one can 
	parametrize $s$ in terms of arclength via 
	\[ \gamma(t) = \gamma(t(s)) = \tilde\gamma(s), \] 
	where we define $\tilde\gamma = \gamma \circ t$. 
\end{solution}

\item Compute the curvature and torsion of the circular helix $\gamma(t) = (4\cos t,4\sin t, 3t), t \in \R$.

\begin{solution}
	Note that $\gamma'(t) = (-4\sin t, 4\cos t, 3)$ with 
	$\|\gamma'(t)\| = 5$, so $\gamma$ is regular but not unit-speed. 
	Next, we have $\gamma''(t) = (-4\cos t, -4\sin t, 0)$ so that 
	\[ \gamma'(t) \times \gamma''(t) = 
	(-4\sin t, 4\cos t, 3) \times (-4\cos t, -4\sin t, 0) 
	= (12\sin t, -12\cos t, 16). \] 
	This gives us $\|\gamma'(t) \times \gamma''(t)\| = 20$, so the 
	curvature of $\gamma$ at $\gamma(t)$ is 
	\[ \kappa(t) = \frac{\|\gamma'(t) \times \gamma''(t)\|}{\|\gamma'(t)\|^3} 
	= \frac{20}{5^3} = \frac{4}{25}. \] 
	Next, we have $\gamma'''(t) = (4\sin t, -4\cos t, 0)$ so that 
	\[ (\gamma'(t) \times \gamma''(t)) \cdot \gamma'''(t) 
	= (12\sin t, -12\cos t, 16) \cdot (4\sin t, -4\cos t, 0) = 48. \] 
	It follows that the torsion of $\gamma$ at $\gamma(t)$ is 
	\[ \tau(t) = \frac{(\gamma'(t) \times \gamma''(t)) \cdot \gamma'''(t)}
	{\|\gamma'(t) \times \gamma''(t)\|^2} = \frac{48}{20^2} = \frac{3}{25}. \] 
\end{solution}

\item Consider the following unit-speed curve $$\displaystyle \gamma(s) = 
\left( \frac{1}{3}(1+s)^{3/2},\frac{1}{3}(1-s)^{3/2},\frac{s}{\sqrt{2}}\right), s\in (-1,1).$$
Find the Frenet frame $\{ T,N,B \}$ and the compute the curvature and the torsion at a general point on the curve. 

\begin{solution}
	First, the unit tangent vector to $\gamma$ at $\gamma(s)$ is
	\[ T(s) = \gamma'(s) = \left( \frac12(1+s)^{1/2}, -\frac12(1-s)^{1/2}, 
	\frac{1}{\sqrt{2}} \right). \] 
	Next, we see that 
	\[ \gamma''(s) = \left( \frac14(1+s)^{-1/2}, \frac14(1-s)^{1/2}, 0 \right), \] 
	so the curvature is 
	\[ \kappa(s) = \|\gamma''(s)\| = \frac{1}{\sqrt{8(1-s^2)}} > 0. \] 
	Then the principal unit normal is 
	\[ N(s) = \frac{\gamma''(s)}{\kappa(s)} = \frac{1}{\sqrt{2}} 
	\left( (1-s)^{1/2}, (1+s)^{1/2}, 0 \right). \] 
	The unit binormal is then 
	\[ B(s) = T(s) \times N(s) = \left( -\frac12(1+s)^{1/2}, 
	\frac12(1-s)^{1/2}, \frac{1}{\sqrt{2}} \right). \] 
	Note that 
	\[ \frac{{\rm d}}{{\rm d}s} B(s) = 
	\left( -\frac14(1+s)^{-1/2}, -\frac14(1+s)^{-1/2}, 0 \right) 
	= -\gamma''(s) = -\kappa(s) N(s), \] 
	which implies that the torsion is 
	\[ \tau(s) = \kappa(s) = \frac{1}{\sqrt{8(1-s^2)}}. \] 
\end{solution}

\item
Let $\gamma(t)$, $t \in (\alpha,\beta)$, be a smooth regular curve in $\mathbb{R}^3$. Prove that $\gamma$ parametrises a straight line if and only if $\kappa(t) = 0$ for all $t \in (\alpha,\beta)$. 

\begin{solution}
	Without loss of generality, we assume that $\gamma$ is parametrized 
	using arclength so that it is unit-speed. 
	Suppose that $\gamma(t) = \vec{x} + t\vec{v}$ is a straight line in $\R^3$. 
	Then we have $\gamma'(t) = \vec{v}$ and $\gamma''(t) = \vec 0$ 
	so that $\kappa(t) = \|\gamma''(t)\| = 0$ for all $t \in (\alpha, \beta)$. 
	All of these steps are in fact reversible, so we are done. 
\end{solution}

\item Let $\gamma(t)$ be a smooth regular curve in $\mathbb{R}^3$. Prove that $\gamma'(t)$ and $\gamma''(t)$ are linearly dependent (that is, $\gamma''(t) = c(t)\gamma'(t)$ for some smooth function $c$) if and only if $\gamma(t)$ parametrises a straight line.

\begin{solution}
	Suppose that $\gamma''(t) = c(t)\gamma'(t)$
	for some smooth scalar function $c$. Then 
	$\gamma'(t) \times \gamma''(t) = \vec 0$ for all $t \in 
	(\alpha, \beta)$ (since they are parallel), which implies that 
	\[ \kappa(t) = \frac{\|\gamma'(t) \times \gamma''(t)\|}{\|\gamma'(t)\|^3} 
	= \frac{0}{\|\gamma'(t)\|^3} = 0 \] 
	for all $t \in (\alpha, \beta)$. Applying the previous problem, 
	this means that $\gamma$ parametrizes a straight line. 
	These steps are reversible and thus the converse holds.
\end{solution}

\item
Let $\gamma(t)$, $t \in (\alpha,\beta)$, be a smooth regular curve in $\mathbb{R}^3$ such that $\kappa(t) > 0$ for all $t \in (\alpha,\beta)$. Prove that $\gamma(t)$ is a plane curve if and only if its torsion $\tau(t)$ is equal to 0 at every $t \in (\alpha,\beta)$. Moreover, if it is a plane curve, then it is contained in the plane passing through
$\gamma(t_0)$ with normal vector $B(t_0)$ for any $t_0 \in (\alpha,\beta)$.

\begin{solution}
	$(\Rightarrow)$ Suppose that $\gamma(t)$ is a plane curve. Let 
	$t_0 \in (\alpha, \beta)$. Note that $\gamma$ is a plane curve 
	if and only if every point $\gamma(t)$ is contained in some fixed plane 
	$\Pi$ in $\R^3$. Then $\gamma(t_0) \in \Pi$. If $\mathbf n_0$ is a 
	unit normal to $\Pi$, then for any $t \in (\alpha, \beta)$, the 
	vector going from $\gamma(t_0)$ to $\gamma(t)$ is parallel to $\Pi$ 
	and thus perpendicular to $\mathbf n_0$. This implies that 
	$\mathbf n_0 \cdot (\gamma(t) - \gamma(t_0)) = 0$, or equivalently 
	$\mathbf n_0 \cdot \gamma(t) = \mathbf n_0 \cdot \gamma(t_0)$ for all 
	$t \in (\alpha, \beta)$. The right-hand side is a constant, so we have 
	\[ 0 = \frac{\rm d}{{\rm d}t} (\mathbf n_0 \cdot \gamma(t)) 
	= \frac{{\rm d}}{{\rm d}t} (\mathbf n_0) \cdot \gamma(t) 
	+ \mathbf n_0 \cdot \gamma'(t) = \mathbf n_0 \cdot T(t). \]
	Similarly, we have $\mathbf n_0 \cdot \gamma''(t) = 0$ with 
	$\gamma''(t) = \kappa(t) N(t)$ for all $t \in (\alpha, \beta)$, so 
	\[ \mathbf n_0 \cdot N(t) = 0. \]
	Then $\mathbf n_0 \perp T(t)$ and $\mathbf n_0 \perp N(t)$ implies that 
	$\mathbf n_0 = \pm T(t) \times N(t) = \pm B(t)$. In particular, we have 
	\[ \frac{{\rm d}B}{{\rm d}t}(t) = \vec 0 \] 
	for all $t \in (\alpha, \beta)$, which implies that $\tau(t) = 0$ 
	for all $t \in (\alpha, \beta)$ since $\frac{{\rm d}B}{{\rm d}t}(t) 
	= -\tau(t)N(t)$ by the Frenet-Serret equations with $N(t) \neq \vec 0$ 
	for all $t \in (\alpha, \beta)$. 

	$(\Leftarrow)$ Suppose that $\tau(t) = 0$ for all $t \in (\alpha, \beta)$. 
	Then $\frac{{\rm d}B}{{\rm d}t}(t) = \vec 0$ and hence $B(t) = 
	\mathbf n_0$ for all $t \in (\alpha, \beta)$ for some fixed 
	$\mathbf n_0 \in \R^3$. We have $T(t) \perp \mathbf n_0$ for all 
	$t \in (\alpha, \beta)$, which implies that $T(t) \cdot \mathbf n_0 = 0$. 
	But $T(t) = \gamma'(t)$, and the product rule gives 
	\[ \frac{{\rm d}}{{\rm d}t} (\gamma(t) \cdot \mathbf n_0) 
	= \gamma'(t) \cdot \mathbf n_0 + \gamma(t) \cdot \frac{\rm d}{{\rm d}t}
	(\mathbf n_0) = T(t) \cdot \mathbf n_0 = 0 \] 
	since $\frac{{\rm d}}{{\rm d}t}(\mathbf n_0) = \vec 0$ and 
	$T(t) \cdot \mathbf n_0 = 0$. It follows that $\gamma(t) \cdot \mathbf n_0 
	= c$ for some constant $c \in \R$. Fixing $t_0 \in (\alpha, \beta)$, 
	we observe that 
	\[ (\gamma(t) - \gamma(t_0)) \cdot \mathbf n_0 = \gamma(t) \cdot 
	\mathbf n_0 - \gamma(t_0) \cdot \mathbf n_0 = c - c = 0 \] 
	for all $t \in (\alpha, \beta)$. This means that $\gamma$ is 
	contained in the plane passing through $\gamma(t_0)$ normal to 
	$B(t_0) = \mathbf n_0$, and $\gamma$ is a plane curve. 
\end{solution}

\item
Show that $\gamma(t) = (t,1+t^{-1},t^{-1}-t)$, $t>0$, is a plane curve and find the equation of the plane in which it lies. 

\begin{solution}
	By the previous problem, it is enough to show that $\tau(t) = 0$ for all 
	$t > 0$. We have $\gamma'(t) = (1, -t^{-2}, -t^{-2} - 1)$,
	$\gamma''(t) = 2t^{-3}(0, 1, 1)$, and $\gamma'''(t) = -6t^{-4}(0, 1, 1)$. 
	This gives us $\gamma'(t) \times \gamma''(t) = 2t^{-3}(1, -1, 1)$ 
	and $(\gamma'(t) \times \gamma''(t)) \cdot \gamma'''(t) 
	= 0$, which implies that $\tau(t) = 0$ for all $t > 0$. Therefore, 
	$\gamma$ is a plane curve. 

	To determine the equation of the plane, we solve for $a, b, c, d \in \R$
	such that 
	\[ at + b(1+t^{-1}) + c(t^{-1} - t) = d. \] 
	This gives us $(b-d) + (a-c)t + (b+c)t^{-1} = 0$, so 
	we have $b = d$, $a = c$, and $b = -c$. Setting $c = 1$, 
	we have that $a = 1$ and $b = d = -1$. Then $\gamma$ is 
	contained in the plane $x - y + z = -1$. 

	Alternatively, the unit binormal vector is 
	\[ B(t) = \frac{\gamma'(t) \times \gamma''(t)}{\|\gamma'(t) \times \gamma''(t)\|}
	= \frac{1}{\sqrt{3}}(1, -1, 1) \] 
	for all $t \in (\alpha, \beta)$, which is a normal vector to the plane 
	that $\gamma$ lies in. Since $\gamma(1) = (1, 2, 0)$ is a point on the curve 
	and therefore the plane, we have that 
	\[ (x-1, y-2, z) \cdot (1, -1, 1) = 0. \] 
	This gives the same plane $x - y + z = -1$. 
\end{solution}

\item
State the Fundamental Theorem of Space Curves.

\begin{solution}
	A rigid motion of $\R^3$ is a rotation followed by a translation. 
	That is, a rigid motion of $\R^3$ is an affine map $M : \R^3 \to \R^3$ 
	of the form $M(v) = Av + b$ where $A$ is an element of the 
	special orthogonal group 
	$\text{SO}(3) = \{C \in M_{3\times 3}(\R) : C^T C = I_{3\times 3},\, 
	\det C = 1\}$. 

	The Fundamental Theorem of Space Curves states that given two 
	unit-speed curves $\gamma_1$ and $\gamma_2$ with the same curvature 
	$\kappa(s)$ and the same torsion $\tau(s)$ for all $s \in (\alpha, \beta)$, 
	there exists a rigid motion $M$ of $\R^3$ such that $\gamma_2(s) = M(\gamma_1(s))$
	for all $s \in (\alpha, \beta)$. Moreover, if $k$ and $t$ are functions 
	of class $C^3$ with $k > 0$ everywhere, then there exists a unit-speed curve 
	in $\R^3$ whose curvature is $k$ and whose torsion is $t$. 
\end{solution}

\item
Show that $\gamma(t)=(2+\sqrt{2}\cos t, 1 - \sin t, 3 + \sin t)$, $t \in \mathbb{R}$, is a circle. 

\begin{solution}
	By the Fundamental Theorem of Space Curves, every circle has 
	constant (positive) curvature and zero torsion. We verify that 
	this is true for $\gamma$. Note that $\gamma'(t) = 
	(-\sqrt{2} \sin t, -\cos t, \cos t)$,
	$\gamma''(t) = (-\sqrt{2}\cos t, \sin t, -\sin t)$, and 
	$\gamma'''(t) = (\sqrt{2}\sin t, \cos t, -\cos t)$. 
	We have that $\|\gamma'(t)\| = \sqrt{2}$ and 
	$\gamma'(t) \times \gamma''(t) = (0, -\sqrt{2}, -\sqrt{2})$
	so that $\|\gamma'(t) \times \gamma''(t)\| = 2$ and 
	$(\gamma'(t) \times \gamma''(t)) \cdot \gamma'''(t) = 0$. 
	This implies that the curvature is 
	\[ \kappa(t) = \frac{\|\gamma'(t) \times \gamma''(t)\|}{\|\gamma'(t)\|^3} 
	= \frac{2}{(\sqrt{2})^3} = \frac{1}{\sqrt{2}} > 0, \] 
	which is constant, and the torsion is 
	\[ \tau(t) = \frac{(\gamma'(t) \times \gamma''(t)) \times \gamma'''(t)}
	{\|\gamma'(t) \times \gamma''(t)\|^2} = 0. \]
\end{solution}

\end{enumerate}

\newpage 
\noindent
{\bf D. Surfaces.}

\begin{enumerate}
\item
Let $S$ be a smooth surface in $\R^3$ and $\sigma: U \subset \R^2 \rightarrow V \subset S$ be a smooth coordinate chart. Define the following:
\begin{enumerate}
\item
The standard unit normal, first and second fundamental forms of $\sigma$.

\begin{solution}
	Let $p_0 = \sigma(u_0, v_0)$ for some $(u_0, v_0) \in U$. 
	The standard unit normal is
	\[ N_\sigma(u_0, v_0) = \frac{\sigma_u(u_0, v_0) \times \sigma_v(u_0, v_0)}
	{\|\sigma_u(u_0, v_0) \times \sigma_v(u_0, v_0)\|}. \] 
	The first fundamental form of $\sigma$ is 
	\[ {\cal F}_{\text{I}} = \begin{bmatrix}
		\sigma_u \cdot \sigma_u & \sigma_u \cdot \sigma_v \\ 
		\sigma_v \cdot \sigma_u & \sigma_v \cdot \sigma_v
	\end{bmatrix}, \] 
	and the second fundamental form of $\sigma$ is 
	\[ {\cal F}_{\text{II}} = \begin{bmatrix}
		\sigma_{uu} \cdot N_\sigma & \sigma_{uv} \cdot N_\sigma \\ 
		\sigma_{vu} \cdot N_\sigma & \sigma_{vv} \cdot N_\sigma
	\end{bmatrix}. \] 
\end{solution}
\item
The principal, normal and geodesic curvatures of $\sigma$.

\begin{solution}
	The principal curvatures of $\sigma$ are the eigenvalues $\kappa_1$ 
	and $\kappa_2$ of the shape operator ${\cal W} := {\cal F}_{\text{I}}^{-1} 
	{\cal F}_{\text{II}}$. For a unit-speed curve $\gamma : (\alpha, \beta) 
	\to V \subset S$ contained in the coordinate patch $\sigma$ and a 
	point $p_0 = \gamma(t_0)$, the normal curvature is
	\[ \kappa_n := \gamma''(t_0) \cdot N_\sigma, \] 
	whereas the geodesic curvature is
	\[ \kappa_g := \gamma''(t_0) \cdot (N_\sigma \times \gamma'(t_0)). \] 
\end{solution}
\item 
The Gaussian and mean curvatures of $\sigma$.

\begin{solution}
	For the shape operator ${\cal W} = {\cal F}_{\text{I}}^{-1} 
	{\cal F}_{\text{II}}$, the Gaussian curvature of $\sigma$ is 
	$K := \det {\cal W}$ and the mean curvature of $\sigma$ is 
	$H := \frac12 \tr{\cal W}$. 
\end{solution}
\item
The geodesics of $\sigma$.

\begin{solution}
	We call a unit-speed curve $\gamma : (\alpha, \beta) \to V \subset S$ 
	that is contained in the coordinate patch $\sigma$ a geodesic 
	if $\kappa_g = 0$ everywhere. Equivalently, we have that 
	$\gamma''(t) = \vec 0$ or $\gamma''(t) \parallel N_\sigma$ for all 
	$t \in (\alpha, \beta)$. 
\end{solution}

\end{enumerate}

\item
Let $\sigma: U \subset \R^2 \rightarrow V \subset S$ and $\tilde{\sigma}: \tilde{U} \subset \R^2 \rightarrow \tilde{V} \subset S$ be smooth coordinate charts of a surface $S$ with $V \cap \tilde{V}$. Set $\Phi := \sigma^{-1} \circ \tilde{\sigma}: \tilde{\sigma}^{-1}(V \cap \tilde{V}) \rightarrow \sigma^{-1}(V \cap \tilde{V})$ so that $\tilde{\sigma} = \sigma \circ \Phi$. 
\begin{enumerate}
\item Show that $N_{\tilde{\sigma}} = \pm N_\sigma$, where $\pm$ is the sign of $\det(D \Phi)$.

\begin{solution}
	Note that $\Phi$ is a smooth diffeomorphism with components 
	\[ (\tilde u, \tilde v) \mapsto (u(\tilde u, \tilde v), v(\tilde u, \tilde v)). \] 
	By the chain rule, we have $D\tilde\sigma = D\sigma D\Phi$, or more explicitly 
	\[ \left[ \begin{array}{c|c} 
		\!\!\!\tilde\sigma_{\tilde u} & \tilde\sigma_{\tilde v}\!\!\!
	\end{array} \right] = \left[ \begin{array}{c|c} 
		\!\!\!\sigma_u & \sigma_v\!\!\!
	\end{array} \right] D\Phi. \]
	Here, $D\Phi$ is the change of basis matrix of $T_pS$. Note that 
	\[ D\Phi = \begin{bmatrix}
		\partial u/\partial\tilde u & \partial u/\partial\tilde v \\ 
		\partial v/\partial\tilde u & \partial v/\partial\tilde v 
	\end{bmatrix}. \] 
	Moreover, we have 
	\begin{align*}
		\tilde\sigma_{\tilde u} &= \sigma_u \cdot \frac{\partial u}{\partial \tilde u} 
		+ \sigma_v \cdot \frac{\partial v}{\partial\tilde u}, \\ 
		\tilde\sigma_{\tilde v} &= \sigma_u \cdot \frac{\partial u}{\partial \tilde v} 
		+ \sigma_v \cdot \frac{\partial v}{\partial\tilde v},
	\end{align*}
	which implies that 
	\begin{align*}
		\tilde\sigma_{\tilde u} \times \tilde\sigma_{\tilde v} 
		&= \left( \sigma_u \cdot \frac{\partial u}{\partial \tilde u} 
		+ \sigma_v \cdot \frac{\partial v}{\partial\tilde u} \right) 
		\times \left( \sigma_u \cdot \frac{\partial u}{\partial \tilde v} 
		+ \sigma_v \cdot \frac{\partial v}{\partial\tilde v} \right) \\ 
		&= (\sigma_u \times \sigma_u) \left( \frac{\partial u}{\partial\tilde u} 
		\frac{\partial v}{\partial\tilde v} \right) + (\sigma_u \times \sigma_v) 
		\left( \frac{\partial u}{\partial\tilde u} \frac{\partial v}{\partial\tilde v}
		- \frac{\partial v}{\partial\tilde u} \frac{\partial u}{\partial\tilde v} \right) 
		+ (\sigma_v \times \sigma_v) \left( \frac{\partial v}{\partial\tilde u}
		\frac{\partial v}{\partial\tilde v} \right) \\ 
		&= \left( \frac{\partial u}{\partial\tilde u} \frac{\partial v}{\partial\tilde v}
		- \frac{\partial v}{\partial\tilde u} \frac{\partial u}{\partial\tilde v} \right)
		(\sigma_u \times \sigma_v) = (\det D\Phi)(\sigma_u \times \sigma_v), 
	\end{align*}
	where we used the fact that $\sigma_u \times \sigma_u = 
	\sigma_v \times \sigma_v = \vec 0$. It follows that 
	\[ N_{\tilde\sigma} = \frac{\tilde\sigma_{\tilde u} \times \tilde\sigma_{\tilde v}}
	{\|\tilde\sigma_{\tilde u} \times \tilde\sigma_{\tilde v}\|} 
	= \frac{\det D\Phi}{\lvert\det D\Phi\rvert} \frac{\sigma_u \times \sigma_v}
	{\|\sigma_u \times \sigma_v\|} = \pm N_\sigma, \] 
	where $\pm$ is the sign of $\det D\Phi$. 
\end{solution}

\item
Recall that if $\mathcal{F}_I, \mathcal{F}_{II}$ and $\widetilde{\mathcal{F}}_I, \widetilde{\mathcal{F}}_{II}$ are the first and second fundamental forms of $\sigma$ and $\tilde{\sigma}$, respectively, then
\[ \widetilde{\mathcal{F}}_I = (D \Phi)^T \mathcal{F}_I (D \Phi)\]
and
\[ \widetilde{\mathcal{F}}_{II} = \pm (D \Phi)^T \mathcal{F}_{II} (D \Phi)\]
where $\pm$ is the sign of $\det(D \Phi)$.
\begin{enumerate}
\item
Let $\mathcal{W}$ and $\tilde{\mathcal{W}}$ be the shape operators associated to $\sigma$ and $\tilde{\sigma}$ respectively. Show that $\tilde{\mathcal{W}} = \pm (D \Phi)^{-1}\mathcal{W} (D \Phi)$, where $\pm$ is the sign of $\det(D \Phi)$.

\begin{solution}
	Observe that 
	\begin{align*}
		\tilde{\cal W} = \tilde{\cal F}_{\text{I}}^{-1} \tilde{\cal F}_{\text{II}} 
		&= \pm [(D\Phi)^T {\cal F}_{\text{I}} (D\Phi)]^{-1} [(D\Phi)^T {\cal F}_{\text{II}} (D\Phi)] \\ 
		&= \pm (D\Phi)^{-1} {\cal F}_{\text{I}}^{-1} ((D\Phi)^T)^{-1} (D\Phi)^T {\cal F}_{\text{II}} (D\Phi) \\ 
		&= \pm (D\Phi)^{-1} {\cal F}_{\text{I}}^{-1} {\cal F}_{\text{II}} (D\Phi) \\
		&= \pm (D\Phi)^{-1} {\cal W} (D\Phi), 
	\end{align*}
	where $\pm$ is the sign of $\det D\Phi$. 
\end{solution}

\item
Show that $\tilde{K} = K$ and $\tilde{H} = \pm H$, i.e., Gaussian curvature is invariant under reparametrisations, whereas mean curvature may change sign under reparametrisations.
Moreover, show that the principal curvatures change as follows under reparametrisations: 
$\widetilde{\kappa}_1 = \pm \kappa_1$ and $\widetilde{\kappa}_2 = \pm \kappa_2$.

\begin{solution}
	Using the previous part and noting that we are working 
	with $2\times 2$ matrices, we have by the multiplicativity of 
	determinant that 
	\[ \tilde K = \det \tilde{\cal W} = 
	(\pm1)^2 \det((D\Phi)^{-1} {\cal W} (D\Phi)) = \det{\cal W} = K. \] 
	Similarly, we know that similar matrices have the same trace, so 
	\[ \tilde H = \frac12\tr\tilde{\cal W} 
	= \frac12 \tr(\pm (D\Phi)^{-1} {\cal W} (D\Phi)) 
	= \pm \frac12 \tr{\cal W} = H. \] 
	Next, we use the fact that similar matrices have the same eigenvalues. 
	The principal curvatures are the eigenvalues of the shape operator, 
	so $\kappa_1$ and $\kappa_2$ are the eigenvalues of ${\cal W}$ 
	whereas $\tilde\kappa_1$ and $\tilde\kappa_2$ are the eigenvalues of 
	$\tilde{\cal W}$. If $\tilde {\cal W} = (D\Phi)^{-1} {\cal W} (D\Phi)$, 
	then $\tilde {\cal W}$ has the same eigenvalues as ${\cal W}$ 
	so that (up to reordering) $\tilde\kappa_1 = \kappa_1$ and 
	$\tilde\kappa_2 = \kappa_2$. If we had $\tilde{\cal W} 
	= - (D\Phi)^{-1} {\cal W} (D\Phi)$, then this means that 
	$\tilde{\cal W}$ and $-{\cal W}$ are similar and have the same 
	eigenvalues. This implies that $\tilde\kappa_1 = -\kappa_1$ and 
	$\tilde\kappa_2 = -\kappa_2$. 
\end{solution}

\end{enumerate}

\item
A smooth surface is called {\em flat} if its Gaussian curvature is everywhere zero. Use (b) to show that the notion of flat surface is independent of the atlas chosen.

\begin{solution}
	Consider the two coordinate charts $\sigma$ and $\tilde\sigma$ with $V \cap 
	\tilde V \neq \varnothing$. By part (b), we know that $K = \tilde K$. 
	Then the Gaussian curvature is independent of the choice of coordinate 
	chart, and therefore the notion of flatness is independent of 
	the choice of atlas.
\end{solution}

\item
A smooth surface is called {\em minimal} if its mean curvature is everywhere zero. Use (b) to show that the notion of minimal surface is independent of the atlas chosen. 

\begin{solution}
	By part (b), we have $H = \tilde H$. In particular, if $H = 0$ everywhere, 
	then $\tilde H = 0$ everywhere as well. This means that the notion of 
	minimal surface is independent of the choice of atlas.
\end{solution}

\item
Show that the Gaussian curvature of a minimal surface is $\leq 0$ everywhere.

\begin{solution}
	By part (d), the notion of minimal surface is independent of the choice of 
	atlas. Therefore, take any atlas and let $\sigma$ be a coordinate chart 
	in the atlas. 
	Let $\kappa_1$ and $\kappa_2$ be the principal curvatures of $\sigma$. 
	For a minimal surface, we have that $H = \frac12(\kappa_1 + \kappa_2) = 0$, 
	which implies that $\kappa_1 = -\kappa_2$. This implies that 
	$K = \kappa_1 \kappa_2 = -\kappa_1^2 \leq 0$. 
\end{solution}

\item
Show that if a unit-speed curve on a smooth surface has zero normal {\em and} geodesic curvatures everywhere, it is part of a straight line.

\begin{solution}
	We know that $\kappa^2 = \kappa_n^2 + \kappa_g^2$ for a 
	unit-speed curve, so if 
	$\kappa_n = \kappa_g = 0$ everywhere, then $\kappa = 0$ everywhere.
	By Problem C.6, the curve parametrizes a straight line.
\end{solution}

\end{enumerate}
		
\item
Let $S$ be the graph of a smooth function $f: U\subset \R^2 \rightarrow \R$ with $U$ an open subset of $\R^2$. Show that at a critical point $(x_0,y_0) \in U$ of $f$, the Gaussian curvature of $S$ is equal to the determinant of the Hessian matrix $H(f)(x_0,y_0)$ of $f$ at $(x_0,y_0)$.

\begin{solution}
	Note that a coordinate chart for $S$ is $\sigma : U \subset \R^2 
	\to S \subset \R^3$ defined by 
	\[ (u, v) \mapsto (u, v, f(u, v)). \] 
	We have that $\sigma_u = (1, 0, f_u)$ and 
	$\sigma_v = (0, 1, f_v)$, so the first fundamental 
	form of $\sigma$ is 
	\[ {\cal F}_{\text{I}} = \begin{bmatrix}
		1 + (f_u)^2 & f_u f_v \\ 
		f_u f_v & 1 + (f_v)^2
	\end{bmatrix}. \] 
	Next, we see that $\sigma_u \times \sigma_v = (-f_u, -f_v, 1)$ 
	and $\|\sigma_u \times \sigma_v\| = \sqrt{1 + (f_u)^2 + (f_v)^2}$, so 
	\[ N_\sigma = \frac{1}{\sqrt{1+(f_u)^2+(f_v)^2}} (-f_u, -f_v, 1). \] 
	Moreover, we have $\sigma_{uu} = (0, 0, f_{uu})$, $\sigma_{uv} 
	= (0, 0, f_{uv})$ and $\sigma_{vv} = (0, 0, f_{vv})$ so that 
	\[ {\cal F}_{\text{II}} = \frac{1}{\sqrt{1+(f_u)^2+(f_v)^2}} 
	\begin{bmatrix}
		f_{uu} & f_{uv} \\ 
		f_{vu} & f_{vv}
	\end{bmatrix} = \frac{1}{\sqrt{1+(f_u)^2+(f_v)^2}} H(f). \] 
	At a critical point $(u_0, v_0) \in U$ of $f$, we have that 
	\[ f_u(u_0, v_0) = f_v(u_0, v_0) = 0. \]
	At such a point, we have ${\cal F}_{\text{I}} = I_{2\times 2}$ 
	and ${\cal F}_{\text{II}} = H(f)$ so that the Gaussian curvature is 
	\[ K = \det{\cal W} = \det H(f). \] 
\end{solution}

\item
Let $S$ be the smooth surface given by $z=ax^2+by^2$, $a,b \in \mathbb{R}$. Show that:
\begin{enumerate}
\item Every point on $S$ is elliptic if and only if $ab > 0$, in which case $S$ is an elliptic paraboloid.
		
\item Every point on $S$ is hyperbolic if and only if $ab < 0$, in which case $S$ is a hyperbolic paraboloid (that is, a saddle surface).

\item Every point on $S$ is parabolic if and only if $ab = 0$ and $a \neq 0$ or $b \neq 0$, in which case $S$ is a parabolic cylinder.

\item Every point on $S$ is planar if and only if $a = b = 0$, in which case $S$ is the $xy$-plane ($z=0$).
\end{enumerate}

\begin{solution}
	We first compute the shape operator. Note that $\sigma(u, v) = 
	(u, v, au^2 + bv^2)$ is a smooth coordinate chart for $S$, 
	so we have $\sigma_u = (1, 0, 2au)$ and $\sigma_v = (0, 1, 2bv)$. 
	This implies that 
	\[ {\cal F}_{\text{I}} = \begin{bmatrix}
		1 + 4a^2u^2 & 4abuv \\ 
		4abuv & 1 + 4b^2v^2 
	\end{bmatrix}. \] 
	Note that $\det{\cal F}_{\text{I}} = (1+4a^2u^2)(1+4b^2v^2) 
	- (4abuv)^2 = 1 + 4a^2u^2 + 4b^2v^2$, so we have 
	\[ {\cal F}_{\text{I}}^{-1} = 
		\frac{1}{1+4a^2u^2 + 4b^2v^2} \begin{bmatrix}
			1+4a^2u^2 & -4abuv \\ 
			-4abuv & 1+4b^2v^2
		\end{bmatrix}. \] 
	Next, we have $\sigma_u \times \sigma_v = (-2au, -2bv, 1)$, so 
	$\|\sigma_u \times \sigma_v\| = \sqrt{1+4a^2u^2 + 4b^2v^2}$ and 
	\[ N_\sigma = \frac{1}{\sqrt{1+4a^2u^2 + 4b^2v^2}} (-2au, -2bv, 1). \] 
	Then $\sigma_{uu} = (0, 0, 2a)$, $\sigma_{uv} = (0, 0, 0)$ and 
	$\sigma_{vv} = (0, 0, 2b)$, giving us 
	\[ {\cal F}_{\text{II}} = \frac{2}{\sqrt{1+4a^2u^2 + 4b^2v^2}} 
	\begin{bmatrix}
		a & 0 \\ 0 & b
	\end{bmatrix}. \] 
	Therefore, the shape operator is 
	\[ {\cal W} = {\cal F}_{\text{I}}^{-1} {\cal F}_{\text{II}} 
	= \frac{2}{(1+4a^2u^2 + 4b^2v^2)^{3/2}} \begin{bmatrix}
		a(1+4a^2u^2) & -4ab^2uv \\ 
		-4a^2buv & b(1+4b^2v^2)
	\end{bmatrix}. \] 
	The Gaussian curvature is 
	\begin{align*} 
		K = \det{\cal W} &= \frac{4}{(1+4a^2u^2+4b^2v^2)^3} [ab (1+4a^2u^2)(1+4b^2v^2) 
		- (-4ab^2uv)(-4a^2buv)] \\ 
		&= \frac{4ab}{(1+4a^2u^2+4b^2v^2)^3} [(1 + 4a^2u^2)(1+4b^2v^2) 
		- (4abuv)^2] \\ 
		&= \frac{4ab}{(1+4a^2u^2+4b^2v^2)^2}, 
	\end{align*}
	and the mean curvature is 
	\[ H = \frac12 \tr{\cal W} = \frac{a(1+4a^2u^2) + b(1+4b^2v^2)}
	{(1+4a^2u^2 + 4b^2v^2)^{3/2}}. \]
	\begin{enumerate}[(a)]
		\item Observe that $K > 0$ if and only if $ab > 0$, with points on 
		$S$ being elliptic there. 
		\item We have $K < 0$ if and only if $ab < 0$, in which we have 
		hyperbolic points. 
		\item Suppose every point on $S$ is parabolic so that 
		$K = 0$ but $H \neq 0$. Then $ab = 0$ since $K = 0$, 
		and we must have $a \neq 0$ or $b \neq 0$ for 
		otherwise $H = 0$. Conversely, if $ab = 0$ with 
		$a \neq 0$ or $b \neq 0$, then $H \neq 0$ and $K = 0$.
		\item Note that $K = H = 0$ if and only if $a = b = 0$. 
	\end{enumerate}
\end{solution}

\item
Classify all the points on the regular surface patch $\sigma(u,v) = (u,v,u^3+v^2), \ (u,v) \in \mathbb{R}^2$, as to being elliptic, hyperbolic, parabolic, or planar.

\begin{solution}
	We will use Problem D.3 to simplify computations a little. 
	Note that $\sigma$ corresponds to the graph of the function 
	$f(u, v) = u^3+v^2$ with $f_u = 3u^2$ and $f_v = 2v$, 
	as well as $f_{uu} = 6u$, $f_{uv} = 0$, and $f_{vv} = 2$. 
	This tells us that
	\[ {\cal F}_{\text{I}} = \begin{bmatrix} 
		1 + (f_u)^2 & f_u f_v \\ 
		f_u f_v & 1 + (f_v)^2 
	\end{bmatrix} = \begin{bmatrix}
		1+9u^4 & 6u^2v \\ 
		6u^2v & 1+4v^2
	\end{bmatrix} \] 
	and we also saw in that problem that 
	\[ {\cal F}_{\text{II}} = \frac{1}{\sqrt{1+(f_u)^2+(f_v)^2}} 
	\begin{bmatrix}
		f_{uu} & f_{uv} \\ 
		f_{vu} & f_{vv} 
	\end{bmatrix} = \frac{1}{\sqrt{1+9u^4+4v^2}} \begin{bmatrix}
		6u & 0 \\ 0 & 2
	\end{bmatrix}. \] 
	Now, observe that $\det{\cal F}_{\text{I}} 
	= (1+9u^4)(1+4v^2) - (6u^2v)^2 = 1+9u^4+4v^2$, which implies that 
	\[ {\cal F}_{\text{I}}^{-1} = \frac{1}{1+9u^4+4v^2} 
	\begin{bmatrix}
		1+9u^4 & -6u^2v \\ -6u^2v & 1+4v^2
	\end{bmatrix}. \] 
	Then the shape operator is 
	\[ {\cal W} = {\cal F}_{\text{I}}^{-1} {\cal F}_{\text{II}} 
	= \frac{1}{(1+9u^4+4v^2)^{3/2}} \begin{bmatrix}
		6u(1+9u^4) & -12u^2v \\ 
		-36u^3v & 2(1+4v^2)
	\end{bmatrix}. \] 
	The Gaussian curvature is 
	\begin{align*} 
		K = \det{\cal W} &= \frac{1}{(1+9u^4+4v^2)^3} 
		[12u(1+9u^4)(1+4v^2) - (-12u^2v)(-36u^3v)] \\ 
		&= \frac{12u}{(1+9u^4+4v^2)^3} [(1+9u^4)(1+4v^2) - 36u^4v^2] \\ 
		&= \frac{12u}{(1+9u^4+4v^2)^2}, 
	\end{align*}
	and the mean curvature is 
	\[ H = \frac12\tr{\cal W} = \frac{3u(1+9u^4) + (1+4v^2)}
	{(1 + 9u^4 + 4v^2)^{3/2}}. \] 
	Let $(u, v) \in \R^2$. We see that $\sigma(u, v)$ is elliptic for 
	points where $u > 0$ and hyperbolic for points where $u < 0$. 
	When $u = 0$, note that $H > 0$ for all $v \in \R$, so 
	$\sigma(u, v)$ is parabolic there. 
\end{solution}

\item
Let $S$ be the circular cylinder $x^2+y^2=1$. Consider the atlas $\{ (S_i, \sigma_i)\}_{i=1}^2$ of $S$,
where each $S_i$ is the image in $\mathbb{R}^3$ of the smooth coordinate charts
\[ \sigma_i(u,v) = (\cos u, \sin u, v), \]
with $(u,v) \in (0,2\pi) \times \mathbb{R}$ when $i=1$, 
and $(u,v) \in (-\pi,\pi) \times \mathbb{R}$ when $i=2$.
\begin{enumerate}
\item
Find the principal curvatures $\kappa_1, \kappa_2$ of $S$ at every point.

\begin{solution}
	We first compute the shape operator. Since $\sigma_1$ and $\sigma_2$ 
	are described using the same function but over different domains, 
	we will denote them simply by $\sigma$ for now. We have 
	$\sigma_u = (-\sin u, \cos u, 0)$ and $\sigma_v = (0, 0, 1)$ so that 
	\[ {\cal F}_{\text{I}} = \begin{bmatrix}
		1 & 0 \\ 0 & 1
	\end{bmatrix}. \] 
	Next, we see that $N_\sigma = \sigma_u \times \sigma_v = (\cos u, \sin u, 0)$
	with $\sigma_{uu} = (-\cos u, -\sin u, 0)$ and $\sigma_{uv} = \sigma_{vv} = (0, 0, 0)$, so 
	\[ {\cal F}_{\text{II}} = \begin{bmatrix}
		-1 & 0 \\ 0 & 0
	\end{bmatrix}. \] 
	It follows that the shape operator is 
	\[ {\cal W} = {\cal F}_{\text{I}}^{-1} {\cal F}_{\text{II}} = \begin{bmatrix}
		-1 & 0 \\ 0 & 0 
	\end{bmatrix}. \] 
	The principal curvatures are the eigenvalues of ${\cal W}$, namely 
	$\kappa_1 = 0$ and $\kappa_2 = -1$. 
\end{solution}

\item
Show that the normal curvature $\kappa_n$ of any unit-speed curve $\gamma$ on $S$ is such that
\[ -1 \leq \kappa_n \leq 0.\]

\begin{solution}
	We recall that the principal curvatures $\kappa_1$ and $\kappa_2$ 
	are the maximum and minimum values of the normal curvature $\kappa_n$. 
	This implies that $-1 \leq \kappa_n \leq 0$. 
\end{solution}

\item
Is the cylinder $S$ a flat surface? a minimal surface? Justify your answer. (See problem 2 for a definition of {\em flat} and {\em minimal}.)

\begin{solution}
	For convenience, we restate the definitions here. A smooth surface is 
	flat if its Gaussian curvature is everywhere zero, and minimal 
	if its mean curvature is everywhere zero. 
	In this case, we have $K = \det{\cal W} = 0$ and $H = 
	\frac12\tr{\cal W} = -\frac12$, so the cylinder $S$ is a flat surface 
	but not minimal. 
\end{solution}

\end{enumerate}

\item
Let $\gamma: (\alpha,\beta) \rightarrow \mathbb{R}^3$ be a smooth unit-speed curve whose curvature is everywhere non-zero, and is such that the map 
\[ \sigma(v,s) = \gamma(s) + v\gamma'(s), s \in (\alpha,\beta), v > 0,\]
is a homeomorphism.
\begin{enumerate}
\item
Show that $\sigma$ is  a smooth embedding.

\begin{solution}
	Since $\gamma$ is smooth, we know that $\sigma$ is smooth. Note that 
	\[ D\sigma(v, s) = (\gamma'(s), \gamma'(s) + v\gamma''(s)). \] 
	This is nonzero everywhere since $\gamma$ is unit-speed. 
	Moreover, we know that $\gamma'(s) \cdot \gamma''(s) = 0$. 
	Since the curvature of $\gamma$ is everywhere nonzero, this implies that 
	$\gamma'(s) \neq \vec 0$ for all $s \in (\alpha, \beta)$ 
	and therefore $\gamma'(s) \perp \gamma''(s)$. In particular, 
	we know that $\gamma'(s)$ and $\gamma'(s) + v\gamma''(s)$ are 
	linearly independent since $v > 0$, so $D\sigma$ has maximal rank $2$ everywhere. 
	Finally, it is given that $\sigma$ is a homeomorphism, so $\sigma$ is 
	an embedding.
\end{solution}

\item
Show that the smooth surface determined by $\sigma$ is flat. (See problem 2 for a definition of {\em flat}.)

{\em Hint:} Start by showing that $L=M=0$ everywhere. Conclude that $K = 0$ everywhere.
\end{enumerate}

\begin{solution}
	Since $K = \det{\cal W} = (\det{\cal F}_{\text{I}})^{-1} 
	\det{\cal F}_{\text{II}}$ by the multiplicativity of the determinant, 
	it is enough to show that 
	$\det{\cal F}_{\text{II}} = 0$. We first compute 
	\[ {\cal F}_{\text{II}} = \begin{bmatrix}
		\sigma_{vv} \cdot N_\sigma & \sigma_{vs} \cdot N_\sigma \\ 
		\sigma_{sv} \cdot N_\sigma & \sigma_{ss} \cdot N_\sigma 
	\end{bmatrix} = \begin{bmatrix}
		L & M \\ M & N
	\end{bmatrix}. \] 
	We have $\sigma_v = \gamma'(s)$ and $\sigma_s = \gamma'(s) + v\gamma''(s)$. 
	Note that $T(s) = \gamma'(s)$ and $N(s) = \gamma''(s)/\|\gamma''(s)\|$, 
	and moreover, we know that  
	\[ B(s) = T(s) \times N(s) = \frac{\gamma'(s) \times \gamma''(s)}
	{\|\gamma''(s)\|} \] 
	with $\|B(s)\| = 1/\|\gamma''(s)\|$ since $\|T(s)\| = \|N(s)\| = 1$. 
	In particular, we have 
	\begin{align*} 
		\sigma_v \times \sigma_s 
		= \gamma'(s) \times (\gamma'(s) + v\gamma''(s)) 
		= v\gamma'(s) \times \gamma''(s) 
		= v \frac{B(s)}{\|B(s)\|} = vB(s)
	\end{align*} 
	and therefore $N_\sigma = B(s)$ since $\|B(s)\| = 1$ and
	$\|\sigma_v \times \sigma_s\| = v$. 
	Finally, we compute that $\sigma_{vv} = \vec 0$ and $\sigma_{vs} 
	= \gamma''(s) = \kappa(s)N(s)$. 
	Observe that $L = \sigma_{vv} \cdot N_\sigma = 0$ 
	and $M = \sigma_{vs} \cdot N_\sigma = \gamma''(s) \cdot B(s) 
	= \kappa(s)N(s) \cdot B(s) = 0$, so it follows that 
	$\det{\cal F}_{\text{II}} = 0$.
\end{solution}

\item 
Let $\gamma: (\alpha, \beta) \rightarrow \mathbb{R}^3$ be a smooth whose curvature is everywhere non-zero, and let $\{ T,N,B \}$
be its Frenet frame. Suppose that the map
\[ \sigma(v,s) = \gamma(s) + vB(s),\]
$s \in (\alpha, \beta) , v \in (-\epsilon,\epsilon)$, is a homeomorphism for small enough $\epsilon > 0$. Let $S$ be image of $\sigma$ in $\mathbb{R}^3$.
\begin{enumerate}
\item Show that $S$ is a smooth surface.

\begin{solution}
	Since $\sigma$ is a smooth homeomorphism with $S$ as its image, 
	it is enough to show that $D\sigma$ has maximal rank everywhere. 
	Note that 
	\[ D\sigma(v, s) = \left(B(s), \gamma'(s) + v \frac{{\rm d}}{{\rm d}s} B(s) 
	\right) = (B(s), T(s) - v\tau(s)N(s)), \] 
	where we used the Frenet-Serret equations. To show that $\sigma_v = B(s)$ 
	and $\sigma_s = T(s) - v\tau(s)N(s)$ are linearly independent, we can show 
	that the area of the parallelogram spanned by them is nonzero. 
	Indeed, we have that 
	\begin{align*} 
		\sigma_v \times \sigma_s
		&= B(s) \times T(s) - v\tau(s) B(s) \times N(s) \\
		&= N(s) + v\tau(s) N(s) \times B(s) \\
		& = N(s) + v\tau(s) T(s). 
	\end{align*}
	Since $N(s) \perp T(s)$, this implies that 
	\[ \|\sigma_v \times \sigma_s\| = 1 + v^2\tau(s)^2 > 0. \] 
	Therefore, $\sigma$ is a smooth embedding and $S$ is a smooth surface. 
\end{solution}

\item Show that $\gamma$ is a geodesic of $\sigma$ and that every point on $\gamma$ is either hyperbolic or parabolic.

\begin{solution}
	Note that $\gamma(s) = \sigma(0, s)$ with $\gamma''(s) = 
	\kappa(s) N(s) = \kappa(s) N_\sigma(0, s)$, so $\gamma$ is a 
	geodesic of $\sigma$. Next, we have $\sigma_v = B(s)$ 
	and $\sigma_s = T(s)$ on $\gamma$, which implies that 
	$\sigma_v \cdot \sigma_v = \sigma_s \cdot \sigma_s = 1$ and $
	\sigma_v \cdot \sigma_s = 0$. Thus, the first fundamental form is 
	\[ {\cal F}_{\text{I}} = \begin{bmatrix}
		1 & 0 \\ 0 & 1
	\end{bmatrix}. \] 
	Recall that $N_\sigma(0, s) = N(s)$. Moreover, we have 
	$\sigma_{vv} = \vec 0$. The Frenet-Serret equations imply that 
	$\sigma_{vs} = \frac{{\rm d}}{{\rm d}s} B(s) = -\tau(s)N(s)$ and 
	$\sigma_{ss} = \frac{{\rm d}}{{\rm d}s} T(s) = \kappa(s) N(s)$. Then 
	\[ {\cal W} = {\cal F}_{\text{II}} = \begin{bmatrix}
		0 & -\tau(s) \\ 
		-\tau(s) & \kappa(s)
	\end{bmatrix} \] 
	since ${\cal F}_{\text{I}} = I_{2\times 2}$. Notice that 
	$K = \det{\cal W} = -\tau(s)^2 \leq 0$ and $H = \frac12\tr{\cal W} = 
	\frac12\kappa(s) > 0$ since the curvature is everywhere nonzero. 
	In particular, we either have $H < 0$ so that the point on 
	$\gamma$ is hyperbolic, or $H = 0$ and $K \neq 0$ so that the point 
	is parabolic.
\end{solution}

\end{enumerate}


\item
\begin{enumerate}
\item
Let $\gamma$ be a unit-speed curve on a smooth surface $S$. Show that  $\gamma$ is a geodesic if and only if $\kappa_g = 0$.

\begin{solution}
	By definition, $\gamma$ is a geodesic if for all $s \in (\alpha, \beta)$, 
	we have $\gamma''(s) = \vec 0$ or $\gamma''(s) \parallel N_\sigma$. 
	This implies that $\kappa_g = \gamma''(s) \cdot (N_\sigma \times \gamma'(s)) 
	= 0$. In the case that $\gamma''(s) = \vec 0$, this is clear. 
	Otherwise, we have $\gamma''(s) \parallel N_\sigma$. 
	Since $N_\sigma \perp N_\sigma \times \gamma'(s)$ by the
	definition of the cross product, we have
	$\gamma''(s) \perp (N_\sigma \times \gamma'(s))$ as well. 
\end{solution}

\item
Prove or disprove (by giving a counterexample) the following statements:
\begin{enumerate}


\item
If the unit-speed curve $\gamma$ is a plane curve, then $\kappa_g = \pm \kappa$.

\begin{solution}
	False. The great circles on a sphere are geodesics, but are also 
	plane curves as they are obtained by intersecting the sphere 
	with a plane passing through the origin. By (a), we have 
	$\kappa_g = 0$ everywhere. However, we have seen that the 
	curvature of a sphere of radius $a$ is $\kappa = 1/a$, so 
	$\kappa_g \neq \pm\kappa$. 
\end{solution}

\item
If the unit-speed curve $\gamma$ is a geodesic, then $\kappa_n = \pm \kappa$.

\begin{solution}
	True. We have that $\kappa^2 = \kappa_n^2 + \kappa_g^2$. Since 
	$\gamma$ is a geodesic, we have by (a) that $\kappa_g = 0$, so 
	$\kappa^2 = \kappa_n^2$ and $\kappa_n = \pm\kappa$. 
\end{solution}

\item
Let $p$ and $q$ be two points on a smooth surface $S$, and let $\gamma$ be a geodesic joining $p$ and $q$. Then, $\gamma$ is the path of shortest length from $p$ to $q$.

\begin{solution}
	False. Pick two points $p$ and $q$ on a great circle of a sphere which are not 
	antipodal. Then there are two arcs from $p$ to $q$ which are both 
	geodesics, but one has greater length than the other. 
\end{solution}

\end{enumerate}

\item
Give an example of an isometry which is not a translation, rotation, or reflection. 

\begin{solution}
	Let $S_1 = \{(x, y, z) \in \R^3 : z = 0,\, 0 < x < 2\pi\}$ be a 
	subset of the $xy$-plane and $S_2 = \{(x, y, z) \in \R^3 : 
	x^2 + y^2 = 1,\, x \neq 1\}$ be the cylinder of radius $1$ minus the 
	line $x = 1$. Consider the diffeomorphism 
	\begin{align*}
		\Phi : S_1 &\to S_2 \\ 
		(x, y, 0) &\mapsto (\cos x, \sin x, y).
	\end{align*}
	Define $\sigma : \R^2 \to S_1$ by $(u, v) \mapsto (u, v, 0)$ and let 
	\begin{align*}
		\tilde\sigma = \Phi \circ \sigma : \R^2 &\to S_2 \\ 
		(u, v) &\mapsto (\cos u, \sin u, v). 
	\end{align*}
	Then $\tilde\sigma$ is the usual parametrization of the cylinder. 
	We have seen in class that ${\cal F}_{\text{I}} = \tilde{\cal F}_{\text{I}} 
	= I_{2\times 2}$, so $\Phi$ is an isometry.
\end{solution}

\item
Describe all geodesics on a plane, a sphere and a cylinder. 

\begin{solution}
	The geodesics on a plane with coordinate $\sigma(u, v) 
	= p_0 + uw_1 + vw_2$ are the lines
	\[ \gamma(t) = \sigma(at+b, ct+d). \] 
	The geodesics on a sphere are the great circles. To find the geodesics of a 
	cylinder, one can use the isometry we gave in part (c), which gives a 
	one-to-one correspondence between geodesics on the plane 
	and geodesics on the cylinder. Here, the geodesics on the cylinder 
	are the images of the lines in the plane under the isometry $\Phi$, so we have 
	\[ \gamma(t) = \Phi(at+b, ct+d, 0) = (\cos(at+b), \sin(at+b), ct+d). \] 
	This is a line if $a = 0$, a circle of $c = 0$, and a helix if $a$ and 
	$c$ are both nonzero. 
\end{solution}
\end{enumerate}
\end{enumerate}

\newpage 
\noindent
{\bf E. Integration.}

\begin{enumerate}
\item 
Let $M$ be a smooth $k$-dimensional submanifold of $\R^n$ and $\alpha: U \subset \R^k \rightarrow V \subset M$ be a smooth coordinate chart of $M$. Assume that $V$ is bounded.
Define the following:
\begin{enumerate}
\item The first fundamental form of $\alpha$.

\begin{solution}
	The first fundamental form of $\alpha$ is defined to be 
	\[ {\cal F}_{\text{I}} := D\alpha^T D\alpha. \] 
\end{solution}
\item The scalar integral of a smooth function $f$ on $V$.

\begin{solution}
	The scalar integral of a smooth function $f$ on $V$ is 
	\[ \int_V f\,\text{dVol} := \int_U f(\alpha(u)) \sqrt{\det{\cal F}_{\text{I}}} 
	\,{\rm d}u_1 \cdots {\rm d}u_k. \] 
\end{solution}

\item The line integral of a smooth vector field on a smooth curve.

\begin{solution}
	Let $\gamma : [a, b] \to \R^n$ be a smooth regular curve contained 
	in $\R^n$, and $F : U \to \R^n$ be a smooth vector field defined 
	on an open set $U \subset \R^n$ containing $\gamma$. The line 
	integral of $F$ on $\gamma$ is defined to be
	\[ \int_\gamma F \cdot {\rm d}r := \int_a^b F(\gamma(t)) \cdot \gamma'(t) 
	\,{\rm d}t. \] 
\end{solution}

\item A hypersurface in $\R^n$.

\begin{solution}
	A hypersurface is a smooth $(n-1)$-dimensional submanifold of $\R^n$, 
	where $n > 3$. 
\end{solution}

\item An orientation of a hypersurface in $\R^n$.

\begin{solution}
	A hypersurface $M$ in $\R^n$ is orientable if it admits a smooth unit 
	normal vector $N$. There are two possible choices for the 
	smooth unit normal vector $N$; such a choice of $N$ 
	is called an orientation of $M$.
\end{solution}

\item The integral of a smooth vector field on a hypersurface in $\R^n$.

\begin{solution}
	Let $F : V \to \R^n$ be a smooth vector field, and let 
	$B$ be a bounded subset of $V$. Let 
	$N$ be a unit normal vector to $M$ (i.e. an orientation). Then we define 
	\[ \int_B F \cdot \text{dVol} = \int_B (F \cdot N)\,{\rm dVol}, \] 
	where the latter integral is the scalar integral 
	of the smooth function $F \cdot N$ as defined in (b). 
\end{solution}

\end{enumerate}

\item
Let $\gamma: (\alpha, \beta) \rightarrow \mathbb{R}^n$ be a smooth regular curve. Verify that the formula for the integral of a scalar function on $\gamma$ is equivalent to
\[  \int_\gamma f \ ds = \int_\alpha^\beta f(\gamma(t))\| \gamma'(t) \| dt\] 
for any smooth scalar function $f:(\alpha,\beta) \rightarrow \R$.

\begin{solution}
	Note that $D\gamma(t) = \gamma'(t)$, so ${\cal F}_{\text{I}}$ is a 
	$1\times 1$ matrix with 
	\[ {\cal F}_{\text{I}} = D\gamma(t)^T D\gamma(t) 
	= \gamma'(t)^T \gamma'(t) = \gamma'(t) \cdot \gamma'(t) 
	= \|\gamma'(t)\|^2. \] 
	It follows that $\det{\cal F}_{\text{I}} = {\cal F}_{\text{I}} 
	= \|\gamma'(t)\|^2$, and therefore 
	\[ \int_\gamma f\,{\rm d}s = 
	\int_\alpha^\beta f(\gamma(t)) \sqrt{\det{\cal F}_{\text{I}}}\,{\rm d}t 
	= \int_\alpha^\beta f(\gamma(t)) \|\gamma'(t)\|\,{\rm d}t. \] 
\end{solution}

\item
Let $S$ be a surface in $\R^3$ and $\sigma: U \subset \R^2 \rightarrow V \subset S$ be a smooth coordinate chart with $V$ bounded. Show that the formulas for the integrals of a scalar function or a vector field on $V$ can be expressed in terms of $\sigma$ as:
\begin{enumerate}
\item
$\int_V f \ dVol = \int_U f(\sigma(u,v))\| \sigma_u \times \sigma_v \| dudv$ for any smooth scalar function $f:V \rightarrow \R$.

\begin{solution}
	From our study of surfaces, we have $\det{\cal F}_{\text{I}} 
	= \|\sigma_u \times \sigma_v\|^2$, so $\sqrt{\det{\cal F}_{\text{I}}} 
	= \|\sigma_u \times \sigma_v\|$ and the formula follows by definition.
\end{solution}

\item
$\int_V F \cdot dVol = \int_U F(\sigma(u,v)) \cdot (\sigma_u \times \sigma_v ) dudv$ with respect to the orientation $N_\sigma$ on $V$.

\begin{solution}
	Using the definition and part (a), we have that 
	\begin{align*}
		\int_V F \cdot \text{dVol} 
		&= \int_V (F \cdot N_\sigma)\,{\rm d}A \\ 
		&= \int_U F(\sigma(u, v)) \cdot N_\sigma \|\sigma_u \times \sigma_v\|\,{\rm d}u\,{\rm d}v \\ 
		&= \int_U F(\sigma(u, v)) \cdot \frac{\sigma_u \times \sigma_v}{\|\sigma_u \times \sigma_v\|} \|\sigma_u \times \sigma_v\|\,{\rm d}u\,{\rm d}v \\ 
		&= \int_U F(\sigma(u, v)) \cdot (\sigma_u \times \sigma_v)\,{\rm d}u\,{\rm d}v. 
	\end{align*}
\end{solution}

\end{enumerate}

\item Prove that any smooth hypersurface $M$ in $\R^n$ is locally orientable. That is, for any point $p \in M$, there exists an open neighbourhood $V$ of $p$ in $M$ such that $V$ is an orientable hypersurface.

\begin{solution}
	The zero set of a smooth scalar function $G : V \subset \R^n \to \R$ 
	with $\nabla G$ of maximal rank everywhere on an open set $V \subset \R^n$ 
	is orientable with possible orientations $\pm\nabla G/\|\nabla G\|$. 
	Since any smooth hypersurface $M$ in $\R^n$ is locally the zero set of such a function, 
	it follows that $M$ is locally orientable. 
\end{solution}

\item Prove that if $M$ be a smooth bounded $n$-dimensional submanifold of $\R^n$,  then $\int_M f dVol$ is just the standard multiple integral $\int_M f(x_1,\dots,x_n)dx_1dx_2 \cdots dx_n$ for any smooth scalar function $f$ on $M$.

\begin{solution}
	We know that the definition of the scalar integral is independent 
	of the choice of coordinate chart. Let $M$ be a smooth 
	bounded $n$-dimensional submanifold of $\R^n$. Then $M$ is an 
	open set in $\R^n$, and we can simply take $\alpha = \text{Id}_M$ 
	as the coordinate chart which has $D\alpha = I_{n\times n}$ 
	and therefore ${\cal F}_{\text{I}} = D\alpha^T D\alpha = I_{n\times n}$. 
	Then $\sqrt{\det {\cal F}_{\text{I}}} = 1$, so 
	\[ \int_M f\,{\rm dVol} = \int_M f(\alpha(x_1, \dots, x_n)) 
	\sqrt{\det{\cal F}_{\text{I}}}\,{\rm d}x_1 \cdots {\rm d}x_n 
	= \int_M f(x_1, \dots, x_n)\,{\rm d}x_1 \cdots {\rm d}x_n. \] 
\end{solution}

\item
\begin{enumerate}
\item
Evaluate the line integral $\int_\gamma x^3yds$ along part of the line $\gamma(t) = (t,-t)$, $t \in [-1, 2]$, in $\mathbb{R}^2$.

\begin{solution}
	Let $f(x, y) = x^3 y$. We have $f(\gamma(t)) = -t^4$ and 
	$\gamma'(t) = (1, -1)$ so that $\|\gamma'(t)\| = \sqrt{2}$. 
	Therefore, using the result from Problem E.2, we obtain 
	\begin{align*}
		\int_\gamma x^3 y\,{\rm d}s 
		&= \int_{-1}^2 f(\gamma(t))\|\gamma'(t)\|\,{\rm d}t \\ 
		&= -\sqrt{2} \int_{-1}^2 t^4\,{\rm d}t \\ 
		&= -\sqrt{2} \left[ \frac{t^5}{5} \right]_{-1}^2 
		= -\sqrt{2} \left( \frac{32}{5} - \frac15 \right) = 
		-\frac{31\sqrt{2}}{5}. 
	\end{align*}
\end{solution}

\item
Evaluate the line integral $\int_\gamma F \cdot dr$ of the vector field $F(x,y,z) = (yz,xz,xy)$ along part of the twisted cubic $\gamma(t) = (t,t^2,t^3)$, $t \in [0,1]$, in $\mathbb{R}^3$.

\begin{solution}
	We have $F(\gamma(t)) = (t^5, t^4, t^3)$ and $\gamma'(t) = 
	(1, 2t, 3t^2)$ so that 
	\[ F(\gamma(t)) \cdot \gamma'(t) = (t^5, t^4, t^3) 
	\cdot (1, 2t, 3t^2) = 6t^5. \] 
	From this, we obtain 
	\[ \int_\gamma F\cdot{\rm d}r = \int_0^1 F(\gamma(t)) 
	\cdot \gamma'(t)\,{\rm d}t = \int_0^1 6t^5\,{\rm d}t 
	= 1^6 - 0^6 = 1. \] 
\end{solution}

\item
Evaluate the surface integral $\int_S 6xz dA$ where 
\[ S = \{ (x,y,z) \in \R^3:  -1 \leq x \leq 3, 0 \leq y \leq 4, z=1-x-y \} \] 
is the portion of the plane $x+y+z = 1$ lying above the rectangle $[-1,3] \times [0,4]$ in the $xy$-plane.

\begin{solution}
	We can parametrize the surface $S$ with the coordinate chart 
	\[ \sigma(u, v) = (u, v, 1-u-v), \] 
	where $(u, v) \in U = [-1, 3] \times [0, 4]$. Note that 
	$\sigma_u = (1, 0, -1)$ and $\sigma_v = (0, 1, -1)$, so 
	$\sigma_u \times \sigma_v = (1, 1, 1)$ and $\|\sigma_u \times 
	\sigma_v\| = \sqrt{3}$. Next, for the smooth scalar 
	function $f(x, y, z) = 6xz$, we have 
	\[ f(\sigma(u, v)) = 6u(1-u-v). \] 
	Applying part (a) of Problem E.3 gives 
	\begin{align*}
		\int_S 6xz\,{\rm d}A 
		&= \int_0^4 \int_{-1}^3 f(\sigma(u, v)) \|\sigma_u \times \sigma_v\|\,{\rm d}u\,{\rm d}v \\ 
		&= 6\sqrt{3} \int_0^4 \int_{-1}^3 (u - u^2 - uv)\,{\rm d}u\,{\rm d}v \\ 
		&= 6\sqrt{3} \int_0^4 \left[ \frac{u^2}{2} - \frac{u^3}{3} - \frac{u^2v}{2} \right]_{-1}^3{\rm d}v \\ 
		&= 6\sqrt{3} \int_0^4 \left[ \left( \frac92 - 9 - \frac{9v}2 \right) - \left( \frac12 + \frac13 - \frac{v}2 \right) \right]{\rm d}v \\ 
		&= 6\sqrt{3} \int_0^4 \left( -\frac{16}{3} - 4v \right){\rm d}v \\ 
		&= 6\sqrt{3} \left[ -\frac{16}{3}v - 2v^2 \right]_0^4 
		= -320\sqrt{3}. 
	\end{align*}
\end{solution}

\item
Evaluate the surface integral $\int_S y dA$ where $S$ is the portion of the cylinder $x^2+y^2 = 3$ that lies between $z=0$ and $z=6$. 

\begin{solution}
	Here, we can parametrize $S$ using the chart 
	\[ \sigma(u, v) = (\sqrt 3\cos u, \sqrt 3\sin u, v) \] 
	where $(u, v) \in U = [0, 2\pi] \times [0, 6]$. We have 
	$\sigma_u = (-\sqrt 3\sin u, \sqrt 3\cos u, 0)$ and $\sigma_v = 
	(0, 0, 1)$ so that $\sigma_u \times \sigma_v = 
	(\sqrt 3\cos u, \sqrt 3\sin u, 0)$ with $\|\sigma_u \times \sigma_v\| = \sqrt 3$. 
	Moreover, we have the scalar function $f(x, y, z) = y$ and 
	$f(\sigma(u, v)) = \sqrt 3\sin u$. Putting everything together,we have  
	\begin{align*}
		\int_S y\,{\rm d}A 
		&= \int_0^6 \int_0^{2\pi} f(\sigma(u, v)) \|\sigma_u\times \sigma_v\|\,{\rm d}u\,{\rm d}v 
		= 3 \int_0^6 \int_0^{2\pi} \sin u \,{\rm d}u\,{\rm d}v \\ 
		&= 3 \int_0^6 (-\cos 2\pi + \cos 0) \,{\rm d}v 
		= 3 \int_0^6 0\,{\rm d}v = 0.
	\end{align*}
\end{solution}

\item
Evaluate the integral $\int_R z dV$ where $R$ is the rectangular box $[-1,4] \times [0,3] \times [-2,5]$ in $\mathbb{R}^3$.

\begin{solution}
	Since $R$ is a $3$-dimensional smooth submanifold of $\R^3$, 
	we can perform a standard triple integral by Problem E.5. Then we have 
	\begin{align*} 
		\int_R z\,{\rm d}V 
		&= \int_{-2}^5 \int_0^3 \int_{-1}^4 z\,{\rm d}x\,{\rm d}y\,{\rm d}z 
		= 5 \int_{-2}^5 \int_0^3 z\,{\rm d}y\,{\rm d}z \\ 
		&= 15 \int_{-2}^5 z\,{\rm d}z  
		= 15 \left[ \frac{z^2}{2} \right]_{-2}^5 = 
		15\left( \frac{25}{2} - 2 \right) = \frac{315}{2}.
	\end{align*}
\end{solution}

\item
Evaluate the integral of the vector field $F(x,y,z) = (-y,x,z)$ on the piece of 
paraboloid $S$ in $\R^3$ parametrised by $\sigma(u, v) = (u, v, u^2+v^2)$, $(u,v) \in [0,1] \times [0,3]$, with respect to the orientation $N_\sigma$.

\begin{solution}
	First, we have that $F(\sigma(u, v)) = (-v, u, u^2 + v^2)$. 
	Next, we have $\sigma_u = (1, 0, 2u)$ and $\sigma_v = (0, 1, 2v)$, so 
	$\sigma_u \times \sigma_v = (-2u, -2v, 1)$. Then 
	\[ F(\sigma(u, v)) \cdot (\sigma_u \times \sigma_v) 
	= (-v, u, u^2+v^2) \cdot (-2u, -2v, 1) = u^2+v^2. \]
	Note that $S$ is bounded, and applying part (b) of Problem E.3 gives 
	\begin{align*}
		\int_S F\cdot {\rm dVol} &= \int_0^3 \int_0^1 
		F(\sigma(u, v)) \cdot (\sigma_u \times \sigma_v)\,{\rm d}u\,{\rm d}v \\ 
		&= \int_0^3 \int_0^1 (u^2 + v^2)\,{\rm d}u\,{\rm d}v \\ 
		&= \int_0^3 \left[ \frac{u^3}{3} + uv^2 \right]_0^1 {\rm d}v \\ 
		&= \int_0^3 \left( \frac13 + v^2 \right){\rm d}v \\ 
		&= \left[ \frac{v}{3} + \frac{v^3}{3} \right]_0^3 = 
		1 + 9 = 10. 
	\end{align*}
\end{solution}

\item
Evaluate the integral of the vector field $F(x,y,z) = (xy,2,z-1)$ on the 
half-cylinder $C$ in $\R^3$ parametrised by $\sigma(u, v) = (\cos u, \sin u, v)$, $(u,v) \in [0,\pi] \times [2,4]$, with respect to the orientation $N_\sigma$.		

\begin{solution}
	We have $F(x, y, z) = (\cos u \sin u, 2, v-1)$. Moreover, 
	we see that $\sigma_u = (-\sin u, \cos u, 0)$ and $\sigma_v 
	= (0, 0, 1)$, so $\sigma_u \times \sigma_v = 
	(\cos u, \sin u, 0)$ and 
	\[ F(\sigma(u, v)) \cdot (\sigma_u \times \sigma_v) = 
	(\cos u \sin u, 2, v-1) \cdot (\cos u, \sin u, 0) = 
	\cos^2 u \sin u + 2\sin u. \] 
	Moreover, $C$ is bounded. Therefore, we can apply part (b) of 
	Problem E.3 to obtain 
	\begin{align*}
		\int_C F \cdot{\rm dVol} 
		&= \int_2^4 \int_0^\pi (\cos^2 u \sin u + 2\sin u)\,{\rm d}u\,{\rm d}v \\ 
		&= \int_2^4 \left[ -\frac13\cos^3 u - 2\cos u \right]_0^\pi {\rm d}v \\ 
		&= \int_2^4 \left[ \left( \frac13 + 2 \right) - \left( -\frac13 - 2 \right) \right]{\rm d}v \\ 
		&= \int_2^4 \frac{14}3\,{\rm d}v = \frac{28}{3}. 
	\end{align*}
\end{solution}

\end{enumerate}
\end{enumerate}


\end{document}