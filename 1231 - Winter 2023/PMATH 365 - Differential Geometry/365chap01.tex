\section{Submanifolds of $\R^n$}\label{sec:1}

\subsection{Preliminaries}\label{subsec:1.1}
To begin, we'll recall some facts about the topology of $\R^n$ and 
vector-valued functions.

In this course, we'll be working with the metric topology with respect 
to the Euclidean norm (or metric). Let $x = (x_1, \dots, x_n),\,
y = (y_1, \dots, y_n) \in \R^n$.
The {\bf Euclidean norm} is defined to be 
\[ \|x\| = \sqrt{x_1^2 + \cdots + x_n^2}, \] 
and {\bf Euclidean distance} is given by 
\[ \dist(x, y) = \|y - x\| = \sqrt{(y_1 - x_1)^2 + 
\cdots + (y_n - x_n)^2}. \] 
We define the {\bf open ball} of radius $r > 0$ centered at $x \in \R^n$ by 
\[ B_r(x) := \{y \in \R^n : \dist(x, y) < r\} \subset \R^n. \] 
A {\bf topology} on $\R^n$ is a collection $\mathcal{U} = \{U_\alpha\}_{\alpha \in A}$ 
of subsets $U_\alpha \subset \R^n$ that satisfy the following properties.
\begin{enumerate}[(i)]
    \item $\varnothing$ and $\R^n$ are in ${\cal U}$. 
    \item For any subcollection $\mathcal{V} = \{U_\beta\}_{\beta \in B}$ 
    with $U_\beta \in {\cal U}$ for all $\beta \in B$, we have 
    $\bigcup_{\beta \in B} U_\beta \in {\cal U}$. 
    \item For any \emph{finite} subcollection $\{U_{\alpha_1}, \dots, 
    U_{\alpha_m}\} \subset {\cal U}$, we have $\bigcap_{i=1}^m 
    U_{\alpha_i} \in {\cal U}$. 
\end{enumerate}
The sets $U_\alpha \in {\cal U}$ are called the {\bf open sets} of the topology; 
their complements $F_\alpha = \R^n \setminus U_\alpha$ are called the 
{\bf closed sets}. 

Note that the sets $\varnothing$ and $\R^n$ are both open and closed. 
Moreover, the notion of a topology can be extended to more general sets 
$X$, not just $\R^n$. A topology can also be defined starting with closed sets,
but we prefer to work with open sets because many nice properties, such as 
differentiability, are better described with them.

Under the metric topology, we say that a set $A \subset \R^n$ is {\bf open} 
if $A = \varnothing$ or if for all $p \in A$, there exists $r > 0$ 
such that $B_r(p) \subset A$. Moreover, $A$ is {\bf closed} if its 
complement $A^c = \R^n \setminus A$ is open. (We leave it as an 
exercise to show that this is indeed a topology.) 

For example, the open balls $B_r(x)$ are open sets for all 
$x \in \R^n$ and $r > 0$. Indeed, for any point $p \in 
B_r(x)$, one sees that by picking $r' = (r - \|p - x\|)/2$, 
we have $B_{r'}(p) \subset B_r(x)$. 

In general, open sets are described with strict inequalities, while closed 
sets are described using equality or inclusive inequalities. However, note 
that most sets are neither open nor closed, such as the half-open interval 
$U = (-1, 1]$ over $\R$. 

The metric topology is not the only topology on $\R^n$; one example is 
the one consisting of only the sets ${\cal U} = \{\varnothing, \R^n\}$.
However, we generally want more open sets to work with since we might want to 
know the behaviour of functions around a point $p \in \R^n$. If the 
only non-empty open set we had was $\R^n$, then this would apply to all points 
in $\R^n$, which does not yield a lot of information. 

Let $p \in \R^n$. The previous paragraph leads us to the definition of an 
{\bf open neighbourhood} of $p$, which is just an open set $U \subset \R^n$ 
such that $p \in U$. 

We now turn our discussion to vector-valued functions. Let $U \subset \R^n$ 
and consider the vector-valued function 
\begin{align*}
    F : U \subset \R^n &\to B \subset \R^m \\ 
    x = (x_1, \dots, x_n) &\mapsto (F_1(x), \dots, F_m(x)). 
\end{align*}
Then $F$ is continuous if and only if the component functions 
$F_i : U \to \R$ are continuous for all $i = 1, \dots, m$. 

We say that $F$ is a {\bf homeomorphism} if it is a continuous bijection 
whose inverse 
\[ F^{-1} : B \subset \R^m \to U \subset \R^n \] 
is also continuous. For example, the identity map $\Id_{\R^n} : \R^n \to \R^n$ 
and the function $f : \R \to \R$ defined by $f(x) = x^3$ are both homeomorphisms. 

It is a known fact that homeomorphisms map open sets to open sets and closed 
sets to closed sets. This follows from the topological characterization 
of continuity, which states that $F$ is continuous if and only if for 
every open (respectively closed) set $V \subset \R^m$, we have that 
$F^{-1}(V)$ is open (respectively closed). In fact, homeomorphisms preserve 
much more structure than this, as we'll see later. 

\subsection{Topological submanifolds of $\R^n$}\label{subsec:1.2}
We now define the main object we'll be working with in this course.

\begin{defn}{defn:1.1}
    A {\bf $k$-dimensional topological submanifold} (or 
    {\bf topological $k$-submanifold}) of $\R^n$ is a subset $M \subset \R^n$ 
    such that for every $p \in M$, there exists an open neighbourhood $V$ 
    of $p$ in $\R^n$, an open set $U \subset \R^k$, and a homeomorphism 
    \[ \alpha : U \subset \R^k \to V \cap M \subset \R^n. \] 
    The homeomorphism $\alpha$ is called a {\bf coordinate chart} 
    (or {\bf patch}) on $M$.
\end{defn}\vspace{-0.25cm}

Note that the open neighbourhood $V \subset \R^n$ of $p$, the 
open set $U \subset \R^k$, and the map $\alpha$ do not need to be unique.
But we'll see later that the dimension $k$ must be unique and is completely 
determined by $M$.

For example, $\R^n$ is a topological $n$-submanifold of $\R^n$ 
by taking $U = V = \R^n$ and $\alpha = \Id_{\R^n}$. Any open set 
$W \subset \R^n$ is a topological $n$-submanifold of $\R^n$ by 
taking $U = V = W$ and $\alpha = \Id_W$. 

Let's now consider some non-trivial examples. Consider 
\[ M = \{(x, y) \in \R^2 : y = x^2\} \subset \R^2, \] 
which is the graph of the parabola $f(x) = x^2$. Then $M$ is a 
topological $1$-submanifold of $\R^2$ by considering the map 
$\alpha : \R^1 \to M \subset \R^2,\, t \mapsto (t, t^2)$.
The inverse $\alpha^{-1} : M \to \R^1$ is just the projection of the 
first coordinate, which is continuous. 

More generally, let $U \subset \R^k$ be an open set. Consider the graph of a 
continuous function 
\[ F : U \subset \R^k \to \R^{n-k},\, x \mapsto (F_1(x), \dots, F_{n-k}(x)). \]
In other words, we are looking at the set 
\[ G = \{(x, y) \in \R^k \times \R^{n-k} : y = F(x),\, x \in U\} \subset \R^n. \] 
We claim that $G$ is a $k$-dimensional topological submanifold of $\R^n$. To 
see this, define $\alpha : U \subset \R^k \to G \subset \R^n$ by 
$x \mapsto (x, F(x))$. Then $\alpha$ is continuous since $F$ is continuous, 
and it is a bijection since we are restricted to $G$. Moreover, 
it has continuous inverse $\alpha^{-1} : G \subset \R^n \to 
U \subset \R^k,\, (x, y) \mapsto x$. 

Here are two more examples of this in action. 
\begin{enumerate}[(1)]
    \item Let $M = \{(x, y, z) \in \R^3 : z = x^2 + y^2\} \subset \R^3$. 
    Then $M$ is the graph of the continuous function $f(x, y) = x^2 + y^2$, 
    so it is a $2$-dimensional topological submanifold of $\R^3$. 
    \item Observe that $M = \{(x, y, z) \in \R^3 : y = x^2,\, z = x^3\} 
    \subset \R^3$ is the graph of the continuous function $F(t) = (t^2, t^3)$, 
    so it is a $1$-dimensional topological submanifold of $\R^3$.
\end{enumerate}
In all the examples above, we only needed one coordinate chart which 
worked for all points. However, this is not always the case! Consider 
the unit circle 
\[ \mathbb{S}^1 := \{(x, y) \in \R^2 : x^2 + y^2 = 1 \} \subset \R^2. \] 
Note that $\mathbb{S}^1$ is compact. Therefore, by Heine-Borel, 
it is closed and bounded. Recall that homeomorphisms preserve closed sets, 
so it is impossible to find a unique chart $\alpha$. Indeed, if we had 
such a homeomorphism $\alpha : U \subset \R^1 \to \mathbb{S}^1 \subset \R^2$ 
for some open set $U$, then $U = \alpha^{-1}(\mathbb{S}^1)$ would be 
a compact subset of $\R^1$. But the only open and compact subset of 
$\R^n$ is $\varnothing$, which is a contradiction! 

Nonetheless, two coordinate charts are enough to cover all points on 
$\mathbb{S}^1$. Define 
\begin{align*}
    V_1 = \R^2 \setminus \{(x, 0) \in \R^2 : x \leq 0\}, \\ 
    V_2 = \R^2 \setminus \{(x, 0) \in \R^2 : x \geq 0\},
\end{align*}
which are both open sets. Then the homeomorphism 
\[ \alpha_1 : U_1 = (-\pi, \pi) \to \mathbb{S}^1 \cap V_1,\,
 t \mapsto (\cos t, \sin t) \]
covers all points on $\mathbb{S}^1$ except for $(-1, 0)$, while 
\[ \alpha_2 : U_2 = (0, 2\pi) \to \mathbb{S}^1 \cap V_2,\,
 t \mapsto (\cos t, \sin t) \]
covers all points on $\mathbb{S}^1$ except for $(1, 0)$. 

\subsection{More preliminaries} \label{subsec:1.3}
We now introduce another definition from topology.

\begin{defn}{defn:1.2}
    Let $A \subset \R^n$. A subset $U \subset A$ is {\bf relatively open} 
    if it is of the form $U = A \cap U'$ for some open set $U' \subset \R^n$. 
    Similarly, we say that $F \subset A$ is {\bf relatively closed} 
    if $F = A \cap F'$ for some closed set $F' \subset \R^n$. 
\end{defn}\vspace{-0.25cm}

For example, consider $\R$ equipped with the metric topology so that 
the open (respectively closed) sets are the unions of open intervals 
(respectively finite intersections of closed intervals). Let 
$A = [-1, 2) \subset \R$, which is neither open nor closed in $\R$. 
Take $U = [-1, 1) \subset \R$, which is again neither open nor closed in $\R$. 
But $U$ is relatively open in $A$ since $U = A \cap (-3, 1)$. Similarly, 
$F = [-1, 1] = A \cap [-1, 1]$ is relatively closed in $A$. 

Using the language of relatively open and closed sets, a lot of statements 
can be made simpler.
\begin{enumerate}[(1)]
    \item We define an {\bf open neighbourhood of $p$ in $A$} to be a relatively 
    open set $U$ containing $p$.
    \item The relatively open sets form a topology on $A$, called the 
    {\bf relative topology} (verify this as an exercise).
    \item We now have a more concise definition of a topological submanifold. 
    Let $M \subseteq \R^n$. Then $M$ is a {\bf $k$-dimensional topological 
    submanifold} of $\R^n$ if for every $p \in M$, there exists an 
    open neighbourhood $V$ of $p$ in $M$, an open set $U \subset \R^k$, 
    and a homeomorphism $\alpha : U \subset \R^k \to V \subset \R^n$. 
\end{enumerate}

\begin{defn}{defn:1.3}
    Let $A \subset \R^n$. Then $A$ is {\bf connected} if it cannot be written 
    in the form $A = U \cup V$ where $U, V \neq \varnothing$ are relatively 
    open in $A$ and $U \cap V = \varnothing$. Otherwise, we say that $A$ 
    is {\bf disconnected}; we call $U$ and $V$ {\bf disconnecting sets} for $A$.
\end{defn}\vspace{-0.25cm}

Let's go over a few example of connected sets. 
It can be shown that an open set in $\R^n$ is connected if and only if 
it is path connected; that is, there is a path between any two points in the set.
This result can help us build some intuition for what a connected set should 
look like. 
\begin{enumerate}[(1)]
    \item $\R^n$ is connected.
    \item Let $\alpha < \beta \in \R$. Then $(\alpha, \beta)$, 
    $(\alpha, \beta]$, $[\alpha, \beta)$, and $[\alpha, \beta]$ are 
    all connected. 
    \item Observe that $A = (-1, 0] \cup [1, 2]$ is a disconnected set because 
    $(-1, 0] = A \cap (-1.3, 0.3)$ and $[1, 2] = A \cap (0.9, 2.1)$ are 
    both relatively open in $A$ and disjoint. 
    \item The open ball $B_r(p)$ is connected for all $p \in \R^n$ and $r > 0$. 
\end{enumerate}
An important property of connected sets is that the continuous image of a 
connected set is connected! This can be used to prove that a subset 
$M \subset \R^n$ is not a submanifold. 

\begin{enumerate}[(1)]
    \item Consider the {\bf $\alpha$-curve} $C := \{(x, y) \in \R^2 : y^2 = x^2(x+1)\} \subset \R^2$.
    This can be parametrized by the map 
    \begin{align*}
        \alpha : \R &\to C \subset \R^2 \\
        t &\mapsto (t^2 - 1, t(t^2 - 1)).
    \end{align*}
    Note that $\alpha$ is not injective since $\alpha(-1) = \alpha(1) = (0, 0)$.
    That is, $\alpha$ is not a homeomorphism on $\R$, but it becomes one 
    if we remove the points $t = \pm1$, whose inverse is 
    \begin{align*}
        \alpha^{-1} : C \setminus \{(0, 0)\} &\to \R \setminus \{\pm1\} \subset \R \\ 
        (x, y) &\mapsto 1/x. 
    \end{align*} 
    Thus, $C$ is a $1$-dimensional submanifold away from the point $(0, 0)$. 
    
    Our goal now is to show that the whole of $C$ is not a topological submanifold 
    of $\R^2$. By contradiction, suppose that it were. By our above discussion, 
    it must have dimension $1$ because it has dimension $1$ away from $(0, 0)$. 
    Since $(0, 0) \in C$, it follows from the definition that there exists an 
    open neighbourhood $V$ of $(0, 0)$ in $C$, 
    an open set $U \subset \R^1$, and a homeomorphism 
    \[ \alpha : U \subset \R^2 \to V \subset C. \] 
    There must be a unique point $t_0 \in U$ such that $\alpha(t_0) = (0, 0)$ 
    since $\alpha$ is a bijection. Since $U$ is open and $t_0 \in U$, there 
    exists $\eps > 0$ such that $B_\eps(t_0) \subset U$. But $U \subset \R^1$, 
    so $B_\eps(t_0) = (t_0 - \eps, t_0 + \eps) =: U'$. Then $\alpha|_{U'}$ 
    is also a homeomorphism. Let $V' = \alpha(U')$. Observe that 
    $V' \setminus \{(0, 0)\}$ has three or four pieces depending on how large 
    the open set $U'$ is: one on the top right quadrant, one on the bottom right 
    quadrant, and one or two on the left of the $y$-axis. On the other hand, 
    $U' \setminus \{t_0\}$ has only two components, contradicting the fact that 
    homeomorphisms preserve the number of connected components. 

    \item Consider the {\bf double cone} $M = \{(x, y, z) \in \R^3 : 
    z^2 = x^2 + y^2\} \subset \R^3$. Away from $(0, 0, 0)$, every point in $M$ 
    lies on the graph of one of the continuous functions $f_1(x, y) =
    \sqrt{x^2+y^2}$ or $f_2(x, y) = -\sqrt{x^2 + y^2}$. Therefore, 
    $M \setminus \{(0, 0, 0)\}$ is a $2$-dimensional topological submanifold 
    of $\R^3$ since $f_1$ and $f_2$ are both functions of two variables. 

    However, there is a problem at the point $(0, 0, 0)$ since it 
    lies on the graph of both $f_1$ and $f_2$. Suppose that $M$ 
    is a topological submanifold of $\R^3$. Then $M$ must necessarily be of 
    dimension $2$ because $M \setminus \{(0, 0, 0)\}$ is of dimension $2$. 
    Then by definition, there exists an open neighbourhood $V$ of $(0, 0, 0)$, an 
    open set $U \subset \R^2$, and a homeomorphism 
    \[ \alpha : U \subset \R^2 \to V \subset M. \] 
    Since $\alpha$ is a bijection, there exists a unique point $(x_0, y_0) \in U$ 
    such that $\alpha(x_0, y_0) = (0, 0, 0)$. After shrinking $U$ (by the 
    same argument as above), we may take $U' = B_{\eps}((x_0, y_0))$ to 
    ensure that we have a connected set. Consider now the restriction 
    $\alpha|_{U'} : U' \to V = \alpha(U')$. Then $U' \setminus \{(x_0, y_0)\}$ 
    has one component, whereas $V' \setminus \{(0, 0, 0)\}$ has two components
    (that is, it is disconnected), which is a contradiction. 
\end{enumerate}

\newpage 
Now, we want to prove the invariance of dimension.
\begin{theo}[Invariance of Dimension]{theo:1.4}
    $\R^m$ is homeomorphic to $\R^n$ if and only if $m = n$. 
\end{theo}\vspace{-0.25cm}
If $m = n$, then $\R^m = \R^n$, so there is nothing to prove. The other 
implication is much harder, and we'll need the following result. 

\begin{theo}[Brouwer Invariance of Domain]{theo:1.5}
    Let $U \subset \R^n$ be open, and let $f : U \subset \R^n \to \R^n$ 
    be an injective continuous map. Then $f(U) \subset \R^n$ is open. 
    In particular, $f$ is a homeomorphism onto its image.
\end{theo}\vspace{-0.25cm}

An elementary proof can be found on 
\href{https://terrytao.wordpress.com/2011/06/13/brouwers-fixed-point-and-invariance-of-domain-theorems-and-hilberts-fifth-problem/}{Terry Tao's blog} 
where he uses the Brouwer Fixed Point Theorem to prove it. Nowadays, the 
standard proof uses algebraic topology. 

It is important that both the domain and codomain involve the same dimension $n$. 
For example, consider the injective continuous map $f : \R \to \R^2$ defined by 
$x \mapsto (x, 0)$. Observe that $f(U)$ is the $x$-axis, which is not open in $\R^2$. 

\begin{pf}[Theorem~\ref{theo:1.4}]
    We proceed by contradiction. Suppose that there is a homeomorphism 
    $f : \R^m \to \R^n$ and that $m > n$. Consider the inclusion 
    \begin{align*}
        \iota : \R^n &\to \R^m \\ 
        (x_1, \dots, x_n) &\mapsto (x_1, \dots, x_m, 0, \dots 0),
    \end{align*} 
    which is an injective continuous map. Then $\iota \circ f : \R^m 
    \to \R^m$ is also an injective continuous map since it is the composition of 
    two injective continuous maps. By Theorem~\ref{theo:1.5}, we have that 
    $\iota \circ f(\R^m)$ is an open set in $\R^m$. But this is impossible 
    because if $(x_1, \dots, x_n, 0, \dots, 0) \in \iota \circ f(\R^m)$, 
    then 
    \[ (x_1, \dots, x_n, \eps/2, 0, \dots, 0) \notin \iota \circ f(\R^m) \]  
    for all $\eps > 0$. Then $B_\eps((x_1, \dots, x_n, 0, \dots, 0)) 
    \not\subset \iota \circ f(\R^m)$ for all $\eps > 0$, implying that 
    $\iota \circ f(\R^m)$ is not open in $\R^m$. Therefore, we must have 
    $m \leq n$. If $n < m$, then we can repeat the same argument with 
    $f^{-1} : \R^n \to \R^m$, which again leads to a contradiction. 
    We conclude that $n = m$. \qed 
\end{pf}\vspace{-0.25cm}

Note that we actually proved something stronger: if $m > n$ and $U$ is a 
nonempty open subset of $\R^m$, then there is no continuous mapping from 
$U$ to $\R^n$. As a consequence, we get the following. 
\begin{prop}{prop:1.6}
    If $M \subset \R^n$ is a $k$-dimensional topological submanifold of 
    $\R^n$, then $k \leq n$. 
\end{prop}\vspace{-0.25cm}
\begin{pf}[Proposition~\ref{prop:1.6}]
    If $M \subset \R^n$ is a $k$-dimensional topological submanifold of $\R^n$, 
    then for all $p \in M$, there exists an open set $U \subset \R^k$, 
    an open neighbourhood $V \subset M$ of $p$, and a homeomorphism
    \[ \alpha : U \subset \R^k \to V \subset M \subset \R^n. \] 
    Since $\alpha$ is an injective continuous map, this forces $k \leq n$ 
    by the above discussion. \qed
\end{pf}\vspace{-0.25cm}
Finally, we must have the same $k$ for any chart $\alpha : U \subset \R^k 
\to V \subset M$. Indeed, let $p \in M$, and suppose that we have two 
different charts, say $\alpha : U \subset \R^k \to V \subset M$ and 
$\beta : U' \subset \R^{k'} \to V' \subset M$ where $p \in V \cap V'$. 
Then $V \cap V' \neq \varnothing$, so we can consider the restrictions 
\begin{align*}
    \alpha|_{\alpha^{-1}(V \cap V')} &: \alpha^{-1}(V \cap V') \to V \cap V', \\
    \beta|_{\beta^{-1}(V \cap V')} &: \beta^{-1}(V \cap V') \to V \cap V'.
\end{align*} 
Then $\beta^{-1} \circ \alpha : \alpha^{-1}(V \cap V') \subset \R^k 
\to \beta^{-1}(V \cap V') \subset \R^{k'}$ is a homeomorphism. Hence, 
$\alpha^{-1}(V \cap V')$ is a $k$-dimensional topological submanifold
of $\R^{k'}$. By Proposition~\ref{prop:1.6}, we have $k \leq k'$. 
Similarly, $\alpha^{-1} \circ \beta : \beta^{-1}(V \cap V') \subset 
\R^{k'} \to \alpha^{-1}(V \cap V') \subset \R^k$ is a homeomorphism, so 
$\beta^{-1}(V \cap V')$ is a $k'$-dimensional submanifold of $\R^k$. 
It follows that $k' \leq k$ and so $k' = k$. 

\subsection{Submanifolds of $\R^n$ of class $C^r$} \label{subsec:1.4}
Let $U \subset \R^n$ be an open set and consider the vector-valued function 
\begin{align*}
    F : U \subset \R^n &\to \R^m \\ 
    x = (x_1, \dots, x_n) &\mapsto (F_1(x), \dots, F_m(x)). 
\end{align*}
Recall that $F$ is of {\bf class $C^r$} for $r \geq 1$ if each component 
function $F_i : U \subset \R^n \to \R$ is of class $C^r$. That is, 
the partial derivatives of $F_i$ exist and are continuous up to order $r$. 
Also, we say that $F$ is of class $C^\infty$ or {\bf smooth} if each 
$F_i$ is smooth (the partial derivatives exist up to any order).
\begin{enumerate}[(1)]
    \item All polynomials are smooth. 
    \item The function $f(x) = x^{4/3}$ is of class $C^1$. Its derivative 
    $f'(x) = \frac43 x^{1/3}$ is continuous, but the second derivative 
    $f''(x) = \frac49 x^{-2/3}$ is not defined at $x = 0$. 
    \item The vector-valued function $F(x, y) = (2\cos x, xy - 1, e^{2\sin y+x})$ 
    is smooth on $\R^2$ because each component function is smooth. 
\end{enumerate}
The {\bf partial derivative} of $F$ with respect to the variable $x_j$ is 
\[ \frac{\partial F}{\partial x_j} := \left( \frac{\partial F_1}{\partial x_j}, 
\dots, \frac{\partial F_m}{\partial x_j} \right). \] 
If we fix a component function $F_i$, its {\bf gradient} is 
\[ \nabla F_i := \left( \frac{\partial F_i}{\partial x_1}, 
\dots, \frac{\partial F_i}{\partial x_n} \right). \] 
The {\bf derivative matrix} or {\bf Jacobian matrix} of $F$ is the $m \times n$ 
matrix 
\[ DF := \begin{bmatrix} 
    \partial F_1/\partial x_1 & \cdots & \partial F_1/\partial x_n \\ 
    \vdots & \ddots & \vdots \\ 
    \partial F_m/\partial x_1 & \cdots & \partial F_m/\partial x_n
\end{bmatrix}. \] 
That is, the rows correspond to the component functions $F_i$, and the 
columns correspond to the variables $x_j$. We can also think of the 
rows as the gradients and the columns as the partial derivatives; that is, 
we have 
\[ DF = \mleft[ \begin{array}{c|c|c}
    \dfrac{\partial F}{\partial x_1} & \cdots & \dfrac{\partial F}{\partial x_n}
\end{array} \mright] = \mleft[ \begin{array}{c}
    \nabla F_1 \\ \hline 
    \vdots \\[2pt] \hline 
    \nabla F_m \\
\end{array} \mright] \] 
In general, we want to work with some differentiability. 
This leads to the following definition. 

\begin{defn}{defn:1.7}
    Let $M \subset \R^n$. Suppose that for every $p \in M$, there exists 
    an open neighbourhood $V$ of $p$ in $M$, an open subset $U \subset \R^k$, 
    and a homeomorphism $\alpha : U \subset \R^k \to V \subset M$ such that 
    \begin{enumerate}[(1)]
        \item $\alpha$ is of class $C^r$ for some $r \geq 1$; 
        \item $D\alpha(x)$ has rank $k$ for all $x \in U$. 
    \end{enumerate}
    Then $M$ is called a {\bf $k$-dimensional submanifold of $\R^n$ of 
    class $C^r$}. We call $\alpha$ a {\bf coordinate chart} (or 
    {\bf coordinate patch}) about $p$.
\end{defn}\vspace{-0.25cm}

Note that every submanifold of class $C^r$ is a topological submanifold. 
We are only imposing the extra conditions (1) and (2) on the coordinate charts. 
We will see that condition (2) will allow us to define tangent spaces to 
the submanifolds at every point. A submanifold of class $C^\infty$ is 
called a {\bf smooth submanifold}. 

As usual, let's go over some examples. 
\begin{enumerate}[(1)]
    \item Let $U \subset \R^n$ be open. Then $\alpha : U \subset \R^n \to 
    V = U \subset \R^n$ sending $x$ to itself is smooth. Since the 
    component functions are $F_i(x) = x_i$ for all $i = 1, \dots, n$, we have 
    \[ D\alpha(x) = \left[ \frac{\partial F_i}{\partial x_j} \right] 
    = \left[ \frac{\partial x_i}{\partial x_j} \right] = [\delta_{ij}], \] 
    where $\delta_{ij}$ is the Kronecker delta. In other words, $D\alpha(x)$ 
    is the $n \times n$ identity matrix and has rank $n$ for all $x \in U$, 
    so $U \subset \R^n$ is a smooth $n$-dimensional submanifold of $\R^n$. 

    \item {\bf Graphs of functions of class $C^r$.} Let $U \subset \R^k$ 
    be an open set and consider a function 
    \begin{align*}
        F : U \subset \R^k &\to \R^{n-k} \\ 
        (x_1, \dots, x_k) &\mapsto (F_1(x), \dots, F_{n-k}(x)) 
    \end{align*} 
    of class $C^r$ (so each $F_i$ is of class $C^r$). Let 
    \[ M = \{(x, F(x)) \in \R^k \times \R^{n-k} : x \in U\} \subset \R^n \] 
    be the graph of $F$. We have already seen that $M$ is a $k$-dimensional 
    submanifold of $\R^n$ by taking $V = M$ and the homeomorphism 
    $\alpha : U \subset \R^k \to V \subset \R^n$ defined by 
    \[ \alpha(x) = (x, F(x)) = (x_1, \dots, x_k, F_1(x), \dots, F_{n-k}(x)). \]
    In particular, $\alpha$ is of class $C^r$ since the identity components 
    are smooth and the $F_i$ are of class $C^r$. Let's look at the 
    derivative matrix in terms of the columns of partial derivatives. We have 
    \[ \frac{\partial\alpha}{\partial x_j} = 
    \left( 0, \dots, 0, 1, 0, \dots, 0, \frac{\partial F_1}{\partial x_j}, 
    \dots, \frac{\partial F_{n-k}}{\partial x_j} \right) \] 
    where the $1$ corresponds to the $j$-th component, and hence 
    \[ D\alpha(x) = \mleft[ \begin{array}{cccc}
        1 & 0 & \cdots & 0 \\ 
        0 & 1 & \cdots & 0 \\ 
        \vdots & \vdots & \ddots & \vdots \\ 
        0 & 0 & \cdots & 1 \\ \hline 
        \partial F_1/\partial x_1 & \partial F_1/\partial x_2 & \cdots & \partial F_1/\partial x_k \\ 
        \vdots & \vdots & \ddots & \vdots \\ 
        \partial F_{n-k}/\partial x_1 & \partial F_{n-k}/\partial x_2 & \cdots & \partial F_{n-k}/\partial x_k \\
    \end{array} \mright] = \mleft[ \begin{array}{c} I_{k\times k} \\\hline DF(x) \end{array} \mright]. \]
    This matrix has rank $k$ for all $x \in U$, so $M$ is a $k$-dimensional 
    submanifold of class $C^r$. 

    \item We saw that the circle $\mathbb{S}^1 = \{(x, y) \in \R^2 : x^2 + y^2 = 1\} 
    \subset \R^2$ was a $1$-dimensional topological submanifold using
    the charts $\alpha_1 : U_1 = (-\pi, \pi) \subset \R^1 \to 
    \mathbb{S}^1 \setminus \{(-1, 0)\} \subset \R^2$ and 
    $\alpha_2 : U_2 = (0, 2\pi) \subset \R^1 \to \mathbb{S}^1 \setminus 
    \{(1, 0)\} \subset \R^2$, both defined by $t \mapsto (\cos t, \sin t)$. 
    Note that both $\alpha_i$ are smooth functions with derivative matrix 
    \[ D\alpha_i = \left[ \frac{\textrm{d}\alpha_i}{\textrm{d}t} \right] = 
    \begin{bmatrix} -\sin t \\ \cos t \end{bmatrix}, \] 
    which has rank $1$ because $\sin t$ and $\cos t$ don't have the same zeroes, 
    and hence $D\alpha_i$ is never the zero vector. Thus, $\mathbb{S}^1$ 
    is a smooth $1$-dimensional submanifold of $\R^2$. 
\end{enumerate}
Not every topological submanifold of $\R^n$ is of class $C^r$ 
for some $r \geq 1$. For example, consider the graph 
\[ M = \{(x, |x|) : x \in \R\} \] 
of the function $f(x) = |x|$ on $\R$. Since $f$ is continuous, we know that 
$M$ is a $1$-dimensional topological submanifold of $\R^2$. Note that $f$ is 
smooth away from $x = 0$, so $M \setminus \{(0, 0)\}$ is a smooth $1$-dimensional
submanifold of $\R^2$. However, we claim that it cannot be a submanifold of 
class $C^r$ on any neighbourhood of the point $(0, 0)$. 

% TODO: Continue from end of Lecture 7