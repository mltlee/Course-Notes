\section{Submanifolds of $\R^n$}\label{sec:1}

\subsection{Preliminaries}\label{subsec:1.1}
To begin, we'll recall some facts about the topology of $\R^n$ and 
vector-valued functions.

In this course, we'll be working with the metric topology with respect 
to the Euclidean norm (or metric). Let $x = (x_1, \dots, x_n),\,
y = (y_1, \dots, y_n) \in \R^n$.
The {\bf Euclidean norm} is defined to be 
\[ \|x\| = \sqrt{x_1^2 + \cdots + x_n^2}, \] 
and {\bf Euclidean distance} is given by 
\[ \dist(x, y) = \|y - x\| = \sqrt{(y_1 - x_1)^2 + 
\cdots + (y_n - x_n)^2}. \] 
We define the {\bf open ball} of radius $r > 0$ centered at $x \in \R^n$ by 
\[ B_r(x) := \{y \in \R^n : \dist(x, y) < r\} \subset \R^n. \] 
A {\bf topology} on $\R^n$ is a collection $\mathcal{U} = \{U_\alpha\}_{\alpha \in A}$ 
of subsets $U_\alpha \subset \R^n$ that satisfy the following properties.
\begin{enumerate}[(i)]
    \item $\varnothing$ and $\R^n$ are in ${\cal U}$. 
    \item For any subcollection $\mathcal{V} = \{U_\beta\}_{\beta \in B}$ 
    with $U_\beta \in {\cal U}$ for all $\beta \in B$, we have 
    $\bigcup_{\beta \in B} U_\beta \in {\cal U}$. 
    \item For any \emph{finite} subcollection $\{U_{\alpha_1}, \dots, 
    U_{\alpha_m}\} \subset {\cal U}$, we have $\bigcap_{i=1}^m 
    U_{\alpha_i} \in {\cal U}$. 
\end{enumerate}
The sets $U_\alpha \in {\cal U}$ are called the {\bf open sets} of the topology; 
their complements $F_\alpha = \R^n \setminus U_\alpha$ are called the 
{\bf closed sets}. 

Note that the sets $\varnothing$ and $\R^n$ are both open and closed. 
Moreover, the notion of a topology can be extended to more general sets 
$X$, not just $\R^n$. A topology can also be defined starting with closed sets,
but we prefer to work with open sets because many nice properties, such as 
differentiability, are better described with them.

Under the metric topology, we say that a set $A \subset \R^n$ is {\bf open} 
if $A = \varnothing$ or if for all $p \in A$, there exists $r > 0$ 
such that $B_r(p) \subset A$. Moreover, $A$ is {\bf closed} if its 
complement $A^c = \R^n \setminus A$ is open. (We leave it as an 
exercise to show that this is indeed a topology.) 

For example, the open balls $B_r(x)$ are open sets for all 
$x \in \R^n$ and $r > 0$. Indeed, for any point $p \in 
B_r(x)$, one sees that by picking $r' = (r - \|p - x\|)/2$, 
we have $B_{r'}(p) \subset B_r(x)$. 

In general, open sets are described with strict inequalities, while closed 
sets are described using equality or inclusive inequalities. However, note 
that most sets are neither open nor closed, such as the half-open interval 
$U = (-1, 1]$ over $\R$. 

The metric topology is not the only topology on $\R^n$; one example is 
the one consisting of only the sets ${\cal U} = \{\varnothing, \R^n\}$.
However, we generally want more open sets to work with since we might want to 
know the behaviour of functions around a point $p \in \R^n$. If the 
only non-empty open set we had was $\R^n$, then this would apply to all points 
in $\R^n$, which does not yield a lot of information. 

Let $p \in \R^n$. The previous paragraph leads us to the definition of an 
{\bf open neighbourhood} of $p$, which is just an open set $U \subset \R^n$ 
such that $p \in U$. 

We now turn our discussion to vector-valued functions. Let $U \subset \R^n$ 
and consider the vector-valued function 
\begin{align*}
    F : U \subset \R^n &\to B \subset \R^m \\ 
    x = (x_1, \dots, x_n) &\mapsto (F_1(x), \dots, F_m(x)). 
\end{align*}
Then $F$ is continuous if and only if the component functions 
$F_i : U \to \R$ are continuous for all $i = 1, \dots, m$. 

We say that $F$ is a {\bf homeomorphism} if it is a continuous bijection 
whose inverse 
\[ F^{-1} : B \subset \R^m \to U \subset \R^n \] 
is also continuous. For example, the identity map $\Id_{\R^n} : \R^n \to \R^n$ 
and the function $f : \R \to \R$ defined by $f(x) = x^3$ are both homeomorphisms. 

It is a known fact that homeomorphisms map open sets to open sets and closed 
sets to closed sets. This follows from the topological characterization 
of continuity, which states that $F$ is continuous if and only if for 
every open (respectively closed) set $V \subset \R^m$, we have that 
$F^{-1}(V)$ is open (respectively closed). In fact, homeomorphisms preserve 
much more structure than this, as we'll see later. 

\subsection{Topological Submanifolds of $\R^n$}\label{subsec:1.2}
We now define the main object we'll be working with in this course.

\begin{defn}{defn:1.1}
    A {\bf $k$-dimensional topological submanifold} (or 
    {\bf topological $k$-submanifold}) of $\R^n$ is a subset $M \subset \R^n$ 
    such that for every $p \in M$, there exists an open neighbourhood $V$ 
    of $p$ in $\R^n$, an open set $U \subset \R^k$, and a homeomorphism 
    \[ \alpha : U \subset \R^k \to V \cap M \subset \R^n. \] 
    The homeomorphism $\alpha$ is called a {\bf coordinate chart} 
    (or {\bf patch}) on $M$.
\end{defn}

Note that the open neighbourhood $V \subset \R^n$ of $p$, the 
open set $U \subset \R^k$, and the map $\alpha$ do not need to be unique.
But we'll see later that the dimension $k$ must be unique and is completely 
determined by $M$.

For example, $\R^n$ is a topological $n$-submanifold of $\R^n$ 
by taking $U = V = \R^n$ and $\alpha = \Id_{\R^n}$. Any open set 
$W \subset \R^n$ is a topological $n$-submanifold of $\R^n$ by 
taking $U = V = W$ and $\alpha = \Id_W$. 

Let's now consider some non-trivial examples. Consider 
\[ M = \{(x, y) \in \R^2 : y = x^2\} \subset \R^2, \] 
which is the graph of the parabola $f(x) = x^2$. Then $M$ is a 
topological $1$-submanifold of $\R^2$ by considering the map 
$\alpha : \R^1 \to M \subset \R^2,\, t \mapsto (t, t^2)$.
The inverse $\alpha^{-1} : M \to \R^1$ is just the projection of the 
first coordinate, which is continuous. 

More generally, let $U \subset \R^k$ be an open set. Consider the graph of a 
continuous function 
\[ F : U \subset \R^k \to \R^{n-k},\, x \mapsto (F_1(x), \dots, F_{n-k}(x)). \]
In other words, we are looking at the set 
\[ G = \{(x, y) \in \R^k \times \R^{n-k} : y = F(x),\, x \in U\} \subset \R^n. \] 
We claim that $G$ is a $k$-dimensional topological submanifold of $\R^n$. To 
see this, define $\alpha : U \subset \R^k \to G \subset \R^n$ by 
$x \mapsto (x, F(x))$. Then $\alpha$ is continuous since $F$ is continuous, 
and it is a bijection since we are restricted to $G$. Moreover, 
it has continuous inverse $\alpha^{-1} : G \subset \R^n \to 
U \subset \R^k,\, (x, y) \mapsto x$. 

Here are two more examples of this in action. 
\begin{enumerate}[(1)]
    \item Let $M = \{(x, y, z) \in \R^3 : z = x^2 + y^2\} \subset \R^3$. 
    Then $M$ is the graph of the continuous function $f(x, y) = x^2 + y^2$, 
    so it is a $2$-dimensional topological submanifold of $\R^3$. 
    \item Observe that $M = \{(x, y, z) \in \R^3 : y = x^2,\, z = x^3\} 
    \subset \R^3$ is the graph of the continuous function $F(t) = (t^2, t^3)$, 
    so it is a $1$-dimensional topological submanifold of $\R^3$.
\end{enumerate}
In all the examples above, we only needed one coordinate chart which 
worked for all points. However, this is not always the case! Consider 
the unit circle 
\[ \mathbb{S}^1 := \{(x, y) \in \R^2 : x^2 + y^2 = 1 \} \subset \R^2. \] 
Note that $\mathbb{S}^1$ is compact. Therefore, by Heine-Borel, 
it is closed and bounded. Recall that homeomorphisms preserve closed sets, 
so it is impossible to find a unique chart $\alpha$. Indeed, if we had 
such a homeomorphism $\alpha : U \subset \R^1 \to \mathbb{S}^1 \subset \R^2$ 
for some open set $U$, then $U = \alpha^{-1}(\mathbb{S}^1)$ would be 
a compact subset of $\R^1$. But the only open and compact subset of 
$\R^n$ is $\varnothing$, which is a contradiction! 

Nonetheless, two coordinate charts are enough to cover all points on 
$\mathbb{S}^1$. Define 
\begin{align*}
    V_1 = \R^2 \setminus \{(x, 0) \in \R^2 : x \leq 0\}, \\ 
    V_2 = \R^2 \setminus \{(x, 0) \in \R^2 : x \geq 0\},
\end{align*}
which are both open sets. Then the homeomorphism 
\[ \alpha_1 : U_1 = (-\pi, \pi) \to \mathbb{S}^1 \cap V_1,\,
 t \mapsto (\cos t, \sin t) \]
covers all points on $\mathbb{S}^1$ except for $(-1, 0)$, while 
\[ \alpha_2 : U_2 = (0, 2\pi) \to \mathbb{S}^1 \cap V_2,\,
 t \mapsto (\cos t, \sin t) \]
covers all points on $\mathbb{S}^1$ except for $(1, 0)$. 