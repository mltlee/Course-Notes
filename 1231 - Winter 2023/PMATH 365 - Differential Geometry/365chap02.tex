\section{Curves in $\R^n$}\label{sec:2}

\subsection{Introduction to curves}\label{subsec:2.1}
What is a curve? Intuitively, it is a $1$-dimensional subset of $\R^n$. 

Observe that the level sets $f(x, y) = k$ of a two variable function over $\R^2$ are curves. 
\begin{enumerate}[(1)]
    \item For the function $f(x, y) = x^2 + y^2$, the level sets 
    $C = \{(x, y) \in \R^2 : x^2 + y^2 = k\}$ are circles centered at 
    $(0, 0)$ of radius $\sqrt{k}$. 
    \item For the function $f(x, y) = x^2 - y$, the level sets are of the form 
    $C = \{(x, y) \in \R^2 : x^2 - y = k\}$.
    This is simply the parabola $y = x^2 - k$.
\end{enumerate}

The intersection of two surfaces in $\R^3$ is also a curve. 
\begin{enumerate}[(1)]
    \item The intersection of $z = x^2 + y^2$ and $z = 2$ gives the circle 
    $x^2 + y^2 = 2$ in the plane $z = 2$. 

    \item {\bf (Twisted cubic.)} Let $C = \{(x, y, z) \in \R^3 : 
    y = x^2,\, z = x^3\}$, which is the intersection of the surfaces 
    $y = x^2$ and $z = x^3$ in $\R^3$. This can be described using the parametrized 
    curve $\gamma : \R \to C \subset \R^3$ defined by $t \mapsto (t, t^2, t^3)$.
\end{enumerate}

We will work with parametrized curves, which are 
vector-valued functions $\gamma : I = (\alpha, \beta) \to \R^n$ of class $C^r$.

{\bf (Circular helix.)} Let $\gamma(t) = (a\cos t, a\sin t, bt)$ for $t \in \R$ and $a, b > 0$. 
Setting $x = a\cos t$ and $y = a\sin t$, we see that $x^2 + y^2 = a^2$, so 
$\gamma(t)$ lies on the cylinder $x^2 + y^2 = a^2$ in $\R^3$. 
We have $\gamma(0) = (a, 0, 0)$ and $\gamma(\pi/2) = (0, a, b\pi/2)$. (The circular helix 
looks like a spiral along the cylinder.)

\begin{defn}{defn:2.1}
    A parameterized curve $\gamma : (\alpha, \beta) \to \R^n$ of class $C^r$ is called 
    {\bf regular} if for all $t \in (\alpha, \beta)$, we have 
    \[ \gamma'(t) = \frac{{\rm d}\gamma}{{\rm d}t}(t) \neq 0. \] 
    We call $\|\gamma'(t)\|$ the {\bf speed of $\gamma$ 
    at $\gamma(t)$} and we say that $\gamma$ is {\bf unit speed} if $\|\gamma'(t)\| = 1$ for all 
    $t \in (\alpha, \beta)$. 
\end{defn}\vspace{-0.25cm}
Note that unit speed implies regular because $\|x\| = 0$ if and only if $x = 0$. 
Let's look at some examples. 
\begin{enumerate}[(1)]
    \item {\bf (Circle.)} Let $\gamma(t) = (\cos(ct), \sin(ct))$ for $t \in \R$, 
    where $c \in \R$. Then $\gamma'(t) = c(-\sin(ct), \cos(ct))$ for all $t \in \R$,
    which implies that $\|\gamma'(t)\| = c$. In particular, $\gamma$ is 
    regular when $c \neq 0$, and is unit speed if and only if $c = 1$. 
    
    \item {\bf (Circular helix.)} Given $a, b > 0$, let $\gamma(t) = 
    (a\cos t, a\sin t, bt)$ for $t \in \R$ as before. Then we have $\gamma'(t) = 
    (-a\sin t, a\cos t, b)$ for all $t \in \R$, which implies that 
    \[ \|\gamma'(t)\| = \sqrt{a^2 + b^2} > 0. \] 
    This is a constant; we see that $\gamma$ is regular for all $a, b > 0$, 
    and unit speed if and only if $a^2 + b^2 = 1$. 

    \item {\bf (Twisted cubic.)} Consider $\gamma(t) = (t, t^2, t^3)$ for $t \in \R$. 
    Then $\gamma'(t) = (1, 2t, 3t^2)$ for all $t \in \R$, so 
    \[ \|\gamma'(t)\| = \sqrt{1 + 4t^2 + 9t^4} \geq 1. \] 
    Notice that $\gamma$ is regular, but it is not 
    a unit speed curve since $\|\gamma'(t)\| = 1$ if and only if $t = 0$. 
\end{enumerate}
The following result will be useful in our study of unit speed curves. 

\begin{prop}{prop:2.2}
    Let $u : (\alpha, \beta) \to \R^n$ be a vector-valued function of class $C^r$ 
    such that $\|u(t)\| = 1$ for all $t \in (\alpha, \beta)$. Then 
    $u(t) \cdot u'(t) = 0$ for all $t \in (\alpha, \beta)$, so
    we have $u'(t) = 0$ or $u(t) \perp u'(t)$.
\end{prop}\vspace{-0.25cm} 
\begin{pf}[Proposition~\ref{prop:2.2}]
    Suppose that $u(t) = (u_1(t), \dots, u_n(t))$. Then 
    $u'(t) = (u'_1(t), \dots, u'_n(t))$. Since $\|u(t)\| = 1$, we have that 
    \[ 1 = u(t) \cdot u(t) = \sum_{i=1}^n (u_i(t))^2. \]
    Differentiating this equation with respect to $t$ gives 
    \begin{align*}
        0 = \frac{\rm d}{{\rm d}t} \left( \sum_{i=1}^n (u_i(t))^2 \right)
        = \sum_{i=1}^n \frac{\rm d}{{\rm d}t} ((u_i(t))^2) 
        = \sum_{i=1}^n 2u_i(t)u'_i(t) = 2u(t) \cdot u'(t). 
    \end{align*}
    This means that $u(t) \cdot u'(t) = 0$ for all $t \in (\alpha, \beta)$, 
    as desired. \qed 
\end{pf}\vspace{-0.25cm} 

In particular, if $\gamma$ is a unit speed curve, then $\gamma'(t) = 0$ 
or $\gamma'(t) \perp \gamma''(t)$ by setting $u(t) = \gamma'(t)$ in 
Proposition~\ref{prop:2.2}. Note that this result also holds if for 
some constant $c \in \R$, we have $\|\gamma'(t)\| = c$ for all $t \in (\alpha, \beta)$.

How do we measure how much a curve ``curves''? We first make a few definitions.

\begin{defn}{defn:2.3}
    Let $\gamma : (\alpha, \beta) \to \R^n$ be a unit speed curve. 
    \begin{itemize}
        \item We define the {\bf unit tangent vector to $\gamma$ at $\gamma(s)$} 
        to be 
        \[ T(s) := \gamma'(s). \] 
        \item The {\bf curvature of $\gamma$ at $\gamma(s)$} is defined to be 
        \[ \kappa(s) := \|\gamma''(s)\|. \] 
        \item If $\kappa(s) > 0$, the {\bf principal unit normal to $\gamma$ 
        at $\gamma(s)$} is 
        \[ N(s) := \frac{\gamma''(s)}{\kappa(s)}, \] 
        and the {\bf radius of curvature at $\gamma(s)$} is $\rho(s) := 1/\kappa(s)$. 
    \end{itemize}
\end{defn}

We make a few remarks about these definitions.
\begin{enumerate}[(1)]
    \item Observe that $T(s)$ is the unique unit vector tangent to $\gamma$ 
    at $\gamma(s)$ that points in the direction in which we are traveling 
    along the curve.

    \item Since $\gamma$ is a unit speed curve, Proposition~\ref{prop:2.2}
    implies $\gamma'(s) \cdot \gamma''(s) = 0$. Assuming that the 
    acceleration $\gamma''(s)$ is nonzero, we must have 
    $\gamma'(s) \perp \gamma''(s)$ so that $T(s) \perp N(s)$. 

    \item Note that $\kappa(s) \geq 0$. Whenever $\kappa(s) > 0$, 
    we see that $N(s) = \gamma''(s)/\kappa(s)$ points in the same 
    direction as $\gamma''(s)$, namely the direction that the velocity 
    vectors are changing.
\end{enumerate}

Next, let's look at some simple examples. 
\begin{enumerate}[(1)]
    \item Let $\gamma : (\alpha, \beta) \to \R^n$ be a unit speed curve 
    of class $C^r$ where $r \geq 2$. Then the image of $\gamma$ in $\R^n$ 
    is a line if and only if $\kappa(s) = 0$ for all $s \in (\alpha, \beta)$. 

    {\bf Proof.} Note that $\kappa(s) = 0$ if and only if $\|\gamma''(s)\| = 0$, 
    which is equivalent to $\gamma''(s) = 0$ for all $s \in (\alpha, \beta)$. 
    This happens if and only if $\gamma'(s) = v$ for some fixed 
    $0 \neq v \in \R^n$ (it is nonzero since $\gamma$ is unit speed), and 
    hence $\gamma(s) = sv + x_0$ for some $x_0 \in \R^n$ and $s \in 
    (\alpha, \beta)$. \qed 

    \item {\bf (Circle.)} For $a > 0$ and $s \in \R$, set 
    $\gamma(s) = (a\cos(s/a), a\sin(s/a))$. Then $\gamma'(s) = 
    (-\sin(s/a), \cos(s/a))$ with $\|\gamma'(s)\| = 1$, so 
    $\gamma$ is unit speed with 
    \[ T(s) = \gamma'(s) = (-\sin(s/a), \cos(s/a)). \] 
    Next, we have $\gamma''(s) = (-\frac1a\cos(s/a), -\frac1a\sin(s/a))$, so 
    \[ \kappa(s) = \|\gamma'(s)\| = 1/a > 0 \] 
    and $\gamma$ has constant and positive curvature. From this, we compute 
    \[ N(s) = \frac{\gamma''(s)}{\kappa(s)} = (-\cos(s/a), -\sin(s/a)), \]
    and the radius of curvature at $\gamma(s)$ is $\rho(s) = 1/\kappa(s) = a$.  

    \item {\bf (Helix.)} Let $a, b > 0$ and set $c = (a^2 + b^2)^{-1/2}$. Then 
    $\gamma(s) = (a\cos(cs), a\sin(cs), bcs)$
    for $s \in \R$ is a unit speed parametrization of the helix. Then 
    \[ T(s) = \gamma'(s) = (-ac\sin(cs), ac\cos(cs), bc) \] 
    and $\gamma''(s) = (-ac^2\cos(cs), -ac^2\sin(cs), 0)$, which implies that 
    \[ \kappa(s) = \|\gamma''(s)\| = ac^2 = \frac{a}{a^2+b^2} \] 
    for all $s \in \R$, so this has constant curvature just like the circle! 
    Therefore, we will need more than curvature to distinguish between the two. 
    We will need torsion, which we discuss later on. The principal unit 
    normal to $\gamma$ at $\gamma(s)$ is 
    \[ N(s) = \frac{\gamma''(s)}{\kappa(s)} = (-\cos(cs), -\sin(cs), 0). \] 
\end{enumerate}
Note that in $\R^2$, there are only two possible unit normal vectors to the 
curve at any point. The principal unit normal points inwards. 

On the other hand, in $\R^3$, there is a whole plane of normal vectors to the curve at 
any point. It is perpendicular to the unit tangent $T$ and called the {\bf normal plane} to 
the curve at that point. The principal unit normal $N$ is a vector in that plane.

\begin{defn}{defn:2.4}
    Let $\gamma : (\alpha, \beta) \to \R^n$ be a unit speed curve. Suppose that 
    $\kappa(s) > 0$ where $s \in (\alpha, \beta)$.
    \begin{itemize} 
        \item The {\bf osculating plane to $\gamma$ at $\gamma(s)$} is defined to be
        $\Span_{\R}\{T(s), N(s)\}$.
        \item The {\bf osculating circle of $\gamma$ at $\gamma(s)$} is the 
        circle of radius $\rho(s)$ in the osculating plane that passes 
        through $\gamma(s)$ and whose center is on the ray of direction 
        $N(s)$ starting at $\gamma(s)$. 
    \end{itemize}
\end{defn}\vspace{-0.25cm} 
If $\gamma$ is a plane curve, then the osculating plane is the plane 
which contains the curve. If $\gamma$ is a circle, then it coincides 
with the osculating circle. The osculating circle is therefore the 
circle that best approximates the curve around the point $\gamma(s)$. 

\subsection{Binormal vectors, Frenet frames, and torsion} \label{subsec:2.2}
We first recall the cross product in $\R^3$. Given $u, v \in \R^3$, we 
denote the cross product of $u$ and $v$ by $u \times v$, and it is the 
unique vector in $\R^3$ such that: 
\begin{itemize}
    \item $u \times v \perp u$ and $u \times v \perp v$, with $u \times v$ 
    pointing in the direction given by the right-hand rule; 
    \item $\|u \times v\|$ is the area of the parallelogram spanned by 
    $u$ and $v$. 
\end{itemize}
Note that $u \times v = \|u\|\|v\| \sin\theta\,\mathbf n$, where $\theta$ 
is the angle between $u$ and $v$, and $\mathbf n$ is the unit normal 
vector perpendicular to the plane containing $u$ and $v$ (if $\theta 
\notin \{0, \pi\}$) with direction such that 
$\{u, v, \mathbf n\}$ is positively oriented. 
\begin{enumerate}[(i)]
    \item If $\theta \in \{0, \pi\}$ so that $u$ and $v$ are parallel, then 
    $\sin\theta = 0$ and $u \times v = \mathbf 0$. In fact, we have 
    $u \times v = \mathbf 0$ if and only if $u$ and $v$ are parallel.

    \item If $\|u\| = \|v\| = 1$ and $\theta = \pi/2$, then $u \times v = \mathbf n$ 
    so that $\|u \times v\| = 1$ and $\{u, v, u \times v\}$ is a positively 
    oriented orthonormal basis. 
\end{enumerate}
Let $\mathbf i = (1, 0, 0)$, $\mathbf j = (0, 1, 0)$, and $\mathbf k 
= (0, 0, 1)$ so that $\{\mathbf i, \mathbf j, \mathbf k\}$ is the standard 
basis of $\R^3$. In practice, given $u = (u_1, u_2, u_3)$ and $v = (v_1, v_2, v_3)$, 
we compute $u \times v$ via 
\begin{align*} 
    u \times v &= \det \begin{bmatrix} 
        \mathbf i & \mathbf j & \mathbf k \\ 
        u_1 & u_2 & u_3 \\ 
        v_1 & v_2 & v_3 
    \end{bmatrix} \\ 
    &= (u_2v_3 - v_2u_3)\mathbf i - (u_1v_3 - v_1u_3)\mathbf j + (u_1v_2 - v_1u_2)\mathbf k \\
    &= (u_2v_3 - v_2u_3, v_1u_3 - u_1v_3, u_1v_2 - v_1u_2).
\end{align*}
Finally, we observe that $v \times u = -u \times v$ because by definition, 
$v \times u$ points in the opposite direction of $u \times v$ but has the 
same length.

\begin{defn}{defn:2.5}
    Let $\gamma : (\alpha, \beta) \to \R^3$ be a unit speed curve with 
    $\kappa > 0$. The {\bf unit binormal vector} is defined to be 
    \[ B := T \times N, \] 
    and we call $\{T, N, B\}$ the {\bf Frenet frame}. 
\end{defn}\vspace{-0.25cm}

Note that $B \perp T$ and $B \perp N$ by definition of the cross product, 
and $\|B\|= 1$ since $\|T\| = \|N\| = 1$ with $T \perp N$. This implies that 
$\{T, N, B\}$ is a (right-handed) orthonormal basis at every point. 
\begin{enumerate}[(i)]
    \item Observe that $B$ is perpendicular to the osculating plane $\Span_{\R}\{T, N\}$. 
    \item We call $\Span_{\R}\{N, B\}$ the {\bf normal plane} because it is 
    the plane perpendicular to $T$. 
    \item If $\gamma$ is a curve in the $xy$-plane, then $B$ must be constantly 
    $(0, 0, 1)$ or $(0, 0, -1)$. In general, if $\gamma$ is a plane curve, 
    then $B(s)$ is a constant vector (exercise) so that $\frac{{\rm d}B}{{\rm d}s}$ is 
    identically zero. In particular, $\frac{{\rm d}B}{{\rm d}s}$ measures how much 
    a curve fails to be a plane curve (i.e. how much it twists). 
    We will see later that $\|\frac{{\rm d}B}{{\rm d}s}\|$ is equal to the absolute value 
    of the torsion. 
\end{enumerate}

Let's consider an example. A circle or helix can parametrized using the 
unit speed curve 
\[ \gamma(s) = (a\cos(cs), a\sin(cs), bcs) \] 
for $s \in \R$, where $a > 0$, $b \geq 0$, and $c = (a^2 + b^2)^{-1/2}$. 
We computed in an earlier example that 
$T(s) = \gamma'(s) = c(-a\sin(cs), a\cos(cs), b)$ and 
$N(s) = \gamma''(s)/\kappa(s) = -(\cos(cs), \sin(cs), 0)$. 
Then the unit binormal vector is 
\[ B = T \times N = \det \begin{bmatrix}
    \mathbf i & \mathbf j & \mathbf k \\ 
    -ac\sin(cs) & ac\cos(cs) & bc \\ 
    -\cos(cs) & -\sin(cs) & 0
\end{bmatrix} = (bc\sin(cs), -bc\cos(cs), ac). \] 
Note that if $b = 0$, then $\gamma$ is a circle in the $xy$-plane. 
In this case, we have $B = (0, 0, ac)$ with $c = 1/a$, so $B = (0, 0, 1)$ 
as expected. On the other hand, if $b > 0$, then $B$ is not a constant 
vector, so the osculating plane changes (and $\frac{{\rm d}B}{{\rm d}s} \neq 0$).

We have seen that $\frac{{\rm d}B}{{\rm d}s}$ measures to what extent the curve 
fails to be a plane curve. In addition, we have the following result. 

\begin{lemma}{lemma:2.6}
    For all $s \in (\alpha, \beta)$, we have $\frac{{\rm d}B}{{\rm d}s}(s) = 0$ 
    or $\frac{{\rm d}B}{{\rm d}s}(s)$ is parallel to $N(s)$.
\end{lemma}\vspace{-0.25cm}
\begin{pf}[Lemma~\ref{lemma:2.6}]
    Since $\|B\| = 1$, we know from Proposition~\ref{prop:2.2} that 
    \[ \frac{{\rm d}B}{{\rm d}s}(s) \cdot B(s) = 0 \] 
    so that $\frac{{\rm d}B}{{\rm d}s}(s) = 0$ or $\frac{{\rm d}B}{{\rm d}s}(s) \
    \perp B(s)$. By the product rule, we obtain 
    \begin{align*}
        \frac{{\rm d}B}{{\rm d}s} = \frac{{\rm d}}{{\rm d}s}(T \times N) 
        = \frac{{\rm d}T}{{\rm d}s} \times N + T \times \frac{{\rm d}N}{{\rm d}s}
        = \kappa N \times N + T \times \frac{{\rm d}N}{{\rm d}s}
        = T \times \frac{{\rm d}N}{{\rm d}s}, 
    \end{align*}
    where we used the fact that $\frac{{\rm d}T}{{\rm d}s} = \kappa N$ 
    and $N \times N = \mathbf 0$. This implies that 
    $\frac{{\rm d}B}{{\rm d}s} \perp T$. This combined with 
    $\frac{{\rm d}B}{{\rm d}s} \perp B$ means that 
    $\frac{{\rm d}B}{{\rm d}s}(s)$ is parallel to $B(s) \times T(s) = N(s)$. \qed 
\end{pf}\vspace{-0.25cm}

In particular, we have that $\frac{{\rm d}B}{{\rm d}s}(s) = -\tau(s)N(s)$ 
for some scalar function $\tau : (\alpha, \beta) \to \R$. 

\begin{defn}{defn:2.7}
    The scalar function $\tau : (\alpha, \beta) \to \R$ such that 
    $\frac{{\rm d}B}{{\rm d}s} = -\tau N$
    is called the {\bf torsion} of $\gamma$. 
\end{defn}\vspace{-0.25cm}
Note that $\tau$ can take both positive and negative values (and can 
even be zero). When $\tau > 0$, this reflects the fact that the slopes 
of lines of intersection with the osculating plane are increasing. 
In other words, the curve is twisting ``up''. Analogously, 
when $\tau < 0$, the curve is twisting ``down''. 

Revisiting the circle and helix example above, we already computed that 
$N(s) = -(\cos(cs), \sin(cs), 0)$ and $B(s) = 
(bc\sin(cs), -bc\cos(cs), ac)$. This gives 
\[ \frac{{\rm d}B}{{\rm d}s}(s) = bc^2(\cos(cs), \sin(cs), 0) = -bc^2 N(s), \] 
so for all $s \in \R$, the torsion of $\gamma$ is given by 
\[ \tau(s) = bc^2 = \frac{b}{a^2+b^2} \geq 0. \]
If $b = 0$, then $\tau \equiv 0$, reflecting the fact that $\gamma$ is a 
plane curve; that is, it is a curve in the $xy$-plane. When $b > 0$, 
then $\tau \equiv b/(a^2 + b^2) > 0$, and we have a ``right-handed'' helix 
that ``goes up''.