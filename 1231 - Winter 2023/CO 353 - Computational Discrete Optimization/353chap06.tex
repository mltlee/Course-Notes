\section{Complexity Theory} \label{sec:6}

\subsection{The Complexity Class $\Poly$} \label{subsec:6.1}
The complexity class $\Poly$ consists of all problems that can be solved
efficiently (in polynomial time). 
\begin{itemize}
    \item The problem $\textsc{MST}$ of finding a minimum spanning tree is in $\Poly$ 
    using Prim's or Kruskal's algorithm. 
    \item The problem $\textsc{ShortestPaths}$ is in $\Poly$ using Dijkstra's algorithm.
    \item The problem $\textsc{MCA}$ of finding a minimum cost arborescence is in 
    $\Poly$ by using Edmonds' algorithm. 
    \item However, we are unsure whether or not $\textsc{MSteinerT}$ and $\textsc{MMSteinerT}$
    are in $\Poly$. We'll discuss this further when we introduce the class $\NP$.
\end{itemize}

Let $X$ be a minimization problem. The {\bf decision problem} corresponding 
to $X$ takes as input an instance of $X$ and a number $k$, and answers 
if there is a feasible solution for the given instance of value at most $k$. 

For example, the decision version of the $\textsc{MSteinerT}$ problem, 
which we'll call $\textsc{DecisionMSteinerT}$, takes as input a graph 
$G = (V, E)$, costs $c_e \geq 0$ for $e \in E$, terminals $T \subseteq V$, 
and a number $k$. To solve the problem, we should output ``yes'' if there 
exists a Steiner tree of cost at most $k$, and output ``no'' otherwise. 

We'll be working with linear programs a lot later in this course. Therefore, 
we should note that solving a linear program is in $\Poly$. 
\begin{itemize}
    \item It is still not known whether the simplex algorithm can be implemented 
    to give a polynomial running time in the worst case. The hard part is the choice of 
    the pivoting rule; it is known that Bland's rule, which ensures that 
    the simplex algorithm never cycles, is quite inefficient.
    \item There are other algorithms that can solve linear programs in 
    polynomial time. The first one to be discovered was the ellipsoid method, 
    but it is also inefficient in practice. 
\end{itemize} 

\subsection{Polynomial Time Reductions} \label{subsec:6.2}
We have seen a glimpse of reductions when we discussed 
$\textsc{MSteinerT}$ and $\textsc{MMSteinerT}$ in Section~\ref{subsec:5.2}. 
Given two problems $X$ and $Y$, we say that $X$ {\bf reduces} in polynomial 
time to $Y$, denoted by $X \leq_P Y$, if one can solve $X$ by using a 
polynomial number of basic operations and a polynomial number of calls 
to an oracle that solves $Y$. Intuitively, this means that $X$ is ``not 
harder'' than $Y$, because if we know how to solve $Y$, then we have 
a way to solve $X$. We look at some examples of reductions.
\begin{itemize}
    \item We can show that $\textsc{MST} \leq_P \textsc{MSteinerT}$. 
    Suppose we are given an instance of $\textsc{MST}$, namely a graph 
    $G = (V, E)$ and edge costs $c_e$ for each $e \in E$. Then we can 
    construct an instance of $\textsc{MSteinerT}$ by simply setting 
    $T = V$. We can pass $(G, c, T)$ to an oracle that solves $\textsc{MSteinerT}$
    and output that result. 
    Overall, this process involved a polynomial number of operations 
    (assigning $T = V$), and then a polynomial number of calls to 
    the oracle (only one call was needed here).

    \item Next, let's show that $\textsc{MST} \leq_P \textsc{MCA}$. As before, 
    suppose we are given an instance of $\textsc{MST}$, so a graph $G = (V, E)$ 
    and edge costs $c_e$ for each $e \in E$. We now construct 
    an instance of $\textsc{MCA}$ as follows. Let $r$ be an 
    arbitrary node in $V$. Let $G' = (V', E')$ where 
    $V' = V$ and 
    \[ E' = \{(u, v) : uv \in E\} \cup \{(v, u) : uv \in E\}. \] 
    Set edge costs $c'_{(u,v)} = c'_{(v,u)} = c_{uv}$ for all $uv \in E$.
    Feed this instance $(G', c', r)$ of $\textsc{MCA}$ to the oracle that 
    solves it and output the obtained arborescence, ignoring directions.
\end{itemize}
However, these reductions were a bit silly. We already know that we can solve 
$\textsc{MST}$ in polynomial time; we didn't even need the oracles for the 
other problems! In particular, any problem in $\Poly$ reduces to 
any other problem simply by using the polynomial time algorithm that solves it.

Suppose that $X$ and $Y$ are problems such that $X \leq_P Y$. We make two 
easy observations: 
\begin{itemize}
    \item If $Y \in \Poly$, then $X \in \Poly$ because adding a polynomial 
    to a product of polynomials is still a polynomial. 
    \item If $X \notin \Poly$, then $Y \notin \Poly$. This is just the 
    contrapositive of the previous point, but this statement 
    is slightly more interesting. 
\end{itemize}

\subsection{The Complexity Class $\NP$} \label{subsec:6.3}
Let $X$ be a decision problem. An {\bf efficient certifier} $B$ for the 
problem $X$ is an algorithm with polynomial running time which takes 
two inputs $s$ and $t$, where $s$ is an instance of $X$ and $t$ is a 
{\bf certificate}. 
\begin{itemize}
    \item If $s$ is a ``yes'' instance of $X$, then there exists a certificate 
    $t$ whose size is polynomial in the size of $s$ such that $B(s, t)$ 
    returns ``yes''.
    \item If $s$ is a ``no'' instance of $X$, then $B(s, t)$ returns ``no'' 
    for all certificates $t$. 
\end{itemize}
Recall that the $\textsc{DecisionMSteinerT}$ problem takes as input a 
graph $G = (V, E)$, costs $c_e \geq 0$ for $e \in E$, terminals $T \subseteq V$, 
and a number $k$. This is a ``yes'' instance if there exists a Steiner tree 
of cost at most $k$, and a ``no'' instance otherwise. 

For this problem, 
a certificate $t$ is an edge set $F$, and the efficient certifier 
outputs ``yes'' if $F$ is a Steiner tree with $c(F) \leq k$, and outputs ``no'' 
otherwise. For checking that $F$ is a Steiner tree, it needs to verify that 
the edges form a  tree and the terminals are all connected, both of which can be 
done in polynomial time.

The complexity class $\NP$ is the class of all decision problems that admit 
an efficient certifier. In other words, $\NP$ consists of problems where 
``yes'' instances can be verified efficiently with a ``short'' certificate.

Clearly, if a decision problem is in $\Poly$, then it is in $\NP$. 
The verifier can use the polynomial time algorithm that solves the 
decision problem and completely ignore the certificate.
Conversely, it is not known whether every decision problem in $\NP$ 
also lies in $\Poly$. This is the famous $\Poly = \NP$ problem. 