\section{Maximum Weight Independent Sets in Matroids}\label{sec:4}

\subsection{Matroids}\label{subsec:4.1}
A {\bf matroid} $M$ is a pair $(S, {\cal I})$ where $S$ is a ground set 
and ${\cal I} \subseteq {\cal P}(S)$ is a family of sets over the 
ground set $S$ satisfying the following properties:
\begin{enumerate}[(1)]
    \item We have $\varnothing \in {\cal I}$.
    \item {\bf Downward closed:} For every $A \in {\cal I}$ and $B \subseteq A$, 
    we have $B \in {\cal I}$. 
    \item {\bf Exchange property:} For every $A, B \in {\cal I}$ with 
    $|A| > |B|$, there exists $e \in A \setminus B$ such that $B \cup \{e\} \in {\cal I}$. 
\end{enumerate}
The sets in the family ${\cal I}$ are called the {\bf independent sets}.

This definition is rather abstract, so we'll go over some examples. 

{\bf Uniform matroids.} Let $S = \{1, \dots, n\}$ and let ${\cal I} = 
\{A \subseteq S : |A| \leq k\}$ for some fixed $k \in \Z^+$. 
\begin{enumerate}[(1)]
    \item We have $\varnothing \in {\cal I}$ since $|\varnothing| = 0 \leq k$. 
    \item If $A \in {\cal I}$ and $B \subseteq A$, then $|B| \leq |A| \leq k$, 
    so $B \in {\cal I}$. 
    \item For every $A, B \in {\cal I}$ with $|A| > |B|$, we have 
    $A \setminus B \neq \varnothing$, so we can pick an element $e \in A 
    \setminus B$. Note that $|B| \leq k-1$, so $|B \cup \{e\}| \leq k$, 
    which implies that $B \cup \{e\} \in {\cal I}$. 
\end{enumerate}

{\bf Partition matroids.} Let $S = \{1, \dots, n\}$ as before, and 
let $S_1, \dots, S_t$ be a partition of $S$. That is, we have 
$\bigcup_{j=1}^t S_j = S$ and $S_i \cap S_j = \varnothing$ whenever 
$i \neq j$. Let ${\cal I} = \{A \subseteq S : |A \cap S_j| \leq r_j 
\text{ for all } j = 1, \dots, t\}$, where $r_j \in \Z^+$ are fixed. 
(Usually, we pick $r_j < |S_j|$, for otherwise this does not put 
any constraints on the independent sets.)
\begin{enumerate}[(1)]
    \item We have $\varnothing \in {\cal I}$ since $|\varnothing \cap S_j| =
    0 \leq r_j$ for all $j = 1, \dots, t$. 
    \item If $A \in {\cal I}$ and $B \subseteq A$, then $|B \cap S_j| 
    \leq |A \cap S_j| \leq r_j$ for all $j = 1, \dots, t$, so $B \in {\cal I}$.
    \item If $A, B \in {\cal I}$ with $|A| > |B|$, then there must exist 
    some $j \in \{1, \dots, t\}$ such that $|A \cap S_j| > |B \cap S_j|$. 
    Then $(A \cap S_j) \setminus (B \cap S_j) \neq \varnothing$, so 
    we can find an element $e \in (A \cap S_j) \setminus (B \cap S_j)$. 
    Observe that 
    \[ |(B \cup \{e\}) \cap S_k| = \begin{cases}
        |B \cap S_k|, & \text{if } k \neq j, \\ 
        |B \cap S_k| + 1, & \text{if } k = j. 
    \end{cases} \] 
    In the first case, we have $|B \cap S_k| \leq r_k$ since $B \in {\cal I}$
    and $e \notin S_k$. For the second case, we have $|B \cap S_j| + 1 
    \leq |A \cap S_j| \leq r_j$ since $A \in {\cal I}$ and 
    $|A \cap S_j| > |B \cap S_j|$. It follows that 
    $B \cup \{e\} \in {\cal I}$. 
\end{enumerate}

The following example is the setting where matroids originated from. 
This is the prototypical example of a matroid given in introductory
talks, and is one we can build our intuition from. 

{\bf Linear matroids.} Let $\mathbb{F}$ be a field. (If you don't remember 
what a field is, it's a set together with two operations satisfying 
some nice axioms; for example, $\mathbb{F} = \mathbb{R}$ is a field under 
the usual addition and multiplication.) 
Let $S$ be a set of vectors in $\mathbb{F}^d$ and let 
${\cal I} = \{A \subseteq S : A \text{ is a linearly independent set over }
\mathbb{F}\}$. Verifying that this is a matroid is an exercise in 
linear algebra, which we won't do here. Such a matroid is said to be 
representable over $\mathbb{F}$.

{\bf Graphic matroids.} Given a graph $G = (V, E)$, let $S = E$, and let 
${\cal I} = \{A \subseteq S : (V, A) \text{ is acyclic}\}$. Then ${\cal I}$ 
consists of all forests (acyclic subgraphs) of $G$.
\begin{enumerate}[(1)]
    \item We see that $(V, \varnothing)$ is acyclic, so $\varnothing \in {\cal I}$. 
    \item If $A \in {\cal I}$ and $B \subseteq A$, then $(V, B)$ has no cycles 
    since $(V, A)$ has no cycles, so $B \in {\cal I}$. 
    \item Let $A, B \in {\cal I}$ with $|A| > |B|$. Then $(V, A)$ has 
    $|V| - |A|$ connected components. Similarly, $(V, B)$ has 
    $|V| - |B|$ connected components, and we have $|V| - |B| > |V| - |A|$. 
    Hence, there is a connected component $C$ of $(V, A)$ that intersects 
    at least two connected components of $(V, B)$, say $C'_1$ and $C'_2$.
    Then $(V, A)$ has a $u, v$-path $P$ where $u \in C'_1$ and 
    $v \in C'_2$. Let $e \in P \cap \delta(C'_1)$, which exists because 
    an edge must cross the boundary of the connected component to 
    get to $C'_2$. Then $B \cup \{e\} \in {\cal I}$.
\end{enumerate}

A maximal independent set $A \in {\cal I}$ is called a {\bf basis}.
It is important to note that maximal is different from maximum. 
Here, we just mean that $A \cup \{e\}$ cannot be independent for any 
$e \in S \setminus A$. 

In the graphic matroid example, the bases correspond to the spanning trees.

Before we move on, we give a few non-examples of matroids. 
\begin{enumerate}[(1)]
    \item Let $S = \{1, 2\}$ and ${\cal I} = \{\{1\}, \{2\}, \{1, 2\}\}$. 
    Then $(S, {\cal I})$ is not a matroid since $\varnothing \notin {\cal I}$. 
    It fails the downward closed property too by considering $\varnothing 
    \subseteq \{1\}$, where $\{1\} \in {\cal I}$.
    \item Let $S = \{1, 2, 3\}$ and ${\cal I} = \{\varnothing, \{1\}, \{2\}, \{3\}, 
    \{1, 2\}\}$. Then $|\{1, 2\}| > |\{3\}|$, but neither $\{1, 3\}$ nor 
    $\{2, 3\}$ are independent sets. Hence, $(S, {\cal I})$ fails the exchange 
    property and is not a matroid.
\end{enumerate}

\begin{lemma}{lemma:4.1}
    All bases for a matroid have the same cardinality.
\end{lemma}\vspace{-0.25cm}
\begin{pf}[Lemma~\ref{lemma:4.1}]
    Let $M = (S, {\cal I})$ be a matroid and let $A, B \in {\cal I}$ be 
    two bases. Suppose by way of contradiction that $|A| > |B|$. 
    Then by the exchange property, we can find $e \in A \setminus B$ 
    such that $B \cup \{e\} \in {\cal I}$, contradicting the maximality of $B$. \qed
\end{pf}\vspace{-0.25cm}

\subsection{Maximum Weight Independent Sets} \label{subsec:4.2}
In the maximum weight independent sets problem, we are given a matroid 
$M = (S, {\cal I})$ and weights $w_e$ for each $e \in S$. The goal 
is to find a maximum weight independent set; that is, a set $A \in {\cal I}$ 
achieving the maximum value of 
\[ w(A) := \sum_{e\in A} w_e. \]
Note that we may assume that $w_e > 0$ for all $e \in S$. Indeed, if 
$A \in {\cal I}$, then by the downward closed property, we have 
$A \setminus \{e \in S : w_e \leq 0\} \in {\cal I}$ as well. But 
\[ w(A \setminus \{e \in S : w_e \leq 0\}) \geq w(A), \] 
so we can simply throw away any elements in $S$ with $w_e \leq 0$ 
and still remain independent.

We now give a greedy algorithm to solve the maximum weight independent 
sets problem under the assumption that $w_e > 0$ for all $e \in S$. 
We notice that it is extremely similar to Kruskal's algorithm.

\begin{mdframed}[
    linewidth=1pt,
    linecolor=black,
    bottomline=false,topline=false,rightline=false,
    innerrightmargin=0pt,innertopmargin=0pt,innerbottommargin=0pt,
    innerleftmargin=1em,% Distance between vertical rule & proof content
    skipabove=0.75\baselineskip
]
{\bf Input.} A matroid $M = (S, {\cal I})$ and weights $w_e > 0$ for all 
$e \in S$.

{\bf Output.} A maximum weight independent set of $M$.
\begin{enumerate}[leftmargin=1.75cm, label={Step \arabic*.}]
    \item Sort the elements in $S$ such that 
    $w_{e_1} \geq w_{e_2} \geq \cdots \geq w_{e_{|S|}}$. 
    Set $A \gets \varnothing$ and $i \gets 1$. 

    \item While $i \neq |S|+1$:
    \begin{enumerate}[label={}]
        \item If $A \cup \{e_i\} \in {\cal I}$, then set $A \gets A \cup \{e_i\}$.
        \item Regardless if we add an element or not, increment $i \gets i+1$.
    \end{enumerate}
\end{enumerate}
\end{mdframed}\vspace{-0.15cm}
It is crucial that we have the assumption $w_e > 0$ for $e \in S$ here, 
because otherwise our greedy algorithm may keep adding negative weight 
elements that keep the set independent, lowering the total weight.

\begin{pf}[correctness of the greedy algorithm]
    Let $A$ denote the output the greedy algorithm. Let $A^*$ be a 
    maximum weight independent set of $M$. It is clear from the construction 
    in Step 2 that $A \in {\cal I}$. Moreover, $A$ is a basis of $M$ 
    because we iterate through every element of $S$ and no element 
    can be further included in $A$ to keep it independent. Similarly, 
    $A^*$ is a basis of $M$. Otherwise, if $A^* \cup \{e\} \in {\cal I}$ 
    for some $e \in S$, then $w(A^* \cup \{e\}) > w(A^*)$ using the 
    assumption that $w_e > 0$, and so $A^*$ was not a maximum weight 
    independent set in the first place. Therefore, we have $|A| = |A^*| = k$ 
    since bases of matroids share the same cardinality by Lemma~\ref{lemma:4.1}.

    Write $A = \{g_1, \dots, g_k\}$ and $A^* = \{g_1^*, \dots, g_k^*\}$ 
    where the elements are sorted such that $w_{g_1} \geq \cdots \geq 
    w_{g_k}$ and $w_{g_1^*} \geq \cdots \geq w_{g_k^*}$. Our aim is to show 
    that $w(A) \geq w(A^*)$. Towards a contradiction, suppose that 
    $w(A) < w(A^*)$ and consider the smallest $j \in \{1, \dots, k\}$ such that 
    \[ w(\{g_1, \dots, g_j\}) < w(\{g_1^*, \dots, g_j^*\}). \] 
    Let $A_{j-1} = \{g_1, \dots, g_{j-1}\}$ and $A_j^* = \{g_1^*, \dots, g_j^*\}$. 
    These are both independent by the downward closed property and we have 
    $|A_j^*| = j > j-1 = |A_{j-1}|$, so it follows by the exchange property 
    that there is some $e \in A_j^* \setminus A_{j-1}$ such that $A_{j-1} 
    \cup \{e\} \in {\cal I}$. Then we obtain 
    \[ w_e \geq w_{g_j}^* > w_{g_j}. \] 
    The first inequality is because $e \in \{g_1^*, \dots, g_j^*\}$ and 
    the edges are sorted by descending weight, and the second inequality 
    is because $w(\{g_1, \dots, g_j\}) < w(\{g_1^*, \dots, g_j^*\})$ 
    while $w(\{g_1, \dots, g_{j-1}\}) \geq w(\{g_1^*, \dots, g_{j-1}^*\})$
    by the way we picked $j$. In particular, let's consider the iteration 
    where $e$ was considered for the greedy algorithm. It must have been 
    before $g_j$ because $w_e > w_{g_j}$. But if $B \cup \{e\} 
    \notin {\cal I}$ for $B \subseteq A_{j-1}$, then $A_{j-1} 
    \cup \{e\} \notin {\cal I}$ as well, which is a contradiction. \qed
\end{pf}\vspace{-0.25cm}

The running time of Step 1 is $O(|S|\log|S|)$ for sorting the elements in $S$. 
In Step 2, there are $|S|$ calls to an independence oracle that tells us 
``yes'' or ``no'' to whether a set is independent, but we do not 
necessarily know the complexity of this oracle. When we discussed Kruskal's 
algorithm, we could efficiently play the role as oracle by using the \textsc{UnionFind} 
data structure, but this is not always the case.

\subsection{Maximum Weight Common Independent Sets} \label{subsec:4.3}
In the maximum weight \emph{common} independent set problem, we are given 
two matroids $M' = (S, {\cal I}')$ and $M'' = (S, {\cal I}'')$ sharing a 
ground set $S$ with weights $w_e$ for $e \in S$. The goal is to find 
a set $A$ such that $A \in {\cal I}'$ and $A \in {\cal I}''$ and 
achieving maximum value 
\[ w(A) = \sum_{e\in A} w_e. \] 
We note that this problem can be solved efficiently with linear programming, 
but we won't state the algorithm here. The extension of this problem 
to even just three matroids becomes $\NP$-hard. For now, we'll give a few 
examples of applications of the two matroid problem. 

{\bf Bipartite matchings.} Let $G = (V, E)$ be a bipartite graph with 
bipartition $(U, W)$. Recall that a bipartite graph is such that 
the vertices are partitioned as $V = U \cup W$, and every edge has 
one endpoint in $U$ and one endpoint in $W$. A {\bf matching} is a set 
of edges $M \subseteq E$ such that for every $v \in V$, we have 
$|M \cap \delta(v)| \leq 1$. In other words, every vertex participates 
in at most one edge. 

Suppose that for a bipartite graph, we want to find a matching $M \subseteq E$
achieving the maximum weight
\[ w(M) = \sum_{e\in M} w_e. \] 
We can formulate this problem as a maximum weight common independent set instance. 
Define matroids $M' = (E, {\cal I}')$ and $M'' = (E, {\cal I}'')$, where 
\begin{align*}
    {\cal I}' &= \{A \subseteq E : |A \cap \delta(u)| \leq 1 \text{ for all } u \in U\}, \\ 
    {\cal I}'' &= \{A \subseteq E : |A \cap \delta(w)| \leq 1 \text{ for all } w \in W\}.
\end{align*}
We see that $\{\delta(u)\}_{u\in U}$ and $\{\delta(w)\}_{w\in W}$ define 
partitions on $E$, and so $M'$ and $M''$ are particular examples of partition 
matroids. Moreover, $M$ is a matching if and only if $M \in {\cal I}'$ and 
$M \in {\cal I}''$. 

{\bf Arborescences.} Let $G = (V, E)$ be a directed graph with edge costs 
$c_e \geq 0$ for all $e \in E$. Let $r \in V$ be a distinguished vertex. 
We can define matroids $M' = (E, {\cal I}')$ 
and $M'' = (E, {\cal I}'')$ where 
\begin{align*}
    {\cal I}' &= \{A \subseteq E : (V, A) \text{ is acyclic when ignoring directions}\}, \\ 
    {\cal I}'' &= \{A \subseteq E : |\delta^{\text{in}}(v) \cap A| \leq 1 - \delta_{vr} \text{ for all } v \in V\}.
\end{align*}
Here, $\delta^{\text{in}}(v)$ denotes the set of edges incoming to $v$, 
and $\delta_{vr}$ is the Kronecker delta so that 
\[ 1 - \delta_{vr} = \begin{cases}
    1, & \text{if } v \neq r, \\ 
    0, & \text{if } v = r.
\end{cases} \] 
We see that $M' = (E, {\cal I}')$ is a graphic matroid and $M'' = 
(E, {\cal I}'')$ is a partition matroid, with $T$ being an arborescence 
rooted at $r$ if and only if $T$ is a basis for both $M'$ and $M''$. 