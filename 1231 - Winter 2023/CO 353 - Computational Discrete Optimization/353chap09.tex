\section{Lagrangian Relaxation and the $k$-median Problem} \label{sec:9}

\subsection{The $k$-median Problem} \label{subsec:9.1}
We now discuss a variant of the uncapacitated facility location problem, 
known as the $k$-median problem. As before, we are given a set of 
facilities $F$, a set of clients $D$, and metric assignment costs $c_{ij}$ 
for all $i \in F$ and $j \in D$. However, there are no longer 
facility opening costs $f_i$ for all $i \in F$; instead, we are given an integer $k$ 
that is an upper bound on the number of facilities we can open. 
In other words, we wish to select facilities $S \subseteq F$ 
such that $|S| \leq k$ while minimizing the cost 
\[ c(S) = \sum_{j\in D} \min_{i\in S} c_{ij}. \]
We define the binary variables $y_i \in \{0, 1\}$ to indicate whether or 
not we open facility $i$, and $x_{ij} \in \{0, 1\}$ to indicate if 
client $j$ is assigned to facility $i$. To limit the number of open facilities, 
we introduce the constraint $\sum_{i\in F} y_i \leq k$. This gives the 
following IP formulation for the $k$-median problem:
\begin{align*}
    \min\quad & \sum_{i\in F} \sum_{j\in D} c_{ij} x_{ij} \\ 
    \text{subject to}\quad & \sum_{i\in F} x_{ij} = 1
    && \text{for all $j \in D$} \\
    & x_{ij} \leq y_i && \text{for all $i \in F$ and $j \in D$} \\
    & \sum_{i\in F} y_i \leq k \\
    & x_{ij} \in \{0, 1\} && \text{for all $i \in F$ and $j \in D$} \\ 
    & y_i \in \{0, 1\} && \text{for all $i \in F$}
\end{align*}
The only difference with the uncapacitated facility location problem 
is the objective function now that there are no opening costs, 
and the additional constraint we mentioned above. The LP relaxation is then 
obtained by replacing $x_{ij} \in \{0, 1\}$ and $y_i \in \{0, 1\}$ 
with $x_{ij} \geq 0$ and $y_i \geq 0$ respectively. We will denote the 
LP relaxation by (PrimalLP).

\subsection{Lagrangian Relaxation} \label{subsec:9.2}
We will use the idea of Lagrangian relaxation to reduce the $k$-median 
problem to the uncapacitated facility location problem. To do this, 
we would like to remove the constraint that $\sum_{i\in F} y_i \leq k$. 
Fix $\lambda \geq 0$, and introduce the penalty term $\lambda \cdot (\sum_{i\in F} y_i - k)$
to the objective function. This penalty will prefer solutions that 
respect the constraint. Our new LP, which we call (LPrimalLP), is given by 
\begin{align*}
    \min\quad & \sum_{i\in F} \sum_{j\in D} c_{ij} x_{ij} + \sum_{i\in F} \lambda y_i - \lambda k \\ 
    \text{subject to}\quad & \sum_{i\in F} x_{ij} = 1
    && \text{for all $j \in D$} \\
    & x_{ij} \leq y_i && \text{for all $i \in F$ and $j \in D$} \\
    & x_{ij} \geq 0 && \text{for all $i \in F$ and $j \in D$} \\ 
    & y_i \geq 0 && \text{for all $i \in F$}
\end{align*}
Observe that aside from the additional constant term $-\lambda k$, the objective 
function looks like that of the LP relaxation for the uncapacitated 
facility problem where the facility opening costs are all $f_i = \lambda$. 
Note that any feasible solution for the LP relaxation (PrimalLP) of the $k$-median 
problem is also feasible for (LPrimalLP). Moreover, any feasible 
solution for (PrimalLP) has lower objective value in (LPrimalLP) 
than (PrimalLP) since $\sum_{i\in F} y_i - k \leq 0$. Therefore, we find that 
\[ \text{OPT}_{\text{PrimalIP}} \geq \text{OPT}_{\text{PrimalLP}} \geq 
\text{OPT}_{\text{LPrimalLP}}. \] 
Taking the dual of (LPrimalLP), say (LDualLP), we obtain 
\begin{align*}
    \max\quad & \sum_{j\in D} v_j - \lambda k \\ 
    \text{subject to}\quad & \sum_{j\in D} w_{ij} \leq \lambda 
    && \text{for all $i \in F$} \\
    & v_j - w_{ij} \leq c_{ij} && \text{for all $i \in F$ and $j \in D$} \\
    & w_{ij} \geq 0 && \text{for all $i \in F$ and $j \in D$} 
\end{align*}
This is the same as the dual of the LP relaxation for the 
uncapacitated facility location problem, except that the facility costs 
are $\lambda$ and there is an extra constant term $-\lambda k$ in the 
objective function. 

\subsection{Approximation Algorithm for $k$-median} \label{subsec:9.3}
We want to use the primal-dual algorithm for the uncapacitated facility 
location problem, where all the facility costs are $f_i = \lambda$ 
for some choice of $\lambda \geq 0$. How do we get a performance 
guarantee? Recall from the previous section (just after the statement
of Lemma~\ref{lemma:8.3}) that the algorithm opens a set $S \subseteq F$ 
of facilities and constructs a feasible dual $(v, w)$ such that 
\[ c(S) + \sum_{i\in S} f_i = \sum_{j\in D} \min_{i\in S} c_{ij} + \sum_{i\in S} f_i \leq 3 
\sum_{j\in D} v_j. \] 
In fact, this claim can be slightly strengthened to the following: 

\begin{exercise}{ex:9.1}
    The primal-dual algorithm for the uncapacitated facility location 
    problem outputs a set $S \subseteq F$ of facilities and a 
    constructs a feasible dual solution $(v, w)$ such that 
    \[ c(S) + 3 \sum_{i\in S} f_i \leq 3 \sum_{j\in D} v_j. \]
\end{exercise}\vspace{-0.25cm}
Substituting $f_i = \lambda$ and rearranging this gives 
\begin{equation}\label{eqn:9.1}
    c(S) \leq 3 \left( \sum_{j\in D} v_j - \lambda|S| \right).
\end{equation}
Note that if, by chance, the algorithm opens a set of facilities $S$ such that 
$|S| = k$, then we have 
\[ c(S) \leq 3\left( \sum_{j\in D} v_j - \lambda k \right) \leq 
3 \cdot \text{OPT}_{\text{LDualLP}} = \text{OPT}_{\text{LPrimalLP}} 
\leq \text{OPT}_{\text{PrimalIP}}. \]
This follows because $(v, w)$ is feasible for (LDualLP). Therefore, 
a natural idea is to try to find a value for $\lambda \geq 0$ such that 
the uncapacitated facility location algorithm outputs a set of 
facilities $S \subseteq F$ with $|S| = k$. We do this by way of 
binary search. To begin, we need two initial values of $\lambda$, one 
of which the uncapacitated facility location algorithm opens 
at least $k$ facilities, and another for which it opens at most $k$ facilities. 
\begin{itemize}
    \item Consider what happens for $\lambda = 0$. If the algorithm 
    opens less than $k$ facilities, then we can open an additional 
    $k - |S|$ facilities at no additional costs and still get the 
    approximation guarantee. Therefore, we may assume that for 
    $\lambda = 0$, the algorithm opens more than $k$ facilities. 
    \item Consider $\lambda = \sum_{i\in F} \sum_{j\in D} c_{ij}$. 
    In this case, the algorithm opens just one facility since 
    this cost to open a facility is extremely expensive and 
    exceeds the sum of any combination of assignment costs. 
\end{itemize}
Therefore, we can start with the values $\lambda_1 = 0$ and 
$\lambda_2 = \sum_{i\in F} \sum_{j\in D} c_{ij}$. As we discussed above, 
these two values of $\lambda$ return solutions $S_1$ and $S_2$ 
(respectively) for which $|S_1| > k$ and $|S_2| < k$. We run the algorithm 
for $\lambda = \frac12(\lambda_1 + \lambda_2)$. 
\begin{enumerate}[(i)]
    \item If the algorithm 
    returns a solution $S$ with exactly $k$ facilities, then we are done 
    and obtain a solution of cost at most $3 \cdot \text{OPT}_{\text{PrimalIP}}$.
    \item If $S$ has more than $k$ facilities, then set $\lambda_1 = \lambda$ 
    and $S_1 = S$. Otherwise, $S$ has less than $k$ facilities, so we set 
    $\lambda_2 = \lambda$ and $S_2 = S$. 
\end{enumerate} 
We repeat this process until the algorithm finds a solution with exactly 
$k$ facilities, or the interval $\lambda_2 - \lambda_1$ becomes small enough, 
at which point we can obtain a solution for the $k$-median problem 
by appropriately combining the solutions from $S_1$ and $S_2$. 

Let $c_{\min}$ be the smallest 
nonzero assignment cost; set $c_{\min} = 0$ if they are all zero. 
Without loss of generality, we assume that $0 < c_{\min} \leq 
\text{OPT}_{\text{PrimalIP}}$. 
The second inequality certainly holds as each client needs to be assigned 
to a facility in a feasible solution, and the first can be assumed for otherwise 
$c_{ij} = 0$ for all $i \in F$ and $j \in D$ so that $\text{OPT}_{\text{PrimalIP}} = 0$. 
In this case, we can find a solution to the $k$-median in polynomial time 
by simply picking $k$ arbitrary facilities. 

We will run the binary search until either 
the algorithm finds a solution with exactly $k$ facilities, or 
\[ \lambda_2 - \lambda_1 \leq \frac{\eps c_{\min}}{|F|} \] 
for some error parameter $\eps > 0$. 
In the latter case, we will use $S_1$ and $S_2$ to obtain a solution $S$
in polynomial time such that $|S| = k$ and $c(S) \leq 6(1+\eps) \cdot 
\text{OPT}_{\text{PrimalIP}}$. This gives us a $6(1+\eps)$-approximation 
algorithm for the $k$-median problem. Note that the binary search makes 
\[ \log \left( \frac{\sum_{i\in F} \sum_{j\in D} c_{ij}}{\eps c_{\min}/|F|} 
\right) = \log \left( \frac{|F| \cdot \sum_{i\in F} \sum_{j\in D} c_{ij}}{\eps c_{\min}} \right) \] 
calls to the uncapacitated facility location algorithm, so the overall 
algorithm runs in polynomial time.

We only need to analyze the case where the binary search 
terminates with $\lambda_2 - \lambda_1 \leq \eps c_{\min}/|F|$, 
where $S_1$ and $S_2$ are the facilities obtained by the algorithm such that 
$|S_1| > k$ and $|S_2| < k$. Let $(v^1, w^1)$ and $(v^2, w^2)$ be the 
corresponding solutions to (LDualLP). From inequality (\ref{eqn:9.1}), we have that 
\begin{align}
    c(S_1) &\leq 3 \left( \sum_{j\in D} v_j^1 - \lambda_1 |S_1| \right), \label{eqn:9.2} \\
    c(S_2) &\leq 3 \left( \sum_{j\in D} v_j^2 - \lambda_2 |S_2| \right). \label{eqn:9.3}
\end{align}
We now show how to use $S_1$ and $S_2$ to obtain a solution $S$ in 
polynomial time with $|S| = k$ and $c(S) \leq 6(1+\eps) \cdot 
\text{OPT}_{\text{PrimalIP}}$. First, we need to relate the costs 
of the two solutions to $\text{OPT}_{\text{PrimalIP}}$. Find 
$\alpha_1, \alpha_2 \geq 0$ such that $\alpha_1 + \alpha_2 = 1$ and 
$\alpha_1|S_1| + \alpha_2|S_2| = k$. This implies that 
\begin{align*}
    \alpha_1 &= \frac{k-|S_2|}{|S_1|-|S_2|}, \qquad \alpha_2 
    = \frac{|S_1|-k}{|S_1|-|S_2|}. 
\end{align*} 
Note that we can obtain a dual solution $(\tilde v, \tilde w)$ by setting 
$\tilde v = \alpha_1 v^1 + \alpha_2 v^2$ and $\tilde w = 
\alpha_1 w^1 + \alpha_2 w^2$. We know that $(v^2, w^2)$ is 
feasible for (LDualLP) with facility costs $\lambda_2$. Moreover, 
$(v^1, w^1)$ is feasible for (LDualLP) with facility costs $\lambda_1$, 
and we know that $\lambda_1 \leq \lambda_2$. It follows that 
\[ \sum_{j\in D} w_{ij}^1 \leq \lambda_1 \leq \lambda_2 \] 
for all $i \in F$, implying that $(v^1, w^1)$ is feasible for (LDualLP) with facility 
costs $\lambda_2$ as well. Therefore, $(\tilde v, \tilde w)$ is feasible for 
(LDualLP) with facility costs $\lambda_2$ since it is a convex combination of 
two feasible dual solutions. We now prove the following lemma, 
which states that the convex combination of the costs of $S_1$ and $S_2$ 
is close to the cost of an optimal solution.

\begin{lemma}{lemma:9.2}
    We have $\alpha_1 c(S_1) + \alpha_2 c(S_2) \leq 3(1+\eps) \cdot 
    \text{OPT}_{\text{PrimalIP}}$. 
\end{lemma}\vspace{-0.25cm} 
\begin{pf}[Lemma~\ref{lemma:9.2}]
    First, observe that 
    \begin{align*}
        c(S_1) \leq 3 \left( \sum_{j\in D} v_j^1 - \lambda_1 |S_1| \right) 
        &= 3 \left( \sum_{j\in D} v_j^1 - \lambda_2 |S_1| \right) 
        + 3(\lambda_2 - \lambda_1) |S_1| \\ 
        &\leq 3 \left( \sum_{j\in D} v_j^1 - \lambda_2 |S_1| \right) 
        + 3 \cdot \frac{\eps c_{\min}}{|F|} |S_1| \\ 
        &\leq 3 \left( \sum_{j\in D} v_j^1 - \lambda_2 |S_1| \right) 
        + 3\eps \cdot \text{OPT}_{\text{PrimalIP}},
    \end{align*}
    where the first inequality is inequality (\ref{eqn:9.2}), and the 
    last inequality uses the fact that $|S_1| \leq |F|$ and 
    $c_{\min} \leq \text{OPT}_{\text{PrimalIP}}$. Moreover, 
    inequality (\ref{eqn:9.3}) states that 
    \[ c(S_2) \leq 3\left( \sum_{j\in D} v_j^2 - \lambda_2 |S_2| \right). \] 
    Recall from our earlier discussion that $(\tilde v, \tilde w)$ is a feasible dual solution 
    to (LDualLP) for facility costs $\lambda_2$ where $\tilde v = 
    \alpha_1 v^1 + \alpha_2 v^2$ and $\tilde w = \alpha_1 w^1 + \alpha_2 w^2$. 
    Moreover, we have $\alpha_1 |S_1| + \alpha_2 |S_2| = k$ and 
    $\alpha_1 \leq 1$. 
    By taking the convex combination of these inequalities, we obtain 
    \begin{align*} 
        \alpha_1 c(S_1) + \alpha_2 c(S_2) 
        &\leq 3\alpha_1 \left( \sum_{j\in D} v_j^1 - \lambda_2 |S_1| \right) 
        + 3\alpha_1\eps \cdot \text{OPT}_{\text{PrimalIP}} + 
        3\alpha_2 \left( \sum_{j\in D} v_j^2 - \lambda_2 |S_2| \right) \\ 
        &= 3 \sum_{j\in D} (\alpha_1 v_j^1 + \alpha_2 v_j^2) - 3\lambda_2 
        (\alpha_1 |S_1| + \alpha_2 |S_2|) + 3\alpha_1\eps \cdot 
        \text{OPT}_{\text{PrimalIP}} \\ 
        &\leq 3 \left( \sum_{j\in D} \tilde v_j - \lambda_2 k \right) 
        + 3\eps \cdot \text{OPT}_{\text{PrimalIP}}  
        \leq 3(1 + \eps) \cdot \text{OPT}_{\text{PrimalIP}},
    \end{align*}
    where the last inequality uses the fact that $(\tilde v, \tilde w)$ 
    is feasible for (LDualLP) with respect to $\lambda_2$. \qed 
\end{pf}\vspace{-0.25cm} 

\subsection{Case Analysis}
The algorithm now splits into two cases: the easier case where 
$\alpha_2 \geq 1/2$, and a harder 
case where $\alpha_2 < 1/2$. When $\alpha_2 \geq 1/2$, 
we see that $S_2$ is a feasible solution for the original problem and 
Lemma~\ref{lemma:9.2} implies that 
\[ c(S_2) \leq 2\alpha_2 c(S_2) \leq 2(\alpha_1 c(S_1) + \alpha_2 c(S_2)) 
\leq 6(1+\eps) \cdot \text{OPT}_{\text{PrimalIP}}. \] 
For all $j \in D$, let $c(j, S) = \min_{i\in S} c_{ij}$ so that $\sum_{j\in D} c(j, S) = c(S)$. 
We will give a randomized algorithm for the remaining case where $\alpha_2 < 1/2$. 
\begin{enumerate}
    \item For each facility $i \in S_2$, we open the closest facility $h \in S_1$; that 
    is, the facility $h \in S_1$ minimizing $c_{ih}$.
    \item If Step 1 doesn't open $|S_2|$ facilities of $S_1$ (because some facilities in $S_2$ are 
    close to the same facility in $S_1$), then we open some arbitrary 
    facilities in $S_1$ so that exactly $|S_2|$ are opened. 
    \item Choose a random subset of $k - |S_2|$ of the remaining $|S_1| - |S_2|$ facilities, 
    and open these ones.
\end{enumerate}
Let $S$ be the resulting set of facilities opened. To finish our 
analysis, we prove the following lemma. 

\begin{lemma}{lemma:9.3}
    If $\alpha_2 < 1/2$, then opening the facilities as above has expected 
    cost $\mathbb{E}[c(S)] \leq 6(1+\eps) \cdot \text{OPT}_{\text{PrimalIP}}$. 
\end{lemma}\vspace{-0.25cm}
\begin{pf}[Lemma~\ref{lemma:9.3}]
    To prove the lemma, we consider the expected cost of assigning a given 
    client $j \in D$ to a facility opened by the randomized algorithm. 
    Suppose that the facility $1 \in S_1$ is the open facility in $S_1$ 
    closest to $j$, so $c_{1j} = c(j, S_1)$. Similarly, let $2 \in S_2$ 
    be the open facility in $S_2$ closest to $j$ so that $c_{2j} = 
    c(j, S_2)$. 
    
    Note that with probability $(k-|S_2|)/(|S_1|-|S_2|) = \alpha_1$, 
    the facility $1 \in S_1$ is opened in the randomized step 
    of the algorithm if it has not been opened in the first step of the algorithm. 
    Therefore, with probability at least $\alpha_1$, the cost of 
    assigning $j$ to the closest open facility in $S$ is at most $c_{1j} = 
    c(j, S_1)$. 
    
    With probability at most $1 - \alpha_1 = \alpha_2$, 
    facility $1 \in S_1$ is not opened. In this case, we can at worst assign 
    $j$ to a facility opened in the first step of the algorithm; 
    in particular, we assign $j$ to the facility in $S_1$ closest to 
    $2 \in S_2$. Let $i \in S_1$ be the closest facility to $2 \in S_2$. 
    By the triangle inequality, we have 
    \[ c_{ij} \leq c_{i2} + c_{2j}. \] 
    Moreover, we know that $c_{i2} \leq c_{12}$ since $i$ is the closest 
    facility in $S_1$ to $2 \in S_2$, so 
    \[ c_{ij} \leq c_{12} + c_{2j}. \] 
    Using the triangle equality again, we have $c_{12} \leq c_{1j} + c_{2j}$, so 
    \[ c_{ij} \leq c_{1j} + c_{2j} + c_{2j} = c(j, S_1) + 2c(j, S_2). \] 
    Therefore, the expected cost of assigning $j$ to the closest facility in $S$ is 
    \[ \mathbb{E}[c(j, S)] \leq \alpha_1 c(j, S_1) + \alpha_2
    (c(j, S_1) + 2c(j, S_2)) = c(j, S_1) + 2\alpha_2 c(j, S_2). \] 
    Since $\alpha_2 < 1/2$, we have $\alpha_1 = 1 - \alpha_2 > 1/2$, 
    implying that 
    \[ \mathbb{E}[c(j, S)] \leq 2(\alpha_1 c(j, S_1) + \alpha_2 c(j, S_2)). \] 
    Summing over all $j \in D$ and applying Lemma~\ref{lemma:9.2} gives 
    \[ \mathbb{E}[c(S)] \leq 2(\alpha_1 c(S_1) + \alpha_2 c(S_2)) \leq 
    6(1+\eps) \cdot \text{OPT}_{\text{PrimalIP}}. \tag*{\qed} \] 
\end{pf}