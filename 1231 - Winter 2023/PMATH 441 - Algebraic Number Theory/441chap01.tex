\section{Algebraic Integers}\label{sec:1}

\subsection{Motivation}\label{subsec:1.1}
At its most elementary, number theory is the study of integers. Some of the 
hot topics typically discussed in a first-year number theory course 
include primes, divisibility, the Euclidean algorithm, and of most 
interest to us, prime factorization. Our goal in this course is to generalize 
these topics using commutative algebra. 

One naive approach would be to consider unique factorization domains, or 
UFDs. However, the canonical example of a principal ideal domain (PID) 
that is not a UFD is $\Z[\sqrt{5}]$, which is far too integer-like 
to be disqualified from our discussion. 

Let's do some investigation. Consider $\alpha = (1 + \sqrt{5})/2$. We have 
$(2\alpha - 1)^2 = 5$, and expanding gives us $4\alpha^2 - 4\alpha - 4 = 0$. 
In particular, we see that 
\[ \alpha^2 = \alpha + 1. \] 
Next, let's consider the ring $\Z[\alpha] = \{f(\alpha) : f(x) \in \Z[x]\}$. 
Since $\alpha^2 = \alpha + 1$, we have that
\[ \Z[\alpha] = \{a + b\alpha : a, b \in \Z\}, \] 
since there are no need for terms $\alpha^n$ with $n \geq 2$. What made 
this simplification work? 
\begin{enumerate}[(a)]
    \item We needed a monic polynomial $f(x) \in \Z[x]$ such that $f(\alpha) = 0$. 
    \item Moreover, notice that $5 \equiv 1 \pmod 4$, so we could nicely 
    divide all the terms by $4$ in the equation $4\alpha^2 - 4\alpha - 4 = 0$. 
\end{enumerate}
More generally, why do we want to work with $\Z[\alpha] = \Z + \Z\alpha 
+ \cdots + \Z\alpha^{n-1}$? This is because it allows us to do 
finite-dimensional $\Z$-linear algebra (actually module theory, since 
$\Z$ is not a field).

\subsection{Algebraic Integers}\label{subsec:1.2}
Now that we are properly motivated, let's introduce the algebraic integers.

\begin{defn}{defn:1.1}
    We call $\alpha \in \C$ an {\bf algebraic integer} if there 
    exists a monic polynomial $f(x) \in \Z[x]$ such that $f(\alpha) = 0$. 
\end{defn}

Note that in the above definition, we do not insist that $f(x) \in \Z[x]$ 
is irreducible.

It is not hard to see that $n$ and $\sqrt{n}$ are algebraic integers for 
all $n \in \mathbb{Z}^+$. By our previous work, we see that $(1 + \sqrt{5})/2$ 
is an algebraic integer. It can also be shown that $i$, $1 + i$ and 
$\zeta_n = e^{2\pi i/n}$ are all algebraic integers.

We can ignore all transcendental numbers here, because they are certainly 
not algebraic integers. But how do we tell if an algebraic number $\alpha \in \C$ 
(i.e. $\alpha$ is algebraic over $\Q$) is an algebraic integer? The following 
theorem gives us a simple test to do so.

\begin{theo}{theo:1.2}
    An algebraic number $\alpha \in \C$ is an algebraic integer if and only if 
    its minimal polynomial over $\Q$ has integer coefficients.
\end{theo}

For example, let us now consider $\beta = (1 + \sqrt{3})/2$ (noting that 
$3 \not\equiv 1 \pmod 4$ here). Performing the same manipulations as before, 
we deduce that $4\beta^2 - 4\beta - 2 = 0$ and hence $\beta^2 - \beta - 1/2 = 0$. 
In fact, $m(x) = x^2 - x - 1/2$ is the minimal polynomial for $\beta$ over 
$\Q$. Indeed, $m(x)$ is monic by performing the eyeball test, and it is 
irreducible since we know the roots are $(1 \pm \sqrt{3})/2$, which are 
not in $\Q$. By applying Theorem~\ref{theo:1.2}, it follows that $\beta$ is 
\emph{not} an algebraic integer. 

Another easy corollary we can obtain is that the only algebraic integers 
in $\Q$ are the ordinary integers. Indeed, the minimal polynomial of 
a rational number $q \in \Q$ is $m(x) = x - q$, which is in $\Z[x]$ 
if and only if $q \in \Z$. 