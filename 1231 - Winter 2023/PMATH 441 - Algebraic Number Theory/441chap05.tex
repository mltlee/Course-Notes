\section{Dirichlet's Unit Theorem}\label{sec:5}

\subsection{Motivation}\label{subsec:5.1}
We begin with a simple example. Let $K = \Q(\sqrt{2})$ and 
$R = {\cal O}_K = \Z[\sqrt{2}]$. We compute the group of units $R^\times$.

From ring theory, we know that $a + b\sqrt{2} \in R^\times$ if and only if 
$|a^2 - 2b^2| = 1$. By inspection, we have that $\pm 1$ and $1 + \sqrt{2}$ 
are units of $R$.

{\bf Claim.} If $u \in R^\times$ with $u > 1$, then $u \geq 1 + \sqrt{2}$. 

{\bf Proof of Claim.} Suppose that $u \in R^\times$ with $1 < u \leq 1 + \sqrt{2}$. 
To prove the claim, it suffices to show that $u = 1 + \sqrt{2}$. 
Write $u = a + b\sqrt{2}$ for some $a, b \in \Z$ and note that 
\[ |a^2 - 2b^2| = |a - b\sqrt{2}||a + b\sqrt{2}| = 1. \] 
Since $u = a + b\sqrt{2} > 1$, we must have $|a - b\sqrt{2}| < 1$, or 
equivalently, $-1 < a - b\sqrt{2} < 1$. Adding these inequalities together with 
$1 < u \leq 1 + \sqrt{2}$ yields $0 < 2a < 2+\sqrt{2}$,
which implies that $a = 1$ since $a \in \Z$. But then $1 < 1 + b\sqrt{2} 
\leq 1 + \sqrt{2}$, and the only solution is $b = 1$. We conclude that 
$a = b = 1$ and $u = 1 + \sqrt{2}$. \hfill$\blacksquare$

Suppose now that $u \in \R^\times \setminus \{\pm1\}$. By considering 
$\pm u$ and $\pm 1/u$, we may assume that $u > 1$. By the claim, we have 
$u \geq 1 + \sqrt{2}$. In particular, we can find $k \in \N$ such that 
\[ (1 + \sqrt{2})^k \leq u < (1 + \sqrt{2})^{k+1}. \] 
Dividing through by $(1 + \sqrt{2})^k$, we have $1 \leq u(1 + \sqrt{2})^{-k} 
< 1 + \sqrt{2}$. Then the claim tells us that $1 = u(1 + \sqrt{2})^{-k}$ 
and thus $u = (1 + \sqrt{2})^k$. By considering $\pm u$ and $\pm 1/u$ as before, 
we can conclude that 
\[ R^\times = \{\pm(1 + \sqrt{2})^k : k \in \Z\}. \] 
In fact, this argument can be generalized to compute ${\cal O}_K^\times$ 
where $K = \Q(\sqrt{m^2+1})$ for some $m \in \Z$.

\subsection{The Unit Theorem} \label{subsec:5.2}
The following definition is the idea of $\Z$-linear independence 
with respect to multiplication instead of addition. 

\begin{defn}{defn:5.1}
    Let $\eps_1, \dots, \eps_m \in \C \setminus \{0\}$. We say
    $\eps_1, \dots, \eps_m$ are {\bf multiplicatively independent} 
    if for $n_1, \dots, n_m \in \Z$, we have 
    $\eps_1^{n_1} \cdots \eps_m^{n_m} = 1$ if and only if 
    $n_1 = \cdots = n_m = 0$. 
\end{defn}\vspace{-0.25cm}

This leads us to Dirichlet's unit theorem, which allows us to characterize 
the units of a ring of integers.

\begin{theo}[Dirichlet's Unit Theorem]{theo:5.2}
    Let $K$ be a number field with $[K : \Q] = n$ and let $R = {\cal O}_K$. 
    Let $r$ be the number of real embeddings of $K$ in $\C$, 
    and let $s$ be the number of complex embedding pairs of $K$ in $\C$. 
    Let $m = r + s - 1$. Then there exists multiplicatively independent 
    $\eps_1, \dots, \eps_m \in K$ such that 
    \[ R^\times = \{\zeta \eps_1^{n_1} \cdots \eps_m^{n_m} : n_i \in \Z,\, 
    \zeta \text{ a root of unity in $K$}\}. \] 
    We call $\eps_1, \dots, \eps_m$ a {\bf fundamental system of units} for $K$.
\end{theo}\vspace{-0.25cm}

In our above example, the multiplicatively independent set consisted 
of just $\eps_1 = 1 + \sqrt{2}$ and the roots of unity in 
$K = \Q(\sqrt{2})$ are $\pm 1$, so the result we got is in line with 
the theorem. 
\newpage
Let's make some observations on what Dirichlet's unit theorem tells us.
\begin{enumerate}[(1)]
    \item Note that $n = r + 2s$. 
    \item Suppose that $\eps_1^{n_1} \cdots \eps_m^{n_m} = \eps_1^{k_1} 
    \cdots \eps_m^{k_m}$. We can write this as 
    \[ \eps_1^{n_1-k_1} \cdots \eps_m^{n_m-k_m} = 1. \] 
    But $\eps_1, \dots, \eps_m$ are multiplicatively independent, 
    so $n_i - k_i = 0$ and hence $n_i = k_i$ for all $i = 1, \dots, m$. 
    So multiplicative independence gives us the unique representation 
    of products in this form, similar to linear independence with addition.
    \item Suppose now that $\zeta_1 \eps_1^{n_1} \cdots \eps_m^{n_m} 
    = \zeta_2 \eps_1^{k_1} \cdots \eps_m^{k_m}$ for some roots of 
    unity $\zeta_1$ and $\zeta_2$. We can find $N \in \N$ such that 
    $\zeta_1^N = \zeta_2^N = 1$, which gives us 
    \[ \eps_1^{Nn_1} \cdots \eps_m^{Nn_m} = \eps_1^{Nk_1} \cdots \eps_m^{Nk_m}. \] 
    By (2), we have $Nn_i = Nk_i$ and hence $n_i = k_i$ for all $i = 1, \dots, m$. 
    But then $\zeta_1 = \zeta_2$ as well, so we also have unique representations 
    in the form $\zeta\eps_1^{n_1} \cdots \eps_m^{n_m}$. 
    \item {\bf Exercise.} By applying Dirichlet's unit theorem, we have 
    \[ R^\times \cong T \times \Z^m, \] 
    where $T$ is the group of roots of unity in $K$.
    This can be done by identifying $\zeta\eps_1^{n_1} \cdots \eps_m^{n_m}$ 
    with $(\zeta, (n_1, \dots, n_m))$. In other words, we are identifying
    $\eps_i$ with the standard basis vector $e_i$. 
    \item Suppose that $r > 0$ so that $K$ has a real embedding $\sigma : K \to \R$. 
    Let $\zeta \in K$ be a root of unity with $\zeta^\ell = 1$ for some 
    $\ell \in \N$. Then we have 
    \[ \sigma(\zeta)^\ell = \sigma(\zeta^\ell) = \sigma(1) = 1. \] 
    Then $\sigma(\zeta)$ is a real root of unity, so $\sigma(\zeta) \in \{\pm1\}$. 
    But $\sigma$ is injective and fixes $\Q$, so we must have $\zeta \in \{\pm1\}$.
    Therefore, in the case that $r > 0$, the group of roots of unity 
    is simply $\{\pm1\}$.
\end{enumerate}
To prove Dirichlet's unit theorem, we need the Minkowski business that 
we worked with on Assignment 5.
\begin{enumerate}[(1)]
    \item A {\bf lattice} in $\R^n$ is a set $L = \Span_{\Z}\{v_1, \dots, v_k\}$
    where $\{v_1, \dots, v_k\} \subseteq \R^n$ is $\R$-linearly independent. 
    We say that $L$ is a {\bf full lattice} if $k = n$. (This is a slight 
    relaxation of the definition we gave in \ref{A5-1}. The definition of 
    lattice we gave there required that $k = n$, which corresponds to 
    a full lattice here.)
    \item Let $K$ be a number field with $[K : \Q] = n$, and let $r$ and 
    $s$ be as usual. Let $\sigma_1, \dots, \sigma_r$ be the real embeddings 
    of $K$ in $\C$, and let $\sigma_{r+1}, \dots, \sigma_{r+s}$ be the 
    representatives of the complex pair embeddings of $K$ in $\C$. 
    Define a map $\psi : K \to \R^n$ by 
    \[ \psi(x) = (\sigma_1(x), \dots, \sigma_r(x), \dots, \Re(\sigma_{r+1}(x)), 
    \Im(\sigma_{r+1}(x)), \dots, \Re(\sigma_{r+s}(x)), \Im(\sigma_{r+s}(x))). \]
    Through an abuse of notation, we will write 
    \[ \psi(x) = (\sigma_1(x), \dots, \sigma_r(x), \sigma_{r+1}(x), \dots, \sigma_{r+s}(x)) \] 
    where we identify each $\sigma_{r+j}(x) = a + bi$ with the pair $(a, b)$.
    We call $\psi$ the {\bf Minkowski embedding}. Note that $\psi$ is 
    an embedding of additive groups; in other words, it is a map 
    $(K, +) \to (\R^n, +)$. 

    We then define $M_K := \psi({\cal O}_K)$, which we saw was a full lattice 
    in $\R^n$ by \ref{A5-2}. We call $M_K$ the {\bf Minkowski lattice} of $K$.
\end{enumerate}
The Minkowski lattice $M_K$ gives us a way to geometrically 
visualize ${\cal O}_K$. For example, consider $K = \Q(\sqrt{2})$. 
Since ${\cal O}_K$ is dense in $\R$, viewing ${\cal O}_K$ on the real line is not of much help. 
But drawing $M_K$ in $\R^2$ gives us a parallelogram type shape, which is a 
lot easier to visualize.

Next, note that $R^\times$ is a multiplicative group and $\R^n$ is an additive 
group, so the restriction of the Minkowski embedding $\psi : R^\times \to 
\psi(R^\times)$ is not a homomorphism. But we can use logarithms to transform 
multiplication into addition!

From now on, we fix the following notation. 
\begin{itemize}
    \item Let $K$ be a number field with $[K : \Q] = n$. 
    \item Let $r$ be the number of real embeddings of $K$ in $\C$, and let 
    $s$ be the number of complex embedding pairs of $K$ in $\C$. 
    Let $\sigma_1, \dots, \sigma_r$ denote the real embeddings, and 
    let $\sigma_{r+1}, \dots, \sigma_{r+s}$ be the complex pair representatives.
    \item Let $\psi : K \to \R^n$ be the Minkowski embedding, where we 
    use the abuse of notation 
    \[ \psi(x) = (\sigma_1(x), \dots, \sigma_{r+s}(x)) \]
    that we discussed above by identifying the real and imaginary parts 
    of $\sigma_{r+s}(x)$ with just $\sigma_{r+s}(x)$ itself.
    \item Let $\varphi : K^\times \to \R^{r+s}$ be the map 
    \[ \varphi(x) = (\log|\sigma_1(x)|, \dots, \log|\sigma_{r+s}(x)|). \] 
    Here, there is no abuse of notation, and $|\cdot|$ denotes the 
    complex modulus.
\end{itemize}

\begin{prop}{prop:5.3}
    The restriction $\varphi : R^\times \to \R^{r+s}$ is a group homomorphism.
\end{prop}\vspace{-0.25cm}

We leave the proof of Proposition~\ref{prop:5.3} as an exercise; 
it really just uses logarithm properties.

\begin{prop}{prop:5.4}
    We have $\varphi(R^\times) \subseteq H$, where $H$ is the hyperplane 
    \[ H = \{x \in \R^{r+s} : x_1 + \cdots + x_r + 2x_{r+1} + \cdots + 2x_{r+s} = 0\}. \] 
\end{prop}\vspace{-0.25cm}
\begin{pf}[Proposition~\ref{prop:5.4}]
    Consider a point $\varphi(a) = (\log|\sigma_1(a)|, \dots, \log|\sigma_{r+s}(a)|)$
    where $a \in R^\times$. Note that 
    \begin{align*}
        &\log|\sigma_1(a)| + \cdots + \log|\sigma_r(a)| + 
        2\log|\sigma_{r+1}(a)| + \cdots + 2\log|\sigma_{r+s}(a)| \\
        &= \log|\sigma_1(a) \cdots \sigma_r(a) \sigma_{r+1}(a)^2 \cdots 
        \sigma_{r+s}(a)^2| \\ 
        &= \log |N_{K/\Q}(a)| \\ 
        &= \log 1 = 0, 
    \end{align*}
    where the second equality follows from part (a) of \ref{A5-4}, so 
    $\varphi(a) \in H$. \qed 
\end{pf}\vspace{-0.25cm}
Recall that the dimension of a hyperplane is always one less than the ambient 
space. In particular, we see that $\dim_{\R}(H) = r + s - 1 = m$. 

We now state a couple of facts about lattices: 
\begin{enumerate}[(1)]
    \item If $L \subseteq \R^n$ is a lattice and $X \subseteq L$ is bounded, then 
    $X$ is finite. 
    \item We say that $A \subseteq \R^n$ is {\bf discrete} if for all $a \in A$, 
    there exists $\eps > 0$ such that $B_\eps(a) \cap A = \{a\}$ (intuitively, 
    the points are spread out). It is known 
    that every discrete subgroup $L \subseteq \R^n$ is a lattice; 
    in fact, this is an equivalent definition of a lattice.
\end{enumerate}

\begin{prop}{prop:5.5}
    The kernel of $\varphi : K^\times \to \R^{r+s}$ is finite. 
\end{prop}\vspace{-0.25cm}
\begin{pf}[Proposition~\ref{prop:5.5}]
    We know that $\ker\varphi \subseteq \{x \in K^\times : |\sigma_i(x)| = 1 
    \text{ for all } i = 1, \dots, r+s\} =: X$. Then $\psi(X) \subseteq M_K$ 
    is bounded as each component of the Minkowski map is in the interval $[-1, 1]$. 
    By fact (1) above, we have that $\psi(X)$ is finite. But $\psi$ is injective, 
    so $X$ is finite, and hence $\ker\varphi \subseteq X$ is finite. \qed  
\end{pf}\vspace{-0.25cm}

We recall the following result from Galois theory. 

\begin{prop}{prop:5.6}
    Let $F$ be a field. Then every finite subgroup $G \subseteq F^\times$ is cyclic.
\end{prop}\vspace{-0.25cm}
\begin{pf}[Proposition~\ref{prop:5.6}]
    By the fundamental theorem of finite abelian groups, we can write 
    \[ G \cong \Z_{n_1} \times \cdots \times \Z_{n_k} \] 
    where each $n_i$ is a prime power. Let $N = n_1 \cdots n_k = |G|$ and 
    $M = \lcm(n_1, \dots, n_k)$. It is clear that $M \leq N$. Conversely, every element 
    of $G$ (of which there are $N$ of them) is a root of $x^M - 1$, which has at most 
    $M$ roots. This means that $N \leq M$, so $M = N$. It follows that the $n_i$'s 
    are powers of distinct primes, so $G \cong \Z_{n_1} \times \cdots \times \Z_{n_k} 
    \cong \Z_N$ is cyclic. \qed 
\end{pf}\vspace{-0.25cm}

Next, let's look at the structure of the kernel of $\varphi$. 

\begin{prop}{prop:5.7}
    We have $\ker\varphi = \{\zeta \in K : \zeta \text{ a root of unity}\}$.
\end{prop}\vspace{-0.25cm}
\begin{pf}[Proposition~\ref{prop:5.7}]
    We know from Proposition~\ref{prop:5.5} that $\ker\varphi$ is finite. Write 
    $N = \lvert\ker\varphi\rvert$. By Lagrange, we know that if $x \in \ker\varphi$, 
    then $x^N = 1$. Conversely, if $\zeta \in K$ with $\zeta^\ell = 1$, we have 
    \[ \sigma_i(\zeta)^\ell = \sigma_i(\zeta^\ell) = \sigma_i(1) = 1 \] 
    for all $i = 1, \dots, r+s$. Then 
    \[ \varphi(\zeta) = (\log |\sigma_1(\zeta)|, \dots, \log|\sigma_{r+s}(\zeta)|) = (0, \dots, 0), \]
    which implies that $\zeta \in \ker\varphi$. \qed 
\end{pf}\vspace{-0.25cm}

Since $K$ is a field and $\ker\varphi \subseteq K^\times$ is a finite subgroup, 
we see that $\ker\varphi$ is cyclic by Proposition~\ref{prop:5.6}.

\begin{prop}{prop:5.8}
    We have that $\varphi(R^\times) \subseteq \R^{r+s}$ is a lattice.
\end{prop}\vspace{-0.25cm}\newpage
\begin{pf}[Proposition~\ref{prop:5.8}]
    It is enough to show that $\varphi(R^\times)$ is discrete because it is 
    clear that $\varphi(R^\times)$ being the image of a group under a 
    homomorphism is a subgroup of $\R^{r+s}$, and so we can apply fact (2)
    about lattices.

    Fix $N \in \N$ and consider the $N$-cube $X = [-N, N]^{r+s} \subseteq \R^{r+s}$. 
    Let $Y = \varphi^{-1}(X)$. For all $u \in Y$, we have $\varphi(u) \in X$, 
    implying that $\lvert\log|\sigma_i(u)|\rvert \leq N$. Then there must exist 
    some $N' \in \N$ such that $|\sigma_i(u)| \leq N'$ for all $u \in Y$.
    (This is because the logarithm is increasing, so if we could find arbitrarily 
    large $|\sigma_i(u)|$, then $\lvert\log |\sigma_i(u)|\rvert$ would also be 
    arbitrarily large.) It follows that $\psi(Y) \subseteq M_K$ is finite by 
    lattice fact (1). But $\psi$ is injective, so $Y$ is finite. 
    By a set counting argument (exercise), it follows that $\varphi(R^\times) \cap X$ 
    is finite. But every finite set is discrete, so we are done. \qed 
\end{pf}\vspace{-0.25cm}

The following result is then enough to prove Dirichlet's unit theorem. 

\begin{prop}{prop:5.9}
    We have $W := \Span_{\R} \varphi(R^\times) = H$. 
\end{prop}\vspace{-0.25cm}

We'll save the proof of Proposition~\ref{prop:5.9} for later, as it is quite tricky. 
First, let's see how it implies Dirichlet's unit theorem (Theorem~\ref{theo:5.2}). 

\begin{pf}[Dirichlet]
    We know that $\varphi(R^\times) \subseteq \R^{r+s}$ is a lattice by 
    Proposition~\ref{prop:5.8}, so we can write $\varphi(R^\times) = 
    \Span_{\Z}\{u_1, \dots, u_k\}$ where $\{u_1, \dots, u_k\}$ is $\R$-linearly independent. 
    Then by Proposition~\ref{prop:5.9}, we have 
    \[ H = \Span_{\R} \varphi(R^\times) = \Span_{\R}\{u_1, \dots, u_k\} \cong \Z^k. \] 
    This means that $\{u_1, \dots, u_k\}$ is a basis for $H$, so $k = \dim_{\R} H = r + s - 1$.
    This gives us $\varphi(R^\times) \cong \Z^{r+s-1}$. Combining this with the first 
    isomorphism theorem, it follows that there exists an isomorphism  
    \[ \rho : R^\times/\ker\varphi \to \Z^{r+s-1}. \] 
    For all $i = 1, \dots, r+s-1$, we can pick a representative $\eps_i \ker\varphi \in 
    R^\times/\ker\varphi$ such that $\rho(\eps_i \ker\varphi) = e_i$, where $e_i$ 
    is the $i$-th standard basis vector in $\Z^{r+s-1}$. For all $u \in R^\times$, we 
    can write 
    \[ \rho(u\ker\varphi) = n_1 e_1 + \cdots + n_m e_m \] 
    for some $n_i \in \Z$. But then $u\ker\varphi = \eps_1^{n_1} \cdots \eps_m^{n_m} 
    \ker\varphi$ since $\varphi$ is an isomorphism. In particular, there 
    exists $\zeta \in \ker\varphi$ such that $u = \zeta \eps_1^{n_1} \cdots \eps_m^{n_m}$. 
    Note that $\eps_1, \dots, \eps_m$ is a multiplicatively independent set since 
    the standard basis vectors $e_i$ are $\Z$-linearly independent, which completes the proof. \qed 
\end{pf}\vspace{-0.25cm}

Finally, we prove Proposition~\ref{prop:5.9} using Minkowski's lemma (\ref{A5-3}),
which states that if a Lebesgue 
measurable set is compact, convex, and symmetric about the origin with ``large'' 
volume, then it contains some nonzero point from the lattice.

\begin{pf}[Proposition~\ref{prop:5.9}]
    We already know that $W \subseteq H$. To prove that $H \subseteq W$, we show that 
    $W^\perp \subseteq H^\perp$, noting that orthogonal complements reverse 
    inclusion. To do this, we'll proceed by contrapositive. 
    Suppose that $z = (z_1, \dots, z_{r+s}) \in \R^{r+s}$ 
    with $z \notin H^\perp$. We claim that $z \notin W^\perp$. 

    Define the map $f : K^\times \to \R$ by $f(x) = z \cdot \varphi(x)$, where $\cdot$ 
    denotes the dot product in $\R^{r+s}$. If we can show that 
    there exists some $u \in R^\times$ such that $f(u) \neq 0$, then we're done. 
    Let $C = (2/\pi)^s \lvert\disc(K)\rvert^{1/2}$ 
    and pick any positive $c_1, \dots, c_{r+s} \in \R$ such that 
    \[ C = c_1 \cdots c_r c_{r+1}^2 \cdots c_{r+s}^2. \] 
    This is always possible since $C \in \R$; one possible choice is $c_1 = C$ and 
    $c_i = 1$ otherwise. 
    Then the set 
    \[ A := \{(x_1, \dots, x_n) \in \R^n : |x_i| \leq c_i \text{ for all $i = 1, \dots, r$ 
    and $x_i^2 + x_{i+s}^2 \leq c_i^2$ for all $i = r+1, \dots, r+s$}\} \] 
    is compact, convex, symmetric about the origin, and Lebesgue measurable. In particular, 
    we have
    \[ m(A) = \left( \prod_{i=1}^r 2c_i \right) \left( \prod_{i=r+1}^{r+s} \pi c_i^2 \right) 
    = 2^r \pi^s C = 2^{r+s} \lvert\disc(K)\rvert^{1/2} = 2^n \cdot \frac1{2^s}\lvert\disc(K)\rvert^{1/2}
    = 2^n \Vol(M_K). \] 
    By Minkowski's lemma (\ref{A5-3}), there exists a nonzero $a \in A \cap M_K$. We write 
    $a = \psi(b)$ for some $b \in {\cal O}_K$. 

    We have that $|N_{K/\Q}(b)| = N(a) \leq c_1 \cdots c_r c_{r+1}^2 
    \cdots c_{r+s}^2 = C$, where $N(\cdot)$ denotes the map from \ref{A5-4} and 
    the inequality is from the fact that $a \in A$.

    {\bf Claim.} We have $|\sigma_i(b)| \geq c_i/C$ for all $i = 1, \dots, r$
    and $|\sigma_i(b)|^2 \geq c_i^2/C$ for all $i = r+1, \dots, r+s$.

    {\bf Proof of Claim.} Since $a = \psi(b) = (\sigma_1(b), \dots, \sigma_{r+s}(b)) \in A$, 
    we have that $|\sigma_i(b)| \leq c_i$ for all $i = 1, \dots, r$ and 
    $|\sigma_i(b)|^2 \leq c_i^2$ for all $i = r+1, \dots, r+s$. 

    Suppose towards a contradiction that there exists $i = 1, \dots, r$ with 
    $|\sigma_i(b)| < c_i/C$. Then we have 
    \[ 1 \leq |N_{K/\Q}(b)| = |\sigma_1(b)| \cdots |\sigma_r(b)| |\sigma_{r+1}(b)|^2 
    \cdots |\sigma_{r+s}(b)|^2 < \frac{c_1 \cdots c_r c_{r+1}^2 \cdots c_{r+s}^2}{C} 
    = \frac{C}{C} = 1, \] 
    which is a contradiction. If we assume there exists $i = r+1, \dots, r+s$ 
    such that $|\sigma_i(b)| < c_i^2/C$, the same argument applies. \hfill $\blacksquare$ 

    There are finitely many nonzero principal ideals of ${\cal O}_K$ 
    with norm at most $C$, say $\langle b_1 \rangle, \dots, \langle b_\ell \rangle$. 
    Without loss of generality, suppose that $\langle b \rangle = \langle b_1 \rangle$. 
    Then we can write $b = ub_1$ for some $u \in R^\times$. 
    
    Let $L = z \cdot (\log c_1, \dots, \log c_{r+s}) = z_1 \log c_1 + \cdots + z_{r+s} \log c_{r+s}$. Note that 
    \begin{align*}
        f(b) = f(ub_1) &= z \cdot \varphi(ub_1) \\ 
        &= z \cdot (\varphi(u) + \varphi(b_1)) \\
        &= f(u) + f(b_1). 
    \end{align*}
    From this, it can be shown using the claim and the triangle inequality that 
    \[ |f(u) - L| \leq |f(b_1)| + \left( \sum_{i=1}^r |z_i| + \frac12 \sum_{i=r+1}^{r+s} |z_i| \right) 
    \cdot \log C =: B. \] 
    Note that $B$ depends only on $z$ and $C$, and not the choice of 
    $c_1, \dots, c_{r+s}$.

    {\bf Case 1.} If $r + s - 1 = 0$, then $\dim_{\R} H = r + s - 1 = 0$ 
    so that $W \subseteq H = \{0\}$, and hence $W = H$. 

    {\bf Case 2.} Suppose that $r + s - 1 > 0$. By replacing $z \notin H^\perp$
    with a multiple of $[1, \dots, 1, 2, \dots, 2]^T \in H^\perp$ 
    (where $1$ appears $r$ times and $2$ appears $s$ times), we 
    may assume that there exists $p, q \in \{1, \dots, r+s\}$ such that 
    $z_p = 0$ and $z_q \neq 0$. Now, pick representatives $c_1, \dots, 
    c_{r+s} > 0$ with $C = c_1 \cdots c_r c_{r+1}^2 \cdots c_{r+s}^2$ such that 
    \begin{enumerate}[(1)]
        \item $c_q$ is arbitrarily large; 
        \item $c_p$ is arbitrarily small; 
        \item $c_i = 1$ for all $i \notin \{p, q\}$. 
    \end{enumerate}
    Note that we have $L = z \cdot (\log c_1, \dots, \log c_{r+s}) 
    = z_q \log c_q$ since $z_p = 0$ and $c_i = 1$ for all $i \notin \{p, q\}$. 
    By a careful choice of $c_p$ and $c_q$, we may assume that $|L| = |z_q \log c_q| > B$. 
    But we also have $|f(u) - L| \leq B$, which implies that $f(u) \neq 0$. 
    We conclude that $f(u) = z \cdot \varphi(u) \neq 0$, so $z \notin W^\perp$. \qed 
\end{pf}\newpage  

We revisit the example from the end of Section~\ref{subsec:4.2}. 
Let $K = \Q(\alpha)$ where $\alpha \in \C$ is a root of $f(x) = x^3 + 4x + 1$. 
The discriminant of $f(x)$ is $-283$, which is prime, so $R = {\cal O}_K = \Z[\alpha]$. 
We claimed that $R$ was not a PID. We had reduced the problem to determining 
if the ideal $P = \langle \alpha + 1, 2 \rangle$ was principal or not, 
and now we have the tools to show that it is in fact not principal.

Suppose that $P = \langle \alpha + 1, 2 \rangle$ were principal with 
$P = \langle \beta \rangle$. Note that $\alpha \in R^\times$ since 
the constant term of $f(x)$ is $1$ and hence $|N_{K/\Q}(\alpha)| = 1$.
In particular, we also have $P = \langle \alpha^t \beta \rangle$ for all $t \in \Z$. 

Let $\alpha_1, \alpha_2, \alpha_3$ be the roots of $f(x)$ where $\alpha_1 \in \R$ 
and $\alpha_2, \alpha_3 \in \C \setminus \R$ with $\overline{\alpha_2} = \alpha_3$. 
The embeddings of $K$ in $\C$ are $\sigma_i(\alpha) = \alpha_i$ for $i \in \{1, 2, 3\}$. 
Using a computer, we find that 
\[ \alpha_1 \approx 0.24627 \approx 1/4. \] 
This implies that $1 = |N_{K/\Q}(\alpha)| = |\alpha_1||\alpha_2|^2 \approx 
|\alpha_2|^2/4$, so $|\alpha_2| \approx 2$. Then we have 
\[ \varphi(\alpha) \approx (\log 1/4, \log 2) = (-\log 4, \log 2). \] 
Write $\varphi(\beta) = (\log |\beta_1|, \log |\beta_2|)$ where 
$\beta_1 = \sigma_1(\beta)$ and $\beta_2 = \sigma_2(\beta)$. This yields 
\[ \varphi(\alpha^t \beta) = t\varphi(\alpha) + \varphi(\beta) 
\approx (-t \log 4 + \log |\beta_1|, t\log 2 + \log|\beta_2|). \] 
As we alluded to before, we now reduce to the ``small'' cases where 
\[ 0 \leq \log |\beta_1| \leq \log 4, \] 
or equivalently, $1 \leq |\beta_1| \leq 4$. But we also have 
$2 = N(P) = |N_{K/\Q}(\beta)| = |\beta_1||\beta_2|^2$, which combined with
\[ |\beta_2|^2 \leq |\beta_1||\beta_2|^2 \leq 4|\beta_2|^2 \] 
gives $\frac{1}{\sqrt{2}} \leq |\beta_2| \leq \sqrt{2}$. In \ref{A9-2}, 
we show that the only possibilities are $\beta \in \{1, 3 + \alpha^2\}$. 

It is clear that $P \neq \langle 1 \rangle$, and we have $P \neq 
\langle 3 + \alpha^2 \rangle$ by checking that $|N_{K/\Q}(3+\alpha^2)| \neq 2$. 
It follows that $P$ is not principal, so $R$ is not a PID. 


