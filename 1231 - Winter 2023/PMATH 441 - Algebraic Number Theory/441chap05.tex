\section{Dirichlet's Unit Theorem}\label{sec:5}

\subsection{Motivation}\label{subsec:5.1}
We begin with a simple example. Let $K = \Q(\sqrt{2})$ and 
$R = {\cal O}_K = \Z[\sqrt{2}]$. We compute the group of units $R^\times$.

From ring theory, we know that $a + b\sqrt{2} \in R^\times$ if and only if 
$|a^2 - 2b^2| = 1$. By inspection, we have that $\pm 1$ and $1 + \sqrt{2}$ 
are units of $R$.

{\bf Claim.} If $u \in R^\times$ with $u > 1$, then $u \geq 1 + \sqrt{2}$. 

{\bf Proof of Claim.} Suppose that $u \in R^\times$ with $1 < u \leq 1 + \sqrt{2}$. 
To prove the claim, it suffices to show that $u = 1 + \sqrt{2}$. 
Write $u = a + b\sqrt{2}$ for some $a, b \in \Z$ and note that 
\[ |a^2 - 2b^2| = |a - b\sqrt{2}||a + b\sqrt{2}| = 1. \] 
Since $u = a + b\sqrt{2} > 1$, we must have $|a - b\sqrt{2}| < 1$, or 
equivalently, $-1 < a - b\sqrt{2} < 1$. Adding these inequalities together with 
$1 < u \leq 1 + \sqrt{2}$ yields $0 < 2a < 2+\sqrt{2}$,
which implies that $a = 1$ since $a \in \Z$. But then $1 < 1 + b\sqrt{2} 
\leq 1 + \sqrt{2}$, and the only solution is $b = 1$. We conclude that 
$a = b = 1$ and $u = 1 + \sqrt{2}$. \hfill$\blacksquare$

Suppose now that $u \in \R^\times \setminus \{\pm1\}$. By considering 
$\pm u$ and $\pm 1/u$, we may assume that $u > 1$. By the claim, we have 
$u \geq 1 + \sqrt{2}$. In particular, we can find $k \in \N$ such that 
\[ (1 + \sqrt{2})^k \leq u < (1 + \sqrt{2})^{k+1}. \] 
Dividing through by $(1 + \sqrt{2})^k$, we have $1 \leq u(1 + \sqrt{2})^{-k} 
< 1 + \sqrt{2}$. Then the claim tells us that $1 = u(1 + \sqrt{2})^{-k}$ 
and thus $u = (1 + \sqrt{2})^k$. By considering $\pm u$ and $\pm 1/u$ as before, 
we can conclude that 
\[ R^\times = \{\pm(1 + \sqrt{2})^k : k \in \Z\}. \] 
In fact, this argument can be generalized to compute ${\cal O}_K^\times$ 
where $K = \Q(\sqrt{m^2+1})$ for some $m \in \Z$.

\subsection{The Unit Theorem} \label{subsec:5.2}
The following definition is the idea of $\Z$-linear independence 
with respect to multiplication instead of addition. 

\begin{defn}{defn:5.1}
    Let $\eps_1, \dots, \eps_m \in \C \setminus \{0\}$. We say
    $\eps_1, \dots, \eps_m$ are {\bf multiplicatively independent} 
    if for $n_1, \dots, n_m \in \Z$, we have 
    $\eps_1^{n_1} \cdots \eps_m^{n_m} = 1$ if and only if 
    $n_1 = \cdots = n_m = 0$. 
\end{defn}\vspace{-0.25cm}

This leads us to Dirichlet's unit theorem, which allows us to characterize 
the units of a ring of integers.

\begin{theo}[Dirichlet's Unit Theorem]{theo:5.2}
    Let $K$ be a number field with $[K : \Q] = n$ and let $R = {\cal O}_K$. 
    Let $r$ be the number of real embeddings of $K$ in $\C$, 
    and let $s$ be the number of complex embedding pairs of $K$ in $\C$. 
    Let $m = r + s - 1$. Then there exists multiplicatively independent 
    $\eps_1, \dots, \eps_m \in K$ such that 
    \[ R^\times = \{\zeta \eps_1^{n_1} \cdots \eps_m^{n_m} : n_i \in \Z,\, 
    \zeta \text{ a root of unity in $K$}\}. \] 
    We call $\eps_1, \dots, \eps_m$ a {\bf fundamental system of units} for $K$.
\end{theo}\vspace{-0.25cm}

In our above example, the multiplicatively independent set consisted 
of just $\eps_1 = 1 + \sqrt{2}$ and the roots of unity in 
$K = \Q(\sqrt{2})$ are $\pm 1$, so the result we got is in line with 
the theorem. 
\newpage
Let's make some observations on what Dirichlet's unit theorem tells us.
\begin{enumerate}[(1)]
    \item Note that $n = r + 2s$. 
    \item Suppose that $\eps_1^{n_1} \cdots \eps_m^{n_m} = \eps_1^{k_1} 
    \cdots \eps_m^{k_m}$. We can write this as 
    \[ \eps_1^{n_1-k_1} \cdots \eps_m^{n_m-k_m} = 1. \] 
    But $\eps_1, \dots, \eps_m$ are multiplicatively independent, 
    so $n_i - k_i = 0$ and hence $n_i = k_i$ for all $i = 1, \dots, m$. 
    So multiplicative independence gives us the unique representation 
    of products in this form, similar to linear independence.
    \item Suppose now that $\zeta_1 \eps_1^{n_1} \cdots \eps_m^{n_m} 
    = \zeta_2 \eps_1^{k_1} \cdots \eps_m^{k_m}$ for some roots of 
    unity $\zeta_1$ and $\zeta_2$. We can find $N \in \N$ such that 
    $\zeta_1^N = \zeta_2^N = 1$, which gives us 
    \[ \eps_1^{Nn_1} \cdots \eps_m^{Nn_m} = \eps_1^{Nk_1} \cdots \eps_m^{Nk_m}. \] 
    By (2), we have $Nn_i = Nk_i$ and hence $n_i = k_i$ for all $i = 1, \dots, m$. 
    But then $\zeta_1 = \zeta_2$ as well, so we also have unique representations 
    in the form $\zeta\eps_1^{n_1} \cdots \eps_m^{n_m}$. 
    \item {\bf Exercise.} By applying Dirichlet's unit theorem, we have 
    \[ R^\times \cong T \times \Z^m, \] 
    where $T$ is the group of roots of unity in $K$.
    This can be done by identifying $\zeta\eps_1^{n_1} \cdots \eps_m^{n_m}$ 
    with $(\zeta, (n_1, \dots, n_m))$. In other words, we are identifying
    $\eps_i$ with the standard basis vector $e_i$. 
    \item Suppose that $r > 0$ so that $K$ has a real embedding $\sigma : K \to \R$. 
    Let $\zeta \in K$ be a root of unity with $\zeta^\ell = 1$ for some 
    $\ell \in \N$. Then we have 
    \[ \sigma(\zeta)^\ell = \sigma(\zeta^\ell) = \sigma(1) = 1. \] 
    Then $\sigma(\zeta)$ is a real root of unity, so $\sigma(\zeta) \in \{\pm1\}$. 
    But $\sigma$ is injective and fixes $\Q$, so we must have $\zeta \in \{\pm1\}$.
    Therefore, in the case that $r > 0$, the group of roots of unity 
    is simply $\{\pm1\}$.
\end{enumerate}
To prove Dirichlet's unit theorem, we need the Minkowski business that 
we worked with on Assignment 5.
\begin{enumerate}[(1)]
    \item A {\bf lattice} in $\R^n$ is a set $L = \Span_{\Z}\{v_1, \dots, v_k\}$
    where $\{v_1, \dots, v_k\} \subseteq \R^n$ is $\R$-linearly independent. 
    We say that $L$ is a {\bf full lattice} if $k = n$. (This is a slight 
    relaxation of the definition we gave in \ref{A5-1}. The definition of 
    lattice we gave there required that $k = n$, which corresponds to 
    a full lattice here.)
    \item Let $K$ be a number field with $[K : \Q] = n$, and let $r$ and 
    $s$ be as usual. Let $\sigma_1, \dots, \sigma_r$ be the real embeddings 
    of $K$ in $\C$, and let $\sigma_{r+1}, \dots, \sigma_{r+s}$ be the 
    representatives of the complex pair embeddings of $K$ in $\C$. 
    Define a map $\psi : K \to \R^n$ by 
    \[ \psi(x) = (\sigma_1(x), \dots, \sigma_r(x), \dots, \Re(\sigma_{r+1}(x)), 
    \Im(\sigma_{r+1}(x)), \dots, \Re(\sigma_{r+s}(x)), \Im(\sigma_{r+s}(x))). \]
    Through an abuse of notation, we will write 
    \[ \psi(x) = (\sigma_1(x), \dots, \sigma_r(x), \sigma_{r+1}(x), \dots, \sigma_{r+s}(x)) \] 
    where we identify each $\sigma_{r+j}(x) = a + bi$ with the pair $(a, b)$.
    We call $\psi$ the {\bf Minkowski embedding}. Note that $\psi$ is 
    an embedding of additive groups; in other words, it is a map 
    $(K, +) \to (\R^n, +)$. 

    We then define $M_K := \psi({\cal O}_K)$, which we saw was a full lattice 
    in $\R^n$ by \ref{A5-2}. We call $M_K$ the {\bf Minkowski lattice} of $K$.
\end{enumerate}
The Minkowski lattice $M_K$ gives us a way to geometrically 
visualize ${\cal O}_K$. For example, consider $K = \Q(\sqrt{2})$. 
Since ${\cal O}_K$ is dense in $\R$, viewing ${\cal O}_K$ on the real line is not of much help. 
But drawing $M_K$ in $\R^2$ gives us a parallelogram type shape, which is a 
lot easier to visualize.

Next, note that $R^\times$ is a multiplicative group and $\R^n$ is an additive 
group, so the restriction of the Minkowski embedding $\psi : R^\times \to 
\psi(R^\times)$ is not a homomorphism. But we can use logarithms to transform 
multiplication into addition!

From now on, we fix the following notation. 
\begin{itemize}
    \item Let $K$ be a number field with $[K : \Q] = n$. 
    \item Let $r$ be the number of real embeddings of $K$ in $\C$, and let 
    $s$ be the number of complex embedding pairs of $K$ in $\C$. 
    Let $\sigma_1, \dots, \sigma_r$ denote the real embeddings, and 
    let $\sigma_{r+1}, \dots, \sigma_{r+s}$ be the complex pair representatives.
    \item Let $\psi : K \to \R^n$ be the Minkowski embedding, where we 
    use the abuse of notation 
    \[ \psi(x) = (\sigma_1(x), \dots, \sigma_{r+s}(x)) \]
    that we discussed above by identifying the real and imaginary parts 
    of $\sigma_{r+s}(x)$ with just $\sigma_{r+s}(x)$ itself.
    \item Let $\varphi : K^\times \to \R^{r+s}$ be the map 
    \[ \varphi(x) = (\log|\sigma_1(x)|, \dots, \log|\sigma_{r+s}(x)|). \] 
    Here, there is no abuse of notation, and $|\cdot|$ denotes the 
    complex modulus.
\end{itemize}

\begin{prop}{prop:5.3}
    The restriction $\varphi : R^\times \to \R^{r+s}$ is a group homomorphism.
\end{prop}\vspace{-0.25cm}

We leave the proof of Proposition~\ref{prop:5.3} as an exercise; 
it really just uses logarithm properties.

\begin{prop}{prop:5.4}
    We have $\varphi(R^\times) \subseteq H$, where $H$ is the hyperplane 
    \[ H = \{x \in \R^{r+s} : x_1 + \cdots + x_r + 2x_{r+1} + \cdots + 2x_{r+s} = 0\}. \] 
\end{prop}\vspace{-0.25cm}
\begin{pf}[Proposition~\ref{prop:5.4}]
    Consider a point $\varphi(a) = (\log|\sigma_1(a)|, \dots, \log|\sigma_{r+s}(a)|)$
    where $a \in R^\times$. Note that 
    \begin{align*}
        &\log|\sigma_1(a)| + \cdots + \log|\sigma_r(a)| + 
        2\log|\sigma_{r+1}(a)| + \cdots + 2\log|\sigma_{r+s}(a)| \\
        &= \log|\sigma_1(a) \cdots \sigma_r(a) \sigma_{r+1}(a)^2 \cdots 
        \sigma_{r+s}(a)^2| \\ 
        &= \log |N_{K/\Q}(a)| \\ 
        &= \log 1 = 0, 
    \end{align*}
    where the second equality follows from part (a) of \ref{A5-4}, so 
    $\varphi(a) \in H$. \qed 
\end{pf}\vspace{-0.25cm}
Recall that the dimension of a hyperplane is always one less than the ambient 
space. In particular, we see that $\dim_{\R}(H) = r + s - 1 = m$. 