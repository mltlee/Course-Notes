\section{Discriminants}\label{sec:2}

\subsection{Elementary Properties}\label{subsec:2.1}
Let $K$ be a number with $[K : \Q] = n$ and consider its ring of integers 
$R = {\cal O}_K$. Given $\{v_1, \dots, v_n\} \subseteq R$, we want to find a 
way to discriminate whether or not $\{v_1, \dots, v_n\}$ is an integral 
basis for $R$. This leads us to the notion of the discriminant. 

\begin{defn}{defn:2.1}
    Let $K$ be a number field with $[K : \Q] = n$. Let $\sigma_1, 
    \dots, \sigma_n$ be the embeddings of $K$ into $\C$. The 
    {\bf discriminant} of $\{a_1, \dots, a_n\} \subseteq K$ is 
    \[ \disc(a_1, \dots, a_n) = \det\,[\sigma_i(a_j)]^2. \] 
\end{defn}\vspace{-0.15cm}

In the matrix $[\sigma_i(a_j)]$ above, the rows are encoded by the 
embeddings $\sigma_i$ and the columns are encoded by the elements $a_j$. 
Now, let's investigate some properties of the discriminant. 
\begin{enumerate}[(1)]
    \item The discriminant is independent of the choice of ordering 
    for both the embeddings $\sigma_i$ and the elements $a_j$. This is 
    because squaring the determinant kills any negatives obtained 
    by flipping rows or columns.

    \item Let $B = [\sigma_i(a_j)]$ and $A = [\sigma_j(a_i)] = B^T$. 
    Taking the transpose leaves the determinant unchanged, so
    \[ \disc(a_1, \dots, a_n) = \det(AB). \] 
    Now, observe that the $(i, j)$-th entry of $AB$ is 
    \[ \begin{bmatrix} \sigma_1(a_i) \\ \sigma_2(a_i) \\ \vdots \\ \sigma_n(a_i) \end{bmatrix} 
    \cdot \begin{bmatrix} \sigma_1(a_j) \\ \sigma_2(a_j) \\ \vdots \\ \sigma_n(a_j) \end{bmatrix}
    = \sum_{k=1}^n \sigma_k(a_i a_j) = \Tr_{K/\Q}(a_ia_j). \]
    Thus, we obtain an equivalent definition of the discriminant seen in 
    some texts, which is given by 
    \[ \disc(a_1, \dots, a_n) = \det\,[\Tr_{K/\Q}(a_ia_j)] \in \Q. \] 
    Moreover, if we also assume that $a_1, \dots, a_n \in {\cal O}_K$, then 
    \[ \disc(a_1, \dots, a_n) \in \Q \cap {\cal O}_K = \Z. \] 

    \item Let $v, w \in K^n$ and $A \in M_n(\Q)$ be such that $Av = w$. 
    Observe that 
    \[ A \begin{bmatrix} \sigma_i(v_1) \\ \vdots \\ \sigma_i(v_n) \end{bmatrix} 
    = \begin{bmatrix} \sigma_i(a_{11}v_1 + \cdots + a_{1n}v_n) \\ \vdots \\ 
        \sigma_i(a_{n1}v_1 + \cdots + a_{nn}v_n) \end{bmatrix} 
    = \begin{bmatrix} \sigma_i(w_1) \\ \vdots \\ \sigma_i(w_n) \end{bmatrix}. \] 
    Therefore, we deduce that 
    \[ A \begin{bmatrix} \sigma_1(v_1) & \cdots & \sigma_n(v_1) \\ 
        \vdots & \ddots & \vdots \\ 
        \sigma_1(v_n) & \cdots & \sigma_n(v_n) \end{bmatrix} 
    = \begin{bmatrix} \sigma_1(w_1) & \cdots & \sigma_n(w_1) \\ 
        \vdots & \ddots & \vdots \\ 
        \sigma_1(w_n) & \cdots & \sigma_n(w_n) \end{bmatrix}. \] 
    The matrices above are the transposes of the matrices in the 
    definition of the discriminant, since the columns encode the embeddings 
    and the rows encode the elements this time. Taking squared determinants 
    gives the nice relationship
    \[ (\det A)^2 \cdot \disc(v) = \disc(w). \] 

    \item Let $\{v_1, \dots, v_n\} \subseteq {\cal O}_K$ be an integral basis.
    Suppose that $\{w_1, \dots, w_n\} \subseteq {\cal O}_K$. Then 
    for all $i = 1, \dots, n$, there must exist some $c_{ij} \in \Z$ such that 
    \[ w_i = c_{i1} v_1 + \cdots + c_{in} v_n. \] 
    Then we can write $w = Cv$ where $C = [c_{ij}]$, which yields 
    \[ \disc(w) = (\det C)^2 \cdot \disc(v). \] 
\end{enumerate}