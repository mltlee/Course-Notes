\section{Prime Factorization}\label{sec:3}

\subsection{Ring Theory}\label{subsec:3.1}
Let $K$ be a number field and let $R = {\cal O}_K$ be its ring of integers. 
Recall that in Section~\ref{subsec:1.4}, we uncovered some purely 
ring theoretic facts about $R$. We'll restate them here as they'll be 
very useful to us soon. Note that when we speak about rings in this 
course, they are always commutative and unital.
\begin{enumerate}[(1)]
    \item Corollary~\ref{cor:1.19}: If $I$ is a nonzero ideal of $R$, then 
    $R/I$ is finite. 
    \item Corollary~\ref{cor:1.20}: Every nonzero prime ideal of $R$ is maximal.
    \item Corollary~\ref{cor:1.21}: $R$ is Noetherian. 
\end{enumerate}

We now give a few characterizations of being a Noetherian ring.

\begin{prop}{prop:3.1}
    Let $R$ be a ring. The following are equivalent:
    \begin{enumerate}[(1)]
        \item $R$ is Noetherian. 
        \item If $I_1 \subseteq I_2 \subseteq \cdots$ is an ascending 
        chain of ideals of $R$, then there exists some $N \in \N$ such that 
        $I_k = I_N$ for all $k \geq N$; that is, the chain terminates.
        \item Every nonempty set of ideals of $R$ has a maximal element 
        (with respect to $\subseteq$). 
    \end{enumerate}
\end{prop}\vspace{-0.25cm}
\begin{pf}[Proposition~\ref{prop:3.1}]
    $(1) \Rightarrow (2)$: Let $I_1 \subseteq I_2 \subseteq \cdots$ be 
    an ascending chain of ideals in $R$. Let $I = \bigcup_{j\in\N} I_j$, 
    and note that $I$ is an ideal of $R$. Since $R$ is Noetherian, 
    we see that $I$ is finitely generated, say by $a_1, \dots, a_s$. 
    Then for each $i = 1, \dots, s$, there exists some $N_i \in \N$ such that 
    $a_i \in I_{N_i}$. Taking $N = \max\{N_1, \dots, N_s\}$, we have 
    $a_i \in I_N$ for all $i = 1, \dots, s$ and thus $I \subseteq I_N$. 
    But $I_N \subseteq I$ by definition, so equality follows. In particular, 
    we have $I_k = I_N$ for all $k \geq N$. 
    
    $(2) \Rightarrow (3)$: Suppose ${\cal I}$ is a nonempty set of ideals of 
    $R$, and let $I_1 \in {\cal I}$ (which exists because ${\cal I}$ is 
    nonempty). If $I_1$ is maximal, we're done. Otherwise, ${\cal I} \setminus 
    \{I_1\}$ must be nonempty; we can find $I_2$ from this collection 
    such that $I_1 \subseteq I_2$ (else $I_1$ was maximal). If 
    $I_2$ is not maximal, pick $I_3 \in {\cal I} \setminus \{I_1, I_2\}$ 
    such that $I_1 \subseteq I_2 \subseteq I_3$. But by assumption, this 
    process terminates; for some $N \in \N$, we have $I_k = I_N$ 
    for all $k \geq N$, and $I_N$ is our desired maximal element in ${\cal I}$.

    $(3) \Rightarrow (1)$: Let $I$ be an ideal of $R$. Let ${\cal I}$ 
    denote the collection of all finitely generated ideals of $R$ 
    contained in $I$, which is nonempty because $\langle 0 \rangle 
    \in {\cal I}$. By assumption, ${\cal I}$ has a maximal element $J$. 
    If $J \neq I$, then we can find some $a \in I \setminus J$. Then 
    $\langle J, a \rangle$ is also finitely generated and contained in 
    ${\cal I}$, contradicting maximality. It follows that $J = I$ 
    and so $I$ is finitely generated. \qed 
\end{pf}\vspace{-0.25cm}

Note that the rings that we work with in this course are not generally UFDs, 
so we do not have the classical prime factorization from first year number 
theory. However, the following proposition gives us the idea to 
consider the factorization of proper ideals into prime ideals. 
The reason why we are only considering proper ideals here is because 
taking $I = R$ fails condition (1); every prime ideal of $R$ is proper by 
definition.

\begin{prop}{prop:3.2}
    Let $R$ be Noetherian and let $I \neq R$ be an ideal. 
    There exist proper ideals $P_1, \dots, P_n$ of $R$ such that 
    \begin{enumerate}[(1)]
        \item $I \subseteq P_i$ for all $i = 1, \dots, n$; and 
        \item $P_1P_2 \cdots P_n \subseteq I$. 
    \end{enumerate}
\end{prop}
