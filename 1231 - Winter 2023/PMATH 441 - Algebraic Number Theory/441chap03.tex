\section{Prime Factorization}\label{sec:3}

\subsection{Ring Theory}\label{subsec:3.1}
Let $K$ be a number field and let $R = {\cal O}_K$ be its ring of integers. 
Recall that in Section~\ref{subsec:1.4}, we uncovered some purely 
ring theoretic facts about $R$. We'll restate them here as they'll be 
very useful to us soon. Note that when we speak about rings in this 
course, they are always commutative and unital.
\begin{enumerate}[(1)]
    \item Corollary~\ref{cor:1.19}: If $I$ is a nonzero ideal of $R$, then 
    $R/I$ is finite. 
    \item Corollary~\ref{cor:1.20}: Every nonzero prime ideal of $R$ is maximal.
    \item Corollary~\ref{cor:1.21}: $R$ is Noetherian. 
\end{enumerate}

We now give a few characterizations of being a Noetherian ring.

\begin{prop}{prop:3.1}
    Let $R$ be a ring. The following are equivalent:
    \begin{enumerate}[(1)]
        \item $R$ is Noetherian. 
        \item If $I_1 \subseteq I_2 \subseteq \cdots$ is an ascending 
        chain of ideals of $R$, then there exists some $N \in \N$ such that 
        $I_k = I_N$ for all $k \geq N$; that is, the chain terminates.
        \item Every nonempty set of ideals of $R$ has a maximal element 
        (with respect to $\subseteq$). 
    \end{enumerate}
\end{prop}\vspace{-0.25cm}
\begin{pf}[Proposition~\ref{prop:3.1}]
    $(1) \Rightarrow (2)$: Let $I_1 \subseteq I_2 \subseteq \cdots$ be 
    an ascending chain of ideals in $R$. Let $I = \bigcup_{j\in\N} I_j$, 
    and note that $I$ is an ideal of $R$. Since $R$ is Noetherian, 
    we see that $I$ is finitely generated, say by $a_1, \dots, a_s$. 
    Then for each $i = 1, \dots, s$, there exists some $N_i \in \N$ such that 
    $a_i \in I_{N_i}$. Taking $N = \max\{N_1, \dots, N_s\}$, we have 
    $a_i \in I_N$ for all $i = 1, \dots, s$ and thus $I \subseteq I_N$. 
    But $I_N \subseteq I$ by definition, so equality follows. In particular, 
    we have $I_k = I_N$ for all $k \geq N$. 
    
    $(2) \Rightarrow (3)$: Suppose ${\cal I}$ is a nonempty set of ideals of 
    $R$, and let $I_1 \in {\cal I}$ (which exists because ${\cal I}$ is 
    nonempty). If $I_1$ is maximal, we're done. Otherwise, ${\cal I} \setminus 
    \{I_1\}$ must be nonempty; we can find $I_2$ from this collection 
    such that $I_1 \subseteq I_2$ (else $I_1$ was maximal). If 
    $I_2$ is not maximal, pick $I_3 \in {\cal I} \setminus \{I_1, I_2\}$ 
    such that $I_1 \subseteq I_2 \subseteq I_3$. But by assumption, this 
    process terminates; for some $N \in \N$, we have $I_k = I_N$ 
    for all $k \geq N$, and $I_N$ is our desired maximal element in ${\cal I}$.

    $(3) \Rightarrow (1)$: Let $I$ be an ideal of $R$. Let ${\cal I}$ 
    denote the collection of all finitely generated ideals of $R$ 
    contained in $I$, which is nonempty because $\langle 0 \rangle 
    \in {\cal I}$. By assumption, ${\cal I}$ has a maximal element $J$. 
    If $J \neq I$, then we can find some $a \in I \setminus J$. Then 
    $\langle J, a \rangle$ is also finitely generated and contained in 
    ${\cal I}$, contradicting maximality. It follows that $J = I$ 
    and so $I$ is finitely generated. \qed 
\end{pf}\vspace{-0.25cm}

Note that the rings that we work with in this course are not generally UFDs, 
so we do not have the classical prime factorization from first year number 
theory. However, the following proposition gives us the idea to 
consider the factorization of proper ideals into prime ideals. 
The reason why we are only considering proper ideals here is because 
taking $I = R$ fails condition (1); every prime ideal of $R$ is proper by 
definition.

\begin{prop}{prop:3.2}
    Let $R$ be Noetherian and let $I \neq R$ be an ideal. 
    There exist proper ideals $P_1, \dots, P_n$ of $R$ such that 
    \begin{enumerate}[(1)]
        \item $I \subseteq P_i$ for all $i = 1, \dots, n$; and 
        \item $P_1P_2 \cdots P_n \subseteq I$. 
    \end{enumerate}
\end{prop}

Note that the prime ideals $P_i$ above are not necessarily distinct. 
To prove this proposition, we will use an extremely common tactic in 
commutative algebra: we assume that the set of objects that does 
not satisfy the property is nonempty, take a maximal element, and derive 
a contradiction. 

\begin{pf}[Proposition~\ref{prop:3.2}]
    Let $X$ be the set of proper ideals of $R$ not having this property. 
    By contradiction, assume that $X \neq \varnothing$. Let $I \in X$ be 
    a maximal element (with respect to $\subseteq$). Then $I$ itself 
    is not prime (otherwise we can simply take $P_1 = I$), so we can find 
    $a, b \in R$ such that $ab \in I$ but $a, b \notin I$. By 
    the maximality of $I$, we have $I + \langle a \rangle, I + \langle b 
    \rangle \notin X$. 

    Note that $(I + \langle a \rangle)(I + \langle b \rangle) 
    \subseteq I$ since $I^2$, $\langle a \rangle I$, $\langle b \rangle I$, 
    and $\langle ab \rangle$ are all subsets of $I$. In particular, we have 
    $I + \langle a \rangle \neq R$ and $I + \langle b \rangle \neq R$ since 
    multiplying by $R$ does not change an ideal. 

    Therefore, we can find prime ideals $P_1, \dots, P_n, Q_1, \dots, Q_m$ 
    of $R$ such that 
    \begin{enumerate}[(1)]
        \item $I + \langle a \rangle \subseteq P_i$ for all $i = 1, \dots, n$ 
        and $I + \langle b \rangle \subseteq Q_j$ for all $j = 1, \dots, m$;
        \item $P_1 P_2 \cdots P_n \subseteq I + \langle a \rangle$ 
        and $Q_1 Q_2 \cdots Q_m \subseteq I + \langle b \rangle$. 
    \end{enumerate}
    But then we have $I \subseteq I + \langle a \rangle \subseteq P_i$ 
    and $I \subseteq I + \langle b \rangle \subseteq Q_j$. Moreover, 
    we see that 
    \[ P_1 \cdots P_n Q_1 \cdots Q_m \subseteq (I + \langle a \rangle) 
    (I + \langle b \rangle) \subseteq I. \] 
    Here, we find that $I \notin X$, which is a contradiction. \qed 
\end{pf}\vspace{-0.25cm}

We now introduce some familiar ring theory from PMATH 347, namely the 
notion of coprime ideals and the (generalized) Chinese remainder theorem. 

\begin{defn}{defn:3.3}
    Let $R$ be a ring, and let $I$ and $J$ be proper ideals of $R$. We say 
    that $I$ and $J$ are {\bf coprime} if $I + J = R$.
\end{defn}\vspace{-0.25cm}

The term coprime is used interchangeably with comaximal, but being a 
number theory course, it feels more appropriate to say coprime.
The following proposition tells us that powers of coprime ideals are also 
coprime. 

\begin{prop}{prop:3.4}
    Let $R$ be a ring. Let $I$ and $J$ be proper ideals such that $I + J = R$. 
    Then for all $n, m \in \N$, we have $I^n + J^m = R$. 
\end{prop}\vspace{-0.25cm}
\begin{pf}[Proposition~\ref{prop:3.4}]
    Suppose that $I^n + J^m \neq R$. Then we have $I^n + J^m \subseteq M$ 
    for some maximal ideal $M$. This gives us $I^n \subseteq M$ and 
    $J^m \subseteq M$ as well. But maximal ideals are prime. 
    In particular, if $a \in I$, then $a^n \in M$, but since $M$ is 
    prime, we also have $a \in M$. Thus, we have $I \subseteq M$, 
    and by an identical argument, we obtain $J \subseteq M$. (This result 
    also follows from another characterization of prime ideals.)
    It follows that $I + J \subseteq M$, which is a contradiction since 
    maximal ideals are proper by definition. \qed 
\end{pf}\vspace{-0.25cm}

This leads us to the following famous theorem that we all know and love. 

\begin{theo}[Chinese Remainder Theorem]{theo:3.5}
    Let $R$ be a ring and let $I$ and $J$ be coprime ideals of $R$. Then 
    \[ R/IJ \cong R/I \times R/J. \]
\end{theo}\vspace{-0.25cm}
\newpage
\begin{pf}[Theorem~\ref{theo:3.5}]
    Define the map $\varphi : R \to R/I \times R/J$ by $\varphi(x) 
    = (x+I, x+J)$. Then we have $\ker\varphi = I \cap J$. But $IJ 
    \subseteq I+J$ always holds and since $I$ and $J$ are coprime, we obtain 
    \begin{align*}
        I \cap J &= (I \cap J)R \\ 
        &= (I \cap J)(I + J) \\ 
        &= (I \cap J)I + (I \cap J)J \subseteq IJ
    \end{align*}
    since $I \cap J \subseteq I$, $I \cap J \subseteq J$, and $R$ is 
    commutative. Thus, we have $\ker\varphi = I \cap J = IJ$. To see that 
    $\varphi$ is surjective, let $a \in I$ and $b \in J$ such that $a + b = 1$
    (which exist because $I$ and $J$ are coprime). Then for $x, y \in R$, we have 
    \begin{align*}
        \varphi(ax + by) &= (ax + by + I, ax + by + J) \\ 
        &= (by + I, ax + J) \\ 
        &= (b + I, a + J)(y + I, x + J) \\ 
        &= (1 + I, 1 + J)(y + I, x + J) \\ 
        &= (y + I, x + J),
    \end{align*}
    where the second last equality follows from looking at $b$ modulo $I$ 
    and $a$ modulo $J$. So $\varphi$ is surjective, and it follows from 
    the first isomorphism theorem that $R/IJ \cong R/I \times R/J$. \qed
\end{pf}\vspace{-0.25cm}

The generalized Chinese remainder theorem then follows from a straightforward 
induction.

\begin{theo}[Generalized Chinese Remainder Theorem]{theo:3.6}
    Let $R$ be a ring, and let $I_1, \dots, I_n$ be pairwise coprime ideals 
    (i.e. $I_i + I_j = R$ when $i \neq j$). Then 
    \[ R/I_1 \cdots I_n \cong R/I_1 \times \cdots \times R/I_n. \] 
\end{theo}\vspace{-0.25cm}

We end this section on ring theory with a property of finite rings. 

\begin{prop}{prop:3.7}
    Let $R$ be a finite ring. There exist distinct prime ideals $P_1, \dots, P_m$ 
    of $R$ and $n_i \in \N$ such that 
    \[ R \cong R/P_1^{n_1} \times \cdots \times R/P_m^{n_m}. \] 
\end{prop}\vspace{-0.25cm}

Note that if $R$ is an integral domain, then this proposition tells us 
almost nothing as we can simply take $P_1 = \{0\}$. This is much 
more interesting when we are not working with integral domains.

\begin{pf}[Proposition~\ref{prop:3.7}]
    We can find prime ideals $Q_1, \dots Q_k$ of $R$ such that 
    $Q_1Q_2 \cdots Q_k = \{0\}$ by using Proposition~\ref{prop:3.2}
    with $I = \{0\}$ and noting that finite rings are Noetherian. 
    Grouping the $Q_i$'s with multiplicity, we can write 
    \[ P_1^{n_1} \cdots P_m^{n_m} = \{0\} \] 
    where $P_i \neq P_j$ for $i \neq j$. Note that each $P_i$ is prime, 
    so $R/P_i$ is a finite integral domain and hence a field. This 
    means that $P_i$ is also maximal. So we have $P_i + P_j = R$ 
    for $i \neq j$ because these are each independently maximal and distinct,
    and hence any bigger ideal must be the whole ring. By Proposition~\ref{prop:3.4},
    we obtain $P_i^{n_i} + P_j^{n_j} = R$, and the generalized Chinese 
    remainder theorem (Theorem~\ref{theo:3.6}) gives 
    \[ R \cong R/P_1^{n_1} \cdots P_m^{n_m} \cong R/P_1^{n_1} \times 
    \cdots \times R/P_n^{n_m}. \eqno\qed \]
\end{pf}

\subsection{Prime Ideals of the Ring of Integers} \label{subsec:3.2}
Let $K$ be a number field of degree $[K : \Q] = n$, and let $R = {\cal O}_K$.
Let $I$ be a nonzero proper ideal of $R$. We know the following facts: 
\begin{enumerate}[(1)]
    \item Corollary~\ref{cor:1.19}: $R/I$ is finite, which allows us to 
    apply Proposition~\ref{prop:3.7}.
    \item Corollary~\ref{cor:1.21}: $R$ is Noetherian, so we can apply 
    Proposition~\ref{prop:3.2}.
    \item {\bf Correspondence theorem for rings.} There is a one-to-one
    correspondence between the ideals of $R$ that contain $I$ and the 
    ideals of the quotient ring $R/I$. In other words, every ideal $\overline{J}$ 
    of $R/I$ is of the form $\overline{J} = J/I$, where $J \subseteq R$ 
    is an ideal such that $I \subseteq J$. Moreover, $\overline{J}$ 
    is prime if and only if $J$ is prime. 
    \item Since $R/I$ is finite, we have by (3), Proposition~\ref{prop:3.7},
    and the third isomorphism theorem that 
    \begin{align*}
        R/I &\cong (R/I)/(P_1^{n_1}/I) \times \cdots \times (R/I)/(P_m^{n_m}/I) \\ 
        &\cong R/P_1^{n_1} \times \cdots \times R/P_m^{n_m} 
    \end{align*}
    where each $P_i \subseteq R$ is prime and $I \subseteq P_i$. 
\end{enumerate}
Therefore, to understand the ideal $I$, we study the prime ideals $P \supseteq I$. 
It turns out that $P \supseteq I$ if and only if $P$ is a ``prime factor'' of $I$,
as we'll see later in the course. 

For now, we'll introduce a couple of big ideas. Note that by the correspondence 
theorem, the prime ideals of $R/I$ are precisely $P/I$ where $P \supseteq I$ 
is a prime ideal. Moreover, for a prime ideal $P \supseteq I$, we know that 
$R/P$ is a finite field, so its cardinality is $|R/P| = p^m$
for some prime number $p \in \N$. This tells us that 
\[ p^m + P = p^m(1 + P) = 0 + P \] 
where the last equality is by Lagrange. Then $p^m \in P$. Since $P$ 
is a prime ideal, we get $p \in P$ and so $\langle p \rangle \subseteq P$. 

Next, we go through a computational example. Let $K = \Q(\sqrt{2})$ 
and let $R = {\cal O}_K = \Z[\sqrt{2}]$. Let's try to find all the 
prime ideals $P$ such that $\langle 5 \rangle \subseteq P$ 
(or equivalently, $5 \in P$). We have the isomorphisms 
\[ R/\langle 5 \rangle = \Z[\sqrt{2}]/\langle 5 \rangle 
\cong \Z[x]/\langle x^2 -2, 5 \rangle \cong \Z_5[x] / \langle x^2 -2 \rangle. \] 
Note that $x^2 - 2$ is irreducible over $\Z_5$ because it has no roots. 
So $\langle x^2 - 2 \rangle$ is a maximal ideal of $\Z_5[x]$ and hence 
$R/\langle 5 \rangle \cong \Z_5[x]/\langle x^2 - 2 \rangle$ is a field. 
This tells us that $\langle 5 \rangle \subseteq R$ is maximal, so 
the only prime ideal $P$ of $R$ sitting above $\langle 5 \rangle$ is 
$P = \langle 5 \rangle$ itself. 

That was rather simple, so let's keep going. Take $K = \Q(\sqrt{2})$ 
and $R = {\cal O}_K = \Z[\sqrt{2}]$ as before. Now, let's try to 
find the prime ideals such that $\langle 7 \rangle \subseteq P$. We have 
the isomorphisms 
\[ R/\langle 7 \rangle \cong \Z[x]/\langle x^2 - 2, 7 \rangle 
\cong \Z_7[x]/\langle x^2 - 2 \rangle, \] 
but this time, we have $x^2 - 2 = (x+3)(x+4)$ over $\Z_7$. Then 
the Chinese remainder theorem implies that 
\[ \Z_7[x]/\langle x^2 - 2 \rangle \cong \Z_7[x]/\langle x+3\rangle 
\times \Z_7[x]/\langle x+4 \rangle \cong \Z_7 \times \Z_7, \] 
where the last isomorphism is obtained by sending $x$ to $-3$ and $-4$ 
respectively (or we can formally justify this using the first isomorphism
theorem if we'd like). 

The prime ideals of $\Z_7 \times \Z_7$ are 
$P_1 = \langle (1, 0) \rangle$ and $P_2 = \langle (0, 1) \rangle$. 
Our goal now is to retrace our isomorphisms backwards. We have 
\begin{align*}
    (1, 0) \in \Z_7 \times \Z_7 
    &\mapsto (1 + \langle x+3 \rangle, 0 + \langle x+4 \rangle) \in \Z_7[x]/\langle x+3\rangle 
    \times \Z_7[x]/\langle x+4 \rangle \\ 
    &\mapsto x+4 + \langle x^2 + 2 \rangle \in \Z_7[x]/\langle x^2 - 2 \rangle \\ 
    &\mapsto x+4 + \langle x^2 - 2, 7 \rangle \in \Z[x]/\langle x^2 - 2, 7 \rangle \\ 
    &\mapsto \sqrt{2} + 4 + \langle 7 \rangle \in R/\langle 7 \rangle. 
\end{align*}
The trickiest one to reverse here is the second one: we needed to reverse 
the map from the Chinese remainder theorem. We needed to find a polynomial 
that was congruent to $1$ mod $x+3$ and congruent to $0$ mod $x+4$, 
and $x+4$ happened to do the trick. Similarly, we have 
\begin{align*}
    (0, 1) \in \Z_7 \times \Z_7 
    &\mapsto (0 + \langle x+3 \rangle, 1 + \langle x+4 \rangle) \in \Z_7[x]/\langle x+3\rangle 
    \times \Z_7[x]/\langle x+4 \rangle \\ 
    &\mapsto -x-3 + \langle x^2 + 2 \rangle \in \Z_7[x]/\langle x^2 - 2 \rangle \\ 
    &\mapsto -x-3 + \langle x^2 - 2, 7 \rangle \in \Z[x]/\langle x^2 - 2, 7 \rangle \\ 
    &\mapsto -\sqrt{2} - 3 + \langle 7 \rangle \in R/\langle 7 \rangle, 
\end{align*}
where $-x-3$ is congruent to $0$ mod $x+3$ and congruent to $1$ mod $x+4$. 
Thus, the prime ideals in $R$ containing $7$ are $Q_1 = \langle \sqrt{2} + 4, 7
\rangle$ and $Q_2 = \langle -\sqrt{2} - 3, 7 \rangle = \langle 2+\sqrt{3}, 7 \rangle$, 
keeping in mind that we are looking for the prime ideals of $R$ and not those 
of $R/\langle 7 \rangle$. Note that we have $(\sqrt{2} + 3)(\sqrt{2} - 3) = -7$
so we in fact have $Q_2 = \langle 2 + \sqrt{3} \rangle$, but it doesn't 
hurt to include the $7$ to ensure that it is actually living above 
$\langle 7 \rangle$.

The main takeaways of this computation were as follows: 
\begin{enumerate}[(1)]
    \item The minimal polynomial $x^2 - 2$ factored as $(x+3)(x+4)$ over $\Z_7$. 
    \item We have $(\sqrt{2} + 3)(\sqrt{2} + 4) = 14 + 7\sqrt{2}$; the coefficients 
    are both divisible by $7$. 
    \item By (2), we have $Q_1Q_2 \subseteq \langle 7 \rangle$. 
    It can be checked that $Q_1Q_2 = \langle 7 \rangle$ is the prime 
    factorization of $\langle 7 \rangle$.
\end{enumerate}
