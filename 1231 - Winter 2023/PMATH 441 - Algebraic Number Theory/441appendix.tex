\appendix 
\section{Assignment Problems}
Sometimes, we'll use facts that we cover on the assignments, 
so we list the problems here for reference. 

{\bf Assignment 1.} 
\begin{enumerate}[leftmargin=1.5cm, label={{\bf A1-\arabic*}}]
    \item \label{A1-1} Let $K$ be a number field. \begin{enumerate}[(a)]
        \item Let $\alpha \in K$. Suppose that $\alpha$ is a root of 
        \[ f(x) = a_n x^n + a_{n-1} x^{n-1} + \cdots + a_1 x + a_0 \in \Z[x]. \] 
        Prove that $a_n \alpha \in {\cal O}_K$. 
        \item Prove that there exists a basis for 
        $K$ over $\Q$ consisting entirely of algebraic integers. 
    \end{enumerate} 
    \item \label{A1-2} Let $K = \Q(\sqrt{3}, \sqrt{7})$ and let 
    \[ \Z[\sqrt{3}, \sqrt{7}] = \{a + b\sqrt{3} + c\sqrt{7} + d\sqrt{21} : 
    a, b, c, d \in \Z\}. \] 
    Prove that ${\cal O}_K \neq \Z[\sqrt{3}, \sqrt{7}]$. 
    \item \label{A1-3} Let $R$ and $S$ be integral domains, where $R$ is a subring of $S$. 
    Suppose $S$ is integral over $R$. That is, assume that every element of 
    $S$ is integral over $R$. 
    \begin{enumerate}[(a)]
        \item Prove that $R$ is a field if and only if $S$ is a field. 
        \item Let $Q$ be a prime ideal of $S$ and let $P = Q \cap R$. 
        Prove that $P$ is a prime ideal of $R$, and that $P$ is maximal 
        if and only if $Q$ is maximal.
    \end{enumerate}
    \item \label{A1-4} Let $L/K$ be a finite extension of number fields. 
    Prove that every embedding (injective ring homomorphism) $\varphi : 
    K \to \C$ can be extended to exactly $[L : K]$ embeddings $\psi : L \to \C$.
\end{enumerate}

{\bf Assignment 2.} 
\begin{enumerate}[leftmargin=1.5cm, label={{\bf A2-\arabic*}}]
    \item \label{A2-1} Let $K = \Q(\sqrt{-5})$. \begin{enumerate}[(a)]
        \item Suppose that $\{a, b\} \subseteq {\cal O}_K$ is an integral basis 
        for ${\cal O}_K$. Prove that we must have 
        \[ \det \begin{bmatrix} a & \bar{a} \\ b & \bar{b} \end{bmatrix}^2 = -20. \]
        Here, we mean $\overline{x + y\sqrt{-5}} = x - y\sqrt{-5}$ for $x, y \in \Q$. 
        \item Suppose that $a, b \in {\cal O}_K$ satisfy 
        \[ \det \begin{bmatrix} a & \bar{a} \\ b & \bar{b} \end{bmatrix}^2 = -20. \] 
        Prove that $\{a, b\}$ is an integral basis for ${\cal O}_K$.  
    \end{enumerate} 
    \item \label{A2-2} Let $\alpha \in \C$ such that $\alpha^4 + 3\alpha^2 + 
    6\alpha - 3 = 0$. \begin{enumerate}[(a)]
        \item Compute $\Tr_{\Q(\alpha)/\Q}(\alpha)$. 
        \item Compute $\Tr_{\Q(\alpha)/\Q}(\alpha^4 + \alpha + 2)$.
    \end{enumerate}
    \item \label{A2-3} Let $R$ be an integral domain and let $I$ and $J$ be 
    ideals of $R$. Recall that 
    \[ IJ := \{a_1b_1 + \cdots + a_nb_n : n \in \N,\, a_i \in I,\, b_i \in J\} \] 
    is also an ideal of $R$. Now, let $X$ be the set of nonzero ideals of $R$. 
    \begin{enumerate}[(a)]
        \item Put a relation $\sim$ on $X$ by $I \sim J$ if and only if 
        $\langle \alpha \rangle I = \langle \beta \rangle J$ for some 
        nonzero $\alpha, \beta \in R$. Prove that $\sim$ is an equivalence 
        relation on $X$. 
        \item Prove that $I, J \in X$ are isomorphic as $R$-modules if and only if 
        $I \sim J$. 
        \item Prove that if $I \in X$ and there exists a nonzero $\alpha \in R$ 
        such that $\langle \alpha \rangle I$ is principal, then $I$ 
        itself is principal. What does this tell you about the principal 
        ideals of $R$ relative to $\sim$?
        \item Prove that the set of ideal classes (with respect to $\sim$) form a 
        group under the operation $[I][J] = [IJ]$ if and only if for all 
        $I \in X$, there exists $J \in X$ such that $IJ$ is principal.
    \end{enumerate} 
    \item \label{A2-4} Let $M = \Z^n$ and let $N$ be a submodule of $M$ such that $\rank(N) = n$. 
    \begin{enumerate}[(a)]
        \item Prove that $M/N$ is finite. 
        \item Read Theorem 2.10 of Keith Conrad's notes, which also 
        requires Definition 2.8. If $\{e_1, \dots, e_n\}$ is 
        the standard integral basis for $M$, this guarantees the existence of a 
        basis $\{d_1e_1, \dots, d_ne_n\}$ for $N$, where $d_i \in \Z$. 
        \item Prove that $M/N \cong \Z_{d_1} \times \cdots \times \Z_{d_n}$. 
        \item Let $\{v_1, \dots, v_n\}$ be an integral basis for $N$. 
        Since $N \subseteq M$, we can find $a_{ij} \in \Z$ such that 
        $v_i = \sum_{j=1}^n a_{ij} e_j$. Use this to construct the matrix 
        $A = [a_{ij}] \in M_n(\Z)$. Prove that $[M : N] = \lvert\det(A)\rvert$, where 
        $[M : N]$ is the index of the subgroup $N$ in $M$.
    \end{enumerate}
\end{enumerate}

{\bf Assignment 3.} 
\begin{enumerate}[leftmargin=1.5cm, label={{\bf A3-\arabic*}}]
    \item \label{A3-1} Let $\alpha \in \C$ be a root of $f(x) = x^4 - x^3 
    + 2x^2 - x + 2$, and let $K = \Q(\alpha)$. Prove that ${\cal O}_K 
    = \Z[\alpha]$. Note that $f(x)$ is irreducible by the rational roots 
    theorem.
    \item \label{A3-2} Let $\alpha \in \C$ be a root of $f(x) = x^3 + 18x - 26$, 
    and let $K = \Q(\alpha)$. Note that $f(x)$ is irreducible by 
    Eisenstein with $p = 2$. 
    \begin{enumerate}[(a)]
        \item Prove that $\beta = \frac{\alpha^2-\alpha+1}{3} \in {\cal O}_K$. 
        \item Compute $\disc(1, \alpha, \beta)$. 
        \item Prove that $[{\cal O}_K : \Z[\alpha]] \in \{3, 6\}$. 
    \end{enumerate}
    \item \label{A3-3} Let $\alpha \in \C$ be a root of $f(x) = x^3 - 
    x^2 - 2x - 8$, and let $K = \Q(\alpha)$. Note that $f(x)$ is irreducible 
    by the rational roots theorem. 
    \begin{enumerate}[(a)]
        \item Prove that $\beta = \frac{4}{\alpha} \in {\cal O}_K$. 
        \item Compute $\Tr_{K/\Q}(\alpha)$, $\Tr_{K/\Q}(\beta)$, 
        $\Tr_{K/\Q}(\alpha^2)$, $\Tr_{K/\Q}(\beta^2)$, and $\Tr_{K/\Q}(\alpha\beta)$. 
        \item Compute $\disc(1, \alpha, \beta)$. 
        \item Prove that $\{1, \alpha, \beta\}$ is an integral basis for ${\cal O}_K$.        
    \end{enumerate}
    \item \label{A3-4} Let $\alpha \in \C$ be a root of $f(x) = x^3 - 
    x^2 - 2x - 8$ as before, and let $K = \Q(\alpha)$. Prove that there 
    does not exist $\theta \in {\cal O}_K$ such that $\{1, \theta, \theta^2\}$ 
    is an integral basis for ${\cal O}_K$. 
    
    {\bf Hint:} By \ref{A3-3}, we know that $\{1, \alpha, \beta\}$ is an 
    integral basis for ${\cal O}_K$, so we have $\theta = a + b\alpha + c\beta$
    and $\theta^2 = A + B\alpha + C\beta$ for some $a, b, c, A, B, C \in \Z$. 
    Try to write $A, B, C$ in terms of $a, b, c$.
\end{enumerate}

{\bf Assignment 4.} 
\begin{enumerate}[leftmargin=1.5cm, label={{\bf A4-\arabic*}}]
    \item \label{A4-1} \begin{enumerate}[(a)]
        \item Prove that every UFD is integrally closed.
        \item Prove that every PID is a Dedekind domain.
        \item Give an example of a UFD that is not a Dedekind domain.
    \end{enumerate}
    \item \label{A4-2} Let $K = \Q(\alpha)$ where $\alpha$ is a root of 
    $f(x) = x^3 - x^2 - 3$. Note that $\disc(f(x)) = -255$. Factor 
    $\langle p \rangle$ as a product of prime ideals in ${\cal O}_K$ 
    for all $p \in \{2, 3, 5, 7\}$. Which of these primes divide $255$? 
    How does this affect the corresponding prime factorization?
    \item \label{A4-3} Suppose that $[K : \Q] = n$ and let $P$ be a nonzero prime 
    ideal of ${\cal O}_K$. Prove that $P$ contains a prime number $p \in \N$ 
    and that ${\cal O}_K/P$ contains at most $p^n$ elements. 
    \item \label{A4-4} Let $\alpha$ be a root of $f(x) = x^3 - 2x - 3$. 
    \begin{enumerate}[(a)]
        \item Prove that $R := \Z[\alpha]$ is a Dedekind domain.
        \item Let $I = \langle \alpha - 5 \rangle \subseteq R$. Prove that $|R/I| = 112$. 
        \item Factor $\langle 2 \rangle$ as a product of prime ideals. 
        \item Factor $\langle 7 \rangle$ as a product of prime ideals. 
        \item Factor $I$ as a product of prime ideals. 
    \end{enumerate}
\end{enumerate}

{\bf Assignment 5.} This assignment was a guided reading exercise on the geometry 
of numbers, following Marcus' \emph{Number Fields} pages 93 to 99.
\begin{enumerate}[leftmargin=1.5cm, label={{\bf A5-\arabic*}}]
    \item \label{A5-1} A {\bf lattice} in $\R^n$ is defined to be a subset 
    $L$ of $\R^n$ of the form $L = \Span_{\Z}\{v_1, \dots, v_n\}$, where 
    $\{v_1, \dots, v_n\}$ is a basis for $\R^n$ over $\R$. We define the 
    {\bf volume} of $L$ by 
    \[ \Vol(L) = \lvert\det\,[v_1|\cdots|v_n]\rvert. \] 
    \begin{enumerate}[(a)]
        \item Prove that the definition of the volume of a lattice is 
        independent of the choice of basis. 
        \item Let $M \subseteq L$ be lattices in $\R^n$. Prove that 
        \[ \Vol(M) = [L : M] \cdot \Vol(L). \] 
    \end{enumerate}
    \item \label{A5-2} Let $K$ be a number field and let $R = {\cal O}_K$. 
    Suppose that $\sigma_1, \dots, \sigma_r$ are the real-valued embeddings 
    of $K$ in $\C$ and that $\tau_1, \dots, \tau_{2s}$ are the non-real-valued 
    (i.e. complex) embeddings of $K$ in $\C$. Assume that $\tau_{i+1} = 
    \overline{\tau_i}$ for all odd $i$. Consider the {\bf Minkowski map} $\psi : 
    K \to \R^n$ given by 
    \[ \psi(\alpha) = (\sigma_1(\alpha), \dots, \sigma_r(\alpha), 
    \Re(\tau_1(\alpha)), \Im(\tau_1(\alpha)), \Re(\tau_3(\alpha)), 
    \Im(\tau_3(\alpha)), \dots). \]
    \begin{enumerate}[(a)]
        \item Prove that $\psi(R)$ is a lattice in $\R^n$. We will denote 
        this lattice by $M_K$ and we call it the {\bf Minkowski lattice} of $K$.
        \item Prove that 
        \[ \Vol(M_K) = \frac{1}{2^s} \lvert\disc(K)\rvert^{1/2}. \] 
        \item Let $I$ be a nonzero ideal of $R$ so that $L_I := \psi(I)$ 
        is a sublattice of $M_K$. Prove that 
        \[ \Vol(L_I) = \frac{1}{2^s} \lvert\disc(K)\rvert^{1/2} N(I). \] 
    \end{enumerate}
    \item \label{A5-3} We now take a look at Minkowski's lemma. Let 
    $L$ be a lattice in $\R^n$ and let $E \subseteq \R^n$ be such that 
    \begin{enumerate}[(1)]
        \item $E$ is convex (for all $a, b \in E$ and $t \in [0, 1]$, we have 
        $(1-t)a + tb \in E$); 
        \item $E$ is Lebesgue measurable; 
        \item if $a \in E$, then $-a \in E$.
    \end{enumerate}
    Furthermore, assume that $E$ is large; more precisely, assume that 
    $m(E) > 2^n \Vol(L)$ where $m(E)$ denotes the Lebesgue measure of $E$. 
    It turns out that $E$ is so big that it must contain a nonzero vector from $L$.
    If $E$ is compact, then the strict inequality can be weakened to $\geq$.
    \begin{enumerate}[(a)]
        \item Read about what Lebesgue measure is from page 96 of Marcus. 
        \item Read the proof of Minkowski's lemma from pages 96 and 97 of Marcus.
    \end{enumerate}
    \item \label{A5-4} Let $K$, $R$, $r$, and $s$ be as in \ref{A5-2}. 
    For $x = (x_1, \dots, x_n) \in \R^n$, let 
    \[ N(x) = x_1 x_2 \cdots x_r (x_{r+1}^2 + x_{r+2}^2) \cdots (x_{n-1}^2 + x_n^2). \] 
    \begin{enumerate}[(a)]
        \item Prove that if $\alpha \in R$ and $x = \psi(\alpha)$, then 
        $N(x) = N_{K/\Q}(\alpha)$. 
        \item Let $A \subseteq \R^n$ be compact and satisfy conditions (1), 
        (2), and (3) from \ref{A5-3}, where $A$ has positive Lebesgue measure. 
        Additionally, assume that $|N(a)| \leq 1$ for all $a \in A$. Prove that 
        if $L$ is a lattice in $\R^n$, then there exists a nonzero vector $x \in L$ 
        with 
        \[ |N(x)| \leq \frac{2^n}{m(A)} \Vol(L). \] 
        {\bf Hint:} You can freely use the fact that if $A \subseteq \R^n$ is 
        Lebesgue measurable and $t \in \R$, then $tA$ is Lebesgue measurable 
        with $m(tA) = |t|^n m(A)$.
    \end{enumerate}
    \item \label{A5-5} Let $K$, $R$, $r$, and $s$ be as in \ref{A5-2}. 
    Consider the set 
    \[ A = \left\{ x \in \R^n : |x_1| + \cdots + |x_r| + 2 
    \left( \sqrt{ x_{r+1}^2 + x_{r+2}^2 } + \cdots + \sqrt{ x_{n-1}^2 + x_n^2 } 
    \right) \leq n \right\}. \] 
    It can be shown that $A$ satisfies the hypotheses from part (b) of \ref{A5-4}.
    \begin{enumerate}[(a)]
        \item From pages 98 and 99 of Marcus, read the integration-based argument that 
        \[ m(A) = \frac{n^n}{n!} 2^r \left( \frac{\pi}{2} \right)^s. \] 
        \item Prove that every lattice $L$ in $\R^n$ contains a nonzero vector $x \in L$ 
        such that 
        \[ |N(x)| \leq \frac{n!}{n^n} \left( \frac{8}{\pi} \right)^s \Vol(L). \] 
        \item Prove that every ideal class of $R$ (as per \ref{A2-3}) contains 
        an ideal $I$ such that 
        \[ N(I) \leq \frac{n!}{n^n} \left( \frac{4}{\pi} \right)^s \lvert\disc(K)\rvert^{1/2}. \] 
    \end{enumerate} 
\end{enumerate}

{\bf Assignment 6.} 
\begin{enumerate}[leftmargin=1.5cm, label={{\bf A6-\arabic*}}]
    \item \label{A6-1} Let $\alpha \in \C$ be an algebraic integer and let 
    $S = \Z[\alpha]$. Suppose that $f(x) \in \Z[x]$ is the minimal polynomial of 
    $\alpha$ and let $p \in \N$ be prime. Suppose the irreducible 
    factorization of $f(x) \in \Z_p[x]$ is 
    \[ f(x) = p_1(x)^{n_1} p_2(x)^{n_2} \cdots p_k(x)^{n_k} \in \Z_p[x]. \] 
    Prove that the prime ideals of $S$ containing $p$ are exactly 
    \[ P_i = \langle p_i(\alpha), p \rangle. \] 
    {\bf Hint:} Retrace isomorphisms. 
    \item \label{A6-2} Let $S = \Z[\sqrt{5}]$ and let $P = \langle \sqrt{5}+1, 2 \rangle$. 
    \begin{enumerate}[(a)]
        \item Prove that $P$ is the only prime ideal of $S$ containing $2$. 
        \item Prove that $S_P$ is not a DVR. 
        \item Prove that $\langle 2 \rangle \neq P^2$. Why does this not 
        contradict our prime factorization fact (Theorem~\ref{theo:3.8})?
    \end{enumerate}
    \item \label{A6-3} \begin{enumerate}[(a)]
        \item Let $\alpha \in \C$ be a root of the irreducible polynomial 
        $f(x) = x^3 - x^2 - 8x - 18$, whose discriminant is 
        $-9300 = -2^2 \cdot 3 \cdot 5^2 \cdot 31$. Let $K = \Q(\alpha)$, 
        let $R = {\cal O}_K$, and let $S = \Z[\alpha]$. Use the DVR 
        characterization (Theorem~\ref{theo:3.31}) to prove that $R = S$. 
        \item Let $\alpha \in \C$ be a root of the irreducible polynomial 
        $f(x) = x^3 - 6x - 16$, whose discriminant is 
        $-6048 = -2^5 \cdot 3^3 \cdot 7$. Let $K = \Q(\alpha)$, 
        let $R = {\cal O}_K$, and let $S = \Z[\alpha]$. Use the DVR 
        characterization (Theorem~\ref{theo:3.31}) to prove that $R \neq S$. 
    \end{enumerate}
    \item \label{A6-4} Let $p \in \N$ be a prime number and let $\alpha \in \C$ 
    be the root of a $p$-Eisenstein polynomial (i.e. a monic, integer polynomial 
    which is irreducible by $p$-Eisenstein). Let $R = {\cal O}_{\Q(\alpha)}$ 
    and $S = \Z[\alpha]$. 
    \begin{enumerate}[(a)]
        \item Prove that there is exactly one prime ideal $P$ of $S$ that contains $p$. 
        \item Let $P$ be the prime ideal from part (a). Prove that $S_P$ is a DVR 
        and prove that $\alpha$ is a uniformizer for $S_P$. 
    \end{enumerate}
\end{enumerate}

{\bf Assignment 7.} 
\begin{enumerate}[leftmargin=1.5cm, label={{\bf A7-\arabic*}}]
    \item \label{A7-1} In this problem, we explore where the ideal class 
    equivalence relation comes from. Let $R$ be a Dedekind domain with 
    field of fractions $K = \Frac(R)$. A {\bf fractional ideal} of 
    $K$ is any set of the form $\alpha I$, where $\alpha \in K^\times$ and 
    $I$ is a nonzero ideal of $R$. We define the product of two fractional 
    ideals by $(\alpha I)(\beta J) = \alpha\beta IJ$. Let $G$ denote 
    the set of all fractional ideals of $K$. 
    \begin{enumerate}[(a)]
        \item List three results from lecture whose proof has used 
        fractional ideals. 
        \item Prove that the multiplication of fractional ideals is well-defined. 
        \item Prove that $G$ forms a group under the above multiplication.
        \item Prove that every proper fractional ideal can be uniquely written 
        in the form $P_1^{n_1} \cdots P_k^{n_k}$, where each $P_i$ is a nonzero 
        prime ideal of $R$ and $n_i \in \Z$. 
        \item Let $H$ denote the set of fractional ideals of the form 
        $\alpha R$ where $\alpha \in K^\times$. Prove that $H$ is a subgroup of $G$. 
        \item Let $R = {\cal O}_K$ where $K$ is a number field. Prove that 
        $G/H$ is isomorphic to the ideal class group of $K$.
    \end{enumerate}
    \item \label{A7-2} Let $K = \Q(\alpha)$ be a number field with $\alpha \in {\cal O}_K$. 
    Let $R = {\cal O}_K$, let $S = \Z[\alpha]$, let $f(x) \in \Z[x]$ be the minimal 
    polynomial of $\alpha$, and let $p \in \N$ be a prime such that 
    $p \nmid \disc(\alpha)$. Prove that $f(x) \in \Z_p[x]$ can be written 
    as a product of distinct irreducible polynomials in $\Z_p[x]$.
    \item \label{A7-3} Let $\alpha \in \C$ be a root of $f(x) = x^3 - 2x + 5$. 
    How many nonzero ideals of $\Z[\alpha]$ have norm $8$ or less?
    \item \label{A7-4} Let $\alpha \in \C$ be a root of the irreducible 
    polynomial $f(x) = x^4 - 2x^3 + 3x^2 - 1$, whose discriminant is $-976$. 
    Let $K = \Q(\alpha)$. 
    \begin{enumerate}[(a)]
        \item Prove that ${\cal O}_K = \Z[\alpha]$. 
        \item Prove that ${\cal O}_K$ is a PID. 
    \end{enumerate}
\end{enumerate}