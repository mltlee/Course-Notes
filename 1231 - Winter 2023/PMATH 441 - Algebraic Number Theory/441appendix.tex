\appendix 
\section{Assignment Problems}
Sometimes, we'll use facts that we cover on the assignments, 
so we list the problems here for reference. 

{\bf Assignment 1.} 
\begin{enumerate}[leftmargin=1.5cm, label={{\bf A1-\arabic*}}]
    \item \label{A1-1} Let $K$ be a number field. \begin{enumerate}[(a)]
        \item Let $\alpha \in K$. Suppose that $\alpha$ is a root of 
        \[ f(x) = a_n x^n + a_{n-1} x^{n-1} + \cdots + a_1 x + a_0 \in \Z[x]. \] 
        Prove that $a_n \alpha \in {\cal O}_K$. 
        \item Prove that there exists a basis for 
        $K$ over $\Q$ consisting entirely of algebraic integers. 
    \end{enumerate} 
    \item \label{A1-2} Let $K = \Q(\sqrt{3}, \sqrt{7})$ and let 
    \[ \Z[\sqrt{3}, \sqrt{7}] = \{a + b\sqrt{3} + c\sqrt{7} + d\sqrt{21} : 
    a, b, c, d \in \Z\}. \] 
    Prove that ${\cal O}_K \neq \Z[\sqrt{3}, \sqrt{7}]$. 
    \item \label{A1-3} Let $R$ and $S$ be integral domains, where $R$ is a subring of $S$. 
    Suppose $S$ is integral over $R$. That is, assume that every element of 
    $S$ is integral over $R$. 
    \begin{enumerate}[(a)]
        \item Prove that $R$ is a field if and only if $S$ is a field. 
        \item Let $Q$ be a prime ideal of $S$ and let $P = Q \cap R$. 
        Prove that $P$ is a prime ideal of $R$, and that $P$ is maximal 
        if and only if $Q$ is maximal.
    \end{enumerate}
    \item \label{A1-4} Let $L/K$ be a finite extension of number fields. 
    Prove that every embedding (injective ring homomorphism) $\varphi : 
    K \to \C$ can be extended to exactly $[L : K]$ embeddings $\psi : L \to \C$.
\end{enumerate}

{\bf Assignment 2.} 
\begin{enumerate}[leftmargin=1.5cm, label={{\bf A2-\arabic*}}]
    \item \label{A2-1} Let $K = \Q(\sqrt{-5})$. \begin{enumerate}[(a)]
        \item Suppose that $\{a, b\} \subseteq {\cal O}_K$ is an integral basis 
        for ${\cal O}_K$. Prove that we must have 
        \[ \det \begin{bmatrix} a & \bar{a} \\ b & \bar{b} \end{bmatrix}^2 = -20. \]
        Here, we mean $\overline{x + y\sqrt{-5}} = x - y\sqrt{-5}$ for $x, y \in \Q$. 
        \item Suppose that $a, b \in {\cal O}_K$ satisfy 
        \[ \det \begin{bmatrix} a & \bar{a} \\ b & \bar{b} \end{bmatrix}^2 = -20. \] 
        Prove that $\{a, b\}$ is an integral basis for ${\cal O}_K$.  
    \end{enumerate} 
    \item \label{A2-2} Let $\alpha \in \C$ such that $\alpha^4 + 3\alpha^2 + 
    6\alpha - 3 = 0$. \begin{enumerate}[(a)]
        \item Compute $\Tr_{\Q(\alpha)/\Q}(\alpha)$. 
        \item Compute $\Tr_{\Q(\alpha)/\Q}(\alpha^4 + \alpha + 2)$.
    \end{enumerate}
    \item \label{A2-3} Let $R$ be an integral domain and let $I$ and $J$ be 
    ideals of $R$. Recall that 
    \[ IJ := \{a_1b_1 + \cdots + a_nb_n : n \in \N,\, a_i \in I,\, b_i \in J\} \] 
    is also an ideal of $R$. Now, let $X$ be the set of nonzero ideals of $R$. 
    \begin{enumerate}[(a)]
        \item Put a relation $\sim$ on $X$ by $I \sim J$ if and only if 
        $\langle \alpha \rangle I = \langle \beta \rangle J$ for some 
        nonzero $\alpha, \beta \in R$. Prove that $\sim$ is an equivalence 
        relation on $X$. 
        \item Prove that $I, J \in X$ are isomorphic as $R$-modules if and only if 
        $I \sim J$. 
        \item Prove that if $I \in X$ and there exists a nonzero $\alpha \in R$ 
        such that $\langle \alpha \rangle I$ is principal, then $I$ 
        itself is principal. What does this tell you about the principal 
        ideals of $R$ relative to $\sim$?
        \item Prove that the set of ideal classes (with respect to $\sim$) form a 
        group under the operation $[I][J] = [IJ]$ if and only if for all 
        $I \in X$, there exists $J \in X$ such that $IJ$ is principal.
    \end{enumerate} 
    \item \label{A2-4} Let $M = \Z^n$ and let $N$ be a submodule of $M$ such that $\rank(N) = n$. 
    \begin{enumerate}[(a)]
        \item Prove that $M/N$ is finite. 
        \item Read Theorem 2.10 of Keith Conrad's notes, which also 
        requires Definition 2.8. If $\{e_1, \dots, e_n\}$ is 
        the standard integral basis for $M$, this guarantees the existence of a 
        basis $\{d_1e_1, \dots, d_ne_n\}$ for $N$, where $d_i \in \Z$. 
        \item Prove that $M/N \cong \Z_{d_1} \times \cdots \times \Z_{d_n}$. 
        \item Let $\{v_1, \dots, v_n\}$ be an integral basis for $N$. 
        Since $N \subseteq M$, we can find $a_{ij} \in \Z$ such that 
        $v_i = \sum_{j=1}^n a_{ij} e_j$. Use this to construct the matrix 
        $A = [a_{ij}] \in M_n(\Z)$. Prove that $[M : N] = \lvert\det(A)\rvert$, where 
        $[M : N]$ is the index of the subgroup $N$ in $M$.
    \end{enumerate}
\end{enumerate}

{\bf Assignment 3.} 
\begin{enumerate}[leftmargin=1.5cm, label={{\bf A3-\arabic*}}]
    \item \label{A3-1} Let $\alpha \in \C$ be a root of $f(x) = x^4 - x^3 
    + 2x^2 - x + 2$, and let $K = \Q(\alpha)$. Prove that ${\cal O}_K 
    = \Z[\alpha]$. Note that $f(x)$ is irreducible by the rational roots 
    theorem.
    \item \label{A3-2} Let $\alpha \in \C$ be a root of $f(x) = x^3 + 18x - 26$, 
    and let $K = \Q(\alpha)$. Note that $f(x)$ is irreducible by 
    Eisenstein with $p = 2$. 
    \begin{enumerate}[(a)]
        \item Prove that $\beta = \frac{\alpha^2-\alpha+1}{3} \in {\cal O}_K$. 
        \item Compute $\disc(1, \alpha, \beta)$. 
        \item Prove that $[{\cal O}_K : \Z[\alpha]] \in \{3, 6\}$. 
    \end{enumerate}
    \item \label{A3-3} Let $\alpha \in \C$ be a root of $f(x) = x^3 - 
    x^2 - 2x - 8$, and let $K = \Q(\alpha)$. Note that $f(x)$ is irreducible 
    by the rational roots theorem. 
    \begin{enumerate}[(a)]
        \item Prove that $\beta = \frac{4}{\alpha} \in {\cal O}_K$. 
        \item Compute $\Tr_{K/\Q}(\alpha)$, $\Tr_{K/\Q}(\beta)$, 
        $\Tr_{K/\Q}(\alpha^2)$, $\Tr_{K/\Q}(\beta^2)$, and $\Tr_{K/\Q}(\alpha\beta)$. 
        \item Compute $\disc(1, \alpha, \beta)$. 
        \item Prove that $\{1, \alpha, \beta\}$ is an integral basis for ${\cal O}_K$.        
    \end{enumerate}
    \item \label{A3-4} Let $\alpha \in \C$ be a root of $f(x) = x^3 - 
    x^2 - 2x - 8$ as before, and let $K = \Q(\alpha)$. Prove that there 
    does not exist $\theta \in {\cal O}_K$ such that $\{1, \theta, \theta^2\}$ 
    is an integral basis for ${\cal O}_K$. 
    
    {\bf Hint:} By \ref{A3-3}, we know that $\{1, \alpha, \beta\}$ is an 
    integral basis for ${\cal O}_K$, so we have $\theta = a + b\alpha + c\beta$
    and $\theta^2 = A + B\alpha + C\beta$ for some $a, b, c, A, B, C \in \Z$. 
    Try to write $A, B, C$ in terms of $a, b, c$.
\end{enumerate}