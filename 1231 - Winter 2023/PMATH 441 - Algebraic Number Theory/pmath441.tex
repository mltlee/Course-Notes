\documentclass[10pt]{article}
\usepackage[T1]{fontenc}
\usepackage{amsmath,amssymb,amsthm}
\usepackage{mathtools}
\usepackage[shortlabels]{enumitem}
\usepackage[english]{babel}
\usepackage[utf8]{inputenc}
\usepackage{fancyhdr}
\usepackage{bold-extra}
\usepackage{color}   
\usepackage{tocloft}
\usepackage{graphicx}
\usepackage{lipsum}
\usepackage{wrapfig}
\usepackage{cutwin}
\usepackage{hyperref}
\usepackage{lastpage}
\usepackage{multicol}
\usepackage{tikz}
\usepackage{xcolor}
\usepackage{microtype}
\usepackage{framed}
\usepackage[framemethod=TikZ]{mdframed}

% some useful math commands
\newcommand{\eps}{\varepsilon}
\newcommand{\R}{\mathbb{R}}
\newcommand{\C}{\mathbb{C}}
\newcommand{\N}{\mathbb{N}}
\newcommand{\Z}{\mathbb{Z}}
\newcommand{\Q}{\mathbb{Q}}
\newcommand{\K}{\mathbb{K}}
\newcommand{\F}{\mathbb{F}}
\newcommand{\T}{\mathbb{T}}

\numberwithin{equation}{section}

\newcommand{\dd}{\,\mathrm{d}}
\newcommand{\ddz}{\frac{\rm d}{{\rm d}z}}
\newcommand{\pv}{\text{p.v.}}

\renewcommand{\Re}{{\rm Re}}

\DeclareMathOperator{\GL}{GL}
\DeclareMathOperator{\id}{id}
\DeclareMathOperator{\Arg}{Arg}
\DeclareMathOperator{\Log}{Log}
\DeclareMathOperator{\PV}{PV}
\DeclareMathOperator{\sech}{sech}
\DeclareMathOperator{\csch}{csch}
\DeclareMathOperator{\Res}{Res}
\DeclareMathOperator{\Li}{Li}
\DeclareMathOperator{\QR}{QR}
\DeclareMathOperator{\NR}{NR}
\DeclareMathOperator{\lcm}{lcm}
\DeclareMathOperator{\divergence}{div}
\DeclareMathOperator*{\esssup}{ess\,sup}
\DeclareMathOperator{\Span}{span}
\DeclareMathOperator{\Pol}{Pol}
\DeclareMathOperator{\Content}{Content}

\DeclarePairedDelimiter\ceil{\lceil}{\rceil}
\DeclarePairedDelimiter\floor{\lfloor}{\rfloor}

\newcommand{\suchthat}{\;\ifnum\currentgrouptype=16 \;\middle|\;\else\mid\fi\;}

% title formatting
\newcommand{\newtitle}[4]{
  \begin{center}
	\huge{\textbf{\textsc{#1 Course Notes}}}
    
	\large{\sc #2}
    
	{\sc #3 \textbullet\, #4 \textbullet\, University of Waterloo}
	\normalsize\vspace{1cm}\hrule
  \end{center}
}

\newcounter{theo}[section]\setcounter{theo}{0}
\renewcommand{\thetheo}{\arabic{section}.\arabic{theo}}
\newenvironment{theo}[2][]{%
\refstepcounter{theo}%
\ifstrempty{#1}%
{\mdfsetup{%
frametitle={%
\tikz[baseline=(current bounding box.east),outer sep=0pt]
\node[anchor=east,rectangle,fill=blue!20]
{\strut {\sc Theorem~\thetheo}};}}
}%
{\mdfsetup{%
frametitle={%
\tikz[baseline=(current bounding box.east),outer sep=0pt]
\node[anchor=east,rectangle,fill=blue!20]
{\strut {\sc Theorem~\thetheo:~#1}};}}%
}%
\mdfsetup{innertopmargin=10pt,linecolor=blue!20,%
linewidth=2pt,topline=true,%
frametitleaboveskip=\dimexpr-\ht\strutbox\relax
}
\begin{mdframed}[nobreak=false]\relax%
\label{#2}}{\end{mdframed}}

%%%%%%%%%%%%%%%%%%%%%%%%%%%%%%
%Definition
\newenvironment{defn}[2][]{%
\refstepcounter{theo}%
\ifstrempty{#1}%
{\mdfsetup{%
frametitle={%
\tikz[baseline=(current bounding box.east),outer sep=0pt]
\node[anchor=east,rectangle,fill=yellow!20]
{\strut {\sc Definition~\thetheo}};}}
}%
{\mdfsetup{%
frametitle={%
\tikz[baseline=(current bounding box.east),outer sep=0pt]
\node[anchor=east,rectangle,fill=yellow!20]
{\strut {\sc Definition~\thetheo:~#1}};}}%
}%
\mdfsetup{innertopmargin=10pt,linecolor=yellow!20,%
linewidth=2pt,topline=true,%
frametitleaboveskip=\dimexpr-\ht\strutbox\relax
}
\begin{mdframed}[nobreak=true]\relax%
\label{#2}}{\end{mdframed}}

%%%%%%%%%%%%%%%%%%%%%%%%%%%%%%
%Example
\newenvironment{exmp}[2][]{%
\refstepcounter{theo}%
\ifstrempty{#1}%
{\mdfsetup{%
frametitle={%
\tikz[baseline=(current bounding box.east),outer sep=0pt]
\node[anchor=east,rectangle,fill=cyan!20]
{\strut {\sc Example~\thetheo}};}}
}%
{\mdfsetup{%
frametitle={%
\tikz[baseline=(current bounding box.east),outer sep=0pt]
\node[anchor=east,rectangle,fill=cyan!20]
{\strut {\sc Example~\thetheo:~#1}};}}%
}%
\mdfsetup{innertopmargin=10pt,linecolor=cyan!20,%
linewidth=2pt,topline=true,%
frametitleaboveskip=\dimexpr-\ht\strutbox\relax
}
\begin{mdframed}[nobreak=false]\relax%
\label{#2}}{\end{mdframed}}

%%%%%%%%%%%%%%%%%%%%%%%%%%%%%%
%Corollary
\newenvironment{cor}[2][]{%
\refstepcounter{theo}%
\ifstrempty{#1}%
{\mdfsetup{%
frametitle={%
\tikz[baseline=(current bounding box.east),outer sep=0pt]
\node[anchor=east,rectangle,fill=lime!20]
{\strut {\sc Corollary~\thetheo}};}}
}%
{\mdfsetup{%
frametitle={%
\tikz[baseline=(current bounding box.east),outer sep=0pt]
\node[anchor=east,rectangle,fill=lime!20]
{\strut {\sc Corollary~\thetheo:~#1}};}}%
}%
\mdfsetup{innertopmargin=10pt,linecolor=lime!20,%
linewidth=2pt,topline=true,%
frametitleaboveskip=\dimexpr-\ht\strutbox\relax
}
\begin{mdframed}[nobreak=true]\relax%
\label{#2}}{\end{mdframed}}

%%%%%%%%%%%%%%%%%%%%%%%%%%%%%%
%Remark
\newenvironment{remark}[2][]{%
\refstepcounter{theo}%
\ifstrempty{#1}%
{\mdfsetup{%
frametitle={%
\tikz[baseline=(current bounding box.east),outer sep=0pt]
\node[anchor=east,rectangle,fill=orange!20]
{\strut {\sc Remark~\thetheo}};}}
}%
{\mdfsetup{%
frametitle={%
\tikz[baseline=(current bounding box.east),outer sep=0pt]
\node[anchor=east,rectangle,fill=orange!20]
{\strut {\sc Remark~\thetheo:~#1}};}}%
}%
\mdfsetup{innertopmargin=10pt,linecolor=orange!20,%
linewidth=2pt,topline=true,%
frametitleaboveskip=\dimexpr-\ht\strutbox\relax
}
\begin{mdframed}[nobreak=true]\relax%
\label{#2}}{\end{mdframed}}

%%%%%%%%%%%%%%%%%%%%%%%%%%%%%%
%Exercise
\newenvironment{exercise}[2][]{%
\refstepcounter{theo}%
\ifstrempty{#1}%
{\mdfsetup{%
frametitle={%
\tikz[baseline=(current bounding box.east),outer sep=0pt]
\node[anchor=east,rectangle,fill=pink!20]
{\strut {\sc Exercise~\thetheo}};}}
}%
{\mdfsetup{%
frametitle={%
\tikz[baseline=(current bounding box.east),outer sep=0pt]
\node[anchor=east,rectangle,fill=pink!20]
{\strut {\sc Exercise~\thetheo:~#1}};}}%
}%
\mdfsetup{innertopmargin=10pt,linecolor=pink!20,%
linewidth=2pt,topline=true,%
frametitleaboveskip=\dimexpr-\ht\strutbox\relax
}
\begin{mdframed}[nobreak=true]\relax%
\label{#2}}{\end{mdframed}}

%%%%%%%%%%%%%%%%%%%%%%%%%%%%%%
%Lemma
\newenvironment{lemma}[2][]{%
\refstepcounter{theo}%
\ifstrempty{#1}%
{\mdfsetup{%
frametitle={%
\tikz[baseline=(current bounding box.east),outer sep=0pt]
\node[anchor=east,rectangle,fill=green!20]
{\strut {\sc Lemma~\thetheo}};}}
}%
{\mdfsetup{%
frametitle={%
\tikz[baseline=(current bounding box.east),outer sep=0pt]
\node[anchor=east,rectangle,fill=green!20]
{\strut {\sc Lemma~\thetheo:~#1}};}}%
}%
\mdfsetup{innertopmargin=10pt,linecolor=green!20,%
linewidth=2pt,topline=true,%
frametitleaboveskip=\dimexpr-\ht\strutbox\relax
}
\begin{mdframed}[nobreak=true]\relax%
\label{#2}}{\end{mdframed}}

%%%%%%%%%%%%%%%%%%%%%%%%%%%%%%
%Proposition
\newenvironment{prop}[2][]{%
\refstepcounter{theo}%
\ifstrempty{#1}%
{\mdfsetup{%
frametitle={%
\tikz[baseline=(current bounding box.east),outer sep=0pt]
\node[anchor=east,rectangle,fill=purple!20]
{\strut {\sc Proposition~\thetheo}};}}
}%
{\mdfsetup{%
frametitle={%
\tikz[baseline=(current bounding box.east),outer sep=0pt]
\node[anchor=east,rectangle,fill=purple!20]
{\strut {\sc Proposition~\thetheo:~#1}};}}%
}%
\mdfsetup{innertopmargin=10pt,linecolor=purple!20,%
linewidth=2pt,topline=true,%
frametitleaboveskip=\dimexpr-\ht\strutbox\relax
}
\begin{mdframed}[nobreak=true]\relax%
\label{#2}}{\end{mdframed}}

%%%%%%%%%%%%%%%%%%%%%%%%%%%%%%
%Fact
\newenvironment{fact}[2][]{%
\refstepcounter{theo}%
\ifstrempty{#1}%
{\mdfsetup{%
frametitle={%
\tikz[baseline=(current bounding box.east),outer sep=0pt]
\node[anchor=east,rectangle,fill=gray!20]
{\strut {\sc Fact~\thetheo}};}}
}%
{\mdfsetup{%
frametitle={%
\tikz[baseline=(current bounding box.east),outer sep=0pt]
\node[anchor=east,rectangle,fill=gray!20]
{\strut {\sc Fact~\thetheo:~#1}};}}%
}%
\mdfsetup{innertopmargin=10pt,linecolor=gray!20,%
linewidth=2pt,topline=true,%
frametitleaboveskip=\dimexpr-\ht\strutbox\relax
}
\begin{mdframed}[nobreak=true]\relax%
\label{#2}}{\end{mdframed}}

\newenvironment{pf}[1][\proofname]
  {\par\noindent\normalfont\textsc{Proof of #1.}\par\nopagebreak%
  \begin{mdframed}[
     linewidth=1pt,
     linecolor=black,
     bottomline=false,topline=false,rightline=false,
     innerrightmargin=0pt,innertopmargin=0pt,innerbottommargin=0pt,
     innerleftmargin=1em,% Distance between vertical rule & proof content
     skipabove=0.75\baselineskip
   ]}
  {\end{mdframed}}

% 1-inch margins
\topmargin 0pt
\advance \topmargin by -\headheight
\advance \topmargin by -\headsep
\textheight 8.9in
\oddsidemargin 0pt
\evensidemargin \oddsidemargin
\marginparwidth 0.5in
\textwidth 6.5in

\parindent 0in
\parskip 1.5ex

\setlist[itemize]{topsep=0pt}
\setlist[enumerate]{topsep=0pt}

\newcommand{\pushright}[1]{\ifmeasuring@#1\else\omit\hfill$\displaystyle#1$\fi\ignorespaces}

% hyperlinks
\hypersetup{
  colorlinks=true, 
  linktoc=all,     % table of contents is clickable  
  allcolors=red    % all hyperlink colours
}

% table of contents
\addto\captionsenglish{
  \renewcommand{\contentsname}%
    {Table of Contents}%
}
\renewcommand{\cftsecfont}{\normalfont}
\renewcommand{\cftsecpagefont}{\normalfont}
\cftsetindents{section}{0em}{2em}

\fancypagestyle{plain}{%
\fancyhf{} % clear all header and footer fields
\lhead{PMATH 441: Winter 2023}
\fancyhead[R]{Table of Contents}
%\headrule
\fancyfoot[R]{{\small Page \thepage\ of \pageref*{LastPage}}}
}

% headers and footers
\pagestyle{fancy}
\renewcommand{\sectionmark}[1]{\markboth{#1}{#1}}
\lhead{PMATH 441: Winter 2023}
\cfoot{}
\setlength\headheight{14pt}

%\setcounter{section}{-1}

\begin{document}

\pagestyle{fancy}
\newtitle{PMATH 441}{Algebraic Number Theory}{Blake Madill}{Winter 2023}
\rhead{Table of Contents}
\rfoot{{\small Page \thepage\ of \pageref*{LastPage}}}

\tableofcontents
\vspace{1cm}\hrule
\fancyhead[R]{\nouppercase\rightmark}
\newpage 
\fancyhead[R]{Section \thesection: \nouppercase\leftmark}

\section{Algebraic Integers}\label{sec:1}

\subsection{Motivation}\label{subsec:1.1}
At its most elementary, number theory is the study of integers. Some of the 
hot topics typically discussed in a first-year number theory course 
include primes, divisibility, the Euclidean algorithm, and of most 
interest to us, prime factorization. Our goal in this course is to generalize 
these topics using commutative algebra. 

One naive approach would be to consider unique factorization domains, or 
UFDs. However, the canonical example of a principal ideal domain (PID) 
that is not a UFD is $\Z[\sqrt{5}]$, which is far too integer-like 
to be disqualified from our discussion. 

Let's do some investigation. Consider $\alpha = (1 + \sqrt{5})/2$. We have 
$(2\alpha - 1)^2 = 5$, and expanding gives us $4\alpha^2 - 4\alpha - 4 = 0$. 
In particular, we see that 
\[ \alpha^2 = \alpha + 1. \] 
Next, let's consider the ring $\Z[\alpha] = \{f(\alpha) : f(x) \in \Z[x]\}$. 
Since $\alpha^2 = \alpha + 1$, we have that
\[ \Z[\alpha] = \{a + b\alpha : a, b \in \Z\}, \] 
since there are no need for terms $\alpha^n$ with $n \geq 2$. What made 
this simplification work? 
\begin{enumerate}[(a)]
    \item We needed a monic polynomial $f(x) \in \Z[x]$ such that $f(\alpha) = 0$. 
    \item Moreover, notice that $5 \equiv 1 \pmod 4$, so we could nicely 
    divide all the terms by $4$ in the equation $4\alpha^2 - 4\alpha - 4 = 0$. 
\end{enumerate}
More generally, why do we want to work with $\Z[\alpha] = \Z + \Z\alpha 
+ \cdots + \Z\alpha^{n-1}$? This is because it allows us to do 
finite-dimensional ``linear algebra'' over $\Z$.

\subsection{Algebraic Integers}\label{subsec:1.2}
Now that we are properly motivated, let's introduce the algebraic integers.

\begin{defn}{defn:1.1}
    We call $\alpha \in \C$ an {\bf algebraic integer} if there 
    exists a monic polynomial $f(x) \in \Z[x]$ such that $f(\alpha) = 0$. 
\end{defn}

Note that in the above definition, we do not insist that $f(x) \in \Z[x]$ 
is irreducible.

It is not hard to see that $n$ and $\sqrt{n}$ are algebraic integers for 
all $n \in \mathbb{Z}$. By our previous work, we see that $(1 + \sqrt{5})/2$ 
is an algebraic integer. It can also be shown that $i$, $1 + i$ and 
$\zeta_n = e^{2\pi i/n}$ are all algebraic integers.

We can ignore all transcendental numbers here, because they are certainly 
not algebraic integers. But how do we tell if an algebraic number $\alpha \in \C$ 
(i.e. $\alpha$ is algebraic over $\Q$) is an algebraic integer? The following 
theorem gives us a simple test to do so.

\begin{theo}{theo:1.2}
    An algebraic number $\alpha \in \C$ is an algebraic integer if and only if 
    its minimal polynomial over $\Q$ has integer coefficients.
\end{theo}

An easy corollary we can obtain is that the only algebraic integers 
in $\Q$ are the ordinary integers. Indeed, the minimal polynomial of 
a rational number $q \in \Q$ is $m(x) = x - q$, which is in $\Z[x]$ 
if and only if $q \in \Z$. 

For another example, let us consider $\beta = (1 + \sqrt{3})/2$ (noting that 
$3 \not\equiv 1 \pmod 4$ here). Performing the same manipulations as before, 
we deduce that $4\beta^2 - 4\beta - 2 = 0$ and hence $\beta^2 - \beta - 1/2 = 0$. 
In fact, $m(x) = x^2 - x - 1/2$ is the minimal polynomial for $\beta$ over 
$\Q$. Indeed, $m(x)$ is monic by performing the eyeball test, and it is 
irreducible since we know the roots are $(1 \pm \sqrt{3})/2$, which are 
not in $\Q$. By applying Theorem~\ref{theo:1.2}, it follows that $\beta$ is 
\emph{not} an algebraic integer. 

A concern one might have is that $\beta = (1 + \sqrt{3})/2$ also seems to be 
integer-like, and so we shouldn't dismiss it. However, we shouldn't expect 
it to work that nicely because it behaves more like a rational; we were 
more lucky with $\alpha = (1 + \sqrt{5})/2$ because it happened to be the 
case that $5 \equiv 1 \pmod 4$, as we observed earlier.

With these examples out of the way, let's jump into the proof of the theorem. 
Recall that for a polynomial $f(x) = a_n x^n + \cdots + a_1 x + a_0 \in \Z[x]$, 
the {\bf content} of $f(x)$ is 
\[ \Content(f(x)) = \gcd(a_n, a_{n-1}, \dots, a_0). \] 
We say that $f(x)$ is {\bf primitive} if $\Content(f(x)) = 1$. Moreover, 
an equivalent formulation of Gauss' lemma states that if $f(x), g(x) \in \Z[x]$ 
are primitive, then $f(x)g(x)$ is also primitive.

\begin{pf}[Theorem~\ref{theo:1.2}]
    $(\Leftarrow)$ This is immediate by considering the minimal polynomial 
    of $\alpha$ over $\Q$, say $m(x) \in \Z[x]$, which is monic and satisfies 
    $m(\alpha) = 0$. 

    $(\Rightarrow)$ Let $\alpha \in \C$ be an algebraic integer and let
    $m(x) \in \Q[x]$ be its minimal polynomial. Let $f(x) \in \Z[x]$ be 
    monic such that $f(\alpha) = 0$. Then by the properties of a minimal 
    polynomial, we have $m(x) \mid f(x)$. That is, we can write $f(x) = 
    m(x)g(x)$ for some $g(x) \in \Q[x]$. 

    Let $N_1, N_2 \in \N$ be minimal such that $N_1 m(x), N_2 g(x) \in \Z[x]$. 
    Note that if $p$ is a prime dividing all coefficients of $N_1 m(x)$, 
    then $(N_1/p)m(x) \in \Z[x]$, and in fact, we also have $N_1/p \in \Z$ 
    since $m(x)$ is monic. This contradicts the minimality of $N_1$, so 
    $N_1 m(x)$ must be primitive. Similarly, $N_2 g(x)$ is primitive by 
    the same argument, noting that $g(x)$ is monic since $f(x)$ and $m(x)$ are. 

    Now, observe that $N_1 N_2 f(x) = (N_1 m(x))(N_2 g(x))$ is primitive by 
    Gauss' lemma. Again, we note that $f(x)$ is monic, so equating contents 
    gives us $N_1 N_2 = 1$. It follows that $N_1 = N_2 = 1$, and 
    in particular, we have $m(x) \in \Z[x]$ as desired. \qed
\end{pf}

\subsection{Rings of Integers}\label{subsec:1.3}
We now work through an example which is considered a rite of passage through 
algebraic number theory. Let $d \in \Z$ be square-free where $d \neq 1$. 
Recall that being square-free means that there is no multiplicity in its 
prime factorization. Consider the field extension
\[ K = \Q(\sqrt{d}) = \{a + b\sqrt{d} : a, b \in \Q\}. \] 
In particular, $K/\Q$ is a finite extension and hence algebraic. We wish to 
find all the algebraic integers in $K$.

Suppose that $\alpha = a + b\sqrt{d}$ is an algebraic integer, and let 
$\overline\alpha = a - b\sqrt{d}$ be its complex conjugate. Using some 
Galois theory, the minimal polynomial of $\alpha$ is 
\[ m(x) = (x - \alpha)(x - \overline\alpha) = x^2 - 2ax + a^2 - db^2. \] 
We know that $m(x) \in \Z[x]$ by Theorem~\ref{theo:1.2}, so we must have 
$2a$, $a^2 - db^2 \in \Z$. Next, we have 
\[ 4(a^2 - db^2) = (2a)^2 - d(2b)^2 \in \Z, \] 
so $d(2b)^2 \in \Z$. Then by a denominator argument, we find that 
$2b \in \Z$ as well since $d$ is square-free.

Write $u = 2a$ and $v = 2b$ so that $a = u/2$ and $b = v/2$. We obtain 
\[ a^2 - db^2 = \left(\frac{u}{2}\right)^2 - d\left(\frac{v}{2}\right)^2 
= \frac{u^2 - dv^2}{4} \in \Z, \] 
so $u^2 - dv^2 \equiv 0 \pmod 4$. We now consider what form $\alpha$ 
can take under a few cases. Note that the $d \equiv 0 \pmod 4$ case is 
impossible since $d$ is square-free. 

\textbf{Case 1.} If $d \equiv 1 \pmod 4$, then $u^2 \equiv v^2 \pmod 4$. 
Recall that the square of an even number is $0 \pmod 4$ and the square 
of an odd number is $1 \pmod 4$, so this is equivalent to $u \equiv v \pmod 2$. 
That is, we have $\alpha = a + b\sqrt{d} = (u/2) + (v/2)\sqrt{d}$ for $u$ 
and $v$ with the same parity.

\textbf{Case 2.} If $d \equiv 2 \pmod 4$ or $d \equiv 3 \pmod 4$, then 
it can be shown that $u^2 - dv^2 \equiv 0 \pmod 4$ is equivalent to having 
$u \equiv v \equiv 0 \pmod 2$. This means that $\alpha = a' + b'\sqrt{d}$ for 
some $a', b' \in \Z$. 

We leave it as an exercise to check that these conditions are also sufficient, 
which can be done by reversing the arguments above.

More generally, given a finite field extension $K/\Q$, we want to describe 
all the algebraic integers in $K$. This leads us to the following definitions.

\begin{defn}{defn:1.3}
    We call a finite field extension $K$ of $\Q$ a {\bf number field}.
    For a number field $K$, we call 
    \[ \mathcal{O}_K = \{\alpha \in K : \alpha \text{ an algebraic integer}\} \] 
    the {\bf ring of integers} of $K$.
\end{defn}

Obviously, we'll need to prove that ${\cal O}_K$ is indeed a ring (namely, 
a subring of $\C$). To do this, we'll define 
\[ \mathbb{A} = \{z \in \C : z \text{ an algebraic integer}\} \] 
and show that $\mathbb{A}$ is a ring, which will imply that ${\cal O}_K 
= \mathbb{A} \cap K$ is a ring too. 

Before that, let's move on to some more definitions. Recall that in 
Section~\ref{subsec:1.1}, we wanted to work with 
\[ \Z[\alpha] = \Z + \Z\alpha + \cdots + \Z\alpha^{n-1} \] 
where $\alpha \in \mathbb{A}$ in order to do ``linear algebra'' over $\Z$. But $\Z$ is not a field, 
so we'll need something more general.

\begin{defn}{defn:1.4}
    Let $R$ be a ring. An {\bf $R$-module} is an abelian group $(M, +)$ 
    together with an operation $\cdot : R \times M \to M$ such that 
    \begin{enumerate}[(i)]
        \item for all $m \in M$, we have $1m = m$; 
        \item for all $r_1, r_2 \in R$ and $m \in M$, we have 
        $(r_1 + r_2)m = r_1m + r_2m$;
        \item for all $r \in R$ and $m_1, m_2 \in M$, we have 
        $r(m_1 + m_2) = rm_1 + rm_2$; 
        \item for all $r_1, r_2 \in R$ and $m \in M$, we have 
        $(r_1r_2)m = r_1(r_2m)$.
    \end{enumerate}
\end{defn}

We can think of the operation $\cdot : R \times M \to M$ as the ``$R$-action 
on $M$''. Note that if $R$ is a field, then an $R$-module is the same as 
an $R$-vector space. \newpage

\end{document}
