\documentclass[10pt]{article}
\usepackage[T1]{fontenc}
\usepackage{amsmath,amssymb,amsthm}
\usepackage{mathtools}
\usepackage[shortlabels]{enumitem}
\usepackage[english]{babel}
\usepackage[utf8]{inputenc}
\usepackage{fancyhdr}
\usepackage{bold-extra}
\usepackage{color}   
\usepackage{tocloft}
\usepackage{graphicx}
\usepackage{lipsum}
\usepackage{wrapfig}
\usepackage{cutwin}
\usepackage{hyperref}
\usepackage{lastpage}
\usepackage{multicol}
\usepackage{tikz}
\usepackage{xcolor}
\usepackage{microtype}
\usepackage{framed}
\usepackage{mleftright}


\usepackage[framemethod=TikZ]{mdframed}

% some useful math commands
\newcommand{\eps}{\varepsilon}
\newcommand{\R}{\mathbb{R}}
\newcommand{\C}{\mathbb{C}}
\newcommand{\N}{\mathbb{N}}
\newcommand{\Z}{\mathbb{Z}}
\newcommand{\Q}{\mathbb{Q}}
\newcommand{\K}{\mathbb{K}}
\newcommand{\F}{\mathbb{F}}
\newcommand{\T}{\mathbb{T}}

\numberwithin{equation}{section}

\newcommand{\dd}{\,\mathrm{d}}
\newcommand{\ddz}{\frac{\rm d}{{\rm d}z}}
\newcommand{\pv}{\text{p.v.}}

\renewcommand{\Re}{{\rm Re}}
\renewcommand{\Im}{{\rm Im}}

\DeclareMathOperator{\GL}{GL}
\DeclareMathOperator{\id}{id}
\DeclareMathOperator{\Arg}{Arg}
\DeclareMathOperator{\Log}{Log}
\DeclareMathOperator{\PV}{PV}
\DeclareMathOperator{\sech}{sech}
\DeclareMathOperator{\csch}{csch}
\DeclareMathOperator{\Res}{Res}
\DeclareMathOperator{\Li}{Li}
\DeclareMathOperator{\QR}{QR}
\DeclareMathOperator{\NR}{NR}
\DeclareMathOperator{\lcm}{lcm}
\DeclareMathOperator{\divergence}{div}
\DeclareMathOperator*{\esssup}{ess\,sup}
\DeclareMathOperator{\Span}{span}
\DeclareMathOperator{\Vol}{Vol}
\DeclareMathOperator{\Frac}{Frac}
\DeclareMathOperator{\Pol}{Pol}
\DeclareMathOperator{\Content}{Content}
\DeclareMathOperator{\rank}{rank}
\DeclareMathOperator{\Tr}{Tr}
\DeclareMathOperator{\diag}{diag}
\DeclareMathOperator{\disc}{disc}
\DeclareMathOperator{\adj}{adj}
\DeclareMathOperator{\Gal}{Gal}
\DeclareMathOperator{\cl}{cl}

\DeclarePairedDelimiter\ceil{\lceil}{\rceil}
\DeclarePairedDelimiter\floor{\lfloor}{\rfloor}

\newcommand{\suchthat}{\;\ifnum\currentgrouptype=16 \;\middle|\;\else\mid\fi\;}

% title formatting
\newcommand{\newtitle}[4]{
  \begin{center}
	\huge{\textbf{\textsc{#1 Course Notes}}}
    
	\large{\sc #2}
    
	{\sc #3 \textbullet\, #4 \textbullet\, University of Waterloo}
	\normalsize\vspace{1cm}\hrule
  \end{center}
}

\newcounter{theo}[section]\setcounter{theo}{0}
\renewcommand{\thetheo}{\arabic{section}.\arabic{theo}}
\newenvironment{theo}[2][]{%
\refstepcounter{theo}%
\ifstrempty{#1}%
{\mdfsetup{%
frametitle={%
\tikz[baseline=(current bounding box.east),outer sep=0pt]
\node[anchor=east,rectangle,fill=blue!20]
{\strut {\sc Theorem~\thetheo}};}}
}%
{\mdfsetup{%
frametitle={%
\tikz[baseline=(current bounding box.east),outer sep=0pt]
\node[anchor=east,rectangle,fill=blue!20]
{\strut {\sc Theorem~\thetheo:~#1}};}}%
}%
\mdfsetup{innertopmargin=10pt,linecolor=blue!20,%
linewidth=2pt,topline=true,%
frametitleaboveskip=\dimexpr-\ht\strutbox\relax
}
\begin{mdframed}[nobreak=false]\relax%
\label{#2}}{\end{mdframed}}

%%%%%%%%%%%%%%%%%%%%%%%%%%%%%%
%Definition
\newenvironment{defn}[2][]{%
\refstepcounter{theo}%
\ifstrempty{#1}%
{\mdfsetup{%
frametitle={%
\tikz[baseline=(current bounding box.east),outer sep=0pt]
\node[anchor=east,rectangle,fill=yellow!20]
{\strut {\sc Definition~\thetheo}};}}
}%
{\mdfsetup{%
frametitle={%
\tikz[baseline=(current bounding box.east),outer sep=0pt]
\node[anchor=east,rectangle,fill=yellow!20]
{\strut {\sc Definition~\thetheo:~#1}};}}%
}%
\mdfsetup{innertopmargin=10pt,linecolor=yellow!20,%
linewidth=2pt,topline=true,%
frametitleaboveskip=\dimexpr-\ht\strutbox\relax
}
\begin{mdframed}[nobreak=true]\relax%
\label{#2}}{\end{mdframed}}

%%%%%%%%%%%%%%%%%%%%%%%%%%%%%%
%Example
\newenvironment{exmp}[2][]{%
\refstepcounter{theo}%
\ifstrempty{#1}%
{\mdfsetup{%
frametitle={%
\tikz[baseline=(current bounding box.east),outer sep=0pt]
\node[anchor=east,rectangle,fill=cyan!20]
{\strut {\sc Example~\thetheo}};}}
}%
{\mdfsetup{%
frametitle={%
\tikz[baseline=(current bounding box.east),outer sep=0pt]
\node[anchor=east,rectangle,fill=cyan!20]
{\strut {\sc Example~\thetheo:~#1}};}}%
}%
\mdfsetup{innertopmargin=10pt,linecolor=cyan!20,%
linewidth=2pt,topline=true,%
frametitleaboveskip=\dimexpr-\ht\strutbox\relax
}
\begin{mdframed}[nobreak=false]\relax%
\label{#2}}{\end{mdframed}}

%%%%%%%%%%%%%%%%%%%%%%%%%%%%%%
%Corollary
\newenvironment{cor}[2][]{%
\refstepcounter{theo}%
\ifstrempty{#1}%
{\mdfsetup{%
frametitle={%
\tikz[baseline=(current bounding box.east),outer sep=0pt]
\node[anchor=east,rectangle,fill=lime!20]
{\strut {\sc Corollary~\thetheo}};}}
}%
{\mdfsetup{%
frametitle={%
\tikz[baseline=(current bounding box.east),outer sep=0pt]
\node[anchor=east,rectangle,fill=lime!20]
{\strut {\sc Corollary~\thetheo:~#1}};}}%
}%
\mdfsetup{innertopmargin=10pt,linecolor=lime!20,%
linewidth=2pt,topline=true,%
frametitleaboveskip=\dimexpr-\ht\strutbox\relax
}
\begin{mdframed}[nobreak=true]\relax%
\label{#2}}{\end{mdframed}}

%%%%%%%%%%%%%%%%%%%%%%%%%%%%%%
%Remark
\newenvironment{remark}[2][]{%
\refstepcounter{theo}%
\ifstrempty{#1}%
{\mdfsetup{%
frametitle={%
\tikz[baseline=(current bounding box.east),outer sep=0pt]
\node[anchor=east,rectangle,fill=orange!20]
{\strut {\sc Remark~\thetheo}};}}
}%
{\mdfsetup{%
frametitle={%
\tikz[baseline=(current bounding box.east),outer sep=0pt]
\node[anchor=east,rectangle,fill=orange!20]
{\strut {\sc Remark~\thetheo:~#1}};}}%
}%
\mdfsetup{innertopmargin=10pt,linecolor=orange!20,%
linewidth=2pt,topline=true,%
frametitleaboveskip=\dimexpr-\ht\strutbox\relax
}
\begin{mdframed}[nobreak=true]\relax%
\label{#2}}{\end{mdframed}}

%%%%%%%%%%%%%%%%%%%%%%%%%%%%%%
%Exercise
\newenvironment{exercise}[2][]{%
\refstepcounter{theo}%
\ifstrempty{#1}%
{\mdfsetup{%
frametitle={%
\tikz[baseline=(current bounding box.east),outer sep=0pt]
\node[anchor=east,rectangle,fill=pink!20]
{\strut {\sc Exercise~\thetheo}};}}
}%
{\mdfsetup{%
frametitle={%
\tikz[baseline=(current bounding box.east),outer sep=0pt]
\node[anchor=east,rectangle,fill=pink!20]
{\strut {\sc Exercise~\thetheo:~#1}};}}%
}%
\mdfsetup{innertopmargin=10pt,linecolor=pink!20,%
linewidth=2pt,topline=true,%
frametitleaboveskip=\dimexpr-\ht\strutbox\relax
}
\begin{mdframed}[nobreak=true]\relax%
\label{#2}}{\end{mdframed}}

%%%%%%%%%%%%%%%%%%%%%%%%%%%%%%
%Lemma
\newenvironment{lemma}[2][]{%
\refstepcounter{theo}%
\ifstrempty{#1}%
{\mdfsetup{%
frametitle={%
\tikz[baseline=(current bounding box.east),outer sep=0pt]
\node[anchor=east,rectangle,fill=green!20]
{\strut {\sc Lemma~\thetheo}};}}
}%
{\mdfsetup{%
frametitle={%
\tikz[baseline=(current bounding box.east),outer sep=0pt]
\node[anchor=east,rectangle,fill=green!20]
{\strut {\sc Lemma~\thetheo:~#1}};}}%
}%
\mdfsetup{innertopmargin=10pt,linecolor=green!20,%
linewidth=2pt,topline=true,%
frametitleaboveskip=\dimexpr-\ht\strutbox\relax
}
\begin{mdframed}[nobreak=true]\relax%
\label{#2}}{\end{mdframed}}

%%%%%%%%%%%%%%%%%%%%%%%%%%%%%%
%Proposition
\newenvironment{prop}[2][]{%
\refstepcounter{theo}%
\ifstrempty{#1}%
{\mdfsetup{%
frametitle={%
\tikz[baseline=(current bounding box.east),outer sep=0pt]
\node[anchor=east,rectangle,fill=purple!20]
{\strut {\sc Proposition~\thetheo}};}}
}%
{\mdfsetup{%
frametitle={%
\tikz[baseline=(current bounding box.east),outer sep=0pt]
\node[anchor=east,rectangle,fill=purple!20]
{\strut {\sc Proposition~\thetheo:~#1}};}}%
}%
\mdfsetup{innertopmargin=10pt,linecolor=purple!20,%
linewidth=2pt,topline=true,%
frametitleaboveskip=\dimexpr-\ht\strutbox\relax
}
\begin{mdframed}[nobreak=true]\relax%
\label{#2}}{\end{mdframed}}

%%%%%%%%%%%%%%%%%%%%%%%%%%%%%%
%Fact
\newenvironment{fact}[2][]{%
\refstepcounter{theo}%
\ifstrempty{#1}%
{\mdfsetup{%
frametitle={%
\tikz[baseline=(current bounding box.east),outer sep=0pt]
\node[anchor=east,rectangle,fill=gray!20]
{\strut {\sc Fact~\thetheo}};}}
}%
{\mdfsetup{%
frametitle={%
\tikz[baseline=(current bounding box.east),outer sep=0pt]
\node[anchor=east,rectangle,fill=gray!20]
{\strut {\sc Fact~\thetheo:~#1}};}}%
}%
\mdfsetup{innertopmargin=10pt,linecolor=gray!20,%
linewidth=2pt,topline=true,%
frametitleaboveskip=\dimexpr-\ht\strutbox\relax
}
\begin{mdframed}[nobreak=true]\relax%
\label{#2}}{\end{mdframed}}

\newenvironment{pf}[1][\proofname]
  {\par\noindent\normalfont\textsc{Proof of #1.}\par\nopagebreak%
  \begin{mdframed}[
     linewidth=1pt,
     linecolor=black,
     bottomline=false,topline=false,rightline=false,
     innerrightmargin=0pt,innertopmargin=0pt,innerbottommargin=0pt,
     innerleftmargin=1em,% Distance between vertical rule & proof content
     skipabove=0.75\baselineskip
   ]}
  {\end{mdframed}}

% 1-inch margins
\topmargin 0pt
\advance \topmargin by -\headheight
\advance \topmargin by -\headsep
\textheight 8.9in
\oddsidemargin 0pt
\evensidemargin \oddsidemargin
\marginparwidth 0.5in
\textwidth 6.5in

\parindent 0in
\parskip 1.5ex

\setlist[itemize]{topsep=0pt}
\setlist[enumerate]{topsep=0pt}

\newcommand{\pushright}[1]{\ifmeasuring@#1\else\omit\hfill$\displaystyle#1$\fi\ignorespaces}

% hyperlinks
\hypersetup{
  colorlinks=true, 
  linktoc=all,     % table of contents is clickable  
  allcolors=red    % all hyperlink colours
}

% table of contents
\addto\captionsenglish{
  \renewcommand{\contentsname}%
    {Table of Contents}%
}
\renewcommand{\cftsecfont}{\normalfont}
\renewcommand{\cftsecpagefont}{\normalfont}
\cftsetindents{section}{0em}{2em}

\fancypagestyle{plain}{%
\fancyhf{} % clear all header and footer fields
\lhead{PMATH 441: Winter 2023}
\fancyhead[R]{Table of Contents}
%\headrule
\fancyfoot[R]{{\small Page \thepage\ of \pageref*{LastPage}}}
}

% headers and footers
\pagestyle{fancy}
\renewcommand{\sectionmark}[1]{\markboth{#1}{#1}}
\lhead{PMATH 441: Winter 2023}
\cfoot{}
\setlength\headheight{14pt}

\cftsetindents{subsection}{1em}{2.75em}

%\setcounter{section}{-1}

\begin{document}

\pagestyle{fancy}
\newtitle{PMATH 441}{Algebraic Number Theory}{Blake Madill}{Winter 2023}
\rhead{Table of Contents}
\rfoot{{\small Page \thepage\ of \pageref*{LastPage}}}

\tableofcontents
\vspace{1cm}\hrule
\fancyhead[R]{\nouppercase\rightmark}
\newpage 
\fancyhead[R]{Section \thesection: \nouppercase\leftmark}

\section{Algebraic Integers}\label{sec:1}

\subsection{Motivation}\label{subsec:1.1}
At its most elementary, number theory is the study of integers. Some of the 
hot topics typically discussed in a first-year number theory course 
include primes, divisibility, the Euclidean algorithm, and of most 
interest to us, prime factorization. Our goal in this course is to generalize 
these topics using commutative algebra. 

One naive approach would be to consider unique factorization domains, or 
UFDs. However, the canonical example of a principal ideal domain (PID) 
that is not a UFD is $\Z[\sqrt{5}]$, which is far too integer-like 
to be disqualified from our discussion. 

Let's do some investigation. Consider $\alpha = (1 + \sqrt{5})/2$. We have 
$(2\alpha - 1)^2 = 5$, and expanding gives us $4\alpha^2 - 4\alpha - 4 = 0$. 
In particular, we see that 
\[ \alpha^2 = \alpha + 1. \] 
Next, let's consider the ring $\Z[\alpha] = \{f(\alpha) : f(x) \in \Z[x]\}$. 
Since $\alpha^2 = \alpha + 1$, we have that
\[ \Z[\alpha] = \{a + b\alpha : a, b \in \Z\}, \] 
since there are no need for terms $\alpha^n$ with $n \geq 2$. What made 
this simplification work? 
\begin{enumerate}[(a)]
    \item We needed a monic polynomial $f(x) \in \Z[x]$ such that $f(\alpha) = 0$. 
    \item Moreover, notice that $5 \equiv 1 \pmod 4$, so we could nicely 
    divide all the terms by $4$ in the equation $4\alpha^2 - 4\alpha - 4 = 0$. 
\end{enumerate}
More generally, why do we want to work with $\Z[\alpha] = \Z + \Z\alpha 
+ \cdots + \Z\alpha^{n-1}$? This is because it allows us to do 
finite-dimensional ``linear algebra'' over $\Z$.

\subsection{Algebraic Integers}\label{subsec:1.2}
Now that we are properly motivated, let's introduce the algebraic integers.

\begin{defn}{defn:1.1}
    We call $\alpha \in \C$ an {\bf algebraic integer} if there 
    exists a monic polynomial $f(x) \in \Z[x]$ such that $f(\alpha) = 0$. 
\end{defn}

Note that in the above definition, we do not insist that $f(x) \in \Z[x]$ 
is irreducible.

It is not hard to see that $n$ and $\sqrt{n}$ are algebraic integers for 
all $n \in \mathbb{Z}$. By our previous work, we see that $(1 + \sqrt{5})/2$ 
is an algebraic integer. It can also be shown that $i$, $1 + i$ and 
$\zeta_n = e^{2\pi i/n}$ are all algebraic integers.

We can ignore all transcendental numbers here, because they are certainly 
not algebraic integers. But how do we tell if an algebraic number $\alpha \in \C$ 
(i.e. $\alpha$ is algebraic over $\Q$) is an algebraic integer? The following 
theorem gives us a simple test to do so.

\begin{theo}{theo:1.2}
    An algebraic number $\alpha \in \C$ is an algebraic integer if and only if 
    its minimal polynomial over $\Q$ has integer coefficients.
\end{theo}

An easy corollary we can obtain is that the only algebraic integers 
in $\Q$ are the ordinary integers. Indeed, the minimal polynomial of 
a rational number $q \in \Q$ is $m(x) = x - q$, which is in $\Z[x]$ 
if and only if $q \in \Z$. 

For another example, let us consider $\beta = (1 + \sqrt{3})/2$ (noting that 
$3 \not\equiv 1 \pmod 4$ here). Performing the same manipulations as before, 
we deduce that $4\beta^2 - 4\beta - 2 = 0$ and hence $\beta^2 - \beta - 1/2 = 0$. 
In fact, $m(x) = x^2 - x - 1/2$ is the minimal polynomial for $\beta$ over 
$\Q$. Indeed, $m(x)$ is monic by performing the eyeball test, and it is 
irreducible since we know the roots are $(1 \pm \sqrt{3})/2$, which are 
not in $\Q$. By applying Theorem~\ref{theo:1.2}, it follows that $\beta$ is 
\emph{not} an algebraic integer. 

A concern one might have is that $\beta = (1 + \sqrt{3})/2$ also seems to be 
integer-like, and so we shouldn't dismiss it. However, we shouldn't expect 
it to work that nicely because it behaves more like a rational; we were 
more lucky with $\alpha = (1 + \sqrt{5})/2$ because it happened to be the 
case that $5 \equiv 1 \pmod 4$, as we observed earlier.

With these examples out of the way, let's jump into the proof of the theorem. 
Recall that for a polynomial $f(x) = a_n x^n + \cdots + a_1 x + a_0 \in \Z[x]$, 
the {\bf content} of $f(x)$ is 
\[ \Content(f(x)) = \gcd(a_n, a_{n-1}, \dots, a_0). \] 
We say that $f(x)$ is {\bf primitive} if $\Content(f(x)) = 1$. Moreover, 
an equivalent formulation of Gauss' lemma states that if $f(x), g(x) \in \Z[x]$ 
are primitive, then $f(x)g(x)$ is also primitive.

\begin{pf}[Theorem~\ref{theo:1.2}]
    $(\Leftarrow)$ This is immediate by considering the minimal polynomial 
    of $\alpha$ over $\Q$, say $m(x) \in \Z[x]$, which is monic and satisfies 
    $m(\alpha) = 0$. 

    $(\Rightarrow)$ Let $\alpha \in \C$ be an algebraic integer and let
    $m(x) \in \Q[x]$ be its minimal polynomial. Let $f(x) \in \Z[x]$ be 
    monic such that $f(\alpha) = 0$. Then by the properties of a minimal 
    polynomial, we have $m(x) \mid f(x)$. That is, we can write $f(x) = 
    m(x)g(x)$ for some $g(x) \in \Q[x]$. 

    Let $N_1, N_2 \in \N$ be minimal such that $N_1 m(x), N_2 g(x) \in \Z[x]$. 
    Note that if $p$ is a prime dividing all coefficients of $N_1 m(x)$, 
    then $(N_1/p)m(x) \in \Z[x]$, and in fact, we also have $N_1/p \in \Z$ 
    since $m(x)$ is monic. This contradicts the minimality of $N_1$, so 
    $N_1 m(x)$ must be primitive. Similarly, $N_2 g(x)$ is primitive by 
    the same argument, noting that $g(x)$ is monic since $f(x)$ and $m(x)$ are. 

    Now, observe that $N_1 N_2 f(x) = (N_1 m(x))(N_2 g(x))$ is primitive by 
    Gauss' lemma. Again, we note that $f(x)$ is monic, so equating contents 
    gives us $N_1 N_2 = 1$. It follows that $N_1 = N_2 = 1$, and 
    in particular, we have $m(x) \in \Z[x]$ as desired. \qed
\end{pf}

\subsection{Rings of Integers}\label{subsec:1.3}
We now work through an example which is considered a rite of passage through 
algebraic number theory. Let $d \in \Z$ be square-free where $d \neq 1$. 
Recall that being square-free means that there is no multiplicity in its 
prime factorization. Consider the field extension
\[ K = \Q(\sqrt{d}) = \{a + b\sqrt{d} : a, b \in \Q\}. \] 
In particular, $K/\Q$ is a finite extension and hence algebraic. We wish to 
find all the algebraic integers in $K$.

Suppose that $\alpha = a + b\sqrt{d}$ is an algebraic integer, and let 
$\overline\alpha = a - b\sqrt{d}$ be its complex conjugate. Using some 
Galois theory, the minimal polynomial of $\alpha$ is 
\[ m(x) = (x - \alpha)(x - \overline\alpha) = x^2 - 2ax + a^2 - db^2. \] 
We know that $m(x) \in \Z[x]$ by Theorem~\ref{theo:1.2}, so we must have 
$2a$, $a^2 - db^2 \in \Z$. Next, we have 
\[ 4(a^2 - db^2) = (2a)^2 - d(2b)^2 \in \Z, \] 
so $d(2b)^2 \in \Z$. Then by a denominator argument, we find that 
$2b \in \Z$ as well since $d$ is square-free.

Write $u = 2a$ and $v = 2b$ so that $a = u/2$ and $b = v/2$. We obtain 
\[ a^2 - db^2 = \left(\frac{u}{2}\right)^2 - d\left(\frac{v}{2}\right)^2 
= \frac{u^2 - dv^2}{4} \in \Z, \] 
so $u^2 - dv^2 \equiv 0 \pmod 4$. We now consider what form $\alpha$ 
can take under a few cases. Note that the $d \equiv 0 \pmod 4$ case is 
impossible since $d$ is square-free. 

\textbf{Case 1.} If $d \equiv 1 \pmod 4$, then $u^2 \equiv v^2 \pmod 4$. 
Recall that the square of an even number is $0 \pmod 4$ and the square 
of an odd number is $1 \pmod 4$, so this is equivalent to $u \equiv v \pmod 2$. 
That is, we have $\alpha = a + b\sqrt{d} = (u/2) + (v/2)\sqrt{d}$ for $u$ 
and $v$ with the same parity.

\textbf{Case 2.} If $d \equiv 2 \pmod 4$ or $d \equiv 3 \pmod 4$, then 
it can be shown that $u^2 - dv^2 \equiv 0 \pmod 4$ is equivalent to having 
$u \equiv v \equiv 0 \pmod 2$. This means that $\alpha = a' + b'\sqrt{d}$ for 
some $a', b' \in \Z$. 

We leave it as an exercise to check that these conditions are also sufficient, 
which can be done by reversing the arguments above.

More generally, given a finite field extension $K/\Q$, we want to describe 
all the algebraic integers in $K$. This leads us to the following definitions.

\begin{defn}{defn:1.3}
    We call a finite field extension $K$ of $\Q$ a {\bf number field}.
    For a number field $K$, we call 
    \[ \mathcal{O}_K = \{\alpha \in K : \alpha \text{ an algebraic integer}\} \] 
    the {\bf ring of integers} of $K$.
\end{defn}

Obviously, we'll need to prove that ${\cal O}_K$ is indeed a ring (namely, 
a subring of $\C$). To do this, we'll define 
\[ \mathbb{A} = \{z \in \C : z \text{ an algebraic integer}\} \] 
and show that $\mathbb{A}$ is a ring, which will imply that ${\cal O}_K 
= \mathbb{A} \cap K$ is a ring too. 

Before that, let's move on to some more definitions. Recall that in 
Section~\ref{subsec:1.1}, we wanted to work with 
\[ \Z[\alpha] = \Z + \Z\alpha + \cdots + \Z\alpha^{n-1} \] 
where $\alpha \in \mathbb{A}$ in order to do ``linear algebra'' over $\Z$. But $\Z$ is not a field, 
so we'll need something more general.

\begin{defn}{defn:1.4}
    Let $R$ be a ring. An {\bf $R$-module} is an abelian group $(M, +)$ 
    together with an operation $\cdot : R \times M \to M$ such that 
    \begin{enumerate}[(i)]
        \item for all $m \in M$, we have $1m = m$; 
        \item for all $r_1, r_2 \in R$ and $m \in M$, we have 
        $(r_1 + r_2)m = r_1m + r_2m$;
        \item for all $r \in R$ and $m_1, m_2 \in M$, we have 
        $r(m_1 + m_2) = rm_1 + rm_2$; 
        \item for all $r_1, r_2 \in R$ and $m \in M$, we have 
        $(r_1r_2)m = r_1(r_2m)$.
    \end{enumerate}
\end{defn}

We can think of the operation $\cdot : R \times M \to M$ as the ``$R$-action 
on $M$''. Note that if $R$ is a field, then an $R$-module is the same as 
an $R$-vector space. \newpage
\section{Discriminants}\label{sec:2}

\subsection{Elementary Properties}\label{subsec:2.1}
Let $K$ be a number with $[K : \Q] = n$ and consider its ring of integers 
$R = {\cal O}_K$. Given $\{v_1, \dots, v_n\} \subseteq R$, we want to find a 
way to discriminate whether or not $\{v_1, \dots, v_n\}$ is an integral 
basis for $R$. This leads us to the notion of the discriminant. 

\begin{defn}{defn:2.1}
    Let $K$ be a number field with $[K : \Q] = n$. Let $\sigma_1, 
    \dots, \sigma_n$ be the embeddings of $K$ into $\C$. The 
    {\bf discriminant} of $\{a_1, \dots, a_n\} \subseteq K$ is 
    \[ \disc(a_1, \dots, a_n) = \det\,[\sigma_i(a_j)]^2. \] 
\end{defn}\vspace{-0.15cm}

In the matrix $[\sigma_i(a_j)]$ above, the rows are encoded by the 
embeddings $\sigma_i$ and the columns are encoded by the elements $a_j$. 
Now, let's investigate some properties of the discriminant. 
\begin{enumerate}[(1)]
    \item The discriminant is independent of the choice of ordering 
    for both the embeddings $\sigma_i$ and the elements $a_j$. This is 
    because squaring the determinant kills any negatives obtained 
    by flipping rows or columns.

    \item Let $B = [\sigma_i(a_j)]$ and $A = [\sigma_j(a_i)] = B^T$. 
    Taking the transpose leaves the determinant unchanged, so
    \[ \disc(a_1, \dots, a_n) = \det(AB). \] 
    Now, observe that the $(i, j)$-th entry of $AB$ is 
    \[ \begin{bmatrix} \sigma_1(a_i) \\ \sigma_2(a_i) \\ \vdots \\ \sigma_n(a_i) \end{bmatrix} 
    \cdot \begin{bmatrix} \sigma_1(a_j) \\ \sigma_2(a_j) \\ \vdots \\ \sigma_n(a_j) \end{bmatrix}
    = \sum_{k=1}^n \sigma_k(a_i a_j) = \Tr_{K/\Q}(a_ia_j). \]
    Thus, we obtain an equivalent definition of the discriminant seen in 
    some texts, which is given by 
    \[ \disc(a_1, \dots, a_n) = \det\,[\Tr_{K/\Q}(a_ia_j)] \in \Q. \] 
    Moreover, if we also assume that $a_1, \dots, a_n \in {\cal O}_K$, then 
    \[ \disc(a_1, \dots, a_n) \in \Q \cap {\cal O}_K = \Z. \] 

    \item Let $v, w \in K^n$ and $A \in M_n(\Q)$ be such that $Av = w$. 
    Observe that 
    \[ A \begin{bmatrix} \sigma_i(v_1) \\ \vdots \\ \sigma_i(v_n) \end{bmatrix} 
    = \begin{bmatrix} \sigma_i(a_{11}v_1 + \cdots + a_{1n}v_n) \\ \vdots \\ 
        \sigma_i(a_{n1}v_1 + \cdots + a_{nn}v_n) \end{bmatrix} 
    = \begin{bmatrix} \sigma_i(w_1) \\ \vdots \\ \sigma_i(w_n) \end{bmatrix}. \] 
    Therefore, we deduce that 
    \[ A \begin{bmatrix} \sigma_1(v_1) & \cdots & \sigma_n(v_1) \\ 
        \vdots & \ddots & \vdots \\ 
        \sigma_1(v_n) & \cdots & \sigma_n(v_n) \end{bmatrix} 
    = \begin{bmatrix} \sigma_1(w_1) & \cdots & \sigma_n(w_1) \\ 
        \vdots & \ddots & \vdots \\ 
        \sigma_1(w_n) & \cdots & \sigma_n(w_n) \end{bmatrix}. \] 
    The matrices above are the transposes of the matrices in the 
    definition of the discriminant, since the columns encode the embeddings 
    and the rows encode the elements this time. Taking squared determinants 
    gives the nice relationship
    \[ (\det A)^2 \cdot \disc(v) = \disc(w). \] 

    \item Let $\{v_1, \dots, v_n\} \subseteq {\cal O}_K$ be an integral basis.
    Suppose that $\{w_1, \dots, w_n\} \subseteq {\cal O}_K$. Then 
    for all $i = 1, \dots, n$, there must exist some $c_{ij} \in \Z$ such that 
    \[ w_i = c_{i1} v_1 + \cdots + c_{in} v_n. \] 
    Then we can write $w = Cv$ where $C = [c_{ij}]$, which yields 
    \[ \disc(w) = (\det C)^2 \cdot \disc(v). \] 
\end{enumerate}\newpage
\section{Prime Factorization}\label{sec:3}

\subsection{Ring Theory}\label{subsec:3.1}
Let $K$ be a number field and let $R = {\cal O}_K$ be its ring of integers. 
Recall that in Section~\ref{subsec:1.4}, we uncovered some purely 
ring theoretic facts about $R$. We'll restate them here as they'll be 
very useful to us soon. Note that when we speak about rings in this 
course, they are always commutative and unital.
\begin{enumerate}[(1)]
    \item Corollary~\ref{cor:1.19}: If $I$ is a nonzero ideal of $R$, then 
    $R/I$ is finite. 
    \item Corollary~\ref{cor:1.20}: Every nonzero prime ideal of $R$ is maximal.
    \item Corollary~\ref{cor:1.21}: $R$ is Noetherian. 
\end{enumerate}

We now give a few characterizations of being a Noetherian ring.

\begin{prop}{prop:3.1}
    Let $R$ be a ring. The following are equivalent:
    \begin{enumerate}[(1)]
        \item $R$ is Noetherian. 
        \item If $I_1 \subseteq I_2 \subseteq \cdots$ is an ascending 
        chain of ideals of $R$, then there exists some $N \in \N$ such that 
        $I_k = I_N$ for all $k \geq N$; that is, the chain terminates.
        \item Every nonempty set of ideals of $R$ has a maximal element 
        (with respect to $\subseteq$). 
    \end{enumerate}
\end{prop}\vspace{-0.25cm}
\begin{pf}[Proposition~\ref{prop:3.1}]
    $(1) \Rightarrow (2)$: Let $I_1 \subseteq I_2 \subseteq \cdots$ be 
    an ascending chain of ideals in $R$. Let $I = \bigcup_{j\in\N} I_j$, 
    and note that $I$ is an ideal of $R$. Since $R$ is Noetherian, 
    we see that $I$ is finitely generated, say by $a_1, \dots, a_s$. 
    Then for each $i = 1, \dots, s$, there exists some $N_i \in \N$ such that 
    $a_i \in I_{N_i}$. Taking $N = \max\{N_1, \dots, N_s\}$, we have 
    $a_i \in I_N$ for all $i = 1, \dots, s$ and thus $I \subseteq I_N$. 
    But $I_N \subseteq I$ by definition, so equality follows. In particular, 
    we have $I_k = I_N$ for all $k \geq N$. 
    
    $(2) \Rightarrow (3)$: Suppose ${\cal I}$ is a nonempty set of ideals of 
    $R$, and let $I_1 \in {\cal I}$ (which exists because ${\cal I}$ is 
    nonempty). If $I_1$ is maximal, we're done. Otherwise, ${\cal I} \setminus 
    \{I_1\}$ must be nonempty; we can find $I_2$ from this collection 
    such that $I_1 \subseteq I_2$ (else $I_1$ was maximal). If 
    $I_2$ is not maximal, pick $I_3 \in {\cal I} \setminus \{I_1, I_2\}$ 
    such that $I_1 \subseteq I_2 \subseteq I_3$. But by assumption, this 
    process terminates; for some $N \in \N$, we have $I_k = I_N$ 
    for all $k \geq N$, and $I_N$ is our desired maximal element in ${\cal I}$.

    $(3) \Rightarrow (1)$: Let $I$ be an ideal of $R$. Let ${\cal I}$ 
    denote the collection of all finitely generated ideals of $R$ 
    contained in $I$, which is nonempty because $\langle 0 \rangle 
    \in {\cal I}$. By assumption, ${\cal I}$ has a maximal element $J$. 
    If $J \neq I$, then we can find some $a \in I \setminus J$. Then 
    $\langle J, a \rangle$ is also finitely generated and contained in 
    ${\cal I}$, contradicting maximality. It follows that $J = I$ 
    and so $I$ is finitely generated. \qed 
\end{pf}\vspace{-0.25cm}

Note that the rings that we work with in this course are not generally UFDs, 
so we do not have the classical prime factorization from first year number 
theory. However, the following proposition gives us the idea to 
consider the factorization of proper ideals into prime ideals. 
The reason why we are only considering proper ideals here is because 
taking $I = R$ fails condition (1); every prime ideal of $R$ is proper by 
definition.

\begin{prop}{prop:3.2}
    Let $R$ be Noetherian and let $I \neq R$ be an ideal. 
    There exist proper ideals $P_1, \dots, P_n$ of $R$ such that 
    \begin{enumerate}[(1)]
        \item $I \subseteq P_i$ for all $i = 1, \dots, n$; and 
        \item $P_1P_2 \cdots P_n \subseteq I$. 
    \end{enumerate}
\end{prop}

Note that the prime ideals $P_i$ above are not necessarily distinct. 
To prove this proposition, we will use an extremely common tactic in 
commutative algebra: we assume that the set of objects that does 
not satisfy the property is nonempty, take a maximal element, and derive 
a contradiction. 

\begin{pf}[Proposition~\ref{prop:3.2}]
    Let $X$ be the set of proper ideals of $R$ not having this property. 
    By contradiction, assume that $X \neq \varnothing$. Let $I \in X$ be 
    a maximal element (with respect to $\subseteq$). Then $I$ itself 
    is not prime (otherwise we can simply take $P_1 = I$), so we can find 
    $a, b \in R$ such that $ab \in I$ but $a, b \notin I$. By 
    the maximality of $I$, we have $I + \langle a \rangle, I + \langle b 
    \rangle \notin X$. 

    Note that $(I + \langle a \rangle)(I + \langle b \rangle) 
    \subseteq I$ since $I^2$, $\langle a \rangle I$, $\langle b \rangle I$, 
    and $\langle ab \rangle$ are all subsets of $I$. In particular, we have 
    $I + \langle a \rangle \neq R$ and $I + \langle b \rangle \neq R$ since 
    multiplying by $R$ does not change an ideal. 

    Therefore, we can find prime ideals $P_1, \dots, P_n, Q_1, \dots, Q_m$ 
    of $R$ such that 
    \begin{enumerate}[(1)]
        \item $I + \langle a \rangle \subseteq P_i$ for all $i = 1, \dots, n$ 
        and $I + \langle b \rangle \subseteq Q_j$ for all $j = 1, \dots, m$;
        \item $P_1 P_2 \cdots P_n \subseteq I + \langle a \rangle$ 
        and $Q_1 Q_2 \cdots Q_m \subseteq I + \langle b \rangle$. 
    \end{enumerate}
    But then we have $I \subseteq I + \langle a \rangle \subseteq P_i$ 
    and $I \subseteq I + \langle b \rangle \subseteq Q_j$. Moreover, 
    we see that 
    \[ P_1 \cdots P_n Q_1 \cdots Q_m \subseteq (I + \langle a \rangle) 
    (I + \langle b \rangle) \subseteq I. \] 
    Here, we find that $I \notin X$, which is a contradiction. \qed 
\end{pf}\vspace{-0.25cm}

We now introduce some familiar ring theory from PMATH 347, namely the 
notion of coprime ideals and the (generalized) Chinese remainder theorem. 

\begin{defn}{defn:3.3}
    Let $R$ be a ring, and let $I$ and $J$ be proper ideals of $R$. We say 
    that $I$ and $J$ are {\bf coprime} if $I + J = R$.
\end{defn}\vspace{-0.25cm}

The term coprime is used interchangeably with comaximal, but being a 
number theory course, it feels more appropriate to say coprime.
The following proposition tells us that powers of coprime ideals are also 
coprime. 

\begin{prop}{prop:3.4}
    Let $R$ be a ring. Let $I$ and $J$ be proper ideals such that $I + J = R$. 
    Then for all $n, m \in \N$, we have $I^n + J^m = R$. 
\end{prop}\vspace{-0.25cm}
\begin{pf}[Proposition~\ref{prop:3.4}]
    Suppose that $I^n + J^m \neq R$. Then we have $I^n + J^m \subseteq M$ 
    for some maximal ideal $M$. This gives us $I^n \subseteq M$ and 
    $J^m \subseteq M$ as well. But maximal ideals are prime. 
    In particular, if $a \in I$, then $a^n \in M$, but since $M$ is 
    prime, we also have $a \in M$. Thus, we have $I \subseteq M$, 
    and by an identical argument, we obtain $J \subseteq M$. (This result 
    also follows from another characterization of prime ideals.)
    It follows that $I + J \subseteq M$, which is a contradiction since 
    maximal ideals are proper by definition. \qed 
\end{pf}\vspace{-0.25cm}

This leads us to the following famous theorem that we all know and love. 

\begin{theo}[Chinese Remainder Theorem]{theo:3.5}
    Let $R$ be a ring and let $I$ and $J$ be coprime ideals of $R$. Then 
    \[ R/IJ \cong R/I \times R/J. \]
\end{theo}\vspace{-0.25cm}
\newpage
\begin{pf}[Theorem~\ref{theo:3.5}]
    Define the map $\varphi : R \to R/I \times R/J$ by $\varphi(x) 
    = (x+I, x+J)$. Then we have $\ker\varphi = I \cap J$. But $IJ 
    \subseteq I+J$ always holds and since $I$ and $J$ are coprime, we obtain 
    \begin{align*}
        I \cap J &= (I \cap J)R \\ 
        &= (I \cap J)(I + J) \\ 
        &= (I \cap J)I + (I \cap J)J \subseteq IJ
    \end{align*}
    since $I \cap J \subseteq I$, $I \cap J \subseteq J$, and $R$ is 
    commutative. Thus, we have $\ker\varphi = I \cap J = IJ$. To see that 
    $\varphi$ is surjective, let $a \in I$ and $b \in J$ such that $a + b = 1$
    (which exist because $I$ and $J$ are coprime). Then for $x, y \in R$, we have 
    \begin{align*}
        \varphi(ax + by) &= (ax + by + I, ax + by + J) \\ 
        &= (by + I, ax + J) \\ 
        &= (b + I, a + J)(y + I, x + J) \\ 
        &= (1 + I, 1 + J)(y + I, x + J) \\ 
        &= (y + I, x + J),
    \end{align*}
    where the second last equality follows from looking at $b$ modulo $I$ 
    and $a$ modulo $J$. So $\varphi$ is surjective, and it follows from 
    the first isomorphism theorem that $R/IJ \cong R/I \times R/J$. \qed
\end{pf}\vspace{-0.25cm}

The generalized Chinese remainder theorem then follows from a straightforward 
induction.

\begin{theo}[Generalized Chinese Remainder Theorem]{theo:3.6}
    Let $R$ be a ring, and let $I_1, \dots, I_n$ be pairwise coprime ideals 
    (i.e. $I_i + I_j = R$ when $i \neq j$). Then 
    \[ R/I_1 \cdots I_n \cong R/I_1 \times \cdots \times R/I_n. \] 
\end{theo}\vspace{-0.25cm}

We end this section on ring theory with a property of finite rings. 

\begin{prop}{prop:3.7}
    Let $R$ be a finite ring. There exist distinct prime ideals $P_1, \dots, P_m$ 
    of $R$ and $n_i \in \N$ such that 
    \[ R \cong R/P_1^{n_1} \times \cdots \times R/P_m^{n_m}. \] 
\end{prop}\vspace{-0.25cm}

Note that if $R$ is an integral domain, then this proposition tells us 
almost nothing as we can simply take $P_1 = \{0\}$. This is much 
more interesting when we are not working with integral domains.

\begin{pf}[Proposition~\ref{prop:3.7}]
    We can find prime ideals $Q_1, \dots Q_k$ of $R$ such that 
    $Q_1Q_2 \cdots Q_k = \{0\}$ by using Proposition~\ref{prop:3.2}
    with $I = \{0\}$ and noting that finite rings are Noetherian. 
    Grouping the $Q_i$'s with multiplicity, we can write 
    \[ P_1^{n_1} \cdots P_m^{n_m} = \{0\} \] 
    where $P_i \neq P_j$ for $i \neq j$. Note that each $P_i$ is prime, 
    so $R/P_i$ is a finite integral domain and hence a field. This 
    means that $P_i$ is also maximal. So we have $P_i + P_j = R$ 
    for $i \neq j$ because these are each independently maximal and distinct,
    and hence any bigger ideal must be the whole ring. By Proposition~\ref{prop:3.4},
    we obtain $P_i^{n_i} + P_j^{n_j} = R$, and the generalized Chinese 
    remainder theorem (Theorem~\ref{theo:3.6}) gives 
    \[ R \cong R/P_1^{n_1} \cdots P_m^{n_m} \cong R/P_1^{n_1} \times 
    \cdots \times R/P_n^{n_m}. \eqno\qed \]
\end{pf}

\subsection{Prime Ideals of the Ring of Integers} \label{subsec:3.2}
Let $K$ be a number field of degree $[K : \Q] = n$, and let $R = {\cal O}_K$.
Let $I$ be a nonzero proper ideal of $R$. We know the following facts: 
\begin{enumerate}[(1)]
    \item Corollary~\ref{cor:1.19}: $R/I$ is finite, which allows us to 
    apply Proposition~\ref{prop:3.7}.
    \item Corollary~\ref{cor:1.21}: $R$ is Noetherian, so we can apply 
    Proposition~\ref{prop:3.2}.
    \item {\bf Correspondence theorem for rings.} There is a one-to-one
    correspondence between the ideals of $R$ that contain $I$ and the 
    ideals of the quotient ring $R/I$. In other words, every ideal $\overline{J}$ 
    of $R/I$ is of the form $\overline{J} = J/I$, where $J \subseteq R$ 
    is an ideal such that $I \subseteq J$. Moreover, $\overline{J}$ 
    is prime if and only if $J$ is prime. 
    \item Since $R/I$ is finite, we have by (3), Proposition~\ref{prop:3.7},
    and the third isomorphism theorem that 
    \begin{align*}
        R/I &\cong (R/I)/(P_1^{n_1}/I) \times \cdots \times (R/I)/(P_m^{n_m}/I) \\ 
        &\cong R/P_1^{n_1} \times \cdots \times R/P_m^{n_m} 
    \end{align*}
    where each $P_i \subseteq R$ is prime and $I \subseteq P_i$. 
\end{enumerate}
Therefore, to understand the ideal $I$, we study the prime ideals $P \supseteq I$. 
It turns out that $P \supseteq I$ if and only if $P$ is a ``prime factor'' of $I$,
as we'll see later in the course. \newpage
\section{Ideal Class Group}\label{sec:4}

\subsection{Preliminaries}\label{subsec:4.1}
Let $K$ be a number field and let $R = {\cal O}_K$. Recall from 
\ref{A2-3} that if $X$ is the set of nonzero ideals of $R$, then we can put 
an equivalence relation on $X$ by $I \sim J$ if and only if $\alpha I = \beta J$ 
for some nonzero $\alpha, \beta \in R$. Then 
\[ G_K := \{[I] : I \in X\} \] 
is a group under the operation $[I][J] = [IJ]$, called the {\bf ideal 
class group} of $K$. The identity element is the equivalence class 
of nonzero principal ideals. In fact, this construction works for 
all Dedekind domains because nonzero ideals are invertible due to 
Proposition~\ref{prop:3.13}.

\begin{defn}{defn:4.1}
    Let $K$ be a number field. The {\bf class number} of $K$ is defined to be 
    \[ \cl(K) = |G_K|. \] 
\end{defn}\vspace{-0.25cm}

The ideal class group $G_K$ is structural information that is attached to 
the number field $K$ or the ring of integers ${\cal O}_K$. The class 
number $\cl(K)$ is a measure (in terms of complexity) of how far away 
${\cal O}_K$ is from being a PID because we have that $\cl(K) = 1$ 
if and only if ${\cal O}_K$ is a PID. 

The next result tells us that $\cl(K)$ is also a measure of how far away
${\cal O}_K$ is from being a UFD, and so having unique prime factorization 
of elements!

\begin{prop}{prop:4.2}
    Let $R$ be a Dedekind domain. Then $R$ is a PID if and only if $R$ is a UFD.
\end{prop}\vspace{-0.25cm}
\begin{pf}[Proposition~\ref{prop:4.2}]
    $(\Rightarrow)$ This holds for any ring $R$. 

    $(\Leftarrow)$ Suppose that $R$ is a UFD and let $I$ be a nonzero proper ideal 
    of $R$. By Proposition~\ref{prop:3.13}, we can find an ideal $J$ of $R$ 
    such that $IJ = \langle \alpha \rangle$, where $\alpha \in I$ is nonzero. 
    But we can write $\alpha = p_1^{n_1} \cdots p_k^{n_k}$ for some 
    prime elements $p_i \in R$ and $n_i \in \N$. Then 
    \[ IJ = \langle \alpha \rangle = \langle p_1^{n_1} \cdots p_k^{n_k} \rangle 
    = \langle p_1 \rangle^{n_1} \cdots \langle p_k \rangle^{n_k}, \] 
    where each $\langle p_i \rangle$ is a prime ideal of $R$. It follows that 
    \[ I = \langle p_{i_1} \rangle^{m_1} \cdots \langle p_{i_\ell} \rangle^{m_\ell} 
    = \langle p_{i_1}^{m_1} \cdots p_{i_\ell}^{m_\ell} \rangle \] 
    for some $i_1, \dots, i_\ell \in \{1, \dots, k\}$, so $I$ 
    is principal and $R$ is a PID. \qed 
\end{pf}\vspace{-0.25cm}

At this point, it is not obvious that $\cl(K) < \infty$. Therefore, 
our next goal is to show that $G_K$ is a finite group. 

\begin{prop}{prop:4.3}
    Let $K$ be a number field and let $R = {\cal O}_K$. Then there exists
    $\lambda > 0$ such that for all nonzero ideals $I$ of $R$, there 
    is an nonzero element $\alpha \in I$ satisfying 
    \[ N(\langle \alpha \rangle) \leq \lambda N(I). \] 
\end{prop}\vspace{-0.25cm}

Note that we always have $N(I) \leq N(\langle \alpha \rangle)$ since 
$\langle \alpha \rangle \subseteq I$ and hence $I \mid \langle \alpha \rangle$. 
This proposition gives us an inequality in the other direction up to 
some $\lambda > 0$. 

\begin{pf}[Proposition~\ref{prop:4.3}]
    Let $n = [K : \Q]$, let $\{v_1, \dots, v_n\}$ be an integral basis for 
    ${\cal O}_K$, and let $\sigma_1, \dots, \sigma_n$ be the embeddings 
    of $K$ in $\C$. Pick $m \in \N$ such that $m^n \leq N(I) < (m+1)^n$. 
    Consider the elements of the form 
    \[ m_1 v_1 + \cdots + m_n v_n, \] 
    where $0 \leq m_i \leq m$ and $m_i \in \Z$. There are $(m+1)^n$ 
    such elements. Since $(m+1)^n > N(I) = |R/I|$, there exist 
    two such elements that are congruent modulo $I$. Subtracting these 
    yields a nonzero $\alpha \in I$ of the form 
    $\alpha = m_1 v_1 + \cdots + m_n v_n$
    where $0 \leq |m_i| \leq m$. Then we obtain 
    \begin{align*}
        N(\langle \alpha \rangle) = |N_{K/\Q}(\alpha)| 
        &= \left| \prod_{i=1}^n \sigma_i(\alpha) \right| 
        = \prod_{i=1}^n |\sigma_i(\alpha)| \\ 
        &\leq \prod_{i=1}^n \sum_{j=1}^n |m_j \sigma_i(v_j)| 
        \leq \prod_{i=1}^n \sum_{j=1}^n m|\sigma_i(v_j)| \\ 
        &= m^n \prod_{i=1}^n \sum_{j=1}^n |\sigma_i(v_j)| 
        \leq N(I) \prod_{i=1}^n \sum_{j=1}^n |\sigma_i(v_j)|.
    \end{align*}
    Taking $\lambda = \prod_{i=1}^n \sum_{j=1}^n |\sigma_i(v_j)|$ 
    gives the result. \qed 
\end{pf}\vspace{-0.25cm}
The next result tells us that every ideal class has a representative 
whose norm is bounded.

\begin{prop}{prop:4.4}
    Let $K$ be a number field and let $R = {\cal O}_K$. Let $\lambda > 0$ 
    be as in Proposition~\ref{prop:4.3}. For all nonzero ideals $I$ of $R$, 
    there exists an ideal $J$ of $R$ such that $[I] = [J]$ and $N(J) \leq \lambda$. 
\end{prop}\vspace{-0.25cm}
\begin{pf}[Proposition~\ref{prop:4.4}]
    Let $I$ be a nonzero ideal of $R$. Consider the ideal class $[I]^{-1} = 
    [I']$ represented by some ideal $I'$ of $R$. By Proposition~\ref{prop:3.13},
    we can find an ideal $J$ of $R$ such that $I'J = \langle \alpha \rangle$, 
    where $\alpha \in I$ is the nonzero element chosen as in 
    Proposition~\ref{prop:4.3}. Taking norms gives us 
    \[ N(I') N(J) = N(\langle \alpha \rangle) \leq \lambda N(I'), \] 
    and thus $N(J) \leq \lambda$. Moreover, we have $[I'][J] 
    = [\langle \alpha \rangle] = [\langle 1 \rangle]$, so 
    $[J] = [I']^{-1} = [I]$. \qed 
\end{pf}\vspace{-0.25cm}

This gives us the finiteness of the ideal class group. 

\begin{cor}{cor:4.5}
    Let $K$ be a number field. Then $G_K$ is finite. 
\end{cor}\vspace{-0.25cm}
\begin{pf}[Corollary~\ref{cor:4.5}]
    Let $[I] \in G_K$ be an ideal class. By Proposition~\ref{prop:4.4}, 
    we may assume that $N(I) \leq \lambda$. Suppose that the 
    prime factorization of $I$ is $I = P_1^{n_1} \cdots P_k^{n_k}$,
    where $N(P_i) = p_i^{f_i}$ for all $i = 1, \dots, k$. Since $p_i \in P_i$, 
    we have $P_i \mid \langle p_i \rangle$. Note that we must have 
    $p_i \leq \lambda$. Moreover, there are only finitely many 
    prime ideals $Q$ of $R$ such that $p \in Q$, where $p \in \N$ is prime 
    with $p \leq \lambda$. \qed 
\end{pf}\newpage 

\subsection{Minkowski's Bound} \label{subsec:4.2}
We know that the ideal class group is finite, but what now? The bound 
$\lambda$ we found has a couple of problems: 
\begin{enumerate}[(1)]
    \item It is difficult to compute; we require an integral basis
    for ${\cal O}_K$, the embeddings of $K$ in $\C$, and there are a large 
    amount of sums and products.
    \item In addition, $\lambda$ can be significantly larger than needed. 
\end{enumerate}
It turns out that there is a much better bound, obtained from 
our work on the geometry of numbers in Assignment 5. 

\begin{theo}[Minkowski]{theo:4.6}
    Let $K$ be a number field with $[K : \Q] = n$, and let $R = {\cal O}_K$.
    Let $s$ is the number of pairs of nonreal embeddings of $K$ in $\C$. 
    Then every ideal class of $R$ contains an ideal $I$ such that 
    \[ N(I) \leq \frac{n!}{n^n} \left( \frac{4}{\pi} \right)^{\!s} \lvert\disc(K)\rvert^{1/2}. \]
\end{theo}\vspace{-0.25cm}

The main thing to notice is that $n!/n^n$ approaches $0$ extremely quickly, 
so this bound tends to be small. This helps us narrow down the 
primes to look at. 

Let's take a look at an example. Let $K = \Q(\alpha)$, where $\alpha$ 
is a root of the $2$-Eisenstein polynomial $f(x) = x^3 - 2x - 2$. 
with discriminant $-76 = -2^2 \cdot 19$. By \ref{A6-4}, we have 
${\cal O}_K = \Z[\alpha]$. 

Since $\disc(f(x)) = -76 < 0$, we see that $f(x)$ must have a pair of 
nonreal roots, because if they were all real, then $\disc(f(x))$ would 
be positive. But $\deg(f(x)) = 3$, so this gives us $s = 1$. 
Then the Minkowski bound is 
\[ B_K = \frac{3!}{3^3} \left( \frac{4}{\pi} \right)^{\!1} \sqrt{76} \approx 2.467. \] 
By Theorem~\ref{theo:4.6}, we have $G_K = \{[I] : N(I) \leq 2\}$.
It follows that $G_K$ is generated by the prime ideals $P$ such that $N(P) = 2$.
But working modulo $2$, we have $f(x) = x^3 \in \Z_2[x]$. Then Kummer-Dedekind 
gives us $\langle 2 \rangle = \langle \alpha, 2 \rangle^3 = \langle \alpha \rangle^3$, 
where $2 = \alpha^3 - 2\alpha \in \langle \alpha \rangle$. Therefore, 
we have $G_K = \{[\langle 1 \rangle], [\langle \alpha \rangle]\} = 
\{[\langle 1 \rangle]\}$ and $\cl(K) = 1$, which shows that ${\cal O}_K$ 
is a PID. \newpage
\appendix 
\section{Assignment Problems}
Sometimes, we'll use facts that we cover on the assignments, 
so we list the problems here for reference. 

{\bf Assignment 1.} 
\begin{enumerate}[leftmargin=1.5cm, label={{\bf A1-\arabic*}}]
    \item \label{A1-1} Let $K$ be a number field. \begin{enumerate}[(a)]
        \item Let $\alpha \in K$. Suppose that $\alpha$ is a root of 
        \[ f(x) = a_n x^n + a_{n-1} x^{n-1} + \cdots + a_1 x + a_0 \in \Z[x]. \] 
        Prove that $a_n \alpha \in {\cal O}_K$. 
        \item Prove that there exists a basis for 
        $K$ over $\Q$ consisting entirely of algebraic integers. 
    \end{enumerate} 
    \item \label{A1-2} Let $K = \Q(\sqrt{3}, \sqrt{7})$ and let 
    \[ \Z[\sqrt{3}, \sqrt{7}] = \{a + b\sqrt{3} + c\sqrt{7} + d\sqrt{21} : 
    a, b, c, d \in \Z\}. \] 
    Prove that ${\cal O}_K \neq \Z[\sqrt{3}, \sqrt{7}]$. 
    \item \label{A1-3} Let $R$ and $S$ be integral domains, where $R$ is a subring of $S$. 
    Suppose $S$ is integral over $R$. That is, assume that every element of 
    $S$ is integral over $R$. 
    \begin{enumerate}[(a)]
        \item Prove that $R$ is a field if and only if $S$ is a field. 
        \item Let $Q$ be a prime ideal of $S$ and let $P = Q \cap R$. 
        Prove that $P$ is a prime ideal of $R$, and that $P$ is maximal 
        if and only if $Q$ is maximal.
    \end{enumerate}
    \item \label{A1-4} Let $L/K$ be a finite extension of number fields. 
    Prove that every embedding (injective ring homomorphism) $\varphi : 
    K \to \C$ can be extended to exactly $[L : K]$ embeddings $\psi : L \to \C$.
\end{enumerate}

{\bf Assignment 2.} 
\begin{enumerate}[leftmargin=1.5cm, label={{\bf A2-\arabic*}}]
    \item \label{A2-1} Let $K = \Q(\sqrt{-5})$. \begin{enumerate}[(a)]
        \item Suppose that $\{a, b\} \subseteq {\cal O}_K$ is an integral basis 
        for ${\cal O}_K$. Prove that we must have 
        \[ \det \begin{bmatrix} a & \bar{a} \\ b & \bar{b} \end{bmatrix}^2 = -20. \]
        Here, we mean $\overline{x + y\sqrt{-5}} = x - y\sqrt{-5}$ for $x, y \in \Q$. 
        \item Suppose that $a, b \in {\cal O}_K$ satisfy 
        \[ \det \begin{bmatrix} a & \bar{a} \\ b & \bar{b} \end{bmatrix}^2 = -20. \] 
        Prove that $\{a, b\}$ is an integral basis for ${\cal O}_K$.  
    \end{enumerate} 
    \item \label{A2-2} Let $\alpha \in \C$ such that $\alpha^4 + 3\alpha^2 + 
    6\alpha - 3 = 0$. \begin{enumerate}[(a)]
        \item Compute $\Tr_{\Q(\alpha)/\Q}(\alpha)$. 
        \item Compute $\Tr_{\Q(\alpha)/\Q}(\alpha^4 + \alpha + 2)$.
    \end{enumerate}
    \item \label{A2-3} Let $R$ be an integral domain and let $I$ and $J$ be 
    ideals of $R$. Recall that 
    \[ IJ := \{a_1b_1 + \cdots + a_nb_n : n \in \N,\, a_i \in I,\, b_i \in J\} \] 
    is also an ideal of $R$. Now, let $X$ be the set of nonzero ideals of $R$. 
    \begin{enumerate}[(a)]
        \item Put a relation $\sim$ on $X$ by $I \sim J$ if and only if 
        $\langle \alpha \rangle I = \langle \beta \rangle J$ for some 
        nonzero $\alpha, \beta \in R$. Prove that $\sim$ is an equivalence 
        relation on $X$. 
        \item Prove that $I, J \in X$ are isomorphic as $R$-modules if and only if 
        $I \sim J$. 
        \item Prove that if $I \in X$ and there exists a nonzero $\alpha \in R$ 
        such that $\langle \alpha \rangle I$ is principal, then $I$ 
        itself is principal. What does this tell you about the principal 
        ideals of $R$ relative to $\sim$?
        \item Prove that the set of ideal classes (with respect to $\sim$) form a 
        group under the operation $[I][J] = [IJ]$ if and only if for all 
        $I \in X$, there exists $J \in X$ such that $IJ$ is principal.
    \end{enumerate} 
    \item \label{A2-4} Let $M = \Z^n$ and let $N$ be a submodule of $M$ such that $\rank(N) = n$. 
    \begin{enumerate}[(a)]
        \item Prove that $M/N$ is finite. 
        \item Read Theorem 2.10 of Keith Conrad's notes, which also 
        requires Definition 2.8. If $\{e_1, \dots, e_n\}$ is 
        the standard integral basis for $M$, this guarantees the existence of a 
        basis $\{d_1e_1, \dots, d_ne_n\}$ for $N$, where $d_i \in \Z$. 
        \item Prove that $M/N \cong \Z_{d_1} \times \cdots \times \Z_{d_n}$. 
        \item Let $\{v_1, \dots, v_n\}$ be an integral basis for $N$. 
        Since $N \subseteq M$, we can find $a_{ij} \in \Z$ such that 
        $v_i = \sum_{j=1}^n a_{ij} e_j$. Use this to construct the matrix 
        $A = [a_{ij}] \in M_n(\Z)$. Prove that $[M : N] = \lvert\det(A)\rvert$, where 
        $[M : N]$ is the index of the subgroup $N$ in $M$.
    \end{enumerate}
\end{enumerate}

{\bf Assignment 3.} 
\begin{enumerate}[leftmargin=1.5cm, label={{\bf A3-\arabic*}}]
    \item \label{A3-1} Let $\alpha \in \C$ be a root of $f(x) = x^4 - x^3 
    + 2x^2 - x + 2$, and let $K = \Q(\alpha)$. Prove that ${\cal O}_K 
    = \Z[\alpha]$. Note that $f(x)$ is irreducible by the rational roots 
    theorem.
    \item \label{A3-2} Let $\alpha \in \C$ be a root of $f(x) = x^3 + 18x - 26$, 
    and let $K = \Q(\alpha)$. Note that $f(x)$ is irreducible by 
    Eisenstein with $p = 2$. 
    \begin{enumerate}[(a)]
        \item Prove that $\beta = \frac{\alpha^2-\alpha+1}{3} \in {\cal O}_K$. 
        \item Compute $\disc(1, \alpha, \beta)$. 
        \item Prove that $[{\cal O}_K : \Z[\alpha]] \in \{3, 6\}$. 
    \end{enumerate}
    \item \label{A3-3} Let $\alpha \in \C$ be a root of $f(x) = x^3 - 
    x^2 - 2x - 8$, and let $K = \Q(\alpha)$. Note that $f(x)$ is irreducible 
    by the rational roots theorem. 
    \begin{enumerate}[(a)]
        \item Prove that $\beta = \frac{4}{\alpha} \in {\cal O}_K$. 
        \item Compute $\Tr_{K/\Q}(\alpha)$, $\Tr_{K/\Q}(\beta)$, 
        $\Tr_{K/\Q}(\alpha^2)$, $\Tr_{K/\Q}(\beta^2)$, and $\Tr_{K/\Q}(\alpha\beta)$. 
        \item Compute $\disc(1, \alpha, \beta)$. 
        \item Prove that $\{1, \alpha, \beta\}$ is an integral basis for ${\cal O}_K$.        
    \end{enumerate}
    \item \label{A3-4} Let $\alpha \in \C$ be a root of $f(x) = x^3 - 
    x^2 - 2x - 8$ as before, and let $K = \Q(\alpha)$. Prove that there 
    does not exist $\theta \in {\cal O}_K$ such that $\{1, \theta, \theta^2\}$ 
    is an integral basis for ${\cal O}_K$. 
    
    {\bf Hint:} By \ref{A3-3}, we know that $\{1, \alpha, \beta\}$ is an 
    integral basis for ${\cal O}_K$, so we have $\theta = a + b\alpha + c\beta$
    and $\theta^2 = A + B\alpha + C\beta$ for some $a, b, c, A, B, C \in \Z$. 
    Try to write $A, B, C$ in terms of $a, b, c$.
\end{enumerate}

{\bf Assignment 4.} 
\begin{enumerate}[leftmargin=1.5cm, label={{\bf A4-\arabic*}}]
    \item \label{A4-1} \begin{enumerate}[(a)]
        \item Prove that every UFD is integrally closed.
        \item Prove that every PID is a Dedekind domain.
        \item Give an example of a UFD that is not a Dedekind domain.
    \end{enumerate}
    \item \label{A4-2} Let $K = \Q(\alpha)$ where $\alpha$ is a root of 
    $f(x) = x^3 - x^2 - 3$. Note that $\disc(f(x)) = -255$. Factor 
    $\langle p \rangle$ as a product of prime ideals in ${\cal O}_K$ 
    for all $p \in \{2, 3, 5, 7\}$. Which of these primes divide $255$? 
    How does this affect the corresponding prime factorization?
    \item \label{A4-3} Suppose that $[K : \Q] = n$ and let $P$ be a nonzero prime 
    ideal of ${\cal O}_K$. Prove that $P$ contains a prime number $p \in \N$ 
    and that ${\cal O}_K/P$ contains at most $p^n$ elements. 
    \item \label{A4-4} Let $\alpha$ be a root of $f(x) = x^3 - 2x - 3$. 
    \begin{enumerate}[(a)]
        \item Prove that $R := \Z[\alpha]$ is a Dedekind domain.
        \item Let $I = \langle \alpha - 5 \rangle \subseteq R$. Prove that $|R/I| = 112$. 
        \item Factor $\langle 2 \rangle$ as a product of prime ideals. 
        \item Factor $\langle 7 \rangle$ as a product of prime ideals. 
        \item Factor $I$ as a product of prime ideals. 
    \end{enumerate}
\end{enumerate}

{\bf Assignment 5.} This assignment was a guided reading exercise on the geometry 
of numbers, following Marcus' \emph{Number Fields} pages 93 to 99.
\begin{enumerate}[leftmargin=1.5cm, label={{\bf A5-\arabic*}}]
    \item \label{A5-1} A {\bf lattice} in $\R^n$ is defined to be a subset 
    $L$ of $\R^n$ of the form $L = \Span_{\Z}\{v_1, \dots, v_n\}$, where 
    $\{v_1, \dots, v_n\}$ is a basis for $\R^n$ over $\R$. We define the 
    {\bf volume} of $L$ by 
    \[ \Vol(L) = \lvert\det\,[v_1|\cdots|v_n]\rvert. \] 
    \begin{enumerate}[(a)]
        \item Prove that the definition of the volume of a lattice is 
        independent of the choice of basis. 
        \item Let $M \subseteq L$ be lattices in $\R^n$. Prove that 
        \[ \Vol(M) = [L : M] \cdot \Vol(L). \] 
    \end{enumerate}
    \item \label{A5-2} Let $K$ be a number field and let $R = {\cal O}_K$. 
    Suppose that $\sigma_1, \dots, \sigma_r$ are the real-valued embeddings 
    of $K$ in $\C$ and that $\tau_1, \dots, \tau_{2s}$ are the non-real-valued 
    (i.e. complex) embeddings of $K$ in $\C$. Assume that $\tau_{i+1} = 
    \overline{\tau_i}$ for all odd $i$. Consider the {\bf Minkowski map} $\psi : 
    K \to \R^n$ given by 
    \[ \psi(\alpha) = (\sigma_1(\alpha), \dots, \sigma_r(\alpha), 
    \Re(\tau_1(\alpha)), \Im(\tau_1(\alpha)), \Re(\tau_3(\alpha)), 
    \Im(\tau_3(\alpha)), \dots). \]
    \begin{enumerate}[(a)]
        \item Prove that $\psi(R)$ is a lattice in $\R^n$. We will denote 
        this lattice by $M_K$ and we call it the {\bf Minkowski lattice} of $K$.
        \item Prove that 
        \[ \Vol(M_K) = \frac{1}{2^s} \lvert\disc(K)\rvert^{1/2}. \] 
        \item Let $I$ be a nonzero ideal of $R$ so that $L_I := \psi(I)$ 
        is a sublattice of $M_K$. Prove that 
        \[ \Vol(L_I) = \frac{1}{2^s} \lvert\disc(K)\rvert^{1/2} N(I). \] 
    \end{enumerate}
    \item \label{A5-3} We now take a look at Minkowski's lemma. Let 
    $L$ be a lattice in $\R^n$ and let $E \subseteq \R^n$ be such that 
    \begin{enumerate}[(1)]
        \item $E$ is convex (for all $a, b \in E$ and $t \in [0, 1]$, we have 
        $(1-t)a + tb \in E$); 
        \item $E$ is Lebesgue measurable; 
        \item if $a \in E$, then $-a \in E$.
    \end{enumerate}
    Furthermore, assume that $E$ is large; more precisely, assume that 
    $m(E) > 2^n \Vol(L)$ where $m(E)$ denotes the Lebesgue measure of $E$. 
    It turns out that $E$ is so big that it must contain a nonzero vector from $L$.
    If $E$ is compact, then the strict inequality can be weakened to $\geq$.
    \begin{enumerate}[(a)]
        \item Read about what Lebesgue measure is from page 96 of Marcus. 
        \item Read the proof of Minkowski's lemma from pages 96 and 97 of Marcus.
    \end{enumerate}
    \item \label{A5-4} Let $K$, $R$, $r$, and $s$ be as in \ref{A5-2}. 
    For $x = (x_1, \dots, x_n) \in \R^n$, let 
    \[ N(x) = x_1 x_2 \cdots x_r (x_{r+1}^2 + x_{r+2}^2) \cdots (x_{n-1}^2 + x_n^2). \] 
    \begin{enumerate}[(a)]
        \item Prove that if $\alpha \in R$ and $x = \psi(\alpha)$, then 
        $N(x) = N_{K/\Q}(\alpha)$. 
        \item Let $A \subseteq \R^n$ be compact and satisfy conditions (1), 
        (2), and (3) from \ref{A5-3}, where $A$ has positive Lebesgue measure. 
        Additionally, assume that $|N(a)| \leq 1$ for all $a \in A$. Prove that 
        if $L$ is a lattice in $\R^n$, then there exists a nonzero vector $x \in L$ 
        with 
        \[ |N(x)| \leq \frac{2^n}{m(A)} \Vol(L). \] 
        {\bf Hint:} You can freely use the fact that if $A \subseteq \R^n$ is 
        Lebesgue measurable and $t \in \R$, then $tA$ is Lebesgue measurable 
        with $m(tA) = |t|^n m(A)$.
    \end{enumerate}
    \item \label{A5-5} Let $K$, $R$, $r$, and $s$ be as in \ref{A5-2}. 
    Consider the set 
    \[ A = \left\{ x \in \R^n : |x_1| + \cdots + |x_r| + 2 
    \left( \sqrt{ x_{r+1}^2 + x_{r+2}^2 } + \cdots + \sqrt{ x_{n-1}^2 + x_n^2 } 
    \right) \leq n \right\}. \] 
    It can be shown that $A$ satisfies the hypotheses from part (b) of \ref{A5-4}.
    \begin{enumerate}[(a)]
        \item From pages 98 and 99 of Marcus, read the integration-based argument that 
        \[ m(A) = \frac{n^n}{n!} 2^r \left( \frac{\pi}{2} \right)^s. \] 
        \item Prove that every lattice $L$ in $\R^n$ contains a nonzero vector $x \in L$ 
        such that 
        \[ |N(x)| \leq \frac{n!}{n^n} \left( \frac{8}{\pi} \right)^s \Vol(L). \] 
        \item Prove that every ideal class of $R$ (as per \ref{A2-3}) contains 
        an ideal $I$ such that 
        \[ N(I) \leq \frac{n!}{n^n} \left( \frac{4}{\pi} \right)^s \lvert\disc(K)\rvert^{1/2}. \] 
    \end{enumerate} 
\end{enumerate}

{\bf Assignment 6.} 
\begin{enumerate}[leftmargin=1.5cm, label={{\bf A6-\arabic*}}]
    \item \label{A6-1} Let $\alpha \in \C$ be an algebraic integer and let 
    $S = \Z[\alpha]$. Suppose that $f(x) \in \Z[x]$ is the minimal polynomial of 
    $\alpha$ and let $p \in \N$ be prime. Suppose the irreducible 
    factorization of $f(x) \in \Z_p[x]$ is 
    \[ f(x) = p_1(x)^{n_1} p_2(x)^{n_2} \cdots p_k(x)^{n_k} \in \Z_p[x]. \] 
    Prove that the prime ideals of $S$ containing $p$ are exactly 
    \[ P_i = \langle p_i(\alpha), p \rangle. \] 
    {\bf Hint:} Retrace isomorphisms. 
    \item \label{A6-2} Let $S = \Z[\sqrt{5}]$ and let $P = \langle \sqrt{5}+1, 2 \rangle$. 
    \begin{enumerate}[(a)]
        \item Prove that $P$ is the only prime ideal of $S$ containing $2$. 
        \item Prove that $S_P$ is not a DVR. 
        \item Prove that $\langle 2 \rangle \neq P^2$. Why does this not 
        contradict our prime factorization fact (Theorem~\ref{theo:3.8})?
    \end{enumerate}
    \item \label{A6-3} \begin{enumerate}[(a)]
        \item Let $\alpha \in \C$ be a root of the irreducible polynomial 
        $f(x) = x^3 - x^2 - 8x - 18$, whose discriminant is 
        $-9300 = -2^2 \cdot 3 \cdot 5^2 \cdot 31$. Let $K = \Q(\alpha)$, 
        let $R = {\cal O}_K$, and let $S = \Z[\alpha]$. Use the DVR 
        characterization (Theorem~\ref{theo:3.31}) to prove that $R = S$. 
        \item Let $\alpha \in \C$ be a root of the irreducible polynomial 
        $f(x) = x^3 - 6x - 16$, whose discriminant is 
        $-6048 = -2^5 \cdot 3^3 \cdot 7$. Let $K = \Q(\alpha)$, 
        let $R = {\cal O}_K$, and let $S = \Z[\alpha]$. Use the DVR 
        characterization (Theorem~\ref{theo:3.31}) to prove that $R \neq S$. 
    \end{enumerate}
    \item \label{A6-4} Let $p \in \N$ be a prime number and let $\alpha \in \C$ 
    be the root of a $p$-Eisenstein polynomial (i.e. a monic, integer polynomial 
    which is irreducible by $p$-Eisenstein). Let $R = {\cal O}_{\Q(\alpha)}$ 
    and $S = \Z[\alpha]$. 
    \begin{enumerate}[(a)]
        \item Prove that there is exactly one prime ideal $P$ of $S$ that contains $p$. 
        \item Let $P$ be the prime ideal from part (a). Prove that $S_P$ is a DVR 
        and prove that $\alpha$ is a uniformizer for $S_P$. 
    \end{enumerate}
\end{enumerate}

{\bf Assignment 7.} 
\begin{enumerate}[leftmargin=1.5cm, label={{\bf A7-\arabic*}}]
    \item \label{A7-1} In this problem, we explore where the ideal class 
    equivalence relation comes from. Let $R$ be a Dedekind domain with 
    field of fractions $K = \Frac(R)$. A {\bf fractional ideal} of 
    $K$ is any set of the form $\alpha I$, where $\alpha \in K^\times$ and 
    $I$ is a nonzero ideal of $R$. We define the product of two fractional 
    ideals by $(\alpha I)(\beta J) = \alpha\beta IJ$. Let $G$ denote 
    the set of all fractional ideals of $K$. 
    \begin{enumerate}[(a)]
        \item List three results from lecture whose proof has used 
        fractional ideals. 
        \item Prove that the multiplication of fractional ideals is well-defined. 
        \item Prove that $G$ forms a group under the above multiplication.
        \item Prove that every proper fractional ideal can be uniquely written 
        in the form $P_1^{n_1} \cdots P_k^{n_k}$, where each $P_i$ is a nonzero 
        prime ideal of $R$ and $n_i \in \Z$. 
        \item Let $H$ denote the set of fractional ideals of the form 
        $\alpha R$ where $\alpha \in K^\times$. Prove that $H$ is a subgroup of $G$. 
        \item Let $R = {\cal O}_K$ where $K$ is a number field. Prove that 
        $G/H$ is isomorphic to the ideal class group of $K$.
    \end{enumerate}
    \item \label{A7-2} Let $K = \Q(\alpha)$ be a number field with $\alpha \in {\cal O}_K$. 
    Let $R = {\cal O}_K$, let $S = \Z[\alpha]$, let $f(x) \in \Z[x]$ be the minimal 
    polynomial of $\alpha$, and let $p \in \N$ be a prime such that 
    $p \nmid \disc(\alpha)$. Prove that $f(x) \in \Z_p[x]$ can be written 
    as a product of distinct irreducible polynomials in $\Z_p[x]$.
    \item \label{A7-3} Let $\alpha \in \C$ be a root of $f(x) = x^3 - 2x + 5$. 
    How many nonzero ideals of $\Z[\alpha]$ have norm $8$ or less?
    \item \label{A7-4} Let $\alpha \in \C$ be a root of the irreducible 
    polynomial $f(x) = x^4 - 2x^3 + 3x^2 - 1$, whose discriminant is $-976$. 
    Let $K = \Q(\alpha)$. 
    \begin{enumerate}[(a)]
        \item Prove that ${\cal O}_K = \Z[\alpha]$. 
        \item Prove that ${\cal O}_K$ is a PID. 
    \end{enumerate}
\end{enumerate}\newpage

\end{document}
