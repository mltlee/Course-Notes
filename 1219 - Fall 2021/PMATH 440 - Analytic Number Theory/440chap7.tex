\section{$L$-functions and Dirichlet's Theorem}\label{sec:7}

\subsection{Some Results on Primes in Arithmetic Progressions}\label{subsec:7.1}
Let $k$ and $\ell$ be coprime positive integers. Recall that Dirichlet's theorem 
asserts that $kn + \ell$ is prime for infinitely many integers $n$. For many 
pairs $(k, \ell)$, this result can be proved using elementary means. However, 
this is hard to prove generally, and we'll require more tools to do so. 

For example, consider the pair $(k, \ell) = (4, 3)$. Suppose that there are only 
finitely many primes $p_1, \dots, p_k$ of the form $4n + 3$. Then $4p_1 
\cdots p_k + 3$ must be divisible by a prime of the form $4n + 3$, since a product 
of primes congruent to $1$ modulo $4$ can only yield numbers congruent to 
$1$ modulo $4$. Notice that such a prime cannot be any of $p_1, \dots, p_k$, 
which is a contradiction. 

The following result is Dirichlet's theorem in the case where $k$ is an 
arbitrary positive integer and $\ell = 1$. 

\begin{theo}{thm:7.1} 
    Let $n \in \Z^+$. There are infinitely many primes congruent to $1$ modulo $n$.
\end{theo}
\begin{pf}
    This proof is due to Birkhoff and Vandiver (1904). Let $a > 2$ be an integer, 
    and consider the $n$-th cyclotomic polynomial
    \[ \Phi_n(x) = \prod_{\substack{1\leq j \leq n \\ \gcd(j, n) = 1}} 
    (x - \zeta_n^j), \] 
    where $\zeta_n = e^{2\pi i/n}$. We know that $\Phi_n(x) \in \Z[x]$ is irreducible, 
    and $x^n - 1 = \prod_{d\mid n} \Phi_d(x)$. We now consider $\Phi_n(x)$ 
    evaluated at $a$. 

    {\sc Claim.} If $p$ is a prime dividing $\Phi_n(a)$, then $p \mid n$ or 
    $p \equiv 1 \pmod n$. 

    {\sc Proof of Claim.} Since $x^n - 1 = \prod_{d\mid n} \Phi_d(x)$, we have 
    $p \mid a^n - 1$. We consider two cases. 

    First, if $p \nmid a^d - 1$ for all proper divisors $d$ of $n$, then the order of 
    $a$ modulo $p$ must be $n$. By Fermat's little theorem (Corollary~\ref{cor:5.3}), 
    we have $n \mid (p-1)$, so $p \equiv 1 \pmod n$. 

    Suppose there is a proper divisor $d$ of $n$ such that $p \mid a^d - 1$. Since 
    $p \mid \Phi_n(a)$, we obtain $p \mid (a^n - 1)/(a^d - 1)$. Notice that 
    \[ a^n = \left( 1 + (a^d - 1) \right)^{n/d} = 1 + \frac{n}{d} (a^d - 1) 
    + \binom{n/d}{2} (a^d - 1)^2 + \binom{n/d}{3} (a^d - 1)^3 + \cdots, \] 
    so it follows that 
    \[ \frac{a^n - 1}{a^d - 1} = \frac{n}{d} + \binom{n/d}{2} (a^d - 1) 
    + \binom{n/d}{3} (a^d - 1)^2 + \cdots. \] 
    Since $p \mid (a^n-1)/(a^d-1)$ and $p \mid (a^d-1)$, we get $p \mid n/d$ as well. 
    Therefore, we have $p \mid n$ as required. \hfill$\blacksquare$ 

    Now, assume that there are only finitely many primes $p_1, \dots, p_k$ 
    congruent to $1$ modulo $n$. The $n$-th cyclotomic polynomial is of the form 
    \[ \Phi_n(x) = x^{\phi(n)} + \cdots \pm 1. \] 
    Let $m$ be an integer. We see that $\Phi_n(np_1 \cdots p_k m)$ is not divisible 
    by $p_i$ for $i = 1, \dots, k$ and is coprime with $n$. Notice that for sufficiently 
    large $m$, we have $\Phi_n(np_1 \cdots p_km) \geq 2$. In particular, 
    $\Phi_n(np_1 \cdots p_km)$ has a prime divisor congruent to $1$ modulo $n$ 
    which is not in the set $\{p_1, \dots, p_k\}$, which is a contradiction. 
\end{pf}

\newpage 
\subsection{Characters}\label{subsec:7.2}

\vspace{2ex}
\begin{defn}{def:7.2}
    Let $G$ be a finite abelian group. A {\bf character} of $G$ is a homomorphism 
    $\chi : G \to \C^*$. Note that the set of characters of $G$ forms a group 
    under the operation $(\chi_1 \cdot \chi_2)(g) = \chi_1(g) \chi_2(g)$. 
    We call this group the {\bf dual group} of $G$, and denote it by $\hat G$. 
    The identity of $\hat G$ is the character $\chi_0$ with $\chi_0(g) = 1$
    for all $g \in G$. We call $\chi_0$ the {\bf principal character}. 
\end{defn}

Notice that if $|G| = n$, then $g^n = e$ for all $g \in G$. In particular, we see that 
$(\chi(g))^n = \chi(g^n) = \chi(e) = 1$, so $\chi(g)$ is an $n$-th root of unity 
for all $g \in G$. 

\begin{theo}{thm:7.3}
    Let $G$ be a finite abelian group. 
    \begin{enumerate}[(1)]
        \item The order of $\hat G$ is equal to the order of $G$. 
        \item The dual group $\hat G$ is isomorphic to $G$. 
        \item We have the formulas 
        \begin{align*}
            \sum_{\chi\in\hat G} \chi(g) = \begin{cases}
                |G| & \text{if } g = e, \\ 
                0 & \text{otherwise;}
            \end{cases} \\ 
            \sum_{g\in G} \chi(g) = \begin{cases} 
                |G| & \text{if } \chi = \chi_0, \\ 
                0 & \text{otherwise.}
            \end{cases}
        \end{align*}
    \end{enumerate}
\end{theo}
\begin{pf}~
    \begin{enumerate}[(1)]
        \item Recall that a finite abelian group is the direct product of cyclic 
        groups. Hence, there exist elements $g_1, \dots, g_r \in G$ and 
        $h_1 \cdots h_r \in \N$ with $h_1 \cdots h_r = |G|$ such that 
        every element $g \in G$ has a unique representation $g = g_1^{a_1} 
        \cdots g_r^{a_r}$ with $1 \leq a_i \leq h_i$, and $g_i^{h_i} = e$ for 
        $i = 1, \dots, r$. 

        Any character $\chi$ is uniquely determined by its action on 
        $g_1, \dots, g_r$. Since $g_i^{h_i} = e$, we have $(\chi(g_i))^{h_1} = 1$, 
        which shows that $\chi(g_i)$ is an $h_i$-th root of unity. Hence, there are 
        at most $h_1 \cdots h_r$ characters. 

        On the other hand, there are at least $h_1 \cdots h_r$ characters because 
        if $\omega_i$ is an $h_i$-th root of unity, then we may define $\chi(g_i)
        = \omega_i$ for $i = 1, \dots, r$ and extend multiplicatively to $G$. 
        We conclude that $|\hat G| = |G|$. 

        \item For each $i = 1, \dots, r$, let $\chi_i$ be the character 
        which sends $g_i$ to $e^{2\pi i/h_i}$ and $g_j$ to $1$ when $j \neq i$. 
        Define $\phi : G \to \hat G$ by 
        \[ \phi(g_1^{a_1} \cdots g_r^{a_r}) = \chi_1^{a_1} \cdots \chi_r^{a_r}. \] 
        Notice that $\phi$ is a homomorphism. We see that $\phi$ is injective 
        because for $j = 1, \dots, r$, we have 
        \[ (\chi_1^{a_1} \cdots \chi_r^{a_r})(g_j) = e^{2\pi ia_j/h_j} \] 
        Since $G$ is finite and $|\hat G| = |G|$ by (1), 
        we see that $\phi$ is also surjective. Therefore, we have $\hat G \cong G$. 

        \item Let $S(g) = \sum_{\chi\in\hat G} \chi(g)$. Notice that 
        $\chi(e) = 1$ for all $\chi \in \hat G$, so we obtain $S(g) = |\hat G| = |G|$. 

        Assume now that $g \neq e$. Then there exists a character $\chi_1 \in \hat G$ 
        such that $\chi_1(g) \neq 1$. Now, we have 
        \[ S(g) = \sum_{\chi \in \hat G} \chi(g) = \sum_{\chi \in \hat G} (\chi_1\chi)(g) 
        = \chi_1(g) \sum_{\chi\in\hat G} \chi(g) = \chi_1(g) S(g). \] 
        Since $\chi_1(g) \neq 1$, we must have $S(g) = 0$. 
        \newpage 
        On the other hand, let $T(\chi) = \sum_{g \in G} \chi(g)$. Notice that $\chi_0(g) 
        = 1$ for all $g \in G$, so $T(\chi_0) = |G|$. 
        
        If $\chi \neq \chi_0$, then there exists $g_1 \in G$ such that $\chi(g_1) 
        \neq 1$. We have 
        \[ T(\chi) = \sum_{g \in G} \chi(g) = \sum_{g \in G} \chi(g_1g) = 
        \chi(g_1) \sum_{g \in G} \chi(g) = \chi(g_1) T(\chi). \] 
        Since $\chi(g_1) \neq 1$, it follows that $T(\chi) = 0$. \qedhere 
    \end{enumerate}
\end{pf}

\subsection{Dirichlet Characters}\label{subsec:7.3}

\vspace{2ex}
\begin{defn}{def:7.4}
    Let $k \in \Z^+$, and denote $(\Z/k\Z)^*$ by $G(k)$. Let $\chi$ be a character 
    on $G(k)$. We can associate $\chi$ with a map $\Z \to \C^*$, which we also 
    call $\chi$, by setting 
    \[ \chi(a) = \begin{cases} 
        \chi([a]) & \text{if } [a] \in G(k), \\ 
        0 & \text{if } [a] \notin G(k), 
    \end{cases} \] 
    where $[a]$ denotes the conjugacy class of $a$. We call $\chi$ a 
    {\bf character modulo $k$}. 
\end{defn}

\begin{theo}{thm:7.5}
    Let $k \in \Z^+$, and let $\chi$ be a character modulo $k$. 
    \begin{enumerate}[(1)]
        \item If $\gcd(n, k) = 1$, then $\chi(n)$ is a $\phi(k)$-th root of unity. 
        \item We have $\chi(mn) = \chi(m)\chi(n)$; that is, $\chi$ is completely 
        multiplicative. 
        \item We have $\chi(n+k) = n$ for all $n \in \Z$; that is, $\chi$ is 
        periodic with period $k$. 
        \item We have the formulas 
        \begin{align*}
            \sum_{n=1}^k \chi(n) &= \begin{cases}
                \phi(k) & \text{if } \chi = \chi_0, \\ 
                0 & \text{otherwise;}
            \end{cases} \\ 
            \sum_{\chi} \chi(n) &= \begin{cases} 
                \phi(k) & \text{if } n \equiv 1 \text{ (mod $k$)}, \\ 
                0 & \text{otherwise,}
            \end{cases}
        \end{align*}
        where the second sum runs through all characters modulo $k$. 
        \item Let $\overline\chi$ be the conjugate character of $\chi$ with $\overline\chi(n) 
        = \overline{\chi(n)}$ for all $n \in \Z$. Let $\chi'$ be a character modulo 
        $k$. Then we have the formulas 
        \begin{align*}
            \sum_{\chi\in\hat G(k)} \chi(n) \overline\chi(m) &= \begin{cases}
                \phi(k) & \text{if } n \equiv m \text{ (mod $k$) and } \gcd(n, k) = 1, \\ 
                0 & \text{otherwise;}
            \end{cases} \\ 
            \sum_{n=1}^k \chi(n) \chi'(n) &= \begin{cases} 
                \phi(k) & \text{if } \chi' = \overline\chi, \\ 
                0 & \text{otherwise.}
            \end{cases}
        \end{align*}
    \end{enumerate}
\end{theo}
\newpage 
\begin{pf}
    Properties (1) to (4) either follow from definitions or Theorem~\ref{thm:7.3}. 
    Note that $\overline\chi(m) \chi(m) = 1 = \chi(m^{-1}) \chi(m)$, 
    where $m^{-1}$ denotes the multiplicative inverse of $m$ in $G(k)$. Therefore, 
    we have $\overline\chi(m) = \chi(m^{-1})$. It follows that 
    \[ \sum_{\chi\in\hat G(k)} \chi(n) \overline\chi(m) 
    = \sum_{\chi\in\hat G(k)} \chi(n) \chi(m^{-1}) 
    = \sum_{\chi\in\hat G(k)} \chi(nm^{-1}). \] 
    By (4), the last sum is $\phi(k)$ if $nm^{-1} \equiv 1 \pmod k$, or equivalently 
    $n \equiv m \pmod k$, and the sum is $0$ otherwise. This gives us the first 
    equation in (5). Moreover, note that if $\chi' = \overline\chi$, then 
    $\chi\chi' = \chi_0$. Otherwise, $\chi\chi'$ is a non-principal character, 
    so the second equation in (5) follows from (4). 
\end{pf}

We now describe the group of characters modulo $k$. By the multiplicative property 
of characters, it is enough to discuss the characters modulo $p^a$ where $p$ is 
prime and $a \in \Z^+$. First, assume that $p$ is an odd prime. Let $g$ be a 
primitive root modulo $p^a$. If $n$ is coprime with $p$, then there is a unique 
integer $1 \leq \nu \leq \phi(p^a)$ such that $n \equiv g^\nu \pmod{p^a}$. 
For each integer $1 \leq b \leq \phi(p^a)$, we define the character $\chi^b$ by 
\[ \chi^b(a) = \exp\left( \frac{2\pi ib\nu}{\phi(p^a)} \right). \] 
In this way, we get $\phi(p^a)$ different characters modulo $p^a$, so this is the 
complete list. Now, let $k = 2^a$. If $a = 1$, then we simply have the principal 
character. For $a = 2$, we have the principal character together with the character 
$\chi_4$ given by 
\[ \chi_4(n) = \begin{cases}
    1 & \text{if } n \equiv 1 \text{ (mod $4$),} \\ 
    -1 & \text{if } n \equiv -1 \text{ (mod $4$),} \\ 
    0 & \text{otherwise.}
\end{cases} \] 
If $a \geq 3$, then $(\Z/2^a\Z)^*$ is not cyclic by Theorem~\ref{thm:6.13}. 
Moreover, we saw in the proof of that theorem that for any odd integer $n$, 
there is a unique pair of integers $(x, y)$ with $x \in \{0, 1\}$ and 
$y \in \{0, \dots, 2^{a-2}-1\}$ such that 
\[ n \equiv (-1)^x 5^y \pmod{2^a}. \] 
For $c, d \in \Z$ with $c \in \{0, 1\}$ and $d \in \{0, \dots, 2^{a-2}-1\}$, we define 
\[ \chi_{2^a}^{c,d}(n) = \begin{cases}
    \exp(\frac{2\pi icx}2 + \frac{2\pi idy}{2^{a-2}}) & \text{if } n \equiv 1 
    \text{ (mod $2$),} \\ 
    0 & \text{otherwise.} 
\end{cases} \] 
We obtain $\phi(2^a)$ different characters modulo $2^a$, giving us the complete list. 

\subsection{Dirichlet $L$-functions}\label{subsec:7.4}
The Riemann zeta function was a powerful tool for studying the prime counting 
function $\pi(x)$. This suggests that it might be helpful to introduce complex 
functions in order to understand the primes in arithmetic progressions. 

\begin{defn}{def:7.6}
    Let $k \in \Z^+$, and let $\chi$ be a character modulo $k$. For $\Re(s) > 1$, 
    we define the {\bf Dirichlet $L$-function} by 
    \[ L(s, \chi) = \sum_{n=1}^\infty \frac{\chi(n)}{n^s}. \] 
\end{defn}

As with the Riemann zeta function, we can establish the analytic continuation of 
$L(s, \chi)$ up to $\Re(s) > 0$. 

\begin{theo}{thm:7.7}
    The function $L(s, \chi)$ can be analytically continued to $\Re(s) > 0$ except 
    when $\chi$ is the principal character. If $\chi_0$ is the principal character 
    modulo $k$, then $L(s, \chi_0)$ can be analytically continued to $\Re(s) > 0$ 
    except at the point $s = 1$, where we have a simple pole with residue $\phi(k)/k$.  
\end{theo}
\begin{pf}
    Let $A(x) = \sum_{n\leq x} \chi(n)$. By (4) of Theorem~\ref{thm:7.5}, we see that 
    \[ A(x) = \begin{cases}
        \floor{\frac{x}{k}} \phi(k) + T(x) & \text{if } \chi = \chi_0, \\ 
        \floor{\frac{x}{k}} 0 + T(x) & \text{if } \chi \neq \chi_0,
    \end{cases} \]
    where $|T(x)| < \phi(k)$. It follows that 
    \[ A(x) = E(\chi) \frac{\phi(k)x}{k} + R(x), \] 
    with $|R(x)| < 2\phi(k)$ and 
    \[ E(\chi) = \begin{cases}
        1 & \text{if } \chi = \chi_0, \\ 
        0 & \text{if } \chi \neq \chi_0.
    \end{cases} \]
    Let $f(n) = 1/n^s$. By Abel's summation formula (Lemma~\ref{lemma:2.8}), we have 
    \begin{align*}
        \sum_{n\leq x} \frac{\chi(n)}{n^s} 
        &= \frac{A(x)}{x^s} + s \int_1^x \frac{A(u)}{u^{s+1}}\dd u \\ 
        &= E(\chi) \frac{\phi(k)}{k} \frac{1}{x^{s-1}} + \frac{R(x)}{x^s} 
        + sE(\chi) \frac{\phi(k)}{k} \left( \frac{-u^{-s+1}}{s-1} \bigg|_1^x \right)
        + s \int_1^x \frac{R(u)}{u^{s+1}}\dd u \\ 
        &= E(\chi) \frac{\phi(k)}{k} \left( x^{1-s} + \frac{s}{1-s} (x^{1-s} - 1) 
        \right) + \frac{R(x)}{x^s} + s \int_1^x \frac{R(u)}{u^{s+1}}\dd u.  
    \end{align*}
    We now consider two cases. 
    \begin{itemize}
        \item If $\chi \neq \chi_0$, then $E(\chi) = 0$. We see from above that 
        \[ \sum_{n\leq x} \frac{\chi(n)}{n^s} = \frac{R(x)}{x^s} + s \int_1^x 
        \frac{R(u)}{u^{s+1}}\dd u. \] 
        We have $|R(x)| < 2\phi(k)$, so by letting $x \to \infty$, we see that 
        \[ L(s, \chi) = s \int_1^\infty \frac{R(u)}{u^{s+1}}. \] 
        This integral converges for $\Re(s) > 0$, so $L(s, \chi)$ has an 
        analytic continuation to $\Re(s) > 0$. 
        \item Note that $E(\chi_0) = 1$. By the above equation, we have 
        \[ \sum_{n\leq x} \frac{\chi_0(n)}{n^s} = \frac{\phi(k)}{k} \left( x^{1-s} + 
        \frac{s}{1-s} (x^{1-s} - 1) \right) + \frac{R(x)}{x^s} + s \int_1^x 
        \frac{R(u)}{u^{s+1}}\dd u. \]
        Since $|R(x)| < 2\phi(k)$, letting $x \to \infty$ again gives 
        \[ L(s, \chi_0) = \frac{\phi(k)}{k} \frac{s}{s-1} + s \int_1^\infty 
        \frac{R(u)}{u^{s+1}}. \] 
        The integral converges for $\Re(s) > 0$, so $L(s, \chi_0)$ has an an analytic 
        continuation to $\Re(s) > 0$, except at the simple pole $s = 1$ with 
        residue $\phi(k)/k$. \qedhere 
    \end{itemize}
\end{pf}

\subsection{Dirichlet Series}\label{subsec:7.5}

\vspace{2ex}
\begin{defn}{def:7.8}  
    Let $\{\lambda_n\}_{n=1}^\infty$ be a strictly increasing sequence of 
    positive real numbers. A {\bf Dirichlet series} attached to 
    $\{\lambda_n\}_{n=1}^\infty$ is a series of the form 
    \[ \sum_{n=1}^\infty a_n e^{-\lambda_n z}, \] 
    where $\{a_n\}_{n=1}^\infty$ is a sequence of complex numbers and $z \in \C$. 
\end{defn}

\begin{theo}{thm:7.9} 
    If the Dirichlet series $\sum_{n=1}^\infty a_n e^{-\lambda_n z}$ converges 
    at $z = z_0$, then it converges uniformly for $\Re(z-z_0) \geq 0$ and 
    $\lvert\arg(z-z_0)\rvert \leq \alpha$ with $\alpha < \pi/2$. 
\end{theo}
\begin{pf}
    Without loss of generality, we may assume that $z_0 = 0$. Note that 
    $\sum_{n=1}^\infty a_n$ converges, so for any $\eps > 0$, there exists 
    $N = N(\eps) \in \N$ such that if $\ell, m > N$, then 
    \[ \left| \sum_{n=\ell}^m a_n \right| < \eps. \] 
    Defining $A_{\ell,m} = \sum_{n=\ell}^m a_n$ and taking the convention that 
    $A_{\ell,\ell-1} = 0$, we have 
    \begin{align*}
        \sum_{n=\ell}^\infty a_n e^{-\lambda_n z}
        &= \sum_{n=\ell}^m (A_{\ell,n} - A_{\ell,n-1}) e^{-\lambda_n z} \\ 
        &= \sum_{n=\ell}^{m-1} A_{\ell,n} (e^{-\lambda_n z} - e^{-\lambda_{n+1}z})
        + A_{\ell,m} e^{-\lambda_m z}. 
    \end{align*}
    For $\Re(z) \geq 0$, we see that 
    \[ \left| \sum_{n=\ell}^m a_n e^{-\lambda_n z} \right| 
    \leq \eps \left( \sum_{n=\ell}^{m-1} |e^{-\lambda_n z} - e^{-\lambda_{n+1}z}| 
    + 1 \right). \] 
    Note that 
    \[ e^{-\lambda_n z} - e^{-\lambda_{n+1}z} = z \int_{\lambda_n}^{\lambda_{n+1}} 
    e^{-tz}\dd t. \] 
    Moreover, for $z = x+iy \in \C$ with $x, y \in \R$, we have $|e^{-tz}| = 
    e^{-tx}$. Hence, we obtain 
    \begin{align*}
        |e^{-\lambda_n z} - e^{-\lambda_{n+1}z}|
        &\leq |z| \int_{\lambda_n}^{\lambda_{n+1}} e^{-tx}\dd t \\ 
        &\leq |z| \left( -\frac{e^{-tx}}{x} \bigg|_{\lambda_n}^{\lambda_{n+1}}
        \right) \\ 
        &= \frac{|z|}{x} (e^{-\lambda_n x} - e^{-\lambda_{n+1}x}). 
    \end{align*}
    It follows that 
    \begin{align*}
        \left| \sum_{n=\ell}^m a_n e^{-\lambda_n z} \right| 
        &\leq \eps \left( \frac{|z|}{x} \sum_{n=\ell}^{m-1} (e^{-\lambda_n x} 
        - e^{-\lambda_{n+1}x}) + 1 \right) 
        = \eps \left( \frac{|z|}{x} (e^{-\lambda_\ell x} - e^{-\lambda_m x})
        + 1 \right). 
    \end{align*}
    Now, for $\lvert\arg(z)\rvert \leq \alpha$ with $\alpha < \pi/2$, we have 
    \[ \frac{|z|}{x} = \frac{1}{\cos(\arg z)} < c \] 
    for some constant $c = c(\alpha)$. Moreover, note that $|e^{-\lambda_\ell x} 
    - e^{-\lambda_m x}| \leq 2$. Therefore, we have 
    \[ \left| \sum_{n=\ell}^m a_n e^{-\lambda_n z} \right| < (2c + 1)\eps. \] 
    In particular, the Dirichlet series converges uniformly for $\Re(z) \geq 0$ 
    and $\lvert\arg(z)\rvert \leq \alpha$. 
\end{pf}

We have proved that if the Dirichlet series converges at $z = z_0$, then 
it determines an analytic function for $\Re(z-z_0) \geq 0$ and 
$\lvert\arg(z-z_0)\rvert \leq \alpha$ with $\alpha < \pi/2$. Next, we'll 
show that if $\{a_n\}_{n=1}^\infty$ is in addition a sequence of non-negative 
real numbers, then the domain of convergence for the analytic function 
determined by the series is limited only by a singularity on the real axis. 

\begin{theo}{thm:7.10}
    Let $f(z) = \sum_{n=1}^\infty a_n e^{-\lambda_n z}$ be a Dirichlet series 
    where $\{a_n\}_{n=1}^\infty$ is a sequence of non-negative real numbers. 
    Suppose that the series converges for $\Re(z) > \sigma_0$ where $\sigma_0 
    \in \R$, and $f$ can be analytically continued in a neighbourhood of $\sigma_0$. 
    Then there exists $\eps > 0$ such that $\sum_{n=1}^\infty a_n e^{-\lambda_n z}$ 
    converges for $\Re(z) > \sigma_0 - \eps$. 
\end{theo}
\begin{pf}
    Without loss of generality, we may assume that $\sigma_0 = 0$. Since this 
    series converges on $\Re(z) > 0$, then for any $z \in \C$ with $\Re(z) > 0$, 
    the series converges uniformly on a neighbourhood centered at $z$ by 
    Theorem~\ref{thm:7.9} (we can find $w \in \C$ with $\Re(w) > 0$ such that 
    the fan-like uniform convergence area covers a neighbourhood of $z$). Then $f$ 
    is analytic at $z$, and hence analytic for $\Re(z) > 0$. Since $f$ is 
    analytic for $\Re(z) > 0$ and $f$ is also analytic in a neighbourhood of 
    $\sigma_0 = 0$, there exists $\eps > 0$ such that $f$ is analytic in 
    $|z - 1| \leq 1 + \eps$. We now consider the Taylor series expansion of $f$ 
    around $1$ in $|z - 1| \leq 1 + \eps$. Note that for $\Re(z) > 0$, we have 
    \[ f^{(m)}(z) = \sum_{n=1}^\infty a_n(-\lambda_n)^m e^{-\lambda_n z}. \] 
    This implies that 
    \[ f^{(m)}(1) = \sum_{n=1}^\infty a_n(-\lambda_n)^m e^{-\lambda_n}. \]
    Now, the Taylor series expansion of $f$ about $1$ in $|z - 1| \leq 1 + \eps$ 
    is of the form 
    \[ \sum_{m=0}^\infty \frac{f^{(m)}(1)}{m!} (z-1)^m. \] 
    Consider $f$ at the point $z = -\eps$. We have 
    \begin{align*} 
        f(-\eps) &= \sum_{m=0}^\infty \left( \sum_{n=1}^\infty a_n (-\lambda_n)^m 
        e^{-\lambda_n} \right) \frac{(-1-\eps)^m}{m!} \\
        &= \sum_{m=0}^\infty \left( \sum_{n=1}^\infty a_n \lambda_n^m 
        e^{-\lambda_n} \right) \frac{(1+\eps)^m}{m!}. 
    \end{align*}
    Since $a_n \geq 0$ and all the other terms above are positive, we can 
    switch the order of summation to obtain 
    \begin{align*}
        f(-\eps) &= \sum_{n=1}^\infty a_n e^{-\lambda_n} \left( \sum_{m=0}^\infty 
        \frac{\lambda_n^m (1+\eps)^m}{m!} \right) 
        = \sum_{n=1}^\infty a_n e^{-\lambda_n} e^{\lambda_n(1+\eps)} 
        = \sum_{n=1}^\infty a_n e^{\lambda_n \eps} 
        = \sum_{n=1}^\infty a_n e^{(-\lambda_n)(-\eps)}.
    \end{align*}
    Hence, the series $\sum_{n=1}^\infty a_n e^{-\lambda_n z}$ converges 
    to $f$ at $z = -\eps$. By Theorem~\ref{thm:7.9}, it converges to $f$ 
    for $\Re(z) > -\eps$ because for any $z \in \C$ with $\Re(z) > -\eps$, 
    we can find some $\alpha < \pi/2$ such that $\lvert\arg(z-(-\eps))\rvert 
    < \alpha$. 
\end{pf}

\subsection{Dirichlet's Theorem}\label{subsec:7.6}

\vspace{2ex}
\begin{theo}{thm:7.11}
    If $\chi$ is a character modulo $k$, then $L(s, \chi)$ is nonzero for 
    $\Re(s) > 1$. Furthermore, if $\chi$ is not principal, then $L(1, \chi)$ 
    is nonzero. 
\end{theo}
\begin{pf}
    Note that $L(s, \chi)$ converges absolutely for $\Re(s) > 1$. Moreover, 
    we showed in Theorem~\ref{thm:7.5} that $\chi$ is completely multiplicative, 
    so $L(s, \chi)$ has an Euler product representation for $\Re(s) > 1$ given by 
    \[ L(s, \chi) = \prod_p \left( 1 - \frac{\chi(p)}{p^s} \right)^{\!-1}. \] 
    Recall that given a sequence of complex numbers $\{a_n\}_{n=1}^\infty$, the 
    product $\prod_{n=1}^\infty (1 + a_n)$ with $1 + a_n \neq 0$ converges 
    absolutely (to a nonzero value) if and only if the series $\sum_{n=1}^\infty 
    |a_n|$ converges. Since $\sum_p |\chi(p)/p^s|$ converges for $\Re(s) > 1$, 
    so does the Euler product representation above. Thus, $L(s, \chi) \neq 0$
    for $\Re(s) > 1$. 

    For the second assertion, we have two cases, depending on whether $\chi$ is a 
    real or complex character. For $\Re(s) > 1$, the Euler product representation
    of $L(s, \chi)$ gives 
    \[ \log^* L(s, \chi) = \sum_p -\log\left(1 - \frac{\chi(p)}{p^s} \right) 
    = \sum_p \sum_{a=1}^\infty \frac{\chi(p^a)}{ap^{as}}, \] 
    where $\log$ denotes the principal branch and $\log^*$ indicates a branch 
    of the logarithm. 

    Let $k \geq 2$ be an integer, and let $\ell$ be an integer coprime with $k$. 
    Then we have 
    \[ \sum_{\chi\in\hat G(k)} \overline\chi(\ell) \log^* L(s, \chi) 
    = \sum_p \sum_{a=1}^\infty \frac{1}{ap^{as}} \sum_{\chi\in\hat G(k)} 
    \overline\chi(\ell) \chi(p^a). \] 
    By (5) of Theorem~\ref{thm:7.5}, we obtain 
    \begin{equation}\label{eq:7.1}
        \sum_{\chi\in\hat G(k)} \overline\chi(\ell) \log^* L(s, \chi) 
        = \phi(k) \sum_{a=1}^\infty \sum_{p^a\equiv\ell\text{ (mod }k)} \frac{1}{ap^{as}}. 
    \end{equation}
    Taking $\ell = 1$ in equation $\eqref{eq:7.1}$ and exponentiating both sides, 
    we get 
    \[ \prod_{\chi\in\hat G(k)} L(s, \chi) = \exp\left( 
        \phi(k) \sum_{a=1}^\infty \sum_{p^a\equiv1\text{ (mod }k)} \frac{1}{ap^{as}} 
        \right). \] 
    Therefore, if $s$ is real with $s > 1$, then 
    \begin{equation}\label{eq:7.2}
        \prod_{\chi\in\hat G(k)} L(s,\chi) \geq 1. 
    \end{equation}
    First, suppose that $L(1, \chi) = 0$ where $\chi$ is not a real character. 
    Then $\overline\chi$ is a character modulo $k$ with $\chi \neq \chi_0$. 
    Notice that when $s$ is real with $s > 1$, we have $\overline{L(s,\chi)} 
    = L(s,\overline\chi)$, and hence 
    \[ L(1, \overline\chi) = \overline{L(1, \chi)} = 0. \] 
    By Theorem~\ref{thm:7.7}, $L(s, \chi_0)$ has a simple pole at $s = 1$, 
    and $L(s, \chi)$ does not have a pole at $s = 1$ when $\chi \neq \chi_0$. 
    Hence, as $s \to 1$ on the real axis, we have 
    \[ \prod_{\chi\in\hat G(k)} L(s, \chi) = O((s-1)^{-1} (s-1)^2) = O(s-1). \] 
    However, this contradicts equation $\eqref{eq:7.2}$, so $L(1, \chi) \neq 0$ 
    when $\chi$ is not a real character. 

    Suppose now that $L(1, \chi) = 0$ where $\chi$ is a real character. 
    For $\Re(s) > 1$, we set 
    \[ g(s) = \frac{\zeta(s) L(s, \chi)}{\zeta(2s)}. \] 
    The Euler product representation of $g$ for $\Re(s) > 1$ is 
    \begin{align*}
        g(s) &= \prod_p \left( \frac{1 - p^{-2s}}{(1 - p^{-s})(1 - \chi(p)/p^s)} \right) \\ 
        &= \prod_p \left( \frac{1 + p^{-s}}{1 - \chi(p)/p^s} \right) \\ 
        &= \prod_p \left( 1 + \frac{1}{p^s} \right) \sum_{a=0}^\infty \frac{\chi(p^a)}{p^{as}} \\ 
        &= \prod_p \left( 1 + \sum_{a=1}^\infty \frac{\chi(p^{a-1}) + \chi(p^a)}{p^{as}} \right) \\ 
        &= \prod_p \left( 1 + \sum_{a=1}^\infty \frac{b(p^a)}{p^{as}} \right), 
    \end{align*}
    where $b(p^a) = \chi(p^{a-1}) + \chi(p^a)$. Since $\chi$ is a real character, 
    it takes on values from $\{-1, 0, 1\}$. Moreover, we know that $\chi$ 
    is multiplicative by Theorem~\ref{thm:7.5}, so we have 
    \[ b(p^a) = \chi(p^{a-1}) + \chi(p^a) = 
    \begin{cases}
        0 & \text{if } \chi(p) = 0, \\ 
        2 & \text{if } \chi(p) = 1, \\ 
        0 & \text{if } \chi(p) = -1. 
    \end{cases} \] 
    In all cases, we have $b(p^a) \geq 0$ for $a \geq 1$. Therefore, we see that 
    $g(s) = \sum_{n=1}^\infty a_n/n^s$ where $a_1 = 1$ and $a_n \geq 0$ for 
    all $n \geq 2$. Moreover, we had set $g(s) = \zeta(s)L(s, \chi)/\zeta(2s)$ 
    for $\Re(s) > 1$. Since $L(1, \chi) = 0$ eliminates the pole of $\zeta(s)$
    at $s = 1$ and $\zeta(2s)$ is nonzero and analytic for $\Re(s) > 1/2$, 
    it follows that $g(s)$ has an analytic continuation to $\Re(s) > 1/2$. 

    We now apply Theorem~\ref{thm:7.10} to conclude that the series defining 
    $g$ converges to $g$ for $\Re(s) > 1/2$. Letting $s \to 1/2$ from above 
    on the real axis, we have 
    \[ g(s) = O(s - 1/2) = o(1) \] 
    since $\zeta(2s)$ has a pole at $s = 1/2$. However, since 
    \[ g(s) = 1 + \sum_{n=2}^\infty \frac{a_n}{n^s} \] 
    with $a_n \geq 0$ for $n \geq 2$, we obtain $g(s) \geq 1$ for $\Re(s) > 1/2$. 
    This is a contradiction, so we must have $L(1, \chi) \neq 0$ when 
    $\chi$ is a real character. 
\end{pf}

\begin{theo}{thm:7.12}
    If $k$ and $\ell$ are coprime integers with $k \geq 2$, then the series 
    \[ \sum_{p\equiv\ell\text{ (mod }k)} \frac{1}{p} \] 
    diverges. Consequently, there are infinitely many primes in the 
    arithmetic progression $kn + \ell$. 
\end{theo}
\begin{pf}
    From equation $\eqref{eq:7.1}$ from Theorem~\ref{thm:7.11}, we have 
    \[ \frac{1}{\phi(k)} \sum_{\chi\in\hat G(k)} \overline\chi(\ell) 
    \log L(s, \chi) = \sum_{a=1}^\infty \sum_{p^a\equiv\ell\text{ (mod }k)}
    \frac{1}{ap^{as}}. \] 
    As $s \to 1$ from the right on the real axis, $(s-1)^{E(\chi)} L(s, \chi)$ 
    tends to a finite nonzero limit, where $E(\chi) = 1$ if $\chi = \chi_0$ 
    and $E(\chi) = 0$ otherwise. Then $E(\chi) \log(s-1) + 
    \log L(s, \chi)$ also tends to a limit. It follows that as $s \to 1$ 
    from the right on the real axis, we have 
    \[ \log L(s, \chi) = -E(\chi)\log(s-1) + O(1). \] 
    Hence, we get 
    \begin{align*}
        \frac{1}{\phi(k)} \sum_{\chi\in\hat G(k)} \overline\chi(\ell) \log L(s, \chi)
        &= \frac{1}{\phi(k)} \log L(s, \chi_0) + \frac{1}{\phi(k)} 
        \sum_{\substack{\chi\in\hat G(k) \\ \chi\neq\chi_0}} \overline\chi(\ell) 
        \log L(s, \chi) \\
        &= -\frac{1}{\phi(k)} \log(s-1) + O(1). 
    \end{align*}
    Combining this with $\eqref{eq:7.1}$ yields 
    \[ \sum_{a=1}^\infty \sum_{p^a\equiv\ell\text{ (mod }k)} \frac{1}{ap^{as}} 
    = -\frac{1}{\phi(k)} \log(s-1) + O(1). \] 
    Thus, we have 
    \[ \sum_{p\equiv\ell\text{ (mod }k)} \frac{1}{p^s} + 
    \sum_{a=2}^\infty \sum_{p^a\equiv\ell\text{ (mod }k)} \frac{1}{ap^{as}} 
    = -\frac{1}{\phi(k)} \log(s-1) + O(1). \] 
    On the other hand, for $s \in \R$ with $s \geq 1$, we have 
    \begin{align*}
        \sum_{a=2}^\infty \sum_{p^a\equiv\ell\text{ (mod }k)} \frac{1}{ap^{as}} 
        &\leq \frac12 \sum_{a=2}^\infty \sum_{p^a\equiv\ell\text{ (mod }k)} \frac{1}{p^{as}} \\ 
        &\leq \frac12 \sum_{n=2}^\infty \left( \frac{1}{n^{2s}} + \frac{1}{n^{3s}} + \cdots \right) \\ 
        &\leq \frac12 \sum_{n=2}^\infty \frac{1}{n^{2s}} \left( \frac{1}{1 - 1/n^s} \right) \\ 
        &\leq \sum_{n=1}^\infty \frac{1}{n^2} = \frac{\pi^2}{6}. 
    \end{align*}
    Therefore, as $s \to 1$ from the right on the real axis, we obtain 
    \[ \sum_{p\equiv\ell\text{ (mod }k)} \frac{1}{p^s} = -\frac{1}{\phi(k)} \log(s-1) + O(1). \] 
    Since the quantity $\log(s-1)$ blows up as $s \to 1$, the series diverges. 
\end{pf}

\subsection{Distribution of Primes in Arithmetic Progressions}\label{subsec:7.7}
Let $k$ and $\ell$ be coprime integers with $k \geq 2$. For each $x \in \R$, 
define $\pi(x, k, \ell)$ to be the number of primes $p$ satisfying 
$p \leq x$ and $p \equiv \ell \pmod k$. Then it can be shown that 
\[ \pi(x, k, \ell) \sim \frac{1}{\phi(k)} \frac{x}{\log x} \sim \frac{\Li(x)}{\phi(k)}. \] 
For $t \in \R$ and $k \in \Z^+$, set $\tau(k, t) = \max\{|t|, k + 2\}$. 
Let $c \in \R$ with $0 < c < 1$, and define the set $R_c(k)$ by 
\[ R_c(k) = \left\{ \sigma + it : 1 - \frac{c}{\log \tau(k, t)} < \sigma \right\}. \] 
One can show that there exists a positive real number $c_0$ such that if 
$\chi$ is a non-real character modulo $k$ for $k \geq 2$, then $L(s, \chi)$ 
is nonzero in $R_{c_0}(k)$. 

When $\chi$ is a real non-principal character, this is not true in general. 
However, such a $c_0$ exists if we allow for the possibility that 
there is a point $\beta$ on the real axis in $R_{c_0}(k)$ where 
$L(s, \chi)$ is zero. 

\begin{defn}{def:7.13} 
    If $L(s, \chi)$ vanishes at $\beta \in R_{c_0}(k)$, then $\beta$ is a 
    simple zero of $L(s, \chi)$ and is called a {\bf Siegel zero}. 
\end{defn}

The extended Riemann hypothesis implies that $L(s, \chi)$ is nonzero for 
$\Re(s) > 1/2$, so no Siegel zero exists under this hypothesis. 

Let $k$ and $\ell$ be coprime integers with $k \geq 2$. Put $b = \beta(k)$ 
if there is a real non-principal character $\chi$ where $\beta$ is a 
zero of $L(s, \chi)$ in $R_{c_0}(k)$, and set $b = 1$ otherwise. Then 
there exists $a > 0$ such that 
\[ \pi(x, k, \ell) = \frac{\Li(x)}{\phi(k)} - \frac{\lambda(b)}{b} \frac{x^b}{\phi(k)} 
+ O(x \exp(-a\sqrt{\log x})), \] 
where $\lambda(b) = 0$ if $b = 1$, and $\lambda(b) = \chi(\ell)$ if $b \neq 1$. 
We would like to know if the term 
\[ \frac{\lambda(b)}{b} \frac{x^b}{\phi(k)} \] 
exists, or in other words, whether a Siegel zero exists. We haven't been 
able to do so yet, however. The best ``effective'' estimate for the size 
of a Siegel zero $\beta(k)$ associated to $L(s, \chi)$ where $\chi$ is a 
real character modulo $k$ is due to Pintz, who proved that 
\[ \beta(k) < 1 - \frac{c}{\sqrt{k}}, \] 
where $c$ is an effectively computable positive number. On the other hand, 
Siegel proved that for every $\eps > 0$, there exists a positive 
number $c(\eps)$ such that 
\[ \beta(k) < 1 - \frac{c(\eps)}{k^{\eps}}. \] 
Unfortunately, there is no known algorithm for computing $c(\eps)$ given 
$\eps > 0$. Using Siegel's estimate, one can prove that if $H$ is a 
positive number satisfying $k \leq (\log x)^H$, then 
\[ \pi(x, k, \ell) = \frac{\Li(x)}{\phi(k)} + O\left( \frac{x}{\exp(C\sqrt{\log x})} \right) \] 
for some $C > 0$. However, notice that this big-$O$ term is quite ineffective. 
