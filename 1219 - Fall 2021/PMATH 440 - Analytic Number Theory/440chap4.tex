\section{Divisor Counting Functions}\label{sec:4}

\subsection{Asymptotic Formulas for Divisor Counting Functions}\label{subsec:4.1}

\vspace{2ex}
\begin{defn}{def:4.1}
    For a positive integer $n \in \N$, we denote by $\Omega(n)$ the number 
    of prime factors of $n$ counted with multiplicity, and $\omega(n)$ the 
    number of distinct prime factors of $n$. 
\end{defn}

For example, if $n = 2^{10} \cdot 3^2 \cdot 7$, then $\Omega(n) = 10 + 2 + 1
= 13$ and $\omega(n) = 3$. 

\begin{defn}{def:4.2}
    Let $k \in \N$. For each real number $x \in \R$, we define $\tau_k(x)$ 
    to be the number of positive integer with $n \leq x$ and $\Omega(n) = k$. 
    That is, 
    \[ \tau_k(x) = \#\{n \leq x : \Omega(n) = k\}. \] 
    Furthermore, we let $\pi_k(x)$ be the number of positive integers 
    $n$ with $n \leq x$ and $\omega(n) = \Omega(n) = k$. That is, 
    \[ \pi_k(x) = \#\{n \leq x : \omega(n) = \Omega(n) = k\}. \] 
    In particular, $\pi_k(x)$ counts the positive integers $n$ up to $x$ 
    which are squarefree and have $k$ prime factors. Note that 
    $\pi(x) = \pi_1(x) = \tau_1(x)$. 
\end{defn}

\begin{theo}[Landau, 1900]{thm:4.3} 
    Let $k \in \N$ be a positive integer. Then 
    \[ \pi_k(x) \sim \tau_k(x) \sim \frac{1}{(k-1)!} \frac{x}{\log x} 
    (\log \log x)^{k-1}. \] 
\end{theo}
\begin{pf}
    We first introduce the functions 
    \begin{align*} 
        L_k(x) &= \sum_{p_1\cdots p_k \leq x}{\vphantom{\sum}}^{\hspace{-2.5ex}*} \hspace{1.5ex} \frac{1}{p_1 \cdots p_k}, & 
        \Pi_k(x) &= \sum_{p_1\cdots p_k \leq x}{\vphantom{\sum}}^{\hspace{-2.5ex}*} \hspace{1.5ex} 1, &
        \Theta_k(x) &= \sum_{p_1\cdots p_k \leq x}{\vphantom{\sum}}^{\hspace{-2.5ex}*} \hspace{1.5ex} \log(p_1 \dots p_k), 
    \end{align*}
    where the $*$ means that the sum is taken over all $k$-tuples of 
    primes $(p_1, \dots, p_k)$ with $p_1 \cdots p_k \leq x$. Note that 
    different $k$-tuples can correspond to the same product $p_1 \cdots p_k$. 

    For each positive integer $n \geq 1$, we let $c_n = c_n(k)$ denote the number 
    of $k$-tuples $(p_1, \dots, p_k)$ such that $p_1 \cdots p_k = n$. 
    Observe that 
    \begin{align*}
        \Pi_k(x) &= \sum_{n\leq x} c_n, \\ 
        \Theta_k(x) &= \sum_{n\leq x} c_n \log n.
    \end{align*}
    Moreover, we have 
    \[ c_n = \begin{cases} 
        0 & \text{ if $n$ is not a product of $k$ primes,} \\
        k! & \text{ if $n$ is squarefree and $\omega(n) = \Omega(n) = k$.} 
    \end{cases} \] 
    We also see that $0 < c_n < k!$ if $\Omega(n) = k$ but $n$ is not squarefree. 
    Therefore, we obtain the inequalities 
    \[ k!\pi_k(x) \leq \Pi_k(x) \leq k!\tau_k(x). \tag{4.1}\label{eq:4.1} \] 
    For $k \geq 2$, note that the number of positive integers up to $x$ with 
    $k$ prime factors and divisible by the square of some prime is 
    $\tau_k(x) - \pi_k(x)$. Therefore, we have 
    \[ \tau_k(x) - \pi_k(x) = 
    \sum_{\substack{p_1\cdots p_k \leq x \\ p_i = p_j \text{ for some } i \neq j}}
    {\vphantom{\sum}}^{\hspace{-6.5ex}*} \hspace{5.5ex} 1 
    \leq \binom{k}{2} \sum_{p_1\cdots p_k \leq x}
    {\vphantom{\sum}}^{\hspace{-2.5ex}*} \hspace{1.5ex} 1 
    = \binom{k}{2} \Pi_{k-1}(x). \]
    {\sc Claim.} We have 
    \[ \Pi_k(x) \sim k \frac{x(\log \log x)^{k-1}}{\log x}. \] 
    {\sc Proof of Claim.} Applying Abel's summation formula with $a_n = c_n$ 
    and $f(u) = \log u$, we have 
    \[ \Theta_k(x) = \sum_{n \leq x} c_n \log n = \Pi_k(x) \log x 
    - \int_1^x \frac{\Pi_k(u)}{u}\dd u. \] 
    Observe that 
    \[ \Pi_k(x) \leq k!\tau_k(x) \leq k!x, \] 
    so $\Pi_k(u) = O(u)$, and hence 
    \[ \Theta_k(x) = \Pi_k(x) \log x + O(x). \] 
    Thus, it suffices to show that for all $k \in \N$, we have 
    \[ \Theta_k(x) \sim kx(\log \log x)^{k-1}. \tag{4.2}\label{eq:4.2} \] 
    We'll proceed by induction on $k$. This will be somewhat similar to the 
    proof of the Prime Number Theorem, but with the weighting function 
    $\log(p_1 \cdots p_k)$ on the $k$-tuple $(p_1, \dots, p_k)$. 

    For $k = 1$, we have $\Theta_1(x) = \theta(x) \sim x$ by Theorem~\ref{thm:2.7}
    and the Prime Number Theorem. Assume now that $\Theta_k(x) 
    \sim kx(\log \log x)^{k-1}$ for some $k \geq 1$. We'll prove the result for 
    $\Theta_{k+1}(x)$. First, note that 
    \[ \left( \sum_{p \leq x^{1/k}} \frac{1}{p} \right)^{\!k} 
    \leq L_k(x) \leq \left( \sum_{p \leq x} \frac{1}{p}\right)^{\!k} \] 
    for all $k \geq 1$. By Theorem~\ref{thm:2.13}, we have 
    \begin{align*} 
        \left( \sum_{p \leq x^{1/k}} \frac{1}{p} \right)^{\!k} &\sim 
        \left(\log \log (x^{1/k})\right)^k, \\ 
        \left( \sum_{p \leq x} \frac{1}{p}\right)^{\!k} &\sim 
        (\log \log x)^k. 
    \end{align*}
    Notice that 
    \[ \left( \log\log(x^{1/k}) \right)^{\!k} = 
    (\log\log x - \log k)^k \sim (\log\log x)^{k}, \] 
    so $L_k \sim (\log\log x)^k$. Therefore, we have 
    \[ \Theta_{k+1}(x) - (k+1)(\log\log x)^k = \Theta_{k+1}(x) 
    - (k+1)xL_k(x) + o\left( x(\log\log x)^k \right). \] 
    Note that 
    \begin{align*} 
        k\Theta_{k+1}(x) &= \sum_{p_1\cdots p_{k+1} \leq x}
        {\vphantom{\sum}}^{\hspace{-3.5ex}*} \hspace{2.5ex}
        k \cdot \log(p_1 \cdots p_{k+1}) \\ 
        &= \sum_{p_1\cdots p_{k+1} \leq x}
        {\vphantom{\sum}}^{\hspace{-3.5ex}*} \hspace{2.5ex}
        \left( \log(p_2 \cdots p_{k+1}) + \log(p_1p_3 \cdots p_{k+1}) 
        + \cdots + \log(p_1 \cdots p_k) \right) \\ 
        &= (k+1) \sum_{p_1 \leq x} \sum_{p_2\cdots p_{k+1} \leq x/p_1}
        {\vphantom{\sum}}^{\hspace{-5ex}*} \hspace{4ex} 
        \log(p_2 \cdots p_{k+1}) \\ 
        &= (k+1) \sum_{p_1 \leq x} \Theta_k \left( \frac{x}{p_1} \right). 
    \end{align*}
    Since $L_0(x) = 1$ and 
    \[ L_k(x) = \sum_{p_1\cdots p_k \leq x}
    {\vphantom{\sum}}^{\hspace{-2.5ex}*} \hspace{1.5ex} 
    \frac{1}{p_1 \cdots p_k} = \sum_{p_1 \leq x} \frac{1}{p_1} L_{k-1} 
    \left( \frac{x}{p_1} \right), \] 
    it follows that 
    \begin{align*}
        \Theta_{k+1}(x) - (k+1)xL_k(x) 
        &= (k+1) \sum_{p_1 \leq x} \left( \frac{1}{k} \Theta_k \left( \frac{x}{p_1} \right) 
        - \frac{x}{p_1} L_{k-1} \left( \frac{x}{p_1} \right) \right) \\ 
        &= \frac{k+1}{k} \sum_{p_1 \leq x} \left( \Theta_k \left( \frac{x}{p_1} \right) 
        - k \frac{x}{p_1} L_{k-1} \left( \frac{x}{p_1} \right) \right). 
    \end{align*}
    By the induction hypothesis, we have 
    \[ \Theta_k(y) - kyL_{k-1}(y) = o(y(\log \log y)^{k-1}). \] 
    Given $\eps > 0$, there exists $x_0 = x_0(\eps, k)$ such that for all 
    $y > x_0$, we have 
    \[ |\Theta_k(y) - kyL_{k-1}(y)| \leq \eps y(\log \log y)^{k-1}. \] 
    Furthermore, there exists a positive constant $c = c(\eps, k) > 0$ such 
    that for all $y \leq x_0$, we have 
    \[ |\Theta_k(y) - kyL_{k-1}(y)| \leq c. \] 
    Note that $x/p_1 > x_0$ implies that $p_1 < x/x_0$, so for sufficiently 
    large $x$, we obtain 
    \begin{align*} 
        |\Theta_{k+1}(x) - (k+1)xL_k(x)| 
        & \leq \frac{k+1}{k} \left( \sum_{\frac{x}{x_0} < p_1 \leq x} c 
        + \sum_{p_1 \leq \frac{x}{x_0}} \eps \frac{x}{p_1} \left( 
        \log\log \frac{x}{p_1} \right)^{\!k-1} \right) \\
        & \leq 2cx + 2\eps x(\log \log x)^{k-1} \sum_{p_1 \leq \frac{x}{x_0}} 
        \frac{1}{p_1} \\ 
        & \leq 2cx + 4\eps x (\log \log x)^k < 5\eps x(\log \log x)^k, 
    \end{align*}
    where the second last inequality comes from choosing $x$ large enough 
    so that 
    \[ \sum_{p\leq x} \frac{1}{p} \leq 2\log\log x. \] 
    Therefore, we see that 
    \[ \Theta_{k+1}(x) - (k+1)xL_k(x) = o\left( x(\log\log x)^k \right). \] 
    We conclude that 
    \[ \Theta_{k+1}(x) \sim (k+1) x(\log\log x)^k, \] 
    which proves the claim. \hfill$\blacksquare$

    From equation $\eqref{eq:4.1}$ and the claim, we have 
    \[ \pi_k(x) \leq \frac{1}{k!} \Pi_k(x) \sim 
    \frac{1}{(k-1)!} \frac{x}{\log x} (\log\log x)^{k-1}. \] 
    Moreover, combining equations $\eqref{eq:4.1}$ and 
    $\eqref{eq:4.2}$ with the claim yields 
    \[ \pi_k(x) = \tau_k(x) + O(\Pi_{k-1}(x)) 
    \geq \frac{1}{k!} \Pi_k(x) + O(\Pi_{k-1}(x)) 
    \sim \frac{1}{(k-1)!} \frac{x}{\log x} (\log \log x)^{k-1}. \] 
    In particular, we get 
    \[ \pi_k(x) \sim \tau_k(x) \sim \frac{1}{(k-1)!} \frac{x}{\log x} 
    (\log\log x)^{k-1}, \] 
    which finishes the proof of the theorem. 
\end{pf}

\subsection{Summatory Functions for $\omega(n)$ and $\Omega(n)$}\label{subsec:4.2} 
Let's now consider the averages of $\omega(n)$ and $\Omega(n)$. 

\begin{theo}{thm:4.4}
    We have 
    \begin{align*}
        \sum_{n\leq x} \omega(n) &= x\log\log x + \beta x + o(x), \\ 
        \sum_{n\leq x} \Omega(n) &= x\log\log x + \tilde\beta x + o(x), 
    \end{align*}
    where $\beta$ is Merten's constant as in Theorem~\ref{thm:2.13} and 
    \[ \tilde\beta = \beta + \sum_p \frac{1}{p(p-1)}. \] 
\end{theo}
\begin{pf}
    Set $S_1 = S_1(x) = \sum_{n\leq x} \omega(n)$. Then we have 
    \[ S_1 = \sum_{n\leq x} \sum_{p\mid n} 1 = \sum_{p\leq x} \floor*{\frac{x}{p}}. \] 
    By Theorem~\ref{thm:2.13}, we obtain 
    \begin{align*} 
        S_1 &= \sum_{p\leq x} \floor*{\frac{x}{p}} \\
        &= x \sum_{p\leq x} \frac{1}{p} + O(\pi(x)) \\
        &= x(\log\log x + \beta + o(1)) + O(\pi(x)) \\
        &= x\log\log x + x\beta + o(x), 
    \end{align*}
    where the last equality follows from the Prime Number Theorem. 

    On the other hand, if we set $S_2 = S_2(x) = \sum_{n\leq x} \Omega(n)$, then 
    \[ S_2 - S_1 = \sum_{p^m \leq x,\, m \geq 2} \floor*{\frac{x}{p^m}} 
    = \sum_{p^m \leq x,\, m\geq 2} \frac{x}{p^m} + 
    O \left( \sum_{p^m \leq x,\, m \geq 2} 1 \right). \] 
    Note that $2^m \leq p^m \leq x$, so $m \leq \frac{\log x}{\log 2}$. 
    Moreover, $p^2 \leq p^m \leq x$ implies that $p \leq x^{1/2}$. Therefore, 
    we have 
    \[ S_2 - S_1 = \sum_{p^m \leq x,\, m\geq 2} \frac{x}{p^m} + 
    O(x^{1/2} \log x) = x \left( \sum_p \left( \frac{1}{p^2} + 
    \frac{1}{p^3} + \cdots \right) - \sum_{p^m \geq x} \frac{1}{p^m} \right) 
    + O(x^{1/2} \log x). \] 
    Observe that 
    \begin{align*}
        \sum_{\substack{p^m > x \\ m \geq x}} \frac{1}{p^m} 
        &\leq \sum_{\substack{p^m > x \\ m \geq 2 \\ 2\,\mid\,n}} \frac{1}{p^m} 
        + \sum_{\substack{p^m > x \\ m \geq 2 \\ 2\,\nmid\,n}} \frac{1}{p^m} 
        \leq \sum_{n^2 > x} \frac{1}{n^2} + \sum_{\substack{p^m > x \\
        m \geq 2 \\ 2\,\mid\,m \\ p \leq \sqrt{x}}} \frac{1}{p^m} 
        + \sum_{\substack{p^m > x \\ 
        m \geq 2 \\ 2\,\mid\,m \\ p > \sqrt{x}}} \frac{1}{p^m}. 
    \end{align*}
    Notice that if $p \leq \sqrt{x}$, then since $p^m > x$, we get 
    $p^{m-1} > x/p > \sqrt{x}$. On the other hand, if $p > \sqrt{x}$, then 
    $p^{m-1} > \sqrt{x}$. Hence, we get 
    \begin{align*} 
        \sum_{\substack{p^m > x \\ m \geq 2}} \frac{1}{p^m} 
        &\leq \sum_{n^2 > x} \frac{1}{n^2} + 2 \sum_{\substack{p^{m-1} > \sqrt{x}
        \\ m \geq 2 \\ 2\,\mid\,m}} \frac{1}{p^{m-1}} 
        \leq \sum_{n^2 > x} \frac{1}{n^2} + 2 \sum_{m^2 > \sqrt{x}} \frac{1}{m^2}  
        \leq 3 \sum_{k > \sqrt[4]{x}} \frac{1}{k^2} = O\left( \frac{1}{\sqrt[4]{x}} \right). 
    \end{align*}
    Therefore, we have 
    \[ S_2 - S_1 = x \left( \sum_p \frac{1}{p(p-1)} + o(1) \right) 
    + O(x^{1/2} \log x) = x \sum_p \frac{1}{p(p-1)} + o(x). \] 
    Together with our estimate of $S_1$, we see that 
    \[ S_2 = x\log\log x + x \left( \beta + \sum_p \frac{1}{p(p-1)} \right) 
    + o(x). \qedhere \] 
\end{pf}

\subsection{Asymptotic Density and Normal Order}\label{subsec:4.3} 

\vspace{2ex}
\begin{defn}{def:4.5}
    Let $A$ be a subset of $\N$. For any $n \in \N$, we set $A(n) 
    = \{1, \dots, n\} \cap A$. We define the {\bf upper asymptotic density}
    of $A$ by 
    \[ \bar{d}(A) := \limsup_{n\to\infty} \frac{|A(n)|}{n}. \] 
    Similarly, we define the {\bf lower asymptotic density} of $A$ to be 
    \[ \underline{d}(A) := \liminf_{n\to\infty} \frac{|A(n)|}{n}. \] 
    We say that $A$ has {\bf asymptotic density} $d(A)$ when 
    $\bar{d}(A) = \underline{d}(A)$, in which case we set $d(A)$ to be 
    this common value. 
\end{defn}

Now, let's look at some simple examples of asymptotic density of subsets 
$A \subseteq \N$. 

\begin{exmp}{exmp:4.6}
    \begin{enumerate}[(1)]
        \item When $A$ is the set of all primes, we have 
        $d(A) = \bar{d}(A) = \underline{d}(A) = 0$. 
        \item For $A = \{n \in \N : n \equiv 0 \pmod 5\}$, we have 
        $d(A) = \bar{d}(A) = \underline{d}(A) = 1/5$.
        \item For $A = \{n \in \N : n \neq k^2 + 1 \text{ for any } k \in \Z\}$, 
        we have $d(A) = \bar{d}(A) = \underline{d}(A) = 1$.
        \item Let $A = \{a \in \N : (2k)! < a < (2k+1)! \text{ for some } k \in \Z\}$. 
        Notice that for $n = (2k+1)!$, any $a \in \N$ satisfying $(2k)! < a < 
        (2k+1)!$ is included in $A(n)$. Therefore, we have 
        \[ 1 \geq \frac{|A((2k+1)!)|}{(2k+1)!} \geq \frac{(2k+1)! - (2k)!}{(2k+1)!} 
        = \frac{2k}{2k+1}. \] 
        By taking $k \to \infty$, we see that 
        \[ \frac{|A((2k+1)!)|}{(2k+1)!} \to 1, \] 
        and hence $\bar{d}(A) = 1$. On the other hand, when $n = (2k)!$, then only 
        $a \in \N$ such that $a < (2k-1)!$ are included in $A(n)$. Thus, we have 
        \[ 0 \leq \frac{|A((2k)!)|}{(2k)!} \leq \frac{(2k-1)!}{(2k)!} = \frac{1}{2k}. \] 
        As $k \to \infty$, we have 
        \[ \frac{|A((2k)!)|}{(2k)!} \to 0, \] 
        and hence $\underline{d}(A) = 0$.
    \end{enumerate} 
\end{exmp}

From asymptotic density, we can define normal order. 
Moreover, we will define average order. 

\begin{defn}{def:4.7}
    Let $f(n)$ and $F(n)$ be functions from $\N$ to $\R$. 
    \begin{itemize}
        \item We say that $f(n)$ has {\bf normal order} $F(n)$ if for every 
        $\eps > 0$, the set 
        \[ A(\eps) = \{n \in \N : (1-\eps)F(n) < f(n) < (1+\eps)F(n)\} \] 
        has the property that $d(A(\eps)) = 1$. Equivalently, if $B(\eps) 
        = \N \setminus A(\eps)$, then $d(B(\eps)) = 0$. 
        \item We say that $f(n)$ has {\bf average order} $F(n)$ if 
        \[ \sum_{j=1}^n f(j) \sim \sum_{j=1}^n F(j). \] 
    \end{itemize}
\end{defn}

These definitions seem rather abstract, so let's look at some examples of normal 
and average order. It's not too difficult to check the details. 

\begin{exmp}{exmp:4.8}
    \begin{enumerate}[(1)]
        \item If we define 
        \[ f(n) = \begin{cases} 1 & \text{if $n \neq k!$ for any $k \in \N$,} \\ 
            n & \text{if $n = k!$ for some $k \in \N$,} \end{cases} \] 
        then $f$ has normal order $1$ but not average order $1$. 
        \item If we define 
        \[ f(n) = \begin{cases} 2 & \text{if $n \equiv 1 \pmod 2$,} \\ 
            0 & \text{if $n \equiv 0 \pmod 2$,} \end{cases} \] 
        then $f$ has average order $1$ but not normal order $1$. 
        \item If we define 
        \[ f(n) = \begin{cases}
            \log n + (\log n)^{1/2} & \text{if $n \equiv 1 \pmod 2$,} \\ 
            \log n - (\log n)^{1/2} & \text{if $n \equiv 0 \pmod 2$,} \\ 
        \end{cases} \] 
        then $f$ has both normal and average order $\log n$.
    \end{enumerate}
\end{exmp}

\begin{theo}{thm:4.9}
    Both $\omega(n)$ and $\Omega(n)$ have average order $\log\log n$. 
\end{theo}
\begin{pf}
    First, note that 
    \begin{align*} 
        \sum_{n\leq x} \log\log n 
        &= \sum_{x^{1/2}<n\leq x} \log\log n + \sum_{n\leq x^{1/2}} \log\log n \\
        &= \sum_{x^{1/2}<n\leq x} \log\log n + O(x^{1/2}\log \log x). 
    \end{align*}
    Moreover, we have 
    \[ \sum_{x^{1/2} < n \leq x} \log\log n \leq \log\log x  
    \sum_{x^{1/2} < n \leq x} 1 = x\log\log x + O(x^{1/2}\log\log x). \] 
    Also, we have the lower bound 
    \[ \sum_{x^{1/2} < n \leq x} \log\log n \geq (\log\log x - \log 2) 
    \sum_{x^{1/2} < n \leq x} 1 = x\log\log x + O(x^{1/2}\log\log x). \] 
    It follows that 
    \[ \sum_{n\leq x} \log\log n = x\log\log x + O(x^{1/2}\log\log x). \] 
    Combining this estimate with Theorem~\ref{thm:4.4} shows that 
    $\omega(n)$ and $\Omega(n)$ both have average order $\log\log n$. 
\end{pf}

\subsection{Normal Order of $\omega(n)$ and $\Omega(n)$}\label{subsec:4.4}
We have shown that $\omega(n)$ and $\Omega(n)$ have average order $\log\log n$. 
In this section, we'll work towards proving that they have normal order $\log\log n$. 

\begin{theo}{thm:4.10}
    Let $\delta > 0$. The number of positive integers $n \leq x$ satisfying 
    \[ |f(n) - \log\log n| > (\log\log n)^{\frac12+\delta} \] 
    is $o(x)$, where $f(n) = \omega(n)$ or $f(n) = \Omega(n)$. In particular, 
    both $\omega(n)$ and $\Omega(n)$ have normal order $\log\log n$. 
\end{theo}
\begin{pf}
    It is enough to prove that the number of positive integers $n \leq x$ with 
    \[ |f(n) - \log\log x| > (\log\log x)^{\frac12+\delta} \] 
    is $o(x)$, because for $x^{1/e} \leq n \leq x$, we have 
    \[ \log\log x \geq \log\log n \geq \log\left(\frac{\log x}e\right) = 
    \log\log x - 1. \] 
    In other words, we can replace $\log\log n$ in the statement of the theorem 
    with $\log\log x$. 

    Moreover, we can restrict our attention to the case where $f(n) = \omega(n)$, 
    because by Theorem~\ref{thm:4.4}, we have 
    \[ \sum_{n\leq x} (\Omega(n) - \omega(n)) = O(x). \] 
    Thus, the number of integers $n \leq x$ for which $\Omega(n) - \omega(n) 
    > (\log\log n)^{1/2}$ is $o(x)$. 

    {\sc Claim.} We have 
    \begin{align*}
        \sum_{n\leq x} \omega(n)^2 = x(\log\log x)^2 + O(x\log\log x), \\ 
        \sum_{n\leq x} (\omega(n) - \log\log x)^2 = O(x\log\log x). 
    \end{align*}

    {\sc Proof of Claim.} For each $n \leq x$, consider the ordered pairs 
    $(p, q)$ where $p$ and $q$ are distinct prime factors of $n$. There are 
    $\omega(n)$ choices for $p$ and $\omega(n) - 1$ choices for $q$, which gives 
    \[ \omega(n)(\omega(n) - 1) = \sum_{\substack{pq\,\mid\,n \\ p \neq q}} 1 
    = \sum_{pq\,\mid\,n} 1 - \sum_{p^2\,\mid\,n} 1. \] 
    Therefore, we have 
    \begin{align*} 
        \sum_{n\leq x} \omega(n)^2 - \sum_{n\leq x} \omega(n) 
        &= \sum_{n\leq x} \omega(n)(\omega(n) - 1) \\ 
        &= \sum_{n\leq x} \left( \sum_{pq\,\mid\,n} 1 - \sum_{p^2\,\mid\,n} 1 \right) \\
        &= \sum_{pq \leq x} \floor*{\frac{x}{pq}} - \sum_{p^2 \leq x} \floor*{\frac{x}{p^2}}.
    \end{align*}
    Observe that 
    \begin{align*} 
        \sum_{p^2 \leq x} \floor*{\frac{x}{p^2}} 
        \leq x \sum_{p^2 \leq x} \frac{1}{p^2} = O(x), \\ 
        \sum_{pq \leq x} \floor*{\frac{x}{pq}} 
        = \sum_{pq \leq x} \frac{x}{pq} + O(x), 
    \end{align*}
    which implies that 
    \[ \sum_{n\leq x} \omega(n)^2 - \sum_{n\leq x} \omega(n) 
    = \sum_{pq\leq x} \frac{x}{pq} + O(x). \tag{4.3}\label{eq:4.3} \] 
    Next, note that 
    \[ \left( \sum_{p\leq x^{1/2}} \frac1p \right)^{\!2} 
    - \left( \sum_{p\leq x} \frac1{p^2} \right) 
    \leq \sum_{pq\leq x} \frac{1}{pq} 
    \leq \left( \sum_{p\leq x} \frac1p \right)^{\!2}. \] 
    Furthermore, Merten's theorem (Theorem~\ref{thm:2.13}) tells us that 
    \[ \left( \sum_{p\leq x} \frac1p \right)^{\!2} = (\log\log x)^2 
    + O(\log\log x), \] 
    so it follows that 
    \[ \left( \sum_{p\leq x^{1/2}} \frac1p \right)^{\!2} 
    = \left(\log\log x^{1/2} + O(1)\right)^2 
    = (\log\log x - \log 2 + O(1))^2 
    = (\log\log x)^2 + O(\log\log x). \] 
    Thus, we obtain 
    \[ \sum_{pq\leq x} \frac{1}{pq} = (\log\log x)^2 + O(\log\log x). 
    \tag{4.4}\label{eq:4.4} \] 
    By Theorem~\ref{thm:4.4}, we get 
    \[ \sum_{n\leq x} \omega(n) = O(x\log\log x). \tag{4.5}\label{eq:4.5} \] 
    Combining equations $\eqref{eq:4.3}$, $\eqref{eq:4.4}$, and $\eqref{eq:4.5}$ 
    together yields 
    \[ \sum_{n\leq x} \omega(n)^2 = x(\log\log x)^2 + O(x\log\log x), \] 
    which proves the first equality. Now, we have 
    \begin{align*}
        \sum_{n\leq x} (\omega(n) - \log\log x)^2 
        &= \sum_{n\leq x} \omega(n)^2 - 2\sum_{n\leq x} \omega(n)\log\log x 
        + \sum_{n\leq x} (\log\log x)^2 \\ 
        &= x(\log\log x)^2 + O(x\log\log x) - 2\log\log x \sum_{n\leq x} \omega(n)
        + \floor{x} (\log\log x)^2 \\ 
        &= x(\log\log x)^2 + O(x\log\log x) - 2x(\log\log x)^2 + O(\log\log x) \\
        &\qquad + x(\log\log x)^2 + O((\log\log x)^2) \\ 
        &= O(x\log\log x), 
    \end{align*}
    where the second last equality follows from Theorem~\ref{thm:4.4}. This 
    finishes the proof of the claim. \hfill $\blacksquare$ 

    Finally, as we stated in the beginning of the proof, it suffices to show that 
    \[ E(x) := \#\{n \leq x : |\omega(n) - \log\log x| > (\log\log x)^{\frac12+\delta}\} \] 
    is $o(x)$. By the claim, we have 
    \[ E(x) \cdot (\log\log x)^{1+2\delta} 
    \leq \sum_{n\leq x} (\omega(n) - \log\log x)^2 = O(x\log\log x). \] 
    It follows that 
    \[ E(x) = O\left( \frac{x\log\log x}{(\log\log x)^{1+2\delta}} 
    \right) = o(x). \qedhere \] 
\end{pf}

\begin{remark}{remark:4.11}
    Since the average order of $\omega(n)$ is $\log\log n$, which is asymptotic 
    to $\log\log x$ for ``almost all'' $n$ (namely, all except $o(x)$ 
    many $n \leq x$), we can view the sum 
    \[ \frac1x \sum_{n\leq x} (\omega(n) - \log\log x)^2 \] 
    as the variance of $\omega(n)$; that is, the squares of the standard deviation. 
    In Homework 3, we will show that 
    \[ \sum_{n\leq x} (\omega(n) - \log\log x)^2 \sim x\log\log x, \] 
    which implies that the standard deviation of $\omega(n)$ is about 
    $\sqrt{\log\log n}$. Now, consider the term 
    \[ \frac{\omega(n) - \log\log n}{\sqrt{\log\log n}}. \] 
    In 1934, Erd\H{o}s and Kac proved (without knowing probability theory) that 
    \[ \lim_{x\to\infty} \frac1x \#\left\{ n\leq x : \frac{\omega(n) - 
    \log\log n}{\sqrt{\log\log n}} \leq \gamma \right\} = G(\gamma), \] 
    where we define 
    \[ G(\gamma) := \frac{1}{\sqrt{2\pi}} \int_{-\infty}^{\gamma} e^{-t^2/2}\dd t \] 
    to be the Gaussian normal distribution. This result forms a foundation 
    of probabilistic number theory. 
\end{remark}

Recall that for all $n \in \N$, the divisor function $d(n)$ gives the number of 
positive divisors of $n$. In particular, if we have $n = p_1^{a_1} \cdots p_r^{a_r}$
where $a_1, \dots, a_r \in \N$ and $p_1, \dots, p_r$ are distinct primes, then
\begin{align*} 
    \omega(n) &= r, \\ 
    \Omega(n) &= a_1 + \cdots + a_r, \\ 
    d(n) &= (a_1 + 1) \cdots (a_r + 1). 
\end{align*}

\begin{theo}{thm:4.12} 
    Given $\eps > 0$, define the set 
    \[ S(\eps) = \{n \in \N : 2^{(1-\eps)\log\log n} < d(n) < 2^{(1+\eps)\log\log n} \}. \] 
    Then $S(\eps)$ has asymptotic density $1$. 
\end{theo}
\begin{pf}
    Note that for any $a \in \N$, we have 
    \[ 2 \leq a + 1 \leq 2^a. \] 
    In particular, we get 
    \[ 2^{\omega(n)} \leq d(n) \leq 2^{\Omega(n)}, \] 
    and the result follows from Theorem~\ref{thm:4.10}. 
\end{pf}

\begin{remark}{remark:4.13}
    We saw in Theorem~\ref{thm:3.9} that 
    \[ \sum_{n\leq x} d(n) \sim x\log x \sim \sum_{n\leq x}\log n. \] 
    Therefore, the average order of $d(n)$ is $\log n$. However, using 
    Theorem~\ref{thm:4.12}, one can show that for almost all $n \in \N$, the 
    divisor function $d(n)$ satisfies 
    \[ (\log n)^{\log 2 - \eps} < d(n) < (\log n)^{\log 2 + \eps} \] 
    for any $\eps > 0$. 
\end{remark}