\documentclass[10pt]{article}
\usepackage[T1]{fontenc}
\usepackage{amsmath,amssymb,amsthm}
\usepackage{mathtools}
\usepackage[shortlabels]{enumitem}
\usepackage[english]{babel}
\usepackage[utf8]{inputenc}
\usepackage{fancyhdr}
\usepackage{bold-extra}
\usepackage{color}   
\usepackage{tocloft}
\usepackage{graphicx}
\usepackage{lipsum}
\usepackage{wrapfig}
\usepackage{cutwin}
\usepackage{hyperref}
\usepackage{lastpage}
\usepackage{multicol}
\usepackage{tikz}
\usepackage{xcolor}
\usepackage{microtype}
\usepackage{titlesec}
\usepackage{tikz-cd}
\usepackage[framemethod=TikZ]{mdframed}

\numberwithin{equation}{section}

% some useful math commands
\newcommand{\eps}{\varepsilon}
\newcommand{\R}{\mathbb{R}}
\newcommand{\C}{\mathbb{C}}
\newcommand{\N}{\mathbb{N}}
\newcommand{\Z}{\mathbb{Z}}
\newcommand{\Q}{\mathbb{Q}}
\newcommand{\K}{\mathbb{K}}
\newcommand{\F}{\mathbb{F}}
\newcommand{\A}{\mathbb{A}}

\DeclareMathOperator{\GL}{GL}
\DeclareMathOperator{\id}{id}
\DeclareMathOperator{\Hom}{Hom}
\DeclareMathOperator{\End}{End}
\DeclareMathOperator{\im}{im}
\DeclareMathOperator{\Ann}{Ann}
\DeclareMathOperator{\op}{op}
\DeclareMathOperator{\Spec}{Spec}
\DeclareMathOperator{\Tr}{Tr}
\DeclareMathOperator{\ch}{char}
\DeclareMathOperator{\fix}{fix}

% title formatting
\newcommand{\newtitle}[4]{
  \begin{center}
	\huge{\textbf{\textsc{#1 Course Notes}}}
    
	\large{\sc #2}
    
	{\sc #3 \textbullet\, #4 \textbullet\, University of Waterloo}
	\normalsize\vspace{1cm}\hrule
  \end{center}
}

\newcounter{theo}[section]\setcounter{theo}{0}
\renewcommand{\thetheo}{\arabic{section}.\arabic{theo}}
\newenvironment{theo}[2][]{%
\refstepcounter{theo}%
\ifstrempty{#1}%
{\mdfsetup{%
frametitle={%
\tikz[baseline=(current bounding box.east),outer sep=0pt]
\node[anchor=east,rectangle,fill=blue!20]
{\strut {\sc Theorem~\thetheo}};}}
}%
{\mdfsetup{%
frametitle={%
\tikz[baseline=(current bounding box.east),outer sep=0pt]
\node[anchor=east,rectangle,fill=blue!20]
{\strut {\sc Theorem~\thetheo:~#1}};}}%
}%
\mdfsetup{innertopmargin=10pt,linecolor=blue!20,%
linewidth=2pt,topline=true,%
frametitleaboveskip=\dimexpr-\ht\strutbox\relax
}
\begin{mdframed}[nobreak=true]\relax%
\label{#2}}{\end{mdframed}}

%%%%%%%%%%%%%%%%%%%%%%%%%%%%%%
%Definition
\newenvironment{defn}[2][]{%
\refstepcounter{theo}%
\ifstrempty{#1}%
{\mdfsetup{%
frametitle={%
\tikz[baseline=(current bounding box.east),outer sep=0pt]
\node[anchor=east,rectangle,fill=yellow!20]
{\strut {\sc Definition~\thetheo}};}}
}%
{\mdfsetup{%
frametitle={%
\tikz[baseline=(current bounding box.east),outer sep=0pt]
\node[anchor=east,rectangle,fill=yellow!20]
{\strut {\sc Definition~\thetheo:~#1}};}}%
}%
\mdfsetup{innertopmargin=10pt,linecolor=yellow!20,%
linewidth=2pt,topline=true,%
frametitleaboveskip=\dimexpr-\ht\strutbox\relax
}
\begin{mdframed}[nobreak=true]\relax%
\label{#2}}{\end{mdframed}}

%%%%%%%%%%%%%%%%%%%%%%%%%%%%%%
%Example
\newenvironment{exmp}[2][]{%
\refstepcounter{theo}%
\ifstrempty{#1}%
{\mdfsetup{%
frametitle={%
\tikz[baseline=(current bounding box.east),outer sep=0pt]
\node[anchor=east,rectangle,fill=cyan!20]
{\strut {\sc Example~\thetheo}};}}
}%
{\mdfsetup{%
frametitle={%
\tikz[baseline=(current bounding box.east),outer sep=0pt]
\node[anchor=east,rectangle,fill=cyan!20]
{\strut {\sc Example~\thetheo:~#1}};}}%
}%
\mdfsetup{innertopmargin=10pt,linecolor=cyan!20,%
linewidth=2pt,topline=true,%
frametitleaboveskip=\dimexpr-\ht\strutbox\relax
}
\begin{mdframed}[]\relax%
\label{#2}}{\end{mdframed}}

%%%%%%%%%%%%%%%%%%%%%%%%%%%%%%
%Corollary
\newenvironment{cor}[2][]{%
\refstepcounter{theo}%
\ifstrempty{#1}%
{\mdfsetup{%
frametitle={%
\tikz[baseline=(current bounding box.east),outer sep=0pt]
\node[anchor=east,rectangle,fill=lime!20]
{\strut {\sc Corollary~\thetheo}};}}
}%
{\mdfsetup{%
frametitle={%
\tikz[baseline=(current bounding box.east),outer sep=0pt]
\node[anchor=east,rectangle,fill=lime!20]
{\strut {\sc Corollary~\thetheo:~#1}};}}%
}%
\mdfsetup{innertopmargin=10pt,linecolor=lime!20,%
linewidth=2pt,topline=true,%
frametitleaboveskip=\dimexpr-\ht\strutbox\relax
}
\begin{mdframed}[nobreak=true]\relax%
\label{#2}}{\end{mdframed}}

%%%%%%%%%%%%%%%%%%%%%%%%%%%%%%
%Remark
\newenvironment{remark}[2][]{%
\refstepcounter{theo}%
\ifstrempty{#1}%
{\mdfsetup{%
frametitle={%
\tikz[baseline=(current bounding box.east),outer sep=0pt]
\node[anchor=east,rectangle,fill=orange!20]
{\strut {\sc Remark~\thetheo}};}}
}%
{\mdfsetup{%
frametitle={%
\tikz[baseline=(current bounding box.east),outer sep=0pt]
\node[anchor=east,rectangle,fill=orange!20]
{\strut {\sc Remark~\thetheo:~#1}};}}%
}%
\mdfsetup{innertopmargin=10pt,linecolor=orange!20,%
linewidth=2pt,topline=true,%
frametitleaboveskip=\dimexpr-\ht\strutbox\relax
}
\begin{mdframed}[nobreak=true]\relax%
\label{#2}}{\end{mdframed}}

%%%%%%%%%%%%%%%%%%%%%%%%%%%%%%
%Exercise
\newenvironment{exercise}[2][]{%
\refstepcounter{theo}%
\ifstrempty{#1}%
{\mdfsetup{%
frametitle={%
\tikz[baseline=(current bounding box.east),outer sep=0pt]
\node[anchor=east,rectangle,fill=pink!20]
{\strut {\sc Exercise~\thetheo}};}}
}%
{\mdfsetup{%
frametitle={%
\tikz[baseline=(current bounding box.east),outer sep=0pt]
\node[anchor=east,rectangle,fill=pink!20]
{\strut {\sc Exercise~\thetheo:~#1}};}}%
}%
\mdfsetup{innertopmargin=10pt,linecolor=pink!20,%
linewidth=2pt,topline=true,%
frametitleaboveskip=\dimexpr-\ht\strutbox\relax
}
\begin{mdframed}[nobreak=true]\relax%
\label{#2}}{\end{mdframed}}

%%%%%%%%%%%%%%%%%%%%%%%%%%%%%%
%Lemma
\newenvironment{lemma}[2][]{%
\refstepcounter{theo}%
\ifstrempty{#1}%
{\mdfsetup{%
frametitle={%
\tikz[baseline=(current bounding box.east),outer sep=0pt]
\node[anchor=east,rectangle,fill=green!20]
{\strut {\sc Lemma~\thetheo}};}}
}%
{\mdfsetup{%
frametitle={%
\tikz[baseline=(current bounding box.east),outer sep=0pt]
\node[anchor=east,rectangle,fill=green!20]
{\strut {\sc Lemma~\thetheo:~#1}};}}%
}%
\mdfsetup{innertopmargin=10pt,linecolor=green!20,%
linewidth=2pt,topline=true,%
frametitleaboveskip=\dimexpr-\ht\strutbox\relax
}
\begin{mdframed}[nobreak=true]\relax%
\label{#2}}{\end{mdframed}}

%%%%%%%%%%%%%%%%%%%%%%%%%%%%%%
%Proposition
\newenvironment{prop}[2][]{%
\refstepcounter{theo}%
\ifstrempty{#1}%
{\mdfsetup{%
frametitle={%
\tikz[baseline=(current bounding box.east),outer sep=0pt]
\node[anchor=east,rectangle,fill=purple!20]
{\strut {\sc Proposition~\thetheo}};}}
}%
{\mdfsetup{%
frametitle={%
\tikz[baseline=(current bounding box.east),outer sep=0pt]
\node[anchor=east,rectangle,fill=purple!20]
{\strut {\sc Proposition~\thetheo:~#1}};}}%
}%
\mdfsetup{innertopmargin=10pt,linecolor=purple!20,%
linewidth=2pt,topline=true,%
frametitleaboveskip=\dimexpr-\ht\strutbox\relax
}
\begin{mdframed}[nobreak=true]\relax%
\label{#2}}{\end{mdframed}}

% new proof environment
\makeatletter
\newenvironment{pf}[1][\proofname]{\par
  \pushQED{\qed}%
  \normalfont \topsep0\p@\relax
  \trivlist
  \item[\hskip\labelsep\scshape
  #1\@addpunct{.}]\ignorespaces
}{%
  \popQED\endtrivlist\@endpefalse
}
\makeatother

% 1-inch margins
\topmargin 0pt
\advance \topmargin by -\headheight
\advance \topmargin by -\headsep
\textheight 8.9in
\oddsidemargin 0pt
\evensidemargin \oddsidemargin
\marginparwidth 0.5in
\textwidth 6.5in

\parindent 0in
\parskip 1.5ex

\setlist[itemize]{topsep=0pt}
\setlist[enumerate]{topsep=0pt}

% hyperlinks
\hypersetup{
    colorlinks=true, 
    linktoc=all,     % table of contents is clickable  
    allcolors=black  % all hyperlink colours
}

% table of contents
\addto\captionsenglish{
    \renewcommand{\contentsname}%
        {Table of Contents}%
}
\renewcommand{\cftsecfont}{\normalfont}
\renewcommand{\cftsecpagefont}{\normalfont}
\cftsetindents{section}{0em}{2em}

\fancypagestyle{plain}{%
    \fancyhf{} % clear all header and footer fields
    \lhead{PMATH 445: Fall 2021}
    \fancyhead[R]{Table of Contents}
    %\headrule
    \fancyfoot[R]{{\small Page \thepage\ of \pageref*{LastPage}}}
}

% headers and footers
\pagestyle{fancy}
\renewcommand{\sectionmark}[1]{\markboth{#1}{#1}}
\lhead{PMATH 445: Fall 2021}
\cfoot{}
\setlength\headheight{14pt}

%\setcounter{section}{-1}

\begin{document}

\pagestyle{fancy}
\newtitle{PMATH 445}{Representations of Finite Groups}{Jason Bell}{Fall 2021}
\rhead{Table of Contents}
\rfoot{{\small Page \thepage\ of \pageref*{LastPage}}}

\tableofcontents
\vspace{1cm}\hrule
\fancyhead[R]{\nouppercase\rightmark}
\newpage 
\fancyhead[R]{Lecture \thesection: \nouppercase\rightmark}

\section{September 8, 2021}

We begin the course by recalling Cayley's theorem, a famous result from group theory. It states that 
every finite group $G$ embeds (there exists an injective homomorphism) into a symmetric group $S_n$. 
The proof is simple: let $G$ act on itself by left multiplication, and show that this gives an 
embedding of $G$ into $S_n$ where $n = |G|$. This result is simple, but it allows us to understand 
finite groups as subgroups of symmetric groups, where one has many tools to use. 

In representation theory, one seeks to understand groups in terms of maps into general linear groups 
$\GL_n(F)$, where $F$ is a field. This is generally more desirable than an embedding into 
a symmetric group, as we obtain the full power of linear algebra at our disposal. We will consider 
all homomorphisms (not just injective ones) from groups to general linear groups, and such 
homomorphisms are called {\bf representations} of our group. First, we show that 
every finite field embeds into $\GL_n(F)$ for some field $F$.

\begin{prop}
Let $F$ be a field. Every finite group embeds into $\GL_n(F)$ for some $n \geq 1$. 
\end{prop}
\begin{pf}
Let $G$ be a finite group. By Cayley's theorem, we have an embedding $G \hookrightarrow S_n$ 
where $n = |G|$. Hence, it suffices to show that $S_n$ embeds into $\GL_n(F)$. Define 
$\varphi : S_n \to \GL_n(F)$ by $\psi(\sigma) = P_\sigma$, where $P_\sigma$ denotes the permutation matrix. Notice that for $\sigma_1, \sigma_2 \in S_n$, we have $\varphi(\sigma_1\sigma_2) 
= P_{\sigma_1\sigma_2} = P_{\sigma_1}P_{\sigma_2} = \varphi(\sigma_1\sigma_2)$, so $\varphi$ is a 
group homomorphism. One can also check that if $\varphi(\sigma) = I$, the identity matrix, then 
$\sigma$ must be the identity permutation, so $\varphi$ is injective.
\end{pf}

It turns out that this result is not true for infinite groups in general. 

\begin{exmp}
Let $G$ be the group consisting of bijective maps from $\Z^+$ to itself such that $f$ fixes all
but finitely integers. We claim that there does not exist a field $F$ and $n \geq 1$ such that 
$G$ embeds into $\GL_n(F)$. 

\begin{pf} 
First, we note the following fact from linear algebra.

{\sc Fact 1.} If $A$ and $B$ are commuting diagonalizable matrices, then they are simultaneously 
diagonalizable. That is, there is a common change of basis that makes both matrices diagonalizable.
This result also extends to families of commuting diagonalizable matrices. 

Now, we denote by $(i, j)$ the bijective mapping from $\Z^+$ to itself which swaps $i$ and $j$ 
and fixes all other integers. Consider the permutations $(1, 2)$, $(3, 4)$, $(5, 6)$, and so on. 
Note that they pairwise commute. Suppose that there exists an injective homomorphism 
$\varphi: G \to \GL_n(F)$ for some $n \geq 1$ and a field $F$. Let $A_1 = \varphi(1, 2)$, 
$A_2 = \varphi(3, 4)$, and so on. Observe that we have 
\[ \varphi((i, i+1)^2) = \varphi(\id) = I, \]
which implies that $A_1^2 = A_2^2 = \cdots = I$. We now recall another fact from linear algebra. 

{\sc Fact 2.} If the minimal polynomial of a matrix has distinct roots over the (algebraically closed) field $F$, then the matrix is diagonalizable. 

We see from above that the minimal polynomial of the $A_i$ must divide $x^2 - 1$, since 
$A_i^2 - I = 0$. As $x^2 - 1$ has distinct roots, it follows from Fact 2 that all the $A_i$ are diagonalizable. Moreover, by Fact 1, we can assume after a change of basis that each $A_i$ is of the form 
\[ A_i = \begin{pmatrix} \eps_{1,i} & & 0 \\ & \ddots & \\ 0 & & \eps_{n,i} \end{pmatrix} \]
where $\eps_{1,i}, \dots, \eps_{n,i} \in \{\pm 1\}$. Now we have a problem: there are only 
$2^n$ such matrices of the above form, and infinitely many positive integers. Thus, there 
exist positive integers $i < j$ such that $\varphi(A_i) = \varphi(A_j)$, so 
$\varphi$ is not injective, and this yields our contradiction.

Note that this argument needs an adjustment for an algebraically closed field of characteristic $2$, 
since $x^2 - 1 = (x - 1)^2$ does not have distinct roots. 
In such a case, we can proceed in the same way, except we use distinct $3$-cycles instead of $2$-cycles.
\end{pf}
\end{exmp}

We now turn to the notion of a group algebra.

\begin{defn}
The {\bf group algebra} of the group $G$ over the field $k$ is defined by 
\[ k[G] = \left\{ \sum_{g\in G} \alpha_g \cdot g : \alpha_g \in k,\, \alpha_g = 0 \text{ for all but 
finitely many $g$} \right\}. \]
\end{defn}

We note that $k[G]$ is a ring with a natural addition, and multiplication given by 
\[ \left( \sum_{g\in G} \alpha_g \cdot g \right) \left( \sum_{h \in G} \beta_h \cdot h \right) 
= \sum_{y \in G} \left( \sum_{(g, h) : gh = y} \alpha_j \cdot \beta_h \right) \cdot y. \]
Notice that the inner sum is finite because by definition, there are only finitely many 
non-zero $\alpha_g$ and $\beta_h$. 

\begin{remark}
We call $k[G]$ a group {\it algebra} because we have a ``copy'' of $k$ in $k[G]$ given by 
$\lambda \mapsto \lambda \cdot 1_G$ for elements $\lambda \in k$, with $\lambda \cdot g = g 
\cdot \lambda$ for all $g \in G$. We see that $k[G]$ is a $k$-vector space of dimension $|G|$. 
\end{remark}

\begin{exercise}
Show that $\C[S_3] \cong \C \times \C \times M_2(\C)$. Fun fact: using the naive approach 
to multiply matrices takes $O(n^3)$ operations, but applying this fact reduces the time complexity to 
$O(n^{2.373})$. 
\end{exercise}

We now prove a version of Cayley's theorem for group algebras. 

\begin{prop}
Let $G$ be a finite group with $n = |G|$. Then $G$ embeds into $\GL_n(k[G])$. 
\end{prop}
\begin{pf}
Let $G$ act on $k[G]$ by left multiplication. That is, for $g \in G$, define 
\[ L_g : k[G] \to k[G] : \sum_{h\in G} \alpha_h \cdot h \mapsto \sum_{h\in G} \alpha_h \cdot (g \cdot h). \]
Observe that for $g_1, g_2 \in G$, we have 
\[ L_{g_1} \circ L_{g_2}(x) = L_{g_1}(L_{g_2}(x)) = L_{g_1}(g_2 \cdot x) = g_1 \cdot (g_2 \cdot x)
= (g_1 \cdot g_2) \cdot x = L_{g_1g_2}(x). \] 
For the second equality, we can think of $G$ as sitting inside $k[G]$ by identifying 
$g \in G$ with $1 \cdot g \in k[G]$, so we simply have multiplication in the group algebra. 
Hence, we see that the map $L : G \to \GL_n(k[G]) : g \mapsto L_g$ is a group homomorphism. Finally, if 
$L_g$ is the identity matrix, then $g \cdot x = x$ for all $x \in k[G]$. This implies that 
$g \cdot 1 = 1$ and so $g = 1$, so $\ker L = \{1\}$. Thus, $L$ is injective and is the desired embedding.
\end{pf}

Later in the course, we will prove the following important theorem. In short, it states that if 
$G$ is a finite group and $k$ is an algebraically closed field of characteristic zero, then $k[G]$ is
isomorphic to a finite direct product of matrix rings over $k$. The isomorphism and these matrix rings
will completely determine the representation of the group $G$. 

\begin{thm}
Let $G$ be a finite group and let $k$ be an algebraically closed field of characteristic zero. 
Then we have 
\[ k[G] \cong \prod_{i=1}^s M_{n_i}(k), \]
where
\begin{enumerate}[(1)]
    \item $s$ is the number of conjugacy classes of $G$;
    \item $n_1^2 + n_2^2 + \cdots + n_s^2 = |G|$;
    \item $|\{i : n_i = 1\}| = |G/G'|$, where $G'$ denotes the commutator subgroup; and 
    \item $n_i \mid |G|$ for all $1 \leq i \leq s$. 
\end{enumerate}
\end{thm}

As a corollary of this theorem, one can prove Exercise 1.5 by noting that $\C[S_3]$ has three 
conjugacy classes: $\{(1)\}$, $\{(12), (13), (23)\}$, and $\{(123), (132)\}$. 

To finish off the lecture, we give one more interesting linear algebra fact. 

{\sc Fact.} If $q = p^j$ where $p$ is prime and $j \geq 1$, then 
\[ \lvert\GL_n(\F_q)\rvert = \prod_{i=0}^{n-1} (q^n - q^i). \]
It is not hard to see why. Let $A$ be an invertible $n \times n$ matrix, and note that $A$ must 
have linearly independent columns. For the first column, say $v_1$, we have 
$q^n - 1$ choices as we can pick any vector except the zero vector. For the second column, 
we can choose any vector except those in the span of $v_1$, which yields $q^n - q$ choices. 
One can repeat this argument to obtain the result. 

\section{September 10, 2021}

Recall that for a ring $R$, the {\bf center} of $R$ is the set 
\[ Z(R) = \{z \in R : zr = rz \text{ for all } r \in R\}. \]
Note that $Z(R)$ is a commutative subring of $R$. We can also write $[z, r] = 0$ to 
denote that $zr = rz$. 

\begin{defn}
Let $k$ be a field. We say that $R$ is a {\bf $k$-algebra} if 
\begin{enumerate}[(a)]
    \item $R$ is a ring; and 
    \item there exists a homomorphism $\phi : k \to Z(R)$ sending $1_k$ to $1_R$ (we assume that $\phi$ is injective).
\end{enumerate}
\end{defn}

Notice that if we identify $k$ with $\phi(k) \subseteq R$, we have a ``copy'' of $k$ in $R$. This 
means that $R$ is a $k$-vector space in addition to being a ring. 

\begin{defn}
Let $k$ be a field, and let $R$ and $S$ be $k$-algebras.
A {\bf $k$-algebra homomorphism} is a ring homomorphism $\psi : R \to S$ such that $\psi(\lambda) = \lambda$ for all 
$\lambda \in k$. 
\end{defn}

We also have the notion of a module. A module over a ring is a generalization of a 
vector space over a field. 

\begin{defn}
A {\bf (left) $R$-module $M$} is an abelian group $(M, +)$ equipped with a map 
\[ \cdot : R \times M \to M \] 
such that for all $r, s \in R$ and $m, n \in M$, we have 
\begin{enumerate}[(a)]
    \item $(r + s) \cdot m = r \cdot m + s \cdot m$; 
    \item $r \cdot (m + n) = r \cdot m + r \cdot n$;
    \item $(r \cdot s) \cdot m = r \cdot (s \cdot m)$; and 
    \item $1 \cdot m = m$.
\end{enumerate}
\end{defn}

\begin{exmp}
Let $R = M_n(\C)$ and $M = \C^{n \times 1}$, the set of column vectors of length $n$. One can check that 
$M$ is a (left) $R$-module equipped with the operation of matrix-vector multiplication.
\end{exmp}

\begin{exmp}
Left ideals are $R$-modules by left multiplication. In particular, $R$ itself is a left $R$-module. If 
$L$ is a left ideal of $R$ (and we write $L \trianglelefteq_\ell R$), then $L$ is a 
submodule of $R$ as a left $R$-module.
\end{exmp}

\begin{remark}
If $M_1$ and $M_2$ are $R$-modules, we define the set 
\[ \Hom_R(M_1, M_2) = \{f : M_1 \to M_2 \mid f \text{ is $R$-linear}\}. \]
That is, we have $f(r \cdot m_1 + m_2) = r \cdot f(m_1) + f(m_2)$ for all $r \in R$ and $m_1, m_2 \in M_1$. 
In the case that $M_1 = M_2 = M$, we write 
\[ \Hom_R(M_1, M_2) = \End_R(M), \]
the endomorphisms from $M$ to itself. Note that $\End_R(M)$ is a ring with composition as 
the multiplication operation, as the set of linear transformations from a vector space to itself 
forms a ring. 
\end{remark}

{\sc Fact.} If $R$ is a $k$-algebra and $M$ is a left $R$-module, then $M$ is a $k$-vector space. 

In a sense, this is clear. We know that $M$ already has scalar multiplication by $R$, and we have a 
copy of $k$ sitting inside of $R$, so if we restrict $R$ to $k$, we obtain scalar multiplication by $k$.

\begin{defn}
A left $R$-module $M$ is {\bf simple} if $M \neq (0)$, and $(0)$ and $M$ are the only submodules of $M$.
\end{defn}

\begin{exmp}
Let $R = \Z$ and $M = \Z/p\Z$ for a prime $p$. Suppose that $N \subseteq M$ is a submodule with 
$N \neq (0)$. Then there exists $i \in \{1, \dots, p-1\}$ such that the coset $[i]_p$ is in $N$. 
But this implies that $[i]_p^{-1} \cdot [i]_p = [1]_p \in N$. It follows that $N = \Z/p\Z$ and hence 
$M$ is simple. 
\end{exmp}

\begin{exercise}
Let $R = M_n(\C)$ and $M = \C^{n\times 1}$. Show that $M$ is simple. (Hint: If we are given a 
non-zero vector, then we can extend it to a basis, and we can always find a linear transformation to 
send a basis wherever we like.)
\end{exercise}

\begin{defn}
A left ideal $L$ of $R$ is a {\bf maximal left ideal} if 
\begin{enumerate}[(a)]
    \item $L \subsetneq R$; and
    \item there does not exist a left ideal $L'$ such that $L \subsetneq L' \subsetneq R$. 
\end{enumerate} 
\end{defn}

\begin{exercise}
Let $R$ be a non-zero ring (that is, $0_R \neq 1_R$). If $L$ is a proper left ideal of $R$, then 
there exists a maximal left ideal $M$ such that $L \subseteq M$. In particular, taking 
$L = (0)$ shows that a maximal left ideal always exists. 
\end{exercise}

{\sc Fact.} If $R$ is a ring and $M$ is a simple left $R$-module, then $M$ is isomorphic to 
$R/L$ (as $R$-modules), where $L$ is a maximal left ideal. 

We give a sketch of the proof of this fact. First, pick $m_0 \in M \setminus \{0\}$. Consider 
$R$ as a left $R$-module over itself (by left multiplication), and define 
\begin{align*}
    \Phi : R &\to M \\ r &\mapsto r \cdot m_0. 
\end{align*}
\begin{enumerate}[(1)]
    \item Check that $\Phi$ is an $R$-module homomorphism. 
    \item Show that $\ker\Phi = L$ is a left ideal.
    \item Show that $\im\Phi$ is a non-zero submodule of $M$, and hence must be equal to $M$ 
    (as $M$ is simple).
\end{enumerate}
By the first isomorphism theorem, it follows that $R/\ker\Phi = R/L$ is isomorphic to $M = \im\Phi$. 

Finally, why must $L$ be maximal? Similarly to the correspondence for groups and rings, we also have 
correspondence for modules. Indeed, for a module $M$ and a submodule $N \trianglelefteq M$, there 
is a bijection between the submodules of $M/N$ and the submodules of $M$ that contain $N$. 

We have the canonical projection 
\begin{align*}
    \pi : M &\to M/N \\ m &\mapsto m + N.
\end{align*} 
For a submodule $Q$ of $M/N$, notice that 
\[ \pi^{-1}(Q) = \{m \in M : \pi(m) \in Q\} \] 
is a submodule of $M$ which contains $N$, since 
\[ N = \pi^{-1}(0) \subseteq \pi^{-1}(Q). \]
Now, there is a correspondence between submodules of $M$ and the left ideals of $R$ containing $L$. 
Since $M$ is simple, it has two submodules. Thus, there are two left ideals of $R$ containing $L$, namely $L$ and $R$. Therefore, there is no ideal $L'$ such that $L \subsetneq L' \subsetneq R$, so $L$ is 
maximal.

As a corollary of the above fact, we obtain the following. 

\begin{cor}
If $R$ is a finite dimensional $k$-algebra and $M$ is a simple left $R$-module, then $M$ is a 
finite dimensional $k$-vector space.
\end{cor}

\begin{pf}
We showed that there was a surjection $\Phi : R \to M : r \mapsto r \cdot m_0$
for $m_0 \neq 0$. The fact that $\Phi$ is an $R$-module homomorphism implies that it is $k$-linear, 
as there is a copy of $k$ sitting inside of $R$. Since $R$ is finite dimensional, it follows that 
$M$ is also finite dimensional.
\end{pf}

For the remainder of this lecture, we will turn to proving Schur's lemma. 

\begin{defn}
A {\bf division ring} $\Delta$ is a ring in which every non-zero element $a \in \Delta$ 
has a multiplicative inverse $b \in \Delta$ (that is, $ab = ba = 1$).
\end{defn}

\begin{exmp}
Consider the quaternions 
\[ \mathbb{H} = \{a + ib + jc + kd : a, b, c, d \in \R\} \]
with properties 
$i^2 = j^2 = k^2 = 1$, $ij = k$, and $ji = -k$. Check that $\mathbb{H}$ is a division ring, 
but not a field.
\end{exmp}

\begin{thm}[Schur's lemma]
Let $R$ be a ring and let $M$ be a simple left $R$-module. Then $\End_R(M)$ is a division ring.
\end{thm}
\begin{pf}
We already know that $\End_R(M)$ is a ring, so we only need to show that every non-zero element 
has an inverse. Let $0 \neq f \in \End_R(M)$, which is a linear map $f : M \to M$. Notice that 
$\ker f$ is a submodule of $M$. Since $M$ is simple, $\ker f$ is either $(0)$ or $M$. The latter 
is impossible as this would mean that $f = 0$, so we have $\ker f = (0)$, so $f$ is injective. 
Similarly, $\im f$ is a submodule of $M$ and $\im f \neq (0)$ since $f \neq 0$, so 
$\im f = M$ since $M$ is simple. This shows that $f$ is surjective. Therefore, $f$ has a 
set theoretic inverse $g$. We now show that $g \in \End_R(M)$; that is, 
\[ g(r \cdot m_1 + m_2) = r \cdot g(m_1) + g(m_2) \]
for all $r \in R$ and $m_1, m_2 \in M$. But $f$ is bijective, so this is equivalent to showing that 
\[ f(g(r \cdot m_1 + m_2)) = f(r \cdot g(m_1) + g(m_2)). \]
This is indeed the case; we have 
\[ f(g(r \cdot m_1 + m_2)) = r \cdot m_1 + m_2 = r \cdot f \circ g(m_1) + f \circ g(m_2) = f(r \cdot g(m_1) + g(m_2)). \qedhere \]
\end{pf}

\section{September 13, 2021}
Recall the setting from before: we had a ring $R$, a simple left $R$-module $M$,
and $\Delta := \End_R(M)$, which we showed was a division ring. 

\begin{prop}~
\begin{enumerate}[(a)]
    \item If $R$ is a $k$-algebra for a field $k$, then $\Delta := \End_R(M)$ is also a $k$-algebra. 
    \item If $k$ is algebraically closed and $R$ is a finite-dimensional $k$-algebra, then $\Delta 
    \cong k$. 
\end{enumerate}
\end{prop}
\begin{pf}~
\begin{enumerate}[(a)]
    \item For each $\lambda \in k$, define the map 
    \begin{align*}
        \Phi_\lambda : M &\to M, \\ \lambda &\mapsto \lambda \cdot m. 
    \end{align*}
    By $\lambda \cdot m$, we mean that we have a copy $k \subseteq Z(R) \subseteq R$, and $M$ 
    is an $R$-module and hence a $k$-vector space. Note that 
    \begin{align*}
        \Phi_\lambda(r \cdot m_1 + m_2) &= \lambda \cdot (r \cdot m_1 + m_2) \\
        &= \lambda \cdot r \cdot m_1 + \lambda \cdot m_2 \\
        &= r \cdot \lambda \cdot m_1 + \lambda \cdot m_2 & (\text{since $k \subseteq Z(R)$}) \\
        &= r \cdot \Phi_\lambda(m_1) + \Phi_\lambda(m_2).
    \end{align*}
    Therefore, $\Phi_\lambda$ is $R$-linear and so $\Phi_\lambda \in \End_R(M)$. However, it is not 
    enough to show that each $\Phi_\lambda$ is in $\Delta = \End_R(M)$. We also require that our map 
    $\lambda \mapsto \Phi_\lambda$ is a map $k \to Z(\Delta)$; in other words, we also need 
    $\Phi_\lambda \in Z(\Delta)$. Let $\psi \in \Delta$, and observe that 
    \[ \Phi_\lambda \circ \psi(m) = \lambda \cdot \psi(m) = \psi(\lambda \cdot m) = \psi \circ 
    \Phi_\lambda(m), \]
    where the second equality is because the $R$-linearity of $\psi$ implies $k$-linearity.
    
    \item Recall that if $R$ is a finite dimensional $k$-algebra, then $M$ is a finite dimensional 
    $k$-vector space (see Corollary 2.12). Suppose that $\dim_k M =: n < \infty$. Then we have 
    \begin{align*}
        \Delta = \End_R(M) &\subseteq \End_k(M) & (\text{$R$-linearity imposes more conditions than 
        $k$-linearity}) \\
        &\cong\End_k(k^n) & (\text{$M$ is an $n$-dimensional $k$-vector space}) \\
        &\cong M_n(k) & (\text{$k$-linear maps $k^n \to k^n$ are the $n \times n$ matrices})
    \end{align*}
    and thus $\dim_k \Delta = m \leq n^2 < \infty$ for some $m \in \Z$. 
    
    We now show that $\Delta \cong k$. Indeed, pick $a \in \Delta$. Notice that $a$ commutes 
    with all elements of $k$ since $k \subseteq Z(\Delta)$, where we can identify $k$ with the set 
    $\{\Phi_\lambda : \lambda \in k\}$. Consider 
    \[ k \subseteq k(a) \subseteq \Delta, \]
    where $k(a)$ is the field formed from adjoining $a$ to $k$; it is a field because $\Delta$ 
    is a division ring and hence $a$ is invertible. Thus, $\dim_k k(a) \leq \dim_k \Delta = m < \infty$. 
    Therefore, $\{1, a, a^2, \dots, a^m\}$ is a linearly dependent set over $k$, so there 
    exist elements $c_0, c_1, \dots, c_m \in k$, not all zero, such that 
    \[ c_0 + c_1a + \cdots + c_m a^m = 0. \]
    But $k$ is algebraically closed, so $a \in k$. This implies that $\Delta \subseteq k$, and hence 
    $\Delta \cong k$. \qedhere 
\end{enumerate}
\end{pf}

\begin{exercise}
Let $\Delta$ be a division ring and let $M$ be a left $\Delta$-module. Then $M$ has a basis 
$B \subseteq M$. That is, there do not exist $\delta_1, \dots, \delta_m \in \Delta$ such that 
\[ \delta_1 b_1 + \cdots + \delta_m b_m = 0 \]
for distinct $b_1, \dots, b_m \in B$, and for all $m \in M$, we can write 
\[ m = \sum_{b \in B} \delta_b b \]
where $\delta_b = 0$ for all but finitely many $b \in B$. (Hint: Use Zorn's lemma.)
\end{exercise}

For this reason, instead of calling it a left $\Delta$-module, we can call it a 
{\bf left $\Delta$-vector space}.  

\begin{remark}
Let $R$ be a ring, let $M$ be a simple left $R$-module, and let $\Delta = \End_R(M)$. Then $M$ 
is a left $\Delta$-vector space. 
\end{remark}
\begin{pf}
Recall that $\delta \in \Delta$ is an $R$-linear map from $M$ to itself. We can consider the operation 
\begin{align*}
    \Delta \times M &\to M, \\
    (\delta, m) &\mapsto \delta \cdot m := \delta(m). 
\end{align*}
It is straightforward to see that 
\begin{itemize}
    \item $(\delta_1 + \delta_2) \cdot m = \delta_1 \cdot m + \delta_2 \cdot m$, 
    \item $\delta \cdot (m_1 + m_2) = \delta \cdot m_1 + \delta \cdot m_2$, 
    \item $\delta \cdot (\delta \cdot m) = (\delta_1 \cdot \delta_2) \cdot m$, and 
    \item $\id \cdot \,m = m$,
\end{itemize} so $M$ is a left $\Delta$-vector space. 
\end{pf}

\begin{exmp}
Let $R = M_n(\C)$ and recall that $M = \C^{n\times 1}$ is a simple left $R$-module. We have 
\[ \Delta = \C = \{\Phi_\lambda : \lambda \in \C\}, \]
where each $\Phi_\lambda(m) = \lambda \cdot m$ for each $\lambda \in \C$. One can show that 
$\Phi(Av) = A \cdot \Phi(v)$ for all $A \in M_n(\C)$ and $v \in \C^{n\times 1}$ only when 
$\Phi = \Phi_\lambda$ for some $\lambda \in \C$. 
\end{exmp}

We now state the Jacobson Density Theorem, which we will prove in the next lecture. 

\begin{thm}[Jacobson Density Theorem]
Let $R$ be a ring, let $M$ be a simple left $R$-module, and let $\Delta = \End_R(M)$. Then we have a 
ring $\End_\Delta(M)$ and a map 
\begin{align*}
    \Phi : R &\to \End_\Delta(M) \\
    r &\mapsto \Phi_r : M \to M \\
    & \hspace{1.33cm} m \mapsto r \cdot m.
\end{align*}
Moreover, we have the following properties. 
\begin{enumerate}[(1)]
    \item The map $\Phi$ is a ring homomorphism and if $R$ is a $k$-algebra, then $\Phi$ is a 
    $k$-algebra homomorphism.
    \item The kernel of $\Phi$ is the annihilator of $M$; that is, 
    \[ \ker\Phi = \{r \in R : r \cdot m = 0 \text{ for all } m \in M\}. \]
    \item {\bf Density:} If $m_1, \dots, m_n \in M$ are left linearly independent over $\Delta$ and 
    $w_1, \dots, w_n \in M$ are arbitrary elements, then there exists $r \in R$ such that 
    \[ \Phi_r(m_i) = w_i \]
    for all $1 \leq i \leq n$ (in particular, we can send finite linear combinations wherever we like).
\end{enumerate}
\end{thm}

For now, let's see why the maps $\Phi_r$ are $\Delta$-linear so that $\Phi_r \in \End_\Delta(M)$
for each $r \in R$. 
Let $\delta \in \Delta$ and $m_1, m_2 \in M$. Then we have 
\begin{align*}
    \Phi_r(\delta \cdot m_1 + m_2) &= r \cdot (\delta \cdot m_1 + m_2) \\
    &= r \cdot (\delta \cdot m_1) + r \cdot m_2 & (\text{multiplication by $R$ is linear}) \\
    &= r \cdot \delta(m_1) + r \cdot m_2 & (\text{we have $\delta \in \Delta$}) \\
    &= \delta(r \cdot m_1) + r \cdot m_2 & (\text{$\delta$ is an $R$-linear map}) \\
    &= \delta \cdot \Phi_r(m_1 ) + \Phi_r(m_2),
\end{align*}
so $\Phi_r$ is $\Delta$-linear as desired. 
\section{Proof of the Jacobson density theorem (09/15/2021)}

We now prove the Jacobson Density Theorem, which we stated in the previous lecture. Recall that 
$R$ is a ring, $M$ is a simple left $R$-module, and $\Delta = \End_R(M)$. We already showed that 
$\End_\Delta(M)$ is a ring, and we defined the map 
\begin{align*}
    \Phi : R &\to \End_\Delta(M) \\
    r &\mapsto \Phi_r : M \to M \\
    & \hspace{1.33cm} m \mapsto r \cdot m.
\end{align*}

{\sc Proof of the Jacobson Density Theorem.}
\begin{enumerate}[(1)]
    \item First, observe that for all $r_1, r_2 \in R$ and $m \in M$, we have
    \begin{align*}
        \Phi(r_1r_2)(m) &= \Phi_{r_1r_2}(m) \\
        &= (r_1 \cdot r_2) \cdot m \\
        &= r_1 \cdot (r_2 \cdot m) \\
        &= r_1 \cdot \Phi_{r_2}(m) \\
        &= \Phi_{r_1} \circ \Phi_{r_2}(m) \\
        &= \Phi(r_1) \circ \Phi(r_2)(m).
    \end{align*}
    Similarly, one can show that $\Phi(r_1 + r_2) = \Phi(r_1) + \Phi(r_2)$ and $\Phi(1) = \id_M$, so 
    $\Phi$ is a ring homomorphism. Moreover, if $R$ is a $k$-algebra, then we can identify 
    $k$ with the set $\{\Phi_\lambda : \lambda \in k\}$, so we have a copy of $k$ in $Z(\Delta)$. 
    
    \item Notice that 
    \begin{align*}
        r \in \ker\Phi &\iff \Phi_r : M \to M \text{ is the zero map} \\
        &\iff \Phi_r(m) = 0 \text{ for all $m \in M$} \\
        &\iff r \cdot m = 0 \text{ for all $m \in M$,}
    \end{align*}
    where the last equivalence follows since we defined $\Phi_r$ to be left multiplication by $r$.
    
    \item We will proceed by induction on $n$. For $n = 1$, linear independence just means that 
    $m_1 \neq 0$. For arbitrary $w_1 \in R$, we need to show that there exists $r \in R$ such that 
    $\Phi_r(m_1) = r \cdot m_1 = w_1$. Let 
    \[ N = \{s \cdot m_1 : s \in R\} \subseteq M, \]
    which is an $R$-submodule of $M$. Notice that $N \neq (0)$ since $m_1 \neq 0$. Since $M$ is 
    simple, we must have $N = M$. In particular, we see that $w_1 \in N$, so there exists 
    $r \in R$ such that $r \cdot m_1 = w_1$. 
    
    Assume the result holds for $1 \leq k \leq n-1$ where $n \geq 2$. Let $m_1, \dots, m_n$ be 
    linearly independent over $\Delta$, and let $w_1, \dots, w_n \in M$ be arbitrary. We wish to find 
    $r \in R$ such that $r \cdot m_i = w_i$ for all $1 \leq i \leq n$. 
    
    By the induction hypothesis, there exists $a \in R$ such that 
    \[ a \cdot m_i = w_i \]
    for all $1 \leq i \leq n-1$. However, we don't know that $a \cdot m_n = w_n$, so we will set 
    $a \cdot m_n =: w \in M$. 
    
    {\sc Claim.} There exists $r \in R$ such that 
    \[ r \cdot m_1 = \cdots = r \cdot m_{n-1} = 0 \]
    and $r \cdot m_n =: w' \neq 0$. 
    
    To complete the proof, it is enough to prove this claim. To see why, suppose we know the claim holds. 
    We know from the base case 
    that we can send $w'$ wherever we like; in particular, there exists $b \in R$ such that 
    $b \cdot w' = w_n - w$. Notice that we have 
    \[ (a + b \cdot r) \cdot m_i = \begin{cases}
    w_i & \text{if $1 \leq i \leq n-1$,} \\ w + (w_n - w) & \text{if $i = n$.} \end{cases} \]
    Thus, choosing $a + b \cdot r \in R$ does the trick.
    
    {\sc Proof of Claim.} Suppose that no such $r \in R$ exists. In particular, if 
    \[ r \cdot m_1 = \cdots = r \cdot m_{n-1} = 0, \]
    then this will force $r \cdot m_n = 0$ as well. For $a_1, \dots, a_{n-1} \in M$, we know by the induction hypothesis that there exists 
    $s \in R$ such that 
    \[ s \cdot m_i = a_i \]
    for all $1 \leq i \leq n-1$. We can define the map $\theta : M^{n-1} \to M$ by 
    \begin{align*}
        \theta(a_1, \dots, a_{n-1}) = s \cdot m_n. 
    \end{align*}  
    Is this well-defined? We know such an $s$ exists, but we aren't guaranteed that it is unique. 
    Suppose that $s_1, s_2 \in R$ are such that 
    \[ a_i = s_1 \cdot m_i = s_2 \cdot m_i \]
    for all $1 \leq i \leq n-1$. Then 
    \[ (s_1 - s_2) \cdot m_i = a_i - a_i = 0 \] 
    for all 
    $1 \leq i \leq n-1$, and by our assumption above, we obtain 
    \[ (s_1 - s_2) \cdot m_n = 0. \]
    This means that $s_1 \cdot m_n = s_2 \cdot m_n$, so even if the choice of $s$ is not unique, the 
    map $\theta$ is well-defined. We now show that $\theta$ is $R$-linear. For $b \in R$, we have 
    \begin{align*}
        \theta(b \cdot (a_1, \dots, a_n)) 
        &= \theta(b \cdot (s \cdot m_1, \dots, s \cdot m_{n-1})) \\
        &= \theta(b \cdot s \cdot (m_1, \dots, m_{n-1})) \\
        &= (b \cdot s) \cdot m_n \\
        &= b \cdot (s \cdot m_n) \\
        &= b \cdot \theta(s \cdot (m_1, \dots, m_{n-1})) \\
        &= b \cdot \theta(a_1, \dots, a_{n-1}).
    \end{align*}
    Addition follows from the module structure, and we leave it as an exercise.
    
    For $1 \leq j \leq n-1$, we define the canonical inclusion maps 
    \begin{align*}
        i_j : M &\to M^{n-1} \\
        m &\mapsto (0, \dots, 0, m, 0, \dots, 0),
    \end{align*}
    where $m$ is placed in the $j$-th coordinate. It is easy to see that the $i_j$ are $R$-linear. 
    We have 
    \[ M \xrightarrow{i_j} M^{n-1} \xrightarrow{\theta} M \]
    where $i_j$ and $\theta$ are both $R$-linear, so we have 
    \[ \delta_j := \theta \circ i_j \in \Delta = \End_R(M). \]
    Now, we obtain 
    \begin{align*}
        \delta_1 \cdot m_1 + \cdots + \delta_{n-1} \cdot m_{n-1} 
        &= \delta_1(m_1) + \cdots + \delta_{n-1}(m_{n-1}) \\
        &= \theta \circ i_1(m_1) + \cdots + \theta \circ i_{n-1}(m_{n-1}) \\
        &= \theta(i_1(m_1) + \cdots + i_{n-1}(m_{n-1})) \\
        &= \theta(m_1, \dots, m_{n-1}) \\
        &= \theta(1 \cdot m_1, \dots, 1 \cdot m_{n-1}) \\
        &= 1 \cdot m_n \\
        &= m_n.
    \end{align*}
    This implies that $m_1, \dots, m_n$ are left linearly dependent over $\Delta$, which is a contradiction.
    Therefore, the claim holds, 
    and the result follows by induction. \qed
    
\end{enumerate}

\section{Faithful and simple modules, left Artinian rings (09/17/2021)}

Note that if $M$ is finite-dimensional as a $\Delta$-vector space, then the map $\Phi$ from the 
Jacobson Density Theorem is surjective. To see why, take a basis $\{m_1, \dots, m_n\}$ for $M$ as a 
$\Delta$-vector space. If $f \in \End_\Delta(M)$, there exist elements $w_1, \dots, w_n \in M$ 
such that $f(m_i) = w_i$ for all $1 \leq i \leq n$. But by the Jacobson Density Theorem, there
exists $r \in R$ such that $r \cdot m_i = w_i$ for all $1 \leq i \leq n$. 
Note that a linear transformation is completely 
determined by where the basis is sent, so $f = \Phi_r$. 

Recall that by (2) in the Jacobson Density Theorem, we have 
\[ \ker \Phi = \Ann_R(M) := \{r \in R : r \cdot m = 0 \text{ for all } m \in M\}. \]

\begin{defn}{}
A left $R$-module $M$ is called {\bf faithful} if $\Ann_R(M) = (0)$. 
\end{defn}

From above, we see that $M$ is faithful if and only if $\Phi$ is injective. 

\begin{defn}{}
A ring $R$ is said to be {\bf primitive} if it has a faithful simple module.
\end{defn}

Putting our above observations together, we obtain the following result. 

\begin{cor}{}
If $R$ has a faithful simple left $R$-module $M$ such that $\dim_\Delta M < \infty$ where 
$\Delta = \End_R(M)$, then the map 
\[ \Phi : R \to \End_\Delta(M) \]
from the Jacobson Density Theorem is an isomorphism. 
\end{cor}
\begin{pf}
The assumption $\dim_\Delta(M) < \infty$ gives surjectivity, and since $M$ is faithful, 
$\Phi$ is also injective.
\end{pf}

\begin{cor}{}
If $k$ is an algebraically closed field, $R$ is a finite-dimensional $k$-algebra, and $M$ is a 
faithful simple left $R$-module, then $R \cong M_n(k)$ where $n = \dim_k(M)$. 
\end{cor}
\begin{pf}
By Proposition 3.1, we know that $\Delta = \End_R(M) \cong k$. Moreover, we showed that 
$\dim_k M = n < \infty$ in Corollary 2.12. Therefore, we see that 
\begin{align*}
    R &\cong \End_\Delta(M) & \text{(by Corollary 5.3)} \\
    &\cong \End_k(M) & \text{(since $\Delta \cong k$)} \\
    &\cong \End_k(k^n) & \text{($M$ is an $n$-dimensional $k$-vector space)} \\
    &\cong M_n(k) & \text{($k$-linear maps $k^n \to k^n$ are the $n \times n$ matrices)} 
\end{align*}
which completes the proof.
\end{pf}

\begin{defn}{}
Let $R$ be a ring. We say that $R$ is {\bf left Artinian} if every descending chain of left ideals of $R$
terminates. That is, for a chain of left ideals 
\[ L_1 \supseteq L_2 \supseteq L_3 \supseteq \cdots \]
of $R$, there exists $n \geq 1$ such that $L_n = L_m$ for all $m \geq n$. 
\end{defn}

\begin{exmp}{}
\begin{enumerate}[(1)]
    \item We see that $\Z$ is not left Artinian since we can take the chain of ideals 
    \[ 2\Z \supsetneq 4\Z \supsetneq 8\Z \supsetneq \cdots. \]
    \item Intuitively, $M_n(\C)$ is left Artinian since it is an $n^2$-dimensional $\C$-vector space, so
    for a chain of ideals
    \[ L_1 \supsetneq L_2 \supsetneq L_3 \supsetneq \cdots, \]
    the dimension must eventually decrease, so the chain terminates.
    \item Similarly, $\Z/6000\Z$ is left Artinian as it only has finitely many subsets, so the 
    sizes of the ideals in a descending chain must decrease. 
\end{enumerate}
\end{exmp}

We can generalize our observations from the previous example. We leave the proof as an exercise.

\begin{remark}{}
\begin{enumerate}[(1)]
    \item If $R$ is a finite-dimensional $k$-algebra, then $R$ is left Artinian.
    \item If $R$ is a finite ring, then $R$ is left Artinian.
\end{enumerate}
\end{remark}

\begin{defn}{}
Let $R$ be a ring. Let $I$ be a two-sided ideal of $R$. We say that $I$ is a {\bf nil ideal} 
if for every $x \in I$,  there exists $n = n(x) \geq 1$ such that $x^n = 0$ (that is, 
every element in $I$ is nilpotent). 
\end{defn}

\begin{exmp}{}
\begin{enumerate}[(1)]
    \item Let $R$ be any ring. Then $(0)$ is a nil ideal. 
    \item Let $R = \Z/2\Z$. Then $I = 6R = \{[0]_R, [6]_R\}$ is a nil ideal since 
    $[0]_R^1 = [0]_R$ and $[6]_R^2 = [36]_R = [0]_R$. 
    \item Let $R$ be the ring of $2 \times 2$ upper triangular matrices over $\C$; that is, 
    \[ R = \left\{ \begin{pmatrix} a&b\\0&c \end{pmatrix} : a, b,c \in \C \right\} \subseteq M_2(\C). \]
    Then one can check that 
    \[ I = \left\{ \begin{pmatrix} 0&\alpha\\0&0 \end{pmatrix} : \alpha \in \C \right\} \]
    is an ideal of $R$, and that it is a nil ideal (by squaring). 
\end{enumerate}
\end{exmp}

We now state the Artin-Wedderburn Theorem and give some remarks. 

\begin{theo}[Artin-Wedderburn]{}
Let $R$ be a left Artinian ring. If $R$ has no non-zero nil ideals, then there exists $s \geq 1$, 
division rings $D_1, \dots, D_s$, and integers $n_1, \dots, n_s \geq 1$ such that 
\[ R \cong M_{n_1}(D_1) \times \cdots \times M_{n_s}(D_s). \]
\end{theo}

\begin{remark}{}
\begin{enumerate}[(1)]
    \item If $k$ is an algebraically closed field and $R$ is a finite-dimensional $k$-algebra, then 
    \[ R \cong M_{n_1}(k) \times \cdots \times M_{n_s}(k). \]
    \item By using Exercise 3 on Assignment 1, we see that if $R$ is also finite, then 
    \[ R \cong M_{n_1}(\F_{q_1}) \times \cdots \times M_{n_s}(\F_{q_s}). \]
    This can be observed by noting that if $R$ is finite, then the division rings must also be finite. 
    \item If $R$ is also commutative, then 
    \[ R \cong F_1 \times \cdots \times F_s \]
    for fields $F_1, \dots, F_s$. This is because commutativity forces the matrix rings to be 
    $1 \times 1$. Moreover, the division rings must also be commutative, and so they are fields.
\end{enumerate}
\end{remark}

We finish the lecture by giving one last definition. 

\begin{defn}{}
Let $R$ be a ring. Then a proper two-sided ideal $P$ is called a {\bf prime ideal} if whenever $a, b \in R$
are such that $aRb = \{arb : r \in R\} \subseteq P$, we either have $a \in P$ or $b \in P$.
We say that $R$ is a {\bf prime ring} if $(0)$ is a prime ideal of $R$. 
\end{defn}

\begin{exmp}{}
Observe that $p\Z$ for a prime $p$ is a prime ideal of $\Z$. Indeed, if $a, b \in \Z$ are such that 
$a\Z b = ab\Z \subseteq p\Z$, then $p \mid ab$. This occurs if and only if $p \mid a$ or $p \mid b$, 
or equivalently, $a \in p\Z$ or $b = p\Z$. In fact, $\Z$ is a prime ring as it has $(0)$ as a prime 
ideal. 
\end{exmp}

\section{Simple rings, characterization of left Artinian rings (09/20/2021)}

When $R$ is commutative, notice that $R$ is prime if and only if $R$ is an integral domain. 
Indeed, observe that 
\begin{align*}
    R \text{ is prime} &\iff aRb = (0) \text{ implies } a = 0 \text{ or } b = 0 \\
    &\iff abR = (0) \text{ implies } a = 0 \text{ or } b = 0 \\
    &\iff ab \cdot 1 = 0 \text{ implies } a = 0 \text{ or } b = 0 \\
    &\iff ab = 0 \text{ implies } a = 0 \text{ or } b = 0 \\
    &\iff R \text{ is an integral domain.}
\end{align*}
Being prime can be thought of as an extension of being an integral domain in the case where $R$ 
is not a commutative ring. 

\begin{defn}{}
A ring $R$ is {\bf simple} if $(0)$ and $R$ are its only two-sided ideals.
\end{defn}

\begin{prop}{}
If $R$ is a simple ring, then $R$ is prime. 
\end{prop}
\begin{pf}
Let $R$ be simple. Suppose that $aRb = (0)$ with $a \neq 0$ and $b \neq 0$. Since $a \neq 0$, the 
two-sided ideal 
\[ RaR := \{u_1av_1 + \cdots + u_sav_s : s \geq 0,\, u_i, \dots, u_s, v_1, \dots, v_s \in R\} \]
is equal to $R$ by the simplicity of $R$. In particular, there exists $s \geq 1$ and 
$u_s, v_1, \dots, v_s \in R$ such that 
\begin{equation}
    1 = u_1av_1 + \cdots + u_sav_s. 
\end{equation}
Similarly, there exists $t \geq 1$ and $y_1, \dots, y_t, z_1, \dots, z_t \in R$ such that 
\begin{equation}
    1 = y_1bz_1 + \cdots + y_taz_t. 
\end{equation}
Multiplying equations (6.1) and (6.2) together gives 
\[ 1 \cdot 1 = (u_1av_1 + \cdots + u_sav_s)(y_1bz_1 + \cdots + y_taz_t) 
= \sum_{i=1}^s \sum_{j=1}^t u_i(av_iy_jb)z_j = 0, \]
where the last equality is because $aRb = (0)$ and thus $av_iy_jb = 0$ for any $1 \leq i \leq s$ 
and $1 \leq j \leq t$. 
\end{pf}

\begin{prop}{}
Let $D$ be a division ring and let $n \geq 1$. Then $M_n(D)$ is simple and hence prime. 
\end{prop}
\begin{pf}
Let $I$ be a non-zero ideal of $M_n(D)$. We want to show that $I = M_n(D)$. Since $I$ is 
non-zero, there exists a matrix 
\[ x = \begin{pmatrix} a_{11} & \cdots & a_{1n} \\ \vdots & \ddots & \vdots \\ a_{n1} & \cdots & a_{nn} \end{pmatrix} \in I \]
with each $a_{ij} \in D$ and $a_{i_0j_0} \neq 0$ for some $1 \leq i_0, j_0 \leq n$. For all 
$1 \leq i, j \leq n$, let $e_{ij}$ be the matrix where the $(i,j)$-th entry is $1$ and all other entries
are $0$. Then we find that 
\[ e_{ii_0}xe_{j_0j} = a_{i_0j_0}e_{ij} \in I. \]
Since $D$ is a division ring, we know that $a_{i_0j_0}$ has an inverse. It follows that 
\[ \begin{pmatrix} a_{i_0j_0}^{-1} & & 0 \\ & \ddots \\ 0 & & a_{i_0j_0}^{-1} \end{pmatrix} a_{i_0j_0} e_{ij} = e_{ij} \in I \]
for all $1 \leq i, j \leq n$. In particular, we obtain 
\[ e_{11} + \cdots + e_{nn} = 1 \in I, \]
so we conclude that $I = M_n(D)$ and hence $M_n(D)$ is simple. 
\end{pf}

We now prove a nice characterization of left Artinian rings. 

\begin{prop}{}
Let $R$ be a ring. Then $R$ is left Artinian if and only if whenever $S$ is a non-empty collection 
of left ideals of $R$, then $S$ has a minimal element with respect to inclusion. 
\end{prop}
\begin{pf}
For the forward direction, suppose that $R$ is left Artinian. Let $S$ be a non-empty collection of 
left ideals of $R$. Pick $L_1 \in S$. If $L_1$ is minimal, we are done. Otherwise, we can pick $L_2 
\in S$ such that $L_2 \subsetneq L_1$. We can continue this process, and since $R$ is left 
Artinian, this terminates at some step $L_n$ as we cannot have an infinite strictly descending chain.

For the converse, assume that for every non-empty collection $S$ of left ideals of $R$, then 
$S$ has a minimal element with respect to inclusion. Let $L_1 \supseteq L_2 \supseteq \cdots$ 
be a descending chain of left ideals of $R$. Let $S = \{L_1, L_2, \dots\}$. By assumption, there 
exists $n \geq 1$ such that $L_n$ is a minimal element of $S$. In particular, we have 
$L_n = L_m$ for all $m \geq n$, so $R$ is left Artinian.
\end{pf}

The next theorem will be a key step towards proving the Artin-Wedderburn theorem.

\begin{theo}{}
Let $R$ be a prime left Artinian ring. Then $R \cong M_n(D)$ for some $n \geq 1$ and division ring 
$D$. (In fact, the converse is also true.)
\end{theo}
\begin{pf}
Let $S = \{Ru : u \in R,\, u \neq 0\}$ be a collection of left ideals of $R$. Note that $S$ 
is non-empty since $R \cdot 1 = R \in S$. Then there exists a minimal element in $S$ by Proposition 
6.4, say $Rb$ for some $b \in R$. Notice that $Rb$ is a left $R$-module (since 
left ideals are left $R$-modules). 

First, we show that $M = Rb$ is simple. If $N \subsetneq M = Rb$ is a proper left ideal with 
$N \neq (0)$, then there exists $u \neq 0$ in $N$ such that 
\[ (0) \subsetneq Ru \subsetneq N \subsetneq M. \]
But we assumed that $M = Rb$ was minimal in $S$, which is a contradiction. 

Next, we show that $M = Rb$ is faithful; that is, $\Ann_R(M) = (0)$. Suppose there exists 
$a \neq 0$ in $\Ann_R(M)$. Then we have $aM = (0)$ and hence $aRb = (0)$. Since $R$ is prime,
it must be that $a = 0$ or $b = 0$. But we assumed that $a \neq 0$ and $b \neq 0$, so this is 
a contradiction. 

Now, we show that $\dim_\Delta M < \infty$ where $\Delta = \End_R(M)$. Suppose to the contrary 
that $M$ were an infinite-dimensional left $\Delta$-vector space. Then there exist elements 
$m_1, m_2, \dots \in M$ that are $\Delta$-linearly independent. By the Jacobson Density Theorem, 
for every $n \geq 1$, there exists $r_n \in R$ such that 
\[ r_nm_1 = r_nm_2 = \cdots = r_nm_{n-1} = 0 \]
and $r_nm_n \neq 0$. Define the left ideal 
\[ L_i = \{r \in R : rm_1 = rm_2 = \cdots = rm_i = 0\} \]
for all $i \geq 1$. Notice that $L_1 \supseteq L_2 \supseteq \cdots$ and $r_n \in L_{n+1} \setminus L_n$,
which implies that these are proper containments. But this is an infinite descending chain of 
left ideals, which contradicts the fact that $R$ is left Artinian. 

In the next lecture, we will finish off the proof of this theorem.
\end{pf}
\section{Opposite ring, spectrum of a ring, Nullstellensatz (09/22/2021)}

\begin{defn}{}
Let $S$ be a ring. The {\bf opposite ring} $S^{\op}$ of $S$ is defined to be another ring with 
the same elements and addition as $S$, but the multiplication $* : S^{\op} \times S^{\op} \to S^{\op}$ is given by 
\[ s_1 * s_2 := s_2 \cdot s_1, \]
where $\cdot$ denotes the multiplication in $S$. 
\end{defn}

\begin{remark}{}
\begin{enumerate}[(1)]
    \item If $S$ is commutative, then $S^{\op}$ is the same as $S$ (since $*$ is the same as $\cdot$).
    \item If $\Delta$ is a division ring, then $\Delta^{\op}$ is also a division ring. Indeed, 
    let $a \in \Delta^{\op}$ be non-zero. Then there exists $b \in \Delta$ such that 
    $a \cdot b = b \cdot a = 1_\Delta$, so we have $b * a = a * b = 1_{\Delta^{\op}}$. Hence, 
    $a$ is invertible. 
\end{enumerate}
\end{remark}

\begin{exercise}{}
Let $\Delta$ be a division ring. If $M$ is an $n$-dimensional left $\Delta$-vector space, then 
$\End_\Delta(M) \cong M_n(\Delta^{\op})$. Note that if $\Delta = k$ for a field $k$, then this is just saying that $\End_k(M) \cong M_n(k)$, which is a familiar fact.

Hint: Construct the map $\Psi : \End_\Delta(M) \to M_n(\Delta^{\op})$ as follows: pick a basis 
$\{m_1, \dots, m_n\}$ as a left $\Delta$-vector space. For $f \in \End_\Delta(M)$, we have 
\[ f(m_j) = \sum_{i=1}^n a_{ij} m_i \]
where $a_{ij} \in \Delta$ since $f$ is $\Delta$-linear. Define $\Psi(f) := (a_{ij})$, and 
show that $\Psi(f \circ g) = \Psi(g) \cdot \Psi(f) = \Psi(f) * \Psi(g)$. 
\end{exercise}

Last time, we were proving Theorem 6.5, which stated that if $R$ is a prime left Artinian ring, 
then $R \cong M_n(D)$ for some $n \geq 1$ and division ring $D$. We can now finish the proof. 

We can set $n := \dim_\Delta M$ as we showed that $M$ is a finite-dimensional left $\Delta$-vector space.
Then we have 
\[ R \cong \End_\Delta(M) \cong M_n(\Delta^{\op}) \]
by Exercise 7.3, and we are done since $D = \Delta^{\op}$ is a division ring. \qed

\begin{cor}{}
Let $k$ be an algebraically closed field, and let $R$ be a prime finite-dimensional $k$-algebra. Then 
$R \cong M_n(k)$. 
\end{cor}
\begin{pf}
Since $R$ is a finite-dimensional $k$-algebra, it is left Artinian (see Remark 5.7). Moreover,
Proposition 3.1 shows that $\Delta = \End_R(M) \cong k$ where $M$ is a simple left 
$R$-module, since $k$ is algebraically closed. 
\end{pf}

For the rest of the lecture, we will consider the connection between prime ideals and nil ideals. 

\begin{defn}{}
We define the {\bf spectrum} of a ring $R$ to be the set of all prime ideals of $R$,
denoted $\Spec(R)$. 
\end{defn}

\begin{exmp}{}
For $R = \Z$, we have $\Spec(\Z) = \{p\Z : p \text{ prime}\} \cup \{(0)\}$. 
\end{exmp}

\begin{exmp}{}
For $R = M_n(\Delta)$ for $\Delta$ a division ring, we have $\Spec(M_n(\Delta)) = \{(0)\}$ as we 
showed that $M_n(\Delta)$ is a simple ring in Proposition 6.3.
\end{exmp}

\begin{exmp}{}
For $R = \C[x, y]$, we have
\[ \Spec(\C[x, y]) = \{(0)\} \cup \{(f) : f \text{ irreducible}\} \cup \{(x-a, y-b) : a, b \in \C\}. \]
Note that $(0)$ is a prime ideal as $\C[x, y]$ is an integral domain. Every maximal ideal of 
$\C[x, y]$ is of the form $(x-a, y-b)$, and this is due to the Nullstellensatz which we will prove. 
The other prime ideals are generated by irreducible polynomials $f$. 
\end{exmp}

\begin{theo}[Nullstellensatz]{}
Let $\mathfrak{M}$ be a maximal ideal of $\C[x_1, \dots, x_n]$. Then there exist 
$a_1, \dots, a_n \in \C$ such that 
\[ \mathfrak M = (x_1-a_1, \dots, x_n-a_n). \]
\end{theo}
\begin{pf}
Let $F = \C[x_1, \dots, x_n]/\mathfrak M$, which is a field because $\mathfrak M$ is maximal. 
Note that $F \supseteq \C$. Moreover, we have $\dim_{\C} F \leq \aleph_0$ since 
$\C[x_1, \dots, x_n]$ has a basis $\{x_1^{i_1} \cdots x_n^{i_n} : i_1, \dots, i_n \geq 0\} 
\cong \N^n$, which is countable. 

We claim that $F$ is algebraic over $\C$. That is, if $t \in F$, then $F$ satisfies $p(t) = 0$ 
for some $p(x) \in \C[x] \setminus \{0\}$. Note that this implies $F = \C$ since $\C$ is 
algebraically closed. 

Let $t \in F \setminus \C$. Consider the set 
\[ S = \left\{ \frac1{t-\lambda} : \lambda \in \C\right\} \subseteq F. \] 
Then $S$ is linearly dependent since $\C$ is uncountable while $\dim_\C F \leq \aleph_0$. 
Thus, there exist distinct elements $\lambda_1, \dots, \lambda_n \in \C$ and $c_1, \dots, c_n 
\in \C$, not all zero, such that 
\[ \frac{c_1}{t - \lambda_1} + \cdots + \frac{c_n}{t - \lambda_n} = 0. \]
Multiplying by $\prod_{i=1}^n (t - \lambda_i)$ gives 
\[ \sum_{i=1}^n c_i \prod_{j\neq i} (t - \lambda_j) = p(t) = 0, \]
where $p(x)$ is a polynomial in $\C[x]$. Note that $p(x)$ is non-trivial since 
\[ p(\lambda_i) = c_i \prod_{j\neq i} (\lambda_i - \lambda_j) \neq 0. \]
This proves the claim, so $\C[x_1, \dots, x_n]/\mathfrak M \cong \C$. Hence, there is a homomorphism 
\[ \phi : \C[x_1, \dots, x_n] \to \C \] 
such that $\ker\phi = \mathfrak M$. Letting $a_i = \phi(x_i)$ for all $1 \leq i \leq n$, we have 
\[ \phi(x_1^{i_1} \cdots x_n^{i_n}) = a_1^{i_1} \cdots a_n^{i_n} \]
as $\phi$ is a homomorphism. Since the $x_1^{i_1} \cdots x_n^{i_n}$ form a basis for 
$\C[x_1, \dots, x_n]$, it then follows that $\phi$ is simply the evaluation at 
$(a_1, \dots, a_n)$ map. In particular, $\mathfrak M = (x_1 - a_1, \dots, x_n - a_n)$ as desired.
\end{pf}

\begin{theo}{}
Let $R$ be a ring. Then the intersection of all prime ideals 
\[ \bigcap_{P \in \Spec(R)} P \]
is a nil ideal of $R$. 
\end{theo}
\begin{pf}
Let $N = \bigcap_{P \in \Spec(R)} P$. Suppose that there exists some $x \in N$ which is not nilpotent,
and let 
\[ {\cal T} = \{1, x, x^2, x^3, \dots \}. \]
Notice that $0 \notin {\cal T}$ since $x$ is not nilpotent. Let 
\[ S = \{I \trianglelefteq R : I \cap {\cal T} = \varnothing\}. \]
We have $(0) \in S$, so $S$ is non-empty. We leave it as an exercise to show that $S$ has a maximal 
element $P$. 

We claim that $P$ is a prime ideal. Suppose otherwise, so that there exists $a, b \notin P$ 
such that $aRb \subseteq P$. Since $a \notin P$, we have $RaR + P \supsetneq P$, and similarly, 
$RbR + P \supsetneq P$ as $b \notin P$. As $P$ is maximal in $S$, this implies that 
$RaR + P \notin S$, so there exists $i \geq 1$ such that $x^i \in RaR + P$. 
Analogously, we have $RbR + P \notin S$, so there exists $j \geq 1$ such that $x^j \in RbR + P$. 
It follows that 
\[ x^{i+j} = x^i \cdot x^j \in (RaR + P)(RbR + P) \subseteq R(aRb)R + P \subseteq P. \]
But this is a contradiction since $P \in S$ implies that 
$P \cap {\cal T} = \varnothing$, but we have $x^{i+j} \in P \cap {\cal T}$. 
Thus, $P$ is a prime ideal, proving the claim. 

Now, we find that $x \in N \subseteq P$ since $P$ is prime and $N$ is the intersection of all 
the prime ideals. However, $x \in {\cal T}$, which again contradicts the fact that 
$P \cap {\cal T} = \varnothing$. We conclude that every $x \in N$ must be nilpotent, so 
$N$ is a nil ideal, as required.
\end{pf}

\section{Sun-tzu, the Artin-Wedderburn theorem (09/24/2021)}
We almost have all the tools we need to prove the Artin-Wedderburn theorem. 
First, we make a remark and prove a couple of results that we need. 

\begin{remark}{}
If $R$ is a left Artinian ring, then so is $R/P$ where $P$ is an ideal of $R$ by correspondence. 
Moreover, if $P$ is a prime ideal, then $R/P$ is a prime ring. The converse of this holds when 
$R$ is commutative. 
\end{remark}

\begin{lemma}{}
Let $R$ be a left Artinian ring. 
\begin{enumerate}[(1)]
    \item Every prime ideal of $R$ is a maximal ideal. 
    \item There are only finitely many prime ideals of $R$. 
    \item Let $P_1, \dots, P_s$ be all the prime ideals of $R$. Then for all $i = 1, \dots, s$, we have 
    \[ P_i + \bigcap_{j \neq i} P_j = R. \]
\end{enumerate}
\end{lemma}
\begin{pf}~
\begin{enumerate}[(1)]
    \item Let $P$ be a prime ideal of $R$. By Remark 8.1, $R/P$ is a prime left Artinian ring 
    and hence 
    \[ R/P \cong M_n(D) \]
    for a division ring $D$ and some $n \geq 1$ by Theorem 6.5. We know that $M_n(D)$ is simple by 
    Proposition 6.3, so its only ideals are $(0)$ and $M_n(D)$. In particular, $R/P$ is also simple. 
    By correspondence, there are only two ideals of $R$ that contain $P$. We already know that 
    $P$ and $R$ are ideals that contain $P$, so they are in fact all of them. Thus, $P$ is 
    maximal. 
    
    \item Suppose towards a contradiction that we have infinitely many distinct prime ideals 
    $P_1, P_2, \dots$ of $R$. Recall that if $I$ and $J$ are ideals of $R$, we define 
    \[ IJ = \left\{\sum_{k=1}^s i_kj_k : s \geq 1,\, i_k \in I,\, j_k \in J\right\}. \]
    We have a descending chain of ideals 
    \[ P_1 \supseteq P_1P_2 \supseteq P_1P_2P_3 \supseteq \cdots \]
    and since $R$ is left Artinian, this chain must terminate. Thus, there exists $n \geq 1$ such that 
    \[ P_1 \cdots P_n = P_1 \cdots P_n P_{n+1} \subseteq P_{n+1}. \]
    Since $P_1 \neq P_{n+1}$ and $P_1$ is a maximal ideal by (1), there exists some element 
    $a_1 \in P_1 \setminus P_{n+1}$. Similarly, for all $i = 1, \dots, n$, there exists 
    $a_i \in P_i \setminus P_{n+1}$. Then, we see that 
    \[ (a_1R)(a_2R)(a_3R) \cdots (a_{n-1}R) a_n \subseteq P_1P_2P_3 \cdots P_n \subseteq P_{n+1}, \]
    so we have 
    \begin{equation}
        a_1Ra_2R \cdots a_{n-1}Ra_n \subseteq P_{n+1}.
    \end{equation} 
    Since neither $a_1$ nor $a_2$ are in $P_{n+1}$, there exists $r_1 \in R$ such that 
    $a_1r_1a_2 \notin P_{n+1}$. This is because $P_{n+1}$ is a prime ideal of $R$, so $a_1, a_2 \notin P_{n+1}$ 
    implies that $a_1 R a_2 \nsubseteq P_{n+1}$. By (8.1), we find that 
    \[ (a_1Ra_2)Ra_3R \cdots a_{n-1}Ra_n \subseteq P_{n+1}. \]
    Now, $a_1ra_2$ and $a_3$ are both not in $P_{n+1}$, so there exists $r_2 \in R$ such that 
    $a_1r_1a_2r_2a_3 \notin P_{n+1}$ since $P_{n+1}$ is a prime ideal. Continuing in this manner, 
    we have elements 
    $r_1, r_2, \dots, r_{n-1} \in R$ such that 
    \[ a_1r_1a_2r_2 \cdots a_{n-1}r_{n-1}a_n \notin P_{n+1}, \]
    which contradicts (8.1) since 
    \[ a_1r_1a_2r_2 \cdots a_{n-1}r_{n-1}a_n \in a_1Ra_2R \cdots a_{n-1}Ra_n. \]
    We conclude that $R$ must have finitely many prime ideals.
    
    \item Let $P_1, \dots, P_s$ be the prime ideals of $R$, and fix $1 \leq i \leq s$. We claim that 
    \[ I_i := P_i + \bigcap_{j\neq i} P_j = R. \]
    Notice that $P_i \subseteq I_i \subseteq R$. Since $P_i$ is a maximal ideal by part (1), we either
    have $I_i = P_i$ or $I_i = R$. Suppose towards a contradiction that $I_i = P_i$, and recall the fact
    that $I + J = I$ if and only if $J \subseteq I$. Then we have 
    \[ \bigcap_{j\neq i} P_j \subseteq P_i, \]
    which implies that 
    \[ P_1P_2 \cdots P_{i-1}P_{i+1} \cdots P_s \subseteq \bigcap_{j\neq i} P_j \subseteq P_i. \]
    Using the same argument as the proof of (2), we can show that $P_1P_2 \cdots P_{i-1}P_{i+1} 
    \cdots P_s$ cannot be contained in the prime ideal $P_i$ as each ideal $P_j$ with $j \neq i$ is 
    prime. This contradicts our assumption that $I_i = P_i$, so we must have 
    $I_i = R$. \qedhere 
\end{enumerate}
\end{pf}

\begin{theo}[Sun-tzu]{}
Let $R$ be a ring, and let $I_1, \dots, I_s$ be two-sided ideals of $R$ such that 
\begin{enumerate}[(i)]
    \item $\bigcap_{j=1}^s I_j = (0)$, and 
    \item for all $i = 1, \dots, s$, we have $I_i + \bigcap_{j\neq i} I_j = R$. 
\end{enumerate}
Then we have 
\[ R \cong \prod_{i=1}^s R/I_i. \]
\end{theo}
\begin{pf}
For $i = 1, \dots, s$, we have a canonical surjection 
\begin{align*} \pi_i : R &\to R/I_i \\ r &\mapsto r + I_i, \end{align*}
which is a ring homomorphism. We define 
\begin{align*} \Psi : R &\to R/I_1 \times \cdots \times R/I_s \\ r &\mapsto (\pi_1(r), \dots, \pi_s(r)). 
\end{align*}
Note that since $\pi_1, \dots, \pi_s$ are ring homomorphisms, $\Psi$ is also a ring homomorphism. 
We now show that $\Psi$ is an isomorphism. Notice that 
\[ \ker\Psi = \{r \in R : \Psi(r) = 0\} = \{r \in R : \pi_1(r) = \cdots = \pi_s(r) = 0\} 
= \bigcap_{j=1}^s I_j = (0) \]
where the last equality follows from (i), so $\Psi$ is injective. Now, fix $1 \leq i \leq s$. 
By (ii), we have 
\[ I_i + \bigcap_{j\neq i} I_j = R, \]
so we can find $a_i \in I_i$ and $b_i \in \bigcap_{j\neq i} I_j$ such that $a_i + b_i = 1$. 
We can write $b_i = 1 - a_i$, so we find that 
\[ \pi_i(b_i) = \pi_i(1) - \pi_i(a_i) = (1 + I_i) - (0 + I_i) = 1 + I_i \]
since $a_i \in I_i$. 
On the other hand, when $n \neq i$, we have $b_i \in \bigcap_{j\neq i} I_j \subseteq I_n$, which 
gives $\pi_n(b_i) = 0 + I_n$. It follows that 
\begin{align*} \Psi(b_i) &= (\pi_1(b_i), \pi_2(b_i), \dots, \pi_i(b_i), \dots, \pi_s(b_i)) \\
&= (0 + I_1, 0 + I_2,\, \dots , 0 + I_{i-1}, 1 + I_i, 0 + I_{i+1},\, \dots, 0 + I_s). \end{align*}
Therefore, given $r_1, \dots, r_s \in R$, we have $r = r_1b_1 + \cdots + r_sb_s \in R$, and we see that
\[ \Psi(r) = \Psi(r_1b_1 + \cdots + r_sb_s) = \Psi(r_1 + I_1,\, \dots, r_s + I_s). \]
We conclude that $\Psi$ is surjective, so it is an isomorphism, as required. 
\end{pf}

\begin{theo}[Artin-Wedderburn]{}
Let $R$ be a left Artinian ring. If $R$ has no nonzero nil ideals, then there exists $s \geq 1$, 
division rings $D_1, \dots, D_s$, and integers $n_1, \dots, n_s \geq 1$ such that 
\[ R \cong \prod_{i=1}^s M_{n_i}(D_i). \]
\end{theo}
\begin{pf}
Let $R$ be a left Artinian ring with no nonzero nil ideals. We showed in Lemma 8.1 that $R$ 
has finitely many prime ideals; call them $P_1, \dots, P_s$. Moreover, observe that $\bigcap_{j=1}^s P_j$ is a nil ideal by Theorem 7.10, and since $R$ has no nonzero nil ideals, it must be that 
$\bigcap_{j=1}^s P_j = (0)$. We also showed that $P_i + \bigcap_{j\neq i} P_j = R$ for all 
$i = 1, \dots s$ in Lemma 8.2. It follows from Sun-tzu that 
\[ R \cong \prod_{i=1}^s R/P_i. \] 
We know that $R/P_i$ is a prime left Artinian ring by Remark 8.1, so Theorem 6.5 implies that 
$R/P_i \cong M_{n_i}(D_i)$ for some $n_i \geq 1$ and division ring $D_i$. Thus, we conclude that 
\[ R \cong \prod_{i=1}^s R/P_i \cong \prod_{i=1}^s M_{n_i}(D_i). \qedhere \]
\end{pf}

\section{Corollary of Sun-tzu, Maschke's theorem (09/27/2021)}

Let's first look at a corollary of Sun-tzu (Theorem 8.3).

\begin{remark}{}
Let $k$ be a field. If $R$ is a $k$-algebra, then each canonical surjection $\pi_i : R \to R/I_i$ 
in the proof of Sun-tzu is a $k$-algebra homomorphism, which means that the map
$\Psi : R \to R/I_1 \times \cdots \times R/I_s$ is also a $k$-algebra homomorphism. 
\end{remark}

\begin{cor}{}
Let $k$ be an algebraically closed field, and let $R$ be a finite-dimensional $k$-algebra with no nonzero
nil ideals. Then we have 
\[ R \cong \prod_{i=1}^s M_{n_i}(k) \]
for some $s \geq 1$ and integers $n_1, \dots, n_s \geq 1$. 
\end{cor}
\begin{pf}
Since $R$ is a finite-dimensional $k$-algebra, we see that $R$ is left Artinian (Remark 5.7).
Hence, $R$ has only finitely many prime ideals (Lemma 8.2), say $P_1, \dots, P_s$. Moreover, 
since $R$ has no nonzero nil ideals, we have $\bigcap_{j=1}^s P_j = (0)$ (Theorem 7.10), and 
$P_i + \bigcap_{j\neq i} P_j = R$ for all $i = 1, \dots, s$ since $R$ is left Artinian (Lemma 8.2). 
By Sun-tzu, we have 
\[ R \cong R/P_1 \times \cdots \times R/P_s. \]
Then, since $k$ is algebraically closed and each $R/P_i$ is a prime finite-dimensional $k$-algebra, 
Corollary 7.4 implies that 
\[ R \cong R/P_1 \times \cdots \times R/P_s \cong M_{n_1}(k) \times \cdots \times M_{n_s}(k). 
\qedhere \]
\end{pf}

Now that we have the Artin-Wedderburn theorem in our toolkit, we can finally get started with 
some representation theory. Let $G$ be a finite group with $|G| = n$, and let $k$ 
be an algebraically closed field. Recall that $V := k[G]$ is the group algebra of the group $G$ over 
the field $k$, with $\dim_k k[G] = |G| = n$. Moreover, we have the embedding 
\[ \Phi : k[G] \to \End_k(V) \cong M_n(k), \]
where we send each $g \in G$ to the left multiplication map 
\begin{align*}
    L_g : V &\to V \\ x &\mapsto g \cdot x
\end{align*}
and extend linearly over $k$. We call this the {\bf left regular representation} of $G$. 

Next, let's recall some linear algebra. Let $W$ be a vector space with basis ${\cal B} 
= \{w_1, \dots, w_n\}$, and let $T : W \to W$ be a linear map. Then we can define a matrix 
$[T]_{\cal B}$ by setting $[T]_{\cal B} = (c_{ij})$ where 
\[ Tw_j = \sum_{i=1}^n c_{ij} w_i. \] 
Observe that the trace of $T$ is given by 
\[ \Tr(T) = \sum_{j=1}^n c_{jj}, \]
which is the sum of the coefficients of $w_j$ in $Tw_j$ (and this is independent of the basis 
${\cal B}$). 

Now, for the map $L_g : V \to V$ with $g \in G$, what is $\Tr(L_g)$? First, note that the elements of 
$G$ form a basis over $V = k[G]$; that is, we have ${\cal B} = \{g_1, \dots, g_n\} = G$ in this case. 
Observe that 
\begin{equation} L_g(g_i) = g \cdot g_i = 1 \cdot g \cdot g_i + \sum_{h \in G \setminus \{gg_i\}} 0 \cdot h. \end{equation}
For $j = 1, \dots, n$, the coefficient of $g_j$ in $gg_j$ is $1$ if $g = 1$, and $0$ if 
$g \neq 1$. Indeed, if $g \neq 1$, then $g_j$ would be different than $gg_j$, so it would have 
coefficient $0$ in equation (9.1). This implies that 
\[ \Tr(L_g) = \sum_{j=1}^n \begin{cases} 1 & \text{if } g = 1 \\ 0 & \text{if } g \neq 1 \end{cases} 
= \begin{cases} n & \text{if } g = 1 \\ 0 & \text{if } g \neq 1. \end{cases} \]

\begin{theo}[Maschke]{}
Let $k$ be a field (not necessarily algebraically closed), and let $G$ be a finite group. If 
$\ch(k) \nmid |G|$, then $k[G]$ has no nonzero nil ideals. 
\end{theo}
\begin{pf}
Suppose towards a contradiction that $k[G]$ has a nonzero nil two-sided ideal $N$. Pick a nonzero 
element 
\[ x = \sum_{g \in G} \alpha_g g \in N. \]
Since $x$ is nonzero, there exists some $g_0 \in G$ such that $\alpha_{g_0} \neq 0$. As 
$N$ is an ideal of $k[G]$, we also have $x \cdot g_0^{-1} \in N$. The coefficient of $1$ in 
$x \cdot g_0^{-1}$ is $\alpha_{g_0} \neq 0$. Thus, we can assume without loss of generality that 
$x = \sum_{g \in G} \alpha_g g \in N$ has $\alpha_1 \neq 0$. 

Let $A \in M_n(k)$ be nilpotent so that $A^n = 0$. Let $v$ be an eigenvector of $A$ with eigenvalue $\lambda$. Then we have 
\[ Av = \lambda v \implies A^2v = \lambda^2v \implies \cdots \implies A^n v = \lambda^n v, \]
which implies that $\lambda = 0$ since $A^n = 0$. In particular, $\Tr(A)$ is the sum of the 
eigenvalues of $A$ (with multiplicity), and the above argument shows that all the eigenvalues of 
$A$ are $0$, so $\Tr(A) = 0$.

Now, consider the injective $k$-algebra homomorphism 
\begin{align*}
    \Phi : k[G] \to \End_k(V) \cong M_n(k) : \sum_{g \in G} \beta_g \cdot g \mapsto \sum_{g \in G} \beta_g \cdot L_g
\end{align*}
we discussed at the beginning of this lecture (the left regular representation of $G$). We had an 
element 
\[ x = \sum_{g \in G} \alpha_g g \in N \]
with $\alpha_1 \neq 0$. Note that $x$ is nilpotent since $N$ is a nil ideal; say
$x^j = 0$ for some $j \geq 1$. Then we have 
\[ \Phi(x)^j = \Phi(x^j) = \Phi(0) = 0, \]
so $\Phi(x)$ is also nilpotent. Our discussion above implies that $\Tr(\Phi(x)) = 0$. However, 
we also see that 
\[ \Tr(\Phi(x)) = \Tr\left( \sum_{g \in G} \alpha_g \cdot L_g \right)
= \sum_{g \in G} \alpha_g \cdot \Tr(L_g) = |G| \cdot \alpha_1 \neq 0 \]
since $\alpha_1 \neq 0$ and $\ch(k) \nmid |G|$. This is a contradiction, so $k[G]$ has 
no nonzero nil ideals. 
\end{pf}

\begin{remark}{}
Question 1 of Assignment 1 shows that the converse of Maschke's theorem also holds. Indeed, we 
proved that if $\ch(k) \mid |G|$, then $k[G]u$ is a nonzero nil ideal where $u = \sum_{g \in G} g$. 
\end{remark}

\begin{cor}{}
Let $k$ be an algebraically closed field, and let $G$ be a finite group. If $\ch(k) \nmid |G|$, then 
there exists $s \geq 1$ and integers $n_1, \dots, n_s \geq 1$ such that 
\[ k[G] \cong M_{n_1}(k) \times \cdots \times M_{n_s}(k). \]
\end{cor}
\begin{pf}
By Maschke's theorem, we know that $k[G]$ has no nonzero nil ideals, and it is left Artinian 
as it is a finite-dimensional $k$-algebra. Since $k$ is algebraically closed, it follows that 
\[ k[G] \cong M_{n_1}(k) \times \cdots \times M_{n_s}(k) \]
for some $s \geq 1$ and integers $n_1, \dots, n_s \geq 1$ by Corollary 9.2. 
\end{pf}

This corollary is important because this gives us a nice relationship between $k[G]$, which is 
something intrinsic to the group $G$, and the direct product of matrix rings, which is more in the 
realm of linear algebra. 
\section{September 29, 2021}
By Corollary 9.5, we know that if $k$ is algebraically closed, $G$ 
is a finite group, and $\ch(k) \nmid |G|$, then 
\[ k[G] \cong M_{n_1}(k) \times \cdots \times M_{n_s}(k) \] 
for some $s \geq 1$ and integers $n_1, \dots, n_s \geq 1$. 
This will be our setting for this lecture, and
we'll derive more properties about the choice of $s \geq 1$ and the integers 
$n_1, \dots, n_s \geq 1$. 

\begin{theo}{}
    We have $|G| = n_1^2 + \cdots + n_s^2$. 
\end{theo}
\begin{pf}
    The isomorphism in Corollary 9.5 is a $k$-algebra isomorphism, so we 
    find that 
    \begin{align*}
        |G| &= \dim_k k[G] \\
        &= \dim_k M_{n_1}(k) \times \cdots \times M_{n_s}(k) \\
        &= \dim_k M_{n_1}(k) + \cdots + \dim_k M_{n_s}(k) \\
        &= n_1^2 + \cdots + n_s^2. \qedhere 
    \end{align*}
\end{pf}

Next, we'll work towards showing that $s$ is the number of conjugacy classes of $G$.

\begin{remark}{}
    If $R$ is a $k$-algebra, then $Z(R)$ is also a $k$-algebra. Indeed, we have
    an embedding $k \hookrightarrow Z(R) \subseteq R$ which sends $1_k$ 
    to $1_R$, and this also gives us an embedding $k \hookrightarrow 
    Z(Z(R)) = Z(R)$. 
\end{remark}

\begin{remark}{}
    If $T_1, \dots, T_s$ are rings, then 
    \[ Z(T_1 \times \cdots \times T_s) = Z(T_1) \times \cdots \times Z(T_s). \]
\end{remark}

\begin{prop}{}
    We have $Z(M_n(k)) = kI_n = \{\lambda I_n : \lambda \in k\}$. 
\end{prop}
\begin{pf}
    We'll give two proofs: an elementary one, and a high level one. 

    For the elementary proof, suppose that $(a_{ij}) \in Z(M_n(k))$. 
    Then observe that 
    \begin{align*} 
         \begin{pmatrix} 
            a_{11} & 0 & \cdots & 0 \\ 
            a_{21} & 0 & \cdots & 0 \\ 
            \vdots & \vdots & \ddots & \vdots \\ 
            a_{n1} & 0 & \cdots & 0 
        \end{pmatrix} &= \begin{pmatrix}
            a_{11} & \cdots & a_{1n} \\ 
            \vdots & \ddots & \vdots \\
            a_{n1} & \cdots & a_{nn} 
        \end{pmatrix} \begin{pmatrix} 
            1 & 0 & \cdots & 0 \\ 
            0 & 0 & \cdots & 0 \\
            \vdots & \vdots & \ddots & \vdots \\
            0 & 0 & \cdots & 0
        \end{pmatrix} \\ &= \begin{pmatrix} 
            1 & 0 & \cdots & 0 \\ 
            0 & 0 & \cdots & 0 \\
            \vdots & \vdots & \ddots & \vdots \\
            0 & 0 & \cdots & 0
        \end{pmatrix} \begin{pmatrix}
            a_{11} & \cdots & a_{1n} \\ 
            \vdots & \ddots & \vdots \\
            a_{n1} & \cdots & a_{nn} 
        \end{pmatrix} \\ &= \begin{pmatrix}
            a_{11} & a_{12} & \cdots & a_{1n} \\
            0 & 0 & \cdots & 0 \\ 
            \vdots & \vdots & \ddots & \vdots \\
            0 & 0 & \cdots & 0
        \end{pmatrix}. 
    \end{align*}
    In particular, we have $a_{21} = \cdots = a_{n1} = 0$ and 
    $a_{12} = \cdots = a_{1n} = 0$, which shows that 
    \[ A = \left(\begin{array}{@{}c|ccc@{}}
        a_{11} & 0 & \cdots & 0 \\\hline
        0 \\
        \vdots & & A' & \\
        0 
      \end{array}\right) \]
    for some smaller matrix $A'$. The argument follows inductively. 

    For a high level proof (where $k$ is algebraically closed), take 
    $R = M_n(k)$ and consider the simple left $R$-module
    \[ M = k^{n\times 1} = \left\{ \begin{pmatrix} \lambda_1 \\ \vdots \\ 
        \lambda_n \end{pmatrix} : \lambda_1, \dots, \lambda_n \in k \right\}. \] 
    We have shown that $\Delta = \End_R(M) \cong k$
    by identifying each $\lambda \in k$ with the map 
    \begin{align*}
        \Phi_\lambda : M &\to M \\ v &\mapsto \lambda \cdot v. 
    \end{align*} 
    Now, if $A \in Z(M_n(k))$, then the map 
    \begin{align*} f : M &\to M \\ v &\mapsto Av \end{align*}
    is $R$-linear; indeed, for $B \in M_n(k) = R$ and $v_1, v_2 \in M$, we have 
    \begin{align*} 
        f(Bv_1 + v_2) &= A(Bv_1 + v_2) \\ 
        &= ABv_1 + Av_2 \\
        &= BAv_1 + Av_2 \\ 
        &= Bf(v_1) + f(v_2).
    \end{align*}
    In particular, we see that $f \in \Delta$, so $f = \Phi_\gamma$ for some 
    $\gamma \in k$. Then $f(v) = Av = \gamma v$ for all $v \in M$, which 
    implies that $A = \gamma I$. 
\end{pf}

Combining Proposition 10.4 with Remark 10.3, we see that 
\begin{align*}
    Z(M_{n_1}(k) \times M_{n_s}(k)) 
    &= Z(M_{n_1}(k)) \times \cdots \times Z(M_{n_s}(k)) \\
    &= \{(\lambda_1 I_{n_1}, \dots, \lambda_s I_{n_s}) : 
    (\lambda_1, \dots, \lambda_s) \in k^s\}. 
\end{align*}
In particular, we have 
$\dim_k Z(k[G]) = \dim_k Z(M_{n_1}(k) \times \cdots \times M_{n_s}(k)) = s$.

\begin{defn}{}
    We say that a function $\alpha : G \to k$ is a {\bf class function} if 
    $\alpha$ is constant when restricted to each conjugacy class of $G$. 
\end{defn} 

\begin{lemma}{}
    Let $\alpha : G \to k$ be a function. Then $z := \sum_{g \in G} 
    \alpha(g) g$ is central in $k[G]$ if and only if $\alpha$ is a 
    class function. 
\end{lemma}
\begin{pf}
    Note that $z \in \sum_{g \in G} \alpha(g)g$ is in $Z(k[G])$ if and only if 
    $xz = zx$ for all $x \in G$; the backwards direction here is because $G$ 
    forms a basis for $k[G]$. This occurs if and only if 
    $z = x^{-1}zx$ for all $x \in G$ by rearranging. Now, this is 
    equivalent to 
    \[ \sum_{g \in G} \alpha(g)g = x^{-1} \left( \sum_{g \in G}
    \alpha(g)g \right) x = \sum_{g \in G} \alpha(g)x^{-1}gx
    = \sum_{h \in G} \alpha(xhx^{-1}) h \] 
    holding for all $x \in G$, where the last equality follows from 
    making the substitution $h = x^{-1}gx$. This is true if and only if 
    the coefficient of the left-hand side is the same as the 
    coefficient of the right-hand side. That is, $\alpha(h) = 
    \alpha(xhx^{-1})$ for all $h \in G$ and $x \in G$, and this is exactly 
    the definition of a class function. 
\end{pf}

Let $G$ be a finite group, and let ${\cal C}_1, 
\dots, {\cal C}_s$ be the conjugacy classes of $G$. For $i = 1, \dots, s$, 
observe that 
\[ \alpha(g) = \begin{cases} 1 & \text{if } g \in {\cal C}_i \\ 
    0 & \text{if } g \notin {\cal C}_i \end{cases} \]
is a class function. Then the elements 
\[ z_i = \sum_{g \in G} \alpha(g) g = \sum_{g \in {\cal C}_i} g \] 
for $i = 1, \dots, s$ are central by Lemma 10.6. 

\begin{prop}{}
    Let $G$ be a finite group with conjugacy classes ${\cal C}_1, 
    \dots, {\cal C}_s$. Then the elements 
    \[ z_i := \sum_{g \in {\cal C}_i} g \] 
    for $i = 1, \dots, s$ form a basis for $Z(k[G])$. 
\end{prop} 
\begin{pf}
    We have already seen that the $z_i$ are central. To show 
    linear independence, suppose that 
    \[ c_1 z_1 + \cdots + c_s z_s = 0, \] 
    where we take $0$ to mean $\sum_{g \in G} 0 \cdot g$ in $k[G]$. 
    If $g \in {\cal C}_i$, then the coefficient of $g$ on the left-hand 
    side is $c_i$. But the coefficient on the right-hand side is 
    always $0$, so $c_1 = \cdots = c_s = 0$. To see that 
    $\{z_1, \dots, z_s\}$ spans $Z(k[G])$, recall that 
    $z \in \sum_{g \in G} \alpha(g)g \in Z(k[G])$ if and only if 
    $\alpha$ is a class function. Let $\beta_i$ be the unique value 
    of $\alpha$ on ${\cal C}_i$. Then we see that 
    \[ z = \sum_{g \in G} \alpha(g)g = \sum_{i=1}^s \sum_{g \in {\cal C}_i} 
    \alpha(g)g = \sum_{i=1}^s \beta_i \sum_{g \in {\cal C}_i} g 
    = \sum_{i=1}^s \beta_i z_i, \] 
    so we can write $z$ as a linear combination of the $z_i$, completing the 
    proof. 
\end{pf}

\begin{theo}{}
    In the setting of Corollary 9.5, $s$ is the number of conjugacy 
    classes of $G$. 
\end{theo} 
\begin{pf}
    We previously showed that $\dim_k Z(k[G]) = \dim_k Z(M_{n_1}(k) \times 
    \cdots \times M_{n_s}(k)) = s$. Proposition 10.7 tells us that 
    $\dim_k Z(k[G])$ is the number of conjugacy classes of $G$. 
\end{pf} 

For the remainder of the lecture, we recall the abelianization of a group $G$.
We denote by $G'$ the commutator (or derived) subgroup of $G$, which is the 
smallest subgroup of $G$ which contains all elements of the form 
$ghg^{-1}h^{-1}$ with $g, h \in G$ (we call these elements commutators). 

Note that $G/G'$ is abelian. Indeed, take $g, h \in G$ so that 
$gG', hG' \in G/G'$. Observe that 
\[ (h^{-1})(g^{-1})(h^{-1})^{-1}(g^{-1})^{-1} = 
    h^{-1}g^{-1}hg \in G', \] 
so we have 
\[ (gG')(hG') = ghG' = gh(h^{-1}g^{-1}hg)G' = hgG' = (hG')(gG'). \]
For this reason, we call $G/G'$ the {\bf abelianization} of $G$. 

\begin{exercise}{}
    Show that $G/G'$ has the universal property that if $A$ is abelian 
    group, $\phi : G \to A$ is a group homomorphism, and $\pi 
    : G \to G/G'$ is the canonical quotient map, then there is a 
    unique group homomorphism $\Phi : G/G' \to A$ such that 
    $\Phi \circ \pi = \phi$. 

    \begin{center}
        \begin{tikzcd}
            G \arrow[rr, "\phi"] \arrow[dd, "\pi"'] &  & A \\
                                                    &  &   \\
            G/G' \arrow[rruu, "\Phi"', dotted]      &  &  
        \end{tikzcd}
    \end{center}
\end{exercise}

\section{October 1, 2021}
There is a similar notion of abelianization for rings. Let $R$ 
be a ring. The {\bf commutator ideal} $[R, R]$ is the smallest ideal 
containing all elements of the form $ab - ba$ where $a, b \in R$. 
We can also think of $[R, R]$ as the intersection of all two-sided ideals 
$J$ such that $ab - ba \in J$ for all $a, b \in R$. 

\begin{remark}{}
    \begin{enumerate}[(1)]
        \item Note that $R/[R, R]$ is commutative because for all $a, b \in R$, 
        we have 
        \begin{align*}
            (a + [R, R])(b + [R, R]) 
            &= ab + [R, R] \\ 
            &= ab + (ba - ab) + [R, R] \\
            &= ba + [R, R] \\
            &= (b + [R, R])(a + [R, R]).
        \end{align*}
        \item We also have a universal property here; if $R$ is a ring, 
        $C$ is a commutative ring, $\phi : R \to C$ is a ring homomorphism, 
        and $\pi : R \to R/[R, R]$ is the canonical quotient map, then 
        there exists a unique ring homomorphism $\Phi : R/[R, R] \to C$ 
        such that $\Phi \circ \pi = \phi$. 
        \begin{center}
            \begin{tikzcd}
                R \arrow[rr, "\phi"] \arrow[dd, "\pi"']  &  & C \\
                                                        &  &   \\
                {R/[R, R]} \arrow[rruu, "\Phi"', dotted] &  &  
            \end{tikzcd}
        \end{center}
    \end{enumerate}
\end{remark}

In Assignment 3, we will prove the following results. 
\begin{enumerate}[(1)]
    \item If $k$ is a field and $G$ is a group, then there is a 
    $k$-algebra homomorphism 
    \[ k[G]/[k[G], k[G]] \cong k[G/G']. \] 
    Said another way, the abelianization of the group algebra is isomorphic 
    to the group algebra of the abelianization. 
    \item If $R = M_{n_1}(k) \times \cdots \times M_{n_s}(k)$, then 
    \[ R/[R, R] \cong k^t, \] 
    where $t = \#\{i : n_i = 1\}$. 
\end{enumerate}

As an immediate consequence of these results, we have the following 
corollary. 

\begin{cor}{}
    If $k[G] \cong M_{n_1}(k) \times \cdots \times M_{n_s}(k)$, then 
    $|G/G'| = \#\{i : n_i = 1\}$. 
\end{cor}
\begin{pf}
    Since $k[G]$ and $M_{n_1}(k) \times \cdots \times M_{n_s}(k)$ 
    are isomorphic, so are their abelianizations. It follows that there 
    is a $k$-algebra homomorphism 
    \[ k[G/G'] \cong k^t \] 
    where $t = \#\{i : n_i = 1\}$. By looking at the dimensions, we see that 
    \[ |G/G'| = \dim_k k[G/G'] = \dim_k k^t = t = \#\{i : n_i = 1\}. 
    \qedhere \] 
\end{pf}

\begin{exmp}{}
    Let's look at $\C[S_4]$. Recall that the conjugacy classes of $S_n$ 
    are just sets of permutations with the same disjoint cycle structure. 
    For $n = 4$, the conjugacy classes are 
    \[ [(1)(2)(3)(4)], [(12)(3)(4)], [(12)(34)], [(123)(4)], [(1234)]. \] 
    Notice that we can think of these as
    \[ 1+1+1+1=4,\; 2+1+1=4,\; 2+2=4,\; 3+1=4,\; 4=4. \] 
    In particular, there is a bijective correspondence between the conjugacy 
    classes of $S_n$ and the partitions of $n$. 

    We can look at the sizes of the conjugacy classes. We have 
    \begin{align*} 
        \#[(1)(2)(3)(4)] &= 1, \\ 
        \#[(1)(2)(34)] &= 6, \\
        \#[(1)(234)] &= 8, \\
        \#[(1234)] &= 6, \\
        \#[(12)(34)] &= 3. 
    \end{align*}
    Since there are $5$ conjugacy classes of $S_4$, we have 
    \[ \C[S_4] \cong M_{n_1}(\C) \times \cdots \times M_{n_5}(\C) \] 
    for some integers $n_1, \dots, n_5 \geq 1$. 

    We claim that $S'_4 = A_4$. First, note that $S'_4 \subseteq A_4$ 
    since $S'_4/A_4$ is abelian. Now, notice that 
    \[ (123) = (12)(13)(12)(13) = (12)(13)(12)^{-1}(13)^{-1} \in S'_4. \] 
    Therefore, every $3$-cycle is in $S'_4$ as normal subgroups 
    are unions of conjugacy classes. Hence, we see that 
    \[ |S'_4| \geq 9 = \#[(1)(234)] + \#[(1)(2)(3)(4)]. \] 
    Since $|S'_4| \mid |S_4| = 24$ by Lagrange's theorem, we obtain 
    $|S'_4| = 12$, and so $S'_4 = A_4$. From this, we have  
    \[ \#\{i : n_i = 1\} = |S_4/S'_4| = |S_4/A_4| = 2. \] 
    Now, we have $n_3, n_4, n_5 \geq 2$ with $n_3^2 + n_4^2 + n_5^2 = 22$. 
    The only solution to this is $2^2 + 3^2 + 3^2 = 22$, so we conclude that 
    \[ \C[S_4] \cong \C \times \C \times M_2(\C) \times M_3(\C) \times M_3(\C). \] 
\end{exmp}

\begin{exmp}{}
    If $A$ is abelian with $|A| = n$, then $|A/A'| = n$ so that 
    $\#\{i : n_i = 1\} = n$. It follows that 
    \[ \C[A] \cong \C^n. \] 
    We can give an alternate argument: if we had $n_i \geq 2$ for some 
    $i = 1, \dots, s$, then the product of matrix rings is no longer 
    commutative, which is a contradiction since $\C[A]$ is commutative. 
\end{exmp}

\begin{exmp}{}
    Let $D_5$ be the dihedral group of order $10$. We can write 
    \[ D_5 = \langle x, y : x^2 = 1,\, y^5 = 1,\, xyx = y^{-1} \rangle. \] 
    We have an isomorphism 
    \[ \C(D_5) \cong M_{n_1}(\C) \times \cdots \times M_{n_s}(\C) \] 
    for some $s \geq 1$ and integers $n_1, \dots, n_s \geq 1$. 
    Notice that 
    \[ 10 = 1^2 + 3^2 = 1^2 + 1^2 + 2^2 + 2^2 = \underbrace{1^2 + \cdots + 1^2}_{6} \,+\, 2^2 
    = \underbrace{1^2 + \cdots + 1^2}_{10}. \] 
    One can check that $D_5$ has $4$ conjugacy classes. Therefore, only the 
    choice $10 = 1^2 + 1^2 + 2^2 + 2^2$ works, and we must have 
    \[ \C(D_5) \cong \C \times \C \times M_2(\C) \times M_2(\C). \] 
\end{exmp}

Let's now study representations. Let $G$ be a group and 
let $k$ be an algebraically closed field such that $\ch(k) \nmid |G|$. 
A {\bf representation} of $G$ is a group homomorphism 
\[ \rho : G \to \GL_n(k). \]

\begin{remark}{}
    The group homomorphism $\rho : G \to \GL_n(k)$ extends to a $k$-algebra 
    homomorphism 
    \[ \tilde\rho : k[G] \to M_n(k). \] 
    Conversely, if $\phi : k[G] \to M_n(k)$ is a $k$-algebra homomorphism, then
    $\phi$ restricted to $G$ is a map  
    \[ \phi|_G : G \to \GL_n(k). \] 
    This is because $\phi(1) = I$, so for any $g \in G$, we have 
    \[ \phi(g) \cdot \phi(g)^{-1} = \phi(g \cdot g^{-1}) = \phi(1) = I. \] 
    Thus, $\phi(g)$ must be an invertible matrix. Moreover, $\tilde\rho|_G 
    = \rho$. 
\end{remark}

\begin{remark}{}
    If $\phi : k[G] \to M_n(k)$ is a $k$-algebra homomorphism, then 
    \[ V = k^{n\times 1} = \left\{ \begin{pmatrix} a_1 \\ \vdots \\ a_n
    \end{pmatrix} : a_1, \dots, a_n \in k \right\} \] 
    can be given a left $k[G]$-module structure with the rule 
    \[ r \cdot v := \Phi(r) \cdot v \] 
    for $r \in k[G]$ and $v \in V$. 
\end{remark} 

Notice that if $V$ is a $k[G]$-module, then we have a $k$-algebra homomorphism
\[ \Psi : k[G] \to \End_k(V) \] 
where we send each $g \in G$ to the map 
\begin{align*} L_g : V &\to V \\ v &\mapsto g\cdot v \end{align*}
and extend linearly over $k$. In particular, 
\[ \Psi|_G : G \to \GL(V) \cong \GL_n(k), \] 
where $\GL(V)$ denotes the invertible linear transformations over $M$. 
For this reason, if $V$ is a $k[G]$-module, we will call $V$ a 
{\bf representation} of $G$. 

\section{October 4, 2021}
We'll first make some definitions. 

\begin{defn}{} 
    Let $V$ and $W$ be $k[G]$-modules; that is, they are 
    representations of $G$. 
    \begin{enumerate}[(1)]
        \item If $V$ is simple, we say that $V$ is a {\bf irreducible representation}.
        \item If $W \subseteq V$ is a $k[G]$-submodule of $V$, we say that $W$ is a 
        {\bf subrepresentation} of $G$. 
        \item We call two representations $V$ and $W$ {\bf equivalent} if 
        $V \cong W$ as left $k[G]$-modules. 
        \item We say that $V$ is {\bf decomposable} if $V = V_1 \oplus V_2$, 
        where $V_1$ and $V_2$ are proper submodules of $V$. 
    \end{enumerate}
\end{defn}

Let's define what a direct sum is. First, let $M$ be an $R$-module. 
If $M_1$ and $M_2$ are submodules of $M$, then 
\[ M_1 + M_2 = \{m_1 + m_2 : m_1 \in M_1,\, m_2 \in M_2\} \subseteq M. \] 
More generally, if we have a family $(M_\alpha)_\alpha \in X$ of submodules of 
$M$, then 
\[ \sum_{\alpha \in X} M_\alpha = \left\{ \sum_{\alpha \in X} m_\alpha : 
m_\alpha \in M_\alpha, \text{ $m_\alpha = 0$ for all but finitely many 
$\alpha$}\right\} \subseteq M. \] 
Then, we call $\sum_{\alpha \in X} M_\alpha \subseteq M$ 
{\bf direct} when $\sum_{\alpha \in X} m_\alpha = 0$ if and only if 
$m_\alpha = 0$ for all $\alpha \in X$. In the case where 
$\sum_{\alpha \in X} M_\alpha$ is a direct sum, we write 
\[ \bigoplus_{\alpha \in X} M_\alpha := \sum_{\alpha \in X} M_\alpha. \] 

\begin{remark}{}
    Let $M$ be an $R$-modules, and let $M_1$ and $M_2$ be submodules of $M$. 
    Then $M_1 + M_2$ is a direct sum if and only if $M_1 \cap M_2 = (0)$. 
\end{remark} 

\begin{exmp}{}
    Let $V$ be a vector space over a field $k$, and let $v_1, v_2 \in V 
    \setminus \{0\}$. Consider the submodules $V_1 = \{\lambda v_1 : \lambda 
    \in k\}$ and $V_2 = \{\lambda v_2 : \lambda \in k\}$. Then $V_1 + V_2$ is a 
    direct sum if and only if $v_1$ and $v_2$ are linearly independent. 
\end{exmp}

We'll study some module theory for the group algebra $k[G]$. Let 
$k$ be an algebraically closed field such that $\ch(k) \nmid |G|$. 

In the next few lectures, we will work towards proving the following theorem. 

\begin{theo}{}
    Suppose that 
    \[ k[G] \cong M_{n_1}(k) \times \cdots \times M_{n_s}(k), \] 
    where $s$ is the number of conjugacy classes of $G$. 
    \begin{enumerate}[(1)]
        \item Up to isomorphism, $k[G]$ has $s$ pairwise non-isomorphic 
        simple left modules $V_1, \dots, V_s$ (that is, $s$ pairwise 
        inequivalent irreducible representations). 
        \item Every $k[G]$-module decomposes as a direct sum of copies of 
        $V_1, \dots, V_s$. 
        \item {\bf Uniqueness:} If $V_1^{a_1} \oplus V_2^{a_2} \oplus 
        \cdots \oplus V_s^{a_s} \cong V_1^{b_1} \oplus V_2^{b_2} \oplus 
        \cdots \oplus V_s^{b_s}$, then $a_i = b_i$ for all 
        $i = 1, \dots, s$, where by $V^a$, we mean $V^a := 
        V \oplus \cdots \oplus V$ ($a$ times).
    \end{enumerate}
\end{theo}

What is the significance of this theorem? First, let's ask what the $V_i$'s are. 
We know that 
\[ k[G] \cong M_{n_1}(k) \times \cdots \times M_{n_s}(k), \] 
and for each $i = 1, \dots, s$, we have the projection 
\[ \pi_i : M_{n_1}(k) \times \cdots \times M_{n_s}(k) \to M_{n_i}(k). \] 
Note that $M_{n_i}(k)$ has the module $V_i \cong k^{n_i \times 1}$. Now, if 
$V$ is a representation of $G$ which looks like 
\[ V = V_{i_1} \oplus \cdots \oplus V_{i_q}, \] 
we recall that $k[G]$ acting on $V$ gives us a representation 
\[ \rho : G \to \GL(V) \] 
given by $\rho(g)(v) := g \cdot v$. Now, what's the significance of the 
decomposition of $V$ above? Suppose for simplicity that we had 
$V = W_1 \oplus W_2$, where $W_1$ and $W_2$ are $k[G]$-submodules of $V$. 
Let $\{v_1, \dots, v_d\}$ be a basis for $W_1$, and let 
$\{v_{d+1}, \dots, v_n\}$ be a basis for $W_2$. Then we have 
\[ \rho(g)(v_j) = g \cdot v_j = \begin{cases} 
    a_{1j}v_1 + \cdots + a_{d_j} v_d, & \text{if $j \leq d$,} \\ 
    a_{(d+1)_j} v_{d+1} + \cdots + a_{n_j} v_n, & \text{if $d + 1 \leq j \leq n$.}
\end{cases} \] 
In particular, by setting ${\cal B} = \{v_1, \dots, v_n\}$, we see that 
\[ [\rho(g)]_{\cal B} = \left(\begin{array}{@{}c|c@{}}
    A_1 & 0 \\\hline
    0 & A_2 
  \end{array}\right), \]
where $A_1$ is a $d \times d$ matrix and $A_2$ is an $(n-d) \times (n-d)$ matrix.
Therefore, the importance of the theorem is that if we are given a finite-dimensional 
representation, we can uniquely write it as a block diagonal matrix 
(up to ordering of the blocks), where the blocks are representations coming 
from the irreducibles $V_i \cong k^{n \times 1}$. Therefore, if we understand 
irreducibles, then we understand every representation. 

First, in order to understand $k[G]$, we looked at Artinian rings with no 
nonzero nil ideals. Now that we understand $k[G]$, our main goal is to understand 
representations. Our discussion above shows that it suffices to understand 
the simples $V_i \cong k^{n \times 1}$. We will use character theory to do so, 
which we will get to soon, and we'll understand everything. 

Due to the isomorphism 
\[ k[G] \cong M_{n_1}(k) \times \cdots \times M_{n_s}(k), \] 
we can just think of $k[G]$ as a product of matrix rings. Therefore, it suffices 
to understand the left modules of $M_{n_1}(k) \times \cdots \times 
M_{n_s}(k)$ to know the left modules of $k[G]$. Let $R$ and $S$ be rings 
with an isomorphism $\phi : R \to S$. If we have an $R$-module $M$, 
we obtain an $S$-module by taking 
\[ s \cdot m := \phi^{-1}(s) \cdot m. \] 
Conversely, if $N$ is a left $S$-module, then we can give it an $R$-module 
structure by 
\[ r \cdot n = \phi(r) \cdot n. \] 
That is, the isomorphism $\phi$ is just a way of relabelling these elements. 

\begin{lemma}{}
    Let $R = M_{n_1}(k) \times \cdots \times M_{n_s}(k)$, and let 
    $V_i = k^{n_i \times 1}$ for each $i = 1, \dots, s$. For each 
    $i = 1, \dots, s$ we define 
    \[ (A_1, \dots, A_i, \dots, A_s) \cdot v := A_i v \] 
    where $(A_1, A_2, \dots, A_s) \in R$ and $v \in V_i$. 
    Then $V_i$ is a simple left $R$-module, and if $i \neq j$, then 
    $V_i \ncong V_j$. 
\end{lemma}
\begin{pf}
    If $(A_1, \dots, A_s), (B_1, \dots, B_s) \in R$ and $v \in V_i$, then 
    \begin{align*} 
        (A_1, \dots, A_i, \dots, A_s) \cdot (B_1, \dots, B_i, \dots, B_s) \cdot v 
        &= (A_1B_1, \dots, A_iB_i, \dots, A_sB_s) v \\ 
        &= A_iB_i v \\ 
        &= A_i(B_i v) \\ 
        &= (A_1, \dots, A_s) [(B_1, \dots, B_s) \cdot v] 
    \end{align*} 
    Therefore, $V_i$ is a left $R$-module; the other properties come for free 
    since $V_i$ is already a module over $M_{n_i}(k)$, and hence it is an abelian 
    group. 

    Why is $V_i$ simple as an $R$-module? Notice that it suffices to show that 
    if $v_1 \in V_i \setminus \{0\}$ and $v_2 \in V_i$, then there exists 
    $r \in R$ such that $r \cdot v_1 = v_2$. Indeed, if $V_i$ is simple, then 
    the only submodules are $(0)$ and $V_i$ itself. If we have a nonzero 
    element of $V_i$, then it should generate all of $V_i$. 

    We already know that $V_i$ is a simple $M_{n_i}(k)$-module. Therefore, 
    there exists $A \in M_{n_i}(k)$ such that $Av_1 = v_2$. Let 
    $r = (0, \dots, 0, A, 0, \dots, 0)$, where $A$ is located at the 
    $i$-th coordinate. Then $r \cdot v_1 = v_2$, so $V_i$ is a simple 
    left $R$-module. 

    Next, we show that if $i \neq j$, then $V_i \ncong V_j$ as left $R$-modules. 
    Suppose that we have an $R$-linear isomorphism $f : V_i \to V_j$. 
    If $v \in V_i$, then 
    \[ f((0, \dots, 0, I, 0, \dots, 0) \cdot v) = f(0 \cdot v) = 0 \in V_j, \] 
    where $I$ is in the $j$-th position, and the $i$-th position is one of the 
    zeros. Since $f$ is $R$-linear, then 
    \[ f((0, \dots, 0, I, 0, \dots, 0) \cdot v) = (0, \dots, 0, I, 0, \dots, 0)
    \cdot f(v) = I \cdot f(v) = f(v). \] 
    This means that $f$ is identically zero, contradicting the fact that 
    it is an isomorphism from $V_i$ to $V_j$. 
\end{pf}

\begin{remark}{}
    In fact, we have shown that if $i \neq j$, then 
    \[ \Hom_R(V_i, V_j) = \{0\}. \] 
    When $i = j$, we have $\Hom_R(V_i, V_j) = \End_R(V_i)$. In the case that 
    $k$ is algebraically closed and $R$ is a finite-dimensional $k$-algebra, 
    we have $\End_R(V_i) \cong k$ by Proposition 3.1, since $V_i$ is a simple 
    left $R$-module. 
\end{remark}
\section{October 6, 2021}
Last time, we were looking at a ring 
\[ R = M_{n_1}(k) \times \cdots \times M_{n_s}(k). \] 
We saw that $V_i = k^{n_i \times 1}$ is a left $R$-module with the action 
\[ (A_1, \dots, A_s) \cdot v := A_i v \in V_i \] 
for $(A_1, \dots, A_s) \in R$ and $v \in V_i$. Moreover, $V_i$ is a 
simple left $R$-module with $V_i \ncong V_j$ when $i \neq j$. 

Now, we observe that 
\begin{align*} 
    \Ann_R(V_i) &= \{(A_1, \dots, A_{i-1}, 0, A_{i+1}, \dots, A_s) : 
    A_j \in M_{n_j}(k)\} \\
    &= M_{n_1}(k) \times \cdots \times M_{n_{i-1}}(k) \times \{0\} 
    \times M_{n_{i+1}}(k) \times \cdots \times M_{n_s}(k). 
\end{align*} 
In particular, if $i \neq j$, then $\Ann_R(V_i) \ncong \Ann_R(V_j)$. 
Assignment 3 Question 2 then gives us another proof of the fact that 
$V_i \ncong V_j$. 

\begin{notation} 
    If $R$ is a ring, we will write ${}_R R$ to mean $R$ considered as a left 
    $R$-module. (Similarly, the notation $R_R$ means $R$ taken as a right 
    $R$-module.)
\end{notation}

Let $A, B \in M_n(k)$. We can think of $B$ as $n$ columns concatenated 
with each other. In particular, if $v_1, \dots, v_n$ are the columns of $B$, 
then the columns of $AB$ are given by $Av_1, \dots, Av_n$. 
Therefore, if $R = M_n(k)$ and $V = k^{n \times 1}$, then 
\[ {}_R R \cong \underbrace{V \oplus \cdots \oplus V}_{n} \] 
with the $R$-module isomorphism 
\[ B \mapsto (v_1, \dots, v_n), \] 
where as before, $v_1, \dots, v_n$ are the columns of $B$. 

More generally, if $R = M_{n_1}(k) \times \cdots \times M_{n_s}(k)$ and 
$V_i = k^{n_i \times 1}$, then we have 
\[ (A_1, \dots, A_s) \cdot v = A_i v \] 
for $(A_1, \dots, A_s) \in R$ and $v \in V_i$, so we see that 
\[ {}_R R \cong \underbrace{V_1 \oplus \cdots \oplus V_1}_{n_1} 
\oplus \underbrace{V_2 \oplus \cdots \oplus V_2}_{n_2} \oplus \cdots 
\oplus \underbrace{V_s \oplus \cdots \oplus V_s}_{n_s} \] 
with the explicit $R$-module isomorphism 
\[ (B_1, \dots, B_s) \in {}_R R \mapsto 
(\underbrace{v_{11}, \dots, v_{1n_1}}_{\in V_1^{n_1}}, 
\dots, \underbrace{v_{s1}, \dots, v_{sn_s}}_{\in V_s^{n_s}}), \] 
where $v_{i1}, \dots, v_{in_i}$ are the columns of $B_i$ for each 
$i = 1, \dots, s$. Indeed, we can see that this is an $R$-module isomorphism 
because for $(A_1, \dots, A_s) \in R$ and $(B_1, \dots, B_s) \in {}_R R$, the 
columns of $A_i B_i$ are $Av_{i1}, \dots, Av_{in_i}$, and on the other hand, 
we have 
\[ (A_1, \dots, A_s) \cdot (v_{11}, \dots, v_{1n_1}, \dots, 
v_{s1}, \dots, v_{sn_s}) = 
(\underbrace{A_1 v_{11}, \dots, A_1 v_{1n_1}}_{A_1B_1}, 
\dots, \underbrace{A_s v_{s1}, \dots, A_s v_{sn_s}}_{A_sB_s}). \] 
From this discussion, we obtain the key fact that if $R = M_{n_1}(k) \times \cdots \times 
M_{n_s}(k)$ and $V_i = k^{n_i \times 1}$, then 
\[ {}_R R \cong V_1^{n_1} \oplus \cdots \oplus V_s^{n_s} \] 
as $R$-modules. Now, we are almost in the position to finish proving Theorem 12.4.

\begin{remark} Let $N$ and $M_\alpha$ for $\alpha \in X$ be left $R$-modules. 
    \begin{enumerate}[(1)]
        \item From Exercise 4 on Assignment 2, we have 
        \[ \Hom_R \left( \bigoplus_{\alpha \in X} M_\alpha, N \right) 
        \cong \prod_{\alpha \in X} \Hom_R(M_\alpha, N). \] 
        \item If $N \neq (0)$ is a left $R$-module, then 
        \[ \Hom_R(R, N) \neq (0). \] 
        Notice that for all $n \in N$, there exists a map 
        \begin{align*} \varphi_n : {}_R R &\to N \\ r &\mapsto r \cdot n. \end{align*} 
        For $a \in R$ and $r \in {}_R R$, we have $\varphi_n(ar) = ar \cdot n 
        = a \cdot (rn) = a \cdot \varphi_n(r)$, so $\varphi_n$ is $R$-linear. 
        Moreover, when $n \neq 0$, we have $\varphi_n(1) = n \neq 0$, so 
        $\varphi_n \neq 0$. 
    \end{enumerate}
\end{remark}

\begin{exercise}
    Define the map 
    \begin{align*} 
        e : \Hom_R({}_R R, N) &\to N \\ \varphi &\mapsto \varphi(1). 
    \end{align*}
    If $R$ is commutative, show that $e$ is an isomorphism of $R$-modules. 
\end{exercise}

Now, we can prove part (1) of Theorem 12.4. 

\begin{thm}
    Let $R \cong M_{n_1}(k) \times \cdots \times M_{n_s}(k)$ and $V_i 
    \cong k^{n_i \times 1}$. If $N$ is a simple left $R$-mdoule, then there 
    exists some $i = 1, \dots, s$ such that $N \cong V_i$ as $R$-modules. 
    In particular, $R$ has exactly $s$ simple left $R$-modules up to 
    isomorphism. 
\end{thm}
\begin{pf}
    Let $N$ be a simple $R$-module. Then we have 
    \[ \Hom_R({}_R R, N) \neq (0) \] 
    since $N \neq (0)$. In particular, we obtain 
    \[ \Hom_R({}_R R, N) \cong \Hom_R(V_1^{n_1} \oplus \cdots \oplus V_s^{n_s}, 
    N) \cong \prod_{i=1}^s \Hom_R(V_i, N)^{n_i}. \] 
    Since $\Hom_R({}_R R, N) \neq (0)$, there exists $i = 1, \dots, s$ 
    such that $\Hom_R(V_i, N) \neq (0)$. Therefore, there exists a 
    nonzero $R$-modulo homomorphism $f : V_i \to N$. Now, $V_i$ is simple, 
    and since $\ker(f)$ is a submodule of $V_i$, we either have $\ker(f) 
    = (0)$ or $\ker(f) = V_i$. But $f$ is nonzero, so we have $\ker(f) = V_i$. 
    Similarly, $\im(f)$ is a submodule of $N$. Again, $f$ is nonzero, so 
    $\im(f) \neq (0)$, and since $N$ is simple, $\im(f) = N$. 

    As in the proof of Schur's lemma (Theorem 2.15), the set theoretic 
    inverse of $f$ is an $R$-module homomorphism with $f \circ f^{-1} 
    = \id_N$ and $f^{-1} \circ f = \id_{V_i}$. Therefore, $f : V_i \to N$ is 
    an $R$-module isomorphism. 
\end{pf}

Towards proving (2) of Theorem 12.4, we'll first prove the following fact. 

\begin{prop}
    Every $R$-module $M$ satisfies 
    \[ M \cong R^{\oplus X}/L \] 
    where $R^{\oplus X} = \bigoplus_{x \in X} R$, and $L$ is a submodule of 
    $R^{\oplus X}$. 
\end{prop}
\begin{pf}
    For each $m \in M$, we will define a formal symbol $e_m$. 
    (We can think of this as a vector $e_m = (0, \dots, 0, 1, 0, \dots, 0)$ 
    where $1$ is in the $m$-th position, but this choice doesn't always work.)
    Define an $R$-module homomorphism $\Psi : \bigoplus_{m \in M} Re_m \to M$ by 
    \[ \sum_{m \in M} r_m e_m \mapsto \sum_{m \in M} r_m \cdot m. \] 
    We note that these must be finite sums in order to make sense. When $r \in R$, 
    we have 
    \[ \Psi\left(r \cdot \sum_{m \in M} r_m e_m \right) 
        = \Psi\left( \sum_{m \in M} r \cdot r_m e_m \right) 
        = \sum_{m \in M} r \cdot r_m \cdot m 
        = r \cdot \sum_{m \in M} r_m \cdot m 
        = r \cdot \Psi \left( \sum_{m \in M} r_m e_m \right). \] 
    Therefore, $\Psi$ is $R$-linear. Moreover, $\Psi(1 \cdot e_m) = m$, so 
    $\Psi$ is onto. Let $L = \ker\Psi$. By the first isomorphism theorem, 
    we obtain $\bigoplus_{m \in M} Re_m/L \cong \im\Psi = M$, 
    and we can take  $\bigoplus_{m \in M} Re_m$ as $R^{\oplus X}$. 
\end{pf}

\section{October 8, 2021}

\begin{defn}
    Let $R$ be a ring. We say that a left $R$-module $M$ is {\bf semisimple} 
    of $M$ can be written as the direct sum of simple submodules. We say that 
    $R$ is a {\bf (left) semisimple ring} if every left $R$-module is semisimple.
\end{defn}

\begin{exmp}
    An example of a non-semisimple ring is $\Z$. The simple modules are 
    isomorphic to $\Z/p\Z$ for primes $p$, and we notice that 
    $\Z/4\Z$ cannot be written as the direct sum of simple submodules. 
\end{exmp}

\begin{exmp}
    Let $k$ be a field. A $k$-module is simply a vector space, and a 
    submodule is a subspace. In particular, a simple $k$-module $V$ is a vector 
    space whose only subspaces are $(0)$ and $V$ itself. Therefore, $V$ 
    is a simple $k$-module if and only if $\dim_k V = 1$. Now, if 
    $W$ is a $k$-vector space, then we can write 
    \[ W = \bigoplus_{\alpha \in {\cal B}} kv_\alpha, \] 
    since every vector space has a basis ${\cal B}$. Hence, $k$ is a semisimple ring. 
\end{exmp}

Notice that the fact that every vector space has a basis relies on Zorn's lemma. 
The following theorem generalizes this example, and at some point, we'll have to use 
Zorn's lemma to prove it. 

\begin{thm}
    Let $R$ be a ring. Then $R$ is a (left) semisimple ring if and only if 
    ${}_R R$ is a semisimple $R$-module. 
\end{thm}

To prove this, we'll use the following lemma. 

\begin{lemma}
    Let $R$ be a ring. 
    \begin{enumerate}[(1)]
        \item If $\{M_\alpha\}_{\alpha \in X}$ is a family of left $R$-modules 
        where each $M_\alpha$ is semisimple, then $\bigoplus_{\alpha \in X} 
        M_\alpha$ is semisimple. 
        \item If $M$ is a semisimple left $R$-module and $L \trianglelefteq M$ 
        is a submodule of $M$, then $M/L$ is also semisimple. 
    \end{enumerate}
\end{lemma}

First, we'll show that Theorem 14.4 easily follows from Lemma 14.5. 

{\sc Proof of Theorem 14.4.} $(\Rightarrow)$ If every left $R$-module is semisimple, 
then certainly ${}_R R$ is semisimple. 

$(\Leftarrow)$ Let $M$ be a left $R$-module. We saw in Proposition 13.5 that 
$M \cong R^{\oplus X}/L$ where $L \trianglelefteq \bigoplus_{x \in X} R$. 
If ${}_R R$ is semisimple, then 
\[ \bigoplus_{x \in X} {}_R R = \bigoplus_{x \in X} R \] 
is semisimple by taking $M_\alpha = {}_R R$ for each $\alpha \in X$ and 
applying (1) of Lemma 14.5. It follows from (2) that $M \cong 
\bigoplus_{x \in X} R/L$ is semisimple. \qed 

Now, it only remains to prove Lemma 14.5. 

{\sc Proof of Lemma 14.5.}~
\begin{enumerate}[(1)]
    \item Write each $M_\alpha$ as a direct sum of simple $R$-modules 
    \[ M_\alpha = \bigoplus_{\beta \in X_\alpha} V_{\alpha, \beta}. \] 
    It follows that 
    \[ \bigoplus_{\alpha \in X} M_\alpha 
    = \bigoplus_{\alpha \in X} \bigoplus_{\beta \in X_\alpha} V_{\alpha, \beta}
    = \bigoplus_{(\alpha, \beta)\,:\,\alpha\in X,\,\beta \in X_\alpha} 
    V_{\alpha, \beta} \] 
    is the direct sum of simple modules, so it is semisimple. 
    
    \item Since $M$ is semisimple, we can write 
    \[ M = \bigoplus_{\alpha \in X} V_\alpha, \] 
    where each $V_\alpha$ is simple. Let $\pi : M \to M/L$ be the canonical 
    projection. Consider $\pi(V_\alpha) \subseteq M/L$. We claim that 
    either $\pi(V_\alpha) \cong V_\alpha$ or $\pi(V_\alpha) = (0)$. 
    First, we restrict $\pi$ to $V_\alpha$ to obtain the map 
    \[ \pi|_{V_\alpha} : V_\alpha \to \pi(V_\alpha), \] 
    which is a surjective $R$-module homomorphism. Since $\ker(\pi|_{V_\alpha})$ 
    is a submodule of $V_\alpha$, we either have $\ker(\pi|_{V_\alpha}) = (0)$ 
    or $\ker(\pi|_{V_\alpha}) = V_\alpha$ by the simplicity of $V_\alpha$. 
    In the first case, $\pi|_{V_\alpha}$ is a bijection, so $\pi(V_\alpha) 
    \cong V_\alpha$. In the second case, we obtain $\pi(V_\alpha) = (0)$. 

    Let $Y = \{\alpha \in X : \pi(V_\alpha) \cong V_\alpha\} \subseteq X$. 
    We claim that 
    \[ M/L = \sum_{\alpha \in Y} \pi(V_\alpha). \] 
    Since $\pi : M \to M/L$ is onto and $M = \sum_{\alpha \in X} V_\alpha$, we have 
    \[ M/L = \pi(M) = \pi\left( \sum_{\alpha \in X} V_\alpha \right) 
    = \sum_{\alpha \in X} \pi(V_\alpha) = \sum_{\alpha \in Y} \pi(V_\alpha), \] 
    where the last equality is because $\pi(V_\alpha) = (0)$ for all 
    $\alpha \in X \setminus Y$, which contributes nothing to the sum. 

    Note that the above sum is not necessarily a direct sum. This is where 
    we'll use Zorn's lemma to show that it is isomorphic to a direct sum of 
    simples. Let 
    \[ {\cal S} = \left\{Z \subseteq Y : \sum_{\alpha \in Z} \pi(V_\alpha) 
    \text{ is a direct sum} \right\}. \] 
    First, we note that ${\cal S} \neq \varnothing$ since empty sums are 
    by definition direct sums, so $\varnothing \in {\cal S}$. Let 
    $\{Z_\gamma\}_{\gamma \in \Gamma}$ be a chain in ${\cal S}$ ordered by 
    inclusion, where $\Gamma$ is a totally ordered set. We must show that 
    \[ Z := \bigcup_{\gamma \in \Gamma} Z_\gamma \in {\cal S}, \] 
    which will be an upper bound in the chain, in which case we can apply 
    Zorn's lemma to ${\cal S}$. So suppose towards a contradiction that 
    $Z \notin {\cal S}$. Then the sum $\sum_{\alpha \in Z} \pi(V_\alpha)$ 
    is not a direct sum. Therefore, there is a finite set of elements 
    $\alpha_1, \dots, \alpha_s \in Z$ such that 
    \[ \pi(V_{\alpha_1}) + \cdots + \pi(V_{\alpha_s}) \] 
    is not direct. There exists some $\gamma \in \Gamma$ such that 
    $\alpha_1, \dots, \alpha_s \in Z_\gamma$, which would imply that 
    $\sum_{\alpha \in Z_\gamma} \pi(V_\alpha)$ is not a direct sum, a contradiction. 
    Therefore, $Z \in {\cal S}$, and by Zorn's lemma, there exists a maximal 
    element $Y_0 \in {\cal S}$ with $Y_0 \subseteq Y$. 

    Finally, we'll show that $N = \sum_{\alpha \in Y_0} \pi(V_\alpha)$ is 
    direct and equal to $M/L$. The fact that it is direct is immediate since 
    $Y_0 \in {\cal S}$. Recall that 
    \[ M/L = \sum_{\alpha \in Y} \pi(V_\alpha). \] 
    If $\pi(V_\alpha) \subseteq N$ for all $\alpha \in Y$, we have 
    $\sum_{\alpha \in Y} \pi(V_\alpha) \subseteq N$. This gives $M/L \subseteq N$, 
    and hence $N = M/L$. Therefore, it is enough to show that $\pi(V_\alpha) 
    \subseteq N$ for any $\alpha \in Y$. Suppose that this was not the case, 
    so $\pi(V_{\alpha_0}) \not\subseteq N$ for some $\alpha_0 \in Y$. 
    We will show that 
    \[ \pi(V_{\alpha_0}) + \bigoplus_{\alpha \in Y_0} \pi(V_\alpha) \] 
    is a direct sum, which will give a contradiction as $\{\alpha_0\} 
    \cup Y_0 \in {\cal S}$ contradicts the maximality of $Y_0$ in ${\cal S}$. 
    Note that $\pi(V_{\alpha_0})$ is simple. If $\pi(V_{\alpha_0}) = (0)$, then 
    $\pi(V_{\alpha_0}) + \bigoplus_{\alpha \in Y_0} \pi(V_\alpha)$ is a direct 
    sum. Otherwise, $\pi(V_{\alpha_0})$ is already in 
    $\bigoplus_{\alpha \in Y_0} \pi(V_\alpha)$, which means that 
    $\pi(V_\alpha) \subseteq N$, a contradiction. \qed 
\end{enumerate}
\section{Character tables, an ``inner product'' of maps (10/20/2021)}
Let $G$ be a finite group, and let $k$ be an algebraically closed field with 
$\ch(k) \nmid |G|$. Then 
\[ k[G] \cong \prod_{i=1}^s M_{n_i}(k) \] 
as $k$-algebras, where $s$ is the number of conjugacy classes of $G$, and 
$\sum_{i=1}^s n_i^2 = |G|$. Moreover, $k[G]$ has $s$ pairwise non-isomorphic 
simple modules (or irreducible representations) $V_1, \dots, V_s$, with 
$V_i \cong k^{n_i\times 1}$. 

Moreover, every left $k[G]$-module $V$ decomposes as 
\[ V \cong V_1^{a_1} \oplus \cdots \oplus V_s^{a_s}, \] 
and this decomposition is unique. We also have a representation 
\[ \rho_V : G \to \GL(V), \] 
which gives rise to the character $\chi_V : G \to k$ of $V$, with the formula 
\[ \chi_V = \sum_{i=1}^s a_i \chi_{V_i}. \] 
Notice that $\chi_V(1) = \Tr(\rho_V(1)) = \Tr(I_n) = n = \dim_k(V)$, so 
by taking the module $L = {}_{k[G]} k[G]$, we have 
\[ \chi_L = \sum_{i=1}^s n_i \chi_{V_i}, \] 
and for $g \in G$, we saw in the previous lecture that  
\[ \chi_L(g) = \begin{cases} 
    |G| & \text{if $g = 1$,} \\ 
    0 & \text{if $g \neq 1$.}
\end{cases} \] 
What are the irreducible representations of a finite abelian group $G$? 
Assume that $|G| = n$, and $k$ is an algebraically closed field with $\ch(k) 
\nmid n$. We see that $k[G] \cong k^n$, which implies that there are $n$ 
inequivalent irreducible representations. Now, since $G$ is abelian, we can write 
\[ G \cong C_{d_1} \times C_{d_2} \times \cdots \times C_{d_r}, \] 
where $C_j$ denotes the cyclic group of order $j$. We can think of this as 
\[ G \cong \langle x_1 \mid x_1^{d_1} = 1 \rangle \times \cdots \times 
\langle x_r \mid x_r^{d_r} = 1 \rangle. \] 
We know all simple $k[G]$-modules are one-dimensional as $k$-vector spaces. 
In particular, if $V$ is simple, then 
\[ \rho_V : G \to \GL_1(k) \cong k^* = k \setminus \{0\}. \] 
This means that if $\rho_V(x_i) = \omega_i \in k^*$, then 
\[ 1 = \rho_V(1) = \rho_V(x_i^{d_i}) = \omega_i^{d_i}. \] 
That is, each $\omega_i$ is a $d_i$-th root of unity, so there are 
$d_1\cdots d_r = |G|$ choices for $\rho_V$. On the other hand, we already know 
that there are $d_1 \cdots d_r$ pairwise inequivalent irreducible representations,
so these are in fact all of them. In this case, notice that $\chi_V = \rho_V$. 

\begin{exmp}{}
    Suppose that $G_1 = C_2 \times C_2 = \langle x \mid x^2 = 1 \rangle 
    \times \langle y \mid y^2 = 1 \rangle$ and $G_2 = C_4 = \langle z \mid 
    z^4 = 1 \rangle$. We compute the character tables of $G_1$ and $G_2$; 
    namely, the characters of the representations. 
    \begin{align*}
        \begin{array}{|c|c|c|c|c|}
            \hline
            G_1    & 1 & x  & y  & xy \\ \hline
            \chi_1 & 1 & 1  & 1  & 1  \\ \hline
            \chi_2 & 1 & 1  & -1 & -1 \\ \hline
            \chi_3 & 1 & -1 & 1  & -1 \\ \hline
            \chi_4 & 1 & -1 & -1 & 1  \\ \hline
        \end{array} & & 
        \begin{array}{|c|c|c|c|c|}
            \hline
            G_2    & 1 & z  & z^2  & z^3 \\ \hline
            \chi_1 & 1 & 1  & 1    & 1  \\ \hline
            \chi_2 & 1 & i  & -1   & -i \\ \hline
            \chi_3 & 1 & -1 & 1    & -1 \\ \hline
            \chi_4 & 1 & -i & -1   & -i  \\ \hline
        \end{array} 
    \end{align*}
\end{exmp}

\begin{exmp}{}
    Let $G = S_3$ and $k = \C$. We recall that $\C[S_3] \cong \C \times 
    \C \times M_2(\C)$. Then $S_3$ has three pairwise inequivalent irreducible 
    representations, and these have dimensions $1$, $1$, and $2$ 
    respectively. Moreover, recall that for a finite-dimensional 
    $k[G]$-module $V$, its character $\chi_V : G \to k$ is a class function. 
    Therefore, to compute the character table of $S_3$, it suffices to 
    consider the values of the characters on each conjugacy class. 

    First, the identity element is always sent to the dimension of the 
    representation, so we obtain $\chi_1(\id) = \chi_2(\id) = 1$ and 
    $\chi_3(\id) = 2$. Since $(12)^2 = \id$, we have $\chi_i((12))^2 
    = \chi_i(\id) = 1$ for $i \in \{1, 2\}$, yielding 
    $\chi_1((12)) = 1$ and $\chi_2((12)) = -1$. 

    Next, recall from Assignment 3 Question 1 that the one-dimensional 
    irreducible representations come from $G/G'$. More precisely, if 
    $V$ is a simple one-dimensional $k[G]$-module, then we obtain the following 
    commutative diagram, where we note that $k^*$ is an abelian group. 
    \begin{center}
        \begin{tikzcd}
            G \arrow[rr, "\rho_V"] \arrow[dd, "\pi"']  &  & \GL(V) \cong k^* \\
                                                    &  &   \\
            {G/G'} \arrow[rruu, "\tilde\rho_V"', dotted] &  &  
        \end{tikzcd}
    \end{center}
    In our case, we have $S'_3 = A_3$ and $S_3/A_3 = C_2$, which gives us 
    the following diagram. 
    \begin{center}
        \begin{tikzcd}
            G \arrow[rr, "\rho_V"] \arrow[dd, "\pi"']  &  & \GL(V) \cong k^* \\
                                                    &  &   \\
            C_2 \arrow[rruu, "\tilde\rho_V"', dotted] &  &  
        \end{tikzcd}
    \end{center}
    In particular, everything in $A_3$ has to be in the kernel and
    gets sent to $1$, so $\chi_1((123)) = \chi_2((123)) = 1$. 
    The last row is more difficult to find directly. However, if we know 
    everything in the character table except for the last row, then there's an 
    easy way to compute it. Let 
    \[ \chi_L = \chi_1 + \chi_2 + 2\chi_3. \] 
    We recall that $\chi_L(\sigma) = 0$ for all $\sigma \neq \id$; in particular, 
    we have $\chi_L((12)) = \chi_L((123)) = 0$. This gives 
    \[ \chi_1((12)) + \chi_2((12)) + 2\chi_3((12)) = 0, \] 
    and rearranging this, we obtain 
    \[ \chi_3((12)) = -\frac12(\chi_1((12)) + \chi_2((12)) 
    = -\frac12(1 + (-1)) = 0. \] 
    Similarly, we find that $\chi_3((123)) = -1$, so we have now completed the 
    character table of $S_3$. 
    \begin{align*}
        \begin{array}{|c|c|c|c|}
            \hline
            S_3    & [\id] & [(12)] & [(123)] \\ \hline
            \chi_1 & 1     & 1      & 1       \\ \hline
            \chi_2 & 1     & -1     & 1       \\ \hline
            \chi_3 & 2     & 0      & -1      \\ \hline
        \end{array} 
    \end{align*}
\end{exmp}

\begin{remark}{}
    Notice that for the $\C[S_3]$-module $V = \C^{3\times1}$ given the permutation 
    representation, $\chi_V(\sigma)$ gives the number of fixed points of 
    $\sigma \in S_3$. Then we have $\chi_V(\id) = 3$, $\chi_V((12)) = 1$, and 
    $\chi_V((123)) = 0$, so we find that $\chi_V = \chi_1 + \chi_3$ simply from 
    the character table. 
\end{remark}

\begin{exmp}{}
    Let $G = Q_8 = \{\pm1, \pm i, \pm j, \pm k\}$, where $ij = k$ and 
    $i^2 = j^2 = k^2 = -1$. Recall that 
    \[ \C[Q_8] \cong \C^4 \times M_2(\C), \] 
    so $Q_8$ has five conjugacy classes; they are $\{1\}$, $\{-1\}$, 
    $\{i, -i\}$, $\{j, -j\}$, and $\{k, -k\}$. Observe that $Q'_8 = \{\pm1\}$, 
    which implies that 
    \[ Q_8/Q'_8 = \langle i, j \mid i^2 = j^2 = 1,\, ij = ji 
    \rangle \cong C_2 \times C_2 . \] 
    Now, the first four rows of the character table are easy to compute, 
    as well as the first column. 
    \begin{align*}
        \begin{array}{|c|c|c|c|c|c|}
            \hline
            Q_8    & [1] & [-1] & [i] & [j] & [k] = [ij] \\ \hline
            \chi_1 & 1   & 1    & 1   & 1   & 1          \\ \hline
            \chi_2 & 1   & 1    & 1   & -1  & -1         \\ \hline
            \chi_3 & 1   & 1    & -1  & 1   & -1         \\ \hline
            \chi_4 & 1   & 1    & -1  & -1  & 1          \\ \hline
            \chi_5 & 2   & *    & *   & *   & *          \\ \hline 
        \end{array} 
    \end{align*}
    To obtain the last row, we apply the same trick as the previous example; 
    we set 
    \[ \chi_L = \chi_1 + \chi_2 + \chi_3 + \chi_4 + 2\chi_5 \] 
    and use the fact that $\chi_L(g) = 0$ when $g \neq 1$. This will finish 
    the character table.
    \begin{align*}
        \begin{array}{|c|c|c|c|c|c|}
            \hline
            Q_8    & [1] & [-1] & [i] & [j] & [k] = [ij] \\ \hline
            \chi_1 & 1   & 1    & 1   & 1   & 1          \\ \hline
            \chi_2 & 1   & 1    & 1   & -1  & -1         \\ \hline
            \chi_3 & 1   & 1    & -1  & 1   & -1         \\ \hline
            \chi_4 & 1   & 1    & -1  & -1  & 1          \\ \hline
            \chi_5 & 2   & -2   & 0   & 0   & 0          \\ \hline 
        \end{array} 
    \end{align*} 
\end{exmp}

We will soon see the interesting fact that if $G$ is a group and 
$\Phi, \Psi : G \to \C$ are maps, then we can define an ``inner product'' by 
\[ \langle \Phi, \Psi \rangle_G = \frac{1}{|G|} \sum_{g \in G} \Phi(g) 
\Psi(g^{-1}). \] 
Note that this is not actually an inner product, as it fails to satisfy 
some properties. However, we see that it is $\C$-bilinear; indeed, for 
$\lambda \in \C$, we have 
\[ \langle \Phi_1 + \lambda\Phi_2, \Psi \rangle_G = 
\frac{1}{|G|} \sum_{g \in G} (\Phi_1(g) + \lambda\Phi_2(g)) \Psi(g^{-1}) 
= \langle \Phi_1, \Psi \rangle_G + \lambda \langle \Phi_2, \Psi \rangle_G, \] 
and the other direction can be checked similarly. Moreover, 
$\langle \cdot, \cdot \rangle_G$ is symmetric, since 
\[ \langle \Phi, \Psi \rangle_G = \frac1{|G|} \sum_{g\in G} 
\Phi(g) \Psi(g^{-1}) = \frac1{|G|} \sum_{x\in G} \Phi(x^{-1}) \Psi(x) 
= \langle \Psi, \Phi \rangle_G, \] 
where in the second equality, we used the substitution $x = g^{-1}$. 

Next time, we'll see that characters are orthonormal with respect to this 
``inner product''. 

\section{Orthonormality of characters over ``inner product'' (10/22/2021)}
For this part of the course, we'll be working over $\C$. Last time, 
for a pair of maps $\Phi, \Psi : G \to \C$, we defined the ``inner product''
\[ \langle \Phi, \Psi \rangle_G = \frac{1}{|G|} \sum_{g \in G} \Phi(g) 
\Psi(g^{-1}), \] 
and saw that this was $\C$-bilinear and symmetric. We now make an important 
remark towards showing that $\langle \cdot, \cdot \rangle_G$ is a 
true inner product when working with characters. 

\begin{remark}{}
    If $\chi$ is a character, then $\chi(g^{-1}) = \overline{\chi(g)}$. 
\end{remark}
\begin{pf}
    We have $\chi(g) = \Tr(\rho(g))$ where $\rho : G \to \GL_n(\C)$ is a 
    representation. Let $g \in G$. Notice that $\rho(g)$ is triangularizable, 
    so 
    \[ S^{-1}\rho(g)S = \begin{pmatrix}
        \lambda_1 & & * \\ & \ddots & \\ 0 & & \lambda_n 
    \end{pmatrix}, \] 
    where $\lambda_1, \dots, \lambda_n$ are the eigenvalues of $\rho(g)$. 
    Since $G$ is finite, there exists an integer $d \geq 1$ such that $g^d = 1$. 
    Then, we can write 
    \[ \rho(g)^d = \rho(g^d) = \rho(1) = I, \] 
    which implies that 
    \[ (S^{-1}\rho(g)S)^d = (S^{-1}\rho(g)S)(S^{-1}\rho(g)S)\cdots 
    (S^{-1}\rho(g)S) = S^{-1}\rho(g)^d S = S^{-1}IS = I. \] 
    Therefore, we find that 
    \[ \begin{pmatrix}
        \lambda_1^d & & * \\ & \ddots & \\ 0 & & \lambda_n^d 
    \end{pmatrix} = I. \] 
    In particular, $\lambda_1, \dots, \lambda_n$ are $d$-th roots of unity. 
    Since $\rho(g^{-1})$ is the inverse of $\rho(g)$, we have 
    \[ \rho(g^{-1}) = \begin{pmatrix}
        \lambda_1^{-1} & & * \\ & \ddots & \\ 0 & & \lambda_n^{-1} 
    \end{pmatrix}. \] 
    Finally, it follows that 
    \[ \chi(g^{-1}) = \Tr(\rho(g^{-1})) = \lambda_1^{-1} + \cdots + \lambda_n^{-1}
    = \overline{\lambda_1} + \cdots + \overline{\lambda_n} = \overline{\chi(g)}. 
    \qedhere \] 
\end{pf}

Let ${\cal C}_1, \dots, {\cal C}_s$ be the conjugacy classes of $G$, and let 
$g_i$ be a representative of ${\cal C}_i$ for each $i = 1, \dots, s$. 
If $\chi_1$ and $\chi_2$ are characters, then 
\begin{align*}
    \langle \chi_1, \chi_2 \rangle_G 
    &= \frac1{|G|} \sum_{g\in G} \chi_1(g) \chi_2(g^{-1}) \\ 
    &= \frac1{|G|} \sum_{g\in G} \chi_1(g) \overline{chi_2(g)} \\ 
    &= \frac1{|G|} \sum_{i=1}^s \sum_{h\in{\cal C}_i} \chi_1(h) \overline{\chi_2(h)} \\
    &= \frac1{|G|} \sum_{i=1}^s |{\cal C}_i| \chi_1(g_i) \overline{\chi_2(g_i)},
\end{align*}
where the last equality follows since $\chi_1$ and $\chi_2$ are class functions. 

We'll now show that if $V_1, \dots, V_s$ are the inequivalent irreducible 
representations of $G$ with $\chi_i = \chi_{V_i}$, then $\langle \chi_i, 
\chi_j \rangle_G = \delta_{ij}$, where $\delta_{ij}$ denotes the Kronecker delta. 

First, we will consider an averaging trick. Let $G$ be a finite group, and let 
$k$ be an algebraically closed field such that $\ch(k) \nmid |G|$. Let 
$V$ and $W$ be $k[G]$-modules. If $T : V \to W$ is a $k$-linear transformation, 
we can define a map $\hat T : V \to W$ where for $v \in V$, we have 
\[ \hat T(v) = \frac1{|G|} \sum_{g \in G} g \cdot T(g^{-1} \cdot v). \] 
This makes sense because $T(g^{-1} \cdot v) \in W$. 

\begin{theo}{}
    If $T : V \to W$ is $k$-linear, then $\hat T : V \to W$ as above is 
    $k[G]$-linear. 
\end{theo}
\begin{pf}
    It suffices to show that $\hat T(h \cdot v) = h \cdot \hat T(v)$ for all 
    $h, v \in V$, because every element in $k[G]$ is a linear combination of 
    elements in $G$. By a direct computation, we have 
    \begin{align*}
        \hat T(h \cdot v) 
        &= \frac1{|G|} \sum_{g\in G} g \cdot T(g^{-1} \cdot (h \cdot v)) \\ 
        &= \frac1{|G|} \sum_{g\in G} g \cdot T((g^{-1} \cdot h) \cdot v) \\ 
        &= \frac1{|G|} \sum_{x\in G} h \cdot x \cdot T(x^{-1} \cdot v) 
        \quad \text{(substitute $x = h^{-1}g$)} \\ 
        &= h \cdot \frac1{|G|} \sum_{x\in G} x \cdot T(x^{-1} \cdot v) 
        = h \cdot \hat T(v). \qedhere 
    \end{align*}. 
\end{pf}

We now return to working over $\C$, and look at some consequences of the previous 
result. 

If $V$ and $W$ are non-isomorphic simple $\C[G]$-modules and $T : V \to W$ 
is $\C$-linear, then $\hat T \equiv 0$ since $\Hom_{\C[G]}(V, W) = (0)$. 
On the other hand, recall that 
\[ \Hom_{\C[G]}(V, V) = \End_{\C[G]}(V) = \C \] 
by Schur's lemma. This implies that if $T : V \to V$ is $\C$-linear, then 
$\hat T : V \to V$ is just scalar multiplication by some $\lambda \in \C$. 
That is, for all $v \in V$, we have 
\[ \hat T(v) = \lambda v. \] 
We now determine what $\lambda$ is. Let $L_g : V \to V$ be the map given by 
$L_g(v) = g \cdot v$. Then 
\begin{align*}
    \Tr(\hat T) 
    &= \Tr\left( \frac1{|G|} \sum_{g\in G} L_g \circ T \circ L_g^{-1} \right) \\
    &= \frac1{|G|} \sum_{g\in G} \Tr(L_g \circ T \circ L_g^{-1}) \\ 
    &= \frac1{|G|} \sum_{g\in G} \Tr(T) = \Tr(T). 
\end{align*}
On the other hand, we already know that $\Tr(\hat T) = \lambda \cdot \dim V$, so 
\[ \lambda = \frac{\Tr(T)}{\dim V}. \] 
Let $V$ and $W$ be two simple non-isomorphic $\C[G]$-modules with bases 
$\{v_1, \dots, v_m\}$ and $\{w_1, \dots, w_n\}$ respectively. Then, we obtain 
representations 
\begin{align*}
    \rho_1 &: G \to \GL_m(\C) \cong \GL(V), \\ 
    \rho_2 &: G \to \GL_n(\C) \cong \GL(W). 
\end{align*}
Note that if we want to find the matrix of a linear transformation, we pick 
our basis, and to get the $j$-th column, we apply the transformation 
to the $j$-th element of our basis. In particular, we find that 
\begin{align*}
    \rho_1(g) \cdot v_j &= \sum_{i=1}^m \rho_1(g)_{ij} \cdot v_i, & 
    \rho_2(g) \cdot w_j &= \sum_{i=1}^n \rho_2(g)_{ij} \cdot w_i. 
\end{align*}
Notice that we can view $\rho_1(g) = (\rho_1(g))_{ij}$ and 
$\rho_2(g) = (\rho_2(g))_{ij}$ as functions from $G$ to $\C$ 
by looking at their $(i, j)$-th entries. 

\begin{theo}{}
    For all $1 \leq i, j \leq m$ and $1 \leq p, q \leq n$, we have 
    $\langle (\rho_1)_{ij}, (\rho_2)_{pq} \rangle_G = 0$. 
\end{theo}
\begin{pf}
    Fix some $1 \leq i \leq m$ and $1 \leq q \leq n$. Define a linear 
    transformation $T : V \to W$ by $T(v_i) = w_q$, and $T(v_k) = 0$
    whenever $k \neq i$. Then $\hat T \equiv 0$ since $V$ and $W$ 
    are non-isomorphic simple modules. For all $1 \leq j \leq m$, we have 
    \[ 0 = \hat T(v_j) = \frac1{|G|} \sum_{g \in G} g \cdot T(g^{-1} \cdot v_j) 
    = \frac1{|G|} \sum_{g\in G} \rho_2(g) \cdot T(\rho_1(g^{-1}) \cdot v_j). \]
    By definition, we have 
    \[ \rho_1(g^{-1}) \cdot v_j = \sum_{k=1}^m \rho_1(g^{-1})_{kj} \cdot v_k, \] 
    which gives us 
    \begin{align*}
        T(\rho_1(g^{-1}) \cdot v_j) 
        &= T \left( \sum_{k=1}^m \rho_1(g^{-1})_{kj} \cdot v_k \right) 
        = \rho_1(g^{-1})_{ij} \cdot T(v_i) 
        = \rho_1(g^{-1})_{ij} \cdot w_q, 
    \end{align*}
    where the second equality comes from throwing everything away except when 
    $k = i$. Finally, we obtain 
    \begin{align*}
        0 &= \frac1{|G|} \sum_{g\in G} \rho_2(g) \cdot T(\rho_1(g) \cdot v_j) \\ 
        &= \frac1{|G|} \sum_{g\in G} \rho_1(g^{-1})_{ij} \cdot \rho_2(g) \cdot w_q \\
        &= \frac1{|G|} \sum_{g\in G} \rho_1(g^{-1})_{ij} 
           \left[ \sum_{k=1}^n \rho_2(g)_{kq} \cdot w_k \right] \\
        &= \frac1{|G|} \sum_{g\in G} \rho_1(g^{-1})_{ij} \cdot \rho_2(g)_{pq} \\ 
        &= \langle (\rho_2)_{pq}, (\rho_1)_{ij} \rangle_G 
        = \langle (\rho_1)_{ij}, (\rho_2)_{pq} \rangle_G.
    \end{align*}
    Since $1 \leq i, j \leq m$ and $1 \leq p, q, \leq n$ were arbitrary, we 
    are done. 
\end{pf}
\section{Character tables of $S_4$ and $S_5$ (10/27/2021)}
Recall that we can write $\C[G] \cong M_{n_i}(\C) \times \cdots \times M_{n_s}(\C)$, 
where $s$ is the number of conjugacy classes of $G$. Then, we obtain $s$ 
irreducible characters $\chi_1, \dots, \chi_s$, with 
\[ \chi_i(1) = n_i, \] 
where we call $n_i$ the degree of $\chi_i$ for each $i = 1, \dots, s$. We also saw 
that 
\[ \delta_{ij} = \langle \chi_i, \chi_j \rangle_G = \frac1{|G|} \sum_{g\in G} 
\chi_i(g) \overline{\chi_j(g)}. \] 
For a $\C[G]$-module $V \cong V_1^{a_1} \oplus \cdots \oplus V_s^{a_s}$, we have 
the formula 
\[ \chi_V = a_1 \chi_i + \cdots + a_s \chi_s, \] 
where $\chi_i = \chi_{V_i}$ for all $i = 1, \dots, s$. We saw that this gives 
$a_i = \langle \chi_V, \chi_i \rangle_G$ by orthonormality. In particular, we 
find that 
\[ \langle \chi_V, \chi_V \rangle_G = \sum_{i=1}^s a_i^2 \|\chi_i\|^2 = 
\sum_{i=1}^s a_i^2. \] 
Let's use our results to find the character tables of $S_4$ and $S_5$. 

\begin{exmp}{}
    To find the character table of $S_4$, we first recall that 
    $\C[S_4] \cong \C \times \C \times M_2(\C) \times M_3(\C) \times M_3(\C)$, 
    which we determined in Example 11.3. We also saw that the $5$ conjugacy 
    classes and their sizes were 
    \begin{align*} 
        \#[\id] &= 1, \\ 
        \#[(12)] &= 6, \\
        \#[(123)] &= 8, \\
        \#[(1234)] &= 6, \\
        \#[(12)(34)] &= 3. 
    \end{align*}
    Since $S'_4 = A_4$, we have $S_4/S'_4 = S_4/A_4 \cong C_2$, so we can immediately 
    determine the degree $1$ characters, as well as the first column as we 
    know the degrees of all the characters. 
    \begin{align*}
        \begin{array}{|c|c|c|c|c|c|}
            \hline
            S_4    & \overset{1}{[\id]} & \overset{6}{[(12)]} & \overset{8}{[(123)]} & \overset{6}{[(1234)]} & \overset{3}{[(12)(34)]} \\ \hline
            \chi_1 & 1                  & 1                   & 1                    & 1                     & 1                       \\ \hline
            \chi_2 & 1                  & -1                  & 1                    & -1                    & 1                       \\ \hline
            \chi_3 & 2                  &                     &                      &                       &                         \\ \hline
            \chi_4 & 3                  &                     &                      &                       &                         \\ \hline
            \chi_5 & 3                  &                     &                      &                       &                         \\ \hline
        \end{array} 
    \end{align*}
    To compute the rest of the character table, we'll usually find a nice 
    representation of the group in practice. Consider the permutation representation
    $P : S_4 \to \GL_4(\C)$ given by 
    \[ P(\sigma)e_j = e_{\sigma(j)}. \] 
    Recall that $\chi_P(\sigma)$ is the number of fixed points of $\sigma \in S_4$. 
    For now, let's add $\chi_P$ to the character table. 
    \begin{align*}
        \begin{array}{|c|c|c|c|c|c|}
            \hline
            S_4    & \overset{1}{[\id]} & \overset{6}{[(12)]} & \overset{8}{[(123)]} & \overset{6}{[(1234)]} & \overset{3}{[(12)(34)]} \\ \hline
            \chi_1 & 1                  & 1                   & 1                    & 1                     & 1                       \\ \hline
            \chi_2 & 1                  & -1                  & 1                    & -1                    & 1                       \\ \hline
            \chi_3 & 2                  &                     &                      &                       &                         \\ \hline
            \chi_4 & 3                  &                     &                      &                       &                         \\ \hline
            \chi_5 & 3                  &                     &                      &                       &                         \\ \hline
            \chi_P & 4                  & 2                   & 1                    & 0                     & 0                       \\ \hline 
        \end{array} 
    \end{align*}
    By Exercise 5 in Homework 3, we find that 
    \[ \langle \chi_P, \chi_P \rangle = \frac1{4!} \sum_{\sigma \in S_4} 
    \fix(\sigma)^2 = \frac1{24} (1 \cdot 4^2 + 6 \cdot 2^2 + 8 \cdot 1^2 + 6 
    \cdot 0^2 + 3 \cdot 0^2) = 2, \] 
    where $\fix(\sigma)$ denotes the number of fixed points of $\sigma$. 
    In particular, we see that $\chi_P$ is not an irreducible character, 
    but recalling the formula 
    \[ \langle \chi_V, \chi_V \rangle = \sum_{i=1}^s a_i^2, \] 
    we see that $\chi_P$ can be decomposed as the sum of two irreducible components. 
    Now, looking at $\langle \chi_P, \chi_1 \rangle$, we obtain 
    \[ \langle \chi_P, \chi_1 \rangle = \frac1{24} (4 \cdot 1 \cdot 1 
    + 2 \cdot 1 \cdot 6 + 1 \cdot 1 \cdot 8) = 1. \] 
    This means that $\chi_P = \chi_1 + \chi$, where $\chi$ is some irreducible 
    character. In other words, $\chi_P - \chi_1$ is an irreducible character. 
    Notice that $\chi_P(\id) - \chi_1(\id) = 4 - 1 = 3$, so this character is 
    one of $\chi_4$ or $\chi_5$. Let's say we have $\chi_4 = \chi_P - \chi_1$, since we 
    can just flip the rows if necessary. This allows us to fill out another row in 
    the character table. 
    \begin{align*}
        \begin{array}{|c|c|c|c|c|c|}
            \hline
            S_4    & \overset{1}{[\id]} & \overset{6}{[(12)]} & \overset{8}{[(123)]} & \overset{6}{[(1234)]} & \overset{3}{[(12)(34)]} \\ \hline
            \chi_1 & 1                  & 1                   & 1                    & 1                     & 1                       \\ \hline
            \chi_2 & 1                  & -1                  & 1                    & -1                    & 1                       \\ \hline
            \chi_3 & 2                  &                     &                      &                       &                         \\ \hline
            \chi_4 & 3                  & 1                   & 0                    & -1                    & -1                      \\ \hline
            \chi_5 & 3                  &                     &                      &                       &                         \\ \hline
            \chi_P & 4                  & 2                   & 1                    & 0                     & 0                       \\ \hline 
        \end{array} 
    \end{align*}
    On Question 1 of Homework 4, we will show that if $\chi$ is a degree $1$ 
    character and for each $d \geq 1$, we let $X_d$ be the set of irreducible 
    characters of degree $d$, then multiplication by $\chi$ permutes $X_d$ 
    for all $d \geq 1$. 

    In particular, since $\chi_2$ is a degree $1$ character, we can multiply 
    $\chi_4$ with $\chi_2$ to obtain the other degree $3$ character 
    $\chi_5 = \chi_4 \cdot \chi_2$, as multiplication by $\chi_2$ permutes 
    $X_3 = \{\chi_4, \chi_5\}$. Similarly, if we multiply $\chi_5$ with 
    $\chi_2$, we again obtain $\chi_4$. 
    \begin{align*}
        \begin{array}{|c|c|c|c|c|c|}
            \hline
            S_4    & \overset{1}{[\id]} & \overset{6}{[(12)]} & \overset{8}{[(123)]} & \overset{6}{[(1234)]} & \overset{3}{[(12)(34)]} \\ \hline
            \chi_1 & 1                  & 1                   & 1                    & 1                     & 1                       \\ \hline
            \chi_2 & 1                  & -1                  & 1                    & -1                    & 1                       \\ \hline
            \chi_3 & 2                  &                     &                      &                       &                         \\ \hline
            \chi_4 & 3                  & 1                   & 0                    & -1                    & -1                      \\ \hline
            \chi_5 & 3                  & -1                  & 0                    & 1                     & -1                      \\ \hline
            \chi_P & 4                  & 2                   & 1                    & 0                     & 0                       \\ \hline 
        \end{array} 
    \end{align*}
    What happens if we do this with $\chi_3$? We have $X_2 = \{\chi_3\}$, 
    and we know that multiplication by $\chi_2$ must permute $X_2$, so that 
    $\chi_3 \cdot \chi_2 = \chi_3$. This tells us that 
    $\chi_3((12)) = \chi_3((1234)) = 0$, since $\chi_2((12)) = \chi_2((1234)) 
    = -1$. Let's now try to find the remaining entries in the row. 
    \begin{align*}
        \begin{array}{|c|c|c|c|c|c|}
            \hline
            S_4    & \overset{1}{[\id]} & \overset{6}{[(12)]} & \overset{8}{[(123)]} & \overset{6}{[(1234)]} & \overset{3}{[(12)(34)]} \\ \hline
            \chi_1 & 1                  & 1                   & 1                    & 1                     & 1                       \\ \hline
            \chi_2 & 1                  & -1                  & 1                    & -1                    & 1                       \\ \hline
            \chi_3 & 2                  & 0                   & a                    & 0                     & b                       \\ \hline
            \chi_4 & 3                  & 1                   & 0                    & -1                    & -1                      \\ \hline
            \chi_5 & 3                  & -1                  & 0                    & 1                     & -1                      \\ \hline
            \chi_P & 4                  & 2                   & 1                    & 0                     & 0                       \\ \hline 
        \end{array} 
    \end{align*}
    We take the inner product of $\chi_3$ with some of the other irreducible 
    characters. Observe that 
    \begin{align*}
        0 &= \langle \chi_3, \chi_1 \rangle = \frac1{24}(2 \cdot 1 \cdot 1 
        + a \cdot 1 \cdot 8 + b \cdot 1 \cdot 3) = \frac1{24}(2 + 8a + 3b), \\ 
        0 &= \langle \chi_3, \chi_4 \rangle = \frac1{24}(2 \cdot 3 \cdot 1 
        + a \cdot 0 \cdot 8 + b \cdot (-1) \cdot 3) = \frac1{24}(6 - 3b). 
    \end{align*}
    Solving this yields $a = -1$ and $b = 2$, so we have now completed 
    the character table of $S_4$. 
    \begin{align*}
        \begin{array}{|c|c|c|c|c|c|}
            \hline
            S_4    & \overset{1}{[\id]} & \overset{6}{[(12)]} & \overset{8}{[(123)]} & \overset{6}{[(1234)]} & \overset{3}{[(12)(34)]} \\ \hline
            \chi_1 & 1                  & 1                   & 1                    & 1                     & 1                       \\ \hline
            \chi_2 & 1                  & -1                  & 1                    & -1                    & 1                       \\ \hline
            \chi_3 & 2                  & 0                   & -1                   & 0                     & 2                       \\ \hline
            \chi_4 & 3                  & 1                   & 0                    & -1                    & -1                      \\ \hline
            \chi_5 & 3                  & -1                  & 0                    & 1                     & -1                      \\ \hline
        \end{array} 
    \end{align*}
\end{exmp}

\begin{exmp}{}
    Let's now find the character table of $S_5$. Note that $S_5$ has $7$ 
    conjugacy classes; their sizes are 
    \begin{align*}
        \#[\id] &= 1, \\ 
        \#[(12)] &= 10, \\ 
        \#[(123)] &= 20, \\ 
        \#[(1234)] &= 30, \\ 
        \#[(12345)] &= 24, \\ 
        \#[(12)(34)] &= 15, \\ 
        \#[(12)(345)] &= 20. 
    \end{align*}
    These sizes can be determined using Exercise 6 of Homework 3. This states that 
    if $\sigma \in S_n$ has disjoint cycle structure with $\lambda_i \geq 0$ 
    $i$-cycles for $i = 1, \dots, n$ (so that $\sum_{i=1}^n i\lambda_i = n$), 
    then the conjugacy class of $\sigma$ has size 
    \[ \frac{n!}{\prod_{i=1}^n \lambda_i! i^{\lambda_i}}. \] 

    \newpage
    Now, we recall that $S'_5 = A_5$, so $S_5/S'_5 = S_5/A_5 \cong C_2$. 
    Then the odd cycles get sent to $-1$ and the even cycles get sent to 
    $1$, so we obtain the first two rows of the character table. 
    \begin{align*}
        \begin{array}{|c|c|c|c|c|c|c|c|}
            \hline
            S_5    & \overset{1}{[\id]} & \overset{10}{[(12)]} & \overset{20}{[(123)]} & \overset{30}{[(1234)]} & \overset{24}{[(12345)]} & \overset{15}{[(12)(34)]} & \overset{20}{[(12)(345)]} \\ \hline
            \chi_1 & 1                  & 1                    & 1                     & 1                      & 1                       & 1                        & 1                         \\ \hline
            \chi_2 & 1                  & -1                   & 1                     & -1                     & 1                       & 1                        & -1                        \\ \hline
            \chi_3 &                    &                      &                       &                        &                         &                          &                           \\ \hline
            \chi_4 &                    &                      &                       &                        &                         &                          &                           \\ \hline
            \chi_5 &                    &                      &                       &                        &                         &                          &                           \\ \hline
            \chi_6 &                    &                      &                       &                        &                         &                          &                           \\ \hline
            \chi_7 &                    &                      &                       &                        &                         &                          &                           \\ \hline 
        \end{array} 
    \end{align*}
    Take the permutation representation $P : S_5 \to \GL_5(\C)$ given by 
    \[ P(\sigma)e_j = e_{\sigma(j)}. \] 
    Again, $\chi_P(\sigma)$ gives the number of fixed points of $\sigma$, so 
    let's add $\chi_P$ to the character table for now. 
    \begin{align*}
        \begin{array}{|c|c|c|c|c|c|c|c|}
            \hline
            S_5    & \overset{1}{[\id]} & \overset{10}{[(12)]} & \overset{20}{[(123)]} & \overset{30}{[(1234)]} & \overset{24}{[(12345)]} & \overset{15}{[(12)(34)]} & \overset{20}{[(12)(345)]} \\ \hline
            \chi_1 & 1                  & 1                    & 1                     & 1                      & 1                       & 1                        & 1                         \\ \hline
            \chi_2 & 1                  & -1                   & 1                     & -1                     & 1                       & 1                        & -1                        \\ \hline
            \chi_3 &                    &                      &                       &                        &                         &                          &                           \\ \hline
            \chi_4 &                    &                      &                       &                        &                         &                          &                           \\ \hline
            \chi_5 &                    &                      &                       &                        &                         &                          &                           \\ \hline
            \chi_6 &                    &                      &                       &                        &                         &                          &                           \\ \hline
            \chi_7 &                    &                      &                       &                        &                         &                          &                           \\ \hline
            \chi_P & 5                  & 3                    & 2                     & 1                      & 0                       & 1                        & 0                         \\ \hline 
        \end{array} 
    \end{align*}
    By Exercise 5 of Homework 3, we have 
    \[ \langle \chi_P, \chi_P \rangle = \frac1{5!} \sum_{\sigma \in S_5} 
    \fix(\sigma)^2 = 2. \] 
    We compute that 
    \[ \langle \chi_P, \chi_1 \rangle = \frac1{5!} 
    (5 \cdot 1 \cdot 1 + 3 \cdot 1 \cdot 10 + 2 \cdot 1 \cdot 20 + 1 \cdot 1 \cdot 30
    + 1 \cdot 1 \cdot 15) = 1. \] 
    As in the previous example, this tells us that $\chi_P - \chi_1$ is an 
    irreducible character. Since $\chi_P(\id) - \chi_1(\id) = 5 - 1 = 4$, 
    we obtain an irreducible character $\chi_3 = \chi_P - \chi_1$ of degree $4$.
    \begin{align*}
        \begin{array}{|c|c|c|c|c|c|c|c|}
            \hline
            S_5    & \overset{1}{[\id]} & \overset{10}{[(12)]} & \overset{20}{[(123)]} & \overset{30}{[(1234)]} & \overset{24}{[(12345)]} & \overset{15}{[(12)(34)]} & \overset{20}{[(12)(345)]} \\ \hline
            \chi_1 & 1                  & 1                    & 1                     & 1                      & 1                       & 1                        & 1                         \\ \hline
            \chi_2 & 1                  & -1                   & 1                     & -1                     & 1                       & 1                        & -1                        \\ \hline
            \chi_3 & 4                  & 2                    & 1                     & 0                      & -1                      & 0                        & -1                        \\ \hline
            \chi_4 &                    &                      &                       &                        &                         &                          &                           \\ \hline
            \chi_5 &                    &                      &                       &                        &                         &                          &                           \\ \hline
            \chi_6 &                    &                      &                       &                        &                         &                          &                           \\ \hline
            \chi_7 &                    &                      &                       &                        &                         &                          &                           \\ \hline
            \chi_P & 5                  & 3                    & 2                     & 1                      & 0                       & 1                        & 0                         \\ \hline 
        \end{array} 
    \end{align*}
    We can also take $\chi_4 = \chi_3 \cdot \chi_2$ to get another irreducible 
    character. 
    \begin{align*}
        \begin{array}{|c|c|c|c|c|c|c|c|}
            \hline
            S_5    & \overset{1}{[\id]} & \overset{10}{[(12)]} & \overset{20}{[(123)]} & \overset{30}{[(1234)]} & \overset{24}{[(12345)]} & \overset{15}{[(12)(34)]} & \overset{20}{[(12)(345)]} \\ \hline
            \chi_1 & 1                  & 1                    & 1                     & 1                      & 1                       & 1                        & 1                         \\ \hline
            \chi_2 & 1                  & -1                   & 1                     & -1                     & 1                       & 1                        & -1                        \\ \hline
            \chi_3 & 4                  & 2                    & 1                     & 0                      & -1                      & 0                        & -1                        \\ \hline
            \chi_4 & 4                  & -2                   & 1                     & 0                      & -1                      & 0                        & 1                         \\ \hline
            \chi_5 &                    &                      &                       &                        &                         &                          &                           \\ \hline
            \chi_6 &                    &                      &                       &                        &                         &                          &                           \\ \hline
            \chi_7 &                    &                      &                       &                        &                         &                          &                           \\ \hline
            \chi_P & 5                  & 3                    & 2                     & 1                      & 0                       & 1                        & 0                         \\ \hline 
        \end{array} 
    \end{align*}
    Now, let $a = \chi_5(\id)$, $b = \chi_6(\id)$, and $c = \chi_7(\id)$. 
    Then we know that 
    \[ 1^2 + 1^2 + 4^2 + 4^2 + a^2 + b^2 + c^2 = 120, \] 
    or equivalently, 
    \[ a^2 + b^2 + c^2 = 86. \] 
    Therefore, we have $2 \leq a, b, c \leq 9$, and we can deduce that 
    $(a, b, c) = (5, 5, 6)$. We have $X_6 = \{\chi_7\}$, and since $\chi_2$ 
    is of degree $1$, we have $\chi_7 \cdot \chi_2 = \chi_7$, which tells us 
    that $\chi_7((12)) = \chi((1234)) = \chi((12)(345)) = 0$, since 
    their counterparts in $\chi_2$ are equal to $-1$. We can now fill out 
    the first column and some of the entries of $\chi_7$. 
    \begin{align*}
        \begin{array}{|c|c|c|c|c|c|c|c|}
            \hline
            S_5    & \overset{1}{[\id]} & \overset{10}{[(12)]} & \overset{20}{[(123)]} & \overset{30}{[(1234)]} & \overset{24}{[(12345)]} & \overset{15}{[(12)(34)]} & \overset{20}{[(12)(345)]} \\ \hline
            \chi_1 & 1                  & 1                    & 1                     & 1                      & 1                       & 1                        & 1                         \\ \hline
            \chi_2 & 1                  & -1                   & 1                     & -1                     & 1                       & 1                        & -1                        \\ \hline
            \chi_3 & 4                  & 2                    & 1                     & 0                      & -1                      & 0                        & -1                        \\ \hline
            \chi_4 & 4                  & -2                   & 1                     & 0                      & -1                      & 0                        & 1                         \\ \hline
            \chi_5 & 5                  &                      &                       &                        &                         &                          &                           \\ \hline
            \chi_6 & 5                  &                      &                       &                        &                         &                          &                           \\ \hline
            \chi_7 & 6                  & 0                    &                       & 0                      &                         &                          & 0                         \\ \hline
            \chi_P & 5                  & 3                    & 2                     & 1                      & 0                       & 1                        & 0                         \\ \hline 
        \end{array} 
    \end{align*}
    Observe that $S_5$ acts on the subsets of size $2$; namely 
    $\{\{i, j\} : i, j \in \{1, 2, 3, 4, 5\}\}$. There are $\binom{5}{2} = 10$ 
    such subsets. Instead of considering the permutation representation 
    $P : S_5 \to \GL_5(\C)$, we have another representation 
    $Q : S_5 \to \GL_{10}(\C)$ where from the basis $\{e_{\{i,j\}} : 
    i, j \in \{1, 2, 3, 4, 5\}\}$, we define 
    \[ \sigma \cdot e_{\{i,j\}} = e_{\sigma(i)\sigma(j)}. \] 
    Let's now determine $\chi_Q$. This time, $\chi_Q(\sigma)$ is the number of sets 
    of size two that are fixed by $\sigma$. 
    \begin{align*}
        \begin{array}{|c|c|c|c|c|c|c|c|}
            \hline
            S_5    & \overset{1}{[\id]} & \overset{10}{[(12)]} & \overset{20}{[(123)]} & \overset{30}{[(1234)]} & \overset{24}{[(12345)]} & \overset{15}{[(12)(34)]} & \overset{20}{[(12)(345)]} \\ \hline
            \chi_1 & 1                  & 1                    & 1                     & 1                      & 1                       & 1                        & 1                         \\ \hline
            \chi_2 & 1                  & -1                   & 1                     & -1                     & 1                       & 1                        & -1                        \\ \hline
            \chi_3 & 4                  & 2                    & 1                     & 0                      & -1                      & 0                        & -1                        \\ \hline
            \chi_4 & 4                  & -2                   & 1                     & 0                      & -1                      & 0                        & 1                         \\ \hline
            \chi_5 & 5                  &                      &                       &                        &                         &                          &                           \\ \hline
            \chi_6 & 5                  &                      &                       &                        &                         &                          &                           \\ \hline
            \chi_7 & 6                  & 0                    &                       & 0                      &                         &                          & 0                         \\ \hline
            \chi_Q & 10                 & 4                    & 1                     & 0                      & 0                       & 2                        & 1                         \\ \hline 
        \end{array} 
    \end{align*}
    For example, note that $(12) \cdot e_{\{1,2\}} 
    = e_{\{1,2\}}$, because the set $\{2, 1\}$ is the same as the set $\{1, 2\}$. 
    Then, we see that the basis elements fixed by $(12)$ are $e_{\{1,2\}}$, 
    $e_{\{3,4\}}$, $e_{\{3,5\}}$, and $e_{\{4,5\}}$, so $\chi_Q((12)) = 4$. 
    The other values of $\chi_Q$ can be found similarly. We now compute that 
    \[ \langle \chi_Q, \chi_Q \rangle = \frac1{5!}(1 \cdot 10^2 
    + 10 \cdot 4^2 + 20 \cdot 1^2 + 15 \cdot 2^2 + 20 \cdot 1^2) = 3. \] 
    Therefore, $\chi_Q$ is the sum of $3$ distinct irreducible characters. 
    We find that 
    \begin{align*}
        \langle \chi_Q, \chi_1 \rangle &= \frac1{5!}(10 \cdot 1 \cdot 1 
        + 4 \cdot 1 \cdot 10 + 1 \cdot 1 \cdot 20 + 2 \cdot 1 \cdot 15 
        + 1 \cdot 1 \cdot 20) = 1, \\ 
        \langle \chi_Q, \chi_3 \rangle &= \frac1{5!}(10 \cdot 4 \cdot 1 
        + 4 \cdot 2 \cdot 10 + 1 \cdot 1 \cdot 20 + 1 \cdot (-1) \cdot 20) = 1. 
    \end{align*}
    Hence, $\chi_Q - \chi_1 - \chi_3$ is an irreducible character. 
    Note that $\chi_Q(\id) - \chi_1(\id) - \chi_3(\id) = 10 - 1 - 4 = 5$, and 
    since $\chi_5(\id) = 5$, we can set $\chi_5 = \chi_Q - \chi_1 - \chi_3$. 
    Then, $X_5 = \{\chi_5, \chi_6\}$, and since $\chi_2$ is a degree $1$ 
    character, it must permute $X_5$, so we can set $\chi_6 = \chi_5 \cdot \chi_2$. 
    This completes two more rows of the character table. 
    \begin{align*}
        \begin{array}{|c|c|c|c|c|c|c|c|}
            \hline
            S_5    & \overset{1}{[\id]} & \overset{10}{[(12)]} & \overset{20}{[(123)]} & \overset{30}{[(1234)]} & \overset{24}{[(12345)]} & \overset{15}{[(12)(34)]} & \overset{20}{[(12)(345)]} \\ \hline
            \chi_1 & 1                  & 1                    & 1                     & 1                      & 1                       & 1                        & 1                         \\ \hline
            \chi_2 & 1                  & -1                   & 1                     & -1                     & 1                       & 1                        & -1                        \\ \hline
            \chi_3 & 4                  & 2                    & 1                     & 0                      & -1                      & 0                        & -1                        \\ \hline
            \chi_4 & 4                  & -2                   & 1                     & 0                      & -1                      & 0                        & 1                         \\ \hline
            \chi_5 & 5                  & 1                    & -1                    & -1                     & 0                       & 1                        & 1                         \\ \hline
            \chi_6 & 5                  & -1                   & -1                    & 1                      & 0                       & 1                        & -1                        \\ \hline
            \chi_7 & 6                  & 0                    & d                     & 0                      & e                       & f                        & 0                         \\ \hline
            \chi_Q & 10                 & 4                    & 1                     & 0                      & 0                       & 2                        & 1                         \\ \hline 
        \end{array} 
    \end{align*}
    Finally, let's compute $\chi_7$ by determining $d = \chi_7((123))$, 
    $e = \chi_7((12345))$, and $f = \chi_7((12)(34))$. We have 
    \begin{align*}
        0 &= \langle \chi_7, \chi_5 \rangle = \frac1{5!}(30 - 20d + 15f), \\ 
        0 &= \langle \chi_7, \chi_3 \rangle = \frac1{5!}(24 + 20d - 24e), \\ 
        0 &= \langle \chi_7, \chi_1 \rangle = \frac1{5!}(6 + 20d + 24e + 15f).
    \end{align*}
    This is a system of $3$ equations and $3$ unknowns; solving it yields 
    $(d, e, f) = (0, 1, -2)$, completing the character table of $S_5$. 
    \begin{align*}
        \begin{array}{|c|c|c|c|c|c|c|c|}
            \hline
            S_5    & \overset{1}{[\id]} & \overset{10}{[(12)]} & \overset{20}{[(123)]} & \overset{30}{[(1234)]} & \overset{24}{[(12345)]} & \overset{15}{[(12)(34)]} & \overset{20}{[(12)(345)]} \\ \hline
            \chi_1 & 1                  & 1                    & 1                     & 1                      & 1                       & 1                        & 1                         \\ \hline
            \chi_2 & 1                  & -1                   & 1                     & -1                     & 1                       & 1                        & -1                        \\ \hline
            \chi_3 & 4                  & 2                    & 1                     & 0                      & -1                      & 0                        & -1                        \\ \hline
            \chi_4 & 4                  & -2                   & 1                     & 0                      & -1                      & 0                        & 1                         \\ \hline
            \chi_5 & 5                  & 1                    & -1                    & -1                     & 0                       & 1                        & 1                         \\ \hline
            \chi_6 & 5                  & -1                   & -1                    & 1                      & 0                       & 1                        & -1                        \\ \hline
            \chi_7 & 6                  & 0                    & 0                     & 0                      & 1                       & -2                       & 0                         \\ \hline
        \end{array} 
    \end{align*}
\end{exmp}

\section{Algebraic numbers and algebraic integers (10/29/2021)}
Write $\C[G] \cong M_{n_1}(\C) \times \cdots \times M_{n_s}(\C)$, and let 
$\chi_1, \dots, \chi_s$ be the irreducible characters. We have $\chi_i(1_G) 
= n_i$, which is the degree of $\chi_i$. Moreover, since $\langle \chi_i, 
\chi_j \rangle = \delta_{ij}$, we know that $\|\chi\|^2 = 1$ if and only if 
$\chi$ is irreducible. In particular, if the character $\chi = a_1 \chi_1 
+ \cdots + a_s \chi_s$ is irreducible, then 
\[ \|\chi\|^2 = a_1^2 + \cdots + a_s^2 = 1 \] 
implies that there is a unique $1 \leq j \leq s$ such that $a_j = 1$ and $a_i = 0$ 
for all $i \neq j$. Hence, $\|\chi\|^2 = 1$ if and only if $\chi = \chi_j$ for some 
$1 \leq j \leq s$. Today, we'll introduce the algebraic integers, which will help 
us to prove the last part of the big theorem, which states that $n_i \mid |G|$ for all 
$1 \leq i \leq s$. 

\begin{defn}{}
    We define the {\bf algebraic numbers} $\A \subseteq \C$ to be the set 
    \[ \A = \{\alpha \in \C : P(\alpha) = 0 \text{ for some monic polynomial }
    P(x) \in \Z[x]\}. \] 
\end{defn}

\begin{exmp}{}
    \begin{enumerate}[(1)]
        \item Observe that $\Z \subseteq \A$, since for any $n \in \Z$, the 
              polynomial $P(x) = x - n$ has $n$ as a root. 
        \item If $\alpha \in \Q$, then $\beta = e^{\pi i\alpha} \in \A$. Indeed, 
              write $\alpha = a/b$ for some $a, b \in \Z$ with $b \neq 0$. 
              Taking the polynomial $P(x) = x^{2b-1}$, we have $P(\beta) = 
              (e^{\pi ia/b})^{2b} - 1 = e^{2\pi ia} - 1 = 0$. 
        \item Let $a \in \Z$ and $b \in \Z^+$. Then $\alpha = \sqrt[b]{a} \in 
              \A$ by taking $P(x) = x^b - a \in \Z[x]$. 
    \end{enumerate}
\end{exmp}

\begin{lemma}{}
    We have $\A \cap \Q = \Z$. 
\end{lemma}
\begin{pf}
    Exercise 1 of Homework 4 shows that $\Z \subseteq \A \cap \Q$. Next, 
    we clearly have $\alpha = 0 \in \Z$. Suppose that we have a 
    nonzero $\alpha = a/b$ with $\gcd(a, b) = 1$ and $b > 0$ which is a root 
    of the monic polynomial 
    \[ x^n + c_{n-1} x^{n-1} + \cdots + c_1 x + c_0 \in \Z[x]. \] 
    Moreover, we can assume that $c_0 \neq 0$, for otherwise we could just 
    reduce the degree of the polynomial. Then by the rational roots theorem, 
    we obtain $b \mid 1$ and $a \mid c_0$, which implies that $\alpha = 
    a/b \in \Z$. Therefore, we conclude that $\A \cap \Q = \Z$. 
\end{pf}

This statement is very simple, but it has some very powerful consequences. 
Recall that a finitely generated $\Z$-submodule of $(\C, +)$ is given by 
all elements of the form 
\[ \{n_1 \lambda_1 + \cdots + n_p \lambda_p : n_1, \dots, n_p \in \Z\} \] 
for some fixed $\lambda_1, \dots, \lambda_p \in \C$. For example, we have 
$\Z[i] = \{a + bi : a, b \in \Z\}$ by choosing $\lambda_1 = 1$ and $\lambda_2 = i$.
We'll now prove some characterizations of algebraic numbers.  

\begin{theo}{}
    Let $\alpha \in \C$. The following are equivalent: 
    \begin{enumerate}[(1)]
        \item $\alpha \in \A$. 
        \item There exists a nonzero finitely generated $\Z$-submodule $M$ of 
              $\C$ such that $\alpha M \subseteq M$. 
        \item There exists $p \geq 1$ and $\lambda_1, \dots, \lambda_p \in 
              \C \setminus \{0\}$ such that $\alpha \lambda_i \in 
              \Z\lambda_1 + \cdots + \Z\lambda_p$ for all $i = 1, \dots, p$. 
    \end{enumerate}
\end{theo}
\begin{pf}
    (1) $\Rightarrow$ (2). If $\alpha \in \A$, then there exists $n \geq 1$ 
    and $c_0, \dots, c_{n-1} \in \Z$ such that 
    \begin{equation}
        \alpha^n + c_{n-1} \alpha^{n-1} + \cdots + c_1 \alpha + c_0 = 0. 
    \end{equation}
    Let $M = \Z + \Z\alpha + \Z\alpha^2 + \cdots + \Z\alpha^{n-1}$. This is a 
    finitely generated $\Z$-submodule, and $M \neq (0)$ since $1 \in M$. 
    Now, we claim that $\alpha M \subseteq M$. To this end, suppose that 
    $a_0 + a_1 \alpha + \cdots + a_{n-1} \alpha^{n-1} \in M$. Then we have 
    \begin{align*}
        \alpha (a_0 + a_1 \alpha + \cdots + a_{n-1} \alpha^{n-1}) 
        &= a_0 \alpha + a_1 \alpha^2 + \cdots + a_{n-2} \alpha^{n-1} 
        + a_{n-1} \alpha^n \\ 
        &= a_0 \alpha + a_1 \alpha^2 + \cdots + a_{n-2} \alpha^{n-1} 
        + a_{n-1} (- c_{n-1} \alpha^{n-1} - \cdots - c_1 \alpha - c_0) \\ 
        &= -a_{n-1} c_0 + (a_0 - a_{n-1} c_1) \alpha + \cdots + 
        (a_{n-2} - a_{n-1} c_{n-1}) \alpha^{n-1},
    \end{align*}
    where the last equality follows from equation (20.1).
    We see that $\alpha(a_0 + a_1 \alpha + \cdots + a_{n-1}) \in M$, so 
    $\alpha M \subseteq M$. 

    (2) $\Rightarrow$ (3). Suppose there exists a nonzero finitely generated 
    $\Z$-submodule $M$ of $\C$ such that $\alpha M \subseteq M$. Then we can 
    pick $\lambda_1, \dots, \lambda_p \in \C \setminus \{0\}$ such that 
    $M = \Z\lambda_1 + \cdots + \Z\lambda_p$. Since $\alpha M \subseteq M$,
    this implies that $\alpha \lambda_i \in M$ for each $i = 1, \dots, p$. 

    (3) $\Rightarrow$ (1). Suppose there exists $p \geq 1$ and $\lambda_1, 
    \dots, \lambda_p \in \C \setminus \{0\}$ such that 
    $\alpha\lambda_i \in \Z\lambda_i + \cdots + \Z\lambda_p$ for all $i = 
    1, \dots, p$. Then for $j = 1, \dots, p$, we can pick $a_{j1}, \dots, 
    a_{jp} \in \Z$ such that 
    \[ \alpha\lambda_i = a_{j1} \lambda_1 + \cdots + a_{jp} \lambda_p. \] 
    We can write this as a matrix equation to get 
    \[ \alpha \begin{pmatrix}
        \lambda_1 \\ \lambda_2 \\ \vdots \\ \lambda_p 
    \end{pmatrix} = \begin{pmatrix}
        a_{11} & a_{12} & \cdots & a_{1p} \\ 
        a_{21} & a_{22} & \cdots & a_{2p} \\
        \vdots & \vdots & \ddots & \vdots \\ 
        a_{p1} & a_{p2} & \cdots & a_{pp}
    \end{pmatrix} \begin{pmatrix}
        \lambda_1 \\ \lambda_2 \\ \vdots \\ \lambda_p 
    \end{pmatrix}. \] 
    Set $v = (\lambda_1, \dots, \lambda_p)^T$, and notice that $v \neq 0$ 
    since $\lambda_1, \dots, \lambda_p \in \C \setminus \{0\}$. 
    Moreover, set $A$ to be the matrix above, and observe that $A \in M_p(\Z)$. 
    Now, we have $\alpha v = Av$, so $\alpha$ is an eigenvalue of $A$. Hence, 
    $\alpha$ is a root of the characteristic polynomial of $A$. That is, for 
    \[ P_A(x) = \det(xI - A) = \det \begin{pmatrix}
        x - a_{11} & \cdots & -a_{1p} \\ 
        \vdots & \ddots & \vdots \\ 
        -a_{p1} & \cdots & x - a_{pp}
    \end{pmatrix}, \] 
    we have $P_A(\alpha) = 0$. Note that $A \in M_p(\Z)$, so $P_A(x)$ is a 
    monic polynomial in $\Z[x]$. It follows that $\alpha \in \A$. \qedhere 
\end{pf}

\begin{theo}{}
    The algebraic integers $\A$ form a subring of $\C$. 
\end{theo}
\begin{pf}
    It suffices to show that $\A$ is closed under addition and multiplication, 
    and that $-1 \in \A$; this last condition assures that we have additive inverses. 

    We have already seen that $\Z \subseteq \A$, so $-1 \in \A$. Let 
    $\alpha, \beta \in \A$. We assume that $\alpha$ and $\beta$ are both 
    nonzero, as the result is trivial otherwise. We want to show that 
    $\alpha + \beta$ and $\alpha\beta$ are in $\A$. By definition, we have 
    \begin{align*}
        \alpha^n + c_{n-1} \alpha^{n-1} + \cdots + c_1 \alpha + c_0 = 0, \\ 
        \beta^m + d_{m-1} \beta^{m-1} + \cdots + d_1 \beta + d_0 = 0,
    \end{align*}
    for some $c_0, \dots, c_{n-1}, d_0, \dots, d_{n-1} \in \Z$. Let 
    \[ M = \sum_{i=0}^{n-1} \sum_{j=0}^{m-1} \Z\alpha^i\beta^j, \] 
    which is a finitely generated $\Z$-submodule of $\C$, and $M \neq (0)$ since 
    $1 \in M$. Now, we claim that $\alpha M \subseteq M$ and $\beta M \subseteq M$. 
    It suffices to show that $\beta (\alpha^i \beta^j)$ and $\alpha (\alpha^i 
    \beta^j)$ are in $M$ for all $0 \leq i \leq n-1$ and $0 \leq j \leq m-1$. 

    We will consider $\alpha(\alpha^i \beta^j)$. When $0 \leq i \leq n-2$, we have 
    $\alpha (\alpha^i \beta^j) = \alpha^{i+1} \beta^j \in M$, since 
    $0 \leq i+1 \leq n-1$. Otherwise, we have $i = n-1$, in which case 
    \[ \alpha(\alpha^{n-1}\beta^j) = \alpha^n \beta^j = 
    (-c_{n-1} \alpha^{n-1} - \cdots - c_1 \alpha- c_0) \beta^j
    \in M. \] 
    This implies that $\alpha M \subseteq M$, and a similar argument shows that 
    $\beta M \subseteq M$. Now, observe that 
    \[ (\alpha + \beta)M \subseteq \alpha M + \beta M \subseteq M + M = M, \] 
    and similarly, we know from our work above that 
    \[ (\alpha \beta) M = \alpha(\beta M) \subseteq \alpha M \subseteq M. \] 
    It follows from Theorem 20.4 that $\alpha + \beta$ and $\alpha\beta$ are in $\A$.
\end{pf}

\begin{remark}{}
    Note that $\A$ is not a field, since $2 \in \A$ but $1/2 \notin \A$. 
\end{remark}

Now, we'll prove an interesting corollary of the fact that $\A$ is a subring of $\C$.

\begin{cor}{}
    If $\alpha \in \Q$ and $\tan(\pi\alpha) \in \Q$, then $\tan(\pi\alpha) 
    \in \{0, 1, -1\}$. 
\end{cor}
\begin{pf}
    Recall that $\tan^2(x) + 1 = 1/\cos^2(x)$. Therefore, if $\tan(\pi\alpha) 
    \in \Q$, then we have $\tan^2(\pi\alpha) + 1 = 1/\cos^2(\pi\alpha) \in \Q$ as 
    well. This implies that $4\cos^2(\pi\alpha) \in \Q$. Now, notice that 
    \[ 4\cos^2(\pi\alpha) = (2 \cos(\pi\alpha))^2 = (e^{i\pi\alpha} + 
    e^{-i\pi\alpha})^2. \] 
    In particular, since $e^{i\pi\alpha}$ and $e^{-i\pi\alpha}$ are both 
    algebraic numbers (as we saw in Example 20.2), this shows that 
    $4\cos^2(\pi\alpha) \in \Q \cap \A = \Z$. Now, we know that 
    $4\cos^2(\pi\alpha) \in \{0, 1, 2, 3, 4\}$. Taking the inverses and 
    multiplying by $4$, we find that $1/\cos^2(\pi\alpha) \in \{4, 2, 4/3, 1\}$. 
    Finally, we have 
    \[ \tan^2(\pi\alpha) = \frac{1}{\cos^2(\pi\alpha)} - 1 \in 
    \left\{3, 1, \frac13, 0\right\}. \]  
    But we assumed that $\tan(\pi\alpha) \in \Q$, so the only possibilities are 
    $\tan(\pi\alpha) \in \{0, 1, -1\}$, as desired. 
\end{pf}
\section{Burnside's theorem (11/03/2021)} 
The fact that the degrees of the irreducible characters divide the order of 
the group has some very powerful consequences. The Feit-Thompson theorem 
states that every group of odd order is solvable. We'll focus on Burnside's 
theorem, which gives the weaker statement that every group of order 
$p^a q^b$ is solvable, where $p$ and $q$ are distinct primes and $a, b \in \Z^+$. 
The proofs of these theorems are very difficult (if not impossible) without 
using character theory, and we'll use it to make our lives easier. 
First, let's recall what it means for a group to be solvable. 

\begin{defn}{}
    A group $G$ is {\bf solvable} if there exists a tower 
    \[ G = G_0 \supseteq G_1 \supseteq G_2 \supseteq \cdots \supseteq G_m = \{1\} \] 
    such that $G_{i+1} \trianglelefteq G_i$ and $G_i/G_{i+1}$ is abelian for 
    all $0 \leq i \leq m-1$. 
\end{defn}

Towards the proof of Burnside's theorem, we'll first prove some intermediate results. 

\begin{lemma}{}
    Let $\chi$ be an irreducible character of degree $n \geq 2$. If $\chi(g)/n 
    \in \A$ for some $g \in G$, then $\chi(g) = 0$. 
\end{lemma}

We leave the proof of this for later. For the moment, we 
will use this lemma to prove the following theorem. 

\begin{theo}{}
    Let $G$ be a non-abelian finite group. Suppose that $G$ has a conjugacy 
    class of size $p^a$ where $p$ is a prime and $a \in \Z^+$. Then $G$ is 
    not simple. 
\end{theo}
\begin{pf}
    Suppose by way of contradiction that there exists a non-abelian finite 
    simple group $G$ with an element $g \in G$ such that the conjugacy class 
    ${\cal C} = {\cal C}(g)$ containing $g$ has size $p^a$, where $p$ is prime and 
    $a \in \Z^+$. Since $G$ is non-abelian and simple, we have $G' = G$. 
    Indeed, note that $G' \trianglelefteq G$. By simplicity, we either 
    have $G' = G$ or $G' = \{1\}$. But $G$ is non-abelian, so we must be in the first 
    case. This means that $G$ has one inequivalent irreducible representation 
    of degree $1$ since $|G/G'| = |G/G| = 1$. The only irreducible character 
    of degree $1$ is the trivial character $\chi$ with $\chi(h) = 1$ for all $h \in G$.

    Let $\chi_1, \dots, \chi_s$ be the irreducible characters of $G$. Without 
    loss of generality, we may assume that $\chi_1$ is the trivial character. 
    Notice that if $n_1, \dots, n_s$ are the degrees of $\chi_1, \dots, \chi_s$ 
    respectively, then $n_1 = 1$ and $n_i > 1$ for $i = 2, \dots, s$. Recall 
    that we have $g \in G$ such that $|{\cal C}| = |{\cal C}(g)| = p^a$, 
    where $p$ is prime and $a \in \Z^+$. This implies that $g \neq 1$, for 
    otherwise its conjugacy class would have size $1$. Now, if $L$ is the 
    left regular representation of $G$, then $\chi_L(g) = 0$. This implies that 
    \[ 0 = \chi_L(g) = \sum_{i=1}^s n_i \chi_i(g) 
    = \chi_1(g) + \sum_{\substack{i\geq 2 \\ p\,\mid\,n_i}} n_i \chi_i(g) 
    + \sum_{\substack{i\geq 2 \\ p\,\nmid\,n_i}} n_i \chi_i(g). \]
    {\sc Claim.} For $i \geq 2$, if $p \nmid n_i$, then $\chi_i(g) = 0$. 

    {\sc Proof of Claim.} We know that $\chi_i(g) \in \A$. We have also showed 
    in Lemma 21.1 that 
    \[ \frac{p^a \chi_i(g)}{n_i} = \frac{|{\cal C}(g)| \chi_i(g)}{n_i} \in \A. \] 
    Since $p \nmid n_i$, we have $\gcd(p^a, n_i) = 1$. Therefore, there 
    exist $c, d \in \Z$ such that $cp^a + dn_i = 1$. We find that 
    \[ \frac{\chi_i(g)}{n_i} = \frac{cp^a + dn_i}{n_i} \chi_i(g) 
    = \frac{cp^a \chi_i(g)}{n_i} + d\chi_i(g) \in \A. \] 
    By Lemma 22.2, we obtain $\chi_i(g) = 0$. \hfill$\blacksquare$ 

    Now, writing $n_i = pn'_i$ for each $i \geq 2$ with $p \mid n_i$, we have 
    \[ 0 = \chi_1(g) + \sum_{\substack{i\geq 2 \\ p\,\mid\,n_i}} n_i \chi_i(g) 
    + \sum_{\substack{i\geq 2 \\ p\,\nmid\,n_i}} n_i \chi_i(g) 
    = \chi_1(g) + p \sum_{\substack{i\geq 2 \\ p\,\mid\,n_i}} n'_i \chi_i(g). \] 
    Let $\alpha$ denote the sum in the above equation, and note that $\alpha \in \A$. 
    Then $1 + p\alpha = 0$, which implies that $\alpha = -1/p$. But this is a 
    contradiction since $\Q \cap \A = \Z$, but $-1/p \notin \Z$. 
\end{pf}

We can now prove Burnside's theorem. First, we'll recall some facts from group theory. 
\begin{enumerate}[(1)]
    \item Let $N \trianglelefteq G$. Then $G$ is solvable if and only if 
          $N$ and $G/N$ are both solvable. 
    \item Let $G$ be a group of order $q^b c$ where $q$ is prime, $b \in \Z^+$, 
          and $q \nmid c$. Then by the first Sylow theorem, $G$ has a subgroup 
          $Q$ of order $q^b$. 
    \item If $G$ is a group of order $q^b$ where $q$ is prime and $b \in \Z^+$, 
          then $G$ has a non-trivial center by the class equation. 
    \item If $g \in G$ and ${\cal C}(g)$ is the conjugacy class of $g$, then 
          $|{\cal C}(g)| = |G|/|C(g)|$, where $C(g) = \{h \in G : hg = gh\}$ 
          is the centralizer of $g$. 
\end{enumerate}

\begin{theo}[Burnside's theorem]{}
    Let $G$ be a group of order $p^a q^b$ where $p$ and $q$ are distinct primes 
    and $a, b \in \Z^+$. Then $G$ is solvable. 
\end{theo}
\begin{pf}
    Suppose the result is not true. Let $G$ be a minimal counterexample; that is, 
    $G$ is a group of order $p^a q^b$ where $p$ and $q$ are distinct primes and 
    $a, b \in \Z^+$, and $G$ has minimal size with respect to this property. 

    {\sc Claim.} $G$ is a non-abelian simple group.  

    {\sc Proof of Claim.} Note that $G$ is non-abelian since if $G$ were 
    abelian, then it would be solvable. Assume that $G$ is not simple. 
    Then there exists a non-trivial proper normal subgroup $N$ of $G$. 
    Then $G/N$ and $N$ are both smaller than $G$ and their sizes divide $p^a q^b$. 
    By minimality, $N$ and $G/N$ are solvable, so $G$ is also solvable by (1), a 
    contradiction. \hfill$\blacksquare$ 

    Pick a subgroup $Q \leq G$ with $|Q| = q^b$, which exists by (2). Moreover, 
    we can choose $z \neq 1$ such that $z \in Z(Q)$, which exists since 
    $Q$ has non-trivial center by (3). Then $Q \subseteq C(z) \subseteq G$, 
    so $|C(z)| = p^c q^b$ where $c \in \{0, \dots, a\}$. By (4), it follows that 
    \[ |{\cal C}(z)| = \frac{|G|}{|C(z)|} = \frac{p^a q^b}{p^c q^b} = p^{a-c}. \] 
    If $c < a$, then $G$ is not simple by Theorem 22.3 since $G$ 
    is non-abelian and $|{\cal C}(z)|$ is a conjugacy class with size which is a 
    prime power, contradicting our claim. So $c = a$, which gives $|{\cal C}(z)| 
    = 1$. But this means that $C(z) = G$. Moreover, $z$ is central, which gives 
    $\{1\} \neq Z(G) \trianglelefteq G$. Since $G$ is simple, we must have 
    $Z(G) = G$. This would mean that $G$ is abelian, once again contradicting 
    our claim. Therefore, no counterexample exists, so the result holds. 
\end{pf}

\section{Galois theory (11/05/2021)}
Let $F$ be a field of characteristic $0$, and let $p(x) = x^n + a_{n-1} x^{n-1} 
+ \cdots + a_1 x + a_0 \in F[x]$ be a monic polynomial with roots $\alpha_1, 
\dots, \alpha_n \in \overline{F}$. Let $K = \{p(\alpha_1, \dots, \alpha_n) : 
p(x_1, \dots, x_n) \in F[x_1, \dots, x_n]\}$. Notice that $K$ is a subring of 
$\overline{F}$, because we may consider the surjective $F$-algebra homomorphism 
\begin{align*}
    \varphi : F[x_1, \dots, x_n] &\mapsto \overline{F}, \\ 
    p(x_1, \dots, x_n) &\mapsto p(\alpha_1, \dots, \alpha_n), 
\end{align*}
whose image is $K$ by definition. Note that $\dim_F K < \infty$ since $K$ 
is spanned by $S = \{\alpha_1^{i_1} \cdots \alpha_n^{i_n} : 0 \leq i_1, \dots, 
i_n \leq n\}$. To see this, let $V$ be the $F$-vector subspace of $K$ spanned 
by $S$. If $V \subsetneq K$, then there exists a polynomial $p(x_1, \dots, x_n) 
\in F[x_1, \dots, x_n]$ such that $p(\alpha_1, \dots, \alpha_n) \notin V$. 
Pick such a $p$ of smallest degree lexicographically; that is, 
$x_1^{i_1} \cdots x_n^{i_n} <_{\textrm{lex}} x_1^{j_1} \cdots x_n^{j_n}$ if for some 
$1 \leq m \leq n$, we have $i_k = j_k$ for all $0 \leq k < m$, and 
$i_m < j_m$. Note that this is a total ordering. 

We claim that all monomials of $p$ have degree less than $n$. Suppose we had 
some $x_1^{i_1} \cdots x_s^{i_s} \cdots x_n^{i_n}$ with $i_s \geq n$. Recall 
that $\alpha_s$ is a root of $p$, so we have 
\[ \alpha_s^n + a_{n-1} \alpha_s^{n-1} + \cdots + a_1 \alpha_s + a_0 = 0. \] 
Rearranging the above equation and multiplying by $\alpha_s^{i_n-n}$ gives 
\[ \alpha_s^{i_n} = -a_{n-1} \alpha_s^{i_n-1} - \cdots - a_0 \alpha_s^{i_n-n}. \] 
From this, it follows that 
\[ \alpha_1^{i_1} \cdots \alpha_s^{i_s} \cdots \alpha_n^{i_n} 
= -a_{n-1} \alpha_1^{i_1} \cdots \alpha_s^{i_n-1} \cdots \alpha_n^{i_n} 
- \cdots - a_0 \alpha_1^{i_1} \cdots \alpha_s^{i_n-n} \cdots \alpha_n^{i_n}. \] 
We can repeat this with all other monomials to reach a contradiction. 

Note that $K$ is a field. Indeed, we have $K \subseteq \overline{F}$, so $K$ is 
an integral domain. Moreover, we showed that $\dim_F K < \infty$. Therefore, 
if $a \in K \setminus \{0\}$, then the map $L_a : K \to K$ given by 
$L_a(x) = ax$ is $K$-linear and onto. Since $L_a$ is onto, there exists $b \in K$ 
such that $L_a(b) = 1$, so $ab = ba = 1$. We call $K$ the splitting field
of $p(x) = x^n + a_{n-1} x^{n-1} + \cdots + a_1 x + a_0 = (x - \alpha_1) 
\cdots (x- \alpha_n)$. By construction, it is the smallest field extension of $F$
containing all the roots of $p(x) \in F[x]$. 

Recall that the Galois group $\Gal(K/F)$ is the set of all $F$-algebra 
automorphisms $\sigma : K \to K$ with $\sigma|_F = \id_F$ together with the 
operation of composition. 

\begin{remark}{}
    $\Gal(K/F)$ embeds into $S_n$. To see why, if $\sigma \in \Gal(K/F)$, then 
    $\sigma$ is uniquely determined by the values $\sigma(\alpha_1), \dots, 
    \sigma(\alpha_n)$. Now, we see that $\sigma$ permutes $\alpha_1, \dots, 
    \alpha_n$ because if $p(\alpha) = 0$ for some $\alpha \in K$, then 
    \[ \alpha^n + a_{n-1} \alpha^{n-1} + \cdots + a_1 \alpha + a_0 = 0. \] 
    Since $\sigma$ is $F$-linear, we obtain 
    \begin{align*}
        p(\sigma(\alpha)) 
        &= \sigma(\alpha)^n + a_{n-1} \sigma(\alpha)^{n-1} + \cdots + a_1 \sigma(\alpha) + a_0 \\
        &= \sigma(\alpha^n + a_{n-1} \alpha^{n-1} + \cdots + a_1 \alpha + a_0) \\
        &= \sigma(0) = 0. 
    \end{align*}
\end{remark}

Now, let $G = \Gal(K/F)$. The fundamental theorem of Galois theory tells us that 
if we are given a subgroup $\{\id\} \subseteq H \subseteq G$, then 
$K^H = \{a \in K : \tau(a) = a \text{ for all } \tau \in H\}$ forms a field, 
and the converse is also true. Moreover, we have $H_1 \subseteq H_2$ if and only if 
$K^{H_1} \supseteq K^{H_2}$; that is, this correspondence is inclusion-reversing. 
In particular, notice that $K^G = \{a \in K : \sigma(a) = a \text{ for all } 
\sigma \in \Gal(K/F)\} = F$. 

\begin{theo}[Kronecker]{}
    Consider the polynomial 
    \[ p(x) = x^n + a_{n-1} x^{n-1} + \cdots + a_1 x + a_0 = 
    (x - \alpha_1) \cdots (x - \alpha_n) \in \Z[x]. \] 
    If $|\alpha_i| \leq 1$ for all $i = 1, \dots, n$, then each $\alpha_i$ is 
    either $0$ or a root of unity. 
\end{theo}
\begin{pf}
    First, note that $\alpha_1, \dots, \alpha_n \in \A$. Let $K$ be the splitting 
    field of $p(x) \in F[x]$, and let $G = \Gal(K/\Q)$. For $j \in \Z^+$, define 
    the polynomial 
    \[ p_j(x) = (x - \alpha_1^j) \cdots (x - \alpha_n^j). \] 
    Note that $p_j(x) \in \Q[x]$. Indeed, if $\sigma \in \Gal(K/\Q)$, then 
    by $K$-linearity, we obtain 
    \[ \sigma(p_j(x)) = \sigma((x - \alpha_1^j) \cdots (x - \alpha_n^j)) 
    = (x - \sigma(\alpha_1^j)) \cdots (x - \sigma(\alpha_n^j)) = p_j(x). \] 
    This holds for all $\sigma \in \Gal(K/\Q)$. The coefficients of $p_j(x)$ 
    are all in $K^G = \Q$ (where this equality holds by the fundamental 
    theorem of Galois theory). It follows that $p_j(x) \in \Z[x]$ since 
    $\alpha_1, \dots, \alpha_n \in \A$, and so the coefficients are in 
    $\A \cap \Q = \Z$. 

    Now, write 
    \[ p_j(x) = x^n + a_{n-1,j} x^{n-1} + \cdots + a_{1,j} x + a_{0,j} \in \Z[x]. \] 
    We claim that $|a_{n,j}| \leq \binom{n}{k}$. Indeed, by expanding out 
    our original definition of $p_j(x)$, we have 
    \[ p_j(x) = x^n - (\alpha_1^j + \cdots + \alpha_n^j) x^{n-1} + \cdots 
    + (-1)^k \left( \sum_{1\leq i_1 < \cdots < i_k \leq n} \alpha_{i_1}^j \cdots 
    \alpha_{i_k}^j \right) x^{n-k} + \cdots. \]
    Then we have 
    \[ a_{n-k,j} = (-1)^k \sum_{1\leq i_1 < \cdots < i_k \leq n} \alpha_{i_1}^j \cdots 
    \alpha_{i_k}^j, \] 
    and since $|\alpha_i| \leq 1$ for all $i = 1, \dots, n$, it follows that 
    $|a_{n-k,j}| \leq \binom{n}{k}$. Now, observe that the set 
    $X = \{x^n + a_{n-1}x^{n-1} + \cdots + a_1x + a_0 \in \Z[x] : 
    |a_i| \leq \binom{n}{i} \text{ for all } i = 0, \dots, n-1\}$ is finite, 
    and $p_j(x) \in X$ for all $j \in \Z^+$. In particular, there exist 
    integers $m_1 < m_2$ such that $p_{m_1}(x) = p_{m_2}(x)$. Notice that 
    this implies that $\{\alpha_1^{m_1}, \dots, \alpha_n^{m_1}\} 
    = \{\alpha_1^{m_2}, \dots, \alpha_n^{m_2}\}$. Then we 
    either have $\alpha_i = 0$ or $\alpha_i^{m_2-m_1} = 1$ for each 
    $i = 1, \dots, n$, completing the proof. 
\end{pf}

\section{Lemma for Burnside's theorem, tensor products (11/08/2021)}
Now that we are equipped with Kronecker's theorem, we can prove the following 
lemma we used towards proving Burnside's theorem. In the interest of time, 
we will prove a slightly weaker version of it. 

\begin{lemma}{}
    Let $G$ be a non-abelian simple group. Let $\chi$ be an irreducible character 
    of degree $n \geq 2$. If $g \in G$ with $g \neq 1$ and $\chi(g)/n \in \A$, 
    then $\chi(g) = 0$. 
\end{lemma}
\begin{pf}
    Note that there exists an integer $d$ such that $g^d = 1$. 
    If $\chi$ is the character of some representation $\rho : G \to \GL_n(\C)$, 
    then 
    \[ \rho(g) \sim \begin{pmatrix}
        \omega_1 & & 0 \\ 
        & \ddots & \\ 
        0 & & \omega_n 
    \end{pmatrix}, \] 
    where $\omega_1, \dots, \omega_n$ are $d$-th roots of unity. Indeed, every 
    matrix over $\C$ is triangularizable. But $g^d = 1$ implies that 
    $\rho(g)^d - I = 0$, so the minimal polynomial of $\rho(g)$ divides 
    $x^d - 1$, and the roots of $x^d - 1$ are distinct. A matrix over 
    an algebraically closed field is diagonalizable if and only if its minimal 
    polynomial has distinct roots, so in particular, $\rho(g)$ is diagonalizable. 
    Now, observe that 
    \[ \frac{\chi(g)}{n} = \frac{\omega_1 + \cdots + \omega_n}{n}. \] 
    By the Cauchy-Schwarz inequality, we see that 
    \[ \left| \frac{\chi(g)}{n} \right| = 
    \left\langle (\omega_1, \dots, \omega_n), 
    \left( \frac1n, \dots, \frac1n \right) \right\rangle 
    \leq \sqrt n \cdot \sqrt{\frac1n} = 1, \] 
    with equality if and only if $\omega_1 = \cdots = \omega_n$. We now 
    consider two cases. 

    If $\omega_1 = \cdots = \omega_n =: \omega$, then 
    \[ \rho(g) \sim \begin{pmatrix}
        \omega & & 0 \\ 
        & \ddots & \\ 
        0 & & \omega
    \end{pmatrix} = \omega I. \] 
    Note that $\omega I$ commutes with everything in $\GL_n(\C)$. Hence, 
    we obtain $\rho(ghg^{-1}h^{-1}) = \rho(g)\rho(h)\rho(g)^{-1}\rho(h)^{-1} 
    = I$ for all $h \in G$. This means that $ghg^{-1}h^{-1} \in \ker\rho$ for 
    all $h \in G$. Note that $\ker\rho \trianglelefteq G$, and since $G$ 
    is simple, we either have $\ker\rho = G$ or $\ker\rho = \{1\}$. The 
    first case is impossible since $\rho$ is irreducible of degree $n \geq 2$. 
    Then $\ker\rho = \{1\}$, so $ghg^{-1}h^{-1} = 1$ for all $h \in G$. 
    This implies that $g \in Z(G) \trianglelefteq G$. But we assumed that 
    $g \neq 1$, so $G$ being simple shows that $Z(G) = G$. This is a 
    contradiction since $G$ is non-abelian. 

    Assume now that $\omega_1, \dots, \omega_n$ are not all equal. In this 
    case, we have 
    \[ \left| \frac{\chi(g)}{n} \right| = \left| \frac{\omega_1 + \cdots + 
    \omega_n}{n} \right| < 1. \] 
    Let $\alpha = \chi(g)/n$, and let $K = \Q(e^{2\pi i/d})$. Notice that 
    $\omega_1, \dots, \omega_n \in K$ as they are all $d$-th roots of unity. 
    A $\Q$-algebra automorphism sends $d$-th roots of unity to other $d$-th 
    roots of unity, since $\omega^d = 1$ implies that $\sigma(\omega)^d 
    = \sigma(\omega^d) = \sigma(1) = 1$. Now, if $\sigma \in \Gal(K/\Q)$, then 
    $\sigma$ is $\Q$-linear, so we get 
    \[ \sigma(\alpha) = \frac{\sigma(\omega_1) + \cdots + \sigma(\omega_n)}{n}. \] 
    Moreover, $\sigma(\omega_1), \dots, \sigma(\omega_n)$ are not all the same 
    since $\sigma$ is an automorphism, and hence injective. Therefore, 
    we have $|\sigma(\alpha)| < 1$ for all $\sigma \in \Gal(K/\Q)$. 

    Assume that $\alpha \in \A$, so $\sigma(\alpha) \in \A$ for all $\sigma 
    \in \Gal(K/\Q)$. Note that the polynomial 
    \[ \prod_{\sigma \in \Gal(K/\Q)} (x - \sigma(\alpha)) \] 
    lives in $\Q[x] \cap \A[x] = \Z[x]$. Note that $\alpha$ is a root of this 
    polynomial with $|\alpha| < 1$. By Kronecker, $\alpha$ is 
    either $0$ or a root of unity, with the latter case being impossible since 
    $|\alpha| \neq 1$. It follows that $\alpha = |\chi(g)|/n$ must be $0$, 
    and hence $\chi(g) = 0$. 
\end{pf}

We now introduce tensor products. Let $R$ be a ring, let $M = M_R$ be a 
right $R$-module, and let $N = {}_R N$ be a left $R$-module. We can form 
the tensor product $M \otimes_R N$ of $M$ and $N$. 

Given an abelian group $A$, we say that a map $f : M \times N \to A$ 
is {\bf bilinear} if 
\begin{enumerate}[(i)]
    \item $f(m_1 + m_2 \cdot a, n) = f(m_1, n) + a \cdot f(m_2, n)$ 
          for all $m_1, m_2 \in M$, $n \in N$, and $a \in \Z$; 
    \item $f(m, n_1 + an_2) = f(m, n_1) + a \cdot f(m, n_2)$ for all 
          $m \in M$, $n_1, n_2 \in N$, and $a \in \Z$; 
    \item $f(m \cdot r, n) = f(m, r \cdot n)$ for all $m \in M$, $n \in N$, and 
          $r \in R$. 
\end{enumerate}
We can think of bilinear maps as an abstraction of multiplication between modules. 

\begin{remark}{}
    Note that we need the left and right pairing of the modules to make this work 
    in general. Suppose that $M$ and $R$ are both right $R$-modules. We would like 
    to impose that $f(m \cdot r, n) = f(m, n \cdot r)$ for all $m \in M$, $n \in N$, and 
    $r \in R$. But this means that for all $r, s \in R$, we have $f(m(rs), n) 
    = f(m, n(rs))$, and also 
    \[ f(m(rs), n) = f(mr, ns) = f(m, n(sr)). \] 
    In particular, we require that $f(m, n(rs - sr)) = 0$ for all $m \in M$, 
    $n \in N$, and $r, s \in R$. This is not an issue for commutative rings, 
    but there are rings $R$ such that $1 \in [R, R]$, in which case the only 
    bilinear map is the zero map. The same issue occurs with two left $R$-modules. 
\end{remark}

We now formally define tensor products. 

\begin{defn}{}
    Let $T$ be the free abelian group on generators $e_{(m,n)}$ where $(m, n) 
    \in M \times N$. That is, we have 
    \[ T = \bigoplus_{(m, n) \in M \times N} \Z e_{(m,n)}. \] 
    Let $U \subseteq T$ be the $\Z$-submodule spanned by the relations 
    \begin{enumerate}[(i)]
        \item $e_{(m_1 + m_2, n)} - e_{(m_1, n)} - e_{(m_2, n)}$; 
        \item $e_{(m, n_1 + n_2)} - e_{(m, n_1)} - e_{(m, n_2)}$; and 
        \item $e_{(mr, n)} = e_{(m, rn)}$. 
    \end{enumerate}
    Then the {\bf tensor product} of $M$ and $N$ is defined to be 
    $M \otimes_R N := T/U$. 
\end{defn}

\end{document}