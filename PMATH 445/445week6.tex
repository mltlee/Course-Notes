\section{October 20, 2021}
Let $G$ be a finite group, and let $k$ be an algebraically closed field with 
$\ch(k) \nmid |G|$. Then 
\[ k[G] \cong \prod_{i=1}^s M_{n_i}(k) \] 
as $k$-algebras, where $s$ is the number of conjugacy classes of $G$, and 
$\sum_{i=1}^s n_i^2 = |G|$. Moreover, $k[G]$ has $s$ pairwise non-isomorphic 
simple modules (or irreducible representations) $V_1, \dots, V_s$, with 
$V_i \cong k^{n_i\times 1}$. 

Moreover, every left $k[G]$-module $V$ decomposes as 
\[ V \cong V_1^{a_1} \oplus \cdots \oplus V_s^{a_s}, \] 
and this decomposition is unique. We also have a representation 
\[ \rho_V : G \to \GL(V), \] 
which gives rise to the character $\chi_V : G \to k$ of $V$, with the formula 
\[ \chi_V = \sum_{i=1}^s a_i \chi_{V_i}. \] 
Notice that $\chi_V(1) = \Tr(\rho_V(1)) = \Tr(I_n) = n = \dim_k(V)$, so 
by taking the module $L = {}_{k[G]} k[G]$, we have 
\[ \chi_L = \sum_{i=1}^s n_i \chi_{V_i}, \] 
and for $g \in G$, we saw in the previous lecture that  
\[ \chi_L(g) = \begin{cases} 
    |G| & \text{if $g = 1$,} \\ 
    0 & \text{if $g \neq 1$.}
\end{cases} \] 
What are the irreducible representations of a finite abelian group $G$? 
Assume that $|G| = n$, and $k$ is an algebraically closed field with $\ch(k) 
\nmid n$. We see that $k[G] \cong k^n$, which implies that there are $n$ 
inequivalent irreducible representations. Now, since $G$ is abelian, we can write 
\[ G \cong C_{d_1} \times C_{d_2} \times \cdots \times C_{d_r}, \] 
where $C_j$ denotes the cyclic group of order $j$. We can think of this as 
\[ G \cong \langle x_1 \mid x_1^{d_1} = 1 \rangle \times \cdots \times 
\langle x_r \mid x_r^{d_r} = 1 \rangle. \] 
We know all simple $k[G]$-modules are one-dimensional as $k$-vector spaces. 
In particular, if $V$ is simple, then 
\[ \rho_V : G \to \GL_1(k) \cong k^* = k \setminus \{0\}. \] 
This means that if $\rho_V(x_i) = \omega_i \in k^*$, then 
\[ 1 = \rho_V(1) = \rho_V(x_i^{d_i}) = \omega_i^{d_i}. \] 
That is, each $\omega_i$ is a $d_i$-th root of unity, so there are 
$d_1\cdots d_r = |G|$ choices for $\rho_V$. On the other hand, we already know 
that there are $d_1 \cdots d_r$ pairwise inequivalent irreducible representations,
so these are in fact all of them. In this case, notice that $\chi_V = \rho_V$. 

\begin{exmp}
    Suppose that $G_1 = C_2 \times C_2 = \langle x \mid x^2 = 1 \rangle 
    \times \langle y \mid y^2 = 1 \rangle$ and $G_2 = C_4 = \langle z \mid 
    z^4 = 1 \rangle$. We compute the character tables of $G_1$ and $G_2$; 
    namely, the characters of the representations. 
    \begin{align*}
        \begin{array}{|c|c|c|c|c|}
            \hline
            G_1    & 1 & x  & y  & xy \\ \hline
            \chi_1 & 1 & 1  & 1  & 1  \\ \hline
            \chi_2 & 1 & 1  & -1 & -1 \\ \hline
            \chi_3 & 1 & -1 & 1  & -1 \\ \hline
            \chi_4 & 1 & -1 & -1 & 1  \\ \hline
        \end{array} & & 
        \begin{array}{|c|c|c|c|c|}
            \hline
            G_2    & 1 & z  & z^2  & z^3 \\ \hline
            \chi_1 & 1 & 1  & 1    & 1  \\ \hline
            \chi_2 & 1 & i  & -1   & -i \\ \hline
            \chi_3 & 1 & -1 & 1    & -1 \\ \hline
            \chi_4 & 1 & -i & -1   & -i  \\ \hline
        \end{array} 
    \end{align*}
\end{exmp}

\begin{exmp}
    Let $G = S_3$ and $k = \C$. We recall that $\C[S_3] \cong \C \times 
    \C \times M_2(\C)$. Then $S_3$ has three pairwise inequivalent irreducible 
    representations, and these have dimensions $1$, $1$, and $2$ 
    respectively. Moreover, recall that for a finite-dimensional 
    $k[G]$-module $V$, its character $\chi_V : G \to k$ is a class function. 
    Therefore, to compute the character table of $S_3$, it suffices to 
    consider the values of the characters on each conjugacy class. 

    First, the identity element is always sent to the dimension of the 
    representation, so we obtain $\chi_1(\id) = \chi_2(\id) = 1$ and 
    $\chi_3(\id) = 2$. Since $(12)^2 = \id$, we have $\chi_i((12))^2 
    = \chi_i(\id) = 1$ for $i \in \{1, 2\}$, yielding 
    $\chi_1((12)) = 1$ and $\chi_2((12)) = -1$. 

    Next, recall from Assignment 3 Question 1 that the one-dimensional 
    irreducible representations come from $G/G'$. More precisely, if 
    $V$ is a simple one-dimensional $k[G]$-module, then we obtain the following 
    commutative diagram, where we note that $k^*$ is an abelian group. 
    \begin{center}
        \begin{tikzcd}
            G \arrow[rr, "\rho_V"] \arrow[dd, "\pi"']  &  & \GL(V) \cong k^* \\
                                                    &  &   \\
            {G/G'} \arrow[rruu, "\tilde\rho_V"', dotted] &  &  
        \end{tikzcd}
    \end{center}
    In our case, we have $S'_3 = A_3$ and $S_3/A_3 = C_2$, which gives us 
    the following diagram. 
    \begin{center}
        \begin{tikzcd}
            G \arrow[rr, "\rho_V"] \arrow[dd, "\pi"']  &  & \GL(V) \cong k^* \\
                                                    &  &   \\
            C_2 \arrow[rruu, "\tilde\rho_V"', dotted] &  &  
        \end{tikzcd}
    \end{center}
    In particular, everything in $A_3$ has to be in the kernel and
    gets sent to $1$, so $\chi_1((123)) = \chi_2((123)) = 1$. 
    The last row is more difficult to find directly. However, if we know 
    everything in the character table except for the last row, then there's an 
    easy way to compute it. Let 
    \[ \chi_L = \chi_1 + \chi_2 + 2\chi_3. \] 
    We recall that $\chi_L(\sigma) = 0$ for all $\sigma \neq \id$; in particular, 
    we have $\chi_L((12)) = \chi_L((123)) = 0$. This gives 
    \[ \chi_1((12)) + \chi_2((12)) + 2\chi_3((12)) = 0, \] 
    and rearranging this, we obtain 
    \[ \chi_3((12)) = -\frac12(\chi_1((12)) + \chi_2((12)) 
    = -\frac12(1 + (-1)) = 0. \] 
    Similarly, we find that $\chi_3((123)) = -1$, so we have now completed the 
    character table of $S_3$. 
    \begin{align*}
        \begin{array}{|c|c|c|c|}
            \hline
            S_3    & [\id] & [(12)] & [(123)] \\ \hline
            \chi_1 & 1     & 1      & 1       \\ \hline
            \chi_2 & 1     & -1     & 1       \\ \hline
            \chi_3 & 2     & 0      & -1      \\ \hline
        \end{array} 
    \end{align*}
\end{exmp}

\begin{remark}
    Notice that for the $\C[S_3]$-module $V = \C^{3\times1}$ given the permutation 
    representation, $\chi_V(\sigma)$ gives the number of fixed points of 
    $\sigma \in S_3$. Then we have $\chi_V(\id) = 3$, $\chi_V((12)) = 1$, and 
    $\chi_V((123)) = 0$, so we find that $\chi_V = \chi_1 + \chi_3$ simply from 
    the character table. 
\end{remark}

\begin{exmp}
    Let $G = Q_8 = \{\pm1, \pm i, \pm j, \pm k\}$, where $ij = k$ and 
    $i^2 = j^2 = k^2 = -1$. Recall that 
    \[ \C[Q_8] \cong \C^4 \times M_2(\C), \] 
    so $Q_8$ has five conjugacy classes; they are $\{1\}$, $\{-1\}$, 
    $\{i, -i\}$, $\{j, -j\}$, and $\{k, -k\}$. Observe that $Q'_8 = \{\pm1\}$, 
    which implies that 
    \[ Q_8/Q'_8 = \langle i, j \mid i^2 = j^2 = 1,\, ij = ji 
    \rangle \cong C_2 \times C_2 . \] 
    Now, the first four rows of the character table are easy to compute, 
    as well as the first column. 
    \begin{align*}
        \begin{array}{|c|c|c|c|c|c|}
            \hline
            Q_8    & [1] & [-1] & [i] & [j] & [k] = [ij] \\ \hline
            \chi_1 & 1   & 1    & 1   & 1   & 1          \\ \hline
            \chi_2 & 1   & 1    & 1   & -1  & -1         \\ \hline
            \chi_3 & 1   & 1    & -1  & 1   & -1         \\ \hline
            \chi_4 & 1   & 1    & -1  & -1  & 1          \\ \hline
            \chi_5 & 2   & *    & *   & *   & *          \\ \hline 
        \end{array} 
    \end{align*}
    To obtain the last row, we apply the same trick as the previous example; 
    we set 
    \[ \chi_L = \chi_1 + \chi_2 + \chi_3 + \chi_4 + 2\chi_5 \] 
    and use the fact that $\chi_L(g) = 0$ when $g \neq 1$. This will finish 
    the character table.
    \begin{align*}
        \begin{array}{|c|c|c|c|c|c|}
            \hline
            Q_8    & [1] & [-1] & [i] & [j] & [k] = [ij] \\ \hline
            \chi_1 & 1   & 1    & 1   & 1   & 1          \\ \hline
            \chi_2 & 1   & 1    & 1   & -1  & -1         \\ \hline
            \chi_3 & 1   & 1    & -1  & 1   & -1         \\ \hline
            \chi_4 & 1   & 1    & -1  & -1  & 1          \\ \hline
            \chi_5 & 2   & -2   & 0   & 0   & 0          \\ \hline 
        \end{array} 
    \end{align*} 
\end{exmp}

We will soon see the interesting fact that if $G$ is a group and 
$\Phi, \Psi : G \to \C$ are maps, then we can define an ``inner product'' by 
\[ \langle \Phi, \Psi \rangle_G = \frac{1}{|G|} \sum_{g \in G} \Phi(g) 
\Psi(g^{-1}). \] 
Note that this is not actually an inner product, as it fails to satisfy 
some properties. However, we see that it is $\C$-bilinear; indeed, for 
$\lambda \in \C$, we have 
\[ \langle \Phi_1 + \lambda\Phi_2, \Psi \rangle_G = 
\frac{1}{|G|} \sum_{g \in G} (\Phi_1(g) + \lambda\Phi_2(g)) \Psi(g^{-1}) 
= \langle \Phi_1, \Psi \rangle_G + \lambda \langle \Phi_2, \Psi \rangle_G, \] 
and the other direction can be checked similarly. Moreover, 
$\langle \cdot, \cdot \rangle_G$ is symmetric, since 
\[ \langle \Phi, \Psi \rangle_G = \frac1{|G|} \sum_{g\in G} 
\Phi(g) \Psi(g^{-1}) = \frac1{|G|} \sum_{x\in G} \Phi(x^{-1}) \Psi(x) 
= \langle \Psi, \Phi \rangle_G, \] 
where in the second equality, we used the substitution $x = g^{-1}$. 

Next time, we'll see that characters are orthonormal with respect to this 
``inner product''. 

\section{October 22, 2021}
For this part of the course, we'll be working over $\C$. Last time, 
for a pair of maps $\Phi, \Psi : G \to \C$, we defined the ``inner product''
\[ \langle \Phi, \Psi \rangle_G = \frac{1}{|G|} \sum_{g \in G} \Phi(g) 
\Psi(g^{-1}), \] 
and saw that this was $\C$-bilinear and symmetric. We now make an important 
remark towards showing that $\langle \cdot, \cdot \rangle_G$ is a 
true inner product when working with characters. 

\begin{remark}
    If $\chi$ is a character, then $\chi(g^{-1}) = \overline{\chi(g)}$. 
\end{remark}
\begin{pf}
    We have $\chi(g) = \Tr(\rho(g))$ where $\rho : G \to \GL_n(\C)$ is a 
    representation. Let $g \in G$. Notice that $\rho(g)$ is triangularizable, 
    so 
    \[ S^{-1}\rho(g)S = \begin{pmatrix}
        \lambda_1 & & * \\ & \ddots & \\ 0 & & \lambda_n 
    \end{pmatrix}, \] 
    where $\lambda_1, \dots, \lambda_n$ are the eigenvalues of $\rho(g)$. 
    Since $G$ is finite, there exists an integer $d \geq 1$ such that $g^d = 1$. 
    Then, we can write 
    \[ \rho(g)^d = \rho(g^d) = \rho(1) = I, \] 
    which implies that 
    \[ (S^{-1}\rho(g)S)^d = (S^{-1}\rho(g)S)(S^{-1}\rho(g)S)\cdots 
    (S^{-1}\rho(g)S) = S^{-1}\rho(g)^d S = S^{-1}IS = I. \] 
    Therefore, we find that 
    \[ \begin{pmatrix}
        \lambda_1^d & & * \\ & \ddots & \\ 0 & & \lambda_n^d 
    \end{pmatrix} = I. \] 
    In particular, $\lambda_1, \dots, \lambda_n$ are $d$-th roots of unity. 
    Since $\rho(g^{-1})$ is the inverse of $\rho(g)$, we have 
    \[ \rho(g^{-1}) = \begin{pmatrix}
        \lambda_1^{-1} & & * \\ & \ddots & \\ 0 & & \lambda_n^{-1} 
    \end{pmatrix}. \] 
    Finally, it follows that 
    \[ \chi(g^{-1}) = \Tr(\rho(g^{-1})) = \lambda_1^{-1} + \cdots + \lambda_n^{-1}
    = \overline{\lambda_1} + \cdots + \overline{\lambda_n} = \overline{\chi(g)}. 
    \qedhere \] 
\end{pf}

Let ${\cal C}_1, \dots, {\cal C}_s$ be the conjugacy classes of $G$, and let 
$g_i$ be a representative of ${\cal C}_i$ for each $i = 1, \dots, s$. 
If $\chi_1$ and $\chi_2$ are characters, then 
\begin{align*}
    \langle \chi_1, \chi_2 \rangle_G 
    &= \frac1{|G|} \sum_{g\in G} \chi_1(g) \chi_2(g^{-1}) \\ 
    &= \frac1{|G|} \sum_{g\in G} \chi_1(g) \overline{chi_2(g)} \\ 
    &= \frac1{|G|} \sum_{i=1}^s \sum_{h\in{\cal C}_i} \chi_1(h) \overline{\chi_2(h)} \\
    &= \frac1{|G|} \sum_{i=1}^s |{\cal C}_i| \chi_1(g_i) \overline{\chi_2(g_i)},
\end{align*}
where the last equality follows since $\chi_1$ and $\chi_2$ are class functions. 

We'll now show that if $V_1, \dots, V_s$ are the inequivalent irreducible 
representations of $G$ with $\chi_i = \chi_{V_i}$, then $\langle \chi_i, 
\chi_j \rangle_G = \delta_{ij}$, where $\delta_{ij}$ denotes the Kronecker delta. 

First, we will consider an averaging trick. Let $G$ be a finite group, and let 
$k$ be an algebraically closed field such that $\ch(k) \nmid |G|$. Let 
$V$ and $W$ be $k[G]$-modules. If $T : V \to W$ is a $k$-linear transformation, 
we can define a map $\hat T : V \to W$ where for $v \in V$, we have 
\[ \hat T(v) = \frac1{|G|} \sum_{g \in G} g \cdot T(g^{-1} \cdot v). \] 
This makes sense because $T(g^{-1} \cdot v) \in W$. 

\begin{thm}
    If $T : V \to W$ is $k$-linear, then $\hat T : V \to W$ as above is 
    $k[G]$-linear. 
\end{thm}
\begin{pf}
    It suffices to show that $\hat T(h \cdot v) = h \cdot \hat T(v)$ for all 
    $h, v \in V$, because every element in $k[G]$ is a linear combination of 
    elements in $G$. By a direct computation, we have 
    \begin{align*}
        \hat T(h \cdot v) 
        &= \frac1{|G|} \sum_{g\in G} g \cdot T(g^{-1} \cdot (h \cdot v)) \\ 
        &= \frac1{|G|} \sum_{g\in G} g \cdot T((g^{-1} \cdot h) \cdot v) \\ 
        &= \frac1{|G|} \sum_{x\in G} h \cdot x \cdot T(x^{-1} \cdot v) 
        \quad \text{(substitute $x = h^{-1}g$)} \\ 
        &= h \cdot \frac1{|G|} \sum_{x\in G} x \cdot T(x^{-1} \cdot v) 
        = h \cdot \hat T(v). \qedhere 
    \end{align*}. 
\end{pf}

We now return to working over $\C$, and look at some consequences of the previous 
result. 

If $V$ and $W$ are non-isomorphic simple $\C[G]$-modules and $T : V \to W$ 
is $\C$-linear, then $\hat T \equiv 0$ since $\Hom_{\C[G]}(V, W) = (0)$. 
On the other hand, recall that 
\[ \Hom_{\C[G]}(V, V) = \End_{\C[G]}(V) = \C \] 
by Schur's lemma. This implies that if $T : V \to V$ is $\C$-linear, then 
$\hat T : V \to V$ is just scalar multiplication by some $\lambda \in \C$. 
That is, for all $v \in V$, we have 
\[ \hat T(v) = \lambda v. \] 
We now determine what $\lambda$ is. Let $L_g : V \to V$ be the map given by 
$L_g(v) = g \cdot v$. Then 
\begin{align*}
    \Tr(\hat T) 
    &= \Tr\left( \frac1{|G|} \sum_{g\in G} L_g \circ T \circ L_g^{-1} \right) \\
    &= \frac1{|G|} \sum_{g\in G} \Tr(L_g \circ T \circ L_g^{-1}) \\ 
    &= \frac1{|G|} \sum_{g\in G} \Tr(T) = \Tr(T). 
\end{align*}
On the other hand, we already know that $\Tr(\hat T) = \lambda \cdot \dim V$, so 
\[ \lambda = \frac{\Tr(T)}{\dim V}. \] 
Let $V$ and $W$ be two simple non-isomorphic $\C[G]$-modules with bases 
$\{v_1, \dots, v_m\}$ and $\{w_1, \dots, w_n\}$ respectively. Then, we obtain 
representations 
\begin{align*}
    \rho_1 &: G \to \GL_m(\C) \cong \GL(V), \\ 
    \rho_2 &: G \to \GL_n(\C) \cong \GL(W). 
\end{align*}
Note that if we want to find the matrix of a linear transformation, we pick 
our basis, and to get the $j$-th column, we apply the transformation 
to the $j$-th element of our basis. In particular, we find that 
\begin{align*}
    \rho_1(g) \cdot v_j &= \sum_{i=1}^m \rho_1(g)_{ij} \cdot v_i, & 
    \rho_2(g) \cdot w_j &= \sum_{i=1}^n \rho_2(g)_{ij} \cdot w_i. 
\end{align*}
Notice that we can view $\rho_1(g) = (\rho_1(g))_{ij}$ and 
$\rho_2(g) = (\rho_2(g))_{ij}$ as functions from $G$ to $\C$ 
by looking at their $(i, j)$-th entries. 

\begin{thm}
    For all $1 \leq i, j \leq m$ and $1 \leq p, q \leq n$, we have 
    $\langle (\rho_1)_{ij}, (\rho_2)_{pq} \rangle_G = 0$. 
\end{thm}
\begin{pf}
    Fix some $1 \leq i \leq m$ and $1 \leq q \leq n$. Define a linear 
    transformation $T : V \to W$ by $T(v_i) = w_q$, and $T(v_k) = 0$
    whenever $k \neq i$. Then $\hat T \equiv 0$ since $V$ and $W$ 
    are non-isomorphic simple modules. For all $1 \leq j \leq m$, we have 
    \[ 0 = \hat T(v_j) = \frac1{|G|} \sum_{g \in G} g \cdot T(g^{-1} \cdot v_j) 
    = \frac1{|G|} \sum_{g\in G} \rho_2(g) \cdot T(\rho_1(g^{-1}) \cdot v_j). \]
    By definition, we have 
    \[ \rho_1(g^{-1}) \cdot v_j = \sum_{k=1}^m \rho_1(g^{-1})_{kj} \cdot v_k, \] 
    which gives us 
    \begin{align*}
        T(\rho_1(g^{-1}) \cdot v_j) 
        &= T \left( \sum_{k=1}^m \rho_1(g^{-1})_{kj} \cdot v_k \right) 
        = \rho_1(g^{-1})_{ij} \cdot T(v_i) 
        = \rho_1(g^{-1})_{ij} \cdot w_q, 
    \end{align*}
    where the second equality comes from throwing everything away except when 
    $k = i$. Finally, we obtain 
    \begin{align*}
        0 &= \frac1{|G|} \sum_{g\in G} \rho_2(g) \cdot T(\rho_1(g) \cdot v_j) \\ 
        &= \frac1{|G|} \sum_{g\in G} \rho_1(g^{-1})_{ij} \cdot \rho_2(g) \cdot w_q \\
        &= \frac1{|G|} \sum_{g\in G} \rho_1(g^{-1})_{ij} 
           \left[ \sum_{k=1}^n \rho_2(g)_{kq} \cdot w_k \right] \\
        &= \frac1{|G|} \sum_{g\in G} \rho_1(g^{-1})_{ij} \cdot \rho_2(g)_{pq} \\ 
        &= \langle (\rho_2)_{pq}, (\rho_1)_{ij} \rangle_G 
        = \langle (\rho_1)_{ij}, (\rho_2)_{pq} \rangle_G.
    \end{align*}
    Since $1 \leq i, j \leq m$ and $1 \leq p, q, \leq n$ were arbitrary, we 
    are done. 
\end{pf}

