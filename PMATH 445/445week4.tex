\section{Class functions, abelianization of a group (09/29/2021)}
By Corollary 9.5, we know that if $k$ is algebraically closed, $G$ 
is a finite group, and $\ch(k) \nmid |G|$, then 
\[ k[G] \cong M_{n_1}(k) \times \cdots \times M_{n_s}(k) \] 
for some $s \geq 1$ and integers $n_1, \dots, n_s \geq 1$. 
This will be our setting for this lecture, and
we'll derive more properties about the choice of $s \geq 1$ and the integers 
$n_1, \dots, n_s \geq 1$. 

\begin{theo}{}
    We have $|G| = n_1^2 + \cdots + n_s^2$. 
\end{theo}
\begin{pf}
    The isomorphism in Corollary 9.5 is a $k$-algebra isomorphism, so we 
    find that 
    \begin{align*}
        |G| &= \dim_k k[G] \\
        &= \dim_k M_{n_1}(k) \times \cdots \times M_{n_s}(k) \\
        &= \dim_k M_{n_1}(k) + \cdots + \dim_k M_{n_s}(k) \\
        &= n_1^2 + \cdots + n_s^2. \qedhere 
    \end{align*}
\end{pf}

Next, we'll work towards showing that $s$ is the number of conjugacy classes of $G$.

\begin{remark}{}
    If $R$ is a $k$-algebra, then $Z(R)$ is also a $k$-algebra. Indeed, we have
    an embedding $k \hookrightarrow Z(R) \subseteq R$ which sends $1_k$ 
    to $1_R$, and this also gives us an embedding $k \hookrightarrow 
    Z(Z(R)) = Z(R)$. 
\end{remark}

\begin{remark}{}
    If $T_1, \dots, T_s$ are rings, then 
    \[ Z(T_1 \times \cdots \times T_s) = Z(T_1) \times \cdots \times Z(T_s). \]
\end{remark}

\begin{prop}{}
    We have $Z(M_n(k)) = kI_n = \{\lambda I_n : \lambda \in k\}$. 
\end{prop}
\begin{pf}
    We'll give two proofs: an elementary one, and a high level one. 

    For the elementary proof, suppose that $(a_{ij}) \in Z(M_n(k))$. 
    Then observe that 
    \begin{align*} 
         \begin{pmatrix} 
            a_{11} & 0 & \cdots & 0 \\ 
            a_{21} & 0 & \cdots & 0 \\ 
            \vdots & \vdots & \ddots & \vdots \\ 
            a_{n1} & 0 & \cdots & 0 
        \end{pmatrix} &= \begin{pmatrix}
            a_{11} & \cdots & a_{1n} \\ 
            \vdots & \ddots & \vdots \\
            a_{n1} & \cdots & a_{nn} 
        \end{pmatrix} \begin{pmatrix} 
            1 & 0 & \cdots & 0 \\ 
            0 & 0 & \cdots & 0 \\
            \vdots & \vdots & \ddots & \vdots \\
            0 & 0 & \cdots & 0
        \end{pmatrix} \\ &= \begin{pmatrix} 
            1 & 0 & \cdots & 0 \\ 
            0 & 0 & \cdots & 0 \\
            \vdots & \vdots & \ddots & \vdots \\
            0 & 0 & \cdots & 0
        \end{pmatrix} \begin{pmatrix}
            a_{11} & \cdots & a_{1n} \\ 
            \vdots & \ddots & \vdots \\
            a_{n1} & \cdots & a_{nn} 
        \end{pmatrix} \\ &= \begin{pmatrix}
            a_{11} & a_{12} & \cdots & a_{1n} \\
            0 & 0 & \cdots & 0 \\ 
            \vdots & \vdots & \ddots & \vdots \\
            0 & 0 & \cdots & 0
        \end{pmatrix}. 
    \end{align*}
    In particular, we have $a_{21} = \cdots = a_{n1} = 0$ and 
    $a_{12} = \cdots = a_{1n} = 0$, which shows that 
    \[ A = \left(\begin{array}{@{}c|ccc@{}}
        a_{11} & 0 & \cdots & 0 \\\hline
        0 \\
        \vdots & & A' & \\
        0 
      \end{array}\right) \]
    for some smaller matrix $A'$. The argument follows inductively. 

    For a high level proof (where $k$ is algebraically closed), take 
    $R = M_n(k)$ and consider the simple left $R$-module
    \[ M = k^{n\times 1} = \left\{ \begin{pmatrix} \lambda_1 \\ \vdots \\ 
        \lambda_n \end{pmatrix} : \lambda_1, \dots, \lambda_n \in k \right\}. \] 
    We have shown that $\Delta = \End_R(M) \cong k$
    by identifying each $\lambda \in k$ with the map 
    \begin{align*}
        \Phi_\lambda : M &\to M \\ v &\mapsto \lambda \cdot v. 
    \end{align*} 
    Now, if $A \in Z(M_n(k))$, then the map 
    \begin{align*} f : M &\to M \\ v &\mapsto Av \end{align*}
    is $R$-linear; indeed, for $B \in M_n(k) = R$ and $v_1, v_2 \in M$, we have 
    \begin{align*} 
        f(Bv_1 + v_2) &= A(Bv_1 + v_2) \\ 
        &= ABv_1 + Av_2 \\
        &= BAv_1 + Av_2 \\ 
        &= Bf(v_1) + f(v_2).
    \end{align*}
    In particular, we see that $f \in \Delta$, so $f = \Phi_\gamma$ for some 
    $\gamma \in k$. Then $f(v) = Av = \gamma v$ for all $v \in M$, which 
    implies that $A = \gamma I$. 
\end{pf}

Combining Proposition 10.4 with Remark 10.3, we see that 
\begin{align*}
    Z(M_{n_1}(k) \times M_{n_s}(k)) 
    &= Z(M_{n_1}(k)) \times \cdots \times Z(M_{n_s}(k)) \\
    &= \{(\lambda_1 I_{n_1}, \dots, \lambda_s I_{n_s}) : 
    (\lambda_1, \dots, \lambda_s) \in k^s\}. 
\end{align*}
In particular, we have 
$\dim_k Z(k[G]) = \dim_k Z(M_{n_1}(k) \times \cdots \times M_{n_s}(k)) = s$.

\begin{defn}{}
    We say that a function $\alpha : G \to k$ is a {\bf class function} if 
    $\alpha$ is constant when restricted to each conjugacy class of $G$. 
\end{defn} 

\begin{lemma}{}
    Let $\alpha : G \to k$ be a function. Then $z := \sum_{g \in G} 
    \alpha(g) g$ is central in $k[G]$ if and only if $\alpha$ is a 
    class function. 
\end{lemma}
\begin{pf}
    Note that $z \in \sum_{g \in G} \alpha(g)g$ is in $Z(k[G])$ if and only if 
    $xz = zx$ for all $x \in G$; the backwards direction here is because $G$ 
    forms a basis for $k[G]$. This occurs if and only if 
    $z = x^{-1}zx$ for all $x \in G$ by rearranging. Now, this is 
    equivalent to 
    \[ \sum_{g \in G} \alpha(g)g = x^{-1} \left( \sum_{g \in G}
    \alpha(g)g \right) x = \sum_{g \in G} \alpha(g)x^{-1}gx
    = \sum_{h \in G} \alpha(xhx^{-1}) h \] 
    holding for all $x \in G$, where the last equality follows from 
    making the substitution $h = x^{-1}gx$. This is true if and only if 
    the coefficient of the left-hand side is the same as the 
    coefficient of the right-hand side. That is, $\alpha(h) = 
    \alpha(xhx^{-1})$ for all $h \in G$ and $x \in G$, and this is exactly 
    the definition of a class function. 
\end{pf}

Let $G$ be a finite group, and let ${\cal C}_1, 
\dots, {\cal C}_s$ be the conjugacy classes of $G$. For $i = 1, \dots, s$, 
observe that 
\[ \alpha(g) = \begin{cases} 1 & \text{if } g \in {\cal C}_i \\ 
    0 & \text{if } g \notin {\cal C}_i \end{cases} \]
is a class function. Then the elements 
\[ z_i = \sum_{g \in G} \alpha(g) g = \sum_{g \in {\cal C}_i} g \] 
for $i = 1, \dots, s$ are central by Lemma 10.6. 

\begin{prop}{}
    Let $G$ be a finite group with conjugacy classes ${\cal C}_1, 
    \dots, {\cal C}_s$. Then the elements 
    \[ z_i := \sum_{g \in {\cal C}_i} g \] 
    for $i = 1, \dots, s$ form a basis for $Z(k[G])$. 
\end{prop} 
\begin{pf}
    We have already seen that the $z_i$ are central. To show 
    linear independence, suppose that 
    \[ c_1 z_1 + \cdots + c_s z_s = 0, \] 
    where we take $0$ to mean $\sum_{g \in G} 0 \cdot g$ in $k[G]$. 
    If $g \in {\cal C}_i$, then the coefficient of $g$ on the left-hand 
    side is $c_i$. But the coefficient on the right-hand side is 
    always $0$, so $c_1 = \cdots = c_s = 0$. To see that 
    $\{z_1, \dots, z_s\}$ spans $Z(k[G])$, recall that 
    $z \in \sum_{g \in G} \alpha(g)g \in Z(k[G])$ if and only if 
    $\alpha$ is a class function. Let $\beta_i$ be the unique value 
    of $\alpha$ on ${\cal C}_i$. Then we see that 
    \[ z = \sum_{g \in G} \alpha(g)g = \sum_{i=1}^s \sum_{g \in {\cal C}_i} 
    \alpha(g)g = \sum_{i=1}^s \beta_i \sum_{g \in {\cal C}_i} g 
    = \sum_{i=1}^s \beta_i z_i, \] 
    so we can write $z$ as a linear combination of the $z_i$, completing the 
    proof. 
\end{pf}

\begin{theo}{}
    In the setting of Corollary 9.5, $s$ is the number of conjugacy 
    classes of $G$. 
\end{theo} 
\begin{pf}
    We previously showed that $\dim_k Z(k[G]) = \dim_k Z(M_{n_1}(k) \times 
    \cdots \times M_{n_s}(k)) = s$. Proposition 10.7 tells us that 
    $\dim_k Z(k[G])$ is the number of conjugacy classes of $G$. 
\end{pf} 

For the remainder of the lecture, we recall the abelianization of a group $G$.
We denote by $G'$ the commutator (or derived) subgroup of $G$, which is the 
smallest subgroup of $G$ which contains all elements of the form 
$ghg^{-1}h^{-1}$ with $g, h \in G$ (we call these elements commutators). 

Note that $G/G'$ is abelian. Indeed, take $g, h \in G$ so that 
$gG', hG' \in G/G'$. Observe that 
\[ (h^{-1})(g^{-1})(h^{-1})^{-1}(g^{-1})^{-1} = 
    h^{-1}g^{-1}hg \in G', \] 
so we have 
\[ (gG')(hG') = ghG' = gh(h^{-1}g^{-1}hg)G' = hgG' = (hG')(gG'). \]
For this reason, we call $G/G'$ the {\bf abelianization} of $G$. 

\begin{exercise}{}
    Show that $G/G'$ has the universal property that if $A$ is abelian 
    group, $\phi : G \to A$ is a group homomorphism, and $\pi 
    : G \to G/G'$ is the canonical quotient map, then there is a 
    unique group homomorphism $\Phi : G/G' \to A$ such that 
    $\Phi \circ \pi = \phi$. 

    \begin{center}
        \begin{tikzcd}
            G \arrow[rr, "\phi"] \arrow[dd, "\pi"'] &  & A \\
                                                    &  &   \\
            G/G' \arrow[rruu, "\Phi"', dotted]      &  &  
        \end{tikzcd}
    \end{center}
\end{exercise}

\section{Applications of the big theorem, representations (10/01/2021)}
There is a similar notion of abelianization for rings. Let $R$ 
be a ring. The {\bf commutator ideal} $[R, R]$ is the smallest ideal 
containing all elements of the form $ab - ba$ where $a, b \in R$. 
We can also think of $[R, R]$ as the intersection of all two-sided ideals 
$J$ such that $ab - ba \in J$ for all $a, b \in R$. 

\begin{remark}{}
    \begin{enumerate}[(1)]
        \item Note that $R/[R, R]$ is commutative because for all $a, b \in R$, 
        we have 
        \begin{align*}
            (a + [R, R])(b + [R, R]) 
            &= ab + [R, R] \\ 
            &= ab + (ba - ab) + [R, R] \\
            &= ba + [R, R] \\
            &= (b + [R, R])(a + [R, R]).
        \end{align*}
        \item We also have a universal property here; if $R$ is a ring, 
        $C$ is a commutative ring, $\phi : R \to C$ is a ring homomorphism, 
        and $\pi : R \to R/[R, R]$ is the canonical quotient map, then 
        there exists a unique ring homomorphism $\Phi : R/[R, R] \to C$ 
        such that $\Phi \circ \pi = \phi$. 
        \begin{center}
            \begin{tikzcd}
                R \arrow[rr, "\phi"] \arrow[dd, "\pi"']  &  & C \\
                                                        &  &   \\
                {R/[R, R]} \arrow[rruu, "\Phi"', dotted] &  &  
            \end{tikzcd}
        \end{center}
    \end{enumerate}
\end{remark}

In Assignment 3, we will prove the following results. 
\begin{enumerate}[(1)]
    \item If $k$ is a field and $G$ is a group, then there is a 
    $k$-algebra homomorphism 
    \[ k[G]/[k[G], k[G]] \cong k[G/G']. \] 
    Said another way, the abelianization of the group algebra is isomorphic 
    to the group algebra of the abelianization. 
    \item If $R = M_{n_1}(k) \times \cdots \times M_{n_s}(k)$, then 
    \[ R/[R, R] \cong k^t, \] 
    where $t = \#\{i : n_i = 1\}$. 
\end{enumerate}

As an immediate consequence of these results, we have the following 
corollary. 

\begin{cor}{}
    If $k[G] \cong M_{n_1}(k) \times \cdots \times M_{n_s}(k)$, then 
    $|G/G'| = \#\{i : n_i = 1\}$. 
\end{cor}
\begin{pf}
    Since $k[G]$ and $M_{n_1}(k) \times \cdots \times M_{n_s}(k)$ 
    are isomorphic, so are their abelianizations. It follows that there 
    is a $k$-algebra homomorphism 
    \[ k[G/G'] \cong k^t \] 
    where $t = \#\{i : n_i = 1\}$. By looking at the dimensions, we see that 
    \[ |G/G'| = \dim_k k[G/G'] = \dim_k k^t = t = \#\{i : n_i = 1\}. 
    \qedhere \] 
\end{pf}

\begin{exmp}{}
    Let's look at $\C[S_4]$. Recall that the conjugacy classes of $S_n$ 
    are just sets of permutations with the same disjoint cycle structure. 
    For $n = 4$, the conjugacy classes are 
    \[ [(1)(2)(3)(4)], [(12)(3)(4)], [(12)(34)], [(123)(4)], [(1234)]. \] 
    Notice that we can think of these as
    \[ 1+1+1+1=4,\; 2+1+1=4,\; 2+2=4,\; 3+1=4,\; 4=4. \] 
    In particular, there is a bijective correspondence between the conjugacy 
    classes of $S_n$ and the partitions of $n$. 

    We can look at the sizes of the conjugacy classes. We have 
    \begin{align*} 
        \#[(1)(2)(3)(4)] &= 1, \\ 
        \#[(1)(2)(34)] &= 6, \\
        \#[(1)(234)] &= 8, \\
        \#[(1234)] &= 6, \\
        \#[(12)(34)] &= 3. 
    \end{align*}
    Since there are $5$ conjugacy classes of $S_4$, we have 
    \[ \C[S_4] \cong M_{n_1}(\C) \times \cdots \times M_{n_5}(\C) \] 
    for some integers $n_1, \dots, n_5 \geq 1$. 

    We claim that $S'_4 = A_4$. First, note that $S'_4 \subseteq A_4$ 
    since $S'_4/A_4$ is abelian. Now, notice that 
    \[ (123) = (12)(13)(12)(13) = (12)(13)(12)^{-1}(13)^{-1} \in S'_4. \] 
    Therefore, every $3$-cycle is in $S'_4$ as normal subgroups 
    are unions of conjugacy classes. Hence, we see that 
    \[ |S'_4| \geq 9 = \#[(1)(234)] + \#[(1)(2)(3)(4)]. \] 
    Since $|S'_4| \mid |S_4| = 24$ by Lagrange's theorem, we obtain 
    $|S'_4| = 12$, and so $S'_4 = A_4$. From this, we have  
    \[ \#\{i : n_i = 1\} = |S_4/S'_4| = |S_4/A_4| = 2. \] 
    Now, we have $n_3, n_4, n_5 \geq 2$ with $n_3^2 + n_4^2 + n_5^2 = 22$. 
    The only solution to this is $2^2 + 3^2 + 3^2 = 22$, so we conclude that 
    \[ \C[S_4] \cong \C \times \C \times M_2(\C) \times M_3(\C) \times M_3(\C). \] 
\end{exmp}

\begin{exmp}{}
    If $A$ is abelian with $|A| = n$, then $|A/A'| = n$ so that 
    $\#\{i : n_i = 1\} = n$. It follows that 
    \[ \C[A] \cong \C^n. \] 
    We can give an alternate argument: if we had $n_i \geq 2$ for some 
    $i = 1, \dots, s$, then the product of matrix rings is no longer 
    commutative, which is a contradiction since $\C[A]$ is commutative. 
\end{exmp}

\begin{exmp}{}
    Let $D_5$ be the dihedral group of order $10$. We can write 
    \[ D_5 = \langle x, y : x^2 = 1,\, y^5 = 1,\, xyx = y^{-1} \rangle. \] 
    We have an isomorphism 
    \[ \C(D_5) \cong M_{n_1}(\C) \times \cdots \times M_{n_s}(\C) \] 
    for some $s \geq 1$ and integers $n_1, \dots, n_s \geq 1$. 
    Notice that 
    \[ 10 = 1^2 + 3^2 = 1^2 + 1^2 + 2^2 + 2^2 = \underbrace{1^2 + \cdots + 1^2}_{6} \,+\, 2^2 
    = \underbrace{1^2 + \cdots + 1^2}_{10}. \] 
    One can check that $D_5$ has $4$ conjugacy classes. Therefore, only the 
    choice $10 = 1^2 + 1^2 + 2^2 + 2^2$ works, and we must have 
    \[ \C(D_5) \cong \C \times \C \times M_2(\C) \times M_2(\C). \] 
\end{exmp}

Let's now study representations. Let $G$ be a group and 
let $k$ be an algebraically closed field such that $\ch(k) \nmid |G|$. 
A {\bf representation} of $G$ is a group homomorphism 
\[ \rho : G \to \GL_n(k). \]

\begin{remark}{}
    The group homomorphism $\rho : G \to \GL_n(k)$ extends to a $k$-algebra 
    homomorphism 
    \[ \tilde\rho : k[G] \to M_n(k). \] 
    Conversely, if $\phi : k[G] \to M_n(k)$ is a $k$-algebra homomorphism, then
    $\phi$ restricted to $G$ is a map  
    \[ \phi|_G : G \to \GL_n(k). \] 
    This is because $\phi(1) = I$, so for any $g \in G$, we have 
    \[ \phi(g) \cdot \phi(g)^{-1} = \phi(g \cdot g^{-1}) = \phi(1) = I. \] 
    Thus, $\phi(g)$ must be an invertible matrix. Moreover, $\tilde\rho|_G 
    = \rho$. 
\end{remark}

\begin{remark}{}
    If $\phi : k[G] \to M_n(k)$ is a $k$-algebra homomorphism, then 
    \[ V = k^{n\times 1} = \left\{ \begin{pmatrix} a_1 \\ \vdots \\ a_n
    \end{pmatrix} : a_1, \dots, a_n \in k \right\} \] 
    can be given a left $k[G]$-module structure with the rule 
    \[ r \cdot v := \Phi(r) \cdot v \] 
    for $r \in k[G]$ and $v \in V$. 
\end{remark} 

Notice that if $V$ is a $k[G]$-module, then we have a $k$-algebra homomorphism
\[ \Psi : k[G] \to \End_k(V) \] 
where we send each $g \in G$ to the map 
\begin{align*} L_g : V &\to V \\ v &\mapsto g\cdot v \end{align*}
and extend linearly over $k$. In particular, 
\[ \Psi|_G : G \to \GL(V) \cong \GL_n(k), \] 
where $\GL(V)$ denotes the invertible linear transformations over $M$. 
For this reason, if $V$ is a $k[G]$-module, we will call $V$ a 
{\bf representation} of $G$. 

\section{Direct sums, decompositions of modules (10/04/2021)}
We'll first make some definitions. 

\begin{defn}{} 
    Let $V$ and $W$ be $k[G]$-modules; that is, they are 
    representations of $G$. 
    \begin{enumerate}[(1)]
        \item If $V$ is simple, we say that $V$ is a {\bf irreducible representation}.
        \item If $W \subseteq V$ is a $k[G]$-submodule of $V$, we say that $W$ is a 
        {\bf subrepresentation} of $G$. 
        \item We call two representations $V$ and $W$ {\bf equivalent} if 
        $V \cong W$ as left $k[G]$-modules. 
        \item We say that $V$ is {\bf decomposable} if $V = V_1 \oplus V_2$, 
        where $V_1$ and $V_2$ are proper submodules of $V$. 
    \end{enumerate}
\end{defn}

Let's define what a direct sum is. First, let $M$ be an $R$-module. 
If $M_1$ and $M_2$ are submodules of $M$, then 
\[ M_1 + M_2 = \{m_1 + m_2 : m_1 \in M_1,\, m_2 \in M_2\} \subseteq M. \] 
More generally, if we have a family $(M_\alpha)_\alpha \in X$ of submodules of 
$M$, then 
\[ \sum_{\alpha \in X} M_\alpha = \left\{ \sum_{\alpha \in X} m_\alpha : 
m_\alpha \in M_\alpha, \text{ $m_\alpha = 0$ for all but finitely many 
$\alpha$}\right\} \subseteq M. \] 
Then, we call $\sum_{\alpha \in X} M_\alpha \subseteq M$ 
{\bf direct} when $\sum_{\alpha \in X} m_\alpha = 0$ if and only if 
$m_\alpha = 0$ for all $\alpha \in X$. In the case where 
$\sum_{\alpha \in X} M_\alpha$ is a direct sum, we write 
\[ \bigoplus_{\alpha \in X} M_\alpha := \sum_{\alpha \in X} M_\alpha. \] 

\begin{remark}{}
    Let $M$ be an $R$-modules, and let $M_1$ and $M_2$ be submodules of $M$. 
    Then $M_1 + M_2$ is a direct sum if and only if $M_1 \cap M_2 = (0)$. 
\end{remark} 

\begin{exmp}{}
    Let $V$ be a vector space over a field $k$, and let $v_1, v_2 \in V 
    \setminus \{0\}$. Consider the submodules $V_1 = \{\lambda v_1 : \lambda 
    \in k\}$ and $V_2 = \{\lambda v_2 : \lambda \in k\}$. Then $V_1 + V_2$ is a 
    direct sum if and only if $v_1$ and $v_2$ are linearly independent. 
\end{exmp}

We'll study some module theory for the group algebra $k[G]$. Let 
$k$ be an algebraically closed field such that $\ch(k) \nmid |G|$. 

In the next few lectures, we will work towards proving the following theorem. 

\begin{theo}{}
    Suppose that 
    \[ k[G] \cong M_{n_1}(k) \times \cdots \times M_{n_s}(k), \] 
    where $s$ is the number of conjugacy classes of $G$. 
    \begin{enumerate}[(1)]
        \item Up to isomorphism, $k[G]$ has $s$ pairwise non-isomorphic 
        simple left modules $V_1, \dots, V_s$ (that is, $s$ pairwise 
        inequivalent irreducible representations). 
        \item Every $k[G]$-module decomposes as a direct sum of copies of 
        $V_1, \dots, V_s$. 
        \item {\bf Uniqueness:} If $V_1^{a_1} \oplus V_2^{a_2} \oplus 
        \cdots \oplus V_s^{a_s} \cong V_1^{b_1} \oplus V_2^{b_2} \oplus 
        \cdots \oplus V_s^{b_s}$, then $a_i = b_i$ for all 
        $i = 1, \dots, s$, where by $V^a$, we mean $V^a := 
        V \oplus \cdots \oplus V$ ($a$ times).
    \end{enumerate}
\end{theo}

What is the significance of this theorem? First, let's ask what the $V_i$'s are. 
We know that 
\[ k[G] \cong M_{n_1}(k) \times \cdots \times M_{n_s}(k), \] 
and for each $i = 1, \dots, s$, we have the projection 
\[ \pi_i : M_{n_1}(k) \times \cdots \times M_{n_s}(k) \to M_{n_i}(k). \] 
Note that $M_{n_i}(k)$ has the module $V_i \cong k^{n_i \times 1}$. Now, if 
$V$ is a representation of $G$ which looks like 
\[ V = V_{i_1} \oplus \cdots \oplus V_{i_q}, \] 
we recall that $k[G]$ acting on $V$ gives us a representation 
\[ \rho : G \to \GL(V) \] 
given by $\rho(g)(v) := g \cdot v$. Now, what's the significance of the 
decomposition of $V$ above? Suppose for simplicity that we had 
$V = W_1 \oplus W_2$, where $W_1$ and $W_2$ are $k[G]$-submodules of $V$. 
Let $\{v_1, \dots, v_d\}$ be a basis for $W_1$, and let 
$\{v_{d+1}, \dots, v_n\}$ be a basis for $W_2$. Then we have 
\[ \rho(g)(v_j) = g \cdot v_j = \begin{cases} 
    a_{1j}v_1 + \cdots + a_{d_j} v_d, & \text{if $j \leq d$,} \\ 
    a_{(d+1)_j} v_{d+1} + \cdots + a_{n_j} v_n, & \text{if $d + 1 \leq j \leq n$.}
\end{cases} \] 
In particular, by setting ${\cal B} = \{v_1, \dots, v_n\}$, we see that 
\[ [\rho(g)]_{\cal B} = \left(\begin{array}{@{}c|c@{}}
    A_1 & 0 \\\hline
    0 & A_2 
  \end{array}\right), \]
where $A_1$ is a $d \times d$ matrix and $A_2$ is an $(n-d) \times (n-d)$ matrix.
Therefore, the importance of the theorem is that if we are given a finite-dimensional 
representation, we can uniquely write it as a block diagonal matrix 
(up to ordering of the blocks), where the blocks are representations coming 
from the irreducibles $V_i \cong k^{n \times 1}$. Therefore, if we understand 
irreducibles, then we understand every representation. 

First, in order to understand $k[G]$, we looked at Artinian rings with no 
nonzero nil ideals. Now that we understand $k[G]$, our main goal is to understand 
representations. Our discussion above shows that it suffices to understand 
the simples $V_i \cong k^{n \times 1}$. We will use character theory to do so, 
which we will get to soon, and we'll understand everything. 

Due to the isomorphism 
\[ k[G] \cong M_{n_1}(k) \times \cdots \times M_{n_s}(k), \] 
we can just think of $k[G]$ as a product of matrix rings. Therefore, it suffices 
to understand the left modules of $M_{n_1}(k) \times \cdots \times 
M_{n_s}(k)$ to know the left modules of $k[G]$. Let $R$ and $S$ be rings 
with an isomorphism $\phi : R \to S$. If we have an $R$-module $M$, 
we obtain an $S$-module by taking 
\[ s \cdot m := \phi^{-1}(s) \cdot m. \] 
Conversely, if $N$ is a left $S$-module, then we can give it an $R$-module 
structure by 
\[ r \cdot n = \phi(r) \cdot n. \] 
That is, the isomorphism $\phi$ is just a way of relabelling these elements. 

\begin{lemma}{}
    Let $R = M_{n_1}(k) \times \cdots \times M_{n_s}(k)$, and let 
    $V_i = k^{n_i \times 1}$ for each $i = 1, \dots, s$. For each 
    $i = 1, \dots, s$ we define 
    \[ (A_1, \dots, A_i, \dots, A_s) \cdot v := A_i v \] 
    where $(A_1, A_2, \dots, A_s) \in R$ and $v \in V_i$. 
    Then $V_i$ is a simple left $R$-module, and if $i \neq j$, then 
    $V_i \ncong V_j$. 
\end{lemma}
\begin{pf}
    If $(A_1, \dots, A_s), (B_1, \dots, B_s) \in R$ and $v \in V_i$, then 
    \begin{align*} 
        (A_1, \dots, A_i, \dots, A_s) \cdot (B_1, \dots, B_i, \dots, B_s) \cdot v 
        &= (A_1B_1, \dots, A_iB_i, \dots, A_sB_s) v \\ 
        &= A_iB_i v \\ 
        &= A_i(B_i v) \\ 
        &= (A_1, \dots, A_s) [(B_1, \dots, B_s) \cdot v] 
    \end{align*} 
    Therefore, $V_i$ is a left $R$-module; the other properties come for free 
    since $V_i$ is already a module over $M_{n_i}(k)$, and hence it is an abelian 
    group. 

    Why is $V_i$ simple as an $R$-module? Notice that it suffices to show that 
    if $v_1 \in V_i \setminus \{0\}$ and $v_2 \in V_i$, then there exists 
    $r \in R$ such that $r \cdot v_1 = v_2$. Indeed, if $V_i$ is simple, then 
    the only submodules are $(0)$ and $V_i$ itself. If we have a nonzero 
    element of $V_i$, then it should generate all of $V_i$. 

    We already know that $V_i$ is a simple $M_{n_i}(k)$-module. Therefore, 
    there exists $A \in M_{n_i}(k)$ such that $Av_1 = v_2$. Let 
    $r = (0, \dots, 0, A, 0, \dots, 0)$, where $A$ is located at the 
    $i$-th coordinate. Then $r \cdot v_1 = v_2$, so $V_i$ is a simple 
    left $R$-module. 

    Next, we show that if $i \neq j$, then $V_i \ncong V_j$ as left $R$-modules. 
    Suppose that we have an $R$-linear isomorphism $f : V_i \to V_j$. 
    If $v \in V_i$, then 
    \[ f((0, \dots, 0, I, 0, \dots, 0) \cdot v) = f(0 \cdot v) = 0 \in V_j, \] 
    where $I$ is in the $j$-th position, and the $i$-th position is one of the 
    zeros. Since $f$ is $R$-linear, then 
    \[ f((0, \dots, 0, I, 0, \dots, 0) \cdot v) = (0, \dots, 0, I, 0, \dots, 0)
    \cdot f(v) = I \cdot f(v) = f(v). \] 
    This means that $f$ is identically zero, contradicting the fact that 
    it is an isomorphism from $V_i$ to $V_j$. 
\end{pf}

\begin{remark}{}
    In fact, we have shown that if $i \neq j$, then 
    \[ \Hom_R(V_i, V_j) = \{0\}. \] 
    When $i = j$, we have $\Hom_R(V_i, V_j) = \End_R(V_i)$. In the case that 
    $k$ is algebraically closed and $R$ is a finite-dimensional $k$-algebra, 
    we have $\End_R(V_i) \cong k$ by Proposition 3.1, since $V_i$ is a simple 
    left $R$-module. 
\end{remark}