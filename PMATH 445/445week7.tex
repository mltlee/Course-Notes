\section{Character tables of $S_4$ and $S_5$ (10/27/2021)}
Recall that we can write $\C[G] \cong M_{n_i}(\C) \times \cdots \times M_{n_s}(\C)$, 
where $s$ is the number of conjugacy classes of $G$. Then, we obtain $s$ 
irreducible characters $\chi_1, \dots, \chi_s$, with 
\[ \chi_i(1) = n_i, \] 
where we call $n_i$ the degree of $\chi_i$ for each $i = 1, \dots, s$. We also saw 
that 
\[ \delta_{ij} = \langle \chi_i, \chi_j \rangle_G = \frac1{|G|} \sum_{g\in G} 
\chi_i(g) \overline{\chi_j(g)}. \] 
For a $\C[G]$-module $V \cong V_1^{a_1} \oplus \cdots \oplus V_s^{a_s}$, we have 
the formula 
\[ \chi_V = a_1 \chi_i + \cdots + a_s \chi_s, \] 
where $\chi_i = \chi_{V_i}$ for all $i = 1, \dots, s$. We saw that this gives 
$a_i = \langle \chi_V, \chi_i \rangle_G$ by orthonormality. In particular, we 
find that 
\[ \langle \chi_V, \chi_V \rangle_G = \sum_{i=1}^s a_i^2 \|\chi_i\|^2 = 
\sum_{i=1}^s a_i^2. \] 
Let's use our results to find the character tables of $S_4$ and $S_5$. 

\begin{exmp}{}
    To find the character table of $S_4$, we first recall that 
    $\C[S_4] \cong \C \times \C \times M_2(\C) \times M_3(\C) \times M_3(\C)$, 
    which we determined in Example 11.3. We also saw that the $5$ conjugacy 
    classes and their sizes were 
    \begin{align*} 
        \#[\id] &= 1, \\ 
        \#[(12)] &= 6, \\
        \#[(123)] &= 8, \\
        \#[(1234)] &= 6, \\
        \#[(12)(34)] &= 3. 
    \end{align*}
    Since $S'_4 = A_4$, we have $S_4/S'_4 = S_4/A_4 \cong C_2$, so we can immediately 
    determine the degree $1$ characters, as well as the first column as we 
    know the degrees of all the characters. 
    \begin{align*}
        \begin{array}{|c|c|c|c|c|c|}
            \hline
            S_4    & \overset{1}{[\id]} & \overset{6}{[(12)]} & \overset{8}{[(123)]} & \overset{6}{[(1234)]} & \overset{3}{[(12)(34)]} \\ \hline
            \chi_1 & 1                  & 1                   & 1                    & 1                     & 1                       \\ \hline
            \chi_2 & 1                  & -1                  & 1                    & -1                    & 1                       \\ \hline
            \chi_3 & 2                  &                     &                      &                       &                         \\ \hline
            \chi_4 & 3                  &                     &                      &                       &                         \\ \hline
            \chi_5 & 3                  &                     &                      &                       &                         \\ \hline
        \end{array} 
    \end{align*}
    To compute the rest of the character table, we'll usually find a nice 
    representation of the group in practice. Consider the permutation representation
    $P : S_4 \to \GL_4(\C)$ given by 
    \[ P(\sigma)e_j = e_{\sigma(j)}. \] 
    Recall that $\chi_P(\sigma)$ is the number of fixed points of $\sigma \in S_4$. 
    For now, let's add $\chi_P$ to the character table. 
    \begin{align*}
        \begin{array}{|c|c|c|c|c|c|}
            \hline
            S_4    & \overset{1}{[\id]} & \overset{6}{[(12)]} & \overset{8}{[(123)]} & \overset{6}{[(1234)]} & \overset{3}{[(12)(34)]} \\ \hline
            \chi_1 & 1                  & 1                   & 1                    & 1                     & 1                       \\ \hline
            \chi_2 & 1                  & -1                  & 1                    & -1                    & 1                       \\ \hline
            \chi_3 & 2                  &                     &                      &                       &                         \\ \hline
            \chi_4 & 3                  &                     &                      &                       &                         \\ \hline
            \chi_5 & 3                  &                     &                      &                       &                         \\ \hline
            \chi_P & 4                  & 2                   & 1                    & 0                     & 0                       \\ \hline 
        \end{array} 
    \end{align*}
    By Exercise 5 in Homework 3, we find that 
    \[ \langle \chi_P, \chi_P \rangle = \frac1{4!} \sum_{\sigma \in S_4} 
    \fix(\sigma)^2 = \frac1{24} (1 \cdot 4^2 + 6 \cdot 2^2 + 8 \cdot 1^2 + 6 
    \cdot 0^2 + 3 \cdot 0^2) = 2, \] 
    where $\fix(\sigma)$ denotes the number of fixed points of $\sigma$. 
    In particular, we see that $\chi_P$ is not an irreducible character, 
    but recalling the formula 
    \[ \langle \chi_V, \chi_V \rangle = \sum_{i=1}^s a_i^2, \] 
    we see that $\chi_P$ can be decomposed as the sum of two irreducible components. 
    Now, looking at $\langle \chi_P, \chi_1 \rangle$, we obtain 
    \[ \langle \chi_P, \chi_1 \rangle = \frac1{24} (4 \cdot 1 \cdot 1 
    + 2 \cdot 1 \cdot 6 + 1 \cdot 1 \cdot 8) = 1. \] 
    This means that $\chi_P = \chi_1 + \chi$, where $\chi$ is some irreducible 
    character. In other words, $\chi_P - \chi_1$ is an irreducible character. 
    Notice that $\chi_P(\id) - \chi_1(\id) = 4 - 1 = 3$, so this character is 
    one of $\chi_4$ or $\chi_5$. Let's say we have $\chi_4 = \chi_P - \chi_1$, since we 
    can just flip the rows if necessary. This allows us to fill out another row in 
    the character table. 
    \begin{align*}
        \begin{array}{|c|c|c|c|c|c|}
            \hline
            S_4    & \overset{1}{[\id]} & \overset{6}{[(12)]} & \overset{8}{[(123)]} & \overset{6}{[(1234)]} & \overset{3}{[(12)(34)]} \\ \hline
            \chi_1 & 1                  & 1                   & 1                    & 1                     & 1                       \\ \hline
            \chi_2 & 1                  & -1                  & 1                    & -1                    & 1                       \\ \hline
            \chi_3 & 2                  &                     &                      &                       &                         \\ \hline
            \chi_4 & 3                  & 1                   & 0                    & -1                    & -1                      \\ \hline
            \chi_5 & 3                  &                     &                      &                       &                         \\ \hline
            \chi_P & 4                  & 2                   & 1                    & 0                     & 0                       \\ \hline 
        \end{array} 
    \end{align*}
    On Question 1 of Homework 4, we will show that if $\chi$ is a degree $1$ 
    character and for each $d \geq 1$, we let $X_d$ be the set of irreducible 
    characters of degree $d$, then multiplication by $\chi$ permutes $X_d$ 
    for all $d \geq 1$. 

    In particular, since $\chi_2$ is a degree $1$ character, we can multiply 
    $\chi_4$ with $\chi_2$ to obtain the other degree $3$ character 
    $\chi_5 = \chi_4 \cdot \chi_2$, as multiplication by $\chi_2$ permutes 
    $X_3 = \{\chi_4, \chi_5\}$. Similarly, if we multiply $\chi_5$ with 
    $\chi_2$, we again obtain $\chi_4$. 
    \begin{align*}
        \begin{array}{|c|c|c|c|c|c|}
            \hline
            S_4    & \overset{1}{[\id]} & \overset{6}{[(12)]} & \overset{8}{[(123)]} & \overset{6}{[(1234)]} & \overset{3}{[(12)(34)]} \\ \hline
            \chi_1 & 1                  & 1                   & 1                    & 1                     & 1                       \\ \hline
            \chi_2 & 1                  & -1                  & 1                    & -1                    & 1                       \\ \hline
            \chi_3 & 2                  &                     &                      &                       &                         \\ \hline
            \chi_4 & 3                  & 1                   & 0                    & -1                    & -1                      \\ \hline
            \chi_5 & 3                  & -1                  & 0                    & 1                     & -1                      \\ \hline
            \chi_P & 4                  & 2                   & 1                    & 0                     & 0                       \\ \hline 
        \end{array} 
    \end{align*}
    What happens if we do this with $\chi_3$? We have $X_2 = \{\chi_3\}$, 
    and we know that multiplication by $\chi_2$ must permute $X_2$, so that 
    $\chi_3 \cdot \chi_2 = \chi_3$. This tells us that 
    $\chi_3((12)) = \chi_3((1234)) = 0$, since $\chi_2((12)) = \chi_2((1234)) 
    = -1$. Let's now try to find the remaining entries in the row. 
    \begin{align*}
        \begin{array}{|c|c|c|c|c|c|}
            \hline
            S_4    & \overset{1}{[\id]} & \overset{6}{[(12)]} & \overset{8}{[(123)]} & \overset{6}{[(1234)]} & \overset{3}{[(12)(34)]} \\ \hline
            \chi_1 & 1                  & 1                   & 1                    & 1                     & 1                       \\ \hline
            \chi_2 & 1                  & -1                  & 1                    & -1                    & 1                       \\ \hline
            \chi_3 & 2                  & 0                   & a                    & 0                     & b                       \\ \hline
            \chi_4 & 3                  & 1                   & 0                    & -1                    & -1                      \\ \hline
            \chi_5 & 3                  & -1                  & 0                    & 1                     & -1                      \\ \hline
            \chi_P & 4                  & 2                   & 1                    & 0                     & 0                       \\ \hline 
        \end{array} 
    \end{align*}
    We take the inner product of $\chi_3$ with some of the other irreducible 
    characters. Observe that 
    \begin{align*}
        0 &= \langle \chi_3, \chi_1 \rangle = \frac1{24}(2 \cdot 1 \cdot 1 
        + a \cdot 1 \cdot 8 + b \cdot 1 \cdot 3) = \frac1{24}(2 + 8a + 3b), \\ 
        0 &= \langle \chi_3, \chi_4 \rangle = \frac1{24}(2 \cdot 3 \cdot 1 
        + a \cdot 0 \cdot 8 + b \cdot (-1) \cdot 3) = \frac1{24}(6 - 3b). 
    \end{align*}
    Solving this yields $a = -1$ and $b = 2$, so we have now completed 
    the character table of $S_4$. 
    \begin{align*}
        \begin{array}{|c|c|c|c|c|c|}
            \hline
            S_4    & \overset{1}{[\id]} & \overset{6}{[(12)]} & \overset{8}{[(123)]} & \overset{6}{[(1234)]} & \overset{3}{[(12)(34)]} \\ \hline
            \chi_1 & 1                  & 1                   & 1                    & 1                     & 1                       \\ \hline
            \chi_2 & 1                  & -1                  & 1                    & -1                    & 1                       \\ \hline
            \chi_3 & 2                  & 0                   & -1                   & 0                     & 2                       \\ \hline
            \chi_4 & 3                  & 1                   & 0                    & -1                    & -1                      \\ \hline
            \chi_5 & 3                  & -1                  & 0                    & 1                     & -1                      \\ \hline
        \end{array} 
    \end{align*}
\end{exmp}

\begin{exmp}{}
    Let's now find the character table of $S_5$. Note that $S_5$ has $7$ 
    conjugacy classes; their sizes are 
    \begin{align*}
        \#[\id] &= 1, \\ 
        \#[(12)] &= 10, \\ 
        \#[(123)] &= 20, \\ 
        \#[(1234)] &= 30, \\ 
        \#[(12345)] &= 24, \\ 
        \#[(12)(34)] &= 15, \\ 
        \#[(12)(345)] &= 20. 
    \end{align*}
    These sizes can be determined using Exercise 6 of Homework 3. This states that 
    if $\sigma \in S_n$ has disjoint cycle structure with $\lambda_i \geq 0$ 
    $i$-cycles for $i = 1, \dots, n$ (so that $\sum_{i=1}^n i\lambda_i = n$), 
    then the conjugacy class of $\sigma$ has size 
    \[ \frac{n!}{\prod_{i=1}^n \lambda_i! i^{\lambda_i}}. \] 

    \newpage
    Now, we recall that $S'_5 = A_5$, so $S_5/S'_5 = S_5/A_5 \cong C_2$. 
    Then the odd cycles get sent to $-1$ and the even cycles get sent to 
    $1$, so we obtain the first two rows of the character table. 
    \begin{align*}
        \begin{array}{|c|c|c|c|c|c|c|c|}
            \hline
            S_5    & \overset{1}{[\id]} & \overset{10}{[(12)]} & \overset{20}{[(123)]} & \overset{30}{[(1234)]} & \overset{24}{[(12345)]} & \overset{15}{[(12)(34)]} & \overset{20}{[(12)(345)]} \\ \hline
            \chi_1 & 1                  & 1                    & 1                     & 1                      & 1                       & 1                        & 1                         \\ \hline
            \chi_2 & 1                  & -1                   & 1                     & -1                     & 1                       & 1                        & -1                        \\ \hline
            \chi_3 &                    &                      &                       &                        &                         &                          &                           \\ \hline
            \chi_4 &                    &                      &                       &                        &                         &                          &                           \\ \hline
            \chi_5 &                    &                      &                       &                        &                         &                          &                           \\ \hline
            \chi_6 &                    &                      &                       &                        &                         &                          &                           \\ \hline
            \chi_7 &                    &                      &                       &                        &                         &                          &                           \\ \hline 
        \end{array} 
    \end{align*}
    Take the permutation representation $P : S_5 \to \GL_5(\C)$ given by 
    \[ P(\sigma)e_j = e_{\sigma(j)}. \] 
    Again, $\chi_P(\sigma)$ gives the number of fixed points of $\sigma$, so 
    let's add $\chi_P$ to the character table for now. 
    \begin{align*}
        \begin{array}{|c|c|c|c|c|c|c|c|}
            \hline
            S_5    & \overset{1}{[\id]} & \overset{10}{[(12)]} & \overset{20}{[(123)]} & \overset{30}{[(1234)]} & \overset{24}{[(12345)]} & \overset{15}{[(12)(34)]} & \overset{20}{[(12)(345)]} \\ \hline
            \chi_1 & 1                  & 1                    & 1                     & 1                      & 1                       & 1                        & 1                         \\ \hline
            \chi_2 & 1                  & -1                   & 1                     & -1                     & 1                       & 1                        & -1                        \\ \hline
            \chi_3 &                    &                      &                       &                        &                         &                          &                           \\ \hline
            \chi_4 &                    &                      &                       &                        &                         &                          &                           \\ \hline
            \chi_5 &                    &                      &                       &                        &                         &                          &                           \\ \hline
            \chi_6 &                    &                      &                       &                        &                         &                          &                           \\ \hline
            \chi_7 &                    &                      &                       &                        &                         &                          &                           \\ \hline
            \chi_P & 5                  & 3                    & 2                     & 1                      & 0                       & 1                        & 0                         \\ \hline 
        \end{array} 
    \end{align*}
    By Exercise 5 of Homework 3, we have 
    \[ \langle \chi_P, \chi_P \rangle = \frac1{5!} \sum_{\sigma \in S_5} 
    \fix(\sigma)^2 = 2. \] 
    We compute that 
    \[ \langle \chi_P, \chi_1 \rangle = \frac1{5!} 
    (5 \cdot 1 \cdot 1 + 3 \cdot 1 \cdot 10 + 2 \cdot 1 \cdot 20 + 1 \cdot 1 \cdot 30
    + 1 \cdot 1 \cdot 15) = 1. \] 
    As in the previous example, this tells us that $\chi_P - \chi_1$ is an 
    irreducible character. Since $\chi_P(\id) - \chi_1(\id) = 5 - 1 = 4$, 
    we obtain an irreducible character $\chi_3 = \chi_P - \chi_1$ of degree $4$.
    \begin{align*}
        \begin{array}{|c|c|c|c|c|c|c|c|}
            \hline
            S_5    & \overset{1}{[\id]} & \overset{10}{[(12)]} & \overset{20}{[(123)]} & \overset{30}{[(1234)]} & \overset{24}{[(12345)]} & \overset{15}{[(12)(34)]} & \overset{20}{[(12)(345)]} \\ \hline
            \chi_1 & 1                  & 1                    & 1                     & 1                      & 1                       & 1                        & 1                         \\ \hline
            \chi_2 & 1                  & -1                   & 1                     & -1                     & 1                       & 1                        & -1                        \\ \hline
            \chi_3 & 4                  & 2                    & 1                     & 0                      & -1                      & 0                        & -1                        \\ \hline
            \chi_4 &                    &                      &                       &                        &                         &                          &                           \\ \hline
            \chi_5 &                    &                      &                       &                        &                         &                          &                           \\ \hline
            \chi_6 &                    &                      &                       &                        &                         &                          &                           \\ \hline
            \chi_7 &                    &                      &                       &                        &                         &                          &                           \\ \hline
            \chi_P & 5                  & 3                    & 2                     & 1                      & 0                       & 1                        & 0                         \\ \hline 
        \end{array} 
    \end{align*}
    We can also take $\chi_4 = \chi_3 \cdot \chi_2$ to get another irreducible 
    character. 
    \begin{align*}
        \begin{array}{|c|c|c|c|c|c|c|c|}
            \hline
            S_5    & \overset{1}{[\id]} & \overset{10}{[(12)]} & \overset{20}{[(123)]} & \overset{30}{[(1234)]} & \overset{24}{[(12345)]} & \overset{15}{[(12)(34)]} & \overset{20}{[(12)(345)]} \\ \hline
            \chi_1 & 1                  & 1                    & 1                     & 1                      & 1                       & 1                        & 1                         \\ \hline
            \chi_2 & 1                  & -1                   & 1                     & -1                     & 1                       & 1                        & -1                        \\ \hline
            \chi_3 & 4                  & 2                    & 1                     & 0                      & -1                      & 0                        & -1                        \\ \hline
            \chi_4 & 4                  & -2                   & 1                     & 0                      & -1                      & 0                        & 1                         \\ \hline
            \chi_5 &                    &                      &                       &                        &                         &                          &                           \\ \hline
            \chi_6 &                    &                      &                       &                        &                         &                          &                           \\ \hline
            \chi_7 &                    &                      &                       &                        &                         &                          &                           \\ \hline
            \chi_P & 5                  & 3                    & 2                     & 1                      & 0                       & 1                        & 0                         \\ \hline 
        \end{array} 
    \end{align*}
    Now, let $a = \chi_5(\id)$, $b = \chi_6(\id)$, and $c = \chi_7(\id)$. 
    Then we know that 
    \[ 1^2 + 1^2 + 4^2 + 4^2 + a^2 + b^2 + c^2 = 120, \] 
    or equivalently, 
    \[ a^2 + b^2 + c^2 = 86. \] 
    Therefore, we have $2 \leq a, b, c \leq 9$, and we can deduce that 
    $(a, b, c) = (5, 5, 6)$. We have $X_6 = \{\chi_7\}$, and since $\chi_2$ 
    is of degree $1$, we have $\chi_7 \cdot \chi_2 = \chi_7$, which tells us 
    that $\chi_7((12)) = \chi((1234)) = \chi((12)(345)) = 0$, since 
    their counterparts in $\chi_2$ are equal to $-1$. We can now fill out 
    the first column and some of the entries of $\chi_7$. 
    \begin{align*}
        \begin{array}{|c|c|c|c|c|c|c|c|}
            \hline
            S_5    & \overset{1}{[\id]} & \overset{10}{[(12)]} & \overset{20}{[(123)]} & \overset{30}{[(1234)]} & \overset{24}{[(12345)]} & \overset{15}{[(12)(34)]} & \overset{20}{[(12)(345)]} \\ \hline
            \chi_1 & 1                  & 1                    & 1                     & 1                      & 1                       & 1                        & 1                         \\ \hline
            \chi_2 & 1                  & -1                   & 1                     & -1                     & 1                       & 1                        & -1                        \\ \hline
            \chi_3 & 4                  & 2                    & 1                     & 0                      & -1                      & 0                        & -1                        \\ \hline
            \chi_4 & 4                  & -2                   & 1                     & 0                      & -1                      & 0                        & 1                         \\ \hline
            \chi_5 & 5                  &                      &                       &                        &                         &                          &                           \\ \hline
            \chi_6 & 5                  &                      &                       &                        &                         &                          &                           \\ \hline
            \chi_7 & 6                  & 0                    &                       & 0                      &                         &                          & 0                         \\ \hline
            \chi_P & 5                  & 3                    & 2                     & 1                      & 0                       & 1                        & 0                         \\ \hline 
        \end{array} 
    \end{align*}
    Observe that $S_5$ acts on the subsets of size $2$; namely 
    $\{\{i, j\} : i, j \in \{1, 2, 3, 4, 5\}\}$. There are $\binom{5}{2} = 10$ 
    such subsets. Instead of considering the permutation representation 
    $P : S_5 \to \GL_5(\C)$, we have another representation 
    $Q : S_5 \to \GL_{10}(\C)$ where from the basis $\{e_{\{i,j\}} : 
    i, j \in \{1, 2, 3, 4, 5\}\}$, we define 
    \[ \sigma \cdot e_{\{i,j\}} = e_{\sigma(i)\sigma(j)}. \] 
    Let's now determine $\chi_Q$. This time, $\chi_Q(\sigma)$ is the number of sets 
    of size two that are fixed by $\sigma$. 
    \begin{align*}
        \begin{array}{|c|c|c|c|c|c|c|c|}
            \hline
            S_5    & \overset{1}{[\id]} & \overset{10}{[(12)]} & \overset{20}{[(123)]} & \overset{30}{[(1234)]} & \overset{24}{[(12345)]} & \overset{15}{[(12)(34)]} & \overset{20}{[(12)(345)]} \\ \hline
            \chi_1 & 1                  & 1                    & 1                     & 1                      & 1                       & 1                        & 1                         \\ \hline
            \chi_2 & 1                  & -1                   & 1                     & -1                     & 1                       & 1                        & -1                        \\ \hline
            \chi_3 & 4                  & 2                    & 1                     & 0                      & -1                      & 0                        & -1                        \\ \hline
            \chi_4 & 4                  & -2                   & 1                     & 0                      & -1                      & 0                        & 1                         \\ \hline
            \chi_5 & 5                  &                      &                       &                        &                         &                          &                           \\ \hline
            \chi_6 & 5                  &                      &                       &                        &                         &                          &                           \\ \hline
            \chi_7 & 6                  & 0                    &                       & 0                      &                         &                          & 0                         \\ \hline
            \chi_Q & 10                 & 4                    & 1                     & 0                      & 0                       & 2                        & 1                         \\ \hline 
        \end{array} 
    \end{align*}
    For example, note that $(12) \cdot e_{\{1,2\}} 
    = e_{\{1,2\}}$, because the set $\{2, 1\}$ is the same as the set $\{1, 2\}$. 
    Then, we see that the basis elements fixed by $(12)$ are $e_{\{1,2\}}$, 
    $e_{\{3,4\}}$, $e_{\{3,5\}}$, and $e_{\{4,5\}}$, so $\chi_Q((12)) = 4$. 
    The other values of $\chi_Q$ can be found similarly. We now compute that 
    \[ \langle \chi_Q, \chi_Q \rangle = \frac1{5!}(1 \cdot 10^2 
    + 10 \cdot 4^2 + 20 \cdot 1^2 + 15 \cdot 2^2 + 20 \cdot 1^2) = 3. \] 
    Therefore, $\chi_Q$ is the sum of $3$ distinct irreducible characters. 
    We find that 
    \begin{align*}
        \langle \chi_Q, \chi_1 \rangle &= \frac1{5!}(10 \cdot 1 \cdot 1 
        + 4 \cdot 1 \cdot 10 + 1 \cdot 1 \cdot 20 + 2 \cdot 1 \cdot 15 
        + 1 \cdot 1 \cdot 20) = 1, \\ 
        \langle \chi_Q, \chi_3 \rangle &= \frac1{5!}(10 \cdot 4 \cdot 1 
        + 4 \cdot 2 \cdot 10 + 1 \cdot 1 \cdot 20 + 1 \cdot (-1) \cdot 20) = 1. 
    \end{align*}
    Hence, $\chi_Q - \chi_1 - \chi_3$ is an irreducible character. 
    Note that $\chi_Q(\id) - \chi_1(\id) - \chi_3(\id) = 10 - 1 - 4 = 5$, and 
    since $\chi_5(\id) = 5$, we can set $\chi_5 = \chi_Q - \chi_1 - \chi_3$. 
    Then, $X_5 = \{\chi_5, \chi_6\}$, and since $\chi_2$ is a degree $1$ 
    character, it must permute $X_5$, so we can set $\chi_6 = \chi_5 \cdot \chi_2$. 
    This completes two more rows of the character table. 
    \begin{align*}
        \begin{array}{|c|c|c|c|c|c|c|c|}
            \hline
            S_5    & \overset{1}{[\id]} & \overset{10}{[(12)]} & \overset{20}{[(123)]} & \overset{30}{[(1234)]} & \overset{24}{[(12345)]} & \overset{15}{[(12)(34)]} & \overset{20}{[(12)(345)]} \\ \hline
            \chi_1 & 1                  & 1                    & 1                     & 1                      & 1                       & 1                        & 1                         \\ \hline
            \chi_2 & 1                  & -1                   & 1                     & -1                     & 1                       & 1                        & -1                        \\ \hline
            \chi_3 & 4                  & 2                    & 1                     & 0                      & -1                      & 0                        & -1                        \\ \hline
            \chi_4 & 4                  & -2                   & 1                     & 0                      & -1                      & 0                        & 1                         \\ \hline
            \chi_5 & 5                  & 1                    & -1                    & -1                     & 0                       & 1                        & 1                         \\ \hline
            \chi_6 & 5                  & -1                   & -1                    & 1                      & 0                       & 1                        & -1                        \\ \hline
            \chi_7 & 6                  & 0                    & d                     & 0                      & e                       & f                        & 0                         \\ \hline
            \chi_Q & 10                 & 4                    & 1                     & 0                      & 0                       & 2                        & 1                         \\ \hline 
        \end{array} 
    \end{align*}
    Finally, let's compute $\chi_7$ by determining $d = \chi_7((123))$, 
    $e = \chi_7((12345))$, and $f = \chi_7((12)(34))$. We have 
    \begin{align*}
        0 &= \langle \chi_7, \chi_5 \rangle = \frac1{5!}(30 - 20d + 15f), \\ 
        0 &= \langle \chi_7, \chi_3 \rangle = \frac1{5!}(24 + 20d - 24e), \\ 
        0 &= \langle \chi_7, \chi_1 \rangle = \frac1{5!}(6 + 20d + 24e + 15f).
    \end{align*}
    This is a system of $3$ equations and $3$ unknowns; solving it yields 
    $(d, e, f) = (0, 1, -2)$, completing the character table of $S_5$. 
    \begin{align*}
        \begin{array}{|c|c|c|c|c|c|c|c|}
            \hline
            S_5    & \overset{1}{[\id]} & \overset{10}{[(12)]} & \overset{20}{[(123)]} & \overset{30}{[(1234)]} & \overset{24}{[(12345)]} & \overset{15}{[(12)(34)]} & \overset{20}{[(12)(345)]} \\ \hline
            \chi_1 & 1                  & 1                    & 1                     & 1                      & 1                       & 1                        & 1                         \\ \hline
            \chi_2 & 1                  & -1                   & 1                     & -1                     & 1                       & 1                        & -1                        \\ \hline
            \chi_3 & 4                  & 2                    & 1                     & 0                      & -1                      & 0                        & -1                        \\ \hline
            \chi_4 & 4                  & -2                   & 1                     & 0                      & -1                      & 0                        & 1                         \\ \hline
            \chi_5 & 5                  & 1                    & -1                    & -1                     & 0                       & 1                        & 1                         \\ \hline
            \chi_6 & 5                  & -1                   & -1                    & 1                      & 0                       & 1                        & -1                        \\ \hline
            \chi_7 & 6                  & 0                    & 0                     & 0                      & 1                       & -2                       & 0                         \\ \hline
        \end{array} 
    \end{align*}
\end{exmp}

\section{Algebraic numbers and algebraic integers (10/29/2021)}
Write $\C[G] \cong M_{n_1}(\C) \times \cdots \times M_{n_s}(\C)$, and let 
$\chi_1, \dots, \chi_s$ be the irreducible characters. We have $\chi_i(1_G) 
= n_i$, which is the degree of $\chi_i$. Moreover, since $\langle \chi_i, 
\chi_j \rangle = \delta_{ij}$, we know that $\|\chi\|^2 = 1$ if and only if 
$\chi$ is irreducible. In particular, if the character $\chi = a_1 \chi_1 
+ \cdots + a_s \chi_s$ is irreducible, then 
\[ \|\chi\|^2 = a_1^2 + \cdots + a_s^2 = 1 \] 
implies that there is a unique $1 \leq j \leq s$ such that $a_j = 1$ and $a_i = 0$ 
for all $i \neq j$. Hence, $\|\chi\|^2 = 1$ if and only if $\chi = \chi_j$ for some 
$1 \leq j \leq s$. Today, we'll introduce the algebraic integers, which will help 
us to prove the last part of the big theorem, which states that $n_i \mid |G|$ for all 
$1 \leq i \leq s$. 

\begin{defn}{}
    We define the {\bf algebraic numbers} $\A \subseteq \C$ to be the set 
    \[ \A = \{\alpha \in \C : P(\alpha) = 0 \text{ for some monic polynomial }
    P(x) \in \Z[x]\}. \] 
\end{defn}

\begin{exmp}{}
    \begin{enumerate}[(1)]
        \item Observe that $\Z \subseteq \A$, since for any $n \in \Z$, the 
              polynomial $P(x) = x - n$ has $n$ as a root. 
        \item If $\alpha \in \Q$, then $\beta = e^{\pi i\alpha} \in \A$. Indeed, 
              write $\alpha = a/b$ for some $a, b \in \Z$ with $b \neq 0$. 
              Taking the polynomial $P(x) = x^{2b-1}$, we have $P(\beta) = 
              (e^{\pi ia/b})^{2b} - 1 = e^{2\pi ia} - 1 = 0$. 
        \item Let $a \in \Z$ and $b \in \Z^+$. Then $\alpha = \sqrt[b]{a} \in 
              \A$ by taking $P(x) = x^b - a \in \Z[x]$. 
    \end{enumerate}
\end{exmp}

\begin{lemma}{}
    We have $\A \cap \Q = \Z$. 
\end{lemma}
\begin{pf}
    Exercise 1 of Homework 4 shows that $\Z \subseteq \A \cap \Q$. Next, 
    we clearly have $\alpha = 0 \in \Z$. Suppose that we have a 
    nonzero $\alpha = a/b$ with $\gcd(a, b) = 1$ and $b > 0$ which is a root 
    of the monic polynomial 
    \[ x^n + c_{n-1} x^{n-1} + \cdots + c_1 x + c_0 \in \Z[x]. \] 
    Moreover, we can assume that $c_0 \neq 0$, for otherwise we could just 
    reduce the degree of the polynomial. Then by the rational roots theorem, 
    we obtain $b \mid 1$ and $a \mid c_0$, which implies that $\alpha = 
    a/b \in \Z$. Therefore, we conclude that $\A \cap \Q = \Z$. 
\end{pf}

This statement is very simple, but it has some very powerful consequences. 
Recall that a finitely generated $\Z$-submodule of $(\C, +)$ is given by 
all elements of the form 
\[ \{n_1 \lambda_1 + \cdots + n_p \lambda_p : n_1, \dots, n_p \in \Z\} \] 
for some fixed $\lambda_1, \dots, \lambda_p \in \C$. For example, we have 
$\Z[i] = \{a + bi : a, b \in \Z\}$ by choosing $\lambda_1 = 1$ and $\lambda_2 = i$.
We'll now prove some characterizations of algebraic numbers.  

\begin{theo}{}
    Let $\alpha \in \C$. The following are equivalent: 
    \begin{enumerate}[(1)]
        \item $\alpha \in \A$. 
        \item There exists a nonzero finitely generated $\Z$-submodule $M$ of 
              $\C$ such that $\alpha M \subseteq M$. 
        \item There exists $p \geq 1$ and $\lambda_1, \dots, \lambda_p \in 
              \C \setminus \{0\}$ such that $\alpha \lambda_i \in 
              \Z\lambda_1 + \cdots + \Z\lambda_p$ for all $i = 1, \dots, p$. 
    \end{enumerate}
\end{theo}
\begin{pf}
    (1) $\Rightarrow$ (2). If $\alpha \in \A$, then there exists $n \geq 1$ 
    and $c_0, \dots, c_{n-1} \in \Z$ such that 
    \begin{equation}
        \alpha^n + c_{n-1} \alpha^{n-1} + \cdots + c_1 \alpha + c_0 = 0. 
    \end{equation}
    Let $M = \Z + \Z\alpha + \Z\alpha^2 + \cdots + \Z\alpha^{n-1}$. This is a 
    finitely generated $\Z$-submodule, and $M \neq (0)$ since $1 \in M$. 
    Now, we claim that $\alpha M \subseteq M$. To this end, suppose that 
    $a_0 + a_1 \alpha + \cdots + a_{n-1} \alpha^{n-1} \in M$. Then we have 
    \begin{align*}
        \alpha (a_0 + a_1 \alpha + \cdots + a_{n-1} \alpha^{n-1}) 
        &= a_0 \alpha + a_1 \alpha^2 + \cdots + a_{n-2} \alpha^{n-1} 
        + a_{n-1} \alpha^n \\ 
        &= a_0 \alpha + a_1 \alpha^2 + \cdots + a_{n-2} \alpha^{n-1} 
        + a_{n-1} (- c_{n-1} \alpha^{n-1} - \cdots - c_1 \alpha - c_0) \\ 
        &= -a_{n-1} c_0 + (a_0 - a_{n-1} c_1) \alpha + \cdots + 
        (a_{n-2} - a_{n-1} c_{n-1}) \alpha^{n-1},
    \end{align*}
    where the second last equality follows from (20.1).
    We see that $\alpha(a_0 + a_1 \alpha + \cdots + a_{n-1}) \in M$, so 
    $\alpha M \subseteq M$. 

    (2) $\Rightarrow$ (3). Suppose there exists a nonzero finitely generated 
    $\Z$-submodule $M$ of $\C$ such that $\alpha M \subseteq M$. Then we can 
    pick $\lambda_1, \dots, \lambda_p \in \C \setminus \{0\}$ such that 
    $M = \Z\lambda_1 + \cdots + \Z\lambda_p$. Since $\alpha M \subseteq M$,
    this implies that $\alpha \lambda_i \in M$ for each $i = 1, \dots, p$. 

    (3) $\Rightarrow$ (1). Suppose there exists $p \geq 1$ and $\lambda_1, 
    \dots, \lambda_p \in \C \setminus \{0\}$ such that 
    $\alpha\lambda_i \in \Z\lambda_i + \cdots + \Z\lambda_p$ for all $i = 
    1, \dots, p$. Then for $j = 1, \dots, p$, we can pick $a_{j1}, \dots, 
    a_{jp} \in \Z$ such that 
    \[ \alpha\lambda_i = a_{j1} \lambda_1 + \cdots + a_{jp} \lambda_p. \] 
    We can write this as a matrix equation to get 
    \[ \alpha \begin{pmatrix}
        \lambda_1 \\ \lambda_2 \\ \vdots \\ \lambda_p 
    \end{pmatrix} = \begin{pmatrix}
        a_{11} & a_{12} & \cdots & a_{1p} \\ 
        a_{21} & a_{22} & \cdots & a_{2p} \\
        \vdots & \vdots & \ddots & \vdots \\ 
        a_{p1} & a_{p2} & \cdots & a_{pp}
    \end{pmatrix} \begin{pmatrix}
        \lambda_1 \\ \lambda_2 \\ \vdots \\ \lambda_p 
    \end{pmatrix}. \] 
    Set $v = (\lambda_1, \dots, \lambda_p)^T$, and notice that $v \neq 0$ 
    since $\lambda_1, \dots, \lambda_p \in \C \setminus \{0\}$. 
    Moreover, set $A$ to be the matrix above, and observe that $A \in M_p(\Z)$. 
    Now, we have $\alpha v = Av$, so $\alpha$ is an eigenvalue of $A$. Hence, 
    $\alpha$ is a root of the characteristic polynomial of $A$. That is, for 
    \[ P_A(x) = \det(xI - A) = \det \begin{pmatrix}
        x - a_{11} & \cdots & -a_{1p} \\ 
        \vdots & \ddots & \vdots \\ 
        -a_{p1} & \cdots & x - a_{pp}
    \end{pmatrix}, \] 
    we have $P_A(\alpha) = 0$. Note that $A \in M_p(\Z)$, so $P_A(x)$ is a 
    monic polynomial in $\Z[x]$. It follows that $\alpha \in \A$. \qedhere 
\end{pf}

\begin{theo}{}
    The algebraic numbers $\A$ form a subring of $\C$. 
\end{theo}
\begin{pf}
    It suffices to show that $\A$ is closed under addition and multiplication, 
    and that $-1 \in \A$; this last condition assures that we have additive inverses. 

    We have already seen that $\Z \subseteq \A$, so $-1 \in \A$. Let 
    $\alpha, \beta \in \A$. We assume that $\alpha$ and $\beta$ are both 
    nonzero, as the result is trivial otherwise. We want to show that 
    $\alpha + \beta$ and $\alpha\beta$ are in $\A$. By definition, we have 
    \begin{align*}
        \alpha^n + c_{n-1} \alpha^{n-1} + \cdots + c_1 \alpha + c_0 = 0, \\ 
        \beta^m + d_{m-1} \beta^{m-1} + \cdots + d_1 \beta + d_0 = 0,
    \end{align*}
    for some $c_0, \dots, c_{n-1}, d_0, \dots, d_{n-1} \in \Z$. Let 
    \[ M = \sum_{i=0}^{n-1} \sum_{j=0}^{m-1} \Z\alpha^i\beta^j, \] 
    which is a finitely generated $\Z$-submodule of $\C$, and $M \neq (0)$ since 
    $1 \in M$. Now, we claim that $\alpha M \subseteq M$ and $\beta M \subseteq M$. 
    It suffices to show that $\beta (\alpha^i \beta^j)$ and $\alpha (\alpha^i 
    \beta^j)$ are in $M$ for all $0 \leq i \leq n-1$ and $0 \leq j \leq m-1$. 

    We will consider $\alpha(\alpha^i \beta^j)$. When $0 \leq i \leq n-2$, we have 
    $\alpha (\alpha^i \beta^j) = \alpha^{i+1} \beta^j \in M$, since 
    $0 \leq i+1 \leq n-1$. Otherwise, we have $i = n-1$, in which case 
    \[ \alpha(\alpha^{n-1}\beta^j) = \alpha^n \beta^j = 
    (-c_{n-1} \alpha^{n-1} - \cdots - c_1 \alpha- c_0) \beta^j
    \in M. \] 
    This implies that $\alpha M \subseteq M$, and a similar argument shows that 
    $\beta M \subseteq M$. Now, observe that 
    \[ (\alpha + \beta)M \subseteq \alpha M + \beta M \subseteq M + M = M, \] 
    and similarly, we know from our work above that 
    \[ (\alpha \beta) M = \alpha(\beta M) \subseteq \alpha M \subseteq M. \] 
    It follows from Theorem 20.4 that $\alpha + \beta$ and $\alpha\beta$ are in $\A$.
\end{pf}

\begin{remark}{}
    Note that $\A$ is not a field, since $2 \in \A$ but $1/2 \notin \A$. 
\end{remark}

Now, we'll prove an interesting corollary of the fact that $\A$ is a subring of $\C$.

\begin{cor}{}
    If $\alpha \in \Q$ and $\tan(\pi\alpha) \in \Q$, then $\tan(\pi\alpha) 
    \in \{0, 1, -1\}$. 
\end{cor}
\begin{pf}
    Recall that $\tan^2(x) + 1 = 1/\cos^2(x)$. Therefore, if $\tan(\pi\alpha) 
    \in \Q$, then we have $\tan^2(\pi\alpha) + 1 = 1/\cos^2(\pi\alpha) \in \Q$ as 
    well. This implies that $4\cos^2(\pi\alpha) \in \Q$. Now, notice that 
    \[ 4\cos^2(\pi\alpha) = (2 \cos(\pi\alpha))^2 = (e^{i\pi\alpha} + 
    e^{-i\pi\alpha})^2. \] 
    In particular, since $e^{i\pi\alpha}$ and $e^{-i\pi\alpha}$ are both 
    algebraic numbers (as we saw in Example 20.2), this shows that 
    $4\cos^2(\pi\alpha) \in \Q \cap \A = \Z$. Now, we know that 
    $4\cos^2(\pi\alpha) \in \{0, 1, 2, 3, 4\}$. Taking the inverses and 
    multiplying by $4$, we find that $1/\cos^2(\pi\alpha) \in \{4, 2, 4/3, 1\}$. 
    Finally, we have 
    \[ \tan^2(\pi\alpha) = \frac{1}{\cos^2(\pi\alpha)} - 1 \in 
    \left\{3, 1, \frac13, 0\right\}. \]  
    But we assumed that $\tan(\pi\alpha) \in \Q$, so the only possibilities are 
    $\tan(\pi\alpha) \in \{0, 1, -1\}$, as desired. 
\end{pf}

\section{Degree of irreps divide order of the group (11/01/2021)} 
Let $G$ be a finite group. Let $\rho : \C[G] \to M_{n_1}(\C) \times \cdots 
\times M_{n_s}(\C)$ be a $\C$-algebra isomorphism. Then, we have the following 
commutative diagram. 
\begin{center}
    \begin{tikzcd}
        {\C[G]} \arrow[rr, "\rho"] \arrow[rrdd, "\rho_i = \pi_i \circ\,\rho"'] &  & M_{n_1}(\C) \times \cdots \times M_{n_s}(\C) \arrow[dd, "\pi_i"] \\
                                                                            &  &                                                                  \\
                                                                            &  & M_{n_i}(\C)                                                     
    \end{tikzcd}
\end{center}
Then, we obtain characters $\chi_i(g) = \Tr(\rho_i(g))$. We also have $\C[G]$-modules 
$V_i = \C^{n_i \times 1}$ with the action $g \cdot v = \rho_i(g) \cdot v$. 
Our goal in this lecture is to prove that $n_i \mid |G|$ for each $i = 1, \dots, s$, 
where $\chi_i(1_G) = n_i$ is the degree of the character. Note that 
$\rho_i(1_G) = I_{n_i}$. 

\begin{lemma}{}
    Let ${\cal C}$ be a conjugacy class of $G$, and let $g \in {\cal C}$. Then 
    for all $i = 1, \dots, s$, we have 
    \[ \frac{|{\cal C}|\chi_i(g)}{n_i} \in \A. \] 
\end{lemma}
\begin{pf}
    Let ${\cal C}_1, \dots, {\cal C}_s$ be the conjugacy classes of $G$. 
    Recall from Proposition 10.7 that a basis for $Z(\C[G])$ is given by 
    $z_1, \dots, z_s$, where 
    \[ z_p = \sum_{h \in {\cal C}_p} h. \] 
    Let $M = \Z z_1 + \cdots + \Z z_s \subseteq Z(\C[G])$. We claim that 
    $z_p M \subseteq M$ for all $p = 1, \dots, s$. 

    It suffices to show that $z_p z_j \in M$ for all $p = 1, \dots, s$ and 
    $j = 1, \dots, s$, as every element in $M$ can be written as a linear 
    combination of $z_1, \dots, z_s$. In particular, notice that 
    \[ z_p(c_1z_1 + \cdots + c_sz_s) = c_1z_pz_1 + \cdots + c_sz_pz_s \in M. \] 
    Notice that $z_p z_j \in Z(\C[G])$, so we have 
    \begin{equation} z_p z_j = \sum_{k=1}^s \lambda_{p,j,k} z_k \end{equation}
    for some $\lambda_{p,j,k} \in \C$. In order to show that $z_p z_j \in M$, 
    we require that $\lambda_{p,j,k} \in \Z$ for all $k = 1, \dots, s$. 
    Pick $h \in {\cal C}_k$ and consider the coefficient of $h$ in $(21.1)$. 
    On the right hand side, it is simply $\lambda_{p,j,k}$. On the left 
    hand side, note that 
    \[ z_p z_j = \left( \sum_{h' \in {\cal C}_p} h' \right) \left( \sum_{h'' \in 
    {\cal C}_j} h'' \right), \] 
    so the coefficient of $h$ is $\#\{(h', h'') \in {\cal C}_p \times {\cal C}_j : 
    h' \cdot h'' = h\} \in \Z_{\geq 0}$. This proves the claim. 

    Now, consider the diagram above, and note that $\rho_i$ is surjective. Since 
    $z_j$ is central in $\C[G]$, we see that $\rho_i(z_j) \in Z(M_{n_i}(\C))$. 
    By surjectivity of $\rho_i$, it follows that $\rho_i(z_j) = 
    \gamma_{i,j} I_{n_i}$ with $\gamma_{i,j} \in \C$; that is, $\rho_i(z_j)$ is a 
    scalar matrix. Next, define 
    \[ M_i = \Z\gamma_{i,1} + \cdots \Z\gamma_{i,s} \subseteq \C, \] 
    which is a finitely generated $\Z$-submodule of $\C$ with $M_i \neq (0)$. 
    Notice that $M_i = \rho_i(M)$ by construction after identifying the 
    scalar matrices $\C \cdot I_{n_i}$ with $\C$. Now, since $z_j M \subseteq M$ 
    by our claim, we obtain $\rho_i(z_j) \rho_i(M) \subseteq \rho_i(M)$, so 
    \[ \gamma_{i,j} M_i \subseteq M_i \] 
    after identification. Hence, for all $1 \leq i, j \leq s$, we get 
    $\gamma_{i,j} \in \A$. Next, let ${\cal C} = {\cal C}_j$ be a conjugacy class 
    and $g \in {\cal C}$. Note that 
    \[ \gamma_{i,j} I_{n_i} = \rho_i(z_j) = \rho_i \left( \sum_{h \in {\cal C}_j} h 
    \right) = \sum_{h \in {\cal C}_j} \rho_i(h). \] 
    Taking the trace of both sides yields 
    \[ \gamma_{i,j} \cdot n_i = \Tr(\gamma_{i,j} I_{n_i}) = 
    \Tr\left( \sum_{h\in {\cal C}_j} \rho_i(h) \right) = \sum_{h \in {\cal C}_j} 
    \chi_i(h) = |{\cal C}| \chi_i(g), \] 
    where the last equality is because $\chi_i$ is a class function. It follows that 
    \[ \frac{|{\cal C}| \chi_i(g)}{n_i} = \gamma_{i,j} \in \A. \qedhere \] 
\end{pf}

\begin{theo}{}
    Let $G$ be a finite group. If $\chi_i$ is an irreducible character with 
    degree $n_i$, then $n_i \mid |G|$. 
\end{theo}
\begin{pf}
    We always have a trivial irreducible character with $\chi(h) = 1$ for all 
    $h \in G$. Without loss of generality, suppose that $\chi_1$ is the trivial 
    representation, which has degree $n_1 = 1$, in which case the result 
    clearly holds. 

    It suffices to consider the case where $i > 1$. Let ${\cal C}_1, \dots, 
    {\cal C}_s$ be the conjugacy classes of $G$. For each $j = 1, \dots, 
    s$, pick a representative $g_j \in {\cal C}_j$. Then we have 
    \[ 1 = \langle \chi_i, \chi_i \rangle = 
    \frac1{|G|} \sum_{j=1}^s |{\cal C}_j| \chi_i(g_j) \overline{\chi_i(g_j)} = 
    \frac{n_i}{|G|} \sum_{j=1}^s \frac{|{\cal C}_j| \chi_i(g_j)}{n_i} 
    \cdot \overline{\chi_i(g_j)}. \] 
    By Lemma 21.1, each $|{\cal C}_j|\chi_i(g_j)/n_i$ is in $\A$. Moreover, 
    we have shown that $\rho_i(g)$ is triangularizable with roots of 
    unity $\omega_1, \dots, \omega_{n_i}$ along the diagonal, so 
    \[ \overline{\chi_i(g_j)} = \overline{\omega_1} + \cdots + \overline{\omega_{n_i}} 
    \in \A. \] 
    In particular, we see that 
    \[ 1 = \frac{n_i}{|G|} \cdot \alpha \] 
    where $\alpha \in \A$. Rearranging gives $|G|/n_i = \alpha$. Observe that 
    $|G|/n_i \in \Q \cap \A = \Z$, so $n_i \mid |G|$. 
\end{pf}