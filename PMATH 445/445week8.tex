\section{Burnside's theorem (11/03/2021)} 
The fact that the degrees of the irreducible characters divide the order of 
the group has some very powerful consequences. The Feit-Thompson theorem 
states that every group of odd order is solvable. We'll focus on Burnside's 
theorem, which gives the weaker statement that every group of order 
$p^a q^b$ is solvable, where $p$ and $q$ are distinct primes and $a, b \in \Z^+$. 
The proofs of these theorems are very difficult (if not impossible) without 
using character theory, and we'll use it to make our lives easier. 
First, let's recall what it means for a group to be solvable. 

\begin{defn}{}
    A group $G$ is {\bf solvable} if there exists a tower 
    \[ G = G_0 \supseteq G_1 \supseteq G_2 \supseteq \cdots \supseteq G_m = \{1\} \] 
    such that $G_{i+1} \trianglelefteq G_i$ and $G_i/G_{i+1}$ is abelian for 
    all $0 \leq i \leq m-1$. 
\end{defn}

Towards the proof of Burnside's theorem, we'll first prove some intermediate results. 

\begin{lemma}{}
    Let $\chi$ be an irreducible character of degree $n \geq 2$. If $\chi(g)/n 
    \in \A$ for some $g \in G$, then $\chi(g) = 0$. 
\end{lemma}

We leave the proof of this lemma for the next lecture. For the moment, we 
will use this lemma to prove the following theorem. 

\begin{theo}{}
    Let $G$ be a non-abelian finite group. Suppose that $G$ has a conjugacy 
    class of size $p^a$ where $p$ is a prime and $a \in \Z^+$. Then $G$ is 
    not simple. 
\end{theo}
\begin{pf}
    Suppose by way of contradiction that there exists a non-abelian finite 
    simple group $G$ with an element $g \in G$ such that the conjugacy class 
    ${\cal C} = {\cal C}(g)$ containing $g$ has size $p^a$, where $p$ is prime and 
    $a \in \Z^+$. Since $G$ is non-abelian and simple, we have $G' = G$. 
    Indeed, note that $G' \trianglelefteq G$. By simplicity, we either 
    have $G' = G$ or $G' = \{1\}$. But $G$ is non-abelian, so we must be in the first 
    case. This means that $G$ has one inequivalent irreducible representation 
    of degree $1$ since $|G/G'| = |G/G| = 1$. The only irreducible character 
    of degree $1$ is the trivial character $\chi$ with $\chi(h) = 1$ for all $h \in G$.

    Let $\chi_1, \dots, \chi_s$ be the irreducible characters of $G$. Without 
    loss of generality, we may assume that $\chi_1$ is the trivial character. 
    Notice that if $n_1, \dots, n_s$ are the degrees of $\chi_1, \dots, \chi_s$ 
    respectively, then $n_1 = 1$ and $n_i > 1$ for $i = 2, \dots, s$. Recall 
    that we have $g \in G$ such that $|{\cal C}| = |{\cal C}(g)| = p^a$, 
    where $p$ is prime and $a \in \Z^+$. This implies that $g \neq 1$, for 
    otherwise its conjugacy class would have size $1$. Now, if $L$ is the 
    left regular representation of $G$, then $\chi_L(g) = 0$. This implies that 
    \[ 0 = \chi_L(g) = \sum_{i=1}^s n_i \chi_i(g) 
    = \chi_1(g) + \sum_{\substack{i\geq 2 \\ p\,\mid\,n_i}} n_i \chi_i(g) 
    + \sum_{\substack{i\geq 2 \\ p\,\nmid\,n_i}} n_i \chi_i(g). \]
    {\sc Claim.} For $i \geq 2$, if $p \nmid n_i$, then $\chi_i(g) = 0$. 

    {\sc Proof of Claim.} We know that $\chi_i(g) \in \A$. We have also showed 
    in Lemma 21.1 that 
    \[ \frac{p^a \chi_i(g)}{n_i} = \frac{|{\cal C}(g)| \chi_i(g)}{n_i} \in \A. \] 
    Since $p \nmid n_i$, we have $\gcd(p^a, n_i) = 1$. Therefore, there 
    exist $c, d \in \Z$ such that $cp^a + dn_i = 1$. We find that 
    \[ \frac{\chi_i(g)}{n_i} = \frac{cp^a + dn_i}{n_i} \chi_i(g) 
    = \frac{cp^a \chi_i(g)}{n_i} + d\chi_i(g) \in \A. \] 
    By Lemma 22.2, we obtain $\chi_i(g) = 0$. \hfill$\blacksquare$ 

    Now, writing $n_i = pn'_i$ for each $i \geq 2$ with $p \mid n_i$, we have 
    \[ 0 = \chi_1(g) + \sum_{\substack{i\geq 2 \\ p\,\mid\,n_i}} n_i \chi_i(g) 
    + \sum_{\substack{i\geq 2 \\ p\,\nmid\,n_i}} n_i \chi_i(g) 
    = \chi_1(g) + p \sum_{\substack{i\geq 2 \\ p\,\mid\,n_i}} n'_i \chi_i(g). \] 
    Let $\alpha$ denote the sum in the above equation, and note that $\alpha \in \A$. 
    Then $1 + p\alpha = 0$, which implies that $\alpha = -1/p$. But this is a 
    contradiction since $\Q \cap \A = \Z$, but $-1/p \notin \Z$. 
\end{pf}

We can now prove Burnside's theorem. First, we'll recall some facts from group theory. 
\begin{enumerate}[(1)]
    \item Let $N \trianglelefteq G$. Then $G$ is solvable if and only if 
          $N$ and $G/N$ are both solvable. 
    \item Let $G$ be a group of order $q^b c$ where $q$ is prime, $b \in \Z^+$, 
          and $q \nmid c$. Then by the first Sylow theorem, $G$ has a subgroup 
          $Q$ of order $q^b$. 
    \item If $G$ is a group of order $q^b$ where $q$ is prime and $b \in \Z^+$, 
          then $G$ has a non-trivial center by the class equation. 
    \item If $g \in G$ and ${\cal C}(g)$ is the conjugacy class of $g$, then 
          $|{\cal C}(g)| = |G|/|C(g)|$, where $C(g) = \{h \in G : hg = gh\}$ 
          is the centralizer of $g$. 
\end{enumerate}

\begin{theo}[Burnside's theorem]{}
    Let $G$ be a group of order $p^a q^b$ where $p$ and $q$ are distinct primes 
    and $a, b \in \Z^+$. Then $G$ is solvable. 
\end{theo}
\begin{pf}
    Suppose the result is not true. Let $G$ be a minimal counterexample; that is, 
    $G$ is a group of order $p^a q^b$ where $p$ and $q$ are distinct primes and 
    $a, b \in \Z^+$, and $G$ has minimal size with respect to this property. 

    {\sc Claim.} $G$ is a non-abelian simple group.  

    {\sc Proof of Claim.} Note that $G$ is non-abelian since if $G$ were 
    abelian, then it would be solvable. Assume that $G$ is not simple. 
    Then there exists a non-trivial proper normal subgroup $N$ of $G$. 
    Then $G/N$ and $N$ are both smaller than $G$ and their sizes divide $p^a q^b$. 
    By minimality, $N$ and $G/N$ are solvable, so $G$ is also solvable by (1), a 
    contradiction. \hfill$\blacksquare$ 

    Pick a subgroup $Q \leq G$ with $|Q| = q^b$, which exists by (2). Moreover, 
    we can choose $z \neq 1$ such that $z \in Z(Q)$, which exists since 
    $Q$ has non-trivial center by (3). Then $Q \subseteq C(z) \subseteq G$, 
    so $|C(z)| = p^c q^b$ where $c \in \{0, \dots, a\}$. By (4), it follows that 
    \[ |{\cal C}(z)| = \frac{|G|}{|C(z)|} = \frac{p^a q^b}{p^c q^b} = p^{a-c}. \] 
    If $c < a$, then $G$ is not simple by Theorem 22.3 since $G$ 
    is non-abelian and $|{\cal C}(z)|$ is a conjugacy class with size which is a 
    prime power, contradicting our claim. So $c = a$, which gives $|{\cal C}(z)| 
    = 1$. But this means that $C(z) = G$. Moreover, $z$ is central, which gives 
    $\{1\} \neq Z(G) \trianglelefteq G$. Since $G$ is simple, we must have 
    $Z(G) = G$. This would mean that $G$ is abelian, once again contradicting 
    our claim. Therefore, no counterexample exists, so the result holds. 
\end{pf}