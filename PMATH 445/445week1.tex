\section{Embeddings and group algebras (09/08/2021)}

We begin the course by recalling Cayley's theorem, a famous result from group theory. It states that 
every finite group $G$ embeds (there exists an injective homomorphism) into a symmetric group $S_n$. 
The proof is simple: let $G$ act on itself by left multiplication, and show that this gives an 
embedding of $G$ into $S_n$ where $n = |G|$. This result is simple, but it allows us to understand 
finite groups as subgroups of symmetric groups, where one has many tools to use. 

In representation theory, one seeks to understand groups in terms of maps into general linear groups 
$\GL_n(F)$, where $F$ is a field. This is generally more desirable than an embedding into 
a symmetric group, as we obtain the full power of linear algebra at our disposal. We will consider 
all homomorphisms (not just injective ones) from groups to general linear groups, and such 
homomorphisms are called {\bf representations} of our group. First, we show that 
every finite field embeds into $\GL_n(F)$ for some field $F$.

\begin{prop}{}
Let $F$ be a field. Every finite group embeds into $\GL_n(F)$ for some $n \geq 1$. 
\end{prop}
\begin{pf}
Let $G$ be a finite group. By Cayley's theorem, we have an embedding $G \hookrightarrow S_n$ 
where $n = |G|$. Hence, it suffices to show that $S_n$ embeds into $\GL_n(F)$. Define 
$\varphi : S_n \to \GL_n(F)$ by $\psi(\sigma) = P_\sigma$, where $P_\sigma$ denotes the permutation matrix. Notice that for $\sigma_1, \sigma_2 \in S_n$, we have $\varphi(\sigma_1\sigma_2) 
= P_{\sigma_1\sigma_2} = P_{\sigma_1}P_{\sigma_2} = \varphi(\sigma_1\sigma_2)$, so $\varphi$ is a 
group homomorphism. One can also check that if $\varphi(\sigma) = I$, the identity matrix, then 
$\sigma$ must be the identity permutation, so $\varphi$ is injective.
\end{pf}

It turns out that this result is not true for infinite groups in general. 

\begin{exmp}{}
Let $G$ be the group consisting of bijective maps from $\Z^+$ to itself such that $f$ fixes all
but finitely integers. There does not exist a field $F$ and $n \geq 1$ such that 
$G$ embeds into $\GL_n(F)$. 
\end{exmp}
\begin{pf} 
First, we note the following fact from linear algebra.

{\sc Fact 1.} If $A$ and $B$ are commuting diagonalizable matrices, then they are simultaneously 
diagonalizable. That is, there is a common change of basis that makes both matrices diagonalizable.
This result also extends to families of commuting diagonalizable matrices. 

Now, we denote by $(i, j)$ the bijective mapping from $\Z^+$ to itself which swaps $i$ and $j$ 
and fixes all other integers. Consider the permutations $(1, 2)$, $(3, 4)$, $(5, 6)$, and so on. 
Note that they pairwise commute. Suppose that there exists an injective homomorphism 
$\varphi: G \to \GL_n(F)$ for some $n \geq 1$ and a field $F$. Let $A_1 = \varphi(1, 2)$, 
$A_2 = \varphi(3, 4)$, and so on. Observe that we have 
\[ \varphi((i, i+1)^2) = \varphi(\id) = I, \]
which implies that $A_1^2 = A_2^2 = \cdots = I$. We now recall another fact from linear algebra. 

{\sc Fact 2.} If the minimal polynomial of a matrix has distinct roots over the (algebraically closed) field $F$, then the matrix is diagonalizable. 

We see from above that the minimal polynomial of the $A_i$ must divide $x^2 - 1$, since 
$A_i^2 - I = 0$. As $x^2 - 1$ has distinct roots, it follows from Fact 2 that all the $A_i$ are diagonalizable. Moreover, by Fact 1, we can assume after a change of basis that each $A_i$ is of the form 
\[ A_i = \begin{pmatrix} \eps_{1,i} & & 0 \\ & \ddots & \\ 0 & & \eps_{n,i} \end{pmatrix} \]
where $\eps_{1,i}, \dots, \eps_{n,i} \in \{\pm 1\}$. Now we have a problem: there are only 
$2^n$ such matrices of the above form, and infinitely many positive integers. Thus, there 
exist positive integers $i < j$ such that $\varphi(A_i) = \varphi(A_j)$, so 
$\varphi$ is not injective, and this yields our contradiction.

Note that this argument needs an adjustment for an algebraically closed field of characteristic $2$, 
since $x^2 - 1 = (x - 1)^2$ does not have distinct roots. 
In such a case, we can proceed in the same way, except we use distinct $3$-cycles instead of $2$-cycles.
\end{pf}

We now turn to the notion of a group algebra.

\begin{defn}{}
The {\bf group algebra} of the group $G$ over the field $k$ is defined by 
\[ k[G] = \left\{ \sum_{g\in G} \alpha_g \cdot g : \alpha_g \in k,\, \alpha_g = 0 \text{ for all but 
finitely many $g$} \right\}. \]
\end{defn}

We note that $k[G]$ is a ring with a natural addition, and multiplication given by 
\[ \left( \sum_{g\in G} \alpha_g \cdot g \right) \left( \sum_{h \in G} \beta_h \cdot h \right) 
= \sum_{y \in G} \left( \sum_{(g, h) : gh = y} \alpha_j \cdot \beta_h \right) \cdot y. \]
Notice that the inner sum is finite because by definition, there are only finitely many 
non-zero $\alpha_g$ and $\beta_h$. 

\begin{remark}{}
We call $k[G]$ a group {\it algebra} because we have a ``copy'' of $k$ in $k[G]$ given by 
$\lambda \mapsto \lambda \cdot 1_G$ for elements $\lambda \in k$, with $\lambda \cdot g = g 
\cdot \lambda$ for all $g \in G$. We see that $k[G]$ is a $k$-vector space of dimension $|G|$. 
\end{remark}

\begin{exercise}{}
Show that $\C[S_3] \cong \C \times \C \times M_2(\C)$. Fun fact: using the naive approach 
to multiply matrices takes $O(n^3)$ operations, but applying this fact reduces the time complexity to 
$O(n^{2.373})$. 
\end{exercise}

We now prove a version of Cayley's theorem for group algebras. 

\begin{prop}{}
Let $G$ be a finite group with $n = |G|$. Then $G$ embeds into $\GL_n(k[G])$. 
\end{prop}
\begin{pf}
Let $G$ act on $k[G]$ by left multiplication. That is, for $g \in G$, define 
\[ L_g : k[G] \to k[G] : \sum_{h\in G} \alpha_h \cdot h \mapsto \sum_{h\in G} \alpha_h \cdot (g \cdot h). \]
Observe that for $g_1, g_2 \in G$, we have 
\[ L_{g_1} \circ L_{g_2}(x) = L_{g_1}(L_{g_2}(x)) = L_{g_1}(g_2 \cdot x) = g_1 \cdot (g_2 \cdot x)
= (g_1 \cdot g_2) \cdot x = L_{g_1g_2}(x). \] 
For the second equality, we can think of $G$ as sitting inside $k[G]$ by identifying 
$g \in G$ with $1 \cdot g \in k[G]$, so we simply have multiplication in the group algebra. 
Hence, we see that the map $L : G \to \GL_n(k[G]) : g \mapsto L_g$ is a group homomorphism. Finally, if 
$L_g$ is the identity matrix, then $g \cdot x = x$ for all $x \in k[G]$. This implies that 
$g \cdot 1 = 1$ and so $g = 1$, so $\ker L = \{1\}$. Thus, $L$ is injective and is the desired embedding.
\end{pf}

Later in the course, we will prove the following important theorem. In short, it states that if 
$G$ is a finite group and $k$ is an algebraically closed field of characteristic zero, then $k[G]$ is
isomorphic to a finite direct product of matrix rings over $k$. The isomorphism and these matrix rings
will completely determine the representation of the group $G$. 

\begin{theo}{}
Let $G$ be a finite group and let $k$ be an algebraically closed field of characteristic zero. 
Then we have 
\[ k[G] \cong \prod_{i=1}^s M_{n_i}(k), \]
where
\begin{enumerate}[(1)]
    \item $s$ is the number of conjugacy classes of $G$;
    \item $n_1^2 + n_2^2 + \cdots + n_s^2 = |G|$;
    \item $|\{i : n_i = 1\}| = |G/G'|$, where $G'$ denotes the commutator subgroup; and 
    \item $n_i \mid |G|$ for all $1 \leq i \leq s$. 
\end{enumerate}
\end{theo}

As a corollary of this theorem, one can prove Exercise 1.5 by noting that $\C[S_3]$ has three 
conjugacy classes: $\{(1)\}$, $\{(12), (13), (23)\}$, and $\{(123), (132)\}$. 

To finish off the lecture, we give one more interesting linear algebra fact. 

{\sc Fact.} If $q = p^j$ where $p$ is prime and $j \geq 1$, then 
\[ \lvert\GL_n(\F_q)\rvert = \prod_{i=0}^{n-1} (q^n - q^i). \]
It is not hard to see why. Let $A$ be an invertible $n \times n$ matrix, and note that $A$ must 
have linearly independent columns. For the first column, say $v_1$, we have 
$q^n - 1$ choices as we can pick any vector except the zero vector. For the second column, 
we can choose any vector except those in the span of $v_1$, which yields $q^n - q$ choices. 
One can repeat this argument to obtain the result. 

\section{$k$-algebras, modules, and Schur's lemma (09/10/2021)}

Recall that for a ring $R$, the {\bf center} of $R$ is the set 
\[ Z(R) = \{z \in R : zr = rz \text{ for all } r \in R\}. \]
Note that $Z(R)$ is a commutative subring of $R$. We can also write $[z, r] = 0$ to 
denote that $zr = rz$. 

\begin{defn}{}
Let $k$ be a field. We say that $R$ is a {\bf $k$-algebra} if 
\begin{enumerate}[(a)]
    \item $R$ is a ring; and 
    \item there exists a homomorphism $\phi : k \to Z(R)$ sending $1_k$ to $1_R$ (we assume that $\phi$ is injective).
\end{enumerate}
\end{defn}

Notice that if we identify $k$ with $\phi(k) \subseteq R$, we have a ``copy'' of $k$ in $R$. This 
means that $R$ is a $k$-vector space in addition to being a ring. 

\begin{defn}{}
Let $k$ be a field, and let $R$ and $S$ be $k$-algebras.
A {\bf $k$-algebra homomorphism} is a ring homomorphism $\psi : R \to S$ such that $\psi(\lambda) = \lambda$ for all 
$\lambda \in k$. 
\end{defn}

We also have the notion of a module. A module over a ring is a generalization of a 
vector space over a field. 

\begin{defn}{}
A {\bf (left) $R$-module $M$} is an abelian group $(M, +)$ equipped with a map 
\[ \cdot : R \times M \to M \] 
such that for all $r, s \in R$ and $m, n \in M$, we have 
\begin{enumerate}[(a)]
    \item $(r + s) \cdot m = r \cdot m + s \cdot m$; 
    \item $r \cdot (m + n) = r \cdot m + r \cdot n$;
    \item $(r \cdot s) \cdot m = r \cdot (s \cdot m)$; and 
    \item $1 \cdot m = m$.
\end{enumerate}
\end{defn}

\begin{exmp}{}
Let $R = M_n(\C)$ and $M = \C^{n \times 1}$, the set of column vectors of length $n$. One can check that 
$M$ is a (left) $R$-module equipped with the operation of matrix-vector multiplication.
\end{exmp}

\begin{exmp}{}
Left ideals are $R$-modules by left multiplication. In particular, $R$ itself is a left $R$-module. If 
$L$ is a left ideal of $R$ (and we write $L \trianglelefteq_\ell R$), then $L$ is a 
submodule of $R$ as a left $R$-module.
\end{exmp}

\begin{remark}{}
If $M_1$ and $M_2$ are $R$-modules, we define the set 
\[ \Hom_R(M_1, M_2) = \{f : M_1 \to M_2 \mid f \text{ is $R$-linear}\}. \]
That is, we have $f(r \cdot m_1 + m_2) = r \cdot f(m_1) + f(m_2)$ for all $r \in R$ and $m_1, m_2 \in M_1$. 
In the case that $M_1 = M_2 = M$, we write 
\[ \Hom_R(M_1, M_2) = \End_R(M), \]
the endomorphisms from $M$ to itself. Note that $\End_R(M)$ is a ring with composition as 
the multiplication operation, as the set of linear transformations from a vector space to itself 
forms a ring. 
\end{remark}

{\sc Fact.} If $R$ is a $k$-algebra and $M$ is a left $R$-module, then $M$ is a $k$-vector space. 

In a sense, this is clear. We know that $M$ already has scalar multiplication by $R$, and we have a 
copy of $k$ sitting inside of $R$, so if we restrict $R$ to $k$, we obtain scalar multiplication by $k$.

\begin{defn}{}
A left $R$-module $M$ is {\bf simple} if $M \neq (0)$, and $(0)$ and $M$ are the only submodules of $M$.
\end{defn}

\begin{exmp}{}
Let $R = \Z$ and $M = \Z/p\Z$ for a prime $p$. Suppose that $N \subseteq M$ is a submodule with 
$N \neq (0)$. Then there exists $i \in \{1, \dots, p-1\}$ such that the coset $[i]_p$ is in $N$. 
But this implies that $[i]_p^{-1} \cdot [i]_p = [1]_p \in N$. It follows that $N = \Z/p\Z$ and hence 
$M$ is simple. 
\end{exmp}

\begin{exercise}{}
Let $R = M_n(\C)$ and $M = \C^{n\times 1}$. Show that $M$ is simple. (Hint: If we are given a 
non-zero vector, then we can extend it to a basis, and we can always find a linear transformation to 
send a basis wherever we like.)
\end{exercise}

\begin{defn}{}
A left ideal $L$ of $R$ is a {\bf maximal left ideal} if 
\begin{enumerate}[(a)]
    \item $L \subsetneq R$; and
    \item there does not exist a left ideal $L'$ such that $L \subsetneq L' \subsetneq R$. 
\end{enumerate} 
\end{defn}

\begin{exercise}{}
Let $R$ be a non-zero ring (that is, $0_R \neq 1_R$). If $L$ is a proper left ideal of $R$, then 
there exists a maximal left ideal $M$ such that $L \subseteq M$. In particular, taking 
$L = (0)$ shows that a maximal left ideal always exists. 
\end{exercise}

{\sc Fact.} If $R$ is a ring and $M$ is a simple left $R$-module, then $M$ is isomorphic to 
$R/L$ (as $R$-modules), where $L$ is a maximal left ideal. 

We give a sketch of the proof of this fact. First, pick $m_0 \in M \setminus \{0\}$. Consider 
$R$ as a left $R$-module over itself (by left multiplication), and define 
\begin{align*}
    \Phi : R &\to M \\ r &\mapsto r \cdot m_0. 
\end{align*}
\begin{enumerate}[(1)]
    \item Check that $\Phi$ is an $R$-module homomorphism. 
    \item Show that $\ker\Phi = L$ is a left ideal.
    \item Show that $\im\Phi$ is a non-zero submodule of $M$, and hence must be equal to $M$ 
    (as $M$ is simple).
\end{enumerate}
By the first isomorphism theorem, it follows that $R/\ker\Phi = R/L$ is isomorphic to $M = \im\Phi$. 

Finally, why must $L$ be maximal? Similarly to the correspondence for groups and rings, we also have 
correspondence for modules. Indeed, for a module $M$ and a submodule $N \trianglelefteq M$, there 
is a bijection between the submodules of $M/N$ and the submodules of $M$ that contain $N$. 

We have the canonical projection 
\begin{align*}
    \pi : M &\to M/N \\ m &\mapsto m + N.
\end{align*} 
For a submodule $Q$ of $M/N$, notice that 
\[ \pi^{-1}(Q) = \{m \in M : \pi(m) \in Q\} \] 
is a submodule of $M$ which contains $N$, since 
\[ N = \pi^{-1}(0) \subseteq \pi^{-1}(Q). \]
Now, there is a correspondence between submodules of $M$ and the left ideals of $R$ containing $L$. 
Since $M$ is simple, it has two submodules. Thus, there are two left ideals of $R$ containing $L$, namely $L$ and $R$. Therefore, there is no ideal $L'$ such that $L \subsetneq L' \subsetneq R$, so $L$ is 
maximal.

As a corollary of the above fact, we obtain the following. 

\begin{cor}{}
If $R$ is a finite dimensional $k$-algebra and $M$ is a simple left $R$-module, then $M$ is a 
finite dimensional $k$-vector space.
\end{cor}

\begin{pf}
We showed that there was a surjection $\Phi : R \to M : r \mapsto r \cdot m_0$
for $m_0 \neq 0$. The fact that $\Phi$ is an $R$-module homomorphism implies that it is $k$-linear, 
as there is a copy of $k$ sitting inside of $R$. Since $R$ is finite dimensional, it follows that 
$M$ is also finite dimensional.
\end{pf}

For the remainder of this lecture, we will turn to proving Schur's lemma. 

\begin{defn}{}
A {\bf division ring} $\Delta$ is a ring in which every non-zero element $a \in \Delta$ 
has a multiplicative inverse $b \in \Delta$ (that is, $ab = ba = 1$).
\end{defn}

\begin{exmp}{}
Consider the quaternions 
\[ \mathbb{H} = \{a + ib + jc + kd : a, b, c, d \in \R\} \]
with properties 
$i^2 = j^2 = k^2 = 1$, $ij = k$, and $ji = -k$. Check that $\mathbb{H}$ is a division ring, 
but not a field.
\end{exmp}

\begin{theo}[Schur's lemma]{}
Let $R$ be a ring and let $M$ be a simple left $R$-module. Then $\End_R(M)$ is a division ring.
\end{theo}
\begin{pf}
We already know that $\End_R(M)$ is a ring, so we only need to show that every non-zero element 
has an inverse. Let $0 \neq f \in \End_R(M)$, which is a linear map $f : M \to M$. Notice that 
$\ker f$ is a submodule of $M$. Since $M$ is simple, $\ker f$ is either $(0)$ or $M$. The latter 
is impossible as this would mean that $f = 0$, so we have $\ker f = (0)$, so $f$ is injective. 
Similarly, $\im f$ is a submodule of $M$ and $\im f \neq (0)$ since $f \neq 0$, so 
$\im f = M$ since $M$ is simple. This shows that $f$ is surjective. Therefore, $f$ has a 
set theoretic inverse $g$. We now show that $g \in \End_R(M)$; that is, 
\[ g(r \cdot m_1 + m_2) = r \cdot g(m_1) + g(m_2) \]
for all $r \in R$ and $m_1, m_2 \in M$. But $f$ is bijective, so this is equivalent to showing that 
\[ f(g(r \cdot m_1 + m_2)) = f(r \cdot g(m_1) + g(m_2)). \]
This is indeed the case; we have 
\[ f(g(r \cdot m_1 + m_2)) = r \cdot m_1 + m_2 = r \cdot f \circ g(m_1) + f \circ g(m_2) = f(r \cdot g(m_1) + g(m_2)). \qedhere \]
\end{pf}

\section{Endomorphism ring, Jacobson density theorem (09/13/2021)}
Recall the setting from before: we had a ring $R$, a simple left $R$-module $M$,
and $\Delta := \End_R(M)$, which we showed was a division ring. 

\begin{prop}{}
\begin{enumerate}[(a)]
    \item If $R$ is a $k$-algebra for a field $k$, then $\Delta := \End_R(M)$ is also a $k$-algebra. 
    \item If $k$ is algebraically closed and $R$ is a finite-dimensional $k$-algebra, then $\Delta 
    \cong k$. 
\end{enumerate}
\end{prop}
\begin{pf}~
\begin{enumerate}[(a)]
    \item For each $\lambda \in k$, define the map 
    \begin{align*}
        \Phi_\lambda : M &\to M, \\ \lambda &\mapsto \lambda \cdot m. 
    \end{align*}
    By $\lambda \cdot m$, we mean that we have a copy $k \subseteq Z(R) \subseteq R$, and $M$ 
    is an $R$-module and hence a $k$-vector space. Note that 
    \begin{align*}
        \Phi_\lambda(r \cdot m_1 + m_2) &= \lambda \cdot (r \cdot m_1 + m_2) \\
        &= \lambda \cdot r \cdot m_1 + \lambda \cdot m_2 \\
        &= r \cdot \lambda \cdot m_1 + \lambda \cdot m_2 & (\text{since $k \subseteq Z(R)$}) \\
        &= r \cdot \Phi_\lambda(m_1) + \Phi_\lambda(m_2).
    \end{align*}
    Therefore, $\Phi_\lambda$ is $R$-linear and so $\Phi_\lambda \in \End_R(M)$. However, it is not 
    enough to show that each $\Phi_\lambda$ is in $\Delta = \End_R(M)$. We also require that our map 
    $\lambda \mapsto \Phi_\lambda$ is a map $k \to Z(\Delta)$; in other words, we also need 
    $\Phi_\lambda \in Z(\Delta)$. Let $\psi \in \Delta$, and observe that 
    \[ \Phi_\lambda \circ \psi(m) = \lambda \cdot \psi(m) = \psi(\lambda \cdot m) = \psi \circ 
    \Phi_\lambda(m), \]
    where the second equality is because the $R$-linearity of $\psi$ implies $k$-linearity.
    
    \item Recall that if $R$ is a finite dimensional $k$-algebra, then $M$ is a finite dimensional 
    $k$-vector space (see Corollary 2.12). Suppose that $\dim_k M =: n < \infty$. Then we have 
    \begin{align*}
        \Delta = \End_R(M) &\subseteq \End_k(M) & (\text{$R$-linearity imposes more conditions than 
        $k$-linearity}) \\
        &\cong\End_k(k^n) & (\text{$M$ is an $n$-dimensional $k$-vector space}) \\
        &\cong M_n(k) & (\text{$k$-linear maps $k^n \to k^n$ are the $n \times n$ matrices})
    \end{align*}
    and thus $\dim_k \Delta = m \leq n^2 < \infty$ for some $m \in \Z$. 
    
    We now show that $\Delta \cong k$. Indeed, pick $a \in \Delta$. Notice that $a$ commutes 
    with all elements of $k$ since $k \subseteq Z(\Delta)$, where we can identify $k$ with the set 
    $\{\Phi_\lambda : \lambda \in k\}$. Consider 
    \[ k \subseteq k(a) \subseteq \Delta, \]
    where $k(a)$ is the field formed from adjoining $a$ to $k$; it is a field because $\Delta$ 
    is a division ring and hence $a$ is invertible. Thus, $\dim_k k(a) \leq \dim_k \Delta = m < \infty$. 
    Therefore, $\{1, a, a^2, \dots, a^m\}$ is a linearly dependent set over $k$, so there 
    exist elements $c_0, c_1, \dots, c_m \in k$, not all zero, such that 
    \[ c_0 + c_1a + \cdots + c_m a^m = 0. \]
    But $k$ is algebraically closed, so $a \in k$. This implies that $\Delta \subseteq k$, and hence 
    $\Delta \cong k$. \qedhere 
\end{enumerate}
\end{pf}

\begin{exercise}{}
Let $\Delta$ be a division ring and let $M$ be a left $\Delta$-module. Then $M$ has a basis 
$B \subseteq M$. That is, there do not exist $\delta_1, \dots, \delta_m \in \Delta$ such that 
\[ \delta_1 b_1 + \cdots + \delta_m b_m = 0 \]
for distinct $b_1, \dots, b_m \in B$, and for all $m \in M$, we can write 
\[ m = \sum_{b \in B} \delta_b b \]
where $\delta_b = 0$ for all but finitely many $b \in B$. (Hint: Use Zorn's lemma.)
\end{exercise}

For this reason, instead of calling it a left $\Delta$-module, we can call it a 
{\bf left $\Delta$-vector space}.  

\begin{remark}{}
Let $R$ be a ring, let $M$ be a simple left $R$-module, and let $\Delta = \End_R(M)$. Then $M$ 
is a left $\Delta$-vector space. 
\end{remark}
\begin{pf}
Recall that $\delta \in \Delta$ is an $R$-linear map from $M$ to itself. We can consider the operation 
\begin{align*}
    \Delta \times M &\to M, \\
    (\delta, m) &\mapsto \delta \cdot m := \delta(m). 
\end{align*}
It is straightforward to see that 
\begin{itemize}
    \item $(\delta_1 + \delta_2) \cdot m = \delta_1 \cdot m + \delta_2 \cdot m$, 
    \item $\delta \cdot (m_1 + m_2) = \delta \cdot m_1 + \delta \cdot m_2$, 
    \item $\delta \cdot (\delta \cdot m) = (\delta_1 \cdot \delta_2) \cdot m$, and 
    \item $\id \cdot \,m = m$,
\end{itemize} so $M$ is a left $\Delta$-vector space. 
\end{pf}

\begin{exmp}{}
Let $R = M_n(\C)$ and recall that $M = \C^{n\times 1}$ is a simple left $R$-module. We have 
\[ \Delta = \C = \{\Phi_\lambda : \lambda \in \C\}, \]
where each $\Phi_\lambda(m) = \lambda \cdot m$ for each $\lambda \in \C$. One can show that 
$\Phi(Av) = A \cdot \Phi(v)$ for all $A \in M_n(\C)$ and $v \in \C^{n\times 1}$ only when 
$\Phi = \Phi_\lambda$ for some $\lambda \in \C$. 
\end{exmp}

We now state the Jacobson Density Theorem, which we will prove in the next lecture. 

\begin{theo}[Jacobson Density Theorem]{}
Let $R$ be a ring, let $M$ be a simple left $R$-module, and let $\Delta = \End_R(M)$. Then we have a 
ring $\End_\Delta(M)$ and a map 
\begin{align*}
    \Phi : R &\to \End_\Delta(M) \\
    r &\mapsto \Phi_r : M \to M \\
    & \hspace{1.33cm} m \mapsto r \cdot m.
\end{align*}
Moreover, we have the following properties. 
\begin{enumerate}[(1)]
    \item The map $\Phi$ is a ring homomorphism and if $R$ is a $k$-algebra, then $\Phi$ is a 
    $k$-algebra homomorphism.
    \item The kernel of $\Phi$ is the annihilator of $M$; that is, 
    \[ \ker\Phi = \{r \in R : r \cdot m = 0 \text{ for all } m \in M\}. \]
    \item {\bf Density:} If $m_1, \dots, m_n \in M$ are left linearly independent over $\Delta$ and 
    $w_1, \dots, w_n \in M$ are arbitrary elements, then there exists $r \in R$ such that 
    \[ \Phi_r(m_i) = w_i \]
    for all $1 \leq i \leq n$ (in particular, we can send finite linear combinations wherever we like).
\end{enumerate}
\end{theo}

For now, let's see why the maps $\Phi_r$ are $\Delta$-linear so that $\Phi_r \in \End_\Delta(M)$
for each $r \in R$. 
Let $\delta \in \Delta$ and $m_1, m_2 \in M$. Then we have 
\begin{align*}
    \Phi_r(\delta \cdot m_1 + m_2) &= r \cdot (\delta \cdot m_1 + m_2) \\
    &= r \cdot (\delta \cdot m_1) + r \cdot m_2 & (\text{multiplication by $R$ is linear}) \\
    &= r \cdot \delta(m_1) + r \cdot m_2 & (\text{we have $\delta \in \Delta$}) \\
    &= \delta(r \cdot m_1) + r \cdot m_2 & (\text{$\delta$ is an $R$-linear map}) \\
    &= \delta \cdot \Phi_r(m_1 ) + \Phi_r(m_2),
\end{align*}
so $\Phi_r$ is $\Delta$-linear as desired. 