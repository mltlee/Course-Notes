\section{October 6, 2021}
Last time, we were looking at a ring 
\[ R = M_{n_1}(k) \times \cdots \times M_{n_s}(k). \] 
We saw that $V_i = k^{n_i \times 1}$ is a left $R$-module with the action 
\[ (A_1, \dots, A_s) \cdot v := A_i v \in V_i \] 
for $(A_1, \dots, A_s) \in R$ and $v \in V_i$. Moreover, $V_i$ is a 
simple left $R$-module with $V_i \ncong V_j$ when $i \neq j$. 

Now, we observe that 
\begin{align*} 
    \Ann_R(V_i) &= \{(A_1, \dots, A_{i-1}, 0, A_{i+1}, \dots, A_s) : 
    A_j \in M_{n_j}(k)\} \\
    &= M_{n_1}(k) \times \cdots \times M_{n_{i-1}}(k) \times \{0\} 
    \times M_{n_{i+1}}(k) \times \cdots \times M_{n_s}(k). 
\end{align*} 
In particular, if $i \neq j$, then $\Ann_R(V_i) \ncong \Ann_R(V_j)$. 
Assignment 3 Question 2 then gives us another proof of the fact that 
$V_i \ncong V_j$. 

\begin{notation} 
    If $R$ is a ring, we will write ${}_R R$ to mean $R$ considered as a left 
    $R$-module. (Similarly, the notation $R_R$ means $R$ taken as a right 
    $R$-module.)
\end{notation}

Let $A, B \in M_n(k)$. We can think of $B$ as $n$ columns concatenated 
with each other. In particular, if $v_1, \dots, v_n$ are the columns of $B$, 
then the columns of $AB$ are given by $Av_1, \dots, Av_n$. 
Therefore, if $R = M_n(k)$ and $V = k^{n \times 1}$, then 
\[ {}_R R \cong \underbrace{V \oplus \cdots \oplus V}_{n} \] 
with the $R$-module isomorphism 
\[ B \mapsto (v_1, \dots, v_n), \] 
where as before, $v_1, \dots, v_n$ are the columns of $B$. 

More generally, if $R = M_{n_1}(k) \times \cdots \times M_{n_s}(k)$ and 
$V_i = k^{n_i \times 1}$, then we have 
\[ (A_1, \dots, A_s) \cdot v = A_i v \] 
for $(A_1, \dots, A_s) \in R$ and $v \in V_i$, so we see that 
\[ {}_R R \cong \underbrace{V_1 \oplus \cdots \oplus V_1}_{n_1} 
\oplus \underbrace{V_2 \oplus \cdots \oplus V_2}_{n_2} \oplus \cdots 
\oplus \underbrace{V_s \oplus \cdots \oplus V_s}_{n_s} \] 
with the explicit $R$-module isomorphism 
\[ (B_1, \dots, B_s) \in {}_R R \mapsto 
(\underbrace{v_{11}, \dots, v_{1n_1}}_{\in V_1^{n_1}}, 
\dots, \underbrace{v_{s1}, \dots, v_{sn_s}}_{\in V_s^{n_s}}), \] 
where $v_{i1}, \dots, v_{in_i}$ are the columns of $B_i$ for each 
$i = 1, \dots, s$. Indeed, we can see that this is an $R$-module isomorphism 
because for $(A_1, \dots, A_s) \in R$ and $(B_1, \dots, B_s) \in {}_R R$, the 
columns of $A_i B_i$ are $Av_{i1}, \dots, Av_{in_i}$, and on the other hand, 
we have 
\[ (A_1, \dots, A_s) \cdot (v_{11}, \dots, v_{1n_1}, \dots, 
v_{s1}, \dots, v_{sn_s}) = 
(\underbrace{A_1 v_{11}, \dots, A_1 v_{1n_1}}_{A_1B_1}, 
\dots, \underbrace{A_s v_{s1}, \dots, A_s v_{sn_s}}_{A_sB_s}). \] 
From this discussion, we obtain the key fact that if $R = M_{n_1}(k) \times \cdots \times 
M_{n_s}(k)$ and $V_i = k^{n_i \times 1}$, then 
\[ {}_R R \cong V_1^{n_1} \oplus \cdots \oplus V_s^{n_s} \] 
as $R$-modules. Now, we are almost in the position to finish proving Theorem 12.4.

\begin{remark} Let $N$ and $M_\alpha$ for $\alpha \in X$ be left $R$-modules. 
    \begin{enumerate}[(1)]
        \item From Exercise 4 on Assignment 2, we have 
        \[ \Hom_R \left( \bigoplus_{\alpha \in X} M_\alpha, N \right) 
        \cong \prod_{\alpha \in X} \Hom_R(M_\alpha, N). \] 
        \item If $N \neq (0)$ is a left $R$-module, then 
        \[ \Hom_R(R, N) \neq (0). \] 
        Notice that for all $n \in N$, there exists a map 
        \begin{align*} \varphi_n : {}_R R &\to N \\ r &\mapsto r \cdot n. \end{align*} 
        For $a \in R$ and $r \in {}_R R$, we have $\varphi_n(ar) = ar \cdot n 
        = a \cdot (rn) = a \cdot \varphi_n(r)$, so $\varphi_n$ is $R$-linear. 
        Moreover, when $n \neq 0$, we have $\varphi_n(1) = n \neq 0$, so 
        $\varphi_n \neq 0$. 
    \end{enumerate}
\end{remark}

\begin{exercise}
    Define the map 
    \begin{align*} 
        e : \Hom_R({}_R R, N) &\to N \\ \varphi &\mapsto \varphi(1). 
    \end{align*}
    If $R$ is commutative, show that $e$ is an isomorphism of $R$-modules. 
\end{exercise}

Now, we can prove part (1) of Theorem 12.4. 

\begin{thm}
    Let $R \cong M_{n_1}(k) \times \cdots \times M_{n_s}(k)$ and $V_i 
    \cong k^{n_i \times 1}$. If $N$ is a simple left $R$-mdoule, then there 
    exists some $i = 1, \dots, s$ such that $N \cong V_i$ as $R$-modules. 
    In particular, $R$ has exactly $s$ simple left $R$-modules up to 
    isomorphism. 
\end{thm}
\begin{pf}
    Let $N$ be a simple $R$-module. Then we have 
    \[ \Hom_R({}_R R, N) \neq (0) \] 
    since $N \neq (0)$. In particular, we obtain 
    \[ \Hom_R({}_R R, N) \cong \Hom_R(V_1^{n_1} \oplus \cdots \oplus V_s^{n_s}, 
    N) \cong \prod_{i=1}^s \Hom_R(V_i, N)^{n_i}. \] 
    Since $\Hom_R({}_R R, N) \neq (0)$, there exists $i = 1, \dots, s$ 
    such that $\Hom_R(V_i, N) \neq (0)$. Therefore, there exists a 
    nonzero $R$-modulo homomorphism $f : V_i \to N$. Now, $V_i$ is simple, 
    and since $\ker(f)$ is a submodule of $V_i$, we either have $\ker(f) 
    = (0)$ or $\ker(f) = V_i$. But $f$ is nonzero, so we have $\ker(f) = V_i$. 
    Similarly, $\im(f)$ is a submodule of $N$. Again, $f$ is nonzero, so 
    $\im(f) \neq (0)$, and since $N$ is simple, $\im(f) = N$. 

    As in the proof of Schur's lemma (Theorem 2.15), the set theoretic 
    inverse of $f$ is an $R$-module homomorphism with $f \circ f^{-1} 
    = \id_N$ and $f^{-1} \circ f = \id_{V_i}$. Therefore, $f : V_i \to N$ is 
    an $R$-module isomorphism. 
\end{pf}

Towards proving (2) of Theorem 12.4, we'll first prove the following fact. 

\begin{prop}
    Every $R$-module $M$ satisfies 
    \[ M \cong R^{\oplus X}/L \] 
    where $R^{\oplus X} = \bigoplus_{x \in X} R$, and $L$ is a submodule of 
    $R^{\oplus X}$. 
\end{prop}
\begin{pf}
    For each $m \in M$, we will define a formal symbol $e_m$. 
    (We can think of this as a vector $e_m = (0, \dots, 0, 1, 0, \dots, 0)$ 
    where $1$ is in the $m$-th position, but this choice doesn't always work.)
    Define an $R$-module homomorphism $\Psi : \bigoplus_{m \in M} Re_m \to M$ by 
    \[ \sum_{m \in M} r_m e_m \mapsto \sum_{m \in M} r_m \cdot m. \] 
    We note that these must be finite sums in order to make sense. When $r \in R$, 
    we have 
    \[ \Psi\left(r \cdot \sum_{m \in M} r_m e_m \right) 
        = \Psi\left( \sum_{m \in M} r \cdot r_m e_m \right) 
        = \sum_{m \in M} r \cdot r_m \cdot m 
        = r \cdot \sum_{m \in M} r_m \cdot m 
        = r \cdot \Psi \left( \sum_{m \in M} r_m e_m \right). \] 
    Therefore, $\Psi$ is $R$-linear. Moreover, $\Psi(1 \cdot e_m) = m$, so 
    $\Psi$ is onto. Let $L = \ker\Psi$. By the first isomorphism theorem, 
    we obtain $\bigoplus_{m \in M} Re_m/L \cong \im\Psi = M$, 
    and we can take  $\bigoplus_{m \in M} Re_m$ as $R^{\oplus X}$. 
\end{pf}

\section{October 8, 2021}

\begin{defn}
    Let $R$ be a ring. We say that a left $R$-module $M$ is {\bf semisimple} 
    of $M$ can be written as the direct sum of simple submodules. We say that 
    $R$ is a {\bf (left) semisimple ring} if every left $R$-module is semisimple.
\end{defn}

\begin{exmp}
    An example of a non-semisimple ring is $\Z$. The simple modules are 
    isomorphic to $\Z/p\Z$ for primes $p$, and we notice that 
    $\Z/4\Z$ cannot be written as the direct sum of simple submodules. 
\end{exmp}

\begin{exmp}
    Let $k$ be a field. A $k$-module is simply a vector space, and a 
    submodule is a subspace. In particular, a simple $k$-module $V$ is a vector 
    space whose only subspaces are $(0)$ and $V$ itself. Therefore, $V$ 
    is a simple $k$-module if and only if $\dim_k V = 1$. Now, if 
    $W$ is a $k$-vector space, then we can write 
    \[ W = \bigoplus_{\alpha \in {\cal B}} kv_\alpha, \] 
    since every vector space has a basis ${\cal B}$. Hence, $k$ is a semisimple ring. 
\end{exmp}

Notice that the fact that every vector space has a basis relies on Zorn's lemma. 
The following theorem generalizes this example, and at some point, we'll have to use 
Zorn's lemma to prove it. 

\begin{thm}
    Let $R$ be a ring. Then $R$ is a (left) semisimple ring if and only if 
    ${}_R R$ is a semisimple $R$-module. 
\end{thm}

To prove this, we'll use the following lemma. 

\begin{lemma}
    Let $R$ be a ring. 
    \begin{enumerate}[(1)]
        \item If $\{M_\alpha\}_{\alpha \in X}$ is a family of left $R$-modules 
        where each $M_\alpha$ is semisimple, then $\bigoplus_{\alpha \in X} 
        M_\alpha$ is semisimple. 
        \item If $M$ is a semisimple left $R$-module and $L \trianglelefteq M$ 
        is a submodule of $M$, then $M/L$ is also semisimple. 
    \end{enumerate}
\end{lemma}

First, we'll show that Theorem 14.4 easily follows from Lemma 14.5. 

{\sc Proof of Theorem 14.4.} $(\Rightarrow)$ If every left $R$-module is semisimple, 
then certainly ${}_R R$ is semisimple. 

$(\Leftarrow)$ Let $M$ be a left $R$-module. We saw in Proposition 13.5 that 
$M \cong R^{\oplus X}/L$ where $L \trianglelefteq \bigoplus_{x \in X} R$. 
If ${}_R R$ is semisimple, then 
\[ \bigoplus_{x \in X} {}_R R = \bigoplus_{x \in X} R \] 
is semisimple by taking $M_\alpha = {}_R R$ for each $\alpha \in X$ and 
applying (1) of Lemma 14.5. It follows from (2) that $M \cong 
\bigoplus_{x \in X} R/L$ is semisimple. \qed 

Now, it only remains to prove Lemma 14.5. 

{\sc Proof of Lemma 14.5.}~
\begin{enumerate}[(1)]
    \item Write each $M_\alpha$ as a direct sum of simple $R$-modules 
    \[ M_\alpha = \bigoplus_{\beta \in X_\alpha} V_{\alpha, \beta}. \] 
    It follows that 
    \[ \bigoplus_{\alpha \in X} M_\alpha 
    = \bigoplus_{\alpha \in X} \bigoplus_{\beta \in X_\alpha} V_{\alpha, \beta}
    = \bigoplus_{(\alpha, \beta)\,:\,\alpha\in X,\,\beta \in X_\alpha} 
    V_{\alpha, \beta} \] 
    is the direct sum of simple modules, so it is semisimple. 
    
    \item Since $M$ is semisimple, we can write 
    \[ M = \bigoplus_{\alpha \in X} V_\alpha, \] 
    where each $V_\alpha$ is simple. Let $\pi : M \to M/L$ be the canonical 
    projection. Consider $\pi(V_\alpha) \subseteq M/L$. We claim that 
    either $\pi(V_\alpha) \cong V_\alpha$ or $\pi(V_\alpha) = (0)$. 
    First, we restrict $\pi$ to $V_\alpha$ to obtain the map 
    \[ \pi|_{V_\alpha} : V_\alpha \to \pi(V_\alpha), \] 
    which is a surjective $R$-module homomorphism. Since $\ker(\pi|_{V_\alpha})$ 
    is a submodule of $V_\alpha$, we either have $\ker(\pi|_{V_\alpha}) = (0)$ 
    or $\ker(\pi|_{V_\alpha}) = V_\alpha$ by the simplicity of $V_\alpha$. 
    In the first case, $\pi|_{V_\alpha}$ is a bijection, so $\pi(V_\alpha) 
    \cong V_\alpha$. In the second case, we obtain $\pi(V_\alpha) = (0)$. 

    Let $Y = \{\alpha \in X : \pi(V_\alpha) \cong V_\alpha\} \subseteq X$. 
    We claim that 
    \[ M/L = \sum_{\alpha \in Y} \pi(V_\alpha). \] 
    Since $\pi : M \to M/L$ is onto and $M = \sum_{\alpha \in X} V_\alpha$, we have 
    \[ M/L = \pi(M) = \pi\left( \sum_{\alpha \in X} V_\alpha \right) 
    = \sum_{\alpha \in X} \pi(V_\alpha) = \sum_{\alpha \in Y} \pi(V_\alpha), \] 
    where the last equality is because $\pi(V_\alpha) = (0)$ for all 
    $\alpha \in X \setminus Y$, which contributes nothing to the sum. 

    Note that the above sum is not necessarily a direct sum. This is where 
    we'll use Zorn's lemma to show that it is isomorphic to a direct sum of 
    simples. Let 
    \[ {\cal S} = \left\{Z \subseteq Y : \sum_{\alpha \in Z} \pi(V_\alpha) 
    \text{ is a direct sum} \right\}. \] 
    First, we note that ${\cal S} \neq \varnothing$ since empty sums are 
    by definition direct sums, so $\varnothing \in {\cal S}$. Let 
    $\{Z_\gamma\}_{\gamma \in \Gamma}$ be a chain in ${\cal S}$ ordered by 
    inclusion, where $\Gamma$ is a totally ordered set. We must show that 
    \[ Z := \bigcup_{\gamma \in \Gamma} Z_\gamma \in {\cal S}, \] 
    which will be an upper bound in the chain, in which case we can apply 
    Zorn's lemma to ${\cal S}$. So suppose towards a contradiction that 
    $Z \notin {\cal S}$. Then the sum $\sum_{\alpha \in Z} \pi(V_\alpha)$ 
    is not a direct sum. Therefore, there is a finite set of elements 
    $\alpha_1, \dots, \alpha_s \in Z$ such that 
    \[ \pi(V_{\alpha_1}) + \cdots + \pi(V_{\alpha_s}) \] 
    is not direct. There exists some $\gamma \in \Gamma$ such that 
    $\alpha_1, \dots, \alpha_s \in Z_\gamma$, which would imply that 
    $\sum_{\alpha \in Z_\gamma} \pi(V_\alpha)$ is not a direct sum, a contradiction. 
    Therefore, $Z \in {\cal S}$, and by Zorn's lemma, there exists a maximal 
    element $Y_0 \in {\cal S}$ with $Y_0 \subseteq Y$. 

    Finally, we'll show that $N = \sum_{\alpha \in Y_0} \pi(V_\alpha)$ is 
    direct and equal to $M/L$. The fact that it is direct is immediate since 
    $Y_0 \in {\cal S}$. Recall that 
    \[ M/L = \sum_{\alpha \in Y} \pi(V_\alpha). \] 
    If $\pi(V_\alpha) \subseteq N$ for all $\alpha \in Y$, we have 
    $\sum_{\alpha \in Y} \pi(V_\alpha) \subseteq N$. This gives $M/L \subseteq N$, 
    and hence $N = M/L$. Therefore, it is enough to show that $\pi(V_\alpha) 
    \subseteq N$ for any $\alpha \in Y$. Suppose that this was not the case, 
    so $\pi(V_{\alpha_0}) \not\subseteq N$ for some $\alpha_0 \in Y$. 
    We will show that 
    \[ \pi(V_{\alpha_0}) + \bigoplus_{\alpha \in Y_0} \pi(V_\alpha) \] 
    is a direct sum, which will give a contradiction as $\{\alpha_0\} 
    \cup Y_0 \in {\cal S}$ contradicts the maximality of $Y_0$ in ${\cal S}$. 
    Note that $\pi(V_{\alpha_0})$ is simple. If $\pi(V_{\alpha_0}) = (0)$, then 
    $\pi(V_{\alpha_0}) + \bigoplus_{\alpha \in Y_0} \pi(V_\alpha)$ is a direct 
    sum. Otherwise, $\pi(V_{\alpha_0})$ is already in 
    $\bigoplus_{\alpha \in Y_0} \pi(V_\alpha)$, which means that 
    $\pi(V_{\alpha_0}) \subseteq N$, a contradiction. \qed 
\end{enumerate}