\section{Quadratic Reciprocity}\label{sec:5}

\subsection{Euler's Totient Function}\label{subsec:5.1}

\begin{defn}\label{def:5.1}
    For $n \in \N$, we define {\bf Euler's totient function} $\phi(n)$ to be 
    the number of integers $m$ such that $1 \leq m \leq n$ and $\gcd(m, n) = 1$. 
    That is, we have 
    \[ \phi(n) = \#\{1 \leq m \leq n : \gcd(m, n) = 1\}. \] 
    A {\bf reduced residue system modulo $n$} is a subset $R \subseteq \Z$ such that 
    \begin{enumerate}[(i)]
        \item $\gcd(r, n) = 1$ for each $r \in R$; 
        \item $R$ contains $\phi(n)$ elements; and 
        \item no two elements of $R$ are congruent modulo $n$. 
    \end{enumerate}
\end{defn}

\begin{thm}\label{thm:5.2}
    Let $a, n \in \N$ with $\gcd(a, n) = 1$. Then 
    \[ a^{\phi(n)} \equiv 1 \pmod n. \] 
\end{thm}
\begin{pf}
    Let $\{c_1, \dots, c_{\phi(n)}\}$ be a reduced residue system modulo $n$. 
    Since $\gcd(a, n) = 1$, $\{ac_1, \dots, ac_{\phi(n)}\}$ is also a 
    reduced residue system modulo $n$. Hence, we have 
    \[ c_1 \cdots c_{\phi(n)} \equiv ac_1 \cdots ac_{\phi(n)} \pmod n. \] 
    In particular, we see that 
    \[ c_1 \cdots c_{\phi(n)} \equiv a^{\phi(n)} c_1 \cdots c_{\phi(n)} \pmod n. \] 
    Since $c_1, \dots, c_{\phi(n)}$ are all coprime with $n$, it follows that 
    \[ a^{\phi(n)} \equiv 1 \pmod n. \qedhere \] 
\end{pf}

Notice that when $p$ is prime, we have $\phi(p) = p-1$, so we immediately 
obtain the following corollary. 

\begin{cor}[Fermat's little theorem]\label{cor:5.3}
    Let $p$ be a prime. For any $a \in \Z$ with $p \nmid a$, we have 
    \[ a^{p-1} \equiv 1 \pmod p. \] 
\end{cor}

\begin{thm}[Wilson]\label{thm:5.4}
    If $p$ is a prme, then $(p-1)! \equiv -1 \pmod p$. 
\end{thm}
\begin{pf}
    Consider the element $x^{p-1} - 1 \in (\Z/p\Z)[x]$. By Fermat's little theorem 
    and using the fact that $\Z/p\Z$ is a field, this factors as 
    \[ x^{p-1} - 1 \equiv (x-1)(x-2) \cdots (x-(p-1)) \pmod p \] 
    in $(\Z/p\Z)[x]$, as $1, 2, \dots, p-1$ are all roots. Looking at the constant 
    coefficient, we find that 
    \[ -1 \equiv (-1)(-2) \cdots (-(p-1)) \pmod p. \] 
    Therefore, we have $-1 \equiv (-1)^{p-1} (p-1)! \pmod p$. When $p = 2$, 
    the result holds since $-1 \equiv 1 \pmod 2$; otherwise, $p$ is odd, 
    so $-1 \equiv (p-1)! \pmod p$ as required. 
\end{pf}