\section{Quadratic Reciprocity}\label{sec:5}

\subsection{Euler's Totient Function}\label{subsec:5.1}

\vspace{2ex}
\begin{defn}{def:5.1}
    For $n \in \N$, we define {\bf Euler's totient function} $\phi(n)$ to be 
    the number of integers $m$ such that $1 \leq m \leq n$ and $\gcd(m, n) = 1$. 
    That is, we have 
    \[ \phi(n) = \#\{1 \leq m \leq n : \gcd(m, n) = 1\}. \] 
    A {\bf reduced residue system modulo $n$} is a subset $R \subseteq \Z$ such that 
    \begin{enumerate}[(i)]
        \item $\gcd(r, n) = 1$ for each $r \in R$; 
        \item $R$ contains $\phi(n)$ elements; and 
        \item no two elements of $R$ are congruent modulo $n$. 
    \end{enumerate}
\end{defn}

\begin{theo}[Euler]{thm:5.2}
    Let $a, n \in \N$ with $\gcd(a, n) = 1$. Then 
    \[ a^{\phi(n)} \equiv 1 \pmod n. \] 
\end{theo}
\begin{pf}
    Let $\{c_1, \dots, c_{\phi(n)}\}$ be a reduced residue system modulo $n$. 
    Since $\gcd(a, n) = 1$, $\{ac_1, \dots, ac_{\phi(n)}\}$ is also a 
    reduced residue system modulo $n$. Hence, we have 
    \[ c_1 \cdots c_{\phi(n)} \equiv ac_1 \cdots ac_{\phi(n)} \pmod n. \] 
    In particular, we see that 
    \[ c_1 \cdots c_{\phi(n)} \equiv a^{\phi(n)} c_1 \cdots c_{\phi(n)} \pmod n. \] 
    Since $c_1, \dots, c_{\phi(n)}$ are all coprime with $n$, it follows that 
    \[ a^{\phi(n)} \equiv 1 \pmod n. \qedhere \] 
\end{pf}

Notice that when $p$ is prime, we have $\phi(p) = p-1$, so we immediately 
obtain the following corollary. 

\begin{cor}[Fermat's little theorem]{cor:5.3}
    Let $p$ be a prime. For any $a \in \Z$ with $p \nmid a$, we have 
    \[ a^{p-1} \equiv 1 \pmod p. \] 
\end{cor}

\begin{theo}[Wilson's theorem]{thm:5.4}
    If $p$ is a prime, then $(p-1)! \equiv -1 \pmod p$. 
\end{theo}
\begin{pf}
    Consider the element $x^{p-1} - 1 \in (\Z/p\Z)[x]$. By Fermat's little theorem 
    and using the fact that $\Z/p\Z$ is a field, this factors as 
    \[ x^{p-1} - 1 \equiv (x-1)(x-2) \cdots (x-(p-1)) \pmod p \] 
    in $(\Z/p\Z)[x]$, as $1, 2, \dots, p-1$ are all roots. Looking at the constant 
    coefficient, we find that 
    \[ -1 \equiv (-1)(-2) \cdots (-(p-1)) \pmod p. \] 
    Therefore, we have $-1 \equiv (-1)^{p-1} (p-1)! \pmod p$. When $p = 2$, 
    the result holds since $-1 \equiv 1 \pmod 2$; otherwise, $p$ is odd, 
    so $-1 \equiv (p-1)! \pmod p$ as required. 
\end{pf}

\subsection{Quadratic Residues}\label{subsec:5.2}

\vspace{2ex}
\begin{defn}{def:5.5}
    Let $p$ be a prime. A nonzero integer $a$ coprime to $p$ that is congruent 
    to a square modulo $p$ is called a {\bf quadratic residue modulo $p$} 
    (or $\QR$ for short). If not, then $a$ is said to be a {\bf quadratic 
    nonresidue modulo $p$} (or $\NR$ for short). Moreover, we define the 
    Legendre symbol $(\frac ap)$ by 
    \[ \left( \frac ap \right) = \begin{cases} 1 & \text{if } 
        x^2 \equiv a \pmod p \text{ has a solution,} \\ 
        -1 & \text{if } x^2 \equiv a \pmod p \text{ has no solution.} \end{cases} \]
    In other words, if $(\frac ap) = 1$, then $a$ is a quadratic residue; otherwise, 
    it is a quadratic nonresidue. 
\end{defn}

\begin{remark}{remark:5.6}
    Let $p$ be an odd prime. Then there are exactly $(p-1)/2$ quadratic residues 
    modulo $p$, and exactly $(p-1)/2$ quadratic nonresidues modulo $p$.
\end{remark}
\begin{pf}
    The quadratic residues modulo $p$ are given by 
    \[ 1^2, 2^2, \dots, \left(\frac{p-1}2\right)^{\!2} \pmod p. \] 
    To see that there are exactly $(p-1)/2$ of them, it suffices to show that these 
    are all different modulo $p$. Indeed, suppose that $1 \leq b_2 \leq b_1 
    \leq (p-1)/2$ with $b_1^2 \equiv b_2^2 \pmod p$. Then we have 
    \[ (b_1 - b_2)(b_1 + b_2) \equiv 0 \pmod p, \] 
    or equivalently, $p \mid (b_1 - b_2)(b_1 + b_2)$. Since $p$ is prime, 
    at least one of $(b_1 - b_2)$ or $(b_1 + b_2)$ must be divisible by $p$. 
    Note that 
    \[ 2 = 1 + 1 \leq b_1 + b_2 \leq \frac{p-1}2 + \frac{p-1}2 = p-1, \] 
    so $p \nmid (b_1 + b_2)$, and hence $p \mid (b_1 - b_2)$. But 
    we know that $0 \leq b_1 - b_2 \leq (p-1)/2 < p$, so $b_1 = b_2$.
\end{pf}

We now consider the products between quadratic residues and nonresidues modulo $p$, 
and derive a nice property about the product of Legendre symbols. 

\newpage 
\begin{lemma}{lemma:5.7} 
    If $a_1$ and $a_2$ are quadratic residues modulo $p$, then so is $a_1a_2$. 
\end{lemma}
\begin{pf}
    Suppose that $b_1^2 \equiv a_1 \pmod p$ and $b_2^2 \equiv a_2 \pmod p$. Then 
    we obtain 
    \[ (b_1b_2)^2 \equiv b_1^2 b_2^2 \equiv a_1a_2 \pmod p. \qedhere \] 
\end{pf}

\begin{lemma}{lemma:5.8}
    If $a_1$ is a quadratic residue and $a_2$ is a quadratic nonresidue modulo $p$, 
    then $a_1a_2$ is a quadratic nonresidue modulo $p$. 
\end{lemma}
\begin{pf}
    Suppose that $b_1^2 \equiv a_1 \pmod p$. Taking the inverse, we have 
    \[ (b_1^{-1})^2 \equiv a_1^{-1} \pmod p. \] 
    Now, if $a_1a_2$ were a quadratic residue modulo $p$, then there would exist 
    an integer $b$ such that 
    \[ b^2 \equiv a_1a_2 \pmod p. \] 
    Multiplying these equations together gives 
    \[ (bb_1^{-1})^2 \equiv a_1a_2 \cdot a_1^{-1} \equiv a_2 \pmod p, \] 
    contradicting our assumption that $a_2$ is a quadratic nonresidue modulo $p$. 
\end{pf}

\begin{theo}[Quadratic residue multiplication rules]{thm:5.9}
    Let $p$ be an odd prime. Then 
    \begin{enumerate}[(i)]
        \item $\QR \times \QR = \QR$; 
        \item $\QR \times \NR = \NR$; and 
        \item $\NR \times \NR = \QR$. 
    \end{enumerate}
    In particular, if $p$ is an odd prime and $a, b \in (\Z/p\Z)^*$, then 
    \[ \left( \frac{ab}p \right) = \left( \frac ap \right) \left( \frac bp \right). \]
\end{theo}
\begin{pf}
    We have already proved (i) and (ii) from Lemma~\ref{lemma:5.7} and 
    Lemma~\ref{lemma:5.8}. Let $A$ be the subset of $(\Z/p\Z)^*$ of quadratic 
    residues, and let $B$ be the subset of $(\Z/p\Z)^*$ of quadratic nonresidues. 
    We know that $A \cap B = \varnothing$ and $A \cup B = (\Z/p\Z)^*$. 
    Moreover, from Remark~\ref{remark:5.6}, we have $|A| = |B| = (p-1)/2$. 
    Notice that if $c$ is a quadratic nonresidue, then 
    \[ c \cdot (\Z/p\Z)^* = (\Z/p\Z)^*. \] 
    Since $\QR \times \NR = \NR$, we see that multiplication by $c$ sends $A$ 
    to $B$, which is a surjective map by counting numbers. In particular, 
    we find that 
    \[ c \cdot B = c \cdot ((\Z/p\Z)^* \setminus A) = (c \cdot (\Z/p\Z)^*) 
    \setminus (c \cdot A) = (\Z/p\Z)^* \setminus B = A, \] 
    and hence $\NR \times \NR = \QR$, proving (iii). 
\end{pf}

\subsection{Special Cases of Quadratic Reciprocity}\label{subsec:5.3}

\vspace{2ex}
\begin{theo}[Euler's criterion]{thm:5.10}
    Let $p$ be an odd prime, and let $a \in (\Z/p\Z)^*$. Then 
    \[ a^{(p-1)/2} \equiv \left( \frac ap \right) \pmod p. \] 
\end{theo}
\begin{pf}
    If $a$ is a quadratic residue modulo $p$, say $a \equiv b^2 \pmod p$, then 
    Fermat's little theorem tells us that 
    \[ a^{(p-1)/2} \equiv (b^2)^{(p-1)/2} \equiv b^{p-1} \equiv 1 \equiv 
    \left( \frac ap \right) \pmod p. \] 
    Next, consider the congruence 
    \[ X^{(p-1)/2} - 1 \equiv 0 \pmod p. \] 
    We just showed that every quadratic residue modulo $p$ is a solution to this 
    congruence, and we know that there are exactly $(p-1)/2$ of them by 
    Remark~\ref{remark:5.6}. Moreover, $X^{(p-1)/2} - 1 \equiv 0 \pmod p$ 
    has at most $(p-1)/2$ distinct solutions since $(\Z/p\Z)^*$ is a field. 
    Therefore, there is a correspondence between the solutions of 
    $X^{(p-1)/2} - 1 \equiv 0 \pmod p$ and the quadratic residues modulo $p$. 

    Now, let $a$ be a quadratic nonresidue modulo $p$. Then Fermat's little 
    theorem tells us that $a^{p-1} \equiv 1 \pmod p$, and hence 
    \[ 0 \equiv a^{p-1} - 1 \equiv \left( a^{(p-1)/2} - 1 \right) 
    \left( a^{(p-1)/2} + 1 \right) \pmod p. \] 
    The first factor is not $0$ modulo $p$ because the solutions of $X^{(p-1)/2}
    - 1 \equiv 0 \pmod p$ are precisely the quadratic residues modulo $p$. 
    Therefore, we have $a^{(p-1)/2} + 1 \equiv 0 \pmod p$, and so 
    \[ a^{(p-1)/2} \equiv -1 \equiv \left( \frac ap \right) \pmod p. \qedhere \] 
\end{pf}

We can use Euler's criterion to compute some Legendre symbols. 

\begin{theo}[Quadratic Reciprocity I]{thm:5.11}
    Let $p$ be an odd prime. Then 
    \[ \left( \frac{-1}p \right) = \begin{cases}
        1 & \text{if } p \equiv 1 \pmod 4, \\ 
        -1 & \text{if } p \equiv 3 \pmod 4. 
    \end{cases} \] 
\end{theo}
\begin{pf}
    By Euler's criterion, we have 
    \[ (-1)^{(p-1)/2} \equiv \left( \frac{-1}p \right) \pmod p. \] 
    Suppose that $p \equiv 1 \pmod 4$. Then $p = 4k+1$ for some integer $k$, and 
    we see that 
    \[ (-1)^{(p-1)/2} = (-1)^{(4k+1-1)/2} = (-1)^{2k} = 1, \] 
    which gives us 
    \[ \left( \frac{-1}p \right) = 1. \] 
    On the other hand, when $p \equiv 3 \pmod 4$ so that $p = 4k+3$ for some 
    integer $k$, then 
    \[ (-1)^{(p-1)/2} = (-1)^{(4k+3-1)/2} = (-1)^{2k+1} = -1, \] 
    and it follows that 
    \[ \left( \frac{-1}p \right) = -1. \qedhere \] 
\end{pf}

To go further, we first require another tool. 

\begin{theo}[Gauss' lemma]{thm:5.12}
    Let $p$ be an odd prime, and let $a$ be an integer coprime to $p$. 
    Take the numbers $a, 2a, \dots, [(p-1)/2]a$ and reduce each of them modulo $p$
    to get numbers lying between $-(p-1)/2$ and $(p-1)/2$. If $s$ is the number 
    of resulting residues less than $0$, then 
    \[ \left( \frac ap \right) = (-1)^s. \] 
\end{theo}
\begin{pf}
    For each $1 \leq i \leq (p-1)/2$, let $u_i$ be an integer such that 
    $ia \equiv u_i \pmod p$ and $-(p-1)/2 \leq u_i \leq (p-1)/2$. Note that $s$ 
    is the number of elements in $u_1, \dots, u_{(p-1)/2}$ less than $0$. 

    We claim that 
    \[ \{|u_1|, |u_2|, \dots, |u_{(p-1)/2}|\} = \{1, 2, \dots, p-1\}. \] 
    It is sufficient to show that no two of the integers in the first set are 
    congruent modulo $p$, as there are exactly $(p-1)/2$ elements in the set, 
    and they are all positive integers not exceeding $(p-1)/2$. Suppose that 
    $|u_i| = |u_j|$. If $u_i = u_j$, then $ia \equiv ja \pmod p$, which implies 
    that $i = j$ since $\gcd(a, p) = 1$ and $1 \leq i, j \leq (p-1)/2$. 
    On the other hand, if $u_i = -u_j$, then $ia \equiv -ja \pmod p$. 
    This implies that $(i+j)a \equiv 0 \pmod p$ and hence 
    \[ i + j \equiv 0 \pmod p \] 
    since $\gcd(a, p) = 1$. Since $1 \leq i, j \leq (p-1)/2$, we have 
    $2 \leq i+j \leq p-1$. But there is no number congruent to $0$ modulo $p$ 
    in this range, so this scenario is impossible. This proves our claim. 

    Finally, we find that 
    \[ a^{(p-1)/2} \prod_{i=1}^{(p-1)/2} i 
    \equiv \prod_{i=1}^{(p-1)/2} ia 
    \equiv \prod_{i=1}^{(p-1)/2} u_i 
    \equiv (-1)^s \prod_{i=1}^{(p-1)/2} i \pmod p, \] 
    where the third congruence follows from the claim. This implies that 
    \[ \left( \frac ap \right) \equiv a^{(p-1)/2} \equiv (-1)^s \pmod p. \qedhere \] 
\end{pf}

Now, we can compute $(\frac2p)$. 

\begin{theo}[Quadratic Reciprocity II]{thm:5.13}
    Let $p$ be an odd prime. Then 
    \[ \left( \frac2p \right) = (-1)^{(p^2-1)/8} = \begin{cases} 
        1 & \text{if } p \equiv \pm1 \pmod 8, \\ 
        -1 & \text{if } p \equiv \pm3 \pmod 8. 
    \end{cases} \] 
\end{theo}
\begin{pf}
    From Gauss' lemma (Theorem~\ref{thm:5.12}), we only need to find the amount 
    of numbers $s$ from the list 
    \[ 1\cdot 2,\,2\cdot 2,\,3 \cdot 2, \dots, \left( \frac{p-1}2 \right) \cdot 2 \] 
    which are greater than $p/2$. Note that for $1 \leq j \leq (p-1)/2$, the 
    integer $2j$ is less than $p/2$ when $j \leq p/4$. Hence, there are 
    $\floor{p/4}$ integers in the set less than $p/2$. Consequently, there are 
    \[ s = \frac{p-1}2 - \floor*{\frac p4} \] 
    of them greater than $p/2$. In particular, we find that 
    \[ \left( \frac2p \right) = (-1)^{\frac{p-1}2 - \floor{p/4}}. \] 
    To finish the proof, it remains to show that 
    \[ \frac{p-1}2 - \floor*{\frac p4} \equiv \frac{p^2 - 1}8 \pmod 2. \] 
    We can verify this case by case. 
    \begin{itemize}
        \item If $p \equiv \pm1 \pmod 8$, then $p = 8k\pm 1$ for some integer $k$. 
        We have 
        \[ \frac{p-1}2 - \floor*{\frac p4} = \frac{(8k\pm 1) - 1}2 
        - \floor*{\frac{8k\pm 1}4} = \frac{-1\pm 1}2 + 4k + \floor*{2k \pm \frac14}. \] 
        If we choose the $+$ sign, we obtain 
        \[ \frac{p-1}2 - \floor*{\frac p4} \equiv \frac{-1 + 1}2 + 
        \floor*{2k + \frac14} \equiv 0 + 2k \equiv 0 \pmod 2, \] 
        whereas if we choose the $-$ sign, we get 
        \[ \frac{p-1}2 - \floor*{\frac p4} \equiv \frac{-1-1}2 + 
        \floor*{2k - \frac14} \equiv -1 + 2k - 1 \equiv 0 \pmod 2. \] 
        On the other hand, we have 
        \[ \frac{p^2-1}8 \equiv \frac{(8k\pm 1)^2 - 1}8 \equiv 
        \frac{64k^2 \pm 16k + 1 - 1}8 \equiv 8k^2 \pm 2k \equiv 0 \pmod 2, \] 
        which proves this case. 
        \item If $p \equiv \pm3 \pmod 8$, then $p = 8k\pm 3$ for some integer $k$. 
        Then, we see that 
        \[ \frac{p-1}2 - \floor*{\frac p4} = \frac{(8k\pm 3) - 1}2 
        - \floor*{\frac{8k\pm 3}4} = \frac{-1\pm 3}2 + 4k + \floor*{2k \pm \frac34}. \]
        Choosing $+$ gives 
        \[ \frac{p-1}2 - \floor*{\frac p4} \equiv \frac{-1 + 3}2 + 
        \floor*{2k + \frac34} \equiv -1 + 2k \equiv 1 \pmod 2, \] 
        while choosing $-$ gives 
        \[ \frac{p-1}2 - \floor*{\frac p4} \equiv \frac{-1-3}2 + 
        \floor*{2k - \frac34} \equiv -2 + 2k - 1 \equiv 1 \pmod 2. \]
        Moreover, we have 
        \[ \frac{p^2-1}8 \equiv \frac{(8k\pm 3)^2 - 1}8 \equiv 
        \frac{64k^2 \pm 48k + 9 - 1}8 \equiv 8k^2 \pm 6k + 1 \equiv 1 \pmod 2, \] 
        which completes the proof of the theorem. \qedhere 
    \end{itemize}
\end{pf}

\subsection{The Law of Quadratic Reciprocity}\label{subsec:5.4}

\vspace{2ex}
\begin{theo}[Law of Quadratic Reciprocity]{thm:5.14}
    Let $p$ and $q$ be odd primes. Then 
    \[ \left( \frac pq \right) \left( \frac qp \right) = (-1)^{\frac{p-1}2 \frac{q-1}2}. \] 
    In particular, we have 
    \[ \left( \frac pq \right) = \begin{cases}
        \left(\dfrac qp \right) & \text{if $p \equiv 1$ (mod $4$) or $q \equiv 1$ (mod $4$),} \\[2ex]
        -\left(\dfrac qp \right) & \text{if $p \equiv q \equiv 3$ (mod $4$).} 
    \end{cases} \] 
\end{theo}

There are many ways to prove this famous theorem. We will take an approach that is 
not the fastest, but is easy to understand. First, we require the following lemma. 

\begin{lemma}{lemma:5.15}
    Let $p$ be an odd prime, and let $a$ be an odd integer such that $a \nmid p$. 
    Then we have 
    \[ \left( \frac ap \right) = (-1)^{T(a, p)}, \] 
    where we define $T(a, p)$ to be 
    \[ T(a, p) = \sum_{j=1}^{(p-1)/2} \floor*{\frac{ja}{p}}. \] 
\end{lemma}
\begin{pf}
    Consider the reduced residues of $a, 2a, \dots, [(p-1)/2]a$ lying between 
    $-(p-1)/2$ and $(p-1)/2$; let $u_1, \dots, u_s$ be those less than $0$, 
    and $v_1, \dots, v_t$ be those greater than $0$. The division algorithm 
    tells us that 
    \[ ja = p\floor*{\frac{ja}{p}} + r, \] 
    where the remainder $r$ is either in the form $p + u_j$ or $v_j$. By 
    adding these $(p-1)/2$ equations, we obtain 
    \[ \sum_{j=1}^{(p-1)/2} ja = \sum_{j=1}^{(p-1)/2} p\floor*{\frac{ja}{p}} 
    + \sum_{j=1}^s (p + u_j) + \sum_{j=1}^t v_j. \tag{5.1}\label{eq:5.1} \] 
    In the proof of Gauss' lemma (Theorem~\ref{thm:5.12}), we saw that the integers 
    \[ -u_1, -u_2, \dots, -u_s, v_1, \dots, v_t \] 
    are precisely the integers $1, 2, \dots, (p-1)/2$. Therefore, we have 
    \[ \sum_{j=1}^{(p-1)/2} j = -\sum_{j=1}^s u_j + \sum_{j=1}^t v_j. 
    \tag{5.2}\label{eq:5.2} \] 
    Subtracting $\eqref{eq:5.2}$ from $\eqref{eq:5.1}$, we find that 
    \[ \sum_{j=1}^{(p-1)/2} ja - \sum_{j=1}^{(p-1)/2} j 
    = \sum_{j=1}^{(p-1)/2} p\floor*{\frac{ja}{p}} + ps + 2 \sum_{j=1}^s u_j. \] 
    Using the definition of $T(a, p)$, we get 
    \[ (a-1) \sum_{j=1}^{(p-1)/2} j = pT(a, p) + ps + 2\sum_{j=1}^s u_j. 
    \tag{5.3}\label{eq:5.3} \] 
    Reducing equation $\eqref{eq:5.3}$ modulo $2$ yields $0 \equiv T(a, p) 
    + s \pmod 2$ since $a$ and $p$ are assumed to be odd, and hence $T(a, p) 
    \equiv s \pmod 2$ since $s$ and $-s$ have the same parity. We conclude that 
    \[ \left( \frac ap \right) = (-1)^s = (-1)^{T(a, p)}. \qedhere \] 
\end{pf}

Now, we are ready to prove our main theorem. 

{\sc Proof of Theorem 5.14.} Consider the pairs of integers $(x, y)$ with 
$1 \leq x \leq (p-1)/2$ and $1 \leq y \leq (q-1)/2$. There are 
$(p-1)/2 \cdot (q-1)/2$ such pairs. Note that none of these pairs satisfy 
$qx = py$ since this would imply that $p \mid x$ and $y \mid q$, which is absurd. 
We divide these $(p-1)/2 \cdot (q-1)/2$ pairs into two groups, depending 
on the relative sizes of $qx$ and $py$. 

The pairs of integers $(x, y)$ satisfying $qx > py$ are precisely those 
with $1 \leq x \leq (p-1)/2$ and $1 \leq y \leq qx/p$. Hence, for fixed 
$1 \leq x \leq (p-1)/2$, there are $\floor{qx/p}$ possible values of $y$, so 
the number of pairs $(x, y)$ satisfying $qx > py$ is 
\[ \sum_{j=1}^{(p-1)/2} \floor*{\frac{qj}{p}}. \] 
Similarly, the pairs of integers $(x, y)$ satisfying $qx < py$ are precisely 
those with $1 \leq y \leq (q-1)/2$ and $1 \leq x \leq py/q$. For fixed 
$1 \leq y \leq (q-1)/2$, there are $\floor{py/q}$ possible values of $x$, 
so the number of pairs $(x, y)$ satisfying $qx < py$ is 
\[ \sum_{j=1}^{(q-1)/2} \floor*{\frac{pj}{q}}. \] 
Consequently, we find that 
\[ \frac{p-1}2 \frac{q-1}2 = \sum_{j=1}^{(p-1)/2} \floor*{\frac{qj}{p}} 
+ \sum_{j=1}^{(q-1)/2} \floor*{\frac{pj}{q}} = T(q, p) + T(p, q). \] 
It follows from Lemma~\ref{lemma:5.15} that 
\[ (-1)^{\frac{p-1}2 \frac{q-1}2} = (-1)^{T(q, p) + T(p, q)} = 
(-1)^{T(q, p)} \cdot (-1)^{T(p, q)} = \left( \frac qp \right) \left( \frac pq 
\right), \] 
which completes the proof. \qed 

\subsection{The Jacobi Symbol}\label{subsec:5.5}
We now give a generalization of the Legendre symbol to all odd positive integers 
instead of only odd primes, called the Jacobi symbol. The Jacobi symbol also 
shares many of the properties of the Legendre symbol, and we will leave the 
proof of these as homework. 

\begin{defn}{def:5.16}
    Let $a \in \Z$, and let $n \in \N$ be odd. Let $n = p_1^{\alpha_1} 
    p_2^{\alpha_2} \cdots p_k^{\alpha_k}$ be the prime factorization of $n$. 
    We define the {\bf Jacobi symbol} $(\frac{a}{n})$ to be 
    \[ \left( \frac{a}{n} \right) := \left( \frac{a}{p_1} \right)^{\!\alpha_1} 
    \left( \frac{a}{p_2} \right)^{\!\alpha_2} \cdots 
    \left( \frac{a}{p_k} \right)^{\!\alpha_k}. \] 
\end{defn}

\begin{theo}[Generalized Law of Quadratic Reciprocity]{thm:5.17}
    Let $a, b \in \N$ be odd with $\gcd(a, b) = 1$. Then we have 
    \begin{enumerate}[(1)]
        \item $\left( \dfrac{-1}{b} \right) = \begin{cases} 
            1 & \text{if } b \equiv 1 \pmod 4, \\ 
            -1 & \text{if } b \equiv 3 \pmod 4; 
        \end{cases}$ 
        \item $\left( \dfrac{2}{b} \right) = \begin{cases} 
            1 & \text{if } b \equiv \pm1 \pmod 8, \\ 
            -1 & \text{if } b \equiv \pm3 \pmod 8; 
        \end{cases}$
        \item $\left( \dfrac{a}{b} \right) = \begin{cases}
            \left(\dfrac ba \right) & \text{if $a \equiv 1$ (mod $4$) or $b \equiv 1$ (mod $4$),} \\[2ex]
            -\left(\dfrac ba \right) & \text{if $a \equiv b \equiv 3$ (mod $4$).} 
        \end{cases}$
    \end{enumerate}
\end{theo}
\begin{pf}
    This is Question 1 of Homework 4.
\end{pf}

To end off the chapter, we give some examples of computing Legendre symbols, 
as well as some applications. 

\begin{exmp}{exmp:5.18}
    Consider the Legendre symbol $(\frac{13}{17})$. By the Law of Quadratic 
    Reciprocity (Theorem~\ref{thm:5.14}), we have 
    \[ \left( \frac{13}{17} \right) = \left( \frac{17}{13} \right) 
    (-1)^{(17-1)(13-1)/4} = \left( \frac{17}{13} \right) 
    = \left( \frac{4}{13} \right) = 1. \] 
\end{exmp}

\begin{exmp}{exmp:5.19}
    Let's compute $(\frac{713}{1009})$. Note that $1009$ is prime and $713 
    = 23 \cdot 31$, so Theorem~\ref{thm:5.9} tells us that 
    \[ \left( \frac{713}{1009} \right) = \left( \frac{23}{1009} \right) 
    \left( \frac{31}{1009} \right). \] 
    By the Law of Quadratic Reciprocity (Theorem~\ref{thm:5.14}), we find that 
    $(\frac{23}{1009}) = (\frac{1009}{23}) = (\frac{20}{23})$ and 
    $(\frac{31}{1009}) = (\frac{1009}{31}) = (\frac{17}{31})$. Then, we see that 
    \[ \left( \frac{20}{23} \right) = \left( \frac{4}{23} \right) 
    \left( \frac{5}{23} \right) = \left( \frac{5}{23} \right) = 
    \left( \frac{23}{5} \right) = \left( \frac{3}{5} \right) = -1. \] 
    On the other hand, we have 
    \[ \left( \frac{17}{31} \right) = \left( \frac{31}{17} \right) = 
    \left( \frac{14}{17} \right) = \left( \frac{2}{17} \right) 
    \left( \frac{7}{17} \right) = \left( \frac{7}{17} \right) = 
    \left( \frac{17}{7} \right) = \left( \frac{3}{7} \right) = -1, \] 
    where the third last equality above comes from Quadratic Reciprocity II 
    (Theorem~\ref{thm:5.13}). Putting these together, we obtain 
    \[ \left( \frac{713}{1009} \right) = \left( \frac{23}{1009} \right) 
    \left( \frac{31}{1009} \right) = (-1)(-1) = 1. \] 
\end{exmp}

\begin{exmp}{exmp:5.20}
    Note that $5$ is a quadratic residue for all primes $p$ of the form $10k \pm 1$, 
    whereas it is a quadratic nonresidue for all primes $p$ of the form $10k \pm 3$. 
    This is because we have 
    \[ \left( \frac{5}{p} \right) = \left( \frac{p}{5} \right) \] 
    by the Law of Quadratic Reciprocity (Theorem~\ref{thm:5.14}); we see that 
    $\pm1$ are quadratic residues modulo $5$, while $\pm3$ are not. 
\end{exmp}

\begin{exmp}{exmp:5.21}
    The equation $x^4 - 17y^4 = 2\omega^2$ has no solution in the integers. 

    Suppose towards a contradiction that an integer solution exists. We may 
    assume without loss of generality that $x$ is coprime with $y$, and hence 
    $x$ is coprime with $\omega$. If $p$ is an odd prime that divides $\omega$, then 
    \[ \left( \frac{17}{p} \right) = 1 \] 
    since $x^4 \equiv 17y^4 \pmod p$. Alternatively, by the Law of Quadratic 
    Reciprocity (Theorem~\ref{thm:5.14}), we obtain 
    \[ \left( \frac{17}{p} \right) = \left( \frac{p}{17} \right) 
    (-1)^{(17-1)(p-1)/4} = 1, \] 
    so $p$ is a quadratic residue of $17$. Furthermore, we have $(\frac{2}{17}) 
    = (-1)^{(17^2-1)/8} = 1$. In particular, every prime factor $p$ of 
    $\omega$ is a quadratic residue modulo $17$, so $\omega \equiv t^2 \pmod{17}$ 
    for some integer $t$. Then $x^4 \equiv 2t^4 \pmod{17}$, and it follows that 
    there exists an integer $r$ such that $r^4 \equiv 2 \pmod{17}$. But the 
    order of $2$ modulo $17$ is $8$, so no such $r$ exists. This is a contradiction, 
    so the equation has no solution in the integers. 
\end{exmp}

\begin{exmp}{exmp:5.22}
    We can check if the congruence $3x^2 - 7x - 42 \equiv 0 \pmod{391}$ has a solution. 

    Note that $391 = 17 \cdot 23$. Multiplying the above equation by $12$, we find that 
    \[ 36x^2 - 84x - 516 \equiv 0 \pmod{391}. \] 
    By completing the square, we see that  
    \[ (6x - 7)^2 \equiv 516 + 49 \equiv 565 \pmod{391}. \] 
    This is now equivalent to solving $x^2 \equiv 174 \pmod{391}$. Notice that 
    $x^2 \equiv 174 \equiv 4 \pmod{17}$, which has the solution $x = 2$. 
    On the other hand, we have $x^2 \equiv 174 \equiv 13 \pmod{23}$. We compute 
    \[ \left( \frac{13}{23} \right) = \left( \frac{23}{13} \right) = 
    \left( \frac{10}{13} \right) = \left( \frac{2}{13} \right) 
    \left( \frac{5}{13} \right) = -\left( \frac{5}{13} \right) = 
    -\left( \frac{13}{5} \right) = -\left(\frac{3}{5} \right) = 1. \] 
    Then $x^2 \equiv 13 \pmod{23}$ has a solution, so it follows from the 
    Chinese remainder theorem that the original congruence has a solution. 
\end{exmp}