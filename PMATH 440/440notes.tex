\documentclass[10pt]{article}
\usepackage[T1]{fontenc}
\usepackage{amsmath,amssymb,amsthm}
\usepackage{mathtools}
\usepackage[shortlabels]{enumitem}
\usepackage[english]{babel}
\usepackage[utf8]{inputenc}
\usepackage{fancyhdr}
\usepackage{bold-extra}
\usepackage{color}   
\usepackage{tocloft}
\usepackage{graphicx}
\usepackage{lipsum}
\usepackage{wrapfig}
\usepackage{cutwin}
\usepackage{hyperref}
\usepackage{lastpage}
\usepackage{multicol}
\usepackage{tikz}
\usepackage{xcolor}
\usepackage{microtype}

% some useful math commands
\newcommand{\eps}{\varepsilon}
\newcommand{\R}{\mathbb{R}}
\newcommand{\C}{\mathbb{C}}
\newcommand{\N}{\mathbb{N}}
\newcommand{\Z}{\mathbb{Z}}
\newcommand{\Q}{\mathbb{Q}}
\newcommand{\K}{\mathbb{K}}
\newcommand{\F}{\mathbb{F}}

\numberwithin{equation}{section}

\newcommand{\dd}{\,\mathrm{d}}
\newcommand{\ddz}{\frac{\rm d}{{\rm d}z}}
\newcommand{\pv}{\text{p.v.}}

\renewcommand{\Re}{{\rm Re}}

\DeclareMathOperator{\GL}{GL}
\DeclareMathOperator{\id}{id}
\DeclareMathOperator{\Arg}{Arg}
\DeclareMathOperator{\Log}{Log}
\DeclareMathOperator{\PV}{PV}
\DeclareMathOperator{\sech}{sech}
\DeclareMathOperator{\csch}{csch}
\DeclareMathOperator{\Res}{Res}
\DeclareMathOperator{\Li}{Li}

\DeclarePairedDelimiter\ceil{\lceil}{\rceil}
\DeclarePairedDelimiter\floor{\lfloor}{\rfloor}

\newcommand{\suchthat}{\;\ifnum\currentgrouptype=16 \;\middle|\;\else\mid\fi\;}

% title formatting
\newcommand{\newtitle}[4]{
  \begin{center}
	\huge{\textbf{\textsc{#1 Course Notes}}}
    
	\large{\sc #2}
    
	{\sc #3 \textbullet\, #4 \textbullet\, University of Waterloo}
	\normalsize\vspace{1cm}\hrule
  \end{center}
}

% for theorems
\newtheoremstyle{newstyle}      
{} %Aboveskip 
{-0.25pt} %Below skip
{\mdseries} %Body font e.g.\mdseries,\bfseries,\scshape,\itshape
{} %Indent
{\scshape} %Head font e.g.\bfseries,\scshape,\itshape
{.} %Punctuation afer theorem header
{ } %Space after theorem header
{} %Heading

\theoremstyle{newstyle}

\newtheorem*{prop*}{Proposition}
\newtheorem*{cor*}{Corollary}
\newtheorem*{exercise*}{Exercise}
\newtheorem*{lemma*}{Lemma}
\newtheorem*{remark*}{Remark}
\newtheorem*{exmp*}{Example}
\newtheorem*{defn*}{Definition}
\newtheorem*{thm*}{Theorem}
\newtheorem*{notation*}{Notation}
\newtheorem{thm}{Theorem}[section]
\newtheorem{fact}[thm]{Fact}
\newtheorem{cor}[thm]{Corollary}
\newtheorem{lemma}[thm]{Lemma}
\newtheorem{remark}[thm]{Remark}
\newtheorem{prop}[thm]{Proposition}
\newtheorem{defn}[thm]{Definition}
\newtheorem{claim}[thm]{Claim}
\newtheorem{axiom}[thm]{Axiom}
\newtheorem{notation}[thm]{Notation}
\newtheorem{exercise}[thm]{Exercise}
\newtheorem{exmp}[thm]{Example}

% new proof environment
\makeatletter
\newenvironment{pf}[1][\proofname]{\par
  \pushQED{\qed}%
  \normalfont \topsep0\p@\relax
  \trivlist
  \item[\hskip\labelsep\scshape
  #1\@addpunct{.}]\ignorespaces
}{%
  \popQED\endtrivlist\@endpefalse
}
\makeatother

% 1-inch margins
\topmargin 0pt
\advance \topmargin by -\headheight
\advance \topmargin by -\headsep
\textheight 8.9in
\oddsidemargin 0pt
\evensidemargin \oddsidemargin
\marginparwidth 0.5in
\textwidth 6.5in

\parindent 0in
\parskip 1.5ex

\setlist[itemize]{topsep=0pt}
\setlist[enumerate]{topsep=0pt}

% hyperlinks
\hypersetup{
  colorlinks=true, 
  linktoc=all,     % table of contents is clickable  
  allcolors=black  % all hyperlink colours
}

% table of contents
\addto\captionsenglish{
  \renewcommand{\contentsname}%
    {Table of Contents}%
}
\renewcommand{\cftsecfont}{\normalfont}
\renewcommand{\cftsecpagefont}{\normalfont}
\cftsetindents{section}{0em}{2em}

\fancypagestyle{plain}{%
\fancyhf{} % clear all header and footer fields
\lhead{PMATH 440: Fall 2021}
\fancyhead[R]{Table of Contents}
%\headrule
\fancyfoot[R]{{\small Page \thepage\ of \pageref*{LastPage}}}
}

% headers and footers
\pagestyle{fancy}
\renewcommand{\sectionmark}[1]{\markboth{#1}{#1}}
\lhead{PMATH 440: Fall 2021}
\cfoot{}
\setlength\headheight{14pt}

%\setcounter{section}{-1}

\begin{document}

\pagestyle{fancy}
\newtitle{PMATH 440}{Analytic Number Theory}{Wentang Kuo}{Fall 2021}
\rhead{Table of Contents}
\rfoot{{\small Page \thepage\ of \pageref*{LastPage}}}

\tableofcontents
\vspace{1cm}\hrule
\fancyhead[R]{\nouppercase\rightmark}
\newpage 
\fancyhead[R]{Section \thesection: \nouppercase\leftmark}

\section{Introduction to Prime Numbers and Their Counting Function}\label{sec:1}

\subsection{Primes}\label{subsec:1.1}

\vspace{2ex}
\begin{defn}{def:1.1}
A {\bf prime number} is a positive integer greater than $1$ such that its only factors are $1$ and 
itself. We denote by ${\cal P}$ the set of all prime numbers. For a positive real number $x$, 
we define the {\bf prime counting function} by 
\[ \pi(x) = \#\{p \leq x : p \in {\cal P}\}, \]
where $\#S$ denotes the cardinality of the set $S$. 
\end{defn}

We would like to know how the primes are distributed among the integers. Let $p_n$ denote the 
$n$-th prime. Is there a formula to obtain $p_n$? Is there a polynomial $f(x) \in \Z[x]$ such that 
$f(n) = p_n$ for all $n \in \N$? The answer to the latter question is no, due to the following result. 

\begin{prop}{prop:1.2}
There is no non-constant polynomial $f(x) \in \Z[x]$ such that $f(n)$ is prime for all $n \in \N$. 
\end{prop}
\begin{pf}
Suppose such a polynomial $f(x) \in \Z[x]$ existed, and write 
\[ f(x) = a_n x^n + \cdots + a_1 x + a_0. \]
Let $q$ be a prime with $f(n) = q$ for some $n \in \N$. Then $q \mid f(n+kq)$ for each $k \in \N$. 
In particular, notice that if $f(m)$ is prime for every positive integer $m$, then $f(x)$ must be 
constant with $f(x) = q$ for some prime $q$. 
\end{pf}

\begin{remark}{remark:1.3}
\begin{enumerate}[(1)]
    \item There are examples of polynomials whose initial values are surprisingly often prime. 
    For example, the polynomial $n^2 + n + 41$ is prime for all $0 \leq n \leq 39$, and the 
    polynomial $(n-40)^2 + (n-40) + 41$ is prime for all $0 \leq n \leq 79$. 
    \item In the 1970s, Matijasevic proved Hilbert's tenth problem, and in the process, he was able to
    show that there is a polynomial $f \in \Z[a, b, \dots, z]$ such that the set of positive 
    values in $f(\N^{26})$ is exactly the set of primes. In 1977, he showed that only $10$ variables 
    are needed.
\end{enumerate}
\end{remark}

Let us instead ask a weaker question. Can we find a non-constant polynomial $f(x) \in \Z[x]$ such that 
$f(n)$ yields a prime for infinitely many $n \in \N$? Trivially, we see that $f(x) = x+k$ works 
for any $k \in \Z$. When the coefficient of $x$ is not equal to $1$, we have the following result, 
which we will prove at the end of this course.

\begin{theo}[Dirichlet]{thm:1.4}
Let $k$ and $\ell$ be coprime positive integers. Then $kn+\ell$ is prime for infinitely many positive
integers $n$. 
\end{theo}

\begin{remark}{remark:1.5}
\begin{enumerate}[(1)]
    \item At the moment, there is no known polynomial of degree greater than $1$ in one variable 
    known to take prime values infinitely often. The best result known to date is that $n^2+1$ is 
    a product of two primes for infinitely many $n$.
    \item If we instead consider polynomials of two variables, we can go further. It is known that 
    an odd prime $p$ is the sum of two squares if and only if $p \equiv 1 \pmod 4$. 
    In 1998, Friedlander and Iwaniec proved that there are infinitely many primes of the 
    form $n^2 + m^4$. In 2001, Heath-Brown showed that there are infinitely many primes of 
    the form $n^3 + 2m^3$. 
\end{enumerate}
\end{remark}

\begin{theo}[Euclid]{thm:1.6}
There are infinitely many prime numbers.
\end{theo}
\begin{pf}
Assume that there are only finitely many primes, say $p_1, \dots, p_n$, and consider 
\[ m = p_1 \cdots p_n + 1. \]
Then $m$ can be written as a product of primes by unique factorization, and $p_k \mid m$ for some $1 \leq
k \leq n$. 
Hence, we see that $p_k \mid m - p_1 \cdots p_n$ and $p_k \mid 1$, which is a contradiction. 
\end{pf}

We would like to estimate the prime counting function $\pi(x)$. 

\begin{prop}{prop:1.7}
For all $n \in \N$, we have $p_n \leq 2^{2^n}$. 
\end{prop}
\begin{pf}
We proceed by induction. For $n = 1$, we have $2 = p_1 \leq 2^{2^1} = 4$. Suppose the result 
holds for all $1 \leq k \leq n$. By Euclid's argument, we obtain $p_{n+1} \leq p_1 \cdots p_n + 1$. 
It follows from induction that 
\[ p_{n+1} \leq 2^{2^1} 2^{2^2} \cdots 2^{2^n} + 1 \leq 2^{2^{n+1}-2} + 1 \leq 2^{2^{n+1}}, \]
which completes the proof. 
\end{pf}

\begin{cor}{cor:1.8}
For all $x \geq 2$, we have $\pi(x) > \log\log x$. (In this course, $\log$ denotes the natural logarithm.)
\end{cor}
\begin{pf}
Let $x \geq 2$, and let $s$ be the integer satisfying 
\[ 2^{2^s} \leq x < 2^{2^{s+1}}. \]
By Proposition~\ref{prop:1.7}, we have $\pi(x) \geq s$. On the other hand, since $x < 2^{2^{s+1}}$, 
taking logarithms yields $\log_2(\log_2 x) < s+1$, and hence 
\[ \frac{\log(\frac{\log x}{\log 2})}{\log 2} < s+1. \]
It follows that 
\[ \pi(x) \geq s > \frac{\log(\frac{\log x}{\log 2})}{\log 2} - 1 \geq \log\log x. \qedhere \]
\end{pf}

There is an alternative way to prove Euclid's theorem, due to Euler, which is left as part of the 
homework. Using the same idea, we can derive a slightly better lower bound for $\pi(x)$. 

\begin{prop}{prop:1.9}
For all $x \geq 2$, we have 
\[ \pi(x) \geq \frac{\log\log x}{\log 2}. \]
\end{prop}
\begin{pf}
Suppose that $x \geq 2$. Then we have 
\[ 2^{\pi(x)} \geq \prod_{p\leq x} \left(1 - \frac1p\right)^{-1} = \prod_{p\leq x} \left(1 + \frac1p
+ \frac{1}{p^2} + \cdots \right) \geq \sum_{n\leq x} \frac1n \geq \int_1^{\lfloor x \rfloor + 1}
\frac1u\dd u \geq \log x, \]
where the product $\prod_{p\leq x}$ means that $p$ runs through all primes at most $x$, and
$\lfloor y \rfloor$ is the greatest integer less than or equal to $y$. We will will use this 
notation for the rest of the course. Taking logarithms yields the desired inequality.
\end{pf}

Fermat had conjectured that the numbers of the form $2^{2^n}+1$ are prime for $n \in \N$. 
He had checked it for the values $0 \leq n \leq 4$. These are known as the {\bf Fermat numbers} and 
are denoted by 
\[ F_n = 2^{2^n} + 1. \] 
In 1732, Euler showed that $641 \mid F_5$. It is also known that $F_6, \dots, F_{21}$ are composite. 
It is quite likely that only finitely many Fermat numbers are prime. 

\begin{theo}[Poly\'a]{thm:1.10}
If $n$ and $m$ are positive integers with $1 \leq n < m$, then $(F_n, F_m) = 1$. 
\end{theo}
\begin{pf}
Write $m = n+k$ with $k \geq 1$. First, we will show that $F_n \mid F_m - 2$. Observe that 
\[ F_m - 2 = (2^{2^{n+k}} + 1) - 2 = 2^{2^{n+k}} - 1. \]
The polynomial $x^{2^k} - 1$ is divisible by $x+1$ in $\Z[x]$. Now, letting $x = 2^{2^n}$, we get 
\[ \frac{F_m-2}{F_n} = \frac{x^{2^k} - 1}{x+1} = x^{2^k-1} - x^{2^k-2} + \cdots - 1 \in \Z. \] 
Hence, we have $F_n \mid F_m - 2$. Suppose now that $d \mid F_n$ and $d \mid F_m$. Then 
$d \mid 2$ and $2 \nmid F_n$, which implies that $d = \pm1$. The result follows. 
\end{pf}

This gives yet another proof of Euclid's theorem, as well as the bound $p_n \leq 2^{2^n} + 1$. 

\subsection{Elementary Approximations of $\pi(x)$}\label{subsec:1.2}
In 1896, Hadamard and de la Vall\'ee Poussin each proved the Prime Number Theorem independently. 

\begin{theo}[Prime Number Theorem]{thm:1.11}
We have 
\[ \lim_{x\to\infty} \frac{\pi(x)}{x/\log x} = 1. \]
\end{theo}

This was initially conjectured by Gauss. We will prove this theorem later in the course; 
for now, we will see how to approach this problem using elementary methods. 

\begin{theo}{thm:1.12}
For all $x \geq 2$, we have 
\[ \pi(x) \geq \frac{\log x}{2\log 2}. \]
Moreover, for all $n \geq 1$, we have $p_n \leq 4^n$.
\end{theo}
\begin{pf}
Let $x \geq 2$ be an integer. Let $p_1, \dots, p_j$ be the primes less than or equal to $x$. 
Note that we have $j = \pi(x)$ here. For every integer $n$ with $n \leq x$, we can write $n = n_1^2m$
where $n_1$ is a positive integer and $m$ is squarefree. Then $m$ is of the form 
\[ m = p_1^{\eps_1} \cdots p_j^{\eps_j}, \]
where $\eps_i \in \{0, 1\}$ for each $1 \leq i \leq j$. We see that there are at most $2^j$ possible 
values for $m$. Moreover, there are at most $\sqrt{x}$ possible values for $n_1$. Hence, we have 
$2^j \sqrt{x} \geq x$, which implies that $2^j \geq \sqrt{x}$. Denote this inequality by $(\star)$. 
Since $j = \pi(x)$, we see that 
\[ \pi(x) \log 2 \geq \frac{\log x}2, \]
so the first equality follows. For the second equality, take $x = p_n$ so that $\pi(p_n) = n$. 
By $(\star)$, we obtain $2^n \geq \sqrt{p_n}$ and hence $4^n \geq p_n$. 
\end{pf}

Let $n$ be a positive integer and let $p$ be a prime. Recall that the exact power of $p$ 
dividing $n!$ is 
\[ \sum_{n=1}^\infty \left\lfloor \frac{n}{p^k} \right\rfloor = \sum_{n=1}^{\left\lfloor \frac{\log n}{\log p} \right\rfloor} \left\lfloor \frac{n}{p^k} \right\rfloor. \]

\begin{theo}{thm:1.13}
For all $x \geq 2$, we have 
\[ \left( \frac{3\log 2}8 \right) \frac{x}{\log x} < \pi(x) < (6\log 2) \frac{x}{\log x}. \]
\end{theo}
\begin{pf}
This argument was given by Erd\H{o}s. First, we will prove the lower bound. Note that $\binom{2n}n$ is 
an integer, and 
\[ \binom{2n}n = \frac{(2n)!}{(n!)^2} \; \bigg\rvert \; \prod_{p\leq 2n} p^{r_p}, \]
where $r_p$ is an integer satisfying $p^{r_p} \leq 2n < p^{r_p+1}$. Indeed, note that 
the exact power of $p$ dividing $(2n)!$ is 
\[ \sum_{k=1}^{r_p} \left\lfloor \frac{2n}{p^k} \right\rfloor, \]
and the exact power of $p$ dividing $n!$ is 
\[ \sum_{k=1}^{r_p} \left\lfloor \frac{n}{p^k} \right\rfloor. \]
Thus, the exact power of $p$ dividing $\binom{2n}n$ is 
\[ \sum_{k=1}^{r_p} \left( \left\lfloor \frac{2n}{p^k} \right\rfloor -  \left\lfloor \frac{n}{p^k} \right\rfloor \right) \leq r_p, \] 
since $\lfloor 2a \rfloor - 2\lfloor a \rfloor \leq 1$ for all $a \in \R$. In particular, we have 
\[ \binom{2n}n \leq \prod_{p \leq 2n} p^{r_p} \leq (2n)^{\pi(2n)}. \]
Notice that 
\[ \binom{2n}n = \frac{2n \cdot (2n-1) \cdots (n+1)}{n \cdot (n-1) \cdots 1} = \frac{2n}n \cdots 
\frac{n+1}1 \geq 2^n. \]
Hence, we get $2^n \leq (2n)^{\pi(2n)}$. Now, we have 
\[ \pi(2n) \geq \left( \frac{\log 2}2 \right) \frac{2n}{\log(2n)}. \]
Recall that $\frac{x}{\log x}$ is increasing for $x > e$. If $x \geq 6$, choose $n \in \N$ such that 
$3x/4 \leq 2n \leq x$. We see that 
\[ \pi(x) \geq \pi(2n) \geq \left( \frac{\log 2}2 \right) \frac{2n}{\log(2n)} 
\geq \left( \frac{\log 2}2 \right) \frac{\frac34x}{\log(\frac34x)} > \frac{3\log 2}8 \frac{x}{\log x}. \]
One can manually check that the result holds for $2 \leq x \leq 6$, which finishes the proof of the lower 
bound. 

We now turn to the upper bound. Observe that 
\[ \prod_{n < p \leq 2n} p \; \bigg\rvert \; \binom{2n}n, \]
so by the binomial theorem, we have 
\[ \prod_{n < p \leq 2n} p \leq \binom{2n}n \leq (1+1)^{2n} = 2^{2n}. \]
On the other hand, notice that 
\[ \prod_{n < p \leq 2n} p \geq n^{\pi(2n) - \pi(n)}, \]
so it follows that 
\[ \pi(2n) \log n - \pi(n) \log(n/2) < (\log 2) 2n + (\log 2)\pi(n) < (3\log 2)n. \]
By taking $n = 2^k, 2^{k-1}, \dots, 4$, we obtain a telescoping collection of inequalities, given by 
\begin{align*}
    \pi(2^{k+1})\log2^k - \pi(2^k)\log2^{k-1} &< (3\log2)2^k, \\
    \pi(2^k)\log2^{k-1} - \pi(2^{k-1})\log2^{k-2} &< (3\log2)2^{k-1}, \\[-0.5em]
    &\;\;\vdots \\
    \pi(8)\log4-\pi(4)\log2 &< (3\log2)4.
\end{align*}
Putting these inequalities together, we have 
\[ \pi(2^{k+1})\log 2^k < (3\log 2)(2^k + 2^{k+1} + \cdots + 4) + \pi(4)\log2 < (3\log2)2^{k+1}, \]
and hence 
\[ \pi(2^{k+1}) < (3\log2) \left( \frac{2^{k+1}}{\log(2^k)} \right). \]
If $x > e$, choose $k$ such that $2^k \leq x \leq 2^{k+1}$. Then $\pi(x) \leq \pi(2^{k+1})$, and so 
\[ \pi(x) \leq (3\log2) \left( \frac{2^{k+1}}{\log(2^k)} \right) \leq (6\log2) \left( \frac{2^k}{\log(2^k)} \right) \leq (6\log 2) \left( \frac{x}{\log x} \right), \] 
where in the last equality, we use the fact that $\frac{x}{\log x}$ is increasing for $x > e$. 
The values $2 \leq x \leq e$ can be checked manually, proving the lower bound. 
\end{pf}

We should note that $\frac{3\log 2}8$ is in some sense arbitrary. In the proof, we could have picked
$n \in \N$ such that $1 - \eps \leq 2n \leq x$ instead of $3x/4 \leq 2n \leq x$ for $\eps$ 
arbitrarily small. However, this comes at the cost that the bound may potentially fail for small $x$,
and there is little purpose in a better lower bound for large $x$ as it is overshadowed by the 
Prime Number Theorem.

\subsection{Bertrand's Postulate}
In 1845, Bertrand showed that there is always a prime $p$ in the interval $[n, 2n]$ for $n \in \Z^+$
provided that $n < 6 \cdot 10^6$, and he had conjectured that this holds for all $n \in \Z^+$. Chebyshev 
proved that this was indeed the case in 1950. 
Note that this is not a trivial result; it doesn't occur for free just because $\pi(x) \sim x/\log x$. 

\begin{prop}{prop:1.14}
For all $n \in \Z^+$, we have 
\[ \prod_{p\leq n} p < 4^n. \]
\end{prop}
\begin{pf}
The result is clearly true for $n = 1$ and $n = 2$. Suppose that it holds for all $1 \leq n \leq k-1$. 
Note that we can restrict our attention to the case where $n$ is odd, because if $n$ is even 
and $n > 2$, then 
\[ \prod_{p \leq n} p = \prod_{p \leq n-1} p, \]
and the result will follow by induction. Write $n = 2m+1$ for some $m \in \Z^+$, and consider 
$\binom{2m+1}m$. In particular, we have 
\[ \prod_{m+1<p\leq2m+1} p \; \bigg\rvert \; \binom{2m+1}m. \]
Since $\binom{2m+1}m$ and $\binom{2m+1}{m+1}$ both appear in the binomial expansion of $(1+1)^{2m+1}$
with $\binom{2m+1}m = \binom{2m+1}{m+1}$, we obtain 
\[ \binom{2m+1}m \leq \frac12 (2^{2m+1}) = 4^m. \]
By our inductive hypothesis and the previous inequality, it follows that 
\[ \prod_{p\leq 2m+1} p = \left( \prod_{p\leq m+1} p \right) \left( \prod_{m+1 < p\leq 2m+1} p \right)
\leq 4^{m+1} 4^m = 4^{2m+1}. \qedhere \]
\end{pf}

\begin{lemma}{lemma:1.15}
If $n \geq 3$ and $p$ is a prime with $\frac23n < p \leq n$, then $p \nmid \binom{2n}n$. 
\end{lemma}
\begin{pf}
Since $n \geq 3$, we see that if $p$ is in the range $\frac23n < p \leq n$, then $p > 2$. Then 
$p$ and $2p$ are the only multiples of $p$ at most $2n$, and so 
\[ p^2 \; \| \; (2n)!, \]
where we write $p^k \; \| \; b$ to mean that $p^{k+1} \nmid b$ and $p^k \mid b$. Furthermore, since 
$\frac23n < p \leq n$, we have $p \;\|\; n!$ and hence $p^2 \;\|\; (n!)^2$. Using the identity 
\[ \binom{2n}n = \frac{(2n)!}{(n!)^2}, \] we see that $p \nmid \binom{2n}n$. 
\end{pf}

\begin{theo}[Chebyshev]{thm:1.16}
For every $n \in \Z^+$, there exists a prime $p$ satisfying $n < p \leq 2n$. 
\end{theo}
\begin{pf}
This argument was given by Erd\H{o}s. Note that the result holds for $1 \leq n \leq 3$. Assume that the 
result is false for some integer $n \geq 4$. By Lemma~\ref{lemma:1.15}, every prime dividing 
$\binom{2n}n$ is at most $\frac23n$. 

Let $p$ be a prime divisor of $\binom{2n}n$ where we have $p \leq \frac23n$. Suppose that 
$p^{\alpha_p} \;\|\; \binom{2n}n$ for some integer $\alpha_p$. Recall that in the proof of 
Theorem~\ref{thm:1.13}, we defined $r_p$ to be the integer satisfying $p^{r_p} \leq 2n < p^{r_p+1}$. 
Then we have $\alpha_p \leq r_p$, and hence $p^{\alpha_p} \leq p^{r_p} \leq 2n$. 

If $\alpha_p \geq 2$, then $p^2 \leq p^{\alpha_p} \leq 2n$ so that $p \leq \sqrt{2n}$. By 
Proposition~\ref{prop:1.14}, we have 
\[ \binom{2n}n \leq \left( \prod_{\substack{p\leq\frac23n\\ \alpha_p\leq 1}} p \right) 
\left( \prod_{\substack{p\leq\frac23n\\ \alpha_p\geq 2}} p \right) \leq 4^{2n/3} (2n)^{\pi(\sqrt{2n})}
\leq 4^{2n/3} (2n)^{\sqrt{2n}}. \]
Note that $\binom{2n}n$ is the largest of the $2n+1$ terms in the binomial expansion of 
\[ (1+1)^{2n} = \binom{2n}0 + \binom{2n}1 + \cdots + \binom{2n}{2n}, \]
so we get 
\[ \binom{2n}n \geq \frac{2^{2n}}{2n+1}. \]
Combining the above inequalities gives 
\[ \frac{4^n}{2n+1} \leq \binom{2n}n \leq 4^{2n/3} (2n)^{\sqrt{2n}}, \]
which implies that 
\[ 4^{n/3} \leq (2n)^{\sqrt{2n}} (2n+1) < (2n)^{\sqrt{2n}+2}. \]
One can check manually that the result holds for $4 \leq n \leq 16$, so assume that $n > 16$. Taking 
logarithms, we find that 
\[ \frac{n}3 \log 4 < (\sqrt{2n}+2)\log(2n) < 2\sqrt n \log(2n) < 2\sqrt n \log(n^{5/4}) < \frac52 \sqrt n \log n. \]
Notice that $\frac{\sqrt n}{\log n}$ is increasing for 
$n > e^2$. Putting this together with the fact that 
\[ \frac{\sqrt{1600}}{\log 1600} \approx 5.421 > 5.410 \approx \frac{15}{2\log4}, \]
we have $n \leq 1600$. Finally, we know that $\{2, 3, 5, 7, 13, 23, 43, 83, 163, 317, 557, 1109, 2207\}$
are all primes, where each number in the set is the largest prime less than twice the previous one. 
Thus, no counterexample exists, and the result holds for all $n \geq 4$.
\end{pf}

\subsection{Gaps Between Twin Primes}
By Theorem~\ref{thm:1.16}, we have 
\[ p_{n+1} - p_n \leq p_n \]
as there is a prime between $p_n$ and $2p_n$. What more can we say about differences of consecutive primes?

By the Prime Number Theorem, there are about $x/\log x$ primes $p$ at most $x$. Therefore, the 
``average gap'' between primes $p$ at most $x$ is $\log x$. However, the value of $p_{n+1} - p_n$ 
can vary widely. 

Notice that for any $n \geq 2$, the numbers $n! + k$ for $2 \leq k \leq n$ are all composite. This 
implies that 
\[ \limsup_{n\to\infty} \;(p_{n+1} - p_n) = \infty. \]
In 1931, Weszynthius showed that 
\[ \limsup_{n\to\infty} \left( \frac{p_{n+1} - p_n}{\log p_n} \right) = \infty. \]
By probabilistic reasoning, Cramer had conjectured in 1936 that 
\[ \limsup_{n\to\infty} \left( \frac{p_{n+1} - p_n}{(\log p_n)^2} \right) \leq 1. \]
In the 1930s, Erd\H{o}s proved that for infinitely many integers $n$, we have 
\[ p_{n+1} - p_n > c \log p_n \frac{\log\log p_n}{(\log\log\log p_n)^2} \]
for some positive constant $c$. In 1938, Rankin added a factor of $\log\log\log\log p_n$. 

What about small gaps between consecutive primes? The famous Twin Prime Conjecture states that 
there are infinitely many $n \in \Z^+$ such that $p_{n+1} - p_n = 2$. Equivalently, it can be stated that
\[ \liminf_{n\to\infty} \; (p_{n+1} - p_n) = 2. \]
If we assume that the primes are randomly distributed and an integer is prime with 
probability $1/\log x$, then we might expect $x$ and $x+2$ to both be prime with probability 
$1/(\log x)^2$. 

Therefore, we expect about $x/(\log x)^2$ primes $p$ such that $p+2$ is also prime and $p \leq x$. 
A more careful heuristic suggests that there are about $C x/(\log x)^2$ such primes $p$ where 
$C > 0$ and $C \neq 1$. In the 1960s, Chen proved that there are more than $0.6x/(\log x)^2$ 
primes $p$ with $p \leq x$ such that $p+2$ is a product of at most two primes (called a $P_2$), 
provided that $x$ is sufficiently large. 

In 2005, Goldston, Pintz, and Yildirim showed that 
\[ \liminf_{n\to\infty} \left( \frac{p_{n+1} - p_n}{\log p_n} \right) = 0. \]
However, this is still quite far from the Twin Prime Conjecture; the bound between consecutive primes 
can still go to infinity. 

Astoundingly, Zhang made a breakthrough in 2013 and showed that 
\[ \liminf_{n\to\infty} \; (p_{n+1} - p_n) \leq 7 \cdot 10^7. \]
This was independently improved by Tao and Maynard (via the Polymath Project) in the same year to get 
\[ \liminf_{n\to\infty} \; (p_{n+1} - p_n) \leq 246. \]\newpage 
\section{Asymptotic Analysis for $\pi(x)$}\label{sec:2}

\subsection{The M\"obius Function}\label{subsec:2.1}

\begin{defn}\label{def:2.1}
Let $f$ and $g$ be functions from $\N$ or $\R^+$ to $\R$, and suppose that $g$ maps to $\R^+$. 
\begin{enumerate}[(1)]
    \item We write $f = O(g)$ if there exist constants $c_1, c_2 > 0$ such that for all $x > c_1$, 
    we have $|f(x)| \leq c_2 g(x)$. 
    \item We write $f = o(g)$ if $\lim_{n\to\infty} f(n)/g(n) = 0$. 
    \item We write $f \sim g$ if $\lim_{n\to\infty} f(n)/g(n) = 1$, and we say that $f$ is 
    {\bf asymptotic} to $g$. 
\end{enumerate}
\end{defn}

By the Prime Number Theorem, we have $\pi(x) \sim x/\log x$, or equivalently, 
\begin{equation}
    \pi(x) = \frac{x}{\log x} + o\left( \frac{x}{\log x} \right). \label{eq:2.1}
\end{equation}
\begin{remark}\label{remark:2.2}
Let $\eps > 0$. Then the number of primes in the interval $[x, (1+\eps)x]$ is 
\[ \pi((1+\eps)x) - \pi(x) = \frac{(1+\eps)x}{\log((1+\eps)x)} - \frac{x}{\log x} + o\left( \frac{x}{\log x} \right). \]
Notice that 
\[ \frac{(1+\eps)x}{\log((1+\eps)x)} = \frac{(1+\eps)x}{\log x + \log(1+\eps)} 
= \frac{(1+\eps)x}{(\log x)(1 + \log(1+\eps)/\log x)} = \frac{(1+\eps)x}{\log x} + o\left( \frac{x}{\log x} \right). \]
Therefore, it follows that 
\[ \pi((1+\eps)x) - \pi(x) = \frac{(1+\eps)x}{\log x} - \frac{x}{\log x} + o\left( \frac{x}{\log x} \right) = \frac{\eps x}{\log x} + o\left( \frac{x}{\log x} \right). \]
By taking $\eps = 1$, we have 
\begin{equation}
    \pi(2x) - \pi(x) = \frac{x}{\log x} + o\left( \frac{x}{\log x} \right). \label{eq:2.2}
\end{equation} 
Equation (\ref{eq:2.2}) might look odd together with equation (\ref{eq:2.1}). Nonetheless, the result is 
correct; it's just that the bounds in the notation $o$ are different. 
\end{remark}

\begin{defn}\label{def:2.3}
We define the {\bf M\"obius function} on $\N$ by 
\[ \mu(n) = \begin{cases} 1 & \text{if } n = 1, \\ 0 & \text{ if $n$ is not squarefree,} \\ 
(-1)^r & \text{if $n$ is a product of $r$ distinct primes.} \end{cases} \]
\end{defn}
For example, we have $\mu(48) = \mu(2^4 \cdot 3) = 0$ and $\mu(30) = \mu(2 \cdot 3 \cdot 5) = 
(-1)^3 = -1$. 

\begin{prop}\label{prop:2.4}
We have 
\[ \sum_{d \mid n} \mu(d) = \begin{cases} 1 & \text{if } n = 1, \\ 0 & \text{otherwise,} \end{cases} \]
where $\sum_{d\mid n}$ means that the summation runs through the positive divisors $d$ of $n$. 
\end{prop}
\begin{pf}
The result is true for $n = 1$. For $n > 1$, let $n = p_1^{a_1} \cdots p_r^{a_r}$ be the unique 
factorization of $n$ into distinct prime numbers. Set $N = p_1 \cdots p_r$ (which is called the 
{\bf radical} of $n$). Since $\mu(d) = 0$ when $d$ is not squarefree, we have 
\[ \sum_{d\mid n} \mu(d) = \sum_{d\mid N} \mu(d). \]
Note that the divisors of $N$ are in bijective correspondence with the subsets of $\{p_1, \dots, p_r\}$. 
Since the number of $k$ element subsets is $\binom{r}{k}$ and the corresponding divisor $d$ of such 
a set satisfies $\mu(d) = (-1)^k$, we have 
\[ \sum_{d \mid n} \mu(d) = \sum_{d \mid N} \mu(d) = \sum_{k=0}^r \binom rk (-1)^k = (1-1)^r = 0. \qedhere \]
\end{pf}

\begin{prop}[M\"obius Inversion Formula]\label{prop:2.5}~
\begin{enumerate}[(1)]
    \item For two functions $f, g : \R^+ \to \C$, we have 
    \[ g(x) = \sum_{1 \leq n \leq x} f(x/n) \]
    if and only if 
    \[ f(x) = \sum_{1 \leq n \leq x} \mu(n) g(x/n). \]
    \item For two functions $f, g : \N \to \C$, we have 
    \[ f(n) = \sum_{d \mid n} g(d) \]
    if and only if 
    \[ g(n) = \sum_{d \mid n} \mu(d) f(n/d). \]
\end{enumerate}
\end{prop}
\begin{pf}
This is on Homework 1.
\end{pf}

\subsection{The von Mangoldt Function}\label{subsec:2.2}

\begin{defn}\label{def:2.6}
We define the {\bf von Mangoldt function} on $\N$ by 
\[ \Lambda(n) = \begin{cases} \log p & \text{if $n = p^k$ for $p$ prime and $k \in \N$,} \\ 0 & \text{otherwise.} \end{cases} \]
Moreover, for all $x \in \R$, we define the functions 
\begin{align*}
    \theta(x) &= \sum_{p\leq x} \log p = \log \prod_{p \leq x} p, \\ 
    \psi(x) &= \sum_{p^k \leq x} \log p = \sum_{n \leq x} \Lambda(n). 
\end{align*}
\end{defn}

Notice that 
\[ \psi(x) = \sum_{p\leq x} \left\lfloor \frac{\log x}{\log p} \right\rfloor \log p. \]
Since $p^2 \leq x$ is equivalent to $p \leq x^{1/2}$ and $p^3 \leq x$ if and only if $p \leq x^{1/3}$, 
we see that 
\[ \psi(x) = \theta(x) + \theta(x^{1/2}) + \theta(x^{1/3}) + \cdots. \]
Note that $\theta(x^{1/m}) = 0$ when $m > \frac{\log x}{\log 2}$. Therefore, we get 
\[ \psi(x) = \sum_{k=1}^{\left\lfloor \frac{\log x}{\log 2} \right\rfloor} \theta(x^{1/k}). \]
Observe that we have the inequality 
\[ \theta(x) = \sum_{p\leq x} \log p \leq x \log x, \]
so it follows that 
\[ \sum_{k \geq 2} \theta(x^{1/k}) = O\left( x^{1/2} (\log x)^2 \right). \]
Therefore, we obtain 
\[ \psi(x) = \theta(x) + O\left( x^{1/2} (\log x)^2 \right) \]
and so by Theorem~\ref{thm:1.12}, we get 
\[ \theta(x) = \sum_{p\leq x}\log p \leq \pi(x) \log x < c_1 x \]
for $x \geq 2$ and a constant $c_1 > 0$. Similarly, one finds that $\psi(x) < c_2x$ for $x \geq 2$
and a positive constant $c_2$. Furthermore, from the proof of Theorem~\ref{thm:1.12}, we have $2^n \leq \binom{2n}n$
and $\binom{2n}n \mid \prod_{p \leq 2n} p^{r_p}$, where $r_p$ is the integer satisfying 
$p^{r_p} \leq 2n < p^{r_p+1}$. It follows that 
\[ n\log 2 = \log(2^n) \leq \log \binom{2n}n \leq \sum_{p\leq 2n} r_p \log p \leq 
\sum_{p \leq 2n} \left\lfloor \frac{\log(2n)}{\log p} \right\rfloor \log p \leq \psi(2n). \]
For $x \geq 2$, choosing $n$ such that $2n \leq x < 2n+2$ gives 
\[ \psi(x) \geq \psi(2n) \geq n\log 2 > \frac{x-2}2 \log 2. \]
Hence, we have $\psi(x) > c_3x$ and $\theta(x) > c_4x$ for positive constants $c_3$ and $c_4$. 

What is the relationship between $\theta(x)$, $\psi(x)$, and $\pi(x)$? We note that 
\[ \theta(x) = \sum_{p \leq x} \log p \leq x \log p \leq \pi(x) \log x, \]
so it follows that 
\[ \pi(x) \geq \frac{\theta(x)}{\log x} > c_4 \frac{x}{\log x}. \]

\begin{thm}\label{thm:2.7}
We have 
\[ \pi(x) \sim \frac{\theta(x)}{\log x} \sim \frac{\psi(x)}{\log x}. \]
\end{thm}
\begin{pf}
Since $\psi(x) = \theta(x) + O(x^{1/2} (\log x)^2)$ and $\theta(x) > c_4x$, we see that $\theta(x) 
\sim \psi(x)$. In particular, we have $\theta(x)/\log x \sim \psi(x)/\log x$, so it only 
remains to show that $\pi(x) \sim \theta(x)/\log x$. 

We have already shown that $\pi(x) \geq \theta(x) \geq \log x$, so 
\[ \liminf_{n\to\infty} \frac{\pi(x) \log x}{\theta(x)} \geq 1. \]
We need an upper bound for $\pi(x)$ in terms of $\theta(x)$. Note that for any $\delta > 0$, we have 
\[ \theta(x) = \sum_{p\leq x} \log p \geq \log(x^{1-\delta}) \sum_{x^{1-\delta} \leq p \leq x} 1 
\geq (1-\delta)(\log x)\left(\pi(x) - \pi(x^{1-\delta})\right). \]
Since $\pi(y) \leq y$ for all real numbers $y > 0$, we get 
\[ \theta(x) + (1-\delta)x^{1-\delta} \log x \geq (1-\delta)(\log x)\pi(x). \]
Rearranging the above gives 
\[ \frac{\theta(x)}{(1-\delta)\log x} + x^{1-\delta} \geq \pi(x), \]
and therefore 
\[ \frac{1}{1-\delta} + \frac{x^{1-\delta}\log x}{\theta(x)} \geq \frac{\pi(x)\log x}{\theta(x)}. \]
Given $\eps > 0$, we can choose $\delta > 0$ such that $\frac{1}{1-\delta} < 1 + \frac{\eps}2$, 
and then pick $x_0$ such that if $x > x_0$, then 
\[ \frac{x^{1-\delta}\log x}{\theta(x)} < \frac\eps2 \] 
since $\theta(x) > c_1x$ for $x \geq 2$. Then for all $x > x_0$, we have 
\[ 1 \leq \frac{\pi(x) \log x}{\theta(x)} < 1 + \eps, \]
which completes the proof.
\end{pf}

\subsection{Abel's Summation Formula}\label{subsec:2.3}
We will prove Abel's summation formula and give some of its
applications.

\begin{lemma}[Abel's summation formula]\label{lemma:2.8}
Let $\{a_n\}_{n=1}^\infty$ be a sequence of complex numbers. 
Let $f : \{x \in \R : x \geq 1\} \to \C$ be a function. 
For all $x \geq 1$, we define 
\[ A(x) := \sum_{n\leq x} a_n, \]
where the summation runs through all positive integers up to 
$x$. If $f'$ is continuous at every $x \geq 1$, then 
\[ \sum_{n\leq x} a_n f(n) = A(x) f(x) - \int_1^x A(u)f'(u)\dd u. \]
\end{lemma}
\begin{pf}
Set $N = \lfloor x \rfloor$. Note that 
$a_n = A(n) - A(n-1)$ for all $n \geq 2$, so we can write 
\begin{align*}
    \sum_{n\leq N} a_n f(n) &= A(1)f(1) + 
    (A(2) - A(1))f(2) + \cdots + (A(N) - A(N-1)) f(N) \\
    &= A(1)(f(1) - f(2)) + \cdots + A(N-1)(f(N-1) - f(N)) + A(N)f(N).
\end{align*}
Observe that if $i \in \Z^+$ and $t \in \R$ with $i \leq t < i+1$, then $A(t) = A(i)$. 
It follows that 
\[ A(i) (f(i) - f(i+1)) = -\int_i^{i+1} A(u) f'(u)\dd u. \]
Therefore, we have 
\[ \sum_{n\leq N} a_n f(n) = -\int_1^N A(u)f'(u)\dd u + 
A(N)f(N), \]
so the result holds when $x$ is an integer. Now, notice that 
$A(t) = A(N)$ for all $x \geq t \geq N$, so we obtain 
\[ \int_N^x A(u) f'(u)\dd u = A(x)(f(x) - f(N)) = 
A(x)f(x) - A(N)f(N). \]
Thus, the result holds for all $x \geq 1$.
\end{pf}

\begin{defn}\label{def:2.9}
Given $x \in \R$, we denote the {\bf fractional part} of 
$x$ by $\{x\}$; that is, 
\[ \{x\} := x - \lfloor x \rfloor. \]
We define {\bf Euler's constant} by 
\[ \gamma := 1 - \int_1^\infty \frac{\{t\}}{t^2}\dd t 
= 1 - \int_1^\infty \frac{t - \lfloor t \rfloor}{t^2}\dd t. \]
Note that $\gamma \approx 0.55721$. 
\end{defn}

This has not been proven, but it has been conjectured that 
$\gamma$ is irrational and transcendental.

\begin{thm}\label{thm:2.10}
We have 
\[ \sum_{n \leq x} \frac1n = \log x + \gamma + O
\left( \frac1x \right). \]
\end{thm}
\begin{pf}
Taking $a_n = 1$ and $f(t) = 1/t$ in Abel's summation formula,
we have 
\[ A(x) = \sum_{n\leq x} a_n = \sum_{n \leq x} 1 = \floor{x} \]
so that 
\begin{align*}
    \sum_{n\leq x} \frac1n &= \frac{\floor{x}}{x} + \int_1^x \frac{\floor{u}}{u^2}\dd u \\
    &= \frac{x - (x - \floor x)}{x} + \int_1^x \frac{u - (u - \floor u)}{u^2}\dd u \\
    &= 1 + O\left(\frac1x\right) + \int_1^x \frac{{\rm d}u}u -
    \int_1^x \frac{u - \floor u}{u^2}\dd u \\
    &= 1 + O\left(\frac1x\right) + \log x - \left( \int_1^\infty \frac{u-\floor u}{u^2}\dd u - \int_x^\infty \frac{u-\floor u}{u^2}\dd u \right) \\
    &= \log x + \gamma + O\left(\frac1x\right) + 
    \int_x^\infty \frac{u-\floor u}{u^2}\dd u \\
    &= \log x + \gamma + O\left(\frac1x\right) 
    + O\left(\int_x^\infty \frac1{u^2}\dd u \right) \\
    &= \log x + \gamma + O\left(\frac1x\right). \qedhere
\end{align*}
\end{pf}

\begin{thm}\label{thm:2.11}
We have 
\[ \sum_{n \leq x} \frac{\Lambda(n)}n = \log x + O(1). \]
\end{thm}
\begin{pf}
First, we apply Abel's summation formula with $a_n = 1$ 
and $f(n) = \log n$ to get 
\begin{align*}
    \sum_{n\leq x} \log n
    &= \floor x \log x - \int_1^x \frac{\floor u}u\dd u \\
    &= (x - (x - \floor x))\log x - \int_1^x 
    \frac{u - (u - \floor u)}u \dd u \\
    &= x\log x + O(\log x) - (x-1) + \int_1^x \frac{u - \floor u}u\dd u \\
    &= x\log x - x + O(\log x).
\end{align*}
On the other hand, we have 
\begin{align*}
    \sum_{n\leq x} \log n 
    = \log(\floor{x}!) 
    &= \sum_{p\leq x} \left( \sum_{k=1}^\infty \floor*{\frac{x}{p^k}} \right)\log p \\
    &= \sum_{p^m \leq x} \floor*{\frac{x}{p^m}} \log p \\
    &= \sum_{n \leq x} \floor*{\frac xn} \Lambda(n) \\
    &= \sum_{n \leq x} \frac xn \Lambda(n) - 
    \sum_{n\leq x} \left( \frac xn - \floor*{\frac xn} \right)
    \Lambda(n) \\
    &= x \sum_{n\leq x} \frac{\Lambda(n)}n - O\left( \sum_{n\leq x} \Lambda(n) \right). 
\end{align*}
Since $\sum_{n\leq x} \Lambda(n) = \psi(x) = O(x)$, we have 
\[ \sum_{n\leq x} \log x = x \sum_{n \leq x} \frac{\Lambda(n)}n - O(n). \]
By the asymptotic formula of $\sum_{n\leq x} \log n$ above, we see that 
\[ x\log x - x + O(\log x) = x\sum_{n\leq x} \frac{\Lambda(n)}n - O(x). \]
Rearranging and tucking some terms under $O(x)$ gives 
\[ x \sum_{n\leq x} \frac{\Lambda(n)}n = x\log x + O(x). \]
Finally, dividing through by $x$ gives 
\[ \sum_{n\leq x} \frac{\Lambda(n)}n = \log x + O(1). \qedhere \]
\end{pf}

\begin{thm}\label{thm:2.12}
We have
\[ \sum_{p\leq x} \frac{\log p}p = \log x + O(1). \]
\end{thm}
\begin{pf}
Note that 
\[ \sum_{p \leq x} \frac{\log p}p 
= \sum_{n \leq x} \frac{\Lambda(n)}n - \sum_{m\geq2} \sum_{p^m\leq x} \frac{\log p}{p^m} \\
= \log x + O(1) - \sum_{m\geq2}\sum_{p^m\leq x}
\frac{\log p}{p^m}. \]
Moreover, we see that 
\[ \sum_{m\geq2}\sum_{p^m\leq x}
\frac{\log p}{p^m} \leq \sum_p \left( \frac1{p^2} + \frac1{p^3} + \cdots \right)\log p 
\leq \sum_p \frac{\log p}{p(p-1)} 
\leq \sum_{n=2}^\infty \frac{\log n}{n(n-1)} = O(1), \]
which completes the proof.
\end{pf}

\begin{thm}[Merten]\label{thm:2.13}
There exists a real number $\beta$ such that 
\[ \sum_{p \leq x} \frac1p = \log\log x + \beta + O\left(\frac1{\log x}\right). \]
\end{thm}
\begin{pf}
We apply Abel's summation formula with 
\[ a_n = \begin{cases} \frac{\log p}p & \text{if $n = p$ for a prime $p$} \\ 0 & \text{otherwise} \end{cases} \]
and $f(n) = 1/\log n$. Setting $A(x) = \sum_{n\leq x} a_n$, we have 
\[ \sum_{p \leq x} \frac1p = \frac{A(x)}{\log x} + 
\int_1^x \frac{A(u)}{u(\log u)^2}\dd u. \]
By Theorem~\ref{thm:2.12}, we have 
\[ A(x) = \sum_{p\leq x} \frac{\log p}p = \log x + O(1), \]
so we see that 
\[ \sum_{p\leq x} \frac1p = 1 + O\left(\frac1{\log x}\right)
+ \int_2^x \frac{\log u + \tau(u)}{u(\log u)^2}\dd u, \]
where $\tau(u) = A(u) - \log u = O(1)$. Therefore, we have 
\begin{align*}
    \sum_{p\leq x} \frac1p 
    &= 1 + O\left( \frac1{\log x} \right) + \log\log x 
    - \log\log 2 + \int_2^x \frac{\tau(u)}{u(\log u)^2}\dd u \\
    &= \log\log x + 1 - \log\log 2 + \int_2^\infty 
    \frac{\tau(u)}{u(\log u)^2}\dd u - 
    \int_x^\infty \frac{\tau(u)}{u(\log u)^2}\dd u + 
    O \left( \frac1{\log x} \right). 
\end{align*}
By setting $\beta$ to the middle terms above, we are done.
\end{pf}

In fact, we have 
\[ \beta = \gamma + \sum_p \left[ \log \left( 1 - \frac1p \right) + \frac1p \right] \approx 0.261497, \]
and $\beta$ is called {\bf Merten's constant}.\newpage 
\newpage 
\section{Riemann's Zeta Function and the Prime Number Theorem}

\subsection{The Riemann Zeta Function}
In order to prove the Prime Number Theorem, we need to 
first introduce the Riemann zeta function. 

\begin{defn}
For $s \in \C$ with $\Re(s) > 1$, we define the 
{\bf Riemann zeta function} by 
\[ \zeta(s) := \sum_{n=1}^\infty \frac1{n^s}. \]
We will denote $s = \sigma + it$ where $\sigma, t \in \R$. 
\end{defn}

Note that the series $\sum_{n=1}^\infty n^{-s}$ converges 
absolutely when $\Re(s) > 1$. 

Recall that the infinite product $\prod_n (1 + a_n)$ 
converges absolutely (that is, it is finite and non-zero)
if and only if $\sum_n |a_n|$ converges. We have the 
{\bf Euler product representation} of $\zeta(s)$ given 
in the following lemma. 

\begin{lemma}[Euler product]
For $s \in \C$ with $\Re(s) > 1$, we have 
\[ \prod_p \left( 1 - \frac1{p^s} \right)^{\!-1} = 
\sum_{n=1}^\infty \frac1{n^s}. \]
\end{lemma}
\begin{pf}
Note that 
\[ \prod_p \left( 1 - \frac1{p^s} \right)^{\!-1} = 
\prod_p \left(1 + \frac1{p^2} + \frac1{p^3} + \cdots \right). \]
A typical term in the sum is of the form 
\[ \frac{1}{p_1^{\alpha_1s} \cdots p_k^{\alpha_ks}}
= \frac{1}{(p_1^{\alpha_1} \cdots p_k^{\alpha_k})^s}. \]
By the Fundamental Theorem of Arithmetic, every positive 
integer can be expressed uniquely as a product of primes, 
so the identity holds.
\end{pf}

\begin{thm}
$\zeta(s)$ can be analytically continued to $s \in \C$ with 
$\Re(s) > 0$ and $s \neq 1$. It is analytic except 
at the point $s = 1$ where it has a simple pole with residue
$1$. 
\end{thm}
\begin{pf}
For $s \in \C$ with $\Re(s) > 1$, we have 
$\zeta(s) = \sum_{n=1}^\infty n^{-s}$. By Abel's 
summation formula with $a_n = 1$ and $f(x) = x^{-s}$, 
we find that 
\[ \sum_{n\leq x} \frac{1}{n^s} = \frac{\floor x}{x^s}
+ s\int_1^x \frac{\floor u}{u^{s+1}}\dd u. \]
Letting $x \to \infty$, we obtain 
\begin{align*}
    \zeta(s) &= 0 + s\int_1^\infty \frac{\floor u}{u^{s+1}}\dd u \\
    &= s \int_1^\infty \frac{u - (u - \floor u)}{u^{s+1}}\dd u \\
    &= s \int_1^\infty \frac{u}{u^{s+1}}\dd u - 
    s \int_1^\infty \frac{u - \floor u}{u^{s+1}}\dd u \\
    &= s \left( \frac{u^{1-s}}{1-s} \biggr\rvert_1^\infty 
    \right)
    - s \int_1^\infty \frac{u - \floor u}{u^{s+1}}\dd u \\
    &= \frac{s}{s-1} - s \int_1^\infty \frac{u - \floor u}{u^{s+1}}\dd u
\end{align*}
for $\Re(s) > 1$. Note that 
\[ \int_1^\infty \frac{u - \floor u}{u^{s+1}}\dd u \]
converges for $\Re(s) > 0$ and represents an analytic function. Therefore, we see that 
\[ \frac{s}{s-1} - s \int_1^\infty \frac{u - \floor u}{u^{s+1}}\dd u \] 
is an analytic function for $\Re(s) > 0$ with $s \neq 1$. 
This gives a meromorphic continuation of $\zeta(s)$ 
to the region $\{s \in \C : \Re(s) > 0\}$. Finally, note that 
$\frac{s}{s-1}$ has a simple pole with residue $1$ at $s = 1$.
\end{pf}

\begin{thm}
$\zeta(s)$ has no zeroes in the region $\{s \in \C : 
\Re(s) \geq 1\}$. 
\end{thm}
\begin{pf}
If $\Re(s) > 1$, then $\prod_p (1 - \frac1{p^s})^{-1}$ 
converges, so $\zeta(s) \neq 0$. 

It only remains to consider the case where $\Re(s) = 1$. 
We will first do some preliminary work. 

Recall that we denote $s = \sigma + it$ where $\sigma, t \in \R$. Let $\sigma > 1$. Then for all $ t \in \R$, we have 
\[ \log^*(\zeta(\sigma + it)) = 
\log \left( \prod_p \left(1 + \frac1{p^s}\right)^{\!-1}\right) = \sum_p \sum_{n=1}^\infty \frac1n \left( \frac1{p^{ns}} \right), \]
where $\log$ denotes the principal branch and 
$\log^*$ denotes some branch of the logarithm (we have to be
careful here as we are considering complex numbers). Comparing
the real parts of the above equality, we have 
\[ \log|\zeta(\sigma+it)| = \sum_p \sum_{n=1}^\infty 
\frac{p^{-\sigma n} \cos(nt \log p)}{n}, \]
since we can write 
\[ p^{-int} = e^{-int \log p} = \cos(-nt \log p) + 
i \sin(-nt \log p) = \cos(nt \log p) - i\sin(nt \log p) \]
and therefore $\Re(p^{-int}) = \cos(nt \log p)$. 
Moreover, observe that we have the inequality 
\begin{align*}
    0 \leq 2(1+\cos\theta)^2 
    &= 2(1 + 2\cos\theta + \cos^2\theta) \\
    &= 2 + 4\cos\theta + 2\cos^2\theta \\
    &= 3 + 4\cos\theta + (2\cos^2\theta - 1) \\
    &= 3 + 4\cos\theta + \cos(2\theta). 
\end{align*}
From this, we can deduce that 
\[ \sum_p \sum_{n=1}^\infty \frac{p^{-\sigma n}}n 
(3 + 4\cos(nt \log p) + \cos(2nt \log p)) \geq 0. \]
Therefore, we have 
\[ \log|\zeta(\sigma)|^3 + \log|\zeta(\sigma+it)|^4 + 
\log|\zeta(\sigma+2it)| \geq 0. \]
In particular, we see that 
\begin{equation}
    |\zeta(\sigma)|^3 \cdot |\zeta(\sigma + it)|^4 
\cdot |\zeta(\sigma+2it)| \geq 1
\end{equation} 
for $\sigma > 1$ and $t \in \R$. 

Suppose now that $1 + it_0$ is a zero of $\zeta(s)$, 
and note that $t_0 \neq 0$ as $\zeta(s)$ has a pole at 
$s = 1$. By taking $t \to 1^+$ (that is, from the right), 
we observe tht 
\[ |\zeta(s)| = O((\sigma - 1)^{-1}) \]
since $1$ is a simple pole of $\zeta(s)$. Moreover, 
since $1 + it_0$ is a zero of $\zeta(s)$, we have 
$|\zeta(\sigma + it_0)| = O(\sigma-1)$ as 
$\sigma \to 1^+$. Finally, we have 
$|\zeta(\sigma + 2it_0)| = O(1)$ as $\sigma \to 1^+$ 
since $1 + 2it_0$ is not a simple pole of $\zeta(s)$. 
It follows that 
\[ |\zeta(\sigma)|^3 \cdot |\zeta(\sigma + it)|^4 
\cdot |\zeta(\sigma + 2it)| = 
O((\sigma - 1)^{-3}) \cdot O((\sigma - 1)^4) \cdot O(1) 
= O(\sigma - 1). \]
Thus, $|\zeta(s)|^3 \cdot |\zeta(\sigma+it)|^4 \cdot 
|\zeta(\sigma + 2it)|$ tends to $0$ as $\sigma \to 1^+$. 
But this contradicts that the lower bound we found in (3.1), 
so we conclude that $\zeta(s)$ cannot have a zero when 
$\Re(s) = 1$. 
\end{pf}

\subsection{Newman's Theorem}

\begin{thm}[Newman]
Let $\{a_n\}_{n=1}^\infty$ be a sequence of complex numbers with $|a_n| \leq 1$ for all $n \geq 1$. 
Consider the series $\sum_{n=1}^\infty a_n/n^s$, which converges to an analytic function 
$F(s)$ for $\Re(s) > 1$. If $F(s)$ can be analytically continued to $\Re(s) \geq 1$, then 
$\sum_{n=1}^\infty a_n/n^s$ converges to $F(s)$ for $\Re(s) \geq 1$. 
\end{thm}
\begin{pf}
Let $w \in \C$ with $\Re(w) \geq 1$. Then $F(z+w)$ is analytic for $\Re(z) \geq 0$. Choose 
$R \geq 1$ and let $\delta = \delta(R) > 0$ so that $F(z+w)$ is analytic on the region 
\[ \tilde\Gamma := \{z \in \C : \Re(z) \geq -\delta \text{ and } |z| \leq R\}. \]
To see why such a $\delta > 0$ exists, first note that $F(z+w)$ is analytic for $\Re(z) \geq 0$. 
Consider the line $L = \{z = iy : |y| \leq R\}$. Every point in $L$ has an open cover such that 
$F(z+w)$ is analytic on that cover; call the union of these covers $U$. Since $L$ is compact\footnote{Recall that a set $X$ is compact if every open cover of $X$ has a finite subcover.}, there exists a finite open subcover $\tilde U$ of $U$ such that 
$L \subseteq \tilde U \subseteq U$. Since the number of open sets in $\tilde U$ is finite, 
it follows that such a $\delta > 0$ exists.

Let $M$ denote the maximum of $|F(z+w)|$ on $\tilde \Gamma$, and let $\Gamma$ denote the contour 
obtained by following the outside of $\tilde\Gamma$ in a counterclockwise path. Let $A$ be the 
part of $\Gamma$ in $\Re(z) > 0$, and let $B = \Gamma \setminus A$. For $N \in \N$, consider the 
function 
\[ F(z+w)N^z \left( \frac1z + \frac z{R^2} \right), \]
which is analytic on $\tilde\Gamma$ except at $z = 0$ where there is a simple pole with residue 
$F(0+w) N^0 = F(w)$. By Cauchy's residue theorem, we obtain 
\begin{align*}
    2\pi i F(w) &= \int_\Gamma F(z+w) N^z \left( \frac1z + \frac z{R^2} \right)\dd z \\
    &= \int_A F(z+w) N^z \left( \frac1z + \frac z{R^2} \right)\dd z + 
    \int_B F(z+w) N^z \left( \frac1z + \frac z{R^2} \right)\dd z. \tag{3.2} 
\end{align*}
Observe that $F(z+w)$ is equal to its series on $A$. We split the series as 
\[ S_N(z+w) = \sum_{n=1}^N \frac{a_n}{n^{z+w}} \]
and $R_N(z+w) = F(z+w) - S_N(z+w)$. Note that $S_N(z+w)$ is analytic for all $z \in \C$. Let 
$C$ be the contour given by the path $|z| = R$ taken in the counterclockwise direction. 
By Cauchy's residue theorem, we obtain 
\[ 2\pi i S_N(w) = \int_C S_N(z+w) N^z \left( \frac1z + \frac z{R^2} \right)\dd z \]
since the integrand has a simple pole at $z = 0$ with residue $S_N(0+w) N^0 = S_N(w)$. Note that 
\[ C = A \cup (-A) \cup \{iR, -iR\}. \]
Therefore, we see that 
\[ 2\pi i S_N(w) = \int_A S_N(z+w) N^z \left( \frac1z + \frac z{R^2} \right)\dd z 
+ \int_{-A} S_N(z+w) N^z \left( \frac1z + \frac z{R^2} \right)\dd z. \]
Consider the second integral above. Using the change of variables $z \to -z$, we find that 
\[ \int_{-A} S_N(z+w) N^z \left( \frac1z + \frac z{R^2} \right)\dd z 
= \int_A S_N(-z+w) N^{-z} \left( \frac1z + \frac z{R^2} \right) \dd z. \]
Thus, we obtain 
\[ 2\pi i S_N(w) = \int_A \left( S_N(z+w) N^z + S_N(-z + w) N^{-z} \right) \left( \frac1z + \frac{z}{R^2} \right)\dd z. \]
Combining the above equality with $(3.2)$, we have 
\begin{align*}
    2\pi i(F(w) - S_N(w)) &= \int_A \left( R_N(z+w) N^z - S_N(-z+w) N^{-z} \right) \left( \frac1z + \frac z{R^2} \right) \dd z\\ &\hspace{1.5cm} + \int_B F(z+w) N^z \left( \frac1z + \frac z{R^2} \right)\dd z. \tag{3.3}
\end{align*}  
Our goal is to show that $S_N(w)$ converges to $F(w)$ as $N \to \infty$. Write $z = x + iy$
where $x, y \in \R$. Then for $z \in A$, we have $x > 0$ and $|z| = R$, so 
\[ \frac1z + \frac z{R^2} = \frac{x-iy}{R^2} + \frac{x + iy}{R^2} = \frac{2x}{R^2}. \]
Since $|n^z| = n^x$, we have 
\[ |R_N(z+w)| \leq \sum_{n=N+1}^\infty \frac{1}{n^{\Re(z+w)}} \leq \sum_{n=N+1}^\infty 
\frac{1}{n^{x+1}} \leq \int_N^\infty \frac{1}{u^{x+1}}\dd u = \frac{1}{xN^x}. \]
Also, we have 
\[ |S_N(-z+w)| \leq \sum_{n=1}^N \frac{1}{n^{-x+1}} \leq N^{x-1} + \int_1^N u^{x-1}\dd u 
\leq N^{x-1} + \frac{N^x}{x} = N^x \left( \frac1N + \frac1x \right). \]
Putting the above estimates together, we get 
\begin{align*}
    \left| \int_A \left( R_N(z+w) N^z - S_N(-z+w) N^{-z} \right) \left( \frac1z + \frac z{R^2} \right) \dd z \right| 
    &\leq \int_A \left( \frac{1}{xN^x} N^x + N^x \left( \frac1N + \frac1x \right) N^{-x} \right) 
    \frac{2x}{R^2}\dd z \\
    &= \int_A \left( \frac2x + \frac1N \right) \frac{2x}{R^2}\dd z \\
    &= \int_A \left( \frac4{R^2} + \frac{2x}{NR^2} \right)\dd z \\
    &\leq \pi R \left( \frac4{R^2} + \frac{2}{NR} \right) \quad \text{(since $x \leq R$)} \\
    &\leq \frac{4\pi}R + \frac{2\pi}N. 
\end{align*}
We now estimate the integral along $B$. We can divide $B$ into two parts; one part with $\Re(z) = 
-\delta$, and the other with $-\delta < \Re(z) \leq 0$. For $z \in B$ with $\Re(z) = -\delta$, 
we use the fact that $|z| \leq R$ to find that 
\[ \left| \frac1z + \frac z{R^2} \right| = \left| \frac1z \right| \left| \frac{\bar z}{z} + 
\frac{z\bar z}{R^2} \right| \leq \frac1\delta \left( 1 + \frac{|z|^2}{R^2} \right) \leq \frac2\delta. \]
Since $|F(z+w)| \leq M$ for $z \in B$, we have 
\begin{align*}
    \left| \int_B F(z+w) N^z \left( \frac1z + \frac z{R^2} \right)\dd z \right| 
    &\leq \int_{-R}^R MN^{-\delta} \frac2\delta\dd z + 2 \left| \int_{-\delta}^0 MN^x \frac{2x}{R^2}\dd x \right| \\
    &= \frac{4MR}{\delta N^\delta} + \frac{4M}{R^2} \left| \int_{-\delta}^0 xN^x \dd x \right| \\
    &\leq \frac{4MR}{\delta N^\delta} + \frac{4M\delta}{R^2} \left( \frac1{(\log N)^2} - \frac{\delta+1}{N^\delta \log N} \right) \\
    &\leq \frac{4MR}{\delta N^\delta} + \frac{4M\delta}{R^2(\log N)^2}. 
\end{align*}
Combining this estimate with $(3.2)$ and $(3.3)$ yields 
\[ |2\pi i(F(w) - S_N(w))| \leq \frac{4\pi}R + \frac{2\pi}N + \frac{4MR}{\delta N^\delta} 
+ \frac{4M\delta}{R^2(\log N)^2}. \] 
That is, we have 
\[ |F(w) - S_N(w)| \leq \frac2R + \frac1N + \frac{MR}{\delta N^\delta} + \frac{M\delta}{R^2(\log N)^2}. \]
Given $\eps > 0$, choose $R = 3/\eps$. Then for sufficiently large $N$, we have 
\[ |F(w) - S_N(w)| < \eps. \]
This implies that $S_N(w) \to F(w)$ as $N \to \infty$, which completes the proof. 
\end{pf}

\subsection{Revisiting the M\"obius Function}
Recall that we defined the M\"obius function $\mu : \N \to \{-1, 0, 1\}$ by 
\[ \mu(n) = \begin{cases} 1 & \text{if } n = 1, \\ 0 & \text{if $n$ is not squarefree,} \\ (-1)^r & \text{if $n$ is the product of $r$ distinct primes.} \end{cases} \]
We will show on Homework 2 that for $\Re(s) > 1$, we have 
\[ \frac1{\zeta(s)} = \prod_p \left( 1 - \frac1{p^s} \right) = \sum_{n=1}^\infty \frac{\mu(n)}{n^s}. \]

\begin{thm}
We have 
\[ \sum_{n=1}^\infty \frac{\mu(n)}n = 0. \]
\end{thm}
\begin{pf}
For all $\Re(s) > 1$, equation (3.4) holds. Moreover, we have shown that $(s-1)\zeta(s)$ is analytic 
and non-zero in $\Re(s) \geq 1$, so $1/\zeta(s)$ is analytic on $\Re(s) \geq 1$. 
Now, $\zeta(s)$ can be analytically continued up to $\Re(s) > 0$ and it is nonzero for $\Re(s) \geq 1$,
so we see that the series 
\[ \sum_{n=1}^\infty \frac{\mu(n)}{n^s} \] 
converges to $1/\zeta(s)$ for $\Re(s) \geq 1$. In particular, it converges at $s = 1$. But 
$\zeta(s)$ has a simple pole at $s = 1$, so $1/\zeta(1) = 0$. 
\end{pf}

\begin{thm}
We have 
\[ \sum_{n\leq x} \mu(n) = o(x). \]
\end{thm}
\begin{pf}
Applying Abel's summation formula with $a_n = \mu(n) / n $ and $f(x) = x$, we obtain 
\[ \sum_{n\leq x} \mu(n) = A(x)x - \int_1^x A(u)\dd u, \]
where we hae 
\[ A(t) = \sum_{n\leq t} \frac{\mu(n)}n. \]
By Theorem 3.5, we know that $A(t) = o(1)$. It follows that $A(x) x = o(x)$ and 
\[ \int_1^x A(u)\dd u = o(x), \]
so the result holds. 
\end{pf}

\subsection{Divisor Function}

\begin{defn}
For a positive integer $n \in \N$, let $d(n)$ be the number of positive integers that 
divide $n$. 
\end{defn}

For example, we have $d(1) = 1$, $d(4) = 3$, and $d(p) = 2$ for all primes $p$. 

\begin{thm}
We have 
\[ \sum_{m=1}^n d(m) = \sum_{m=1}^n \floor*{\frac nm} = n\log n + (2\gamma - 1)n + O(n^{1/2}). \]
where $\gamma$ denotes Euler's constant. 
\end{thm}
\begin{pf}
Let $D_n$ be the region in the upper right-hand quadrant not containing the $x$ or $y$ axes, 
which is under and includes the hyperbola $xy = n$. That is, 
\[ D_n := \{(x, y) \in \R^2 : x > 0,\, y > 0,\, xy \leq n\}. \]
Define a {\bf lattice point} to be a point in the plane with integer coordinates; that is, a point 
$(x, y) \in \R^2$ with $x, y \in \Z$. Notice that every lattice point in $D_n$ is contained in 
some hyperbola $xy = s$ where $s$ is an integer with $1 \leq s \leq n$. 

Therefore, $\sum_{s=1}^n d(s)$ is the number of lattice points in $D_n$; that is, 
\[ \sum_{s=1}^n d(s) = \#\{(x, y) \in \R^2 : x, y \in \N,\, xy \leq n\}. \]
We now count the number of lattice points in a different way. Given $x \in \N$ with $1 \leq x \leq n$, 
there are exactly $\floor{\frac xn}$ many integers $y$ such that $xy \leq n$. Thus, we see that 
\[ \#\{(x, y) \in \R^2 : x, y \in \N,\, xy \leq n\} = \sum_{x=1}^n \floor*{\frac nx}. \]
Observe that the number of lattice points above the line $x = y$ inside $D_n$ is equal to the number of 
lattice points below it. Divide the lattice points in $D_n$ into three disjoint regions given by 
\begin{align*}
    D_{n,1} &= \{(x, y) \in \N^2 : xy \leq n,\, x < y\}, \\ 
    D_{n,2} &= \{(x, y) \in \N^2 : xy \leq n,\, x > y\}, \\
    D_{n,3} &= \{(x, y) \in \N^2 : xy \leq n,\, x = y\}.
\end{align*}
Our observation above shows that $|D_{n,1}| = |D_{n,2}|$. Suppose that $(x, y) \in D_{n,1}$. Then 
$x^2 < xy \leq n$, which implies that $x < \sqrt n$. Moreover, for a fixed integer $x$, the number of 
integers $y$ satisfying $xy \leq n$ and $y > x$ is $\floor{\frac nx} - \floor x$. We also see that 
$|D_{n,3}| = \floor{\sqrt n}$, so we obtain 
\begin{align*}
    \sum_{x=1}^n \floor*{\frac nx} &= |D_{n,1}| + |D_{n,2}| + |D_{n,3}| \\
    &= 2 \sum_{x=1}^{\floor{\sqrt n}} \left( \floor*{\frac nx} - \floor x \right) + \floor{\sqrt n} \\
    &= 2 \sum_{x=1}^{\floor{\sqrt n}} \left( \frac nx - x + O(1) \right) + \floor{\sqrt n}. 
\end{align*}
By Theorem 2.10, we see that 
\[ \sum_{x=1}^n \floor*{\frac nx} = 2n \left( \log\floor{\sqrt n} + \gamma + O\left( \frac1{\sqrt n} \right) \right) - \left( n + O(\sqrt n) \right) + O(\sqrt n). \]
Note that if we use the fact that $\log\floor{\sqrt n} = \log \sqrt n + O(1)$, then the resulting 
error term $O(n)$ will be too large. Therefore, we need a finer estimate. Indeed, since 
$\floor{\sqrt n} = \sqrt n - \{\sqrt n\}$ where $\{t\}$ denotes the fractional part of $t$ for 
$t \in \R$, we have 
\begin{align*} \log \floor{\sqrt n} = \log \left( \sqrt n - \{\sqrt n\} \right) 
&= \log \left( \sqrt n \left( 1 - \frac{\{\sqrt n\}}{\sqrt n} \right) \right) \\
&= \log \sqrt n + \log \left( 1 - \frac{\{\sqrt n\}}{\sqrt n} \right) \\
&= \log \sqrt n + O\left( \frac1{\sqrt n} \right). \end{align*}
Combining this with the previous equality gives 
\[ \sum_{x=1}^n \floor*{\frac nx} = n\log n + (2\gamma - 1)n + O(\sqrt n). \qedhere \]
\end{pf}

\subsection{The Prime Number Theorem}
We now have everything we need to prove the Prime Number Theorem. 

\begin{thm}[Prime Number Theorem] 
    We have 
    \[ \pi(x) \sim \frac{x}{\log x}. \] 
\end{thm}
\begin{pf}
    In Theorem 2.7, we showed that 
    \[ \pi(x) \sim \frac{\psi(x)}{\log x}. \] 
    Therefore, it suffices to show that $\psi(x) \sim x$. Define the function 
    \[ F(x) = \sum_{n\leq x} \left( \psi\left( \frac xn \right) - \floor*{\frac xn}
    + 2\gamma \right), \] 
    where $\gamma$ denotes Euler's constant. By the M\"obius inversion formula 
    (Proposition 2.5), we have 
    \[ \psi(x) - \floor{x} + 2\gamma = \sum_{n\leq x} \mu(n) F\left( \frac xn \right). \] 
    In particular, we get 
    \[ \psi(x) = x + O(1) + \sum_{n \leq x} \mu(n) F\left( \frac xn \right). \] 
    Now, it is enough to show that $\sum_{n\leq x} \mu(n) F(x/n) = o(x)$. First, 
    we will estimate $F(x)$. Observe that 
    \[ F(x) = \sum_{n\leq x} \psi\left( \frac xn \right) - \sum_{n \leq x} 
    \floor*{\frac xn} + 2\gamma\floor{x}. \tag{3.4} \]
    Looking at the first sum in $(3.4)$, we have 
    \begin{align*}
        \sum_{n\leq x} \psi\left( \frac xn \right) 
        &= \sum_{n \leq x} \sum_{m \leq \frac xn} \Lambda(m) \\
        &= \sum_{n \leq x} \Lambda(n) \sum_{m\leq \frac xn} 1 \\
        &= \sum_{n \leq x} \Lambda(n) \floor*{\frac xn} \\
        &= \sum_{p^k \leq x} \log p \floor*{\frac x{p^k}} \\ 
        &= \sum_{p \leq x} \left( \floor*{\frac xp} + \floor*{\frac x{p^2}} 
        + \cdots + \floor*{\frac x{p^k}} \right) \quad \text{(where $p^k\;\|\;\floor{x}$)} \\
        &= \log(\floor{x}!) = \sum_{n \leq x} \log n. 
    \end{align*}
    In the proof of Theorem 2.11, we showed that 
    \[ \sum_{n\leq x} \log n = x\log x - x + O(\log x). \] 
    Hence, we obtain 
    \[ \sum_{n\leq x} \psi\left( \frac xn \right) = x\log x - x + O(\log x). \tag{3.5} \] 
    Moreover, by Theorem 3.9, we have 
    \[ \sum_{n=1}^{\floor{x}} \floor*{\frac{\floor x}n} = \floor{x}\log\floor{x} 
    + (2\gamma - 1)\floor{x} + O(x^{1/2}). \] 
    For all $y \in \R$, notice that $\floor{y} \leq y \leq \floor{y}+1$. In particular, 
    we obtain the inequalities 
    \[ \sum_{n=1}^{\floor{x}} \floor*{\frac{\floor{x}}n} \leq 
    \sum_{n=1}^{\floor{x}} \floor*{\frac{x}n} \leq 
    \sum_{n=1}^{\floor{x}+1} \floor*{\frac{\floor{x}+1}n}, \] 
    and it follows that 
    \[ \sum_{n=1}^{\floor{x}} \floor*{\frac{x}{n}} = x\log x + (2\gamma - 1) 
    + O(x^{1/2}). \tag{3.6} \] 
    Combining equations $(3.4)$, $(3.5)$, and $(3.6)$ gives 
    \[ F(x) = (x\log x - x + O(\log x)) - (x\log x + (2\gamma-1)x + O(x^{1/2})) 
    + (2\gamma x + O(1)) = O(x^{1/2}). \] 
    Therefore, there exists a positive constant $c > 0$ such that 
    \[ |F(x)| \leq cx^{1/2} \] 
    for all $x \geq 1$. If $t > 1$ is an integer, then 
    \begin{align*}
        \left| \sum_{n\leq \frac xt} \mu(n) F\left( \frac xn \right) \right| 
        &\leq \sum_{n\leq \frac xt} \left| F\left( \frac xn \right) \right| \\ 
        &\leq \sum_{n\leq \frac xt} c\left( \frac xn \right)^{\!1/2} \\
        &\leq cx^{1/2} \left( 1 + \int_1^{x/t} \frac{1}{u^{1/2}}\dd u \right) \\
        &= cx^{1/2} \left(1 + 2 \left( \frac xt \right)^{\!1/2} - 2 \right) \\ 
        &\leq 2 \cdot \frac{cx}{t^{1/2}}. \tag{3.7}
    \end{align*}
    Observe that $F$ is a step function. That is, if $a$ is an integer and 
    $a \leq x < a + 1$, then $F(x) = F(a)$. Therefore, we have 
    \[ \sum_{\frac xt < n \leq x} \mu(n) F\left(\frac xn \right) 
    = F(1) \sum_{\frac x2 < n \leq x} \mu(n) + F(2) \sum_{\frac x3 < n \leq 
    \frac x2} \mu(n) + \cdots + F(t-1) \sum_{\frac xt < n \leq \frac{x}{t-1}} \mu(n). \] 
    We see that 
    \begin{align*}
        \left| \sum_{\frac xt < n \leq x} \mu(n) F\left(\frac xn \right) \right| 
        &\leq |F(1)| \left| \sum_{\frac x2 < n \leq x} \mu(n) \right| + 
        |F(2)| \left| \sum_{\frac x3 < n \leq \frac x2} \mu(n) \right| + \cdots + 
        |F(t-1)| \left| \sum_{\frac xt < n \leq \frac{x}{t-1}} \mu(n) \right| \\ 
        &\leq (|F(1)| + \cdots + |F(t-1)|) \max_{2\leq i \leq t} 
        \left| \sum_{\frac xi < n \leq \frac{x}{i-1}} \mu(n) \right| \\ 
        &\leq \left( \sum_{i=1}^t ci^{1/2} \right) \max_{2\leq i \leq t} 
        \left| \sum_{\frac xi < n \leq \frac{x}{i-1}} \mu(n) \right|. 
    \end{align*}
    Notice that 
    \[ \sum_{\frac xi < n \leq \frac{x}{i-1}} \mu(n) = 
    \sum_{n \leq \frac{x}{i-1}} \mu(n) - \sum_{\frac{x}{i} < n} \mu(n) = o(x), \] 
    so we obtain 
    \[ \left| \sum_{\frac xt < n \leq x} \mu(n) F\left( \frac xn \right) \right| 
    = o(t^{3/2}x). \] 
    By Theorem 3.7, we have $\sum_{n\leq x} \mu(n) = o(x)$. Hence, 
    for any $\eps > 0$, we can find sufficiently large $x$ such that 
    \[ -\eps x \leq \sum_{n\leq x} \mu(n) \leq \eps x. \] 
    In particular, when $x$ is sufficiently large, we get 
    \[ -\frac{\eps x}{i-1} - \frac{\eps x}{i} 
    \leq \sum_{\frac{x}{i} < n \leq \frac{x}{i-1}} \mu(n) 
    \leq \frac{\eps x}{i-1} + \frac{\eps x}{i}. \] 
    For any given $\eps > 0$, choose $t = t(\eps)$ such that 
    \[ \frac{2c}{t^{1/2}} < \frac{\eps}2. \] 
    By equation $(3.7)$, we have 
    \[ \left| \sum_{n\leq \frac{x}{t}} \mu(n) F\left( \frac xn \right) \right| 
    \leq 2 \cdot \frac{cx}{t^{1/2}} < \frac{\eps}2 x. \tag{3.8} \] 
    For fixed $\eps > 0$ and $t$ as above, we can choose $x$ sufficiently large 
    so that $o(xt^{3/2}) \leq \eps x/2$. Indeed, we have 
    $2c/t^{1/2} < \eps/2$ if and only if $t > (4c)^2/\eps^2$. In particular, 
    we have $t = A^2\eps^{-2}$ for some $A > 4c$, and we can pick $x$ large 
    enough so that 
    \[ o(x) \leq \frac{\eps^4}{2A^3}x. \] 
    Then we get 
    \[ o(xt^{3/2}) \leq \frac{\eps^4}{2A^3} x \cdot A^3\eps^{-3} = \frac{\eps}2 x. \] 
    It follows that 
    \[ \left| \sum_{\frac xt < n \leq x} \mu(n) F\left( \frac xn \right) \right| < 
    \frac{\eps}2. \tag{3.9} \] 
    Combining inequalities $(3.8)$ and $(3.9)$ yields 
    \[ \left| \sum_{n\leq x} \mu(n) F\left( \frac xn \right) \right| = o(x), \] 
    which completes the proof. 
\end{pf}

\begin{remark}~
    \begin{enumerate}[(1)]
        \item In 1896, Hadamard and de la Vall\'ee Poussin proved the Prime 
        Number Theorem independently. Consider the logarithmic integral 
        \[ \Li(x) = \int_2^x \frac{1}{\log t}\dd t \sim \frac{x}{\log x} 
        \sum_{k=0}^\infty \frac{k!}{(\log x)^k}. \] 
        In 1899, de la Vall\'ee Poussin proved that as $x \to \infty$, there 
        exists some $a > 0$ such that 
        \[ \pi(x) = \Li(x) + O(xe^{-a\sqrt{\log x}}). \]
        \item The main ingredient of our proof of the Prime Number Theorem is the 
        fact that $\sum_{n\leq x} \mu(n) = o(x)$, which is a consequence of the 
        analytic continuation and non-vanishing of $\zeta(s)$ at $\Re(s) = 1$. 
        The {\bf Riemann hypothesis}, proposed by Riemann in 1859, states that 
        the non-trivial zeros of $\zeta(s)$ all have real part $1/2$. 
        (The trivial zeros of $\zeta(s)$ are of the form $2n$ for $n \in \Z$ and 
        $n < 0$; these can be obtained by functional equations.) In 1901, Helge von 
        Koch proved that the Riemann hypothesis is true if and only if 
        \[ \pi(x) = \Li(x) + O(\sqrt x \log x). \] 
    \end{enumerate}
\end{remark}


\end{document}