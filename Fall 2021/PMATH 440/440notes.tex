\documentclass[10pt]{article}
\usepackage[T1]{fontenc}
\usepackage{amsmath,amssymb,amsthm}
\usepackage{mathtools}
\usepackage[shortlabels]{enumitem}
\usepackage[english]{babel}
\usepackage[utf8]{inputenc}
\usepackage{fancyhdr}
\usepackage{bold-extra}
\usepackage{color}   
\usepackage{tocloft}
\usepackage{graphicx}
\usepackage{lipsum}
\usepackage{wrapfig}
\usepackage{cutwin}
\usepackage{hyperref}
\usepackage{lastpage}
\usepackage{multicol}
\usepackage{tikz}
\usepackage{xcolor}
\usepackage{microtype}
\usepackage[framemethod=TikZ]{mdframed}

% some useful math commands
\newcommand{\eps}{\varepsilon}
\newcommand{\R}{\mathbb{R}}
\newcommand{\C}{\mathbb{C}}
\newcommand{\N}{\mathbb{N}}
\newcommand{\Z}{\mathbb{Z}}
\newcommand{\Q}{\mathbb{Q}}
\newcommand{\K}{\mathbb{K}}
\newcommand{\F}{\mathbb{F}}

\numberwithin{equation}{section}

\newcommand{\dd}{\,\mathrm{d}}
\newcommand{\ddz}{\frac{\rm d}{{\rm d}z}}
\newcommand{\pv}{\text{p.v.}}

\renewcommand{\Re}{{\rm Re}}

\DeclareMathOperator{\GL}{GL}
\DeclareMathOperator{\id}{id}
\DeclareMathOperator{\Arg}{Arg}
\DeclareMathOperator{\Log}{Log}
\DeclareMathOperator{\PV}{PV}
\DeclareMathOperator{\sech}{sech}
\DeclareMathOperator{\csch}{csch}
\DeclareMathOperator{\Res}{Res}
\DeclareMathOperator{\Li}{Li}
\DeclareMathOperator{\QR}{QR}
\DeclareMathOperator{\NR}{NR}
\DeclareMathOperator{\lcm}{lcm}

\DeclarePairedDelimiter\ceil{\lceil}{\rceil}
\DeclarePairedDelimiter\floor{\lfloor}{\rfloor}

\newcommand{\suchthat}{\;\ifnum\currentgrouptype=16 \;\middle|\;\else\mid\fi\;}

% title formatting
\newcommand{\newtitle}[4]{
  \begin{center}
	\huge{\textbf{\textsc{#1 Course Notes}}}
    
	\large{\sc #2}
    
	{\sc #3 \textbullet\, #4 \textbullet\, University of Waterloo}
	\normalsize\vspace{1cm}\hrule
  \end{center}
}

\newcounter{theo}[section]\setcounter{theo}{0}
\renewcommand{\thetheo}{\arabic{section}.\arabic{theo}}
\newenvironment{theo}[2][]{%
\refstepcounter{theo}%
\ifstrempty{#1}%
{\mdfsetup{%
frametitle={%
\tikz[baseline=(current bounding box.east),outer sep=0pt]
\node[anchor=east,rectangle,fill=blue!20]
{\strut {\sc Theorem~\thetheo}};}}
}%
{\mdfsetup{%
frametitle={%
\tikz[baseline=(current bounding box.east),outer sep=0pt]
\node[anchor=east,rectangle,fill=blue!20]
{\strut {\sc Theorem~\thetheo:~#1}};}}%
}%
\mdfsetup{innertopmargin=10pt,linecolor=blue!20,%
linewidth=2pt,topline=true,%
frametitleaboveskip=\dimexpr-\ht\strutbox\relax
}
\begin{mdframed}[nobreak=true]\relax%
\label{#2}}{\end{mdframed}}

%%%%%%%%%%%%%%%%%%%%%%%%%%%%%%
%Definition
\newenvironment{defn}[2][]{%
\refstepcounter{theo}%
\ifstrempty{#1}%
{\mdfsetup{%
frametitle={%
\tikz[baseline=(current bounding box.east),outer sep=0pt]
\node[anchor=east,rectangle,fill=yellow!20]
{\strut {\sc Definition~\thetheo}};}}
}%
{\mdfsetup{%
frametitle={%
\tikz[baseline=(current bounding box.east),outer sep=0pt]
\node[anchor=east,rectangle,fill=yellow!20]
{\strut {\sc Definition~\thetheo:~#1}};}}%
}%
\mdfsetup{innertopmargin=10pt,linecolor=yellow!20,%
linewidth=2pt,topline=true,%
frametitleaboveskip=\dimexpr-\ht\strutbox\relax
}
\begin{mdframed}[nobreak=true]\relax%
\label{#2}}{\end{mdframed}}

%%%%%%%%%%%%%%%%%%%%%%%%%%%%%%
%Example
\newenvironment{exmp}[2][]{%
\refstepcounter{theo}%
\ifstrempty{#1}%
{\mdfsetup{%
frametitle={%
\tikz[baseline=(current bounding box.east),outer sep=0pt]
\node[anchor=east,rectangle,fill=cyan!20]
{\strut {\sc Example~\thetheo}};}}
}%
{\mdfsetup{%
frametitle={%
\tikz[baseline=(current bounding box.east),outer sep=0pt]
\node[anchor=east,rectangle,fill=cyan!20]
{\strut {\sc Example~\thetheo:~#1}};}}%
}%
\mdfsetup{innertopmargin=10pt,linecolor=cyan!20,%
linewidth=2pt,topline=true,%
frametitleaboveskip=\dimexpr-\ht\strutbox\relax
}
\begin{mdframed}[nobreak=true]\relax%
\label{#2}}{\end{mdframed}}

%%%%%%%%%%%%%%%%%%%%%%%%%%%%%%
%Corollary
\newenvironment{cor}[2][]{%
\refstepcounter{theo}%
\ifstrempty{#1}%
{\mdfsetup{%
frametitle={%
\tikz[baseline=(current bounding box.east),outer sep=0pt]
\node[anchor=east,rectangle,fill=lime!20]
{\strut {\sc Corollary~\thetheo}};}}
}%
{\mdfsetup{%
frametitle={%
\tikz[baseline=(current bounding box.east),outer sep=0pt]
\node[anchor=east,rectangle,fill=lime!20]
{\strut {\sc Corollary~\thetheo:~#1}};}}%
}%
\mdfsetup{innertopmargin=10pt,linecolor=lime!20,%
linewidth=2pt,topline=true,%
frametitleaboveskip=\dimexpr-\ht\strutbox\relax
}
\begin{mdframed}[nobreak=true]\relax%
\label{#2}}{\end{mdframed}}

%%%%%%%%%%%%%%%%%%%%%%%%%%%%%%
%Remark
\newenvironment{remark}[2][]{%
\refstepcounter{theo}%
\ifstrempty{#1}%
{\mdfsetup{%
frametitle={%
\tikz[baseline=(current bounding box.east),outer sep=0pt]
\node[anchor=east,rectangle,fill=orange!20]
{\strut {\sc Remark~\thetheo}};}}
}%
{\mdfsetup{%
frametitle={%
\tikz[baseline=(current bounding box.east),outer sep=0pt]
\node[anchor=east,rectangle,fill=orange!20]
{\strut {\sc Remark~\thetheo:~#1}};}}%
}%
\mdfsetup{innertopmargin=10pt,linecolor=orange!20,%
linewidth=2pt,topline=true,%
frametitleaboveskip=\dimexpr-\ht\strutbox\relax
}
\begin{mdframed}[nobreak=true]\relax%
\label{#2}}{\end{mdframed}}

%%%%%%%%%%%%%%%%%%%%%%%%%%%%%%
%Exercise
\newenvironment{exercise}[2][]{%
\refstepcounter{theo}%
\ifstrempty{#1}%
{\mdfsetup{%
frametitle={%
\tikz[baseline=(current bounding box.east),outer sep=0pt]
\node[anchor=east,rectangle,fill=pink!20]
{\strut {\sc Exercise~\thetheo}};}}
}%
{\mdfsetup{%
frametitle={%
\tikz[baseline=(current bounding box.east),outer sep=0pt]
\node[anchor=east,rectangle,fill=pink!20]
{\strut {\sc Exercise~\thetheo:~#1}};}}%
}%
\mdfsetup{innertopmargin=10pt,linecolor=pink!20,%
linewidth=2pt,topline=true,%
frametitleaboveskip=\dimexpr-\ht\strutbox\relax
}
\begin{mdframed}[nobreak=true]\relax%
\label{#2}}{\end{mdframed}}

%%%%%%%%%%%%%%%%%%%%%%%%%%%%%%
%Lemma
\newenvironment{lemma}[2][]{%
\refstepcounter{theo}%
\ifstrempty{#1}%
{\mdfsetup{%
frametitle={%
\tikz[baseline=(current bounding box.east),outer sep=0pt]
\node[anchor=east,rectangle,fill=green!20]
{\strut {\sc Lemma~\thetheo}};}}
}%
{\mdfsetup{%
frametitle={%
\tikz[baseline=(current bounding box.east),outer sep=0pt]
\node[anchor=east,rectangle,fill=green!20]
{\strut {\sc Lemma~\thetheo:~#1}};}}%
}%
\mdfsetup{innertopmargin=10pt,linecolor=green!20,%
linewidth=2pt,topline=true,%
frametitleaboveskip=\dimexpr-\ht\strutbox\relax
}
\begin{mdframed}[nobreak=true]\relax%
\label{#2}}{\end{mdframed}}

%%%%%%%%%%%%%%%%%%%%%%%%%%%%%%
%Proposition
\newenvironment{prop}[2][]{%
\refstepcounter{theo}%
\ifstrempty{#1}%
{\mdfsetup{%
frametitle={%
\tikz[baseline=(current bounding box.east),outer sep=0pt]
\node[anchor=east,rectangle,fill=purple!20]
{\strut {\sc Proposition~\thetheo}};}}
}%
{\mdfsetup{%
frametitle={%
\tikz[baseline=(current bounding box.east),outer sep=0pt]
\node[anchor=east,rectangle,fill=purple!20]
{\strut {\sc Proposition~\thetheo:~#1}};}}%
}%
\mdfsetup{innertopmargin=10pt,linecolor=purple!20,%
linewidth=2pt,topline=true,%
frametitleaboveskip=\dimexpr-\ht\strutbox\relax
}
\begin{mdframed}[nobreak=true]\relax%
\label{#2}}{\end{mdframed}}

% new proof environment
\makeatletter
\newenvironment{pf}[1][\proofname]{\par
  \pushQED{\qed}%
  \normalfont \topsep0\p@\relax
  \trivlist
  \item[\hskip\labelsep\scshape
  #1\@addpunct{.}]\ignorespaces
}{%
  \popQED\endtrivlist\@endpefalse
}
\makeatother

% 1-inch margins
\topmargin 0pt
\advance \topmargin by -\headheight
\advance \topmargin by -\headsep
\textheight 8.9in
\oddsidemargin 0pt
\evensidemargin \oddsidemargin
\marginparwidth 0.5in
\textwidth 6.5in

\parindent 0in
\parskip 1.5ex

\setlist[itemize]{topsep=0pt}
\setlist[enumerate]{topsep=0pt}

\newcommand{\pushright}[1]{\ifmeasuring@#1\else\omit\hfill$\displaystyle#1$\fi\ignorespaces}

% hyperlinks
\hypersetup{
  colorlinks=true, 
  linktoc=all,     % table of contents is clickable  
  allcolors=red    % all hyperlink colours
}

% table of contents
\addto\captionsenglish{
  \renewcommand{\contentsname}%
    {Table of Contents}%
}
\renewcommand{\cftsecfont}{\normalfont}
\renewcommand{\cftsecpagefont}{\normalfont}
\cftsetindents{section}{0em}{2em}

\fancypagestyle{plain}{%
\fancyhf{} % clear all header and footer fields
\lhead{PMATH 440: Fall 2021}
\fancyhead[R]{Table of Contents}
%\headrule
\fancyfoot[R]{{\small Page \thepage\ of \pageref*{LastPage}}}
}

% headers and footers
\pagestyle{fancy}
\renewcommand{\sectionmark}[1]{\markboth{#1}{#1}}
\lhead{PMATH 440: Fall 2021}
\cfoot{}
\setlength\headheight{14pt}

%\setcounter{section}{-1}

\begin{document}

\pagestyle{fancy}
\newtitle{PMATH 440}{Analytic Number Theory}{Wentang Kuo}{Fall 2021}
\rhead{Table of Contents}
\rfoot{{\small Page \thepage\ of \pageref*{LastPage}}}

\tableofcontents
\vspace{1cm}\hrule
\fancyhead[R]{\nouppercase\rightmark}
\newpage 
\fancyhead[R]{Section \thesection: \nouppercase\leftmark}

\section{Introduction to Prime Numbers and Their Counting Function}\label{sec:1}

\subsection{Primes}\label{subsec:1.1}

\vspace{2ex}
\begin{defn}{def:1.1}
A {\bf prime number} is a positive integer greater than $1$ such that its only factors are $1$ and 
itself. We denote by ${\cal P}$ the set of all prime numbers. For a positive real number $x$, 
we define the {\bf prime counting function} by 
\[ \pi(x) = \#\{p \leq x : p \in {\cal P}\}, \]
where $\#S$ denotes the cardinality of the set $S$. 
\end{defn}

We would like to know how the primes are distributed among the integers. Let $p_n$ denote the 
$n$-th prime. Is there a formula to obtain $p_n$? Is there a polynomial $f(x) \in \Z[x]$ such that 
$f(n) = p_n$ for all $n \in \N$? The answer to the latter question is no, due to the following result. 

\begin{prop}{prop:1.2}
There is no non-constant polynomial $f(x) \in \Z[x]$ such that $f(n)$ is prime for all $n \in \N$. 
\end{prop}
\begin{pf}
Suppose such a polynomial $f(x) \in \Z[x]$ existed, and write 
\[ f(x) = a_n x^n + \cdots + a_1 x + a_0. \]
Let $q$ be a prime with $f(n) = q$ for some $n \in \N$. Then $q \mid f(n+kq)$ for each $k \in \N$. 
In particular, notice that if $f(m)$ is prime for every positive integer $m$, then $f(x)$ must be 
constant with $f(x) = q$ for some prime $q$. 
\end{pf}

\begin{remark}{remark:1.3}
\begin{enumerate}[(1)]
    \item There are examples of polynomials whose initial values are surprisingly often prime. 
    For example, the polynomial $n^2 + n + 41$ is prime for all $0 \leq n \leq 39$, and the 
    polynomial $(n-40)^2 + (n-40) + 41$ is prime for all $0 \leq n \leq 79$. 
    \item In the 1970s, Matijasevic proved Hilbert's tenth problem, and in the process, he was able to
    show that there is a polynomial $f \in \Z[a, b, \dots, z]$ such that the set of positive 
    values in $f(\N^{26})$ is exactly the set of primes. In 1977, he showed that only $10$ variables 
    are needed.
\end{enumerate}
\end{remark}

Let us instead ask a weaker question. Can we find a non-constant polynomial $f(x) \in \Z[x]$ such that 
$f(n)$ yields a prime for infinitely many $n \in \N$? Trivially, we see that $f(x) = x+k$ works 
for any $k \in \Z$. When the coefficient of $x$ is not equal to $1$, we have the following result, 
which we will prove at the end of this course.

\begin{theo}[Dirichlet]{thm:1.4}
Let $k$ and $\ell$ be coprime positive integers. Then $kn+\ell$ is prime for infinitely many positive
integers $n$. 
\end{theo}

\begin{remark}{remark:1.5}
\begin{enumerate}[(1)]
    \item At the moment, there is no known polynomial of degree greater than $1$ in one variable 
    known to take prime values infinitely often. The best result known to date is that $n^2+1$ is 
    a product of two primes for infinitely many $n$.
    \item If we instead consider polynomials of two variables, we can go further. It is known that 
    an odd prime $p$ is the sum of two squares if and only if $p \equiv 1 \pmod 4$. 
    In 1998, Friedlander and Iwaniec proved that there are infinitely many primes of the 
    form $n^2 + m^4$. In 2001, Heath-Brown showed that there are infinitely many primes of 
    the form $n^3 + 2m^3$. 
\end{enumerate}
\end{remark}

\begin{theo}[Euclid]{thm:1.6}
There are infinitely many prime numbers.
\end{theo}
\begin{pf}
Assume that there are only finitely many primes, say $p_1, \dots, p_n$, and consider 
\[ m = p_1 \cdots p_n + 1. \]
Then $m$ can be written as a product of primes by unique factorization, and $p_k \mid m$ for some $1 \leq
k \leq n$. 
Hence, we see that $p_k \mid m - p_1 \cdots p_n$ and $p_k \mid 1$, which is a contradiction. 
\end{pf}

We would like to estimate the prime counting function $\pi(x)$. 

\begin{prop}{prop:1.7}
For all $n \in \N$, we have $p_n \leq 2^{2^n}$. 
\end{prop}
\begin{pf}
We proceed by induction. For $n = 1$, we have $2 = p_1 \leq 2^{2^1} = 4$. Suppose the result 
holds for all $1 \leq k \leq n$. By Euclid's argument, we obtain $p_{n+1} \leq p_1 \cdots p_n + 1$. 
It follows from induction that 
\[ p_{n+1} \leq 2^{2^1} 2^{2^2} \cdots 2^{2^n} + 1 \leq 2^{2^{n+1}-2} + 1 \leq 2^{2^{n+1}}, \]
which completes the proof. 
\end{pf}

\begin{cor}{cor:1.8}
For all $x \geq 2$, we have $\pi(x) > \log\log x$. (In this course, $\log$ denotes the natural logarithm.)
\end{cor}
\begin{pf}
Let $x \geq 2$, and let $s$ be the integer satisfying 
\[ 2^{2^s} \leq x < 2^{2^{s+1}}. \]
By Proposition~\ref{prop:1.7}, we have $\pi(x) \geq s$. On the other hand, since $x < 2^{2^{s+1}}$, 
taking logarithms yields $\log_2(\log_2 x) < s+1$, and hence 
\[ \frac{\log(\frac{\log x}{\log 2})}{\log 2} < s+1. \]
It follows that 
\[ \pi(x) \geq s > \frac{\log(\frac{\log x}{\log 2})}{\log 2} - 1 \geq \log\log x. \qedhere \]
\end{pf}

There is an alternative way to prove Euclid's theorem, due to Euler, which is left as part of the 
homework. Using the same idea, we can derive a slightly better lower bound for $\pi(x)$. 

\begin{prop}{prop:1.9}
For all $x \geq 2$, we have 
\[ \pi(x) \geq \frac{\log\log x}{\log 2}. \]
\end{prop}
\begin{pf}
Suppose that $x \geq 2$. Then we have 
\[ 2^{\pi(x)} \geq \prod_{p\leq x} \left(1 - \frac1p\right)^{-1} = \prod_{p\leq x} \left(1 + \frac1p
+ \frac{1}{p^2} + \cdots \right) \geq \sum_{n\leq x} \frac1n \geq \int_1^{\lfloor x \rfloor + 1}
\frac1u\dd u \geq \log x, \]
where the product $\prod_{p\leq x}$ means that $p$ runs through all primes at most $x$, and
$\lfloor y \rfloor$ is the greatest integer less than or equal to $y$. We will will use this 
notation for the rest of the course. Taking logarithms yields the desired inequality.
\end{pf}

Fermat had conjectured that the numbers of the form $2^{2^n}+1$ are prime for $n \in \N$. 
He had checked it for the values $0 \leq n \leq 4$. These are known as the {\bf Fermat numbers} and 
are denoted by 
\[ F_n = 2^{2^n} + 1. \] 
In 1732, Euler showed that $641 \mid F_5$. It is also known that $F_6, \dots, F_{21}$ are composite. 
It is quite likely that only finitely many Fermat numbers are prime. 

\begin{theo}[Poly\'a]{thm:1.10}
If $n$ and $m$ are positive integers with $1 \leq n < m$, then $(F_n, F_m) = 1$. 
\end{theo}
\begin{pf}
Write $m = n+k$ with $k \geq 1$. First, we will show that $F_n \mid F_m - 2$. Observe that 
\[ F_m - 2 = (2^{2^{n+k}} + 1) - 2 = 2^{2^{n+k}} - 1. \]
The polynomial $x^{2^k} - 1$ is divisible by $x+1$ in $\Z[x]$. Now, letting $x = 2^{2^n}$, we get 
\[ \frac{F_m-2}{F_n} = \frac{x^{2^k} - 1}{x+1} = x^{2^k-1} - x^{2^k-2} + \cdots - 1 \in \Z. \] 
Hence, we have $F_n \mid F_m - 2$. Suppose now that $d \mid F_n$ and $d \mid F_m$. Then 
$d \mid 2$ and $2 \nmid F_n$, which implies that $d = \pm1$. The result follows. 
\end{pf}

This gives yet another proof of Euclid's theorem, as well as the bound $p_n \leq 2^{2^n} + 1$. 

\subsection{Elementary Approximations of $\pi(x)$}\label{subsec:1.2}
In 1896, Hadamard and de la Vall\'ee Poussin each proved the Prime Number Theorem independently. 

\begin{theo}[Prime Number Theorem]{thm:1.11}
We have 
\[ \lim_{x\to\infty} \frac{\pi(x)}{x/\log x} = 1. \]
\end{theo}

This was initially conjectured by Gauss. We will prove this theorem later in the course; 
for now, we will see how to approach this problem using elementary methods. 

\begin{theo}{thm:1.12}
For all $x \geq 2$, we have 
\[ \pi(x) \geq \frac{\log x}{2\log 2}. \]
Moreover, for all $n \geq 1$, we have $p_n \leq 4^n$.
\end{theo}
\begin{pf}
Let $x \geq 2$ be an integer. Let $p_1, \dots, p_j$ be the primes less than or equal to $x$. 
Note that we have $j = \pi(x)$ here. For every integer $n$ with $n \leq x$, we can write $n = n_1^2m$
where $n_1$ is a positive integer and $m$ is squarefree. Then $m$ is of the form 
\[ m = p_1^{\eps_1} \cdots p_j^{\eps_j}, \]
where $\eps_i \in \{0, 1\}$ for each $1 \leq i \leq j$. We see that there are at most $2^j$ possible 
values for $m$. Moreover, there are at most $\sqrt{x}$ possible values for $n_1$. Hence, we have 
$2^j \sqrt{x} \geq x$, which implies that $2^j \geq \sqrt{x}$. Denote this inequality by $(\star)$. 
Since $j = \pi(x)$, we see that 
\[ \pi(x) \log 2 \geq \frac{\log x}2, \]
so the first equality follows. For the second equality, take $x = p_n$ so that $\pi(p_n) = n$. 
By $(\star)$, we obtain $2^n \geq \sqrt{p_n}$ and hence $4^n \geq p_n$. 
\end{pf}

Let $n$ be a positive integer and let $p$ be a prime. Recall that the exact power of $p$ 
dividing $n!$ is 
\[ \sum_{n=1}^\infty \left\lfloor \frac{n}{p^k} \right\rfloor = \sum_{n=1}^{\left\lfloor \frac{\log n}{\log p} \right\rfloor} \left\lfloor \frac{n}{p^k} \right\rfloor. \]

\begin{theo}{thm:1.13}
For all $x \geq 2$, we have 
\[ \left( \frac{3\log 2}8 \right) \frac{x}{\log x} < \pi(x) < (6\log 2) \frac{x}{\log x}. \]
\end{theo}
\begin{pf}
This argument was given by Erd\H{o}s. First, we will prove the lower bound. Note that $\binom{2n}n$ is 
an integer, and 
\[ \binom{2n}n = \frac{(2n)!}{(n!)^2} \; \bigg\rvert \; \prod_{p\leq 2n} p^{r_p}, \]
where $r_p$ is an integer satisfying $p^{r_p} \leq 2n < p^{r_p+1}$. Indeed, note that 
the exact power of $p$ dividing $(2n)!$ is 
\[ \sum_{k=1}^{r_p} \left\lfloor \frac{2n}{p^k} \right\rfloor, \]
and the exact power of $p$ dividing $n!$ is 
\[ \sum_{k=1}^{r_p} \left\lfloor \frac{n}{p^k} \right\rfloor. \]
Thus, the exact power of $p$ dividing $\binom{2n}n$ is 
\[ \sum_{k=1}^{r_p} \left( \left\lfloor \frac{2n}{p^k} \right\rfloor -  \left\lfloor \frac{n}{p^k} \right\rfloor \right) \leq r_p, \] 
since $\lfloor 2a \rfloor - 2\lfloor a \rfloor \leq 1$ for all $a \in \R$. In particular, we have 
\[ \binom{2n}n \leq \prod_{p \leq 2n} p^{r_p} \leq (2n)^{\pi(2n)}. \]
Notice that 
\[ \binom{2n}n = \frac{2n \cdot (2n-1) \cdots (n+1)}{n \cdot (n-1) \cdots 1} = \frac{2n}n \cdots 
\frac{n+1}1 \geq 2^n. \]
Hence, we get $2^n \leq (2n)^{\pi(2n)}$. Now, we have 
\[ \pi(2n) \geq \left( \frac{\log 2}2 \right) \frac{2n}{\log(2n)}. \]
Recall that $\frac{x}{\log x}$ is increasing for $x > e$. If $x \geq 6$, choose $n \in \N$ such that 
$3x/4 \leq 2n \leq x$. We see that 
\[ \pi(x) \geq \pi(2n) \geq \left( \frac{\log 2}2 \right) \frac{2n}{\log(2n)} 
\geq \left( \frac{\log 2}2 \right) \frac{\frac34x}{\log(\frac34x)} > \frac{3\log 2}8 \frac{x}{\log x}. \]
One can manually check that the result holds for $2 \leq x \leq 6$, which finishes the proof of the lower 
bound. 

We now turn to the upper bound. Observe that 
\[ \prod_{n < p \leq 2n} p \; \bigg\rvert \; \binom{2n}n, \]
so by the binomial theorem, we have 
\[ \prod_{n < p \leq 2n} p \leq \binom{2n}n \leq (1+1)^{2n} = 2^{2n}. \]
On the other hand, notice that 
\[ \prod_{n < p \leq 2n} p \geq n^{\pi(2n) - \pi(n)}, \]
so it follows that 
\[ \pi(2n) \log n - \pi(n) \log(n/2) < (\log 2) 2n + (\log 2)\pi(n) < (3\log 2)n. \]
By taking $n = 2^k, 2^{k-1}, \dots, 4$, we obtain a telescoping collection of inequalities, given by 
\begin{align*}
    \pi(2^{k+1})\log2^k - \pi(2^k)\log2^{k-1} &< (3\log2)2^k, \\
    \pi(2^k)\log2^{k-1} - \pi(2^{k-1})\log2^{k-2} &< (3\log2)2^{k-1}, \\[-0.5em]
    &\;\;\vdots \\
    \pi(8)\log4-\pi(4)\log2 &< (3\log2)4.
\end{align*}
Putting these inequalities together, we have 
\[ \pi(2^{k+1})\log 2^k < (3\log 2)(2^k + 2^{k+1} + \cdots + 4) + \pi(4)\log2 < (3\log2)2^{k+1}, \]
and hence 
\[ \pi(2^{k+1}) < (3\log2) \left( \frac{2^{k+1}}{\log(2^k)} \right). \]
If $x > e$, choose $k$ such that $2^k \leq x \leq 2^{k+1}$. Then $\pi(x) \leq \pi(2^{k+1})$, and so 
\[ \pi(x) \leq (3\log2) \left( \frac{2^{k+1}}{\log(2^k)} \right) \leq (6\log2) \left( \frac{2^k}{\log(2^k)} \right) \leq (6\log 2) \left( \frac{x}{\log x} \right), \] 
where in the last equality, we use the fact that $\frac{x}{\log x}$ is increasing for $x > e$. 
The values $2 \leq x \leq e$ can be checked manually, proving the lower bound. 
\end{pf}

We should note that $\frac{3\log 2}8$ is in some sense arbitrary. In the proof, we could have picked
$n \in \N$ such that $1 - \eps \leq 2n \leq x$ instead of $3x/4 \leq 2n \leq x$ for $\eps$ 
arbitrarily small. However, this comes at the cost that the bound may potentially fail for small $x$,
and there is little purpose in a better lower bound for large $x$ as it is overshadowed by the 
Prime Number Theorem.

\subsection{Bertrand's Postulate}
In 1845, Bertrand showed that there is always a prime $p$ in the interval $[n, 2n]$ for $n \in \Z^+$
provided that $n < 6 \cdot 10^6$, and he had conjectured that this holds for all $n \in \Z^+$. Chebyshev 
proved that this was indeed the case in 1950. 
Note that this is not a trivial result; it doesn't occur for free just because $\pi(x) \sim x/\log x$. 

\begin{prop}{prop:1.14}
For all $n \in \Z^+$, we have 
\[ \prod_{p\leq n} p < 4^n. \]
\end{prop}
\begin{pf}
The result is clearly true for $n = 1$ and $n = 2$. Suppose that it holds for all $1 \leq n \leq k-1$. 
Note that we can restrict our attention to the case where $n$ is odd, because if $n$ is even 
and $n > 2$, then 
\[ \prod_{p \leq n} p = \prod_{p \leq n-1} p, \]
and the result will follow by induction. Write $n = 2m+1$ for some $m \in \Z^+$, and consider 
$\binom{2m+1}m$. In particular, we have 
\[ \prod_{m+1<p\leq2m+1} p \; \bigg\rvert \; \binom{2m+1}m. \]
Since $\binom{2m+1}m$ and $\binom{2m+1}{m+1}$ both appear in the binomial expansion of $(1+1)^{2m+1}$
with $\binom{2m+1}m = \binom{2m+1}{m+1}$, we obtain 
\[ \binom{2m+1}m \leq \frac12 (2^{2m+1}) = 4^m. \]
By our inductive hypothesis and the previous inequality, it follows that 
\[ \prod_{p\leq 2m+1} p = \left( \prod_{p\leq m+1} p \right) \left( \prod_{m+1 < p\leq 2m+1} p \right)
\leq 4^{m+1} 4^m = 4^{2m+1}. \qedhere \]
\end{pf}

\begin{lemma}{lemma:1.15}
If $n \geq 3$ and $p$ is a prime with $\frac23n < p \leq n$, then $p \nmid \binom{2n}n$. 
\end{lemma}
\begin{pf}
Since $n \geq 3$, we see that if $p$ is in the range $\frac23n < p \leq n$, then $p > 2$. Then 
$p$ and $2p$ are the only multiples of $p$ at most $2n$, and so 
\[ p^2 \; \| \; (2n)!, \]
where we write $p^k \; \| \; b$ to mean that $p^{k+1} \nmid b$ and $p^k \mid b$. Furthermore, since 
$\frac23n < p \leq n$, we have $p \;\|\; n!$ and hence $p^2 \;\|\; (n!)^2$. Using the identity 
\[ \binom{2n}n = \frac{(2n)!}{(n!)^2}, \] we see that $p \nmid \binom{2n}n$. 
\end{pf}

\begin{theo}[Chebyshev]{thm:1.16}
For every $n \in \Z^+$, there exists a prime $p$ satisfying $n < p \leq 2n$. 
\end{theo}
\begin{pf}
This argument was given by Erd\H{o}s. Note that the result holds for $1 \leq n \leq 3$. Assume that the 
result is false for some integer $n \geq 4$. By Lemma~\ref{lemma:1.15}, every prime dividing 
$\binom{2n}n$ is at most $\frac23n$. 

Let $p$ be a prime divisor of $\binom{2n}n$ where we have $p \leq \frac23n$. Suppose that 
$p^{\alpha_p} \;\|\; \binom{2n}n$ for some integer $\alpha_p$. Recall that in the proof of 
Theorem~\ref{thm:1.13}, we defined $r_p$ to be the integer satisfying $p^{r_p} \leq 2n < p^{r_p+1}$. 
Then we have $\alpha_p \leq r_p$, and hence $p^{\alpha_p} \leq p^{r_p} \leq 2n$. 

If $\alpha_p \geq 2$, then $p^2 \leq p^{\alpha_p} \leq 2n$ so that $p \leq \sqrt{2n}$. By 
Proposition~\ref{prop:1.14}, we have 
\[ \binom{2n}n \leq \left( \prod_{\substack{p\leq\frac23n\\ \alpha_p\leq 1}} p \right) 
\left( \prod_{\substack{p\leq\frac23n\\ \alpha_p\geq 2}} p \right) \leq 4^{2n/3} (2n)^{\pi(\sqrt{2n})}
\leq 4^{2n/3} (2n)^{\sqrt{2n}}. \]
Note that $\binom{2n}n$ is the largest of the $2n+1$ terms in the binomial expansion of 
\[ (1+1)^{2n} = \binom{2n}0 + \binom{2n}1 + \cdots + \binom{2n}{2n}, \]
so we get 
\[ \binom{2n}n \geq \frac{2^{2n}}{2n+1}. \]
Combining the above inequalities gives 
\[ \frac{4^n}{2n+1} \leq \binom{2n}n \leq 4^{2n/3} (2n)^{\sqrt{2n}}, \]
which implies that 
\[ 4^{n/3} \leq (2n)^{\sqrt{2n}} (2n+1) < (2n)^{\sqrt{2n}+2}. \]
One can check manually that the result holds for $4 \leq n \leq 16$, so assume that $n > 16$. Taking 
logarithms, we find that 
\[ \frac{n}3 \log 4 < (\sqrt{2n}+2)\log(2n) < 2\sqrt n \log(2n) < 2\sqrt n \log(n^{5/4}) < \frac52 \sqrt n \log n. \]
Notice that $\frac{\sqrt n}{\log n}$ is increasing for 
$n > e^2$. Putting this together with the fact that 
\[ \frac{\sqrt{1600}}{\log 1600} \approx 5.421 > 5.410 \approx \frac{15}{2\log4}, \]
we have $n \leq 1600$. Finally, we know that $\{2, 3, 5, 7, 13, 23, 43, 83, 163, 317, 557, 1109, 2207\}$
are all primes, where each number in the set is the largest prime less than twice the previous one. 
Thus, no counterexample exists, and the result holds for all $n \geq 4$.
\end{pf}

\subsection{Gaps Between Twin Primes}
By Theorem~\ref{thm:1.16}, we have 
\[ p_{n+1} - p_n \leq p_n \]
as there is a prime between $p_n$ and $2p_n$. What more can we say about differences of consecutive primes?

By the Prime Number Theorem, there are about $x/\log x$ primes $p$ at most $x$. Therefore, the 
``average gap'' between primes $p$ at most $x$ is $\log x$. However, the value of $p_{n+1} - p_n$ 
can vary widely. 

Notice that for any $n \geq 2$, the numbers $n! + k$ for $2 \leq k \leq n$ are all composite. This 
implies that 
\[ \limsup_{n\to\infty} \;(p_{n+1} - p_n) = \infty. \]
In 1931, Weszynthius showed that 
\[ \limsup_{n\to\infty} \left( \frac{p_{n+1} - p_n}{\log p_n} \right) = \infty. \]
By probabilistic reasoning, Cramer had conjectured in 1936 that 
\[ \limsup_{n\to\infty} \left( \frac{p_{n+1} - p_n}{(\log p_n)^2} \right) \leq 1. \]
In the 1930s, Erd\H{o}s proved that for infinitely many integers $n$, we have 
\[ p_{n+1} - p_n > c \log p_n \frac{\log\log p_n}{(\log\log\log p_n)^2} \]
for some positive constant $c$. In 1938, Rankin added a factor of $\log\log\log\log p_n$. 

What about small gaps between consecutive primes? The famous Twin Prime Conjecture states that 
there are infinitely many $n \in \Z^+$ such that $p_{n+1} - p_n = 2$. Equivalently, it can be stated that
\[ \liminf_{n\to\infty} \; (p_{n+1} - p_n) = 2. \]
If we assume that the primes are randomly distributed and an integer is prime with 
probability $1/\log x$, then we might expect $x$ and $x+2$ to both be prime with probability 
$1/(\log x)^2$. 

Therefore, we expect about $x/(\log x)^2$ primes $p$ such that $p+2$ is also prime and $p \leq x$. 
A more careful heuristic suggests that there are about $C x/(\log x)^2$ such primes $p$ where 
$C > 0$ and $C \neq 1$. In the 1960s, Chen proved that there are more than $0.6x/(\log x)^2$ 
primes $p$ with $p \leq x$ such that $p+2$ is a product of at most two primes (called a $P_2$), 
provided that $x$ is sufficiently large. 

In 2005, Goldston, Pintz, and Yildirim showed that 
\[ \liminf_{n\to\infty} \left( \frac{p_{n+1} - p_n}{\log p_n} \right) = 0. \]
However, this is still quite far from the Twin Prime Conjecture; the bound between consecutive primes 
can still go to infinity. 

Astoundingly, Zhang made a breakthrough in 2013 and showed that 
\[ \liminf_{n\to\infty} \; (p_{n+1} - p_n) \leq 7 \cdot 10^7. \]
This was independently improved by Tao and Maynard (via the Polymath Project) in the same year to get 
\[ \liminf_{n\to\infty} \; (p_{n+1} - p_n) \leq 246. \]\newpage 
\section{Asymptotic Analysis for $\pi(x)$}\label{sec:2}

\subsection{The M\"obius Function}\label{subsec:2.1}

\begin{defn}\label{def:2.1}
Let $f$ and $g$ be functions from $\N$ or $\R^+$ to $\R$, and suppose that $g$ maps to $\R^+$. 
\begin{enumerate}[(1)]
    \item We write $f = O(g)$ if there exist constants $c_1, c_2 > 0$ such that for all $x > c_1$, 
    we have $|f(x)| \leq c_2 g(x)$. 
    \item We write $f = o(g)$ if $\lim_{n\to\infty} f(n)/g(n) = 0$. 
    \item We write $f \sim g$ if $\lim_{n\to\infty} f(n)/g(n) = 1$, and we say that $f$ is 
    {\bf asymptotic} to $g$. 
\end{enumerate}
\end{defn}

By the Prime Number Theorem, we have $\pi(x) \sim x/\log x$, or equivalently, 
\begin{equation}
    \pi(x) = \frac{x}{\log x} + o\left( \frac{x}{\log x} \right). \label{eq:2.1}
\end{equation}
\begin{remark}\label{remark:2.2}
Let $\eps > 0$. Then the number of primes in the interval $[x, (1+\eps)x]$ is 
\[ \pi((1+\eps)x) - \pi(x) = \frac{(1+\eps)x}{\log((1+\eps)x)} - \frac{x}{\log x} + o\left( \frac{x}{\log x} \right). \]
Notice that 
\[ \frac{(1+\eps)x}{\log((1+\eps)x)} = \frac{(1+\eps)x}{\log x + \log(1+\eps)} 
= \frac{(1+\eps)x}{(\log x)(1 + \log(1+\eps)/\log x)} = \frac{(1+\eps)x}{\log x} + o\left( \frac{x}{\log x} \right). \]
Therefore, it follows that 
\[ \pi((1+\eps)x) - \pi(x) = \frac{(1+\eps)x}{\log x} - \frac{x}{\log x} + o\left( \frac{x}{\log x} \right) = \frac{\eps x}{\log x} + o\left( \frac{x}{\log x} \right). \]
By taking $\eps = 1$, we have 
\begin{equation}
    \pi(2x) - \pi(x) = \frac{x}{\log x} + o\left( \frac{x}{\log x} \right). \label{eq:2.2}
\end{equation} 
Equation (\ref{eq:2.2}) might look odd together with equation (\ref{eq:2.1}). Nonetheless, the result is 
correct; it's just that the bounds in the notation $o$ are different. 
\end{remark}

\begin{defn}\label{def:2.3}
We define the {\bf M\"obius function} on $\N$ by 
\[ \mu(n) = \begin{cases} 1 & \text{if } n = 1, \\ 0 & \text{ if $n$ is not squarefree,} \\ 
(-1)^r & \text{if $n$ is a product of $r$ distinct primes.} \end{cases} \]
\end{defn}
For example, we have $\mu(48) = \mu(2^4 \cdot 3) = 0$ and $\mu(30) = \mu(2 \cdot 3 \cdot 5) = 
(-1)^3 = -1$. 

\begin{prop}\label{prop:2.4}
We have 
\[ \sum_{d \mid n} \mu(d) = \begin{cases} 1 & \text{if } n = 1, \\ 0 & \text{otherwise,} \end{cases} \]
where $\sum_{d\mid n}$ means that the summation runs through the positive divisors $d$ of $n$. 
\end{prop}
\begin{pf}
The result is true for $n = 1$. For $n > 1$, let $n = p_1^{a_1} \cdots p_r^{a_r}$ be the unique 
factorization of $n$ into distinct prime numbers. Set $N = p_1 \cdots p_r$ (which is called the 
{\bf radical} of $n$). Since $\mu(d) = 0$ when $d$ is not squarefree, we have 
\[ \sum_{d\mid n} \mu(d) = \sum_{d\mid N} \mu(d). \]
Note that the divisors of $N$ are in bijective correspondence with the subsets of $\{p_1, \dots, p_r\}$. 
Since the number of $k$ element subsets is $\binom{r}{k}$ and the corresponding divisor $d$ of such 
a set satisfies $\mu(d) = (-1)^k$, we have 
\[ \sum_{d \mid n} \mu(d) = \sum_{d \mid N} \mu(d) = \sum_{k=0}^r \binom rk (-1)^k = (1-1)^r = 0. \qedhere \]
\end{pf}

\begin{prop}[M\"obius Inversion Formula]\label{prop:2.5}~
\begin{enumerate}[(1)]
    \item For two functions $f, g : \R^+ \to \C$, we have 
    \[ g(x) = \sum_{1 \leq n \leq x} f(x/n) \]
    if and only if 
    \[ f(x) = \sum_{1 \leq n \leq x} \mu(n) g(x/n). \]
    \item For two functions $f, g : \N \to \C$, we have 
    \[ f(n) = \sum_{d \mid n} g(d) \]
    if and only if 
    \[ g(n) = \sum_{d \mid n} \mu(d) f(n/d). \]
\end{enumerate}
\end{prop}
\begin{pf}
This is on Homework 1.
\end{pf}

\subsection{The von Mangoldt Function}\label{subsec:2.2}

\begin{defn}\label{def:2.6}
We define the {\bf von Mangoldt function} on $\N$ by 
\[ \Lambda(n) = \begin{cases} \log p & \text{if $n = p^k$ for $p$ prime and $k \in \N$,} \\ 0 & \text{otherwise.} \end{cases} \]
Moreover, for all $x \in \R$, we define the functions 
\begin{align*}
    \theta(x) &= \sum_{p\leq x} \log p = \log \prod_{p \leq x} p, \\ 
    \psi(x) &= \sum_{p^k \leq x} \log p = \sum_{n \leq x} \Lambda(n). 
\end{align*}
\end{defn}

Notice that 
\[ \psi(x) = \sum_{p\leq x} \left\lfloor \frac{\log x}{\log p} \right\rfloor \log p. \]
Since $p^2 \leq x$ is equivalent to $p \leq x^{1/2}$ and $p^3 \leq x$ if and only if $p \leq x^{1/3}$, 
we see that 
\[ \psi(x) = \theta(x) + \theta(x^{1/2}) + \theta(x^{1/3}) + \cdots. \]
Note that $\theta(x^{1/m}) = 0$ when $m > \frac{\log x}{\log 2}$. Therefore, we get 
\[ \psi(x) = \sum_{k=1}^{\left\lfloor \frac{\log x}{\log 2} \right\rfloor} \theta(x^{1/k}). \]
Observe that we have the inequality 
\[ \theta(x) = \sum_{p\leq x} \log p \leq x \log x, \]
so it follows that 
\[ \sum_{k \geq 2} \theta(x^{1/k}) = O\left( x^{1/2} (\log x)^2 \right). \]
Therefore, we obtain 
\[ \psi(x) = \theta(x) + O\left( x^{1/2} (\log x)^2 \right) \]
and so by Theorem~\ref{thm:1.12}, we get 
\[ \theta(x) = \sum_{p\leq x}\log p \leq \pi(x) \log x < c_1 x \]
for $x \geq 2$ and a constant $c_1 > 0$. Similarly, one finds that $\psi(x) < c_2x$ for $x \geq 2$
and a positive constant $c_2$. Furthermore, from the proof of Theorem~\ref{thm:1.12}, we have $2^n \leq \binom{2n}n$
and $\binom{2n}n \mid \prod_{p \leq 2n} p^{r_p}$, where $r_p$ is the integer satisfying 
$p^{r_p} \leq 2n < p^{r_p+1}$. It follows that 
\[ n\log 2 = \log(2^n) \leq \log \binom{2n}n \leq \sum_{p\leq 2n} r_p \log p \leq 
\sum_{p \leq 2n} \left\lfloor \frac{\log(2n)}{\log p} \right\rfloor \log p \leq \psi(2n). \]
For $x \geq 2$, choosing $n$ such that $2n \leq x < 2n+2$ gives 
\[ \psi(x) \geq \psi(2n) \geq n\log 2 > \frac{x-2}2 \log 2. \]
Hence, we have $\psi(x) > c_3x$ and $\theta(x) > c_4x$ for positive constants $c_3$ and $c_4$. 

What is the relationship between $\theta(x)$, $\psi(x)$, and $\pi(x)$? We note that 
\[ \theta(x) = \sum_{p \leq x} \log p \leq x \log p \leq \pi(x) \log x, \]
so it follows that 
\[ \pi(x) \geq \frac{\theta(x)}{\log x} > c_4 \frac{x}{\log x}. \]

\begin{thm}\label{thm:2.7}
We have 
\[ \pi(x) \sim \frac{\theta(x)}{\log x} \sim \frac{\psi(x)}{\log x}. \]
\end{thm}
\begin{pf}
Since $\psi(x) = \theta(x) + O(x^{1/2} (\log x)^2)$ and $\theta(x) > c_4x$, we see that $\theta(x) 
\sim \psi(x)$. In particular, we have $\theta(x)/\log x \sim \psi(x)/\log x$, so it only 
remains to show that $\pi(x) \sim \theta(x)/\log x$. 

We have already shown that $\pi(x) \geq \theta(x) \geq \log x$, so 
\[ \liminf_{n\to\infty} \frac{\pi(x) \log x}{\theta(x)} \geq 1. \]
We need an upper bound for $\pi(x)$ in terms of $\theta(x)$. Note that for any $\delta > 0$, we have 
\[ \theta(x) = \sum_{p\leq x} \log p \geq \log(x^{1-\delta}) \sum_{x^{1-\delta} \leq p \leq x} 1 
\geq (1-\delta)(\log x)\left(\pi(x) - \pi(x^{1-\delta})\right). \]
Since $\pi(y) \leq y$ for all real numbers $y > 0$, we get 
\[ \theta(x) + (1-\delta)x^{1-\delta} \log x \geq (1-\delta)(\log x)\pi(x). \]
Rearranging the above gives 
\[ \frac{\theta(x)}{(1-\delta)\log x} + x^{1-\delta} \geq \pi(x), \]
and therefore 
\[ \frac{1}{1-\delta} + \frac{x^{1-\delta}\log x}{\theta(x)} \geq \frac{\pi(x)\log x}{\theta(x)}. \]
Given $\eps > 0$, we can choose $\delta > 0$ such that $\frac{1}{1-\delta} < 1 + \frac{\eps}2$, 
and then pick $x_0$ such that if $x > x_0$, then 
\[ \frac{x^{1-\delta}\log x}{\theta(x)} < \frac\eps2 \] 
since $\theta(x) > c_1x$ for $x \geq 2$. Then for all $x > x_0$, we have 
\[ 1 \leq \frac{\pi(x) \log x}{\theta(x)} < 1 + \eps, \]
which completes the proof.
\end{pf}

\subsection{Abel's Summation Formula}\label{subsec:2.3}
We will prove Abel's summation formula and give some of its
applications.

\begin{lemma}[Abel's summation formula]\label{lemma:2.8}
Let $\{a_n\}_{n=1}^\infty$ be a sequence of complex numbers. 
Let $f : \{x \in \R : x \geq 1\} \to \C$ be a function. 
For all $x \geq 1$, we define 
\[ A(x) := \sum_{n\leq x} a_n, \]
where the summation runs through all positive integers up to 
$x$. If $f'$ is continuous at every $x \geq 1$, then 
\[ \sum_{n\leq x} a_n f(n) = A(x) f(x) - \int_1^x A(u)f'(u)\dd u. \]
\end{lemma}
\begin{pf}
Set $N = \lfloor x \rfloor$. Note that 
$a_n = A(n) - A(n-1)$ for all $n \geq 2$, so we can write 
\begin{align*}
    \sum_{n\leq N} a_n f(n) &= A(1)f(1) + 
    (A(2) - A(1))f(2) + \cdots + (A(N) - A(N-1)) f(N) \\
    &= A(1)(f(1) - f(2)) + \cdots + A(N-1)(f(N-1) - f(N)) + A(N)f(N).
\end{align*}
Observe that if $i \in \Z^+$ and $t \in \R$ with $i \leq t < i+1$, then $A(t) = A(i)$. 
It follows that 
\[ A(i) (f(i) - f(i+1)) = -\int_i^{i+1} A(u) f'(u)\dd u. \]
Therefore, we have 
\[ \sum_{n\leq N} a_n f(n) = -\int_1^N A(u)f'(u)\dd u + 
A(N)f(N), \]
so the result holds when $x$ is an integer. Now, notice that 
$A(t) = A(N)$ for all $x \geq t \geq N$, so we obtain 
\[ \int_N^x A(u) f'(u)\dd u = A(x)(f(x) - f(N)) = 
A(x)f(x) - A(N)f(N). \]
Thus, the result holds for all $x \geq 1$.
\end{pf}

\begin{defn}\label{def:2.9}
Given $x \in \R$, we denote the {\bf fractional part} of 
$x$ by $\{x\}$; that is, 
\[ \{x\} := x - \lfloor x \rfloor. \]
We define {\bf Euler's constant} by 
\[ \gamma := 1 - \int_1^\infty \frac{\{t\}}{t^2}\dd t 
= 1 - \int_1^\infty \frac{t - \lfloor t \rfloor}{t^2}\dd t. \]
Note that $\gamma \approx 0.55721$. 
\end{defn}

This has not been proven, but it has been conjectured that 
$\gamma$ is irrational and transcendental.

\begin{thm}\label{thm:2.10}
We have 
\[ \sum_{n \leq x} \frac1n = \log x + \gamma + O
\left( \frac1x \right). \]
\end{thm}
\begin{pf}
Taking $a_n = 1$ and $f(t) = 1/t$ in Abel's summation formula,
we have 
\[ A(x) = \sum_{n\leq x} a_n = \sum_{n \leq x} 1 = \floor{x} \]
so that 
\begin{align*}
    \sum_{n\leq x} \frac1n &= \frac{\floor{x}}{x} + \int_1^x \frac{\floor{u}}{u^2}\dd u \\
    &= \frac{x - (x - \floor x)}{x} + \int_1^x \frac{u - (u - \floor u)}{u^2}\dd u \\
    &= 1 + O\left(\frac1x\right) + \int_1^x \frac{{\rm d}u}u -
    \int_1^x \frac{u - \floor u}{u^2}\dd u \\
    &= 1 + O\left(\frac1x\right) + \log x - \left( \int_1^\infty \frac{u-\floor u}{u^2}\dd u - \int_x^\infty \frac{u-\floor u}{u^2}\dd u \right) \\
    &= \log x + \gamma + O\left(\frac1x\right) + 
    \int_x^\infty \frac{u-\floor u}{u^2}\dd u \\
    &= \log x + \gamma + O\left(\frac1x\right) 
    + O\left(\int_x^\infty \frac1{u^2}\dd u \right) \\
    &= \log x + \gamma + O\left(\frac1x\right). \qedhere
\end{align*}
\end{pf}

\begin{thm}\label{thm:2.11}
We have 
\[ \sum_{n \leq x} \frac{\Lambda(n)}n = \log x + O(1). \]
\end{thm}
\begin{pf}
First, we apply Abel's summation formula with $a_n = 1$ 
and $f(n) = \log n$ to get 
\begin{align*}
    \sum_{n\leq x} \log n
    &= \floor x \log x - \int_1^x \frac{\floor u}u\dd u \\
    &= (x - (x - \floor x))\log x - \int_1^x 
    \frac{u - (u - \floor u)}u \dd u \\
    &= x\log x + O(\log x) - (x-1) + \int_1^x \frac{u - \floor u}u\dd u \\
    &= x\log x - x + O(\log x).
\end{align*}
On the other hand, we have 
\begin{align*}
    \sum_{n\leq x} \log n 
    = \log(\floor{x}!) 
    &= \sum_{p\leq x} \left( \sum_{k=1}^\infty \floor*{\frac{x}{p^k}} \right)\log p \\
    &= \sum_{p^m \leq x} \floor*{\frac{x}{p^m}} \log p \\
    &= \sum_{n \leq x} \floor*{\frac xn} \Lambda(n) \\
    &= \sum_{n \leq x} \frac xn \Lambda(n) - 
    \sum_{n\leq x} \left( \frac xn - \floor*{\frac xn} \right)
    \Lambda(n) \\
    &= x \sum_{n\leq x} \frac{\Lambda(n)}n - O\left( \sum_{n\leq x} \Lambda(n) \right). 
\end{align*}
Since $\sum_{n\leq x} \Lambda(n) = \psi(x) = O(x)$, we have 
\[ \sum_{n\leq x} \log x = x \sum_{n \leq x} \frac{\Lambda(n)}n - O(n). \]
By the asymptotic formula of $\sum_{n\leq x} \log n$ above, we see that 
\[ x\log x - x + O(\log x) = x\sum_{n\leq x} \frac{\Lambda(n)}n - O(x). \]
Rearranging and tucking some terms under $O(x)$ gives 
\[ x \sum_{n\leq x} \frac{\Lambda(n)}n = x\log x + O(x). \]
Finally, dividing through by $x$ gives 
\[ \sum_{n\leq x} \frac{\Lambda(n)}n = \log x + O(1). \qedhere \]
\end{pf}

\begin{thm}\label{thm:2.12}
We have
\[ \sum_{p\leq x} \frac{\log p}p = \log x + O(1). \]
\end{thm}
\begin{pf}
Note that 
\[ \sum_{p \leq x} \frac{\log p}p 
= \sum_{n \leq x} \frac{\Lambda(n)}n - \sum_{m\geq2} \sum_{p^m\leq x} \frac{\log p}{p^m} \\
= \log x + O(1) - \sum_{m\geq2}\sum_{p^m\leq x}
\frac{\log p}{p^m}. \]
Moreover, we see that 
\[ \sum_{m\geq2}\sum_{p^m\leq x}
\frac{\log p}{p^m} \leq \sum_p \left( \frac1{p^2} + \frac1{p^3} + \cdots \right)\log p 
\leq \sum_p \frac{\log p}{p(p-1)} 
\leq \sum_{n=2}^\infty \frac{\log n}{n(n-1)} = O(1), \]
which completes the proof.
\end{pf}

\begin{thm}[Merten]\label{thm:2.13}
There exists a real number $\beta$ such that 
\[ \sum_{p \leq x} \frac1p = \log\log x + \beta + O\left(\frac1{\log x}\right). \]
\end{thm}
\begin{pf}
We apply Abel's summation formula with 
\[ a_n = \begin{cases} \frac{\log p}p & \text{if $n = p$ for a prime $p$} \\ 0 & \text{otherwise} \end{cases} \]
and $f(n) = 1/\log n$. Setting $A(x) = \sum_{n\leq x} a_n$, we have 
\[ \sum_{p \leq x} \frac1p = \frac{A(x)}{\log x} + 
\int_1^x \frac{A(u)}{u(\log u)^2}\dd u. \]
By Theorem~\ref{thm:2.12}, we have 
\[ A(x) = \sum_{p\leq x} \frac{\log p}p = \log x + O(1), \]
so we see that 
\[ \sum_{p\leq x} \frac1p = 1 + O\left(\frac1{\log x}\right)
+ \int_2^x \frac{\log u + \tau(u)}{u(\log u)^2}\dd u, \]
where $\tau(u) = A(u) - \log u = O(1)$. Therefore, we have 
\begin{align*}
    \sum_{p\leq x} \frac1p 
    &= 1 + O\left( \frac1{\log x} \right) + \log\log x 
    - \log\log 2 + \int_2^x \frac{\tau(u)}{u(\log u)^2}\dd u \\
    &= \log\log x + 1 - \log\log 2 + \int_2^\infty 
    \frac{\tau(u)}{u(\log u)^2}\dd u - 
    \int_x^\infty \frac{\tau(u)}{u(\log u)^2}\dd u + 
    O \left( \frac1{\log x} \right). 
\end{align*}
By setting $\beta$ to the middle terms above, we are done.
\end{pf}

In fact, we have 
\[ \beta = \gamma + \sum_p \left[ \log \left( 1 - \frac1p \right) + \frac1p \right] \approx 0.261497, \]
and $\beta$ is called {\bf Merten's constant}.\newpage 
\newpage 
\section{Riemann's Zeta Function and the Prime Number Theorem}

\subsection{The Riemann Zeta Function}
In order to prove the Prime Number Theorem, we need to 
first introduce the Riemann zeta function. 

\begin{defn}
For $s \in \C$ with $\Re(s) > 1$, we define the 
{\bf Riemann zeta function} by 
\[ \zeta(s) := \sum_{n=1}^\infty \frac1{n^s}. \]
We will denote $s = \sigma + it$ where $\sigma, t \in \R$. 
\end{defn}

Note that the series $\sum_{n=1}^\infty n^{-s}$ converges 
absolutely when $\Re(s) > 1$. 

Recall that the infinite product $\prod_n (1 + a_n)$ 
converges absolutely (that is, it is finite and non-zero)
if and only if $\sum_n |a_n|$ converges. We have the 
{\bf Euler product representation} of $\zeta(s)$ given 
in the following lemma. 

\begin{lemma}[Euler product]
For $s \in \C$ with $\Re(s) > 1$, we have 
\[ \prod_p \left( 1 - \frac1{p^s} \right)^{\!-1} = 
\sum_{n=1}^\infty \frac1{n^s}. \]
\end{lemma}
\begin{pf}
Note that 
\[ \prod_p \left( 1 - \frac1{p^s} \right)^{\!-1} = 
\prod_p \left(1 + \frac1{p^2} + \frac1{p^3} + \cdots \right). \]
A typical term in the sum is of the form 
\[ \frac{1}{p_1^{\alpha_1s} \cdots p_k^{\alpha_ks}}
= \frac{1}{(p_1^{\alpha_1} \cdots p_k^{\alpha_k})^s}. \]
By the Fundamental Theorem of Arithmetic, every positive 
integer can be expressed uniquely as a product of primes, 
so the identity holds.
\end{pf}

\begin{thm}
$\zeta(s)$ can be analytically continued to $s \in \C$ with 
$\Re(s) > 0$ and $s \neq 1$. It is analytic except 
at the point $s = 1$ where it has a simple pole with residue
$1$. 
\end{thm}
\begin{pf}
For $s \in \C$ with $\Re(s) > 1$, we have 
$\zeta(s) = \sum_{n=1}^\infty n^{-s}$. By Abel's 
summation formula with $a_n = 1$ and $f(x) = x^{-s}$, 
we find that 
\[ \sum_{n\leq x} \frac{1}{n^s} = \frac{\floor x}{x^s}
+ s\int_1^x \frac{\floor u}{u^{s+1}}\dd u. \]
Letting $x \to \infty$, we obtain 
\begin{align*}
    \zeta(s) &= 0 + s\int_1^\infty \frac{\floor u}{u^{s+1}}\dd u \\
    &= s \int_1^\infty \frac{u - (u - \floor u)}{u^{s+1}}\dd u \\
    &= s \int_1^\infty \frac{u}{u^{s+1}}\dd u - 
    s \int_1^\infty \frac{u - \floor u}{u^{s+1}}\dd u \\
    &= s \left( \frac{u^{1-s}}{1-s} \biggr\rvert_1^\infty 
    \right)
    - s \int_1^\infty \frac{u - \floor u}{u^{s+1}}\dd u \\
    &= \frac{s}{s-1} - s \int_1^\infty \frac{u - \floor u}{u^{s+1}}\dd u
\end{align*}
for $\Re(s) > 1$. Note that 
\[ \int_1^\infty \frac{u - \floor u}{u^{s+1}}\dd u \]
converges for $\Re(s) > 0$ and represents an analytic function. Therefore, we see that 
\[ \frac{s}{s-1} - s \int_1^\infty \frac{u - \floor u}{u^{s+1}}\dd u \] 
is an analytic function for $\Re(s) > 0$ with $s \neq 1$. 
This gives a meromorphic continuation of $\zeta(s)$ 
to the region $\{s \in \C : \Re(s) > 0\}$. Finally, note that 
$\frac{s}{s-1}$ has a simple pole with residue $1$ at $s = 1$.
\end{pf}

\begin{thm}
$\zeta(s)$ has no zeroes in the region $\{s \in \C : 
\Re(s) \geq 1\}$. 
\end{thm}
\begin{pf}
If $\Re(s) > 1$, then $\prod_p (1 - \frac1{p^s})^{-1}$ 
converges, so $\zeta(s) \neq 0$. 

It only remains to consider the case where $\Re(s) = 1$. 
We will first do some preliminary work. 

Recall that we denote $s = \sigma + it$ where $\sigma, t \in \R$. Let $\sigma > 1$. Then for all $ t \in \R$, we have 
\[ \log^*(\zeta(\sigma + it)) = 
\log \left( \prod_p \left(1 + \frac1{p^s}\right)^{\!-1}\right) = \sum_p \sum_{n=1}^\infty \frac1n \left( \frac1{p^{ns}} \right), \]
where $\log$ denotes the principal branch and 
$\log^*$ denotes some branch of the logarithm (we have to be
careful here as we are considering complex numbers). Comparing
the real parts of the above equality, we have 
\[ \log|\zeta(\sigma+it)| = \sum_p \sum_{n=1}^\infty 
\frac{p^{-\sigma n} \cos(nt \log p)}{n}, \]
since we can write 
\[ p^{-int} = e^{-int \log p} = \cos(-nt \log p) + 
i \sin(-nt \log p) = \cos(nt \log p) - i\sin(nt \log p) \]
and therefore $\Re(p^{-int}) = \cos(nt \log p)$. 
Moreover, observe that we have the inequality 
\begin{align*}
    0 \leq 2(1+\cos\theta)^2 
    &= 2(1 + 2\cos\theta + \cos^2\theta) \\
    &= 2 + 4\cos\theta + 2\cos^2\theta \\
    &= 3 + 4\cos\theta + (2\cos^2\theta - 1) \\
    &= 3 + 4\cos\theta + \cos(2\theta). 
\end{align*}
From this, we can deduce that 
\[ \sum_p \sum_{n=1}^\infty \frac{p^{-\sigma n}}n 
(3 + 4\cos(nt \log p) + \cos(2nt \log p)) \geq 0. \]
Therefore, we have 
\[ \log|\zeta(\sigma)|^3 + \log|\zeta(\sigma+it)|^4 + 
\log|\zeta(\sigma+2it)| \geq 0. \]
In particular, we see that 
\begin{equation}
    |\zeta(\sigma)|^3 \cdot |\zeta(\sigma + it)|^4 
\cdot |\zeta(\sigma+2it)| \geq 1
\end{equation} 
for $\sigma > 1$ and $t \in \R$. 

Suppose now that $1 + it_0$ is a zero of $\zeta(s)$, 
and note that $t_0 \neq 0$ as $\zeta(s)$ has a pole at 
$s = 1$. By taking $t \to 1^+$ (that is, from the right), 
we observe tht 
\[ |\zeta(s)| = O((\sigma - 1)^{-1}) \]
since $1$ is a simple pole of $\zeta(s)$. Moreover, 
since $1 + it_0$ is a zero of $\zeta(s)$, we have 
$|\zeta(\sigma + it_0)| = O(\sigma-1)$ as 
$\sigma \to 1^+$. Finally, we have 
$|\zeta(\sigma + 2it_0)| = O(1)$ as $\sigma \to 1^+$ 
since $1 + 2it_0$ is not a simple pole of $\zeta(s)$. 
It follows that 
\[ |\zeta(\sigma)|^3 \cdot |\zeta(\sigma + it)|^4 
\cdot |\zeta(\sigma + 2it)| = 
O((\sigma - 1)^{-3}) \cdot O((\sigma - 1)^4) \cdot O(1) 
= O(\sigma - 1). \]
Thus, $|\zeta(s)|^3 \cdot |\zeta(\sigma+it)|^4 \cdot 
|\zeta(\sigma + 2it)|$ tends to $0$ as $\sigma \to 1^+$. 
But this contradicts that the lower bound we found in (3.1), 
so we conclude that $\zeta(s)$ cannot have a zero when 
$\Re(s) = 1$. 
\end{pf}

\subsection{Newman's Theorem}

\begin{thm}[Newman]
Let $\{a_n\}_{n=1}^\infty$ be a sequence of complex numbers with $|a_n| \leq 1$ for all $n \geq 1$. 
Consider the series $\sum_{n=1}^\infty a_n/n^s$, which converges to an analytic function 
$F(s)$ for $\Re(s) > 1$. If $F(s)$ can be analytically continued to $\Re(s) \geq 1$, then 
$\sum_{n=1}^\infty a_n/n^s$ converges to $F(s)$ for $\Re(s) \geq 1$. 
\end{thm}
\begin{pf}
Let $w \in \C$ with $\Re(w) \geq 1$. Then $F(z+w)$ is analytic for $\Re(z) \geq 0$. Choose 
$R \geq 1$ and let $\delta = \delta(R) > 0$ so that $F(z+w)$ is analytic on the region 
\[ \tilde\Gamma := \{z \in \C : \Re(z) \geq -\delta \text{ and } |z| \leq R\}. \]
To see why such a $\delta > 0$ exists, first note that $F(z+w)$ is analytic for $\Re(z) \geq 0$. 
Consider the line $L = \{z = iy : |y| \leq R\}$. Every point in $L$ has an open cover such that 
$F(z+w)$ is analytic on that cover; call the union of these covers $U$. Since $L$ is compact\footnote{Recall that a set $X$ is compact if every open cover of $X$ has a finite subcover.}, there exists a finite open subcover $\tilde U$ of $U$ such that 
$L \subseteq \tilde U \subseteq U$. Since the number of open sets in $\tilde U$ is finite, 
it follows that such a $\delta > 0$ exists.

Let $M$ denote the maximum of $|F(z+w)|$ on $\tilde \Gamma$, and let $\Gamma$ denote the contour 
obtained by following the outside of $\tilde\Gamma$ in a counterclockwise path. Let $A$ be the 
part of $\Gamma$ in $\Re(z) > 0$, and let $B = \Gamma \setminus A$. For $N \in \N$, consider the 
function 
\[ F(z+w)N^z \left( \frac1z + \frac z{R^2} \right), \]
which is analytic on $\tilde\Gamma$ except at $z = 0$ where there is a simple pole with residue 
$F(0+w) N^0 = F(w)$. By Cauchy's residue theorem, we obtain 
\begin{align*}
    2\pi i F(w) &= \int_\Gamma F(z+w) N^z \left( \frac1z + \frac z{R^2} \right)\dd z \\
    &= \int_A F(z+w) N^z \left( \frac1z + \frac z{R^2} \right)\dd z + 
    \int_B F(z+w) N^z \left( \frac1z + \frac z{R^2} \right)\dd z. \tag{3.2} 
\end{align*}
Observe that $F(z+w)$ is equal to its series on $A$. We split the series as 
\[ S_N(z+w) = \sum_{n=1}^N \frac{a_n}{n^{z+w}} \]
and $R_N(z+w) = F(z+w) - S_N(z+w)$. Note that $S_N(z+w)$ is analytic for all $z \in \C$. Let 
$C$ be the contour given by the path $|z| = R$ taken in the counterclockwise direction. 
By Cauchy's residue theorem, we obtain 
\[ 2\pi i S_N(w) = \int_C S_N(z+w) N^z \left( \frac1z + \frac z{R^2} \right)\dd z \]
since the integrand has a simple pole at $z = 0$ with residue $S_N(0+w) N^0 = S_N(w)$. Note that 
\[ C = A \cup (-A) \cup \{iR, -iR\}. \]
Therefore, we see that 
\[ 2\pi i S_N(w) = \int_A S_N(z+w) N^z \left( \frac1z + \frac z{R^2} \right)\dd z 
+ \int_{-A} S_N(z+w) N^z \left( \frac1z + \frac z{R^2} \right)\dd z. \]
Consider the second integral above. Using the change of variables $z \to -z$, we find that 
\[ \int_{-A} S_N(z+w) N^z \left( \frac1z + \frac z{R^2} \right)\dd z 
= \int_A S_N(-z+w) N^{-z} \left( \frac1z + \frac z{R^2} \right) \dd z. \]
Thus, we obtain 
\[ 2\pi i S_N(w) = \int_A \left( S_N(z+w) N^z + S_N(-z + w) N^{-z} \right) \left( \frac1z + \frac{z}{R^2} \right)\dd z. \]
Combining the above equality with $(3.2)$, we have 
\begin{align*}
    2\pi i(F(w) - S_N(w)) &= \int_A \left( R_N(z+w) N^z - S_N(-z+w) N^{-z} \right) \left( \frac1z + \frac z{R^2} \right) \dd z\\ &\hspace{1.5cm} + \int_B F(z+w) N^z \left( \frac1z + \frac z{R^2} \right)\dd z. \tag{3.3}
\end{align*}  
Our goal is to show that $S_N(w)$ converges to $F(w)$ as $N \to \infty$. Write $z = x + iy$
where $x, y \in \R$. Then for $z \in A$, we have $x > 0$ and $|z| = R$, so 
\[ \frac1z + \frac z{R^2} = \frac{x-iy}{R^2} + \frac{x + iy}{R^2} = \frac{2x}{R^2}. \]
Since $|n^z| = n^x$, we have 
\[ |R_N(z+w)| \leq \sum_{n=N+1}^\infty \frac{1}{n^{\Re(z+w)}} \leq \sum_{n=N+1}^\infty 
\frac{1}{n^{x+1}} \leq \int_N^\infty \frac{1}{u^{x+1}}\dd u = \frac{1}{xN^x}. \]
Also, we have 
\[ |S_N(-z+w)| \leq \sum_{n=1}^N \frac{1}{n^{-x+1}} \leq N^{x-1} + \int_1^N u^{x-1}\dd u 
\leq N^{x-1} + \frac{N^x}{x} = N^x \left( \frac1N + \frac1x \right). \]
Putting the above estimates together, we get 
\begin{align*}
    \left| \int_A \left( R_N(z+w) N^z - S_N(-z+w) N^{-z} \right) \left( \frac1z + \frac z{R^2} \right) \dd z \right| 
    &\leq \int_A \left( \frac{1}{xN^x} N^x + N^x \left( \frac1N + \frac1x \right) N^{-x} \right) 
    \frac{2x}{R^2}\dd z \\
    &= \int_A \left( \frac2x + \frac1N \right) \frac{2x}{R^2}\dd z \\
    &= \int_A \left( \frac4{R^2} + \frac{2x}{NR^2} \right)\dd z \\
    &\leq \pi R \left( \frac4{R^2} + \frac{2}{NR} \right) \quad \text{(since $x \leq R$)} \\
    &\leq \frac{4\pi}R + \frac{2\pi}N. 
\end{align*}
We now estimate the integral along $B$. We can divide $B$ into two parts; one part with $\Re(z) = 
-\delta$, and the other with $-\delta < \Re(z) \leq 0$. For $z \in B$ with $\Re(z) = -\delta$, 
we use the fact that $|z| \leq R$ to find that 
\[ \left| \frac1z + \frac z{R^2} \right| = \left| \frac1z \right| \left| \frac{\bar z}{z} + 
\frac{z\bar z}{R^2} \right| \leq \frac1\delta \left( 1 + \frac{|z|^2}{R^2} \right) \leq \frac2\delta. \]
Since $|F(z+w)| \leq M$ for $z \in B$, we have 
\begin{align*}
    \left| \int_B F(z+w) N^z \left( \frac1z + \frac z{R^2} \right)\dd z \right| 
    &\leq \int_{-R}^R MN^{-\delta} \frac2\delta\dd z + 2 \left| \int_{-\delta}^0 MN^x \frac{2x}{R^2}\dd x \right| \\
    &= \frac{4MR}{\delta N^\delta} + \frac{4M}{R^2} \left| \int_{-\delta}^0 xN^x \dd x \right| \\
    &\leq \frac{4MR}{\delta N^\delta} + \frac{4M\delta}{R^2} \left( \frac1{(\log N)^2} - \frac{\delta+1}{N^\delta \log N} \right) \\
    &\leq \frac{4MR}{\delta N^\delta} + \frac{4M\delta}{R^2(\log N)^2}. 
\end{align*}
Combining this estimate with $(3.2)$ and $(3.3)$ yields 
\[ |2\pi i(F(w) - S_N(w))| \leq \frac{4\pi}R + \frac{2\pi}N + \frac{4MR}{\delta N^\delta} 
+ \frac{4M\delta}{R^2(\log N)^2}. \] 
That is, we have 
\[ |F(w) - S_N(w)| \leq \frac2R + \frac1N + \frac{MR}{\delta N^\delta} + \frac{M\delta}{R^2(\log N)^2}. \]
Given $\eps > 0$, choose $R = 3/\eps$. Then for sufficiently large $N$, we have 
\[ |F(w) - S_N(w)| < \eps. \]
This implies that $S_N(w) \to F(w)$ as $N \to \infty$, which completes the proof. 
\end{pf}

\subsection{Revisiting the M\"obius Function}
Recall that we defined the M\"obius function $\mu : \N \to \{-1, 0, 1\}$ by 
\[ \mu(n) = \begin{cases} 1 & \text{if } n = 1, \\ 0 & \text{if $n$ is not squarefree,} \\ (-1)^r & \text{if $n$ is the product of $r$ distinct primes.} \end{cases} \]
We will show on Homework 2 that for $\Re(s) > 1$, we have 
\[ \frac1{\zeta(s)} = \prod_p \left( 1 - \frac1{p^s} \right) = \sum_{n=1}^\infty \frac{\mu(n)}{n^s}. \]

\begin{thm}
We have 
\[ \sum_{n=1}^\infty \frac{\mu(n)}n = 0. \]
\end{thm}
\begin{pf}
For all $\Re(s) > 1$, equation (3.4) holds. Moreover, we have shown that $(s-1)\zeta(s)$ is analytic 
and non-zero in $\Re(s) \geq 1$, so $1/\zeta(s)$ is analytic on $\Re(s) \geq 1$. 
Now, $\zeta(s)$ can be analytically continued up to $\Re(s) > 0$ and it is nonzero for $\Re(s) \geq 1$,
so we see that the series 
\[ \sum_{n=1}^\infty \frac{\mu(n)}{n^s} \] 
converges to $1/\zeta(s)$ for $\Re(s) \geq 1$. In particular, it converges at $s = 1$. But 
$\zeta(s)$ has a simple pole at $s = 1$, so $1/\zeta(1) = 0$. 
\end{pf}

\begin{thm}
We have 
\[ \sum_{n\leq x} \mu(n) = o(x). \]
\end{thm}
\begin{pf}
Applying Abel's summation formula with $a_n = \mu(n) / n $ and $f(x) = x$, we obtain 
\[ \sum_{n\leq x} \mu(n) = A(x)x - \int_1^x A(u)\dd u, \]
where we hae 
\[ A(t) = \sum_{n\leq t} \frac{\mu(n)}n. \]
By Theorem 3.5, we know that $A(t) = o(1)$. It follows that $A(x) x = o(x)$ and 
\[ \int_1^x A(u)\dd u = o(x), \]
so the result holds. 
\end{pf}

\subsection{Divisor Function}

\begin{defn}
For a positive integer $n \in \N$, let $d(n)$ be the number of positive integers that 
divide $n$. 
\end{defn}

For example, we have $d(1) = 1$, $d(4) = 3$, and $d(p) = 2$ for all primes $p$. 

\begin{thm}
We have 
\[ \sum_{m=1}^n d(m) = \sum_{m=1}^n \floor*{\frac nm} = n\log n + (2\gamma - 1)n + O(n^{1/2}). \]
where $\gamma$ denotes Euler's constant. 
\end{thm}
\begin{pf}
Let $D_n$ be the region in the upper right-hand quadrant not containing the $x$ or $y$ axes, 
which is under and includes the hyperbola $xy = n$. That is, 
\[ D_n := \{(x, y) \in \R^2 : x > 0,\, y > 0,\, xy \leq n\}. \]
Define a {\bf lattice point} to be a point in the plane with integer coordinates; that is, a point 
$(x, y) \in \R^2$ with $x, y \in \Z$. Notice that every lattice point in $D_n$ is contained in 
some hyperbola $xy = s$ where $s$ is an integer with $1 \leq s \leq n$. 

Therefore, $\sum_{s=1}^n d(s)$ is the number of lattice points in $D_n$; that is, 
\[ \sum_{s=1}^n d(s) = \#\{(x, y) \in \R^2 : x, y \in \N,\, xy \leq n\}. \]
We now count the number of lattice points in a different way. Given $x \in \N$ with $1 \leq x \leq n$, 
there are exactly $\floor{\frac xn}$ many integers $y$ such that $xy \leq n$. Thus, we see that 
\[ \#\{(x, y) \in \R^2 : x, y \in \N,\, xy \leq n\} = \sum_{x=1}^n \floor*{\frac nx}. \]
Observe that the number of lattice points above the line $x = y$ inside $D_n$ is equal to the number of 
lattice points below it. Divide the lattice points in $D_n$ into three disjoint regions given by 
\begin{align*}
    D_{n,1} &= \{(x, y) \in \N^2 : xy \leq n,\, x < y\}, \\ 
    D_{n,2} &= \{(x, y) \in \N^2 : xy \leq n,\, x > y\}, \\
    D_{n,3} &= \{(x, y) \in \N^2 : xy \leq n,\, x = y\}.
\end{align*}
Our observation above shows that $|D_{n,1}| = |D_{n,2}|$. Suppose that $(x, y) \in D_{n,1}$. Then 
$x^2 < xy \leq n$, which implies that $x < \sqrt n$. Moreover, for a fixed integer $x$, the number of 
integers $y$ satisfying $xy \leq n$ and $y > x$ is $\floor{\frac nx} - \floor x$. We also see that 
$|D_{n,3}| = \floor{\sqrt n}$, so we obtain 
\begin{align*}
    \sum_{x=1}^n \floor*{\frac nx} &= |D_{n,1}| + |D_{n,2}| + |D_{n,3}| \\
    &= 2 \sum_{x=1}^{\floor{\sqrt n}} \left( \floor*{\frac nx} - \floor x \right) + \floor{\sqrt n} \\
    &= 2 \sum_{x=1}^{\floor{\sqrt n}} \left( \frac nx - x + O(1) \right) + \floor{\sqrt n}. 
\end{align*}
By Theorem 2.10, we see that 
\[ \sum_{x=1}^n \floor*{\frac nx} = 2n \left( \log\floor{\sqrt n} + \gamma + O\left( \frac1{\sqrt n} \right) \right) - \left( n + O(\sqrt n) \right) + O(\sqrt n). \]
Note that if we use the fact that $\log\floor{\sqrt n} = \log \sqrt n + O(1)$, then the resulting 
error term $O(n)$ will be too large. Therefore, we need a finer estimate. Indeed, since 
$\floor{\sqrt n} = \sqrt n - \{\sqrt n\}$ where $\{t\}$ denotes the fractional part of $t$ for 
$t \in \R$, we have 
\begin{align*} \log \floor{\sqrt n} = \log \left( \sqrt n - \{\sqrt n\} \right) 
&= \log \left( \sqrt n \left( 1 - \frac{\{\sqrt n\}}{\sqrt n} \right) \right) \\
&= \log \sqrt n + \log \left( 1 - \frac{\{\sqrt n\}}{\sqrt n} \right) \\
&= \log \sqrt n + O\left( \frac1{\sqrt n} \right). \end{align*}
Combining this with the previous equality gives 
\[ \sum_{x=1}^n \floor*{\frac nx} = n\log n + (2\gamma - 1)n + O(\sqrt n). \qedhere \]
\end{pf}

\subsection{The Prime Number Theorem}
We now have everything we need to prove the Prime Number Theorem. 

\begin{thm}[Prime Number Theorem] 
    We have 
    \[ \pi(x) \sim \frac{x}{\log x}. \] 
\end{thm}
\begin{pf}
    In Theorem 2.7, we showed that 
    \[ \pi(x) \sim \frac{\psi(x)}{\log x}. \] 
    Therefore, it suffices to show that $\psi(x) \sim x$. Define the function 
    \[ F(x) = \sum_{n\leq x} \left( \psi\left( \frac xn \right) - \floor*{\frac xn}
    + 2\gamma \right), \] 
    where $\gamma$ denotes Euler's constant. By the M\"obius inversion formula 
    (Proposition 2.5), we have 
    \[ \psi(x) - \floor{x} + 2\gamma = \sum_{n\leq x} \mu(n) F\left( \frac xn \right). \] 
    In particular, we get 
    \[ \psi(x) = x + O(1) + \sum_{n \leq x} \mu(n) F\left( \frac xn \right). \] 
    Now, it is enough to show that $\sum_{n\leq x} \mu(n) F(x/n) = o(x)$. First, 
    we will estimate $F(x)$. Observe that 
    \[ F(x) = \sum_{n\leq x} \psi\left( \frac xn \right) - \sum_{n \leq x} 
    \floor*{\frac xn} + 2\gamma\floor{x}. \tag{3.4} \]
    Looking at the first sum in $(3.4)$, we have 
    \begin{align*}
        \sum_{n\leq x} \psi\left( \frac xn \right) 
        &= \sum_{n \leq x} \sum_{m \leq \frac xn} \Lambda(m) \\
        &= \sum_{n \leq x} \Lambda(n) \sum_{m\leq \frac xn} 1 \\
        &= \sum_{n \leq x} \Lambda(n) \floor*{\frac xn} \\
        &= \sum_{p^k \leq x} \log p \floor*{\frac x{p^k}} \\ 
        &= \sum_{p \leq x} \left( \floor*{\frac xp} + \floor*{\frac x{p^2}} 
        + \cdots + \floor*{\frac x{p^k}} \right) \quad \text{(where $p^k\;\|\;\floor{x}$)} \\
        &= \log(\floor{x}!) = \sum_{n \leq x} \log n. 
    \end{align*}
    In the proof of Theorem 2.11, we showed that 
    \[ \sum_{n\leq x} \log n = x\log x - x + O(\log x). \] 
    Hence, we obtain 
    \[ \sum_{n\leq x} \psi\left( \frac xn \right) = x\log x - x + O(\log x). \tag{3.5} \] 
    Moreover, by Theorem 3.9, we have 
    \[ \sum_{n=1}^{\floor{x}} \floor*{\frac{\floor x}n} = \floor{x}\log\floor{x} 
    + (2\gamma - 1)\floor{x} + O(x^{1/2}). \] 
    For all $y \in \R$, notice that $\floor{y} \leq y \leq \floor{y}+1$. In particular, 
    we obtain the inequalities 
    \[ \sum_{n=1}^{\floor{x}} \floor*{\frac{\floor{x}}n} \leq 
    \sum_{n=1}^{\floor{x}} \floor*{\frac{x}n} \leq 
    \sum_{n=1}^{\floor{x}+1} \floor*{\frac{\floor{x}+1}n}, \] 
    and it follows that 
    \[ \sum_{n=1}^{\floor{x}} \floor*{\frac{x}{n}} = x\log x + (2\gamma - 1) 
    + O(x^{1/2}). \tag{3.6} \] 
    Combining equations $(3.4)$, $(3.5)$, and $(3.6)$ gives 
    \[ F(x) = (x\log x - x + O(\log x)) - (x\log x + (2\gamma-1)x + O(x^{1/2})) 
    + (2\gamma x + O(1)) = O(x^{1/2}). \] 
    Therefore, there exists a positive constant $c > 0$ such that 
    \[ |F(x)| \leq cx^{1/2} \] 
    for all $x \geq 1$. If $t > 1$ is an integer, then 
    \begin{align*}
        \left| \sum_{n\leq \frac xt} \mu(n) F\left( \frac xn \right) \right| 
        &\leq \sum_{n\leq \frac xt} \left| F\left( \frac xn \right) \right| \\ 
        &\leq \sum_{n\leq \frac xt} c\left( \frac xn \right)^{\!1/2} \\
        &\leq cx^{1/2} \left( 1 + \int_1^{x/t} \frac{1}{u^{1/2}}\dd u \right) \\
        &= cx^{1/2} \left(1 + 2 \left( \frac xt \right)^{\!1/2} - 2 \right) \\ 
        &\leq 2 \cdot \frac{cx}{t^{1/2}}. \tag{3.7}
    \end{align*}
    Observe that $F$ is a step function. That is, if $a$ is an integer and 
    $a \leq x < a + 1$, then $F(x) = F(a)$. Therefore, we have 
    \[ \sum_{\frac xt < n \leq x} \mu(n) F\left(\frac xn \right) 
    = F(1) \sum_{\frac x2 < n \leq x} \mu(n) + F(2) \sum_{\frac x3 < n \leq 
    \frac x2} \mu(n) + \cdots + F(t-1) \sum_{\frac xt < n \leq \frac{x}{t-1}} \mu(n). \] 
    We see that 
    \begin{align*}
        \left| \sum_{\frac xt < n \leq x} \mu(n) F\left(\frac xn \right) \right| 
        &\leq |F(1)| \left| \sum_{\frac x2 < n \leq x} \mu(n) \right| + 
        |F(2)| \left| \sum_{\frac x3 < n \leq \frac x2} \mu(n) \right| + \cdots + 
        |F(t-1)| \left| \sum_{\frac xt < n \leq \frac{x}{t-1}} \mu(n) \right| \\ 
        &\leq (|F(1)| + \cdots + |F(t-1)|) \max_{2\leq i \leq t} 
        \left| \sum_{\frac xi < n \leq \frac{x}{i-1}} \mu(n) \right| \\ 
        &\leq \left( \sum_{i=1}^t ci^{1/2} \right) \max_{2\leq i \leq t} 
        \left| \sum_{\frac xi < n \leq \frac{x}{i-1}} \mu(n) \right|. 
    \end{align*}
    Notice that 
    \[ \sum_{\frac xi < n \leq \frac{x}{i-1}} \mu(n) = 
    \sum_{n \leq \frac{x}{i-1}} \mu(n) - \sum_{\frac{x}{i} < n} \mu(n) = o(x), \] 
    so we obtain 
    \[ \left| \sum_{\frac xt < n \leq x} \mu(n) F\left( \frac xn \right) \right| 
    = o(t^{3/2}x). \] 
    By Theorem 3.7, we have $\sum_{n\leq x} \mu(n) = o(x)$. Hence, 
    for any $\eps > 0$, we can find sufficiently large $x$ such that 
    \[ -\eps x \leq \sum_{n\leq x} \mu(n) \leq \eps x. \] 
    In particular, when $x$ is sufficiently large, we get 
    \[ -\frac{\eps x}{i-1} - \frac{\eps x}{i} 
    \leq \sum_{\frac{x}{i} < n \leq \frac{x}{i-1}} \mu(n) 
    \leq \frac{\eps x}{i-1} + \frac{\eps x}{i}. \] 
    For any given $\eps > 0$, choose $t = t(\eps)$ such that 
    \[ \frac{2c}{t^{1/2}} < \frac{\eps}2. \] 
    By equation $(3.7)$, we have 
    \[ \left| \sum_{n\leq \frac{x}{t}} \mu(n) F\left( \frac xn \right) \right| 
    \leq 2 \cdot \frac{cx}{t^{1/2}} < \frac{\eps}2 x. \tag{3.8} \] 
    For fixed $\eps > 0$ and $t$ as above, we can choose $x$ sufficiently large 
    so that $o(xt^{3/2}) \leq \eps x/2$. Indeed, we have 
    $2c/t^{1/2} < \eps/2$ if and only if $t > (4c)^2/\eps^2$. In particular, 
    we have $t = A^2\eps^{-2}$ for some $A > 4c$, and we can pick $x$ large 
    enough so that 
    \[ o(x) \leq \frac{\eps^4}{2A^3}x. \] 
    Then we get 
    \[ o(xt^{3/2}) \leq \frac{\eps^4}{2A^3} x \cdot A^3\eps^{-3} = \frac{\eps}2 x. \] 
    It follows that 
    \[ \left| \sum_{\frac xt < n \leq x} \mu(n) F\left( \frac xn \right) \right| < 
    \frac{\eps}2. \tag{3.9} \] 
    Combining inequalities $(3.8)$ and $(3.9)$ yields 
    \[ \left| \sum_{n\leq x} \mu(n) F\left( \frac xn \right) \right| = o(x), \] 
    which completes the proof. 
\end{pf}

\begin{remark}~
    \begin{enumerate}[(1)]
        \item In 1896, Hadamard and de la Vall\'ee Poussin proved the Prime 
        Number Theorem independently. Consider the logarithmic integral 
        \[ \Li(x) = \int_2^x \frac{1}{\log t}\dd t \sim \frac{x}{\log x} 
        \sum_{k=0}^\infty \frac{k!}{(\log x)^k}. \] 
        In 1899, de la Vall\'ee Poussin proved that as $x \to \infty$, there 
        exists some $a > 0$ such that 
        \[ \pi(x) = \Li(x) + O(xe^{-a\sqrt{\log x}}). \]
        \item The main ingredient of our proof of the Prime Number Theorem is the 
        fact that $\sum_{n\leq x} \mu(n) = o(x)$, which is a consequence of the 
        analytic continuation and non-vanishing of $\zeta(s)$ at $\Re(s) = 1$. 
        The {\bf Riemann hypothesis}, proposed by Riemann in 1859, states that 
        the non-trivial zeros of $\zeta(s)$ all have real part $1/2$. 
        (The trivial zeros of $\zeta(s)$ are of the form $2n$ for $n \in \Z$ and 
        $n < 0$; these can be obtained by functional equations.) In 1901, Helge von 
        Koch proved that the Riemann hypothesis is true if and only if 
        \[ \pi(x) = \Li(x) + O(\sqrt x \log x). \] 
    \end{enumerate}
\end{remark}
\newpage 
\section{Divisor Counting Functions}\label{sec:4}

\subsection{Asymptotic Formulas for Divisor Counting Functions}\label{subsec:4.1}

\vspace{2ex}
\begin{defn}{def:4.1}
    For a positive integer $n \in \N$, we denote by $\Omega(n)$ the number 
    of prime factors of $n$ counted with multiplicity, and $\omega(n)$ the 
    number of distinct prime factors of $n$. 
\end{defn}

For example, if $n = 2^{10} \cdot 3^2 \cdot 7$, then $\Omega(n) = 10 + 2 + 1
= 13$ and $\omega(n) = 3$. 

\begin{defn}{def:4.2}
    Let $k \in \N$. For each real number $x \in \R$, we define $\tau_k(x)$ 
    to be the number of positive integer with $n \leq x$ and $\Omega(n) = k$. 
    That is, 
    \[ \tau_k(x) = \#\{n \leq x : \Omega(n) = k\}. \] 
    Furthermore, we let $\pi_k(x)$ be the number of positive integers 
    $n$ with $n \leq x$ and $\omega(n) = \Omega(n) = k$. That is, 
    \[ \pi_k(x) = \#\{n \leq x : \omega(n) = \Omega(n) = k\}. \] 
    In particular, $\pi_k(x)$ counts the positive integers $n$ up to $x$ 
    which are squarefree and have $k$ prime factors. Note that 
    $\pi(x) = \pi_1(x) = \tau_1(x)$. 
\end{defn}

\begin{theo}[Landau, 1900]{thm:4.3} 
    Let $k \in \N$ be a positive integer. Then 
    \[ \pi_k(x) \sim \tau_k(x) \sim \frac{1}{(k-1)!} \frac{x}{\log x} 
    (\log \log x)^{k-1}. \] 
\end{theo}
\begin{pf}
    We first introduce the functions 
    \begin{align*} 
        L_k(x) &= \sum_{p_1\cdots p_k \leq x}{\vphantom{\sum}}^{\hspace{-2.5ex}*} \hspace{1.5ex} \frac{1}{p_1 \cdots p_k}, & 
        \Pi_k(x) &= \sum_{p_1\cdots p_k \leq x}{\vphantom{\sum}}^{\hspace{-2.5ex}*} \hspace{1.5ex} 1, &
        \Theta_k(x) &= \sum_{p_1\cdots p_k \leq x}{\vphantom{\sum}}^{\hspace{-2.5ex}*} \hspace{1.5ex} \log(p_1 \dots p_k), 
    \end{align*}
    where the $*$ means that the sum is taken over all $k$-tuples of 
    primes $(p_1, \dots, p_k)$ with $p_1 \cdots p_k \leq x$. Note that 
    different $k$-tuples can correspond to the same product $p_1 \cdots p_k$. 

    For each positive integer $n \geq 1$, we let $c_n = c_n(k)$ denote the number 
    of $k$-tuples $(p_1, \dots, p_k)$ such that $p_1 \cdots p_k = n$. 
    Observe that 
    \begin{align*}
        \Pi_k(x) &= \sum_{n\leq x} c_n, \\ 
        \Theta_k(x) &= \sum_{n\leq x} c_n \log n.
    \end{align*}
    Moreover, we have 
    \[ c_n = \begin{cases} 
        0 & \text{ if $n$ is not a product of $k$ primes,} \\
        k! & \text{ if $n$ is squarefree and $\omega(n) = \Omega(n) = k$.} 
    \end{cases} \] 
    We also see that $0 < c_n < k!$ if $\Omega(n) = k$ but $n$ is not squarefree. 
    Therefore, we obtain the inequalities 
    \[ k!\pi_k(x) \leq \Pi_k(x) \leq k!\tau_k(x). \tag{4.1}\label{eq:4.1} \] 
    For $k \geq 2$, note that the number of positive integers up to $x$ with 
    $k$ prime factors and divisible by the square of some prime is 
    $\tau_k(x) - \pi_k(x)$. Therefore, we have 
    \[ \tau_k(x) - \pi_k(x) = 
    \sum_{\substack{p_1\cdots p_k \leq x \\ p_i = p_j \text{ for some } i \neq j}}
    {\vphantom{\sum}}^{\hspace{-6.5ex}*} \hspace{5.5ex} 1 
    \leq \binom{k}{2} \sum_{p_1\cdots p_k \leq x}
    {\vphantom{\sum}}^{\hspace{-2.5ex}*} \hspace{1.5ex} 1 
    = \binom{k}{2} \Pi_{k-1}(x). \]
    {\sc Claim.} We have 
    \[ \Pi_k(x) \sim k \frac{x(\log \log x)^{k-1}}{\log x}. \] 
    {\sc Proof of Claim.} Applying Abel's summation formula with $a_n = c_n$ 
    and $f(u) = \log u$, we have 
    \[ \Theta_k(x) = \sum_{n \leq x} c_n \log n = \Pi_k(x) \log x 
    - \int_1^x \frac{\Pi_k(u)}{u}\dd u. \] 
    Observe that 
    \[ \Pi_k(x) \leq k!\tau_k(x) \leq k!x, \] 
    so $\Pi_k(u) = O(u)$, and hence 
    \[ \Theta_k(x) = \Pi_k(x) \log x + O(x). \] 
    Thus, it suffices to show that for all $k \in \N$, we have 
    \[ \Theta_k(x) \sim kx(\log \log x)^{k-1}. \tag{4.2}\label{eq:4.2} \] 
    We'll proceed by induction on $k$. This will be somewhat similar to the 
    proof of the Prime Number Theorem, but with the weighting function 
    $\log(p_1 \cdots p_k)$ on the $k$-tuple $(p_1, \dots, p_k)$. 

    For $k = 1$, we have $\Theta_1(x) = \theta(x) \sim x$ by Theorem~\ref{thm:2.7}
    and the Prime Number Theorem. Assume now that $\Theta_k(x) 
    \sim kx(\log \log x)^{k-1}$ for some $k \geq 1$. We'll prove the result for 
    $\Theta_{k+1}(x)$. First, note that 
    \[ \left( \sum_{p \leq x^{1/k}} \frac{1}{p} \right)^{\!k} 
    \leq L_k(x) \leq \left( \sum_{p \leq x} \frac{1}{p}\right)^{\!k} \] 
    for all $k \geq 1$. By Theorem~\ref{thm:2.13}, we have 
    \begin{align*} 
        \left( \sum_{p \leq x^{1/k}} \frac{1}{p} \right)^{\!k} &\sim 
        \left(\log \log (x^{1/k})\right)^k, \\ 
        \left( \sum_{p \leq x} \frac{1}{p}\right)^{\!k} &\sim 
        (\log \log x)^k. 
    \end{align*}
    Notice that 
    \[ \left( \log\log(x^{1/k}) \right)^{\!k} = 
    (\log\log x - \log k)^k \sim (\log\log x)^{k}, \] 
    so $L_k \sim (\log\log x)^k$. Therefore, we have 
    \[ \Theta_{k+1}(x) - (k+1)(\log\log x)^k = \Theta_{k+1}(x) 
    - (k+1)xL_k(x) + o\left( x(\log\log x)^k \right). \] 
    Note that 
    \begin{align*} 
        k\Theta_{k+1}(x) &= \sum_{p_1\cdots p_{k+1} \leq x}
        {\vphantom{\sum}}^{\hspace{-3.5ex}*} \hspace{2.5ex}
        k \cdot \log(p_1 \cdots p_{k+1}) \\ 
        &= \sum_{p_1\cdots p_{k+1} \leq x}
        {\vphantom{\sum}}^{\hspace{-3.5ex}*} \hspace{2.5ex}
        \left( \log(p_2 \cdots p_{k+1}) + \log(p_1p_3 \cdots p_{k+1}) 
        + \cdots + \log(p_1 \cdots p_k) \right) \\ 
        &= (k+1) \sum_{p_1 \leq x} \sum_{p_2\cdots p_{k+1} \leq x/p_1}
        {\vphantom{\sum}}^{\hspace{-5ex}*} \hspace{4ex} 
        \log(p_2 \cdots p_{k+1}) \\ 
        &= (k+1) \sum_{p_1 \leq x} \Theta_k \left( \frac{x}{p_1} \right). 
    \end{align*}
    Since $L_0(x) = 1$ and 
    \[ L_k(x) = \sum_{p_1\cdots p_k \leq x}
    {\vphantom{\sum}}^{\hspace{-2.5ex}*} \hspace{1.5ex} 
    \frac{1}{p_1 \cdots p_k} = \sum_{p_1 \leq x} \frac{1}{p_1} L_{k-1} 
    \left( \frac{x}{p_1} \right), \] 
    it follows that 
    \begin{align*}
        \Theta_{k+1}(x) - (k+1)xL_k(x) 
        &= (k+1) \sum_{p_1 \leq x} \left( \frac{1}{k} \Theta_k \left( \frac{x}{p_1} \right) 
        - \frac{x}{p_1} L_{k-1} \left( \frac{x}{p_1} \right) \right) \\ 
        &= \frac{k+1}{k} \sum_{p_1 \leq x} \left( \Theta_k \left( \frac{x}{p_1} \right) 
        - k \frac{x}{p_1} L_{k-1} \left( \frac{x}{p_1} \right) \right). 
    \end{align*}
    By the induction hypothesis, we have 
    \[ \Theta_k(y) - kyL_{k-1}(y) = o(y(\log \log y)^{k-1}). \] 
    Given $\eps > 0$, there exists $x_0 = x_0(\eps, k)$ such that for all 
    $y > x_0$, we have 
    \[ |\Theta_k(y) - kyL_{k-1}(y)| \leq \eps y(\log \log y)^{k-1}. \] 
    Furthermore, there exists a positive constant $c = c(\eps, k) > 0$ such 
    that for all $y \leq x_0$, we have 
    \[ |\Theta_k(y) - kyL_{k-1}(y)| \leq c. \] 
    Note that $x/p_1 > x_0$ implies that $p_1 < x/x_0$, so for sufficiently 
    large $x$, we obtain 
    \begin{align*} 
        |\Theta_{k+1}(x) - (k+1)xL_k(x)| 
        & \leq \frac{k+1}{k} \left( \sum_{\frac{x}{x_0} < p_1 \leq x} c 
        + \sum_{p_1 \leq \frac{x}{x_0}} \eps \frac{x}{p_1} \left( 
        \log\log \frac{x}{p_1} \right)^{\!k-1} \right) \\
        & \leq 2cx + 2\eps x(\log \log x)^{k-1} \sum_{p_1 \leq \frac{x}{x_0}} 
        \frac{1}{p_1} \\ 
        & \leq 2cx + 4\eps x (\log \log x)^k < 5\eps x(\log \log x)^k, 
    \end{align*}
    where the second last inequality comes from choosing $x$ large enough 
    so that 
    \[ \sum_{p\leq x} \frac{1}{p} \leq 2\log\log x. \] 
    Therefore, we see that 
    \[ \Theta_{k+1}(x) - (k+1)xL_k(x) = o\left( x(\log\log x)^k \right). \] 
    We conclude that 
    \[ \Theta_{k+1}(x) \sim (k+1) x(\log\log x)^k, \] 
    which proves the claim. \hfill$\blacksquare$

    From equation $\eqref{eq:4.1}$ and the claim, we have 
    \[ \pi_k(x) \leq \frac{1}{k!} \Pi_k(x) \sim 
    \frac{1}{(k-1)!} \frac{x}{\log x} (\log\log x)^{k-1}. \] 
    Moreover, combining equations $\eqref{eq:4.1}$ and 
    $\eqref{eq:4.2}$ with the claim yields 
    \[ \pi_k(x) = \tau_k(x) + O(\Pi_{k-1}(x)) 
    \geq \frac{1}{k!} \Pi_k(x) + O(\Pi_{k-1}(x)) 
    \sim \frac{1}{(k-1)!} \frac{x}{\log x} (\log \log x)^{k-1}. \] 
    In particular, we get 
    \[ \pi_k(x) \sim \tau_k(x) \sim \frac{1}{(k-1)!} \frac{x}{\log x} 
    (\log\log x)^{k-1}, \] 
    which finishes the proof of the theorem. 
\end{pf}

\subsection{Summatory Functions for $\omega(n)$ and $\Omega(n)$}\label{subsec:4.2} 
Let's now consider the averages of $\omega(n)$ and $\Omega(n)$. 

\begin{theo}{thm:4.4}
    We have 
    \begin{align*}
        \sum_{n\leq x} \omega(n) &= x\log\log x + \beta x + o(x), \\ 
        \sum_{n\leq x} \Omega(n) &= x\log\log x + \tilde\beta x + o(x), 
    \end{align*}
    where $\beta$ is Merten's constant as in Theorem~\ref{thm:2.13} and 
    \[ \tilde\beta = \beta + \sum_p \frac{1}{p(p-1)}. \] 
\end{theo}
\begin{pf}
    Set $S_1 = S_1(x) = \sum_{n\leq x} \omega(n)$. Then we have 
    \[ S_1 = \sum_{n\leq x} \sum_{p\mid n} 1 = \sum_{p\leq x} \floor*{\frac{x}{p}}. \] 
    By Theorem~\ref{thm:2.13}, we obtain 
    \begin{align*} 
        S_1 &= \sum_{p\leq x} \floor*{\frac{x}{p}} \\
        &= x \sum_{p\leq x} \frac{1}{p} + O(\pi(x)) \\
        &= x(\log\log x + \beta + o(1)) + O(\pi(x)) \\
        &= x\log\log x + x\beta + o(x), 
    \end{align*}
    where the last equality follows from the Prime Number Theorem. 

    On the other hand, if we set $S_2 = S_2(x) = \sum_{n\leq x} \Omega(n)$, then 
    \[ S_2 - S_1 = \sum_{p^m \leq x,\, m \geq 2} \floor*{\frac{x}{p^m}} 
    = \sum_{p^m \leq x,\, m\geq 2} \frac{x}{p^m} + 
    O \left( \sum_{p^m \leq x,\, m \geq 2} 1 \right). \] 
    Note that $2^m \leq p^m \leq x$, so $m \leq \frac{\log x}{\log 2}$. 
    Moreover, $p^2 \leq p^m \leq x$ implies that $p \leq x^{1/2}$. Therefore, 
    we have 
    \[ S_2 - S_1 = \sum_{p^m \leq x,\, m\geq 2} \frac{x}{p^m} + 
    O(x^{1/2} \log x) = x \left( \sum_p \left( \frac{1}{p^2} + 
    \frac{1}{p^3} + \cdots \right) - \sum_{p^m \geq x} \frac{1}{p^m} \right) 
    + O(x^{1/2} \log x). \] 
    Observe that 
    \begin{align*}
        \sum_{\substack{p^m > x \\ m \geq x}} \frac{1}{p^m} 
        &\leq \sum_{\substack{p^m > x \\ m \geq 2 \\ 2\,\mid\,n}} \frac{1}{p^m} 
        + \sum_{\substack{p^m > x \\ m \geq 2 \\ 2\,\nmid\,n}} \frac{1}{p^m} 
        \leq \sum_{n^2 > x} \frac{1}{n^2} + \sum_{\substack{p^m > x \\
        m \geq 2 \\ 2\,\mid\,m \\ p \leq \sqrt{x}}} \frac{1}{p^m} 
        + \sum_{\substack{p^m > x \\ 
        m \geq 2 \\ 2\,\mid\,m \\ p > \sqrt{x}}} \frac{1}{p^m}. 
    \end{align*}
    Notice that if $p \leq \sqrt{x}$, then since $p^m > x$, we get 
    $p^{m-1} > x/p > \sqrt{x}$. On the other hand, if $p > \sqrt{x}$, then 
    $p^{m-1} > \sqrt{x}$. Hence, we get 
    \begin{align*} 
        \sum_{\substack{p^m > x \\ m \geq 2}} \frac{1}{p^m} 
        &\leq \sum_{n^2 > x} \frac{1}{n^2} + 2 \sum_{\substack{p^{m-1} > \sqrt{x}
        \\ m \geq 2 \\ 2\,\mid\,m}} \frac{1}{p^{m-1}} 
        \leq \sum_{n^2 > x} \frac{1}{n^2} + 2 \sum_{m^2 > \sqrt{x}} \frac{1}{m^2}  
        \leq 3 \sum_{k > \sqrt[4]{x}} \frac{1}{k^2} = O\left( \frac{1}{\sqrt[4]{x}} \right). 
    \end{align*}
    Therefore, we have 
    \[ S_2 - S_1 = x \left( \sum_p \frac{1}{p(p-1)} + o(1) \right) 
    + O(x^{1/2} \log x) = x \sum_p \frac{1}{p(p-1)} + o(x). \] 
    Together with our estimate of $S_1$, we see that 
    \[ S_2 = x\log\log x + x \left( \beta + \sum_p \frac{1}{p(p-1)} \right) 
    + o(x). \qedhere \] 
\end{pf}

\subsection{Asymptotic Density and Normal Order}\label{subsec:4.3} 

\vspace{2ex}
\begin{defn}{def:4.5}
    Let $A$ be a subset of $\N$. For any $n \in \N$, we set $A(n) 
    = \{1, \dots, n\} \cap A$. We define the {\bf upper asymptotic density}
    of $A$ by 
    \[ \bar{d}(A) := \limsup_{n\to\infty} \frac{|A(n)|}{n}. \] 
    Similarly, we define the {\bf lower asymptotic density} of $A$ to be 
    \[ \underline{d}(A) := \liminf_{n\to\infty} \frac{|A(n)|}{n}. \] 
    We say that $A$ has {\bf asymptotic density} $d(A)$ when 
    $\bar{d}(A) = \underline{d}(A)$, in which case we set $d(A)$ to be 
    this common value. 
\end{defn}

Now, let's look at some simple examples of asymptotic density of subsets 
$A \subseteq \N$. 

\begin{exmp}{exmp:4.6}
    \begin{enumerate}[(1)]
        \item When $A$ is the set of all primes, we have 
        $d(A) = \bar{d}(A) = \underline{d}(A) = 0$. 
        \item For $A = \{n \in \N : n \equiv 0 \pmod 5\}$, we have 
        $d(A) = \bar{d}(A) = \underline{d}(A) = 1/5$.
        \item For $A = \{n \in \N : n \neq k^2 + 1 \text{ for any } k \in \Z\}$, 
        we have $d(A) = \bar{d}(A) = \underline{d}(A) = 1$.
        \item Let $A = \{a \in \N : (2k)! < a < (2k+1)! \text{ for some } k \in \Z\}$. 
        Notice that for $n = (2k+1)!$, any $a \in \N$ satisfying $(2k)! < a < 
        (2k+1)!$ is included in $A(n)$. Therefore, we have 
        \[ 1 \geq \frac{|A((2k+1)!)|}{(2k+1)!} \geq \frac{(2k+1)! - (2k)!}{(2k+1)!} 
        = \frac{2k}{2k+1}. \] 
        By taking $k \to \infty$, we see that 
        \[ \frac{|A((2k+1)!)|}{(2k+1)!} \to 1, \] 
        and hence $\bar{d}(A) = 1$. On the other hand, when $n = (2k)!$, then only 
        $a \in \N$ such that $a < (2k-1)!$ are included in $A(n)$. Thus, we have 
        \[ 0 \leq \frac{|A((2k)!)|}{(2k)!} \leq \frac{(2k-1)!}{(2k)!} = \frac{1}{2k}. \] 
        As $k \to \infty$, we have 
        \[ \frac{|A((2k)!)|}{(2k)!} \to 0, \] 
        and hence $\underline{d}(A) = 0$.
    \end{enumerate} 
\end{exmp}

From asymptotic density, we can define normal order. 
Moreover, we will define average order. 

\begin{defn}{def:4.7}
    Let $f(n)$ and $F(n)$ be functions from $\N$ to $\R$. 
    \begin{itemize}
        \item We say that $f(n)$ has {\bf normal order} $F(n)$ if for every 
        $\eps > 0$, the set 
        \[ A(\eps) = \{n \in \N : (1-\eps)F(n) < f(n) < (1+\eps)F(n)\} \] 
        has the property that $d(A(\eps)) = 1$. Equivalently, if $B(\eps) 
        = \N \setminus A(\eps)$, then $d(B(\eps)) = 0$. 
        \item We say that $f(n)$ has {\bf average order} $F(n)$ if 
        \[ \sum_{j=1}^n f(j) \sim \sum_{j=1}^n F(j). \] 
    \end{itemize}
\end{defn}

These definitions seem rather abstract, so let's look at some examples of normal 
and average order. It's not too difficult to check the details. 

\begin{exmp}{exmp:4.8}
    \begin{enumerate}[(1)]
        \item If we define 
        \[ f(n) = \begin{cases} 1 & \text{if $n \neq k!$ for any $k \in \N$,} \\ 
            n & \text{if $n = k!$ for some $k \in \N$,} \end{cases} \] 
        then $f$ has normal order $1$ but not average order $1$. 
        \item If we define 
        \[ f(n) = \begin{cases} 2 & \text{if $n \equiv 1 \pmod 2$,} \\ 
            0 & \text{if $n \equiv 0 \pmod 2$,} \end{cases} \] 
        then $f$ has average order $1$ but not normal order $1$. 
        \item If we define 
        \[ f(n) = \begin{cases}
            \log n + (\log n)^{1/2} & \text{if $n \equiv 1 \pmod 2$,} \\ 
            \log n - (\log n)^{1/2} & \text{if $n \equiv 0 \pmod 2$,} \\ 
        \end{cases} \] 
        then $f$ has both normal and average order $\log n$.
    \end{enumerate}
\end{exmp}

\begin{theo}{thm:4.9}
    Both $\omega(n)$ and $\Omega(n)$ have average order $\log\log n$. 
\end{theo}
\begin{pf}
    First, note that 
    \begin{align*} 
        \sum_{n\leq x} \log\log n 
        &= \sum_{x^{1/2}<n\leq x} \log\log n + \sum_{n\leq x^{1/2}} \log\log n \\
        &= \sum_{x^{1/2}<n\leq x} \log\log n + O(x^{1/2}\log \log x). 
    \end{align*}
    Moreover, we have 
    \[ \sum_{x^{1/2} < n \leq x} \log\log n \leq \log\log x  
    \sum_{x^{1/2} < n \leq x} 1 = x\log\log x + O(x^{1/2}\log\log x). \] 
    Also, we have the lower bound 
    \[ \sum_{x^{1/2} < n \leq x} \log\log n \geq (\log\log x - \log 2) 
    \sum_{x^{1/2} < n \leq x} 1 = x\log\log x + O(x^{1/2}\log\log x). \] 
    It follows that 
    \[ \sum_{n\leq x} \log\log n = x\log\log x + O(x^{1/2}\log\log x). \] 
    Combining this estimate with Theorem~\ref{thm:4.4} shows that 
    $\omega(n)$ and $\Omega(n)$ both have average order $\log\log n$. 
\end{pf}

\subsection{Normal Order of $\omega(n)$ and $\Omega(n)$}\label{subsec:4.4}
We have shown that $\omega(n)$ and $\Omega(n)$ have average order $\log\log n$. 
In this section, we'll work towards proving that they have normal order $\log\log n$. 

\begin{theo}{thm:4.10}
    Let $\delta > 0$. The number of positive integers $n \leq x$ satisfying 
    \[ |f(n) - \log\log n| > (\log\log n)^{\frac12+\delta} \] 
    is $o(x)$, where $f(n) = \omega(n)$ or $f(n) = \Omega(n)$. In particular, 
    both $\omega(n)$ and $\Omega(n)$ have normal order $\log\log n$. 
\end{theo}
\begin{pf}
    It is enough to prove that the number of positive integers $n \leq x$ with 
    \[ |f(n) - \log\log x| > (\log\log x)^{\frac12+\delta} \] 
    is $o(x)$, because for $x^{1/e} \leq n \leq x$, we have 
    \[ \log\log x \geq \log\log n \geq \log\left(\frac{\log x}e\right) = 
    \log\log x - 1. \] 
    In other words, we can replace $\log\log n$ in the statement of the theorem 
    with $\log\log x$. 

    Moreover, we can restrict our attention to the case where $f(n) = \omega(n)$, 
    because by Theorem~\ref{thm:4.4}, we have 
    \[ \sum_{n\leq x} (\Omega(n) - \omega(n)) = O(x). \] 
    Thus, the number of integers $n \leq x$ for which $\Omega(n) - \omega(n) 
    > (\log\log n)^{1/2}$ is $o(x)$. 

    {\sc Claim.} We have 
    \begin{align*}
        \sum_{n\leq x} \omega(n)^2 = x(\log\log x)^2 + O(x\log\log x), \\ 
        \sum_{n\leq x} (\omega(n) - \log\log x)^2 = O(x\log\log x). 
    \end{align*}

    {\sc Proof of Claim.} For each $n \leq x$, consider the ordered pairs 
    $(p, q)$ where $p$ and $q$ are distinct prime factors of $n$. There are 
    $\omega(n)$ choices for $p$ and $\omega(n) - 1$ choices for $q$, which gives 
    \[ \omega(n)(\omega(n) - 1) = \sum_{\substack{pq\,\mid\,n \\ p \neq q}} 1 
    = \sum_{pq\,\mid\,n} 1 - \sum_{p^2\,\mid\,n} 1. \] 
    Therefore, we have 
    \begin{align*} 
        \sum_{n\leq x} \omega(n)^2 - \sum_{n\leq x} \omega(n) 
        &= \sum_{n\leq x} \omega(n)(\omega(n) - 1) \\ 
        &= \sum_{n\leq x} \left( \sum_{pq\,\mid\,n} 1 - \sum_{p^2\,\mid\,n} 1 \right) \\
        &= \sum_{pq \leq x} \floor*{\frac{x}{pq}} - \sum_{p^2 \leq x} \floor*{\frac{x}{p^2}}.
    \end{align*}
    Observe that 
    \begin{align*} 
        \sum_{p^2 \leq x} \floor*{\frac{x}{p^2}} 
        \leq x \sum_{p^2 \leq x} \frac{1}{p^2} = O(x), \\ 
        \sum_{pq \leq x} \floor*{\frac{x}{pq}} 
        = \sum_{pq \leq x} \frac{x}{pq} + O(x), 
    \end{align*}
    which implies that 
    \[ \sum_{n\leq x} \omega(n)^2 - \sum_{n\leq x} \omega(n) 
    = \sum_{pq\leq x} \frac{x}{pq} + O(x). \tag{4.3}\label{eq:4.3} \] 
    Next, note that 
    \[ \left( \sum_{p\leq x^{1/2}} \frac1p \right)^{\!2} 
    - \left( \sum_{p\leq x} \frac1{p^2} \right) 
    \leq \sum_{pq\leq x} \frac{1}{pq} 
    \leq \left( \sum_{p\leq x} \frac1p \right)^{\!2}. \] 
    Furthermore, Merten's theorem (Theorem~\ref{thm:2.13}) tells us that 
    \[ \left( \sum_{p\leq x} \frac1p \right)^{\!2} = (\log\log x)^2 
    + O(\log\log x), \] 
    so it follows that 
    \[ \left( \sum_{p\leq x^{1/2}} \frac1p \right)^{\!2} 
    = \left(\log\log x^{1/2} + O(1)\right)^2 
    = (\log\log x - \log 2 + O(1))^2 
    = (\log\log x)^2 + O(\log\log x). \] 
    Thus, we obtain 
    \[ \sum_{pq\leq x} \frac{1}{pq} = (\log\log x)^2 + O(\log\log x). 
    \tag{4.4}\label{eq:4.4} \] 
    By Theorem~\ref{thm:4.4}, we get 
    \[ \sum_{n\leq x} \omega(n) = O(x\log\log x). \tag{4.5}\label{eq:4.5} \] 
    Combining equations $\eqref{eq:4.3}$, $\eqref{eq:4.4}$, and $\eqref{eq:4.5}$ 
    together yields 
    \[ \sum_{n\leq x} \omega(n)^2 = x(\log\log x)^2 + O(x\log\log x), \] 
    which proves the first equality. Now, we have 
    \begin{align*}
        \sum_{n\leq x} (\omega(n) - \log\log x)^2 
        &= \sum_{n\leq x} \omega(n)^2 - 2\sum_{n\leq x} \omega(n)\log\log x 
        + \sum_{n\leq x} (\log\log x)^2 \\ 
        &= x(\log\log x)^2 + O(x\log\log x) - 2\log\log x \sum_{n\leq x} \omega(n)
        + \floor{x} (\log\log x)^2 \\ 
        &= x(\log\log x)^2 + O(x\log\log x) - 2x(\log\log x)^2 + O(\log\log x) \\
        &\qquad + x(\log\log x)^2 + O((\log\log x)^2) \\ 
        &= O(x\log\log x), 
    \end{align*}
    where the second last equality follows from Theorem~\ref{thm:4.4}. This 
    finishes the proof of the claim. \hfill $\blacksquare$ 

    Finally, as we stated in the beginning of the proof, it suffices to show that 
    \[ E(x) := \#\{n \leq x : |\omega(n) - \log\log x| > (\log\log x)^{\frac12+\delta}\} \] 
    is $o(x)$. By the claim, we have 
    \[ E(x) \cdot (\log\log x)^{1+2\delta} 
    \leq \sum_{n\leq x} (\omega(n) - \log\log x)^2 = O(x\log\log x). \] 
    It follows that 
    \[ E(x) = O\left( \frac{x\log\log x}{(\log\log x)^{1+2\delta}} 
    \right) = o(x). \qedhere \] 
\end{pf}

\begin{remark}{remark:4.11}
    Since the average order of $\omega(n)$ is $\log\log n$, which is asymptotic 
    to $\log\log x$ for ``almost all'' $n$ (namely, all except $o(x)$ 
    many $n \leq x$), we can view the sum 
    \[ \frac1x \sum_{n\leq x} (\omega(n) - \log\log x)^2 \] 
    as the variance of $\omega(n)$; that is, the squares of the standard deviation. 
    In Homework 3, we will show that 
    \[ \sum_{n\leq x} (\omega(n) - \log\log x)^2 \sim x\log\log x, \] 
    which implies that the standard deviation of $\omega(n)$ is about 
    $\sqrt{\log\log n}$. Now, consider the term 
    \[ \frac{\omega(n) - \log\log n}{\sqrt{\log\log n}}. \] 
    In 1934, Erd\H{o}s and Kac proved (without knowing probability theory) that 
    \[ \lim_{x\to\infty} \frac1x \#\left\{ n\leq x : \frac{\omega(n) - 
    \log\log n}{\sqrt{\log\log n}} \leq \gamma \right\} = G(\gamma), \] 
    where we define 
    \[ G(\gamma) := \frac{1}{\sqrt{2\pi}} \int_{-\infty}^{\gamma} e^{-t^2/2}\dd t \] 
    to be the Gaussian normal distribution. This result forms a foundation 
    of probabilistic number theory. 
\end{remark}

Recall that for all $n \in \N$, the divisor function $d(n)$ gives the number of 
positive divisors of $n$. In particular, if we have $n = p_1^{a_1} \cdots p_r^{a_r}$
where $a_1, \dots, a_r \in \N$ and $p_1, \dots, p_r$ are distinct primes, then
\begin{align*} 
    \omega(n) &= r, \\ 
    \Omega(n) &= a_1 + \cdots + a_r, \\ 
    d(n) &= (a_1 + 1) \cdots (a_r + 1). 
\end{align*}

\begin{theo}{thm:4.12} 
    Given $\eps > 0$, define the set 
    \[ S(\eps) = \{n \in \N : 2^{(1-\eps)\log\log n} < d(n) < 2^{(1+\eps)\log\log n} \}. \] 
    Then $S(\eps)$ has asymptotic density $1$. 
\end{theo}
\begin{pf}
    Note that for any $a \in \N$, we have 
    \[ 2 \leq a + 1 \leq 2^a. \] 
    In particular, we get 
    \[ 2^{\omega(n)} \leq d(n) \leq 2^{\Omega(n)}, \] 
    and the result follows from Theorem~\ref{thm:4.10}. 
\end{pf}

\begin{remark}{remark:4.13}
    We saw in Theorem~\ref{thm:3.9} that 
    \[ \sum_{n\leq x} d(n) \sim x\log x \sim \sum_{n\leq x}\log n. \] 
    Therefore, the average order of $d(n)$ is $\log n$. However, using 
    Theorem~\ref{thm:4.12}, one can show that for almost all $n \in \N$, the 
    divisor function $d(n)$ satisfies 
    \[ (\log n)^{\log 2 - \eps} < d(n) < (\log n)^{\log 2 + \eps} \] 
    for any $\eps > 0$. 
\end{remark}\newpage 
\section{Quadratic Reciprocity}\label{sec:5}

\subsection{Euler's Totient Function}\label{subsec:5.1}

\begin{defn}\label{def:5.1}
    For $n \in \N$, we define {\bf Euler's totient function} $\phi(n)$ to be 
    the number of integers $m$ such that $1 \leq m \leq n$ and $\gcd(m, n) = 1$. 
    That is, we have 
    \[ \phi(n) = \#\{1 \leq m \leq n : \gcd(m, n) = 1\}. \] 
    A {\bf reduced residue system modulo $n$} is a subset $R \subseteq \Z$ such that 
    \begin{enumerate}[(i)]
        \item $\gcd(r, n) = 1$ for each $r \in R$; 
        \item $R$ contains $\phi(n)$ elements; and 
        \item no two elements of $R$ are congruent modulo $n$. 
    \end{enumerate}
\end{defn}

\begin{thm}\label{thm:5.2}
    Let $a, n \in \N$ with $\gcd(a, n) = 1$. Then 
    \[ a^{\phi(n)} \equiv 1 \pmod n. \] 
\end{thm}
\begin{pf}
    Let $\{c_1, \dots, c_{\phi(n)}\}$ be a reduced residue system modulo $n$. 
    Since $\gcd(a, n) = 1$, $\{ac_1, \dots, ac_{\phi(n)}\}$ is also a 
    reduced residue system modulo $n$. Hence, we have 
    \[ c_1 \cdots c_{\phi(n)} \equiv ac_1 \cdots ac_{\phi(n)} \pmod n. \] 
    In particular, we see that 
    \[ c_1 \cdots c_{\phi(n)} \equiv a^{\phi(n)} c_1 \cdots c_{\phi(n)} \pmod n. \] 
    Since $c_1, \dots, c_{\phi(n)}$ are all coprime with $n$, it follows that 
    \[ a^{\phi(n)} \equiv 1 \pmod n. \qedhere \] 
\end{pf}

Notice that when $p$ is prime, we have $\phi(p) = p-1$, so we immediately 
obtain the following corollary. 

\begin{cor}[Fermat's little theorem]\label{cor:5.3}
    Let $p$ be a prime. For any $a \in \Z$ with $p \nmid a$, we have 
    \[ a^{p-1} \equiv 1 \pmod p. \] 
\end{cor}

\begin{thm}[Wilson]\label{thm:5.4}
    Consider the element $x^{p-1} - 1 \in (\Z/p\Z)[x]$. By Fermat's little theorem 
    and using the fact that $\Z/p\Z$ is a field, this factors as 
    \[ x^{p-1} - 1 \equiv (x-1)(x-2) \cdots (x-(p-1)) \pmod p \] 
    in $(\Z/p\Z)[x]$, as $1, 2, \dots, p-1$ are all roots. Looking at the constant 
    coefficient, we find that 
    \[ -1 \equiv (-1)(-2) \cdots (-(p-1)) \pmod p. \] 
    Therefore, we have $-1 \equiv (-1)^{p-1} (p-1)! \pmod p$. When $p = 2$, 
    the result holds since $-1 \equiv 1 \pmod 2$; otherwise, $p$ is odd, 
    so $-1 \equiv (p-1)! \pmod p$ as required. 
\end{thm}\newpage 
\section{Primitive Roots}\label{sec:6}

\subsection{Cyclicity of $(\Z/p\Z)^*$}\label{subsec:6.1}
Recall that for $a, b \in \Z$, we can find $x, y \in \Z$ such that $ax + by = 
\gcd(a, b)$ using the Euclidean algorithm. 

\begin{theo}[Chinese remainder theorem]{thm:6.1}
    Let $m_1, \dots, m_t \in \N$ with $\gcd(m_i, m_j) = 1$ whenever $i \neq j$, 
    and set $m = m_1 \cdots m_t$. Let $b_1, \dots, b_t \in \Z$. Then the simultaneous 
    congruences 
    \begin{align*}
        x &\equiv b_1 \pmod{m_1}, \\ 
        x &\equiv b_2 \pmod{m_2}, \\ 
          &\qquad\quad\, \vdots \\ 
        x &\equiv b_t \pmod{m_t}
    \end{align*}
    has a unique solution modulo $m$. 
\end{theo}

\begin{theo}{thm:6.2}
    Let $m_1, \dots, m_t \in \N$ with $\gcd(m_i, m_j) = 1$ whenever $i \neq j$, 
    and set $m = m_1 \cdots m_t$. Then we have the ring isomorphism 
    \[ \Z/m\Z \cong \Z/m_1\Z \times \cdots \times \Z/m_t\Z, \] 
    as well as the group isomorphism
    \[ (\Z/m\Z)^* \cong (\Z/m_1\Z)^* \times \cdots \times (\Z/m_t\Z)^*. \] 
\end{theo}
\begin{pf}
    Let $\psi : \Z \to \Z/m_1\Z \times \cdots \times \Z/m_t\Z$ be defined by 
    \[ \psi(n) = (n+m_1\Z, \dots, n+m_t\Z). \] 
    It is readily checked that $\psi$ is a ring homomorphism. By the Chinese 
    remainder theorem, $\psi$ is surjective and $\ker\psi = m\Z$. It follows 
    from the first isomorphism theorem that 
    \[ \Z/m\Z \cong \Z/m_1\Z \times \cdots \times \Z/m_t\Z. \] 
    On the other hand, if we define $\lambda : (\Z/m\Z)^* \to
    (\Z/m_1\Z)^* \times \cdots \times (\Z/m_t\Z)^*$ by 
    \[ \lambda(n + m\Z) = (n+m_1\Z, \dots, n+m_t\Z), \] 
    then $\lambda$ is a group homomorphism, and it is bijective by the Chinese 
    remainder theorem. 
\end{pf}

\begin{cor}{cor:6.3}
    Let $m_1, \dots, m_t$ be pairwise coprime positive integers. Set 
    $m = m_1 \cdots m_t$. Then 
    \[ \phi(m) = \phi(m_1) \cdots \phi(m_t). \] 
\end{cor}
\begin{pf}
    Recall that $\phi(m) = |(\Z/m\Z)^*|$, and 
    \[ \phi(m_1) \cdots \phi(m_t) = |(\Z/m_1\Z)^*| \cdots |(\Z/m_t\Z)^*| 
    = |(\Z/m_1\Z)^* \times \cdots \times (\Z/m_t\Z)^*|. \] 
    The result follows from Theorem~\ref{thm:6.2}. 
\end{pf}

\begin{cor}{cor:6.4}
    Let $m = p_1^{a_1} \cdots p_t^{a_t}$, where $p_1, \dots, p_t$ are distinct 
    primes and $a_1, \dots, a_t$ are positive integers. Then 
    \[ \phi(m) = m \cdot \prod_{i=1}^t \left( 1 - \frac{1}{p_i} \right). \] 
\end{cor}
\begin{pf}
    Select $m_i = p_i^{a_i}$ for $i = 1, \dots, t$ in Corollary~\ref{cor:6.3}. 
    Observe that 
    \[ \phi(p_i^{a_i}) = p_i^{a_i} - p_i^{a_i-1} = p_i^{a_i} \left( 1 - 
    \frac{1}{p_i} \right). \] 
    It follows that 
    \[ \phi(m) = \phi(p_1^{a_1}) \cdots \phi(p_t^{a_t}) = 
    p_1^{a_1} \cdots p_t^{a_t} \left( 1 - \frac{1}{p_1} \right) 
    \cdots \left( 1 - \frac{1}{p_t} \right) = m \cdot \prod_{i=1}^t 
    \left( 1 - \frac{1}{p_i} \right). \qedhere \] 
\end{pf}

\begin{prop}{prop:6.5} 
    Let $p$ be prime. If $d \mid (p-1)$, then $x^d \equiv 1 \pmod p$ has exactly 
    $d$ solutions modulo $p$. 
\end{prop}
\begin{pf}
    Write $p-1 = dk$ for some integer $k$. Then we have 
    \[ \frac{x^{p-1}-1}{x^d-1} = \frac{(x^d)^k-1}{x^d-1} = (x^d)^{k-1} 
    + \cdots + x^d + 1 = g(x) \in (\Z/p\Z)[x]. \] 
    By Fermat's little theorem (Corollary~\ref{cor:5.3}), $x^{p-1}-1$ has 
    $p-1$ distinct roots in $\Z/p\Z$. Since $\Z/p\Z$ is a field, any 
    polynomial of degree $n$ in $(\Z/p\Z)[x]$ has at most $n$ roots. In particular, 
    $(x^d-1)g(x)$ factors into linear polynomials in $(\Z/p\Z)[x]$, and the 
    result follows. 
\end{pf}

\begin{theo}{thm:6.6}
    If $p$ is a prime, then $(\Z/p\Z)^*$ is a cyclic group. 
\end{theo}
\begin{pf}
    For each divisor $d$ of $p-1$, let $\lambda(d)$ denote the number of elements 
    in $(\Z/p\Z)^*$ of order $d$. By Proposition~\ref{prop:6.5}, there are 
    exactly $d$ elements of $(\Z/p\Z)^*$ whose order divides $d$, so we obtain 
    \[ d = \sum_{c\mid d} \lambda(c). \] 
    By the M\"obius inversion formula (Proposition~\ref{prop:2.5}), we have 
    \[ \lambda(d) = \sum_{c\mid d} \mu(c) \frac{d}{c} = d \cdot \sum_{c \mid d}
    \frac{\mu(c)}{c} = d \cdot \prod_{p\mid d} \left( 1 - \frac{1}{p} \right) 
    = \phi(d), \] 
    where the final equality follows from Corollary~\ref{cor:6.4}. Hence, there 
    are $\phi(p-1)$ elements of $(\Z/p\Z)^*$ of order $p-1$, so 
    $(\Z/p\Z)^*$ is cyclic. 
\end{pf}

\subsection{Primitive Roots: The Prime Power Case}

\vspace{2ex}
\begin{defn}{defn:6.7}
    Let $n \in \Z^+$, and let $a \in \Z$. We say that $a$ is a {\bf primitive 
    root} modulo $n$ if $a + n\Z$ generates $(\Z/n\Z)^*$. 
\end{defn}

\begin{remark}{remark:6.8}
    \begin{enumerate}[(1)]
        \item For any prime $p$, we saw that $(\Z/p\Z)^*$ is cyclic by 
              Theorem~\ref{thm:6.6}, so there exists a primitive root modulo $p$; 
              in fact, there are $\phi(p-1)$ of them. 

              Artin conjectured that if $a$ is a positive integer that is not 
              a perfect square, then $a$ is a primitive root modulo $p$ for 
              infinitely many primes $p$. This conjecture is still open, but 
              it can be deduced from the generalized Riemann hypothesis by 
              the work of Hooley. 
        \item Why do we require that $a$ is not a perfect square? Note that if 
              $p$ is an odd prime, then $p-1$ is even. We want $a$ to have order 
              $p-1$. Assume that $a = k^2$ for some integer $k$, and that 
              $a$ is a primitive root modulo $p$. Then there exists an integer 
              $i$ such that $a^i \equiv k \pmod p$. We see that $a^{2i} \equiv a 
              \pmod p$, and hence $a^{2i-1} \equiv 1 \pmod p$. Since the 
              order of $a$ is $p-1$, we have $(p-1) \mid (2i-1)$. But 
              $p-1$ is even and $2i-1$ is odd, which is a contradiction. 
        \item Observe that $2$ is a primitive root modulo $5$, but $2$ is not 
              a primitive root modulo $7$ since $2^3 \equiv 1 \pmod 7$. In 
              general $(\Z/n\Z)^*$ is not cyclic, as primitive roots might not 
              exist modulo $n$. For example, $(\Z/8\Z)^* = \{[1], [3], [5], [7]\}$ 
              with $1^2 \equiv 3^2 \equiv 5^2 \equiv 7^2 \equiv 1 \pmod 8$, 
              so this group is not cyclic. 
    \end{enumerate}
\end{remark}

\begin{prop}{prop:6.9}
    Let $p$ be prime, and let $\ell$ be a positive integer. If $a \equiv b 
    \pmod{p^\ell}$, then 
    \[ a^p \equiv b^p \pmod{p^{\ell+1}}. \] 
\end{prop}
\begin{pf}
    Write $a = b + cp^\ell$ for some $c \in \Z$. Then we have 
    \[ a^p = (b + cp^\ell)^p = b^p + \binom{p}{1} b^{p-1} cp^\ell + 
    \binom{p}{2} b^{p-2} (cp^\ell)^2 + \cdots + \binom{p}{p} (cp^\ell)^p, \] 
    so $a^p \equiv b^p \pmod{p^{\ell+1}}$, since $2\ell \geq \ell+1$. 
\end{pf}

\begin{prop}{prop:6.10}
    If $\ell \geq 2$ is an integer and $p$ is an odd prime, then for any 
    $a \in \Z$, we have 
    \[ (1+ap)^{p^{\ell-2}} \equiv 1 + ap^{\ell-1} \pmod{p^\ell}. \] 
\end{prop}
\begin{pf}
    We proceed by induction on $\ell$. The result is clear for $\ell = 2$. 
    Suppose the result holds for some integer $\ell \geq 2$. We prove it for 
    $\ell + 1$. 

    By Proposition~\ref{prop:6.9} and our inductive hypothesis, we have 
    \[ (1 + ap)^{p^{\ell-1}} \equiv (1 + ap^{\ell-1})^p 
    \equiv 1 + \binom{p}{1} ap^{\ell-1} + \binom{p}{2} (ap^{\ell-1})^2 
    + \cdots + \binom{p}{p} (ap^{\ell-1})^p \pmod{p^{\ell+1}}. \] 
    Since $\ell \geq 2$ implies $2(\ell-1) + 1 \leq 3(\ell-1) \leq k(\ell-1)$, 
    we see that $p^{2(\ell-1)+1}$ divides $(ap^{\ell-1})^k$ for $k = 3, \dots, p$. 
    Furthermore, $p^{2(\ell-1)+1}$ divides $\binom{p}{2}(ap^{\ell-1})^2$ since 
    \[ \binom{p}{2} (ap^{\ell-1})^2 = \frac{p(p-1)}2 (ap^{\ell-1})^2 
    = \frac{p-1}2 a^2 p^{2\ell-1}. \] 
    Note that $(p-1)/2$ is an integer since $p$ is odd. Hence, $p^{2(\ell-1)+1}$ 
    divides the sum 
    \[ \binom{p}{2} (ap^{\ell-1})^2 + \cdots + \binom{p}{p} (ap^{\ell-1})^p 
    \pmod{p^{\ell+1}}. \] 
    Now, since $\ell \geq 2$ implies $2(\ell-1) + 1 \geq \ell+1$ and $p$ is odd, 
    we have 
    \[ 1 + \binom{p}{1} ap^{\ell-1} + \binom{p}{2} (ap^{\ell-1})^2 
    + \cdots + \binom{p}{p} (ap^{\ell-1})^p \equiv 1 + \binom{p}{1} ap^{\ell-1}
    \equiv 1 + ap^\ell \pmod{p^{\ell+1}}. \] 
    The result holds for all integers $\ell \geq 2$ by induction. 
\end{pf}

\begin{prop}{prop:6.11}
    Let $p$ be an odd prime, $\ell$ be a positive integer, and $a$ be an integer 
    coprime with $p$. Then $1 + ap$ has order $p^{\ell-1}$ in $(\Z/p^\ell\Z)^*$. 
\end{prop}
\begin{pf}
    By Proposition~\ref{prop:6.10}, we have 
    \[ (1+ap)^{p^{\ell-2}} \equiv 1 + ap^{\ell-1} \pmod{p^\ell}. \] 
    Since $a$ is coprime with $p$, we see that 
    \[ (1+ap)^{p^{\ell-2}} \not\equiv 1 \pmod{p^\ell}. \] 
    Applying Proposition~\ref{prop:6.10} again, we have 
    \[ (1+ap)^{p^{\ell-1}} \equiv 1 + ap^\ell \pmod{p^{\ell+1}}, \] 
    which implies that 
    \[ (1+ap)^{p^{\ell-1}} \equiv 1 \pmod{p^\ell}. \] 
    Thus, $1 + ap$ has order $p^{\ell-1}$ in $(\Z/p^\ell\Z)^*$. 
\end{pf}

\begin{theo}{thm:6.12}
    Let $p$ be an odd prime, and $\ell$ be a positive integer. Then 
    $(\Z/p^\ell\Z)^*$ is cyclic group. 
\end{theo}
\begin{pf}
    Since $(\Z/p\Z)^*$ is cyclic by Theorem~\ref{thm:6.6}, there is a primitive 
    root $g$ modulo $p$. If $g^{p-1} \equiv 1 \pmod{p^2}$, then 
    \[ (g+p)^{p-1} \equiv g^{p-1} + \binom{p-1}{1} g^{p-2}p + \binom{p-1}{2} 
    g^{p-3}p^2 + \cdots + \binom{p}{p} p^{p-1} \equiv 1 + \binom{p-1}{1} 
    g^{p-2}p \pmod{p^2}, \] 
    so $(g+p)^{p-1} \not\equiv 1 \pmod{p^2}$. In particular, at least one of 
    $g^{p-1}$ and $(g+p)^{p-1}$ is not congruent to $1$ modulo $p^2$. 
    Without loss of generality, we assume that $g^{p-1} \not\equiv 1 \pmod{p^2}$. 
    We claim that $g$ is a primitive root modulo $p^\ell$, and it will follow that 
    $(\Z/p^\ell\Z)^*$ is cyclic. 

    Suppose that $g$ has order $m$ in $(\Z/p^\ell\Z)^*$. By Euler's theorem 
    (Theorem~\ref{thm:5.2}), we have 
    \[ g^{\phi(p^\ell)} \equiv 1 \pmod{p^\ell}, \] 
    and hence $m \mid (p^\ell - p^{\ell-1}) = (p-1)p^{\ell-1}$. Write 
    $m = dp^s$ where $d \mid (p-1)$ and $0 \leq s \leq \ell-1$. By Fermat's 
    little theorem (Corollary~\ref{cor:5.3}), we have $g^p \equiv g \pmod p$. 
    Provided that $s \neq 0$, we have 
    \[ g^{p^s} \equiv g \pmod p. \] 
    However, we have $g^m \equiv 1 \pmod{p^\ell}$, so $g^m \equiv 1 \pmod p$, 
    which implies that $g^d \equiv 1 \pmod p$ as well. Since $g$ is a 
    primitive root modulo $p$, we see that $(p-1) \mid d$. Then $d = p-1$,
    so $m = (p-1)p^s$. Since $g^{p-1} \not\equiv 1 \pmod{p^2}$ and 
    $g^{p-1} \equiv 1 \pmod p$, there exists an integer $a$ coprime with $p$ 
    such that $g^{p-1} \equiv 1 + ap \pmod{p^2}$. By Proposition~\ref{prop:6.11}, 
    $1 + ap$ has order $p^{\ell-1}$ in $(\Z/p\Z)^*$. Then $g$ has order 
    $(p-1)p^\ell$ in $(\Z/p^\ell\Z)^*$, so $g$ is a primitive root of $p^\ell$. 
\end{pf}

\subsection{Primitive Roots: The General Case}

\vspace{2ex}
\begin{theo}{thm:6.13}
    If $\ell = 1, 2$, then $(\Z/2^\ell\Z)^*$ is cyclic. For $\ell \geq 3$, 
    we have the group isomorphism 
    \[ (\Z/2^\ell\Z)^* \cong \Z/2\Z \times \Z/2^{\ell-2}\Z. \] 
    In particular, we can write 
    \[ (\Z/2^\ell\Z)^* = \{(-1)^a 5^b + 2^\ell\Z : a \in \{0, 1\},\, 
    b \in \{0, \dots, 2^{\ell-2}-1\}\}. \] 
\end{theo}
\begin{pf} 
    It is clear that $(\Z/2\Z)^*$ and $(\Z/4\Z)^*$ are cyclic. Suppose that $\ell 
    \geq 3$. We claim that 
    \begin{equation}\label{eq:6.1}
        5^{2^{\ell-3}} \equiv 1 + 2^{\ell-1} \pmod{2^\ell}. 
    \end{equation}
    We proceed by induction on $\ell$. For $\ell = 3$, we have $5 \equiv 
    1 + 2^2 \pmod{2^3}$. Assume that $\eqref{eq:6.1}$ holds for some $\ell \geq 3$. 
    Note that $(1 + 2^{\ell-1})^2 = 1 + 2^\ell + 2^{2(\ell-1)}$ and 
    $2(\ell-1) \geq \ell+1$ for $\ell \geq 3$. By the induction hypothesis, 
    we know that $5^{2^{\ell-3}} = 1 + 2^{\ell-1} + k2^\ell$ for some $k \in \Z$. 
    It follows that 
    \begin{align*}
        5^{2^{\ell-2}} 
        &= (1 + 2^{\ell-1} + k2^\ell)^2 \\ 
        &= 1 + (2^{\ell-1})^2 + (k2^\ell)^2 + 2 \cdot 2^{\ell-1} + 2 \cdot k2^\ell 
        + 2 \cdot 2^{\ell-1} \cdot k2^\ell \\ 
        &= 1 + 2^\ell + k2^{\ell+1} + 2^{2\ell-2} + k2^{2\ell} + k^2 2^{2\ell}. 
    \end{align*}
    Since $2\ell - 2 \geq \ell + 1$ for $\ell \geq 3$, we see that 
    \[ 5^{2^{\ell-2}} \equiv 1 + 2^\ell \pmod{2^{\ell+1}}, \] 
    which completes the induction. In particular, we have $5^{2^{\ell-3}} 
    \not\equiv 1 \pmod{2^\ell}$ and $5^{2^{\ell-2}} \equiv 1 \pmod{2^\ell}$, 
    so $5$ has order $2^{\ell-2}$ in $(\Z/2^\ell\Z)^*$. 

    We now show that the numbers $(-1)^a 5^b$ with $a \in \{0, 1\}$ and 
    $b \in \{0, \dots, 2^{\ell-2} - 1\}$ are distinct modulo $2^\ell$ for 
    $\ell \geq 3$. Suppose that 
    \[ (-1)^{a_1} 5^{b_1} \equiv (-1)^{a_2} 5^{b_2} \pmod{2^\ell} \] 
    for some $a_1, a_2 \in \{0, 1\}$ and $b_1, b_2 \in \{0, \dots, 2^{\ell-2} - 1\}$. 
    Then, we obtain 
    \[ (-1)^{a_1} 5^{b_1} \equiv (-1)^{a_2} 5^{b_2} \pmod 4. \] 
    Since $5 \equiv 1 \pmod 4$, we have 
    \[ (-1)^{a_1} \equiv (-1)^{a_2} \pmod{4}, \] 
    which implies that $a_1 = a_2$. On the other hand, we get 
    \[ 5^{b_1} \equiv 5^{b_2} \pmod{2^\ell}, \] 
    and since $5$ has order $2^{\ell-2}$ with $b_1, b_2 \in \{0, \dots, 
    2^{\ell-2}-1\}$, we have $b_1 = b_2$. 
\end{pf}

\begin{theo}{thm:6.14}
    The only positive integers that have primitive roots are $1$, $2$, 
    $4$, $p^a$, or $2p^a$, where $p$ is prime and $a$ is a positive integer. 
\end{theo}
\begin{pf}
    Let $n = 2^{\ell_0} p_1^{\ell_1} \cdots p_r^{\ell_r}$, where $p_1, \dots, p_r$ 
    are distinct odd primes and $\ell_0, \dots, \ell_r$ are non-negative integers. 
    We have shown in Theorem~\ref{thm:6.2} that 
    \[ (\Z/n\Z)^* \cong (\Z/2^{\ell_0}\Z)^* \times (\Z/p_1^{\ell_1}\Z)^* 
    \times \cdots \times (\Z/p_r^{\ell_r}\Z)^*. \] 
    By Theorem~\ref{thm:6.12}, $(\Z/p_i^{\ell_i}\Z)^*$ is cyclic for all 
    $i = 1, \dots, r$. Moreover, by Theorem~\ref{thm:6.13}, $(\Z/2^{\ell_0}\Z)^*$ 
    is cyclic for $0 \leq \ell_0 \leq 2$ and is isomorphic to 
    $\Z/2\Z \times \Z/2^{\ell_0-2}\Z$ for $\ell \geq 3$. Hence, the order 
    of any element of $\Z/n\Z$ is a divisor of 
    $\lambda(n) = \lcm(b, \phi(p_1^{\ell_1}), \dots, \phi(p_r^{\ell_r}))$, 
    where we define 
    \[ b = \begin{cases} 
        \phi(2^{\ell_0}) & \text{if } 0 \leq \ell_0 \leq 2, \\ 
        \phi(2^{\ell_0})/2 & \text{if } \ell \geq 3.
    \end{cases} \] 
    It is clear that $\lambda(n) < \phi(2^{\ell_0}) \phi(p_1^{\ell_1}) 
    \cdots \phi(p_r^{\ell_r})$ except in the cases where $n$ is of the form 
    $1$, $2$, $4$, $p^a$, or $2p^a$ where $p$ is a prime and $a$ is a positive 
    integer. 
\end{pf}

\begin{defn}{defn:6.15}
    Write $n = 2^{\ell_0} p_1^{\ell_1} \cdots p_r^{\ell_r}$, where 
    $p_1, \dots, p_r$ are distinct odd primes and $\ell_0, \ell_1, \dots, 
    \ell_r$ are non-negative integers. If we define 
    \[ b = \begin{cases} 
        \phi(2^{\ell_0}) & \text{if } 0 \leq \ell_0 \leq 2, \\ 
        \phi(2^{\ell_0})/2 & \text{if } \ell \geq 3,
    \end{cases} \]
    then $\lambda(n) = \lcm(b, \phi(p_1^{\ell_1}), \dots, \phi(p_r^{\ell_r}))$ 
    is called the {\bf universal exponent} of $n$. 
\end{defn}

\begin{theo}{thm:6.16}
    Let $n$ be a positive integer, and let $\lambda(n)$ be the universal 
    exponent of $n$. Then for any integer $a$ coprime with $n$, we have 
    \[ a^{\lambda(n)} \equiv 1 \pmod n. \] 
\end{theo}
\begin{pf}
    This follows from Theorem~\ref{thm:6.14}. 
\end{pf}

This theorem gives us a strengthening of Euler's theorem (Theorem~\ref{thm:5.2}). 
Given a prime $p$, one can ask what an upper bound is for the smallest positive 
integer $a$ which is a primitive root modulo $p$. Hua proved that 
\[ a < 2^{\omega(p-1)+1} \sqrt{p}. \] 

\begin{theo}{thm:6.17}
    If $p$ is a prime of the form $4q+1$ where $q$ is an odd prime, then $2$ is a 
    primitive root modulo $p$. 
\end{theo}
\begin{pf}
    Let $t$ be the order of $2$ modulo $p$. By Fermat's little theorem 
    (Corollary~\ref{cor:5.3}), we have $t \mid (p-1)$ and hence $t \mid 4q$.
    By Theorem~\ref{thm:6.14}, $t$ is one of $1$, $2$, $4$, $2q$, or $4q$. 
    Note that $p = 13$ or $p > 20$, so $t$ cannot be $1$, $2$, or $4$. 
    Furthermore, by Euler's criterion (Theorem~\ref{thm:5.10}), we have 
    \[ 2^{(p-1)/2} \equiv 2^{2q} \equiv \left( \frac{2}{p} \right) \pmod p. \] 
    But we have 
    \[ \left( \frac{2}{p} \right) = (-1)^{(p^2-1)/8} = (-1)^{[(4q)^2+8q]/8}
    = (-1)^q = -1. \] 
    Then $t$ cannot be $q$ or $2q$, so we must have $t = 4q = p-1$, as required. 
\end{pf}\newpage 
\section{$L$-functions and Dirichlet's Theorem}\label{sec:7}

\subsection{Some Results on Primes in Arithmetic Progressions}\label{subsec:7.1}
Let $k$ and $\ell$ be coprime positive integers. Recall that Dirichlet's theorem 
asserts that $kn + \ell$ is prime for infinitely many integers $n$. For many 
pairs $(k, \ell)$, this result can be proved using elementary means. However, 
this is hard to prove generally, and we'll require more tools to do so. 

For example, consider the pair $(k, \ell) = (4, 3)$. Suppose that there are only 
finitely many primes $p_1, \dots, p_k$ of the form $4n + 3$. Then $4p_1 
\cdots p_k + 3$ must be divisible by a prime of the form $4n + 3$, since a product 
of primes congruent to $1$ modulo $4$ can only yield numbers congruent to 
$1$ modulo $4$. Notice that such a prime cannot be any of $p_1, \dots, p_k$, 
which is a contradiction. 

The following result is Dirichlet's theorem in the case where $k$ is an 
arbitrary positive integer and $\ell = 1$. 

\begin{theo}{thm:7.1} 
    Let $n \in \Z^+$. There are infinitely many primes congruent to $1$ modulo $n$.
\end{theo}
\begin{pf}
    This proof is due to Birkhoff and Vandiver (1904). Let $a > 2$ be an integer, 
    and consider the $n$-th cyclotomic polynomial
    \[ \Phi_n(x) = \prod_{\substack{1\leq j \leq n \\ \gcd(j, n) = 1}} 
    (x - \zeta_n^j), \] 
    where $\zeta_n = e^{2\pi i/n}$. We know that $\Phi_n(x) \in \Z[x]$ is irreducible, 
    and $x^n - 1 = \prod_{d\mid n} \Phi_d(x)$. We now consider $\Phi_n(x)$ 
    evaluated at $a$. 

    {\sc Claim.} If $p$ is a prime dividing $\Phi_n(a)$, then $p \mid n$ or 
    $p \equiv 1 \pmod n$. 

    {\sc Proof of Claim.} Since $x^n - 1 = \prod_{d\mid n} \Phi_d(x)$, we have 
    $p \mid a^n - 1$. We consider two cases. 

    First, if $p \nmid a^d - 1$ for all proper divisors $d$ of $n$, then the order of 
    $a$ modulo $p$ must be $n$. By Fermat's little theorem (Corollary~\ref{cor:5.3}), 
    we have $n \mid (p-1)$, so $p \equiv 1 \pmod n$. 

    Suppose there is a proper divisor $d$ of $n$ such that $p \mid a^d - 1$. Since 
    $p \mid \Phi_n(a)$, we obtain $p \mid (a^n - 1)/(a^d - 1)$. Notice that 
    \[ a^n = \left( 1 + (a^d - 1) \right)^{n/d} = 1 + \frac{n}{d} (a^d - 1) 
    + \binom{n/d}{2} (a^d - 1)^2 + \binom{n/d}{3} (a^d - 1)^3 + \cdots, \] 
    so it follows that 
    \[ \frac{a^n - 1}{a^d - 1} = \frac{n}{d} + \binom{n/d}{2} (a^d - 1) 
    + \binom{n/d}{3} (a^d - 1)^2 + \cdots. \] 
    Since $p \mid (a^n-1)/(a^d-1)$ and $p \mid (a^d-1)$, we get $p \mid n/d$ as well. 
    Therefore, we have $p \mid n$ as required. \hfill$\blacksquare$ 

    Now, assume that there are only finitely many primes $p_1, \dots, p_k$ 
    congruent to $1$ modulo $n$. The $n$-th cyclotomic polynomial is of the form 
    \[ \Phi_n(x) = x^{\phi(n)} + \cdots \pm 1. \] 
    Let $m$ be an integer. We see that $\Phi_n(np_1 \cdots p_k m)$ is not divisible 
    by $p_i$ for $i = 1, \dots, k$ and is coprime with $n$. Notice that for sufficiently 
    large $m$, we have $\Phi_n(np_1 \cdots p_km) \geq 2$. In particular, 
    $\Phi_n(np_1 \cdots p_km)$ has a prime divisor congruent to $1$ modulo $n$ 
    which is not in the set $\{p_1, \dots, p_k\}$, which is a contradiction. 
\end{pf}

\newpage 
\subsection{Characters}\label{subsec:7.2}

\vspace{2ex}
\begin{defn}{def:7.2}
    Let $G$ be a finite abelian group. A {\bf character} of $G$ is a homomorphism 
    $\chi : G \to \C^*$. Note that the set of characters of $G$ forms a group 
    under the operation $(\chi_1 \cdot \chi_2)(g) = \chi_1(g) \chi_2(g)$. 
    We call this group the {\bf dual group} of $G$, and denote it by $\hat G$. 
    The identity of $\hat G$ is the character $\chi_0$ with $\chi_0(g) = 1$
    for all $g \in G$. We call $\chi_0$ the {\bf principal character}. 
\end{defn}

Notice that if $|G| = n$, then $g^n = e$ for all $g \in G$. In particular, we see that 
$(\chi(g))^n = \chi(g^n) = \chi(e) = 1$, so $\chi(g)$ is an $n$-th root of unity 
for all $g \in G$. 

\begin{theo}{thm:7.3}
    Let $G$ be a finite abelian group. 
    \begin{enumerate}[(1)]
        \item The order of $\hat G$ is equal to the order of $G$. 
        \item The dual group $\hat G$ is isomorphic to $G$. 
        \item We have the formulas 
        \begin{align*}
            \sum_{\chi\in\hat G} \chi(g) = \begin{cases}
                |G| & \text{if } g = e, \\ 
                0 & \text{otherwise;}
            \end{cases} \\ 
            \sum_{g\in G} \chi(g) = \begin{cases} 
                |G| & \text{if } \chi = \chi_0, \\ 
                0 & \text{otherwise.}
            \end{cases}
        \end{align*}
    \end{enumerate}
\end{theo}
\begin{pf}~
    \begin{enumerate}[(1)]
        \item Recall that a finite abelian group is the direct product of cyclic 
        groups. Hence, there exist elements $g_1, \dots, g_r \in G$ and 
        $h_1 \cdots h_r \in \N$ with $h_1 \cdots h_r = |G|$ such that 
        every element $g \in G$ has a unique representation $g = g_1^{a_1} 
        \cdots g_r^{a_r}$ with $1 \leq a_i \leq h_i$, and $g_i^{h_i} = e$ for 
        $i = 1, \dots, r$. 

        Any character $\chi$ is uniquely determined by its action on 
        $g_1, \dots, g_r$. Since $g_i^{h_i} = e$, we have $(\chi(g_i))^{h_1} = 1$, 
        which shows that $\chi(g_i)$ is an $h_i$-th root of unity. Hence, there are 
        at most $h_1 \cdots h_r$ characters. 

        On the other hand, there are at least $h_1 \cdots h_r$ characters because 
        if $\omega_i$ is an $h_i$-th root of unity, then we may define $\chi(g_i)
        = \omega_i$ for $i = 1, \dots, r$ and extend multiplicatively to $G$. 
        We conclude that $|\hat G| = |G|$. 

        \item For each $i = 1, \dots, r$, let $\chi_i$ be the character 
        which sends $g_i$ to $e^{2\pi i/h_i}$ and $g_j$ to $1$ when $j \neq i$. 
        Define $\phi : G \to \hat G$ by 
        \[ \phi(g_1^{a_1} \cdots g_r^{a_r}) = \chi_1^{a_1} \cdots \chi_r^{a_r}. \] 
        Notice that $\phi$ is a homomorphism. We see that $\phi$ is injective 
        because for $j = 1, \dots, r$, we have 
        \[ (\chi_1^{a_1} \cdots \chi_r^{a_r})(g_j) = e^{2\pi ia_j/h_j} \] 
        Since $G$ is finite and $|\hat G| = |G|$ by (1), 
        we see that $\phi$ is also surjective. Therefore, we have $\hat G \cong G$. 

        \item Let $S(g) = \sum_{\chi\in\hat G} \chi(g)$. Notice that 
        $\chi(e) = 1$ for all $\chi \in \hat G$, so we obtain $S(g) = |\hat G| = |G|$. 

        Assume now that $g \neq e$. Then there exists a character $\chi_1 \in \hat G$ 
        such that $\chi_1(g) \neq 1$. Now, we have 
        \[ S(g) = \sum_{\chi \in \hat G} \chi(g) = \sum_{\chi \in \hat G} (\chi_1\chi)(g) 
        = \chi_1(g) \sum_{\chi\in\hat G} \chi(g) = \chi_1(g) S(g). \] 
        Since $\chi_1(g) \neq 1$, we must have $S(g) = 0$. 
        \newpage 
        On the other hand, let $T(\chi) = \sum_{g \in G} \chi(g)$. Notice that $\chi_0(g) 
        = 1$ for all $g \in G$, so $T(\chi_0) = |G|$. 
        
        If $\chi \neq \chi_0$, then there exists $g_1 \in G$ such that $\chi(g_1) 
        \neq 1$. We have 
        \[ T(\chi) = \sum_{g \in G} \chi(g) = \sum_{g \in G} \chi(g_1g) = 
        \chi(g_1) \sum_{g \in G} \chi(g) = \chi(g_1) T(\chi). \] 
        Since $\chi(g_1) \neq 1$, it follows that $T(\chi) = 0$. \qedhere 
    \end{enumerate}
\end{pf}

\subsection{Dirichlet Characters}\label{subsec:7.3}

\vspace{2ex}
\begin{defn}{def:7.4}
    Let $k \in \Z^+$, and denote $(\Z/k\Z)^*$ by $G(k)$. Let $\chi$ be a character 
    on $G(k)$. We can associate $\chi$ with a map $\Z \to \C^*$, which we also 
    call $\chi$, by setting 
    \[ \chi(a) = \begin{cases} 
        \chi([a]) & \text{if } [a] \in G(k), \\ 
        0 & \text{if } [a] \notin G(k), 
    \end{cases} \] 
    where $[a]$ denotes the conjugacy class of $a$. We call $\chi$ a 
    {\bf character modulo $k$}. 
\end{defn}

\begin{theo}{thm:7.5}
    Let $k \in \Z^+$, and let $\chi$ be a character modulo $k$. 
    \begin{enumerate}[(1)]
        \item If $\gcd(n, k) = 1$, then $\chi(n)$ is a $\phi(k)$-th root of unity. 
        \item We have $\chi(mn) = \chi(m)\chi(n)$; that is, $\chi$ is completely 
        multiplicative. 
        \item We have $\chi(n+k) = n$ for all $n \in \Z$; that is, $\chi$ is 
        periodic with period $k$. 
        \item We have the formulas 
        \begin{align*}
            \sum_{n=1}^k \chi(n) &= \begin{cases}
                \phi(k) & \text{if } \chi = \chi_0, \\ 
                0 & \text{otherwise;}
            \end{cases} \\ 
            \sum_{\chi} \chi(n) &= \begin{cases} 
                \phi(k) & \text{if } n \equiv 1 \text{ (mod $k$)}, \\ 
                0 & \text{otherwise,}
            \end{cases}
        \end{align*}
        where the second sum runs through all characters modulo $k$. 
        \item Let $\overline\chi$ be the conjugate character of $\chi$ with $\overline\chi(n) 
        = \overline{\chi(n)}$ for all $n \in \Z$. Let $\chi'$ be a character modulo 
        $k$. Then we have the formulas 
        \begin{align*}
            \sum_{\chi\in\hat G(k)} \chi(n) \overline\chi(m) &= \begin{cases}
                \phi(k) & \text{if } n \equiv m \text{ (mod $k$) and } \gcd(n, k) = 1, \\ 
                0 & \text{otherwise;}
            \end{cases} \\ 
            \sum_{n=1}^k \chi(n) \chi'(n) &= \begin{cases} 
                \phi(k) & \text{if } \chi' = \overline\chi, \\ 
                0 & \text{otherwise.}
            \end{cases}
        \end{align*}
    \end{enumerate}
\end{theo}
\newpage 
\begin{pf}
    Properties (1) to (4) either follow from definitions or Theorem~\ref{thm:7.3}. 
    Note that $\overline\chi(m) \chi(m) = 1 = \chi(m^{-1}) \chi(m)$, 
    where $m^{-1}$ denotes the multiplicative inverse of $m$ in $G(k)$. Therefore, 
    we have $\overline\chi(m) = \chi(m^{-1})$. It follows that 
    \[ \sum_{\chi\in\hat G(k)} \chi(n) \overline\chi(m) 
    = \sum_{\chi\in\hat G(k)} \chi(n) \chi(m^{-1}) 
    = \sum_{\chi\in\hat G(k)} \chi(nm^{-1}). \] 
    By (4), the last sum is $\phi(k)$ if $nm^{-1} \equiv 1 \pmod k$, or equivalently 
    $n \equiv m \pmod k$, and the sum is $0$ otherwise. This gives us the first 
    equation in (5). Moreover, note that if $\chi' = \overline\chi$, then 
    $\chi\chi' = \chi_0$. Otherwise, $\chi\chi'$ is a non-principal character, 
    so the second equation in (5) follows from (4). 
\end{pf}

We now describe the group of characters modulo $k$. By the multiplicative property 
of characters, it is enough to discuss the characters modulo $p^a$ where $p$ is 
prime and $a \in \Z^+$. First, assume that $p$ is an odd prime. Let $g$ be a 
primitive root modulo $p^a$. If $n$ is coprime with $p$, then there is a unique 
integer $1 \leq \nu \leq \phi(p^a)$ such that $n \equiv g^\nu \pmod{p^a}$. 
For each integer $1 \leq b \leq \phi(p^a)$, we define the character $\chi^b$ by 
\[ \chi^b(a) = \exp\left( \frac{2\pi ib\nu}{\phi(p^a)} \right). \] 
In this way, we get $\phi(p^a)$ different characters modulo $p^a$, so this is the 
complete list. Now, let $k = 2^a$. If $a = 1$, then we simply have the principal 
character. For $a = 2$, we have the principal character together with the character 
$\chi_4$ given by 
\[ \chi_4(n) = \begin{cases}
    1 & \text{if } n \equiv 1 \text{ (mod $4$),} \\ 
    -1 & \text{if } n \equiv -1 \text{ (mod $4$),} \\ 
    0 & \text{otherwise.}
\end{cases} \] 
If $a \geq 3$, then $(\Z/2^a\Z)^*$ is not cyclic by Theorem~\ref{thm:6.13}. 
Moreover, we saw in the proof of that theorem that for any odd integer $n$, 
there is a unique pair of integers $(x, y)$ with $x \in \{0, 1\}$ and 
$y \in \{0, \dots, 2^{a-2}-1\}$ such that 
\[ n \equiv (-1)^x 5^y \pmod{2^a}. \] 
For $c, d \in \Z$ with $c \in \{0, 1\}$ and $d \in \{0, \dots, 2^{a-2}-1\}$, we define 
\[ \chi_{2^a}^{c,d}(n) = \begin{cases}
    \exp(\frac{2\pi icx}2 + \frac{2\pi idy}{2^{a-2}}) & \text{if } n \equiv 1 
    \text{ (mod $2$),} \\ 
    0 & \text{otherwise.} 
\end{cases} \] 
We obtain $\phi(2^a)$ different characters modulo $2^a$, giving us the complete list. 

\subsection{Dirichlet $L$-functions}\label{subsec:7.4}
The Riemann zeta function was a powerful tool for studying the prime counting 
function $\pi(x)$. This suggests that it might be helpful to introduce complex 
functions in order to understand the primes in arithmetic progressions. 

\begin{defn}{def:7.6}
    Let $k \in \Z^+$, and let $\chi$ be a character modulo $k$. For $\Re(s) > 1$, 
    we define the {\bf Dirichlet $L$-function} by 
    \[ L(s, \chi) = \sum_{n=1}^\infty \frac{\chi(n)}{n^s}. \] 
\end{defn}

As with the Riemann zeta function, we can establish the analytic continuation of 
$L(s, \chi)$ up to $\Re(s) > 0$. 

\begin{theo}{thm:7.7}
    The function $L(s, \chi)$ can be analytically continued to $\Re(s) > 0$ except 
    when $\chi$ is the principal character. If $\chi_0$ is the principal character 
    modulo $k$, then $L(s, \chi_0)$ can be analytically continued to $\Re(s) > 0$ 
    except at the point $s = 1$, where we have a simple pole with residue $\phi(k)/k$.  
\end{theo}
\begin{pf}
    Let $A(x) = \sum_{n\leq x} \chi(n)$. By (4) of Theorem~\ref{thm:7.5}, we see that 
    \[ A(x) = \begin{cases}
        \floor{\frac{x}{k}} \phi(k) + T(x) & \text{if } \chi = \chi_0, \\ 
        \floor{\frac{x}{k}} 0 + T(x) & \text{if } \chi \neq \chi_0,
    \end{cases} \]
    where $|T(x)| < \phi(k)$. It follows that 
    \[ A(x) = E(\chi) \frac{\phi(k)x}{k} + R(x), \] 
    with $|R(x)| < 2\phi(k)$ and 
    \[ E(\chi) = \begin{cases}
        1 & \text{if } \chi = \chi_0, \\ 
        0 & \text{if } \chi \neq \chi_0.
    \end{cases} \]
    Let $f(n) = 1/n^s$. By Abel's summation formula (Lemma~\ref{lemma:2.8}), we have 
    \begin{align*}
        \sum_{n\leq x} \frac{\chi(n)}{n^s} 
        &= \frac{A(x)}{x^s} + s \int_1^x \frac{A(u)}{u^{s+1}}\dd u \\ 
        &= E(\chi) \frac{\phi(k)}{k} \frac{1}{x^{s-1}} + \frac{R(x)}{x^s} 
        + sE(\chi) \frac{\phi(k)}{k} \left( \frac{-u^{-s+1}}{s-1} \bigg|_1^x \right)
        + s \int_1^x \frac{R(u)}{u^{s+1}}\dd u \\ 
        &= E(\chi) \frac{\phi(k)}{k} \left( x^{1-s} + \frac{s}{1-s} (x^{1-s} - 1) 
        \right) + \frac{R(x)}{x^s} + s \int_1^x \frac{R(u)}{u^{s+1}}\dd u.  
    \end{align*}
    We now consider two cases. 
    \begin{itemize}
        \item If $\chi \neq \chi_0$, then $E(\chi) = 0$. We see from above that 
        \[ \sum_{n\leq x} \frac{\chi(n)}{n^s} = \frac{R(x)}{x^s} + s \int_1^x 
        \frac{R(u)}{u^{s+1}}\dd u. \] 
        We have $|R(x)| < 2\phi(k)$, so by letting $x \to \infty$, we see that 
        \[ L(s, \chi) = s \int_1^\infty \frac{R(u)}{u^{s+1}}. \] 
        This integral converges for $\Re(s) > 0$, so $L(s, \chi)$ has an 
        analytic continuation to $\Re(s) > 0$. 
        \item Note that $E(\chi_0) = 1$. By the above equation, we have 
        \[ \sum_{n\leq x} \frac{\chi_0(n)}{n^s} = \frac{\phi(k)}{k} \left( x^{1-s} + 
        \frac{s}{1-s} (x^{1-s} - 1) \right) + \frac{R(x)}{x^s} + s \int_1^x 
        \frac{R(u)}{u^{s+1}}\dd u. \]
        Since $|R(x)| < 2\phi(k)$, letting $x \to \infty$ again gives 
        \[ L(s, \chi_0) = \frac{\phi(k)}{k} \frac{s}{s-1} + s \int_1^\infty 
        \frac{R(u)}{u^{s+1}}. \] 
        The integral converges for $\Re(s) > 0$, so $L(s, \chi_0)$ has an an analytic 
        continuation to $\Re(s) > 0$, except at the simple pole $s = 1$ with 
        residue $\phi(k)/k$. \qedhere 
    \end{itemize}
\end{pf}

\subsection{Dirichlet Series}\label{subsec:7.5}

\vspace{2ex}
\begin{defn}{def:7.8}  
    Let $\{\lambda_n\}_{n=1}^\infty$ be a strictly increasing sequence of 
    positive real numbers. A {\bf Dirichlet series} attached to 
    $\{\lambda_n\}_{n=1}^\infty$ is a series of the form 
    \[ \sum_{n=1}^\infty a_n e^{-\lambda_n z}, \] 
    where $\{a_n\}_{n=1}^\infty$ is a sequence of complex numbers and $z \in \C$. 
\end{defn}

\begin{theo}{thm:7.9} 
    If the Dirichlet series $\sum_{n=1}^\infty a_n e^{-\lambda_n z}$ converges 
    at $z = z_0$, then it converges uniformly for $\Re(z-z_0) \geq 0$ and 
    $\lvert\arg(z-z_0)\rvert \leq \alpha$ with $\alpha < \pi/2$. 
\end{theo}
\begin{pf}
    Without loss of generality, we may assume that $z_0 = 0$. Note that 
    $\sum_{n=1}^\infty a_n$ converges, so for any $\eps > 0$, there exists 
    $N = N(\eps) \in \N$ such that if $\ell, m > N$, then 
    \[ \left| \sum_{n=\ell}^m a_n \right| < \eps. \] 
    Defining $A_{\ell,m} = \sum_{n=\ell}^m a_n$ and taking the convention that 
    $A_{\ell,\ell-1} = 0$, we have 
    \begin{align*}
        \sum_{n=\ell}^\infty a_n e^{-\lambda_n z}
        &= \sum_{n=\ell}^m (A_{\ell,n} - A_{\ell,n-1}) e^{-\lambda_n z} \\ 
        &= \sum_{n=\ell}^{m-1} A_{\ell,n} (e^{-\lambda_n z} - e^{-\lambda_{n+1}z})
        + A_{\ell,m} e^{-\lambda_m z}. 
    \end{align*}
    For $\Re(z) \geq 0$, we see that 
    \[ \left| \sum_{n=\ell}^m a_n e^{-\lambda_n z} \right| 
    \leq \eps \left( \sum_{n=\ell}^{m-1} |e^{-\lambda_n z} - e^{-\lambda_{n+1}z}| 
    + 1 \right). \] 
    Note that 
    \[ e^{-\lambda_n z} - e^{-\lambda_{n+1}z} = z \int_{\lambda_n}^{\lambda_{n+1}} 
    e^{-tz}\dd t. \] 
    Moreover, for $z = x+iy \in \C$ with $x, y \in \R$, we have $|e^{-tz}| = 
    e^{-tx}$. Hence, we obtain 
    \begin{align*}
        |e^{-\lambda_n z} - e^{-\lambda_{n+1}z}|
        &\leq |z| \int_{\lambda_n}^{\lambda_{n+1}} e^{-tx}\dd t \\ 
        &\leq |z| \left( -\frac{e^{-tx}}{x} \bigg|_{\lambda_n}^{\lambda_{n+1}}
        \right) \\ 
        &= \frac{|z|}{x} (e^{-\lambda_n x} - e^{-\lambda_{n+1}x}). 
    \end{align*}
    It follows that 
    \begin{align*}
        \left| \sum_{n=\ell}^m a_n e^{-\lambda_n z} \right| 
        &\leq \eps \left( \frac{|z|}{x} \sum_{n=\ell}^{m-1} (e^{-\lambda_n x} 
        - e^{-\lambda_{n+1}x}) + 1 \right) 
        = \eps \left( \frac{|z|}{x} (e^{-\lambda_\ell x} - e^{-\lambda_m x})
        + 1 \right). 
    \end{align*}
    Now, for $\lvert\arg(z)\rvert \leq \alpha$ with $\alpha < \pi/2$, we have 
    \[ \frac{|z|}{x} = \frac{1}{\cos(\arg z)} < c \] 
    for some constant $c = c(\alpha)$. Moreover, note that $|e^{-\lambda_\ell x} 
    - e^{-\lambda_m x}| \leq 2$. Therefore, we have 
    \[ \left| \sum_{n=\ell}^m a_n e^{-\lambda_n z} \right| < (2c + 1)\eps. \] 
    In particular, the Dirichlet series converges uniformly for $\Re(z) \geq 0$ 
    and $\lvert\arg(z)\rvert \leq \alpha$. 
\end{pf}

We have proved that if the Dirichlet series converges at $z = z_0$, then 
it determines an analytic function for $\Re(z-z_0) \geq 0$ and 
$\lvert\arg(z-z_0)\rvert \leq \alpha$ with $\alpha < \pi/2$. Next, we'll 
show that if $\{a_n\}_{n=1}^\infty$ is in addition a sequence of non-negative 
real numbers, then the domain of convergence for the analytic function 
determined by the series is limited only by a singularity on the real axis. 

\begin{theo}{thm:7.10}
    Let $f(z) = \sum_{n=1}^\infty a_n e^{-\lambda_n z}$ be a Dirichlet series 
    where $\{a_n\}_{n=1}^\infty$ is a sequence of non-negative real numbers. 
    Suppose that the series converges for $\Re(z) > \sigma_0$ where $\sigma_0 
    \in \R$, and $f$ can be analytically continued in a neighbourhood of $\sigma_0$. 
    Then there exists $\eps > 0$ such that $\sum_{n=1}^\infty a_n e^{-\lambda_n z}$ 
    converges for $\Re(z) > \sigma_0 - \eps$. 
\end{theo}
\begin{pf}
    Without loss of generality, we may assume that $\sigma_0 = 0$. Since this 
    series converges on $\Re(z) > 0$, then for any $z \in \C$ with $\Re(z) > 0$, 
    the series converges uniformly on a neighbourhood centered at $z$ by 
    Theorem~\ref{thm:7.9} (we can find $w \in \C$ with $\Re(w) > 0$ such that 
    the fan-like uniform convergence area covers a neighbourhood of $z$). Then $f$ 
    is analytic at $z$, and hence analytic for $\Re(z) > 0$. Since $f$ is 
    analytic for $\Re(z) > 0$ and $f$ is also analytic in a neighbourhood of 
    $\sigma_0 = 0$, there exists $\eps > 0$ such that $f$ is analytic in 
    $|z - 1| \leq 1 + \eps$. We now consider the Taylor series expansion of $f$ 
    around $1$ in $|z - 1| \leq 1 + \eps$. Note that for $\Re(z) > 0$, we have 
    \[ f^{(m)}(z) = \sum_{n=1}^\infty a_n(-\lambda_n)^m e^{-\lambda_n z}. \] 
    This implies that 
    \[ f^{(m)}(1) = \sum_{n=1}^\infty a_n(-\lambda_n)^m e^{-\lambda_n}. \]
    Now, the Taylor series expansion of $f$ about $1$ in $|z - 1| \leq 1 + \eps$ 
    is of the form 
    \[ \sum_{m=0}^\infty \frac{f^{(m)}(1)}{m!} (z-1)^m. \] 
    Consider $f$ at the point $z = -\eps$. We have 
    \begin{align*} 
        f(-\eps) &= \sum_{m=0}^\infty \left( \sum_{n=1}^\infty a_n (-\lambda_n)^m 
        e^{-\lambda_n} \right) \frac{(-1-\eps)^m}{m!} \\
        &= \sum_{m=0}^\infty \left( \sum_{n=1}^\infty a_n \lambda_n^m 
        e^{-\lambda_n} \right) \frac{(1+\eps)^m}{m!}. 
    \end{align*}
    Since $a_n \geq 0$ and all the other terms above are positive, we can 
    switch the order of summation to obtain 
    \begin{align*}
        f(-\eps) &= \sum_{n=1}^\infty a_n e^{-\lambda_n} \left( \sum_{m=0}^\infty 
        \frac{\lambda_n^m (1+\eps)^m}{m!} \right) 
        = \sum_{n=1}^\infty a_n e^{-\lambda_n} e^{\lambda_n(1+\eps)} 
        = \sum_{n=1}^\infty a_n e^{\lambda_n \eps} 
        = \sum_{n=1}^\infty a_n e^{(-\lambda_n)(-\eps)}.
    \end{align*}
    Hence, the series $\sum_{n=1}^\infty a_n e^{-\lambda_n z}$ converges 
    to $f$ at $z = -\eps$. By Theorem~\ref{thm:7.9}, it converges to $f$ 
    for $\Re(z) > -\eps$ because for any $z \in \C$ with $\Re(z) > -\eps$, 
    we can find some $\alpha < \pi/2$ such that $\lvert\arg(z-(-\eps))\rvert 
    < \alpha$. 
\end{pf}

\subsection{Dirichlet's Theorem}\label{subsec:7.6}

\vspace{2ex}
\begin{theo}{thm:7.11}
    If $\chi$ is a character modulo $k$, then $L(s, \chi)$ is nonzero for 
    $\Re(s) > 1$. Furthermore, if $\chi$ is not principal, then $L(1, \chi)$ 
    is nonzero. 
\end{theo}
\begin{pf}
    Note that $L(s, \chi)$ converges absolutely for $\Re(s) > 1$. Moreover, 
    we showed in Theorem~\ref{thm:7.5} that $\chi$ is completely multiplicative, 
    so $L(s, \chi)$ has an Euler product representation for $\Re(s) > 1$ given by 
    \[ L(s, \chi) = \prod_p \left( 1 - \frac{\chi(p)}{p^s} \right)^{\!-1}. \] 
    Recall that given a sequence of complex numbers $\{a_n\}_{n=1}^\infty$, the 
    product $\prod_{n=1}^\infty (1 + a_n)$ with $1 + a_n \neq 0$ converges 
    absolutely (to a nonzero value) if and only if the series $\sum_{n=1}^\infty 
    |a_n|$ converges. Since $\sum_p |\chi(p)/p^s|$ converges for $\Re(s) > 1$, 
    so does the Euler product representation above. Thus, $L(s, \chi) \neq 0$
    for $\Re(s) > 1$. 

    For the second assertion, we have two cases, depending on whether $\chi$ is a 
    real or complex character. For $\Re(s) > 1$, the Euler product representation
    of $L(s, \chi)$ gives 
    \[ \log^* L(s, \chi) = \sum_p -\log\left(1 - \frac{\chi(p)}{p^s} \right) 
    = \sum_p \sum_{a=1}^\infty \frac{\chi(p^a)}{ap^{as}}, \] 
    where $\log$ denotes the principal branch and $\log^*$ indicates a branch 
    of the logarithm. 

    Let $k \geq 2$ be an integer, and let $\ell$ be an integer coprime with $k$. 
    Then we have 
    \[ \sum_{\chi\in\hat G(k)} \overline\chi(\ell) \log^* L(s, \chi) 
    = \sum_p \sum_{a=1}^\infty \frac{1}{ap^{as}} \sum_{\chi\in\hat G(k)} 
    \overline\chi(\ell) \chi(p^a). \] 
    By (5) of Theorem~\ref{thm:7.5}, we obtain 
    \begin{equation}\label{eq:7.1}
        \sum_{\chi\in\hat G(k)} \overline\chi(\ell) \log^* L(s, \chi) 
        = \phi(k) \sum_{a=1}^\infty \sum_{p^a\equiv\ell\text{ (mod }k)} \frac{1}{ap^{as}}. 
    \end{equation}
    Taking $\ell = 1$ in equation $\eqref{eq:7.1}$ and exponentiating both sides, 
    we get 
    \[ \prod_{\chi\in\hat G(k)} L(s, \chi) = \exp\left( 
        \phi(k) \sum_{a=1}^\infty \sum_{p^a\equiv1\text{ (mod }k)} \frac{1}{ap^{as}} 
        \right). \] 
    Therefore, if $s$ is real with $s > 1$, then 
    \begin{equation}\label{eq:7.2}
        \prod_{\chi\in\hat G(k)} L(s,\chi) \geq 1. 
    \end{equation}
    First, suppose that $L(1, \chi) = 0$ where $\chi$ is not a real character. 
    Then $\overline\chi$ is a character modulo $k$ with $\chi \neq \chi_0$. 
    Notice that when $s$ is real with $s > 1$, we have $\overline{L(s,\chi)} 
    = L(s,\overline\chi)$, and hence 
    \[ L(1, \overline\chi) = \overline{L(1, \chi)} = 0. \] 
    By Theorem~\ref{thm:7.7}, $L(s, \chi_0)$ has a simple pole at $s = 1$, 
    and $L(s, \chi)$ does not have a pole at $s = 1$ when $\chi \neq \chi_0$. 
    Hence, as $s \to 1$ on the real axis, we have 
    \[ \prod_{\chi\in\hat G(k)} L(s, \chi) = O((s-1)^{-1} (s-1)^2) = O(s-1). \] 
    However, this contradicts equation $\eqref{eq:7.2}$, so $L(1, \chi) \neq 0$ 
    when $\chi$ is not a real character. 

    Suppose now that $L(1, \chi) = 0$ where $\chi$ is a real character. 
    For $\Re(s) > 1$, we set 
    \[ g(s) = \frac{\zeta(s) L(s, \chi)}{\zeta(2s)}. \] 
    The Euler product representation of $g$ for $\Re(s) > 1$ is 
    \begin{align*}
        g(s) &= \prod_p \left( \frac{1 - p^{-2s}}{(1 - p^{-s})(1 - \chi(p)/p^s)} \right) \\ 
        &= \prod_p \left( \frac{1 + p^{-s}}{1 - \chi(p)/p^s} \right) \\ 
        &= \prod_p \left( 1 + \frac{1}{p^s} \right) \sum_{a=0}^\infty \frac{\chi(p^a)}{p^{as}} \\ 
        &= \prod_p \left( 1 + \sum_{a=1}^\infty \frac{\chi(p^{a-1}) + \chi(p^a)}{p^{as}} \right) \\ 
        &= \prod_p \left( 1 + \sum_{a=1}^\infty \frac{b(p^a)}{p^{as}} \right), 
    \end{align*}
    where $b(p^a) = \chi(p^{a-1}) + \chi(p^a)$. Since $\chi$ is a real character, 
    it takes on values from $\{-1, 0, 1\}$. Moreover, we know that $\chi$ 
    is multiplicative by Theorem~\ref{thm:7.5}, so we have 
    \[ b(p^a) = \chi(p^{a-1}) + \chi(p^a) = 
    \begin{cases}
        0 & \text{if } \chi(p) = 0, \\ 
        2 & \text{if } \chi(p) = 1, \\ 
        0 & \text{if } \chi(p) = -1. 
    \end{cases} \] 
    In all cases, we have $b(p^a) \geq 0$ for $a \geq 1$. Therefore, we see that 
    $g(s) = \sum_{n=1}^\infty a_n/n^s$ where $a_1 = 1$ and $a_n \geq 0$ for 
    all $n \geq 2$. Moreover, we had set $g(s) = \zeta(s)L(s, \chi)/\zeta(2s)$ 
    for $\Re(s) > 1$. Since $L(1, \chi) = 0$ eliminates the pole of $\zeta(s)$
    at $s = 1$ and $\zeta(2s)$ is nonzero and analytic for $\Re(s) > 1/2$, 
    it follows that $g(s)$ has an analytic continuation to $\Re(s) > 1/2$. 

    We now apply Theorem~\ref{thm:7.10} to conclude that the series defining 
    $g$ converges to $g$ for $\Re(s) > 1/2$. Letting $s \to 1/2$ from above 
    on the real axis, we have 
    \[ g(s) = O(s - 1/2) = o(1) \] 
    since $\zeta(2s)$ has a pole at $s = 1/2$. However, since 
    \[ g(s) = 1 + \sum_{n=2}^\infty \frac{a_n}{n^s} \] 
    with $a_n \geq 0$ for $n \geq 2$, we obtain $g(s) \geq 1$ for $\Re(s) > 1/2$. 
    This is a contradiction, so we must have $L(1, \chi) \neq 0$ when 
    $\chi$ is a real character. 
\end{pf}

\begin{theo}{thm:7.12}
    If $k$ and $\ell$ are coprime integers with $k \geq 2$, then the series 
    \[ \sum_{p\equiv\ell\text{ (mod }k)} \frac{1}{p} \] 
    diverges. Consequently, there are infinitely many primes in the 
    arithmetic progression $kn + \ell$. 
\end{theo}
\begin{pf}
    From equation $\eqref{eq:7.1}$ from Theorem~\ref{thm:7.11}, we have 
    \[ \frac{1}{\phi(k)} \sum_{\chi\in\hat G(k)} \overline\chi(\ell) 
    \log L(s, \chi) = \sum_{a=1}^\infty \sum_{p^a\equiv\ell\text{ (mod }k)}
    \frac{1}{ap^{as}}. \] 
    As $s \to 1$ from the right on the real axis, $(s-1)^{E(\chi)} L(s, \chi)$ 
    tends to a finite nonzero limit, where $E(\chi) = 1$ if $\chi = \chi_0$ 
    and $E(\chi) = 0$ otherwise. Then $E(\chi) \log(s-1) + 
    \log L(s, \chi)$ also tends to a limit. It follows that as $s \to 1$ 
    from the right on the real axis, we have 
    \[ \log L(s, \chi) = -E(\chi)\log(s-1) + O(1). \] 
    Hence, we get 
    \begin{align*}
        \frac{1}{\phi(k)} \sum_{\chi\in\hat G(k)} \overline\chi(\ell) \log L(s, \chi)
        &= \frac{1}{\phi(k)} \log L(s, \chi_0) + \frac{1}{\phi(k)} 
        \sum_{\substack{\chi\in\hat G(k) \\ \chi\neq\chi_0}} \overline\chi(\ell) 
        \log L(s, \chi) \\
        &= -\frac{1}{\phi(k)} \log(s-1) + O(1). 
    \end{align*}
    Combining this with $\eqref{eq:7.1}$ yields 
    \[ \sum_{a=1}^\infty \sum_{p^a\equiv\ell\text{ (mod }k)} \frac{1}{ap^{as}} 
    = -\frac{1}{\phi(k)} \log(s-1) + O(1). \] 
    Thus, we have 
    \[ \sum_{p\equiv\ell\text{ (mod }k)} \frac{1}{p^s} + 
    \sum_{a=2}^\infty \sum_{p^a\equiv\ell\text{ (mod }k)} \frac{1}{ap^{as}} 
    = -\frac{1}{\phi(k)} \log(s-1) + O(1). \] 
    On the other hand, for $s \in \R$ with $s \geq 1$, we have 
    \begin{align*}
        \sum_{a=2}^\infty \sum_{p^a\equiv\ell\text{ (mod }k)} \frac{1}{ap^{as}} 
        &\leq \frac12 \sum_{a=2}^\infty \sum_{p^a\equiv\ell\text{ (mod }k)} \frac{1}{p^{as}} \\ 
        &\leq \frac12 \sum_{n=2}^\infty \left( \frac{1}{n^{2s}} + \frac{1}{n^{3s}} + \cdots \right) \\ 
        &\leq \frac12 \sum_{n=2}^\infty \frac{1}{n^{2s}} \left( \frac{1}{1 - 1/n^s} \right) \\ 
        &\leq \sum_{n=1}^\infty \frac{1}{n^2} = \frac{\pi^2}{6}. 
    \end{align*}
    Therefore, as $s \to 1$ from the right on the real axis, we obtain 
    \[ \sum_{p\equiv\ell\text{ (mod }k)} \frac{1}{p^s} = -\frac{1}{\phi(k)} \log(s-1) + O(1). \] 
    Since the quantity $\log(s-1)$ blows up as $s \to 1$, the series diverges. 
\end{pf}

\subsection{Distribution of Primes in Arithmetic Progressions}\label{subsec:7.7}
Let $k$ and $\ell$ be coprime integers with $k \geq 2$. For each $x \in \R$, 
define $\pi(x, k, \ell)$ to be the number of primes $p$ satisfying 
$p \leq x$ and $p \equiv \ell \pmod k$. Then it can be shown that 
\[ \pi(x, k, \ell) \sim \frac{1}{\phi(k)} \frac{x}{\log x} \sim \frac{\Li(x)}{\phi(k)}. \] 
For $t \in \R$ and $k \in \Z^+$, set $\tau(k, t) = \max\{|t|, k + 2\}$. 
Let $c \in \R$ with $0 < c < 1$, and define the set $R_c(k)$ by 
\[ R_c(k) = \left\{ \sigma + it : 1 - \frac{c}{\log \tau(k, t)} < \sigma \right\}. \] 
One can show that there exists a positive real number $c_0$ such that if 
$\chi$ is a non-real character modulo $k$ for $k \geq 2$, then $L(s, \chi)$ 
is nonzero in $R_{c_0}(k)$. 

When $\chi$ is a real non-principal character, this is not true in general. 
However, such a $c_0$ exists if we allow for the possibility that 
there is a point $\beta$ on the real axis in $R_{c_0}(k)$ where 
$L(s, \chi)$ is zero. 

\begin{defn}{def:7.13} 
    If $L(s, \chi)$ vanishes at $\beta \in R_{c_0}(k)$, then $\beta$ is a 
    simple zero of $L(s, \chi)$ and is called a {\bf Siegel zero}. 
\end{defn}

The extended Riemann hypothesis implies that $L(s, \chi)$ is nonzero for 
$\Re(s) > 1/2$, so no Siegel zero exists under this hypothesis. 

Let $k$ and $\ell$ be coprime integers with $k \geq 2$. Put $b = \beta(k)$ 
if there is a real non-principal character $\chi$ where $\beta$ is a 
zero of $L(s, \chi)$ in $R_{c_0}(k)$, and set $b = 1$ otherwise. Then 
there exists $a > 0$ such that 
\[ \pi(x, k, \ell) = \frac{\Li(x)}{\phi(k)} - \frac{\lambda(b)}{b} \frac{x^b}{\phi(k)} 
+ O(x \exp(-a\sqrt{\log x})), \] 
where $\lambda(b) = 0$ if $b = 1$, and $\lambda(b) = \chi(\ell)$ if $b \neq 1$. 
We would like to know if the term 
\[ \frac{\lambda(b)}{b} \frac{x^b}{\phi(k)} \] 
exists, or in other words, whether a Siegel zero exists. We haven't been 
able to do so yet, however. The best ``effective'' estimate for the size 
of a Siegel zero $\beta(k)$ associated to $L(s, \chi)$ where $\chi$ is a 
real character modulo $k$ is due to Pintz, who proved that 
\[ \beta(k) < 1 - \frac{c}{\sqrt{k}}, \] 
where $c$ is an effectively computable positive number. On the other hand, 
Siegel proved that for every $\eps > 0$, there exists a positive 
number $c(\eps)$ such that 
\[ \beta(k) < 1 - \frac{c(\eps)}{k^{\eps}}. \] 
Unfortunately, there is no known algorithm for computing $c(\eps)$ given 
$\eps > 0$. Using Siegel's estimate, one can prove that if $H$ is a 
positive number satisfying $k \leq (\log x)^H$, then 
\[ \pi(x, k, \ell) = \frac{\Li(x)}{\phi(k)} + O\left( \frac{x}{\exp(C\sqrt{\log x})} \right) \] 
for some $C > 0$. However, notice that this big-$O$ term is quite ineffective. 


\end{document}