\section{Digital Signature Schemes}
Cryptographic digital signatures aim to transfer our conventional handwritten 
signatures into some electronic form. The man-in-the middle attack to the 
Diffie-Hellman protocol as described in Section 6.2.2 motivates the need for 
cryptographic signatures. If Alice was able to generate a signature $\sigma$ 
for her message $g^\alpha$ to Bob in the protocol, and Bob was able to 
verify Alice's signatures on her messages, then the adversary would not be 
able to mount a man-in-the-middle attack unless they were able to forge 
signatures on behalf of Alice. 

Notice that in the Diffie-Hellman protocol, Alice and Bob are not concerned 
about the confidentiality of the messages $g^\alpha$ and $g^\beta$ being 
sent over the channel, but they are worried about the \emph{integrity} and 
\emph{authentication} of the messages they receive. In other words, Bob 
should be able to verify 
\begin{enumerate}[(i)]
    \item whether the message he received was indeed originated from Alice; and 
    \item whether the message he received from Alice has been altered by 
          other parties.
\end{enumerate}
Alice should be able to do the same for Bob's messages. 

We have previously covered the RSA and ElGamal public key encryption schemes. 
Now, we will present the RSA and ElGamal public key digital signature schemes. 
Practical applications require signing long messages which after being encoded as 
integers would exceed the size of the underlying cryptographic parameters. 
The RSA modulus $N$ and the ElGamal group parameter $p$ are examples of this. 
Therefore, signature schemes generally apply a cryptographic hash function 
$H$ to a message $m$ before signing it. In particular, $H : \{0, 1\}^* 
\to \{0, 1\}^n$ is a function that maps strings of arbitrary length into 
\emph{small} fixed length strings. For security and efficiency reasons, 
$H$ must satisfy the following properties. 
\begin{enumerate}
    \item \textbf{Efficiently computable.} Given $H$ and a message $m \in \{0, 1\}^*$, 
          it should be efficient to compute $H(m)$. 
    \item \textbf{Preimage resistant.} Given a hash value $h \in \{0, 1\}^n$ 
          chosen uniformly at random, it should be computationally infeasible 
          to find any preimage $m \in \{0, 1\}^*$ such that $H(m) = h$. 
    \item \textbf{Second preimage resistant.} Given an input value $m_1 \in \{0, 1\}^*$
          chosen uniformly at random, it should be computationally infeasible 
          to find a second preimage $m_2 \in \{0, 1\}^*$ such that $H(m_1) = H(m_2)$. 
    \item \textbf{Collision resistant.} It should be computationally infeasible 
          to find inputs $m_1, m_2 \in \{0, 1\}^*$ such that $m_1 \neq m_2$ and 
          $H(m_1) = H(m_2)$. Such a pair $(m_1, m_2)$ is called a \textbf{collision} 
          for $H$. 
\end{enumerate}
NIST specifies a list of standardized hash algorithms (SHA); among them is 
the commonly deployed hash algorithm SHA-$256$, which takes a string of 
arbitrary length as input and outputs a $256$-bit string. For a full list of 
SHA and the current NIST policy on hash functions, see 
\href{https://csrc.nist.gov/Projects/Hash-Functions/NIST-Policy-on-Hash-Functions}
{NIST Policy on Hash Functions}. 

\begin{remark}
    We mentioned above that a cryptographic hash function is applied to messages 
    before they are signed. In standardized signature schemes, messages go 
    through additional preprocessing such as encoding, padding, and 
    randomization. In this course, we will present textbook versions of 
    signature schemes for simplicity, and skip details of preprocessing 
    messages before signing. For example, in the RSA signature scheme, 
    we will assume that messages to be signed already belong to $\Z_N$. 
    We refer to \href{https://nvlpubs.nist.gov/nistpubs/FIPS/NIST.FIPS.186-4.pdf}
    {FIPS 186-4} for further details on some standardized
    signature schemes. 
\end{remark}

\subsection{The RSA Signature Scheme}
The RSA signature scheme consists of the following three algorithms. 
\begin{enumerate}
    \item \textbf{Key generation.} This is exactly the same key generation 
    algorithm as in the RSA encryption scheme. The public key and secret 
    key are $\PubKey = [N, e]$ and $\SecKey = [p, q, d]$, where 
    $N = pq$ is a product of two randomly chosen distinct primes $p$ and $q$, 
    and $e, d \in (1, \phi)$ satisfy $\gcd(e, \phi) = 1$ and 
    $ed \equiv 1 \pmod\phi$ where $\phi = (p-1)(q-1)$. 

    \item \textbf{Signature generation algorithm.} For a given public modulus $N$ 
    and the secret exponent $d$, the signature generation algorithm takes a 
    message $m$ as input, and outputs the signature $\sigma = m^d \in \Z_N$. 
    We can denote this process with 
    \begin{align*}
        \Sig_{N,d} : \Z_N &\to \Z_N \\ 
        m &\mapsto \sigma = m^d \pmod N, 
    \end{align*}
    or simply by $\Sig(m) = m^d \pmod N$ when $N$ and $d$ are clear from the context. 

    \item \textbf{Signature verification algorithm.} For a given public key 
    $\PubKey = [N, e]$, the signature verification algorithm takes a message 
    $m \in \Z_N$ and a signature $\sigma \in \Z_N$ as input, and outputs 
    $\True$ (indicating the validity of the message) if and only if 
    $m = \sigma^e \pmod N$. That is, we write $\Ver_{N,e}(m, \sigma) 
    = \True$ if and only if $m = \sigma^e$. 
\end{enumerate}

\begin{exmp}
    Suppose that the public key is $\PubKey = [49163, 20771]$ and the secret key is 
    $\SecKey = [233, 211, 36971]$. We can generate a signature for the message 
    $m = 37917$ by 
    \[ \sigma = m^d \text{ (mod $N$)} = 37917^{36971} \text{ (mod $49163$)} = 123. \] 
    The validity of the signature $\sigma = 123$ on the message $m = 37917$ can be 
    verified because given the public key $\PubKey = [N, e] = [49163, 20771]$, we have 
    \[ \sigma^e \text{ (mod $N$)} = 123^{20771} \text{ (mod $49163$)} = 37917 = m. \] 
\end{exmp}

\subsection{The ElGamal Signature Scheme}
As with the RSA signature scheme, the ElGamal signature scheme consists of three algorithms. 
\begin{enumerate}
    \item \textbf{Key generation.} The purpose of this algorithm is to generate 
    a public key and secret key pair, where the secret key is used to generate 
    signatures and the public key is used to verify signatures on a given message. 
    The public key is a tuple 
    \[ \PubKey = [p, g, y], \] 
    where $p$ is prime, $g$ is a generator for $\Z_p^*$, and $y = g^x$ for some 
    integer $x \in [1, p-1]$. The integer $x$ is chosen uniformly at random, 
    and is the secret key; that is, we have 
    \[ \SecKey = [x]. \] 

    \item \textbf{Signature generation algorithm.} For a given public key 
    $\PubKey = [p, g, y]$ and the secret key $x$, the signature generation 
    algorithm takes a message $m \in \Z_p^*$ as input and generates a 
    signature $\sigma = [r, s]$ as follows. 
    \begin{enumerate}[(i)]
        \item Select an integer $k \in \Z_{p-1}^*$ uniformly at random, and notice 
        that $\gcd(k, p-1) = 1$. 
        \item Compute $r = g^k \pmod p$. 
        \item Compute $s = (m - xr)k^{-1} \pmod{p-1}$. 
        \item If $s = 0$, then go back to (i); otherwise, output $\sigma = [r, s]$. 
    \end{enumerate}

    \item \textbf{Signature verification algorithm.} 
\end{enumerate}