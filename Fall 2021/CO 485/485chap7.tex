\section{Digital Signature Schemes}
Cryptographic digital signatures aim to transfer our conventional handwritten 
signatures into some electronic form. The man-in-the middle attack to the 
Diffie-Hellman protocol as described in Section 6.2.2 motivates the need for 
cryptographic signatures. If Alice was able to generate a signature $\sigma$ 
for her message $g^\alpha$ to Bob in the protocol, and Bob was able to 
verify Alice's signatures on her messages, then the adversary would not be 
able to mount a man-in-the-middle attack unless they were able to forge 
signatures on behalf of Alice. 

Notice that in the Diffie-Hellman protocol, Alice and Bob are not concerned 
about the confidentiality of the messages $g^\alpha$ and $g^\beta$ being 
sent over the channel, but they are worried about the \emph{integrity} and 
\emph{authentication} of the messages they receive. In other words, Bob 
should be able to verify 
\begin{enumerate}[(i)]
    \item whether the message he received was indeed originated from Alice; and 
    \item whether the message he received from Alice has been altered by 
          other parties.
\end{enumerate}
Alice should be able to do the same for Bob's messages. 

We have previously covered the RSA and ElGamal public key encryption schemes. 
Now, we will present the RSA and ElGamal public key digital signature schemes. 
Practical applications require signing long messages which after being encoded as 
integers would exceed the size of the underlying cryptographic parameters. 
The RSA modulus $N$ and the ElGamal group parameter $p$ are examples of this. 
Therefore, signature schemes generally apply a cryptographic hash function 
$H$ to a message $m$ before signing it. In particular, $H : \{0, 1\}^* 
\to \{0, 1\}^n$ is a function that maps strings of arbitrary length into 
\emph{small} fixed length strings. For security and efficiency reasons, 
$H$ must satisfy the following properties. 
\begin{enumerate}
    \item \textbf{Efficiently computable.} Given $H$ and a message $m \in \{0, 1\}^*$, 
          it should be efficient to compute $H(m)$. 
    \item \textbf{Preimage resistant.} Given a hash value $h \in \{0, 1\}^n$ 
          chosen uniformly at random, it should be computationally infeasible 
          to find any preimage $m \in \{0, 1\}^*$ such that $H(m) = h$. 
    \item \textbf{Second preimage resistant.} Given an input value $m_1 \in \{0, 1\}^*$
          chosen uniformly at random, it should be computationally infeasible 
          to find a second preimage $m_2 \in \{0, 1\}^*$ such that $H(m_1) = H(m_2)$. 
    \item \textbf{Collision resistant.} It should be computationally infeasible 
          to find inputs $m_1, m_2 \in \{0, 1\}^*$ such that $m_1 \neq m_2$ and 
          $H(m_1) = H(m_2)$. Such a pair $(m_1, m_2)$ is called a \textbf{collision} 
          for $H$. 
\end{enumerate}
NIST specifies a list of standardized hash algorithms (SHA); among them is 
the commonly deployed hash algorithm SHA-$256$, which takes a string of 
arbitrary length as input and outputs a $256$-bit string. For a full list of 
SHA and the current NIST policy on hash functions, see 
\href{https://csrc.nist.gov/Projects/Hash-Functions/NIST-Policy-on-Hash-Functions}
{NIST Policy on Hash Functions}. 

\begin{remark}
    We mentioned above that a cryptographic hash function is applied to messages 
    before they are signed. In standardized signature schemes, messages go 
    through additional preprocessing such as encoding, padding, and 
    randomization. In this course, we will present textbook versions of 
    signature schemes for simplicity, and skip details of preprocessing 
    messages before signing. For example, in the RSA signature scheme, 
    we will assume that messages to be signed already belong to $\Z_N$. 
    We refer to \href{https://nvlpubs.nist.gov/nistpubs/FIPS/NIST.FIPS.186-4.pdf}
    {FIPS 186-4} for further details on some standardized
    signature schemes. 
\end{remark}

\subsection{The RSA Signature Scheme}
The RSA signature scheme consists of the following three algorithms. 
\begin{enumerate}
    \item \textbf{Key generation.} This is exactly the same key generation 
    algorithm as in the RSA encryption scheme. The public key and secret 
    key are $\PubKey = [N, e]$ and $\SecKey = [p, q, d]$, where 
    $N = pq$ is a product of two randomly chosen distinct primes $p$ and $q$, 
    and $e, d \in (1, \phi)$ satisfy $\gcd(e, \phi) = 1$ and 
    $ed \equiv 1 \pmod\phi$ where $\phi = (p-1)(q-1)$. 

    \item \textbf{Signature generation algorithm.} For a given public modulus $N$ 
    and the secret exponent $d$, the signature generation algorithm takes a 
    message $m$ as input, and outputs the signature $\sigma = m^d \in \Z_N$. 
    We can denote this process with 
    \begin{align*}
        \Sig_{N,d} : \Z_N &\to \Z_N \\ 
        m &\mapsto \sigma = m^d \pmod N, 
    \end{align*}
    or simply by $\Sig(m) = m^d \pmod N$ when $N$ and $d$ are clear from the context. 

    \item \textbf{Signature verification algorithm.} For a given public key 
    $\PubKey = [N, e]$, the signature verification algorithm takes a message 
    $m \in \Z_N$ and a signature $\sigma \in \Z_N$ as input, and outputs 
    $\True$ (indicating the validity of the message) if and only if 
    $m = \sigma^e \pmod N$. That is, we write $\Ver_{N,e}(m, \sigma) 
    = \True$ if and only if $m = \sigma^e$. 
\end{enumerate}

\begin{exmp}
    Suppose that the public key is $\PubKey = [49163, 20771]$ and the secret key is 
    $\SecKey = [233, 211, 36971]$. We can generate a signature for the message 
    $m = 37917$ by 
    \[ \sigma = m^d \text{ (mod $N$)} = 37917^{36971} \text{ (mod $49163$)} = 123. \] 
    The validity of the signature $\sigma = 123$ on the message $m = 37917$ can be 
    verified because given the public key $\PubKey = [N, e] = [49163, 20771]$, we have 
    \[ \sigma^e \text{ (mod $N$)} = 123^{20771} \text{ (mod $49163$)} = 37917 = m. \] 
\end{exmp}

\subsection{The ElGamal Signature Scheme}
As with the RSA signature scheme, the ElGamal signature scheme consists of three algorithms. 
\begin{enumerate}
    \item \textbf{Key generation.} The purpose of this algorithm is to generate 
    a public key and secret key pair, where the secret key is used to generate 
    signatures and the public key is used to verify signatures on a given message. 
    The public key is a tuple 
    \[ \PubKey = [p, g, y], \] 
    where $p$ is prime, $g$ is a generator for $\Z_p^*$, and $y = g^x$ for some 
    integer $x \in [1, p-1]$. The integer $x$ is chosen uniformly at random, 
    and is the secret key; that is, we have 
    \[ \SecKey = [x]. \] 

    \item \textbf{Signature generation algorithm.} For a given public key 
    $\PubKey = [p, g, y]$ and the secret key $x$, the signature generation 
    algorithm takes a message $m \in \Z_p^*$ as input and generates a 
    signature $\sigma = [r, s]$ as follows. 
    \begin{enumerate}[(i)]
        \item Select an integer $k \in \Z_{p-1}^*$ uniformly at random, and notice 
        that $\gcd(k, p-1) = 1$. 
        \item Compute $r = g^k \pmod p$. 
        \item Compute $s = (m - xr)k^{-1} \pmod{p-1}$. 
        \item If $s = 0$, then go back to (i); otherwise, output $\sigma = [r, s]$. 
    \end{enumerate}

    \item \textbf{Signature verification algorithm.} For a given public key 
    $\PubKey = [p, g, y]$, the signature verification algorithm takes a message 
    $m \in \Z_p^*$ and a signature $\sigma = [r, s]$ as input, and outputs 
    $\True$ (indicating the validity of the message) if and only if 
    $r$ is non-zero and $g^m \equiv y^r r^s \pmod p$. Note that if the 
    signature $\sigma = [r, s]$ on a message $m$ is properly generated, then 
    the verification step would output $\True$ as expected since 
    $s = (m - xr)k^{-1} \pmod{p-1}$ gives $m \equiv xr + ks \pmod{p-1}$, and hence 
    \[ g^m \equiv g^{xr+ks} \equiv y^r r^s \pmod p. \] 
\end{enumerate}

\begin{exmp}
    Suppose the public key is $\PubKey = [p, g, y] = [3967526699, 2, 729342158]$ 
    and the secret key is $\SecKey = [x] = [596305913]$. We can generate a 
    signature on the message $m = 37917$ as follows. 
    \begin{enumerate}[(i)]
        \item Select $k = 1094081451$. 
        \item Compute $r = g^k \pmod p = 989111823$. 
        \item Compute $s = (m - xr)k^{-1} = 390136376$. 
        \item Output $\sigma = [r, s] = [989111823, 390136376]$. 
    \end{enumerate}
    We can verify the validity of the signature $\sigma = [r, s] 
    = [989111823, 390136376]$ on the message $m = 37917$ because given the 
    public key $\PubKey = [p, g, y] = [3967526699, 2, 729342158]$, we have 
    \[ y^r r^s \text{ (mod $p$)} = 739342158^{989111823} \cdot 989111823^{390136376}
    \text{( mod $3967526699$)} = 2621047120 = g^m. \] 
\end{exmp}

\subsection{The Digital Signature Algorithm}
An ElGamal digital signature is a pair of integers $[r, s]$, where $r$ is a 
non-zero integer modulo $p$ and $s$ is a non-zero integer modulo $p-1$. The 
digital signature algorithm (DSA) offers an improvement over the ElGamal signatures 
by working over the subgroup of $\Z_p^*$ of order $q$, where $q$ is a prime 
divisor of $p-1$, with $q$ much smaller than $p$. As a result, a DSA signature 
consists of a pair of integers $[r, s]$ with $r, s \in [1, q-1]$. For $128$-bit 
security, it is suggested (for example, \href{https://www.keylength.com/en/4/}
{BlueKrypt Cryptographic Key Recommendation}) to choose $p$ as a $3072$-bit prime, 
and $q$ as a $256$-bit prime. This choice yields $2 \cdot 256 = 512$-bit DSA 
signatures, which are much shorter than the $2 \cdot 3072 = 6144$-bit ElGamal 
signatures. Otherwise, DSA is very similar to the ElGamal signature scheme. 
\begin{enumerate}
    \item \textbf{Key generation.} The purpose of this algorithm is to generate a 
    public key and secret key pair, where the secret key is used to generate signatures, 
    and the public key is used to verify signatures on a given message. The 
    public key is a tuple 
    \[ \PubKey = [p, q, g, y], \] 
    where $p$ and $q$ are primes such that $q \mid (p-1)$, $g$ is a generator of the 
    subgroup of $\Z_p^*$ of order $q$, and $y = g^x$ for some integer $x \in [1, 
    q-1]$. The integer $x$ is chosen uniformly at random, and is the secret key; 
    that is, we have 
    \[ \SecKey = [x]. \] 

    \item \textbf{Signature generation algorithm.} For a given public key 
    $\PubKey = [p, q, g, y]$ and the secret key $x$, the signature generation 
    algorithm takes a message $m \in \Z_q^*$ as input and generates a signature 
    $\sigma = [r, s]$ as follows. 
    \begin{enumerate}[(i)]
        \item Select an integer $k \in \Z_q^*$ uniformly at random.
        \item Compute $r = (g^k \pmod p) \pmod q$. 
        \item If $r = 0$, then go back to (i); otherwise, continue to (iv). 
        \item Compute $s = (m + xr)k^{-1} \pmod q$. 
        \item If $s = 0$, then go back to (i); otherwise, output $\sigma = [r, s]$. 
    \end{enumerate}

    \item \textbf{Signature verification algorithm.} For a given public key 
    $\PubKey = [p, q, g, y]$, the signature verification algorithm takes a 
    message $m \in \Z_q^*$ and a signature $\sigma = [r, s]$ as input.
    It outputs $\True$ (indicating the validity of the message) if and only if 
    $r$ and $s$ are both non-zero, and $v = r$, where $v$ is computed as follows. 
    \begin{enumerate}[(i)]
        \item Compute $w = s^{-1} \pmod q$. 
        \item Compute $u_1 = m \cdot w \pmod q$ and $u_2 = r \cdot w \pmod q$. 
        \item Compute $v = (g^{u_1} y^{u_2} \pmod p) \pmod q$. 
    \end{enumerate}
    If the signature $\sigma = [r, s]$ on a message $m$ is properly generated, 
    then the verification step outputs $\True$ as expected since 
    $s = (m + xr)k^{-1} \pmod q$ implies $k \equiv ms^{-1} + xrs^{-1} \pmod q$. 
    Then we have 
    \[ g^k \equiv g^{ms^{-1} + xrs^{-1}} \equiv g^{u_1} y^{u_2} \pmod p, \] 
    and it follows that 
    \[ r = (g^k \text{ (mod $p$)}) \text{ (mod $q$)} = (g^{u_1} y^{u_2} 
    \text{ (mod $p$)}) \text{ (mod $q$)} = v. \]  
\end{enumerate}

\subsection{Security of Digital Signature Schemes}
Suppose that Alice is a user with a public key and secret key pair 
$(\PubKey, \SecKey)$ who generates digital signatures using her secret key 
$\SecKey$. When analyzing the security of digital signature schemes, we allow 
computationally bounded adversaries to interact with Alice to achieve one 
of the following goals. 
\begin{enumerate}
    \item \textbf{Total break.} The adversary recovers Alice's secret key 
    $\SecKey$, so they can forge Alice's signature on any message. 
    \item \textbf{Universal forgery.} The adversary does not necessarily 
    recover Alice's secret key $\SecKey$, but discovers a method to forge 
    Alice's signatures. 
    \item \textbf{Selective forgery.} The adversary discovers a method to 
    forge Alice's signatures for a selected subset of messages of their choice, 
    which have not been previously signed by Alice. 
    \item \textbf{Existential forgery.} The adversary discovers a method 
    to forge Alice's signature for a single message of their choice. 
\end{enumerate}
Clearly, we see that a total break implies a universal forgery, which implies 
a selective forgery, which then implies an existential forgery. Adversaries 
can also differ in their interactions with Alice, and we categorize them as follows. 
\begin{enumerate}
    \item \textbf{Key-only attack.} The adversary only knows Alice's public key. 
    \item \textbf{Known-message attack.} The adversary knows some message and 
    signature pairs, where the messages are not selected by the adversary. 
    \item \textbf{Chosen-message attack.} The adversary chooses a subset of messages 
    and obtains Alice's signatures on these messages. 
    \item \textbf{Adaptive chosen-message attack.} This is the same as the  
    chosen-message attack, except the adversary can choose their messages 
    depending on the previous signatures they obtain from Alice. 
\end{enumerate}
We say that a signature scheme is \textbf{secure} if the strongest adversary 
cannot achieve the weakest of goals. More formally, we say that a signature 
scheme is secure if it is existentially unforgeable by a computationally
bounded adversary that is mounting an adaptive chosen-message attack. 

\begin{exmp}
    We consider existential forgery against the textbook RSA signature scheme.
    Suppose that Alice has public key $\PubKey = [N, e]$ and secret key 
    $\SecKey = [p, q, d]$, and that a computationally bounded adversary 
    knows Alice's public key. The adversary chooses a message $m \in \Z_N^*$ 
    and constructs $m_1, m_2 \in \Z_N^*$ such that $m \equiv m_1 m_2 \pmod N$
    and $m \neq m_i$ for $i = 1, 2$. Next, the adversary runs a chosen-message 
    attack and asks Alice to sign $m_1$ and $m_2$. Alice generates two signatures 
    $\sigma_1 = m_1^d \pmod N$ and $\sigma_2 = m_2^d \pmod N$ and sends them to 
    the adversary. The adversary outputs $\sigma = \sigma_1 \sigma_2 \pmod N$ 
    as a forgery of Alice's signature on the message $m$. Note that $\sigma$ 
    is a valid signature on $m$ since 
    \[ \sigma^e \equiv \sigma_1^e \sigma_2^e \equiv m_1^{ed} m_2^{ed} 
    \equiv m_1 m_2 \equiv m \pmod N. \] 
\end{exmp}

Fortunately, the textbook RSA signature scheme can be modified in such a way 
that one can prove the modified signature scheme is secure under reasonable 
assumptions. This modified scheme is called the RSA full-domain hash signature 
scheme (RSA-FDH), where a cryptographic hash function $H : \{0, 1\}^* 
\to \Z_N^*$ is applied to a message $m$ before it gets signed. In particular, 
the signature on $m$ is now calculated as $\sigma = H(m)^d \pmod N$. 
Bellare and Rogaway \cite{10.1007/3-540-68339-9_34} proved in 1996 that if the 
\href{https://en.wikipedia.org/wiki/RSA_problem}{RSA problem} is hard and 
$H$ is modeled as a \href{https://en.wikipedia.org/wiki/Random_oracle}{random 
oracle}, then RSA-FDH is secure. 

\begin{exmp}
    Now, let's consider existential forgery against the textbook ElGamal signature 
    scheme. Suppose that Alice has public key $\PubKey = [p, g, y]$ and 
    secret key $\SecKey = [x]$, and that a computationally bounded adversary 
    knows Alice's public key. The adversary selects an integer $e \in [1, p-1]$ 
    and sets $r = g^e y \pmod p$, $s = -r \pmod{p-1}$, and $m = es \pmod{p-1}$. 
    Notice that this is a successful forgery since 
    \[ g^m \equiv g^{es} \equiv y^{-s}(g^e y)^s \equiv y^r r^s \pmod p, \] 
    and hence $[r, s]$ forms a valid signature on the message $m$. 
\end{exmp}

As with the RSA signature scheme, applying a cryptographic hash function to messages 
before they get signed prevents the attack described above. Regarding the security 
of the ElGamal signature scheme and DSA in general, an adversary can successfully 
forge signatures if they can solve $\DLP$ in the underlying (sub)groups. In fact, 
it appears that the only way to forge signatures in these schemes is to solve 
$\DLP$ (although there is no known proof of this statement). However, it is 
worth noting that the ElGamal signature scheme and DSA can be severely attacked 
if the protocols are not properly implemented; for instance, one could fail 
to choose $k$ independently and uniformly at random, or fail to check the
non-zero conditions on $r$ and $s$. 