\section{Elliptic Curve Cryptography}
In Sections 6 and 7, we have described some discrete logarithm based cryptosystems.
Generic descriptions of these cryptosystems assume a cyclic group $G$ with some 
basic assumptions, namely that the elements of $G$ can be represented efficiently,
the group operation in $G$ can be performed efficiently, but $\DLP$ in $G$ is 
intractable. We have also worked with discrete logarithm based cryptosystems 
using the multiplicative group $\Z_p^*$. This allowed us to realize a 
practical implementation of the protocols and quantify the security of the 
systems. In particular, the index calculus algorithm (with subexponential
complexity) together with Pollard's $\rho$-algorithm (with square root 
complexity) solves $\DLP$ in $\Z_p^*$ by working with the $256$-bit prime 
order subgroup of $\Z_p^*$, where $p$ is a $3072$-bit prime to ensure 
$128$-bit security of DSA. 

In this section, we introduce elliptic curves over finite fields whose points 
form an additive abelian group, which can be used to instantiate discrete 
logarithm based cryptosystems. In fact, elliptic curve groups offer more 
efficient implementations for key agreement, encryption, and digital signature 
algorithms because the best algorithms to solve $\DLP$ in carefully chosen 
elliptic curve groups have square root complexity, as opposed to the algorithms 
with subexponential complexity that solve $\DLP$ in $\Z_p^*$. 

Let $\K$ be a field. For simplicity, we will assume that $\K$ is not of 
characteristic $2$ or $3$. An elliptic curve $E$ over $\K$ is defined by 
a set of the form 
\[ E/\K = \{(x, y) \in \K \times \K : y^2 = x^3 + ax + b\} \] 
for some $a, b \in \K$ such that the cubic polynomial $x^3 + ax + b$
has no repeated roots in the algebraic closure $\overline{\K}$ of $\K$.
Given an elliptic curve $E/\K$ and a field $\bbL$ as an extension of $\K$
(which might be $\bbL = \K$), the set of points in $E$ with coordinates 
in $\bbL$ is defined as 
\[ E(\bbL) = \{(x, y) \in \bbL \times \bbL : y^2 = x^3 + ax + b\} \cup \{\cal O\}, \] 
where ${\cal O}$ is an extra auxiliary point. This point is sometimes 
denoted as $\infty$ because it actually corresponds to the point at infinity 
$[0, 1, 0]$ when $E$ is considered in the projective space through its equation 
\[ Y^2 Z = X^3 + aXZ + bZ^3 \] 
with $a, b \in \K$. In this course, we choose the notation ${\cal O}$ because 
elliptic curve points form an additive abelian group with ${\cal O}$ 
being the identity element, as we will see later. 

\begin{exmp}
    Recall that $\C$ is the algebraic closure of $\R$. 
    \begin{enumerate}[(1)]
        \item $E/\R = \{(x, y) \in \R \times \R : y^2 = x^3 - x\} \cup \{\infty\}$
        is an elliptic curve over $\R$ since 
        \[ x^3 - x = x(x-1)(x+1) \] 
        has three distinct roots in $\C$. In fact, all three roots are in $\R$. 
        \item $E/\R = \{(x, y) \in \R \times \R : y^2 = x^3 + x\} \cup \{\infty\}$
        is an elliptic curve over $\R$ since
        \[ x^3 + x = x(x-i)(x+i) \] 
        has three distinct roots in $\C$. 
        \item $E/\R = \{(x, y) \in \R \times \R : y^2 = x^3\} \cup \{\infty\}$
        is not an elliptic curve over $\R$ since $x^3$ has the repeated root 
        $0$ in $\C$. 
        \item $E/\R = \{(x, y) \in \R \times \R : y^2 = x^3 - (1/3)x + 
        2/27\} \cup \{\infty\}$ is not an elliptic curve over $\R$ since 
        \[ x^3 - \frac13x + \frac{2}{27} = \left(x - \frac13\right)^{\!2}\left(x + \frac23\right) \] 
        has a repeated root in $\C$.
    \end{enumerate}
\end{exmp}

\begin{remark}
    Let $\K$ be a field whose characteristic is not $2$ or $3$. Let 
    $a, b \in \K$. One can show that if 
    \[ x^3 + ax + b = (x - r_1)(x - r_2)(x - r_3) \] 
    in $\overline{\K}$, then $[(x - r_1)(x - r_2)(x - r_3)]^2 = -(4a^3 + 27b^2)$. 
    In particular, we see that 
    \[ E/\K = \{(x, y) \in \R \times \R : y^2 = x^3 - x\} \] 
    is an elliptic curve if and only if $4a^3 + 27b^2 \neq 0$. 
\end{remark}