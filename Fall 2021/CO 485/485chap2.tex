\section{Some Number Theory}

\subsection{Modular Arithmetic}
We begin with some notation and definitions. The set of integers is denoted by $\Z$. For an 
integer $n \geq 2$, we define the sets 
\begin{align*}
    \Z_n &= \{a \in \Z : 0 \leq a < n\}, \\ \Z_n^* &= \{a \in \Z_n : \gcd(a, n) = 1\}. 
\end{align*}
For example, we have 
\begin{align*}
    \Z_{15} &= \{0, 1, 2, 3, 4, 5, 6, 7, 8, 9, 10, 11, 12, 13, 14\}, \\
    \Z_{15}^* &= \{1,2,4,7,8,11,13,14\},
\end{align*}
where we exclude all multiples of $3$ and $5$ in the latter set. 
For $a, b \in \Z_n$, we define the operations 
\begin{align*}
    a \oplus b &= a + b \pmod n, \\
    a \odot b &= a \cdot b \pmod n.
\end{align*}
For instance, given $n = 15$, one can check that $11 \oplus 13 = 9$ and $11 \odot 13 = 8$. 
Most of the algebraic operations we work with in this course are modular operations, so we will simply 
write $+$ and $\cdot$ instead of $\oplus$ and $\odot$ when it is clear from the context 
whether $+$ and $\cdot$ are regular or modular operations. 
Notice that when $a, b \in \Z_n$, we have $a + b \in \Z_n$ and $a \cdot b \in \Z_n$; moreover, 
for $a, b \in \Z_n^*$, we have $a \cdot b \in \Z_n^*$. In fact, elements in $\Z_n$ and 
$\Z_n^*$ satisfy a larger set of properties, known as the group axioms, which we will 
formally define in the following definition.

\begin{defn}
A {\bf group} $(\G, *)$ is a non-empty set $\G$ together with a binary operation $*$ satisfying the 
following properties:
\begin{enumerate}[(1)]
    \item $\G$ is {\bf closed}: For all $a, b \in \G$, we have $a * b \in \G$. 
    \item $\G$ is {\bf associative}: For all $a, b, c \in \G$, we have $(a * b) * c = a * (b * c)$. 
    \item $\G$ has an {\bf identity}: There exists $e \in \G$ such that $a * e = e * a = a$ for all $a \in \G$.
    \item Elements in $\G$ are {\bf invertible}: For all $a \in \G$, there exists $a^{-1} \in \G$ 
    such that $a * a^{-1} = a^{-1} * a = e$. 
\end{enumerate}
\end{defn}

As we noted before, $(\Z_n, +)$ is a group with identity $0$ and $(\Z_n^*, \cdot)$ is a group 
with identity $1$. However, $\Z_n$ is not a group with respect to multiplication 
because $0$ is not invertible, and 
$\Z_n^*$ is not a group with respect to addition because there is no identity element $0$. 

Finally, for a finite group $\G$, the {\bf order} of $\G$ is the number of elements in 
$\G$, and is denoted by $|\G|$. 

\subsection{GCD and Modular Inverses}
Proving the existence of inverses is interesting, but what about finding the inverse of a particular
element in a group? Finding inverses of elements in $(\Z_n, +)$ is easy: simply take the 
additive inverse $-a \pmod n$ of $a \in \Z$. On the other hand, finding the multiplicative 
inverse of an element in $(\Z_n^*, \cdot)$ takes some more work. In particular, we care about 
efficient algorithms to compute modular inverses because the secret exponent $d$ in the RSA 
encryption scheme (as in Section 1.2) is the multiplicative inverse of $e$ modulo $\phi$, where 
$e$ is the public key exponent and $\phi$ is the secret modulus. Fortunately, modular multiplicative
inverses can be computed using extended Euclidean type algorithms, and we describe one below. 

\newpage 
\begin{algo}[Modular Multiplicative Inverses]~

{\bf Input:} $n \geq 2$ and $a \in \Z_n^*$.

{\bf Output:} $b \in \Z_n^*$ such that $a \cdot b \equiv 1 \pmod n$. 
\begin{enumerate}
    \item Set the initial state $t_a = a$, $t_n = n$, $u = [u_0, u_1] = [1, 0]$, $v = [v_0, v_1] = [0, 1]$. 
    \item {\bf if $t_n > t_a$:}
    \begin{itemize}
        \item Use the division algorithm to write $t_n = qt_a + r$ for some 
        $q \geq 0$ and $0 \leq r < t_a$.
        \item Update the state: $t_n \gets r$, $v \gets v - q \cdot u$.
    \end{itemize}
    {\bf else if $t_a > t_n$:}
    \begin{itemize}
        \item Use the division algorithm to write $t_a = qt_n + r$ for some 
        $q \geq 0$ and $0 \leq r < t_n$.
        \item Update the state: $t_a \gets r$, $u \gets u - q \cdot v$. 
    \end{itemize}
    \item {\bf if $t_n = 1$:}
    \begin{itemize}
        \item Set $b = v_0 \pmod n$ and output $b$.
    \end{itemize}
    {\bf else if $t_a = 1$:} 
    \begin{itemize}
        \item Set $b = u_0 \pmod n$ and output $b$.
    \end{itemize}
    {\bf else:}
    \begin{itemize}
        \item Go back to step 2. 
    \end{itemize}
\end{enumerate}
\end{algo}

Notice that the input $(a, n)$ of Algorithm 2.2 assumes that $\gcd(a, n) = 1$ since $a \in \Z_n^*$. 
Therefore, on a more general input $(a, n)$ with $a \in \Z_n$, one first has to check that 
$\gcd(a, n) = 1$ is satisfied before Algorithm 2.2 can be run. We can address this nuisance with 
some minor modifications to the algorithm.

\begin{algo}[GCD and Modular Multiplicative Inverses]~

{\bf Input:} $n \geq 2$ and $0 \neq a \in \Z_n$.

{\bf Output:} $\gcd = \gcd(a, n)$, and if $\gcd = 1$, then $b \in \Z_n^*$ such that $a \cdot b \equiv 1 \pmod n$. 
\begin{enumerate}
    \item Set the initial state $t_a = a$, $t_n = n$, $u = [u_0, u_1] = [1, 0]$, $v = [v_0, v_1] = [0, 1]$. 
    \item {\bf if $t_n > t_a$:}
    \begin{itemize}
        \item Use the division algorithm to write $t_n = qt_a + r$ for some 
        $q \geq 0$ and $0 \leq r < t_a$.
        \item Update the state: $t_n \gets r$, $v \gets v - q \cdot u$.
    \end{itemize}
    {\bf else if $t_a > t_n$:}
    \begin{itemize}
        \item Use the division algorithm to write $t_a = qt_n + r$ for some 
        $q \geq 0$ and $0 \leq r < t_n$.
        \item Update the state: $t_a \gets r$, $u \gets u - q \cdot v$. 
    \end{itemize}
    \item {\bf if $t_a = 0$:}
    \begin{itemize}
        \item {\bf if $t_n = 1$:} 
        \begin{itemize}[$\circ$]
            \item Set $\gcd = 1$, $b = v_0 \pmod n$ and output $(\gcd, b)$.
        \end{itemize}
        {\bf else:} 
        \begin{itemize}[$\circ$]
            \item Set $\gcd = t_n$ and output $\gcd$. 
        \end{itemize}
    \end{itemize}
    {\bf else if $t_n = 0$:}
    \begin{itemize}
        \item {\bf if $t_a = 1$:} 
        \begin{itemize}[$\circ$]
            \item Set $\gcd = 1$, $b = u_0 \pmod n$ and output $(\gcd, b)$.
        \end{itemize}
        {\bf else:} 
        \begin{itemize}[$\circ$]
            \item Set $\gcd = t_a$ and output $\gcd$.  
        \end{itemize}
    \end{itemize}
    {\bf else}
    \begin{itemize}
        \item Go back to step 2. 
    \end{itemize}
\end{enumerate}
\end{algo}

\newpage
\begin{lemma}[Facts about Algorithm 2.3]~
\begin{enumerate}[(1)]
    \item Algorithm 2.3 terminates with $t_a = 0$ or $t_n = 0$. 
    \item $\gcd(t_a, t_n)$ is invariant throughout Algorithm 2.3.
    \item We have $u \cdot (a, n) = u_0a + u_1n = t_a$ and $v \cdot (a, n) = v_0a + v_1n = t_n$
    throughout Algorithm 2.3. 
\end{enumerate}
\end{lemma}
\begin{pf}
We leave the proof as an exercise. 
\end{pf}

\begin{thm}[Correctness of Algorithm 2.3]
Algorithm 2.3 terminates and its output is correct.
\end{thm}
\begin{pf}
By Lemma 2.4, we know that Algorithm 2.3 terminates with $t_a = 0$ or $t_n = 0$. Assume without 
loss of generality that it terminates with $t_a = 0$. In the beginning of the algorithm,
we set $t_a = a$ and $t_n = n$. Since $\gcd(t_a, t_n)$ is invariant throughout the algorithm by 
Lemma 2.4, we see that 
\[ \gcd(a, n) = \gcd(t_a, t_n) = \gcd(0, t_n) = t_n, \]
which shows that the first output $\gcd = t_n$ is correct. Furthermore, if $t_n = 1$ at the 
end of the algorithm, then Lemma 2.4 shows that 
\[ v \cdot (a, n) = v_0a + v_1n = t_n = 1, \]
and this implies that
\[ v_0 \cdot a \equiv 1 \pmod n. \]
Therefore, the second output $b = v_0 \pmod n$ is also correct, completing the proof.
\end{pf}

Algorithm 2.3 can be used to calculate the public key and secret key pair $(e, d)$ in the RSA 
encryption scheme. For a given RSA public modulus $N$ and secret modulus $\phi$, choose 
$e \in \Z_N$ and call Algorithm 2.3 with input $a = e$ and $n = \phi$. Repeat this until 
$\gcd = \gcd(e, \phi) = 1$ and then set the secret exponent as $d = b$, where $\gcd$ and $b$ are the 
outputs given by Algorithm 2.3. 

\subsection{Modular Exponentiation}
Let $e$ and $n\geq 2$ be integers. Let $a \in \Z_n$. Given a base $a$ and an exponent $e$, 
a modular exponential algorithm computes $a^e \in \Z_n$. We have already seen an application of modular
exponential in the RSA encryption scheme; we compute $m^e \in \Z_N$ to encrypt a message $m$, 
and compute $c^d \in \Z_N$ to decrypt a ciphertext $c$, where $e$, $d$, and $N$ are the public 
exponent, secret exponent, and public modulus, respectively. Moreover, we will see later 
that modular exponentiation is the main operation in several other cryptographic schemes. 
These include Diffie-Hellman key exchange, the elliptic curve digital signature algorithm, 
and isogeny-based cryptosystems. This gives us a lot of motivation to design and implement 
efficient modular exponentiation algorithms. 

First, we introduce some notation. We assume that exponents are positive integers unless 
otherwise stated. 
\begin{itemize}
    \item We denote the {\bf binary representation} of an $\ell$-bit integer $e = \sum_{i=0}^{\ell-1} e_i 2^i$
    where each $e_i \in \{0, 1\}$ and $e_{\ell-1} = 1$ by $(e_{\ell-1}\,e_{\ell-2}\,\cdots\,e_0)_2$. 
    Moreover, for an $\ell$-bit integer $e$, we define 
    \[ e[i : j]_2 := (e_i\,e_{i-1}\,\cdots\,e_j)_2 = \sum_{k=j}^i e_k 2^{k-j} \]
    for $0 \leq j \leq i \leq \ell-1$. Note that $e[\ell-1 : 0]_2 = e$ and $e[i : i] = e_i$.
    \item In some of the algorithms, $e$ will be represented in a more general form using a base 
    $b \geq 2$. More specifically, we write $e = \sum_{i=0}^{\ell-1} e_i b^i$ where each 
    $0 \leq e_i < b$ and $e_{\ell-1} \neq 0$. We call this the 
    {\bf $b$-ary representation} of $e$, and denote it by $e = (e_{\ell-1}\,e_{\ell-2}\,\cdots\,e_0)_b$.
    Note that the binary representation of $e$ is obtained by setting $b = 2$. 
    \item One may further relax the condition $0 \leq e_i < b$ and allow for a more general 
    digit set $D$ for the $e_i$ \cite{10.1145/322344.322355}. To be more specific, if $e$ can be written as 
    \[ e = \sum_{i=0}^{\ell-1} e_i b^i \]
    where each $e_i \in D$ for a digit set $D$, then we will still denote $e = (e_{\ell-1}\,e_{\ell-2}\,\cdots\,e_0)_b$ and extend our notation $e[i : j]_2$ to 
    \[ e[i : j]_b = (e_i\,e_{i-1}\,\cdots\,e_j)_b = \sum_{k=j}^i e_k b^{k-j}. \]
    If the base $b$ is clear from the context, we can drop it and simply write $e[i : j]$. 
    \item The complexity of some algorithms will depend on the {\bf weight} of the $b$-ary 
    representation of $e$ (the number of indices $i$ with $e_i \neq 0$), which we will denote by 
    $\w_b(a)$. 
    \item Digit sets may contain negative digits, and it will be convenient to denote a negative 
    digit $-d$ by $\overline{d}$. 
\end{itemize}

\begin{exmp}
The $2^w$-ary representations of $20771$ for $w = 1, 2, 3, 4$ are 
\begin{align*}
    20771 &= (1\,0\,1\,0\,0\,0\,1\,0\,0\,1\,0\,0\,0\,1\,1)_2 \\
    &= (1\,1\,0\,1\,0\,2\,0\,3)_4 \\
    &= (5\,0\,4\,4\,3)_8 \\
    &= (5\,1\,2\,3)_{16}
\end{align*}
A binary representation of $20771$ using the digit set $D = \{0, 1, \overline1, 3, \overline3\}$ 
and a $4$-ary representation of $20771$ using the digit set $D = \{1, \overline1, 3, \overline3\}$ are given by 
\begin{align*}
    20771 &= (1\,0\,0\,\overline3\,0\,0\,0\,1\,0\,0\,1\,0\,0\,0\,0\,3)_2 \\
    &= (1\,1\,1\,\overline3\,1\,\overline1\,\overline3\,\overline1)_4.
\end{align*}
To obtain the binary representation, we notice that 
\[ 20771 = 2^{15} - 3 \cdot 2^{12} + 2^8 + 2^5 + 3 \cdot 2^5. \]
Observe that the last representation does not use any $0$ digit and maximizes the weight 
$\w_4(e)$, whereas the second last representation is sparse with a relatively low weight $\w_2(e) = 5$.
\end{exmp}

The most widely known efficient method to perform exponentiation dates back to 200 BC and is called the 
{\bf square and multiply method}. Brauer \cite{bams/1183502136} generalized the square and multiply method using $b = 2^w$
representations of integers, where a set of elements $T = \{a_i = a^i : 1 \leq i < 2^w\}$ is 
precomputed and stored. We present this method in the following algorithm. The parameter 
$w$ used in the algorithm is also called the {\bf window size} because we iterate through $w$ 
bits at a time and we imagine placing (and moving) a window of size $w$ on the binary 
representation of the exponent. 

\begin{algo}[$2^w$-ary Square and Multiply Method]~

{\bf Input:} $w \in \Z$, $w \geq 1$, $b = 2^w$, $e = (e_{\ell-1}\,e_{\ell-2}\,\cdots\,e_0)_b$, 
$0 \leq e_i < 2^w$, $a \in \Z_n$, $n \in \Z$, $n \geq 2$.

{\bf Output:} $a^e \in \Z_n$. 
\begin{enumerate}
    \item Set $a_1 \gets a$, $a_i \gets a_{i-1} \cdot a$ for $2 \leq i \leq 2^w$ (we can ignore this when $w = 1$). 
    \item Set $t \gets 1$. 
    \item {\bf for $i = \ell - 1$ to $0$ by $-1$ do:}
    \begin{itemize}
        \item $t \gets t^{2^w}$ (this step requires $w$ successive squaring operations)
        \item {\bf if $e_i \neq 0$:} 
        \begin{itemize}[$\circ$]
            \item $t \gets t \cdot a_{e_i}$ (the multiply step; if $w = 1$, then $a_{e_i}$ is always $a$)
        \end{itemize}
    \end{itemize}
    \item Output $t$. 
\end{enumerate}
\end{algo}

\newpage
\begin{thm}[Correctness of Algorithm 2.7]
Algorithm 2.7 terminates and its output is correct.
\end{thm}
\begin{pf}
The proof follows from induction and observing that $e[\ell-1 : 0]_b = 2^w \cdot (e[\ell-1 : 1]_b) + e_0$. 
We leave the details as an exercise. 
\end{pf}

\subsection{Quadratic Residues}
We now introduce quadratic residues from number theory. This will prepare us for our next 
topics on primality testing and other cryptographic constructions such as random number generators 
and public key encryption algorithms. 

Let $p$ be an odd prime. Since the only integer $a \in \Z_p$ that does not satisfy $\gcd(a, p) = 1$ is 
$a = 0$, we have $\Z_p^* = \{1, 2, \dots, p-1\}$ and $|\Z_p^*| = p-1$. 

Recall that $\Z_p^*$ is a group under multiplication. An interesting fact in algebra states that 
there always exists an element $g \in \Z_p^*$ such that for all $c \in \Z_p^*$, there is a unique 
integer $1 \leq k \leq p-1$ such that $c = g^k$. In other words, for such an element $g$, we can write 
\[ \Z_p^* = \{g^k : 1 \leq k \leq p-1\}. \]
More generally, if a (multiplicative) group $\G$ has an element such that 
\[ \G = \{g^k : 1 \leq k \leq |\G|-1\}, \]
then we call $\G$ a {\bf cyclic group} generated by $g$, and we write $\G = \langle g \rangle$. 
Equivalently, we say that $g$ is a {\bf generator} of $\G$. We note that a cyclic group can 
have more than one generator, and that not every group is cyclic. Indeed, $\Z_5^* = \{1, 2, 3, 4\}$ 
and $\Z_8^* = \{1, 3, 5, 7\}$ are both groups of order $4$; we have that $\Z_5^* = \langle 2 
\rangle = \langle 3 \rangle$ is cyclic, whereas $\Z_8^*$ is not since $a^2 = 1$ for all $a \in \Z_8^*$. 

It can be shown that $\Z_n^*$ is a cyclic group if and only if $n$ is one of $2$, $4$, $p^k$, or 
$2p^k$ where $p$ is an odd prime and $k$ is a positive integer. 

For a given positive integer $n$, Euler's totient function, denoted $\phi$, counts the number of positive
integers $1 \leq a \leq n$ such that $\gcd(a, n) = 1$. That is, 
\[ \phi(n) = \#\{a \in \Z : \gcd(a, n) = 1,\, 1 \leq a \leq n\}. \]
By definition, we have $|\Z_n^*| = \phi(n)$, so it is of interest to us to compute $\phi(n)$. 
Notice that we have the following properties:
\begin{enumerate}[(1)]
    \item If $p$ is a prime and $k \geq 1$ is an integer, then $\phi(p^k) = p^k - p^{k-1}$. 
    \item If $m$ and $n$ are positive integers with $\gcd(m, n) = 1$, then 
    $\phi(m \cdot n) = \phi(m) \phi(n)$. 
\end{enumerate}
In particular, for two distinct primes $p$ and $q$, we know that $|\Z_p^*| = \phi(p) = p-1$ 
and $|\Z_q^*| = \phi(q) = q-1$. It follows that if $N = p \cdot q$, then 
\[ |\Z_N^*| = \phi(N) = \phi(p)\phi(q) = (p-1)(q-1). \]
This is exactly why we defined $\phi = (p-1)(q-1)$ in the RSA encryption scheme, where we have been 
practically working with $N$. 

We now recall some important definitions and facts about finite groups. 

\begin{prop}[Properties of Finite Groups]
Let $\G$ be a finite group with identity $1$. 
\begin{itemize}
    \item For all $g \in \G$, we have $g^{|\G|} = 1$. 
    \item The {\bf order} of an element $g \in \G$ is the smallest positive integer $s$ such that 
    $g^s = 1$, denoted by $\ord(g)$. 
    \item If $g^a = 1$ for some positive integer $a$, then $\ord(g) \mid a$. In particular, we see that 
    $\ord(g) \mid |\G|$. 
    \item If $\ord(g) = s$, then 
    \[ \ord(g^a) = \frac{s}{\gcd(a, s)}. \]
    In particular, we have $\ord(g) = \ord(g^a)$ if and only if $\gcd(a, \ord(g)) = 1$. 
    \item A cyclic group $\G = \langle g \rangle$ has $\phi(|\G|)$ generators, and the set of 
    generators is given by 
    \[ \{g^a : \gcd(a, |\G|) = 1\}. \]
    \item We have $g^a = g^b$ if and only if $a \equiv b \pmod{\ord(g)}$. 
\end{itemize}
\end{prop}

Let $p$ be a prime, and consider the linear equation 
\[ ax + b \equiv 0 \pmod p \]
for $a, b \in \Z_p$ with $a \neq 0$. This equation has a solution $x = -b \cdot a^{-1}$, and one can 
show that this solution is unique. Next, consider the quadratic equation 
\[ ax^2 + bx + c \equiv 0 \pmod p \]
where $a, b, c \in \Z_p$ with $a \neq 0$. Then we can consider two cases: either the equation 
has no solution in $\Z_p$ (for instance, take $p = 3$, $a = c = 1$, and $b = 0$), or 
it has at least one solution. Suppose we are in the second case, and let $s_1$ be a solution. Then 
we can write 
\begin{align*}
    ax^2 + bx + c &= (ax^2 + bx + c) - (as_1^2 + bs_1 + c) \\
    &= a(x^2 - s_1^2) + b(x - s_1) \\
    &= (x - s_1)(ax + as_1 + b). 
\end{align*}
This gives rise to another solution $s_2 = -(s_1 + ba^{-1}) \in \Z_p$ to the same equation. 
Therefore, we can either have $0$, $1$ (when $s_1 = s_2$), or $2$ (when $s_1 \neq s_2$) solutions in 
$\Z_p$. By setting $a = 1$, $b = 0$, and taking a non-zero $c \in \Z_p$, we conclude that 
\[ x^2 \equiv c \pmod p \]
has either no solution, or solution set $\{s, -s\}$ for some $s \in \Z_p^*$. Note that $s$ cannot 
be zero, and for $p$ an odd prime, $s$ and $-s$ are pairwise distinct. This motivates the 
definition of a quadratic residue modulo $p$. 

\begin{defn}[Quadratic Residues and Non-Residues]
Let $p$ be a prime. An element $c \in \Z_p^*$ is said to be a {\bf quadratic residue} modulo $p$ if 
the equation 
\[ x^2 \equiv c \pmod p \]
as a solution in $\Z_p^*$. If $c \in \Z_p^*$ is not a quadratic residue, we call $c$ a 
{\bf quadratic non-residue} modulo $p$. The set of all quadratic residues modulo $p$ is denoted by 
$\QR_p$, and the set of all quadratic non-residues modulo $p$ is denoted by $\QNR_p$. 
\end{defn}

The following theorem gives two characterizations of $\QR_p$. 

\begin{thm}[Characterizations of $\QR_p$]
Let $p$ be an odd prime and let $g$ be a generator of $\Z_p^*$. Let $c = g^k$ for some $1 \leq k 
\leq p-1$.
\begin{enumerate}[(1)]
    \item We have $c \in \QR_p$ if and only if $k$ is even. 
    \item We have $c \in \QR_p$ if and only if $c^{(p-1)/2} = 1$ (Euler's criterion). 
\end{enumerate}
\end{thm}
\begin{pf}
To prove (1), suppose that $c \in \QR_p$. By definition, there exists $x \in \Z_p^*$ such that 
$c = x^2 = g^{2\alpha}$, where the last equality follows because $g$ is a generator of 
$\Z_p^*$. Setting $k = 2\alpha$ proves the forward direction. For the converse, let $k$ be even. 
Then we can write $c = g^k = g^{2\alpha} = x^2$ for $x = g^\alpha$, finishing the proof of (1). 
The proof of (2) follows from (1) and Proposition 2.9, and we leave it as an exercise.
\end{pf}

Euler's criterion in Theorem 2.11 can be used to test whether an element $c \in \Z_p^*$ 
is in $\QR_p$ for an odd prime $p$. 