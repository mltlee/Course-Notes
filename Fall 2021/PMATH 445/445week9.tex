\section{Universal property, uniqueness of tensor products (11/10/2021)}
Let $R$ be a ring, let $M = M_R$ be a right $R$-module, and let $N = {}_R N$
a left $R$-module. Then $M \otimes_R N = T/U$ is an abelian group, with 
$T$ and $U$ as in the above definition. 
We write $m \otimes n$ to mean the coset $e_{(m, n)} + U$ in $M \otimes_R N$. 

\begin{remark}{}
    Not every element in $M \otimes_R N$ can be expressed in the form  
    $m \otimes n$ in general, but every element is a $\Z$-linear combination of 
    elements of the form $m \otimes n$. 
\end{remark}

We have a map 
\begin{align*}
    \pi : M \times N &\to M \otimes_R N \\ 
    (m, n) &\mapsto m \otimes n
\end{align*}  
which is bilinear and hence satisfies the property 
\[ \pi(m_1 + m_2, n) = (m_1 + m_2) \otimes n = 
m_1 \otimes n + m_2 \otimes n. \] 
This is the case since $e_{(m_1+m_2, n)} \equiv e_{(m_1, n)} 
+ e_{(m_2, n)} \pmod{U}$. Moreover, $e_{(mr, n)} \equiv 
e_{(m, rn)} \pmod{U}$, so $mr \otimes n = m \otimes rn$. 
We now consider the universal property of tensor products. 

\begin{prop}[Universal Property of Tensor Products]{}
    If $A$ is an abelian group and $\theta : M \times N \to A$ is a bilinear map, 
    then there exists a unique group homomorphism $\tilde\theta : M \otimes_R N 
    \to A$.
    \begin{center}
        \begin{tikzcd}
            M \times N \arrow[rr, "\theta"] \arrow[dd, "\pi"'] &  & A \\
                                                            &  &   \\
            M \otimes_R N \arrow[rruu, "\tilde\theta"']          &  &  
        \end{tikzcd}
    \end{center}
\end{prop}  

\begin{pf}
    Given a bilinear map $\theta : M \times N \to A$, define 
    $\hat\theta : T \to A$ by 
    \[ \hat\theta(e_{(m,n)}) = \theta(m, n). \] 
    This extends uniquely to a group homomorphism since the elements 
    $e_{(m,n)}$ generate $T$. We claim that $U \subseteq 
    \ker\hat\theta$. To see this, it suffices to show that 
    \begin{enumerate}[(i)]
        \item $\hat\theta(e_{(m_1+m_2a, n)} - e_{(m_1, n)} 
        - ae_{(m_2, n)}) = 0$; 
        \item $\hat\theta(e_{(m,n_1+an_2)} - e_{(m,n_1)} - 
        ae_{(m,n_2)}) = 0$; 
        \item $\hat\theta(e_{(mr, n)} + e_{(m, rn)}) = 0$. 
    \end{enumerate}
    We check (iii), and we leave (i) and (ii) as exercises. We have 
    \begin{align*}
        \hat\theta(e_{(mr, n)}) - e_{(m, rn)}) = 
        \theta(mr, n) - \theta(m, rn) = 0, 
    \end{align*}
    since $\theta$ is bilinear. Now, using the map $\psi : M \times N \to T$ 
    given by 
    \[ \psi(m, n) \mapsto e_{(m, n)}, \] 
    the map $\hat\theta : T \to A$ is unique. Since $U$ is in $\ker\hat\theta$, 
    we have $\hat\theta(t + u) = \hat\theta(t)$ for all $u \in U$. This allows 
    us to extend $\hat\theta$ uniquely to a map $\tilde\theta$ on the cosets; 
    that is, we have a map from $T/U = M \oplus_R N$ to $A$. Letting 
    $\psi' : T \to T/U$ be the canonical projection, we obtain the following 
    diagram. 
    \begin{center}
        \begin{tikzcd}
            M \times N \arrow[rr, "\theta"] \arrow[dd, "\psi"'] &  & A \\
                                                                &  &   \\
            T \arrow[rruu, "\hat\theta"'] \arrow[dd, "\psi'"']  &  &   \\
                                                                &  &   \\
            T/U \arrow[rruuuu, "\tilde\theta"']                 &  &  
        \end{tikzcd}
    \end{center}
    Setting $\pi = \psi' \circ \psi$ will give us the diagram above, 
    completing the proof. 
\end{pf}

\begin{prop}[Uniqueness of Tensor Products]{}
    We have the bilinear map 
    \begin{align*} 
        \pi : M \times N &\to M \otimes_R N \\ 
        (m, n) &\mapsto m \otimes n
    \end{align*}
    which satisfies the universal property. Suppose that $\Omega$ is 
    another abelian group with a bilinear map $\pi' : M \times N 
    \to \Omega$ which also has the universal property. Then 
    $M \otimes_R N \cong \Omega$. 
\end{prop}
\begin{pf}
    By the universal property of $\Omega$, there exists a unique group 
    homomorphism $\tilde\pi' : M \otimes_R N \to \Omega$ with $\tilde\pi' \circ \pi = \pi'$. 
    Similarly, by the universal property of $M \otimes_R N$, there 
    is a unique group homomorphism $\tilde\pi : \Omega \to M \otimes_R N$
    with $\tilde\pi \circ \pi' = \pi$. 
    \begin{center}
        \begin{tikzcd}
            M \times N \arrow[rr, "\pi'"] \arrow[dd, "\pi"'] &  & \Omega &  & M \times N \arrow[rr, "\pi"] \arrow[dd, "\pi'"'] &  & M \otimes_R N \\
                                                             &  &        &  &                                                  &  &               \\
            M \otimes_R N \arrow[rruu, "\tilde\pi'"']        &  &        &  & \Omega \arrow[rruu, "\tilde\pi"']                 &  &              
        \end{tikzcd}
    \end{center} 
    Clearly, there is a unique map $\id : \Omega \to \Omega$. 
    Using the maps $\tilde\pi : \Omega \to M \otimes_R N$ and 
    $\tilde\pi' : M \otimes_R N \to \Omega$., we obtain the following diagram. 
    \begin{center}
        \begin{tikzcd}
            M \times N \arrow[rr, "\pi'"] \arrow[dd, "\pi'"']    &  & \Omega                                  \\
                                                                 &  &                                         \\
            \Omega \arrow[rruu, "\id"'] \arrow[rr, "\tilde\pi"'] &  & M \otimes_R N \arrow[uu, "\tilde\pi'"']
        \end{tikzcd}
    \end{center} 
    Notice that 
    \[ (\tilde\pi' \circ \tilde\pi) \circ \pi' 
    = \tilde\pi' \circ (\tilde\pi \circ \pi') 
    = \tilde\pi' \circ \pi = \pi'. \] 
    Therefore, we get the following diagram. 
    \begin{center}
        \begin{tikzcd}
            M \times N \arrow[rr, "\pi'"] \arrow[dd, "\pi'"']  &  & \Omega \\
                                                               &  &        \\
            \Omega \arrow[rruu, "\tilde\pi' \circ \tilde\pi"'] &  &       
        \end{tikzcd}
    \end{center}
    But the map $\Omega \to \Omega$ is unique by the universal property, 
    and so we must have $\tilde\pi' \circ \tilde\pi = \id_\Omega$. By 
    an analogous argument, we find that $\tilde\pi \circ \tilde\pi' = 
    \id_{M \otimes_R N}$. This means that $\tilde\pi : \Omega 
    \to M \otimes_R N$ and $\tilde\pi' : M \otimes_R N \to \Omega$ are 
    inverses. In particular, it follows that these maps are isomorphisms. 
\end{pf}

\begin{exmp}{}
    Let $R = \Z$, $M = \Q$, and $N = \Z/3\Z$. Note that 
    $\Q \otimes_{\Z} \Z/3\Z = (0)$. Indeed, in this case, we have 
    \[ T = \bigoplus_{(a, b) \in \Q \times \Z/3\Z} \Z e_{(a, b)}. \] 
    Consider $a \otimes [1]_3$. We have 
    \[ a \otimes [1]_3 = \frac{a}3 \cdot 3 \otimes [1]_3 
    = \frac{a}3 \otimes 3 \cdot [1]_3 = \frac{a}3 \otimes 0
    = \frac{a}3 \otimes 0 \cdot 0 = 0 \otimes 0 = 0. \] 
    To see that $0 \otimes 0 = 0$, note that by bilinearity, we have 
    $0 \otimes 0 = (0 + 0) \otimes 0 = 0 \otimes 0 + 0 \otimes 0$.
    Subtracting both sides by $0 \otimes 0$ gives the result. 
    Notice that there is nothing special about $1$ here; we could take 
    $[0]_3$ and $[2]_3$ as well. 
    In general, tensoring an abelian group with $\Q$ kills the torsion.
\end{exmp}

\begin{exmp}{}
    Let $m, n \in \Z^+$. Let $d = \gcd(m, n)$. We claim that 
    $\Z/m\Z \otimes_{\Z} \Z/n\Z = \Z/d\Z$. We have 
    \[ [i]_m \otimes [j]_n = [i]_m \otimes j \cdot [1]_n 
    = [ij]_m \otimes [1]_n. \] 
    For $a \in \Z$, we can write $a = u + ms + nt$ where 
    $u \in \{0, 1, \dots, d-1\}$. This gives 
    \begin{align*} 
        [ij]_m \otimes [1]_n 
        &= [u + ms + nt]_m \otimes [1]_n \\
        &= [u + nt]_m \otimes [1]_n \\
        &= ([u]_m + [t]_m \cdot n) \otimes [1]_n \\
        &= [u]_m \otimes [1]_n + [t]_m \cdot n \otimes [1]_m \\
        &= u \cdot [1]_m \otimes [1]_n. 
    \end{align*}
    Hence, we have $|\Z/m\Z \otimes_{\Z} \Z/n\Z| \leq d$ since 
    $u \in \{0, 1, \dots, d-1\}$. 
    Notice we can define a multiplication via 
    \begin{align*}
        \theta : \Z/m\Z \times \Z/n\Z &\to \Z/d\Z \\ 
        ([a]_m, [b]_n) &\mapsto [ab]_d, 
    \end{align*}
    and this map is onto. Using the universal property, we have 
    map $\tilde\theta : \Z/m\Z \otimes_{\Z} \Z/n\Z \to \Z/d\Z$ 
    onto, and $\tilde\theta$ is an isomorphism. 
    \begin{center}
        \begin{tikzcd}
            \Z/m\Z \times \Z/n\Z \arrow[rr, "\theta"] \arrow[dd, "\pi"'] &  & \Z/d\Z \\
                                                                         &  &        \\
            \Z/m\Z \otimes_{\Z} \Z/n\Z \arrow[rruu, "\tilde\theta"']     &  &       
        \end{tikzcd}
    \end{center}
\end{exmp}

\begin{remark}{}
    Let $S$ and $R$ be rings. If $M = {}_S M_R$ (that is, $M$ is a 
    left $S$-module and right $R$-module) and $N = {}_R N$, then 
    $M \otimes_R N$ is a left $S$-module via the action
    $s \cdot (m \otimes n) = s \cdot m \otimes n$. 
\end{remark}