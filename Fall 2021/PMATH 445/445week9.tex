\section{Universal property, uniqueness of tensor products (11/10/2021)}
Let $R$ be a ring, let $M = M_R$ be a right $R$-module, and let $N = {}_R N$
a left $R$-module. Then $M \otimes_R N = T/U$ is an abelian group, with 
$T$ and $U$ as in the above definition. 
We write $m \otimes n$ to mean the coset $e_{(m, n)} + U$ in $M \otimes_R N$. 

\begin{remark}{}
    Not every element in $M \otimes_R N$ can be expressed in the form  
    $m \otimes n$ in general, but every element is a $\Z$-linear combination of 
    elements of the form $m \otimes n$. 
\end{remark}

We have a map 
\begin{align*}
    \pi : M \times N &\to M \otimes_R N \\ 
    (m, n) &\mapsto m \otimes n
\end{align*}  
which is bilinear and hence satisfies the property 
\[ \pi(m_1 + m_2, n) = (m_1 + m_2) \otimes n = 
m_1 \otimes n + m_2 \otimes n. \] 
This is the case since $e_{(m_1+m_2, n)} \equiv e_{(m_1, n)} 
+ e_{(m_2, n)} \pmod{U}$. Moreover, $e_{(mr, n)} \equiv 
e_{(m, rn)} \pmod{U}$, so $mr \otimes n = m \otimes rn$. 
We now consider the universal property of tensor products. 

\begin{prop}[Universal Property of Tensor Products]{}
    If $A$ is an abelian group and $\theta : M \times N \to A$ is a bilinear map, 
    then there exists a unique group homomorphism $\tilde\theta : M \otimes_R N 
    \to A$.
    \begin{center}
        \begin{tikzcd}
            M \times N \arrow[rr, "\theta"] \arrow[dd, "\pi"'] &  & A \\
                                                            &  &   \\
            M \otimes_R N \arrow[rruu, "\tilde\theta"']          &  &  
        \end{tikzcd}
    \end{center}
\end{prop}  

\begin{pf}
    Given a bilinear map $\theta : M \times N \to A$, define 
    $\hat\theta : T \to A$ by 
    \[ \hat\theta(e_{(m,n)}) = \theta(m, n). \] 
    This extends uniquely to a group homomorphism since the elements 
    $e_{(m,n)}$ generate $T$. We claim that $U \subseteq 
    \ker\hat\theta$. To see this, it suffices to show that 
    \begin{enumerate}[(i)]
        \item $\hat\theta(e_{(m_1+m_2a, n)} - e_{(m_1, n)} 
        - ae_{(m_2, n)}) = 0$; 
        \item $\hat\theta(e_{(m,n_1+an_2)} - e_{(m,n_1)} - 
        ae_{(m,n_2)}) = 0$; 
        \item $\hat\theta(e_{(mr, n)} - e_{(m, rn)}) = 0$. 
    \end{enumerate}
    We check (iii), and we leave (i) and (ii) as exercises. We have 
    \begin{align*}
        \hat\theta(e_{(mr, n)}) - e_{(m, rn)}) = 
        \theta(mr, n) - \theta(m, rn) = 0, 
    \end{align*}
    since $\theta$ is bilinear. Now, using the map $\psi : M \times N \to T$ 
    given by 
    \[ \psi(m, n) \mapsto e_{(m, n)}, \] 
    the map $\hat\theta : T \to A$ is unique. Since $U$ is in $\ker\hat\theta$, 
    we have $\hat\theta(t + u) = \hat\theta(t)$ for all $u \in U$. This allows 
    us to extend $\hat\theta$ uniquely to a map $\tilde\theta$ on the cosets; 
    that is, we have a map from $T/U = M \otimes_R N$ to $A$. Letting 
    $\psi' : T \to T/U$ be the canonical projection, we obtain the following 
    diagram. 
    \begin{center}
        \begin{tikzcd}
            M \times N \arrow[rr, "\theta"] \arrow[dd, "\psi"'] &  & A \\
                                                                &  &   \\
            T \arrow[rruu, "\hat\theta"'] \arrow[dd, "\psi'"']  &  &   \\
                                                                &  &   \\
            T/U \arrow[rruuuu, "\tilde\theta"']                 &  &  
        \end{tikzcd}
    \end{center}
    Setting $\pi = \psi' \circ \psi$ will give us the diagram above, 
    completing the proof. 
\end{pf}

\begin{prop}[Uniqueness of Tensor Products]{}
    We have the bilinear map 
    \begin{align*} 
        \pi : M \times N &\to M \otimes_R N \\ 
        (m, n) &\mapsto m \otimes n
    \end{align*}
    which satisfies the universal property. Suppose that $\Omega$ is 
    another abelian group with a bilinear map $\pi' : M \times N 
    \to \Omega$ which also has the universal property. Then 
    $M \otimes_R N \cong \Omega$. 
\end{prop}
\begin{pf}
    By the universal property of $\Omega$, there exists a unique group 
    homomorphism $\tilde\pi' : M \otimes_R N \to \Omega$ with $\tilde\pi' \circ \pi = \pi'$. 
    Similarly, by the universal property of $M \otimes_R N$, there 
    is a unique group homomorphism $\tilde\pi : \Omega \to M \otimes_R N$
    with $\tilde\pi \circ \pi' = \pi$. 
    \begin{center}
        \begin{tikzcd}
            M \times N \arrow[rr, "\pi'"] \arrow[dd, "\pi"'] &  & \Omega &  & M \times N \arrow[rr, "\pi"] \arrow[dd, "\pi'"'] &  & M \otimes_R N \\
                                                             &  &        &  &                                                  &  &               \\
            M \otimes_R N \arrow[rruu, "\tilde\pi'"']        &  &        &  & \Omega \arrow[rruu, "\tilde\pi"']                 &  &              
        \end{tikzcd}
    \end{center} 
    Clearly, there is a unique map $\id : \Omega \to \Omega$. 
    Using the maps $\tilde\pi : \Omega \to M \otimes_R N$ and 
    $\tilde\pi' : M \otimes_R N \to \Omega$., we obtain the following diagram. 
    \begin{center}
        \begin{tikzcd}
            M \times N \arrow[rr, "\pi'"] \arrow[dd, "\pi'"']    &  & \Omega                                  \\
                                                                 &  &                                         \\
            \Omega \arrow[rruu, "\id"'] \arrow[rr, "\tilde\pi"'] &  & M \otimes_R N \arrow[uu, "\tilde\pi'"']
        \end{tikzcd}
    \end{center} 
    Notice that 
    \[ (\tilde\pi' \circ \tilde\pi) \circ \pi' 
    = \tilde\pi' \circ (\tilde\pi \circ \pi') 
    = \tilde\pi' \circ \pi = \pi'. \] 
    Therefore, we get the following diagram. 
    \begin{center}
        \begin{tikzcd}
            M \times N \arrow[rr, "\pi'"] \arrow[dd, "\pi'"']  &  & \Omega \\
                                                               &  &        \\
            \Omega \arrow[rruu, "\tilde\pi' \circ \tilde\pi"'] &  &       
        \end{tikzcd}
    \end{center}
    But the map $\Omega \to \Omega$ is unique by the universal property, 
    and so we must have $\tilde\pi' \circ \tilde\pi = \id_\Omega$. By 
    an analogous argument, we find that $\tilde\pi \circ \tilde\pi' = 
    \id_{M \otimes_R N}$. This means that $\tilde\pi : \Omega 
    \to M \otimes_R N$ and $\tilde\pi' : M \otimes_R N \to \Omega$ are 
    inverses. In particular, it follows that these maps are isomorphisms. 
\end{pf}

\begin{exmp}{}
    Let $R = \Z$, $M = \Q$, and $N = \Z/3\Z$. Note that 
    $\Q \otimes_{\Z} \Z/3\Z = (0)$. Indeed, in this case, we have 
    \[ T = \bigoplus_{(a, b) \in \Q \times \Z/3\Z} \Z e_{(a, b)}. \] 
    Consider $a \otimes [1]_3$. We have 
    \[ a \otimes [1]_3 = \frac{a}3 \cdot 3 \otimes [1]_3 
    = \frac{a}3 \otimes 3 \cdot [1]_3 = \frac{a}3 \otimes 0
    = \frac{a}3 \otimes 0 \cdot 0 = 0 \otimes 0 = 0. \] 
    To see that $0 \otimes 0 = 0$, note that by bilinearity, we have 
    $0 \otimes 0 = (0 + 0) \otimes 0 = 0 \otimes 0 + 0 \otimes 0$.
    Subtracting both sides by $0 \otimes 0$ gives the result. 
    Notice that there is nothing special about $1$ here; we could take 
    $[0]_3$ and $[2]_3$ as well. 
    In general, tensoring an abelian group with $\Q$ kills the torsion.
\end{exmp}

\begin{exmp}{}
    Let $m, n \in \Z^+$. Let $d = \gcd(m, n)$. We claim that 
    $\Z/m\Z \otimes_{\Z} \Z/n\Z = \Z/d\Z$. We have 
    \[ [i]_m \otimes [j]_n = [i]_m \otimes j \cdot [1]_n 
    = [ij]_m \otimes [1]_n. \] 
    For $a \in \Z$, we can write $a = u + ms + nt$ where 
    $u \in \{0, 1, \dots, d-1\}$. This gives 
    \begin{align*} 
        [ij]_m \otimes [1]_n 
        &= [u + ms + nt]_m \otimes [1]_n \\
        &= [u + nt]_m \otimes [1]_n \\
        &= ([u]_m + [t]_m \cdot n) \otimes [1]_n \\
        &= [u]_m \otimes [1]_n + [t]_m \cdot n \otimes [1]_m \\
        &= u \cdot [1]_m \otimes [1]_n. 
    \end{align*}
    Hence, we have $|\Z/m\Z \otimes_{\Z} \Z/n\Z| \leq d$ since 
    $u \in \{0, 1, \dots, d-1\}$. 
    Notice we can define a multiplication via 
    \begin{align*}
        \theta : \Z/m\Z \times \Z/n\Z &\to \Z/d\Z \\ 
        ([a]_m, [b]_n) &\mapsto [ab]_d, 
    \end{align*}
    and this map is onto. Using the universal property, we have a
    map $\tilde\theta : \Z/m\Z \otimes_{\Z} \Z/n\Z \to \Z/d\Z$ which is
    onto, and $\tilde\theta$ is an isomorphism. 
    \begin{center}
        \begin{tikzcd}
            \Z/m\Z \times \Z/n\Z \arrow[rr, "\theta"] \arrow[dd, "\pi"'] &  & \Z/d\Z \\
                                                                         &  &        \\
            \Z/m\Z \otimes_{\Z} \Z/n\Z \arrow[rruu, "\tilde\theta"']     &  &       
        \end{tikzcd}
    \end{center}
\end{exmp}

\begin{remark}{}
    Let $S$ and $R$ be rings. If $M = {}_S M_R$ (that is, $M$ is a 
    left $S$-module and right $R$-module, and we call $M$ a 
    \textbf{$(S, R)$-bimodule}) and $N = {}_R N$, then 
    $M \otimes_R N$ is a left $S$-module via the action
    $s \cdot (m \otimes n) = (s \cdot m) \otimes n$. 
    
    Similarly, if $S'$ is another ring and $N = {}_R M_{S'}$ is an 
    $(R, S')$-bimodule, then $M \otimes_R N$ is a right $S'$-module via the 
    action $(m \otimes n) \cdot s' = m \otimes (n \cdot s')$. 
\end{remark}

\section{More on tensor products, induced representations (11/12/2021)}
We first make the small remark that every ring $R$ is a $\Z$-algebra, 
as there is always a ring homomorphism $\Z \to R$. Now, we'll cover 
another example of a tensor product. 

\begin{exmp}{}
    Let $k$ be a field. Let $V$ and $W$ be $k$-vector spaces with bases 
    $\{v_\alpha\}_{\alpha\in X}$ and $\{w_\beta\}_{\beta\in Y}$ respectively. 
    Then $V \otimes_k W$ is a $k$-vector space with basis 
    $\{v_\alpha \otimes w_\beta\}_{(\alpha, \beta) \in X \times Y}$. 
\end{exmp}
\begin{pf}
    We can think of $V$ as a $(k, k)$-bimodule with $\lambda \cdot v = v \cdot \lambda$. 
    Then $V \otimes_k W$ is a $k$-vector space by Remark 25.6. Notice that 
    $V \otimes_k W$ is spanned by elements of the form $v \otimes w$ 
    with $v \in V$ and $w \in W$. Writing $v = \sum_{\alpha\in X} c_\alpha v_\alpha$ 
    and $w = \sum_{\beta\in Y} d_\beta w_\beta$, we have 
    \begin{align*}
        v \otimes w &= \sum_{\alpha \in X} c_\alpha v_\alpha \otimes w \\
        &= \sum_{\alpha \in X} c_\alpha \cdot (v_\alpha \otimes w) \\ 
        &= \sum_{\alpha \in X} c_\alpha v_\alpha \otimes \sum_{\beta \in Y} d_\beta w_\beta \\ 
        &= \sum_{\alpha \in X} \sum_{\beta \in Y} c_\alpha v_\alpha \otimes (d_\beta w_\beta) \\ 
        &= \sum_{\alpha \in X} \sum_{\beta \in Y} c_\alpha d_\beta (v_\alpha \otimes w_\beta). 
    \end{align*}
    To show linear independence, let $U$ be the $k$-vector space with basis
    $\{e_{(\alpha, \beta)} : (\alpha, \beta) \in X \times Y\}$. Define the map 
    $\phi : V \times W \to U$ by the rule 
    \[ \phi \left( \sum_{\alpha \in X} c_\alpha v_\alpha, 
    \sum_{\beta \in Y} d_\beta w_\beta \right) = \sum_{\alpha \in X} 
    \sum_{\beta \in Y} c_\alpha d_\beta e_{(\alpha, \beta)}. \]     
    We see that $\phi$ is bilinear by construction. In particular, 
    note that $\phi(\lambda v, w) = \lambda \cdot \phi(v, w)$. Moreover, 
    we know that $\lambda(v \otimes w) = \lambda v \otimes w$. By the universal 
    property, there exists a unique map $\tilde\phi : V \otimes_k W \to U$ 
    which makes the following diagram commute. 
    \begin{center}
        \begin{tikzcd}
            V \times W \arrow[rr, "\phi"] \arrow[dd, "\pi"'] &  & U \\
                                                            &  &   \\
            V \otimes_k W \arrow[rruu, "\tilde\phi"']        &  &  
        \end{tikzcd}
    \end{center}
    In fact, notice that $\tilde\phi$ is $k$-linear. Now, we have 
    \[ \tilde\phi(v_\alpha \otimes w_\beta) = \tilde\phi \circ \pi(v_\alpha, 
    w_\beta) = \phi(v_\alpha, w_\beta) = e_{(\alpha, \beta)}. \] 
    Suppose that we have $\sum c_{\alpha, \beta} v_\alpha \otimes w_\beta = 0$. 
    Then we obtain 
    \[ 0 = \tilde\phi \left( \sum c_{\alpha,\beta} v_\alpha \otimes w_\beta \right) 
    = \sum c_{\alpha,\beta} \tilde\phi(v_\alpha \otimes w_\beta) = 
    \sum c_{\alpha,\beta} e_{(\alpha,\beta)}, \] 
    and hence $c_{\alpha,\beta} = 0$ for all $(\alpha, \beta) \in X \times Y$. 
    Therefore, $\{v_\alpha \otimes w_\beta\}_{(\alpha, \beta) \in X \times Y}$ 
    forms a basis for $V \otimes_k W$. 
\end{pf}

Let $R$ be a ring, let $M = M_R$ be a right $R$-module, and let $N = {}_R N$ 
be a left $R$-module. By Exercise (iii) in Assignment 5, we have $R_R \otimes_R N 
\cong N$ (as left $R$-modules) and $M \otimes_R {}_R R \cong M$ (as 
right $R$-modules). This can be shown by considering the bilinear maps 
$(r, n) \mapsto rn$ and $(m, r) \mapsto mr$, and showing that the universal
property holds. 

\begin{exercise}{}
    Let $R$ be a ring. Let $(M_\alpha)$ be a family of left $R$-modules, 
    and $N$ be a right $R$-module. Show that 
    \[ \left( \bigoplus_\alpha M_\alpha \right) \otimes_R N \cong 
    \bigoplus_{\alpha} (M_\alpha \otimes_R N). \] 
    Similarly, if $(N_\beta)$ is a family of right $R$-modules and $M$ 
    is a left $R$-module, then show that 
    \[ M \otimes_R \left( \bigoplus_\beta N_\beta \right) \cong 
    \bigoplus_\beta (M \otimes_R N_\beta). \] 
    Hint: Show that these have the universal property. 
\end{exercise}

Intuitively, this makes sense because we can think of a tensor product as 
multiplication, and we have a direct sum of modules. We would expect 
multiplication to distribute over the sum. 

\begin{cor}{}
    Let $k$ be a field. If $V$ and $W$ are $k$-vector spaces with bases 
    $\{v_\alpha\}_{\alpha\in X}$ and $\{w_\beta\}_{\beta\in Y}$ respectively, 
    then 
    \[ V \otimes_k W \cong \bigoplus kv_\alpha \otimes w_\beta. \] 
\end{cor}
\begin{pf}
    Since $\{v_\alpha\}_{\alpha\in X}$ and $\{w_\beta\}_{\beta\in Y}$ are 
    bases for $V$ and $W$ respectively, we can write $V \cong \bigoplus_{\alpha\in X}
    kv_\alpha$ and $W \cong \bigoplus_{\beta\in Y} kw_\beta$. Using 
    Exercise 26.2, it follows that 
    \begin{align*}
        V \otimes_k W 
        &\cong \left( \bigoplus_{\alpha\in X} kv_\alpha \right) \otimes_k W \\ 
        &\cong \bigoplus_{\alpha\in X} kv_\alpha \otimes_k W \\ 
        &\cong \bigoplus_{\alpha\in X} (kv_\alpha) \otimes 
        \left( \bigoplus_{\beta\in Y} kw_\beta \right) \\ 
        &\cong \bigoplus_{\alpha\in X} \bigoplus_{\beta\in Y} kv_\alpha \otimes kw_\beta \\ 
        &\cong \bigoplus_{\alpha\in X} \bigoplus_{\beta\in Y} kv_\alpha \otimes w_\beta. \qedhere 
    \end{align*}
\end{pf}

We now begin induced representations. Suppose we are given a subgroup 
$H \leq G$, and a representation $W$ of $H$ (that is, $W$ is a left 
$k[H]$-module). Then we can find a representation $W \uparrow_H^G$ of $G$ 
(that is, a left $k[G]$-module), called the {\bf induced representation} of $W$
up to $G$. Let's see how this is done. 

Since $H$ is a subgroup of $G$, we see that $k[H]$ is a subring of $k[G]$. 
We can think of $k[G]$ as a $(k[G], k[G])$-bimodule. In particular, we can 
view $k[G]$ as a left $k[G]$-module, and a right $k[H]$-module, forgetting 
the fact that it is a right $k[G]$-module. Then we can form a 
$k[G]$-module from $W$ by  
\[ W \uparrow_H^G := k[G] \otimes_{k[H]} W. \] 
Notice that this tensor product makes sense because $k[G]$ is a right 
$k[H]$-module, and $W$ is a left $k[H]$-module. 

Recall that $k[G]$ is a free right $k[H]$-module with basis $x_1, \dots, x_m$, 
where $G = x_1H \cup x_2H \cup \cdots \cup x_mH$, where the union is disjoint 
so that each $x_i$ is a left coset representative. Indeed, we can write 
\[ \sum_{g\in G} \alpha_g g = \sum_{h\in H} \alpha_{x_1h}(x_1h) + \cdots + 
\sum_{h\in H} \alpha_{x_mh}(x_mh), \] 
so every element in $k[G]$ decomposes uniquely in the form 
\[ \sum_{g\in G} \alpha_g g = x_1 \sum_{h\in H} \alpha_{x_1h}h + 
\cdots + x_m \sum_{h\in H} \alpha_{x_mh}h. \] 
Since each $\sum_{h\in H}\alpha_{x_ih}h \in k[H]$, we get 
\[ k[G] \cong x_1 k[H] \oplus \cdots \oplus x_m k[H]. \] 
Now, it follows that 
\begin{align*}
    k[G] \otimes_{k[H]} W 
    &\cong (x_1 k[H] \oplus \cdots \oplus x_m k[H]) \otimes_{k[H]} W \\ 
    &\cong (x_1 k[H] \otimes_{k[H]} W) \oplus \cdots \oplus 
    (x_m k[H] \otimes_{k[H]} W). 
\end{align*}
By Exercise (iii) in Assignment 5, we have $k[H] \otimes_{k[H]} W \cong W$. 
Hence, we obtain 
\[ k[G] \otimes_{k[H]} W \cong x_1 W \oplus \cdots \oplus x_m W. \] 
More precisely, if $\{w_1, \dots, w_p\}$ is a basis for $W$, then 
$W \uparrow_H^G$ is the $k$-span of $\{x_i \otimes w_j : 
1 \leq i \leq m,\, 1 \leq j \leq p\}$. Notice that each $x_i W$ is 
$p$-dimensional, so $W \uparrow_H^G$ is $mp$-dimensional. Now, how does 
$g \in G$ act on this basis? First, note that $g \cdot (x_i \otimes w) = 
(g \cdot x_i) \otimes w$ by bilinearity. Once again, we recall that 
$G = x_1H \cup \cdots \cup x_m H$, so there exists a unique $j \in 
\{1, \dots, m\}$ and $h \in H$ such that $g \cdot x_i = x_j \cdot h$. We can take 
\[ g \cdot (x_i \otimes w) = (g \cdot x_i) \otimes w = (x_j \cdot h) \otimes w 
= x_j \otimes h \cdot w \] 
to be our action. Since $W$ is a representation of $H$, we have a homomorphism
$\rho : H \to \GL(W)$. In particular, $\rho(h)$ looks like an $m \times m$ 
block matrix with copies of $\rho(h)$ or the zero matrix in the blocks. 

\begin{exmp}{}
    Let $k = \C$. Let $G = S_3$, and let $H = A_3 = \{\id, (123), (132)\}$ be the 
    subgroup of index $3$. Notice that $S_3 = \id \cdot\,H \cup (12) \cdot H$, 
    so we can take $x_1 = \id$ and $x_2 = (12)$ to be our coset representatives. 
    Take $W = \C$ to be our $H$-module, with action given by 
    $\id \cdot\,1 = 1$, $(123) \cdot 1 = e^{2\pi i/3}$, 
    and $(132) \cdot 1 = e^{4\pi i/3}$.

    Now, the basis of $W \uparrow_H^{S_3}$ is simply given by $\{x_1 \otimes 1, 
    x_2 \otimes 1\}$, since $\{1\}$ forms a basis for $W = \C$. Write 
    $v_1 = x_1 \otimes 1$ and $v_2 = x_2 \otimes 1$. Let's see how the elements 
    of $S_3$ act on $W \uparrow_H^{S_3}$. 

    We first consider $(12)$. Notice that 
    \[ (12) \cdot v_1 = (12) \cdot (x_1 \otimes 1) = 
    ((12) \cdot x_1) \otimes 1 = x_2 \otimes 1 = 0 \cdot v_1 + 1 \cdot v_2. \] 
    Similarly, we find that 
    \[ (12) \cdot v_2 = (12) \cdot (x_2 \otimes 1) = 
    ((12) \cdot x_2) \otimes 1 = x_1 \otimes 1 = 1 \cdot v_1 + 0 \cdot v_2. \] 
    Therefore, we have 
    \[ \rho((12)) = \begin{pmatrix}
        0 & 1 \\ 
        1 & 0 
    \end{pmatrix}. \] 
    \newpage 
    What about the action of $(123)$? Notice that $(123) \cdot v_1 = 
    (123) \cdot (x_1 \otimes 1) = (123) \cdot x_1 \otimes 1$, and since 
    $(123) \in H$, we can move it to the other side. This becomes 
    \[ x_1 \otimes ((123) \cdot 1) = x_1 \otimes e^{2\pi i/3} = 
    e^{2\pi i/3} (x_1 \otimes 1) = e^{2\pi i/3} \cdot v_1 + 0 \cdot v_2. \] 
    On the other hand, we have $(123) \cdot v_2 = (123) \cdot (x_2 \otimes 1) 
    = (123) \cdot (12) \otimes 1 = (12) \cdot (132) \otimes 1$. Notice that 
    $(132) \in H$, so we can move it to the other side to obtain 
    \[ (12) \otimes ((132) \cdot 1) = (12) \otimes e^{4\pi i/3} 
    = e^{4\pi i/3} (x_2 \otimes 1) = 0 \cdot v_1 + e^{4\pi i/3} \cdot v_2. \] 
    Therefore, we obtain 
    \[ \rho((123)) = \begin{pmatrix}
        e^{2\pi i/3} & 0 \\ 
        0 & e^{4\pi i/3}
    \end{pmatrix}. \] 
    Now that we have the matrices for $(12)$ and $(123)$, we can compute the 
    rest by multiplying, since these elements generate $S_3$. 

    Note that $\rho$ is an irreducible $2$-dimensional representation of $S_3$. 
    It is clear that it is $2$-dimensional. Notice that if $\chi$ is the 
    character of $\rho$, then $\chi((12)) = 0$ by construction, 
    $\chi((123)) = e^{2\pi i/3} + e^{4\pi i/3} = -1$, and $\chi(\id) = 2$. Since 
    $\chi$ is a class function, it is now completely determined. Computing the 
    inner product of $\chi$ with itself, we have 
    \[ \langle \chi, \chi \rangle = \frac{1}{6}(2^2 \cdot 1 + 0^2 \cdot 3 
    + (-1)^2 \cdot 2) = 1, \] 
    so $\rho$ is in fact irreducible. These values also check out with the 
    character table of $S_3$ that we computed in Example 16.2!
\end{exmp}

\section{Induced characters and transitivity (11/15/2021)}
Let $k$ be a field, and let $H$ be a subgroup of $G$. Then $G$ is the 
union of disjoint cosets $x_1H \cup \cdots \cup x_mH$. If $W$ is a 
left $k[H]$-module with $k$-basis $\{w_1, \dots, w_p\}$, then we obtain 
a representation
\[ \rho : H \to \GL_p(k) \cong \GL(W) \] 
which can be written in matrix form as $\rho(h) = (\rho_{ij}(h))_{1\leq i,j\leq p}$
with action 
\[ h \cdot w_j = \sum_{i=1}^p \rho_{ij}(h) \cdot w_i. \] 
Recall that the induced representation of $W$ up to $G$ is a left $k[G]$-module 
given by $W \uparrow_H^G\,= k[G] \otimes_{k[H]} W$. This has basis 
$\{x_i \otimes w_j : 1 \leq i \leq m,\, 1 \leq j \leq p\}$. Note that 
$g \cdot x_i \in x_qH$ for some $q \in \{1, \dots, m\}$, so we can take the action 
\[ g \cdot (x_i \otimes w) = x_q \otimes (h \cdot w). \] 
With this basis, we get a representation 
\[ \rho \uparrow_H^G\;: G \to \GL_{mp}(k). \] 
What is $\rho \uparrow_H^G(g)$? If we treat each $\{x_j \otimes w_1, \dots, 
x_j \otimes w_p\}$ as a ``block'', we obtain 
\begin{align*}
    g \cdot (x_j \otimes w_1) = x_q \otimes (h \cdot w_1) &= 
    x_q \otimes [\rho_{11}(h) w_1 + \cdots + \rho_{p1}(h) w_p] \\ 
    &\;\;\vdots \\ 
    g \cdot (x_j \otimes w_p) = x_q \otimes (h \cdot w_p) &= 
    x_q \otimes [\rho_{1p}(h) w_1 + \cdots + \rho_{pp}(h) w_p]. 
\end{align*}
Then the $j$-th column in the block matrix $\rho \uparrow_H^G(g)$ has 
$\rho(h)$ in the $q$-th row, and $0$ in the other rows. This gives us 
an $m \times m$ block matrix from $\rho(h)$, where the blocks are 
$p \times p$ matrices, and the $(i, j)$-th block is $0$ if $gx_j \in x_i H$, 
and it is $\rho(x_i^{-1}gx_j)$ if $gx_j \notin x_i H$ (since 
$gx_j = x_ih$ implies that $h = x_i^{-1}gx_j$).

Now, what is the character $\chi \uparrow_H^G\;= \Tr(\rho \uparrow_H^G)$? 
If we write 
\[ \rho \uparrow_H^G(g) = \begin{pmatrix}
    A_{11} & \cdots & A_{1m} \\ 
    \vdots & \ddots & \vdots \\ 
    A_{m1} & \cdots & A_{mm}
\end{pmatrix} \] 
where each $A_{ij}$ is a $p \times p$ matrix, we get 
\[ \chi \uparrow_H^G(g) = \Tr(A_{11}) + \Tr(A_{22}) + \cdots + \Tr(A_{mm}). \] 
Notice that $A_{ii} \neq 0$ when $gx_i \in x_iH$, or equivalently, 
$x_i^{-1}gx_i \in H$. In this case, we have $A_{ii} = \rho(x_i^{-1}gx_i)$. 
So if $\chi = \Tr(\rho)$, then we get 
\[ \chi \uparrow_H^G(g) = \sum_{\{i\,:\,x_i^{-1}gx_i \in H\}} 
\Tr(\rho(x_i^{-1}gx_i)) = \sum_{\{i\,:\,x_i^{-1}gx_i \in H\}} \chi(x_i^{-1}gx_i). \] 

\begin{exercise}{}
    Show that we have the equivalent formula 
    \[ \chi \uparrow_H^G(g) = \frac{1}{|H|} \sum_{\{x\,:\,x^{-1}gx \in H\}} 
    \chi(x^{-1}gx). \] 
\end{exercise}

\begin{exmp}{}
    In Assignment 4, we computed the character table of 
    $G = \langle y \mid y^7 = 1 \rangle \rtimes \langle x \mid x^3 = 1 \rangle$ 
    with $xyx^{-1} = y^2$. We can compute the irreducible characters of degree 
    $3$ more easily by using induced representations. Let $W = \C$ be a 
    $1$-dimensional representation of $H = \langle y \rangle$ with action 
    \[ y^j \cdot 1 = \zeta^j, \] 
    where $\zeta = e^{2\pi i/7}$ is a $7$-th root of unity. 
    Notice that $G = H \cup xH \cup x^2H$, so $\C[G] \otimes_{\C[H]} W$ 
    has basis $\{v_1, v_2, v_3\}$, where $v_1 = 1 \otimes 1$, $v_2 = x \otimes 1$, 
    and $v_3 = x^2 \otimes 1$. Observe that 
    \begin{align*}
        x \cdot v_1 &= v_2, \\ 
        x \cdot v_2 &= v_3, \\ 
        x \cdot v_3 &= v_1.
    \end{align*}
    This gives us 
    \[ \rho(x) = \begin{pmatrix}
        0 & 0 & 1 \\ 
        1 & 0 & 0 \\ 
        0 & 1 & 0
    \end{pmatrix}, \] 
    which has trace $0$, so $\chi(x) = \chi(x^2) = 0$. 
    
    On the other hand, we compute 
    \begin{align*}
        y \cdot v_1 &= y \cdot 1 \otimes 1 = 1 \otimes (y \cdot 1) = 
        \zeta \cdot v_1, \\ 
        y \cdot v_2 &= y \cdot x \otimes 1 = x(x^{-1}yx) \otimes 1 = 
        xy^4 \otimes 1 = x \otimes (y^4 \cdot 1) = \zeta^4 \cdot v_2, \\
        y \cdot v_3 &= y \cdot x^2 \otimes 1 = x^2(x^{-2}yx^2) \otimes 1 
        = x^2y^2 \otimes 1 = x^2 \otimes (y^2 \cdot 1) = \zeta^2 \cdot v_3. 
    \end{align*} 
    Therefore, we obtain 
    \[ \rho(y) = \begin{pmatrix}
        \zeta & 0 & 0 \\ 
        0 & \zeta^4 & 0 \\ 
        0 & 0 & \zeta^2
    \end{pmatrix}, \] 
    so $\chi(y) = \zeta + \zeta^2 + \zeta^4$. Then, we can deduce that 
    \[ \chi(y^3) = \chi(y^6) = \chi(y^{-1}) = \overline{\chi(y)} 
    = \zeta^3 + \zeta^5 + \zeta^6. \] 
\end{exmp}

We can extend the above formula to general class functions. 

\begin{defn}{}
    Let $H$ be a subgroup of $G$. Let $f : H \to k$ be a class function 
    where $k$ is a field. Then we define 
    \[ f \uparrow_H^G(g) := \frac{1}{|H|} \sum_{\{x\in G\,:\,x^{-1}gx \in H\}} 
    f(x^{-1}gx). \] 
\end{defn}

In particular, $\chi \uparrow_H^G$ is the character of the induced representation
from $H$ to $G$ with character $\chi$. 

\begin{prop}[Transitivity of induction]{}
    Let $H \leq G \leq K$ be groups, and let $f : H \to k$ be a class function. 
    Then $(f \uparrow_H^G) \uparrow_G^K = f \uparrow_H^K$. 
\end{prop}
\begin{pf}
    We'll prove this using a routine calculation, but we note that this can be 
    done in a coordinate-free way by using tensor products, which we leave 
    to Assignment 5. For $u \in K$, we have 
    \begin{align*}
        (f \uparrow_H^G) \uparrow_G^K(u)
        &= \frac{1}{|G|} \sum_{\{x\in K\,:\,x^{-1}ux\in G\}} f \uparrow_H^G (x^{-1}ux) \\ 
        &= \frac{1}{|G|} \sum_{\{x\in K\,:\,x^{-1}ux\in G\}} \left( 
        \frac{1}{|H|} \sum_{\{y\in G\,:\,y^{-1}(x^{-1}ux)y\in H\}} f(y^{-1}(x^{-1}ux)y) \right) \\ 
        &= \frac{1}{|G|} \frac{1}{|H|} \sum_{\{x\in K\,:\,x^{-1}ux\in G\}}
        \sum_{\{y\in G\,:\,y^{-1}x^{-1}uxy\in H\}} f(y^{-1}x^{-1}uxy). 
    \end{align*}
    Notice that $y^{-1}x^{-1}uxy \in H$ with $y \in G$ implies that 
    $x^{-1}ux \in yHy^{-1} \in G$. Now, by making the substitution $z = xy$, 
    we obtain 
    \begin{align*}
        (f \uparrow_H^G) \uparrow_G^K(u)
        &= \frac{1}{|G|} \frac{1}{|H|} \sum_{\{z\in K\,:\,z^{-1}uz \in H\}} 
        \sum_{y \in G} f(z^{-1}uz) \\ 
        &= \frac{1}{|G|} \frac{1}{|H|} \sum_{\{z\in K\,:\,z^{-1}uz \in H\}} 
        |G| f(z^{-1}uz) \\ 
        &= f \uparrow_H^K(u). \qedhere 
    \end{align*}
\end{pf}
