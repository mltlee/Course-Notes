\section{Burnside's theorem (11/03/2021)} 
The fact that the degrees of the irreducible characters divide the order of 
the group has some very powerful consequences. The Feit-Thompson theorem 
states that every group of odd order is solvable. We'll focus on Burnside's 
theorem, which gives the weaker statement that every group of order 
$p^a q^b$ is solvable, where $p$ and $q$ are distinct primes and $a, b \in \Z^+$. 
The proofs of these theorems are very difficult (if not impossible) without 
using character theory, and we'll use it to make our lives easier. 
First, let's recall what it means for a group to be solvable. 

\begin{defn}{}
    A group $G$ is {\bf solvable} if there exists a tower 
    \[ G = G_0 \supseteq G_1 \supseteq G_2 \supseteq \cdots \supseteq G_m = \{1\} \] 
    such that $G_{i+1} \trianglelefteq G_i$ and $G_i/G_{i+1}$ is abelian for 
    all $0 \leq i \leq m-1$. 
\end{defn}

Towards the proof of Burnside's theorem, we'll first prove some intermediate results. 

\begin{lemma}{}
    Let $\chi$ be an irreducible character of degree $n \geq 2$. If $\chi(g)/n 
    \in \A$ for some $g \in G$, then $\chi(g) = 0$. 
\end{lemma}

We leave the proof of this for later. For the moment, we 
will use this lemma to prove the following theorem. 

\begin{theo}{}
    Let $G$ be a non-abelian finite group. Suppose that $G$ has a conjugacy 
    class of size $p^a$ where $p$ is a prime and $a \in \Z^+$. Then $G$ is 
    not simple. 
\end{theo}
\begin{pf}
    Suppose by way of contradiction that there exists a non-abelian finite 
    simple group $G$ with an element $g \in G$ such that the conjugacy class 
    ${\cal C} = {\cal C}(g)$ containing $g$ has size $p^a$, where $p$ is prime and 
    $a \in \Z^+$. Since $G$ is non-abelian and simple, we have $G' = G$. 
    Indeed, note that $G' \trianglelefteq G$. By simplicity, we either 
    have $G' = G$ or $G' = \{1\}$. But $G$ is non-abelian, so we must be in the first 
    case. This means that $G$ has one inequivalent irreducible representation 
    of degree $1$ since $|G/G'| = |G/G| = 1$. The only irreducible character 
    of degree $1$ is the trivial character $\chi$ with $\chi(h) = 1$ for all $h \in G$.

    Let $\chi_1, \dots, \chi_s$ be the irreducible characters of $G$. Without 
    loss of generality, we may assume that $\chi_1$ is the trivial character. 
    Notice that if $n_1, \dots, n_s$ are the degrees of $\chi_1, \dots, \chi_s$ 
    respectively, then $n_1 = 1$ and $n_i > 1$ for $i = 2, \dots, s$. Recall 
    that we have $g \in G$ such that $|{\cal C}| = |{\cal C}(g)| = p^a$, 
    where $p$ is prime and $a \in \Z^+$. This implies that $g \neq 1$, for 
    otherwise its conjugacy class would have size $1$. Now, if $L$ is the 
    left regular representation of $G$, then $\chi_L(g) = 0$. This implies that 
    \[ 0 = \chi_L(g) = \sum_{i=1}^s n_i \chi_i(g) 
    = \chi_1(g) + \sum_{\substack{i\geq 2 \\ p\,\mid\,n_i}} n_i \chi_i(g) 
    + \sum_{\substack{i\geq 2 \\ p\,\nmid\,n_i}} n_i \chi_i(g). \]
    {\sc Claim.} For $i \geq 2$, if $p \nmid n_i$, then $\chi_i(g) = 0$. 

    {\sc Proof of Claim.} We know that $\chi_i(g) \in \A$. We have also showed 
    in Lemma 21.1 that 
    \[ \frac{p^a \chi_i(g)}{n_i} = \frac{|{\cal C}(g)| \chi_i(g)}{n_i} \in \A. \] 
    Since $p \nmid n_i$, we have $\gcd(p^a, n_i) = 1$. Therefore, there 
    exist $c, d \in \Z$ such that $cp^a + dn_i = 1$. We find that 
    \[ \frac{\chi_i(g)}{n_i} = \frac{cp^a + dn_i}{n_i} \chi_i(g) 
    = \frac{cp^a \chi_i(g)}{n_i} + d\chi_i(g) \in \A. \] 
    By Lemma 22.2, we obtain $\chi_i(g) = 0$. \hfill$\blacksquare$ 

    Now, writing $n_i = pn'_i$ for each $i \geq 2$ with $p \mid n_i$, we have 
    \[ 0 = \chi_1(g) + \sum_{\substack{i\geq 2 \\ p\,\mid\,n_i}} n_i \chi_i(g) 
    + \sum_{\substack{i\geq 2 \\ p\,\nmid\,n_i}} n_i \chi_i(g) 
    = \chi_1(g) + p \sum_{\substack{i\geq 2 \\ p\,\mid\,n_i}} n'_i \chi_i(g). \] 
    Let $\alpha$ denote the sum in the above equation, and note that $\alpha \in \A$. 
    Then $1 + p\alpha = 0$, which implies that $\alpha = -1/p$. But this is a 
    contradiction since $\Q \cap \A = \Z$, but $-1/p \notin \Z$. 
\end{pf}

We can now prove Burnside's theorem. First, we'll recall some facts from group theory. 
\begin{enumerate}[(1)]
    \item Let $N \trianglelefteq G$. Then $G$ is solvable if and only if 
          $N$ and $G/N$ are both solvable. 
    \item Let $G$ be a group of order $q^b c$ where $q$ is prime, $b \in \Z^+$, 
          and $q \nmid c$. Then by the first Sylow theorem, $G$ has a subgroup 
          $Q$ of order $q^b$. 
    \item If $G$ is a group of order $q^b$ where $q$ is prime and $b \in \Z^+$, 
          then $G$ has a non-trivial center by the class equation. 
    \item If $g \in G$ and ${\cal C}(g)$ is the conjugacy class of $g$, then 
          $|{\cal C}(g)| = |G|/|C(g)|$, where $C(g) = \{h \in G : hg = gh\}$ 
          is the centralizer of $g$. 
\end{enumerate}

\begin{theo}[Burnside's theorem]{}
    Let $G$ be a group of order $p^a q^b$ where $p$ and $q$ are distinct primes 
    and $a, b \in \Z^+$. Then $G$ is solvable. 
\end{theo}
\begin{pf}
    Suppose the result is not true. Let $G$ be a minimal counterexample; that is, 
    $G$ is a group of order $p^a q^b$ where $p$ and $q$ are distinct primes and 
    $a, b \in \Z^+$, and $G$ has minimal size with respect to this property. 

    {\sc Claim.} $G$ is a non-abelian simple group.  

    {\sc Proof of Claim.} Note that $G$ is non-abelian since if $G$ were 
    abelian, then it would be solvable. Assume that $G$ is not simple. 
    Then there exists a non-trivial proper normal subgroup $N$ of $G$. 
    Then $G/N$ and $N$ are both smaller than $G$ and their sizes divide $p^a q^b$. 
    By minimality, $N$ and $G/N$ are solvable, so $G$ is also solvable by (1), a 
    contradiction. \hfill$\blacksquare$ 

    Pick a subgroup $Q \leq G$ with $|Q| = q^b$, which exists by (2). Moreover, 
    we can choose $z \neq 1$ such that $z \in Z(Q)$, which exists since 
    $Q$ has non-trivial center by (3). Then $Q \subseteq C(z) \subseteq G$, 
    so $|C(z)| = p^c q^b$ where $c \in \{0, \dots, a\}$. By (4), it follows that 
    \[ |{\cal C}(z)| = \frac{|G|}{|C(z)|} = \frac{p^a q^b}{p^c q^b} = p^{a-c}. \] 
    If $c < a$, then $G$ is not simple by Theorem 22.3 since $G$ 
    is non-abelian and $|{\cal C}(z)|$ is a conjugacy class with size which is a 
    prime power, contradicting our claim. So $c = a$, which gives $|{\cal C}(z)| 
    = 1$. But this means that $C(z) = G$. Moreover, $z$ is central, which gives 
    $\{1\} \neq Z(G) \trianglelefteq G$. Since $G$ is simple, we must have 
    $Z(G) = G$. This would mean that $G$ is abelian, once again contradicting 
    our claim. Therefore, no counterexample exists, so the result holds. 
\end{pf}

\section{Galois theory (11/05/2021)}
Let $F$ be a field of characteristic $0$, and let $p(x) = x^n + a_{n-1} x^{n-1} 
+ \cdots + a_1 x + a_0 \in F[x]$ be a monic polynomial with roots $\alpha_1, 
\dots, \alpha_n \in \overline{F}$. Let $K = \{p(\alpha_1, \dots, \alpha_n) : 
p(x_1, \dots, x_n) \in F[x_1, \dots, x_n]\}$. Notice that $K$ is a subring of 
$\overline{F}$, because we may consider the surjective $F$-algebra homomorphism 
\begin{align*}
    \varphi : F[x_1, \dots, x_n] &\mapsto \overline{F}, \\ 
    p(x_1, \dots, x_n) &\mapsto p(\alpha_1, \dots, \alpha_n), 
\end{align*}
whose image is $K$ by definition. Note that $\dim_F K < \infty$ since $K$ 
is spanned by $S = \{\alpha_1^{i_1} \cdots \alpha_n^{i_n} : 0 \leq i_1, \dots, 
i_n \leq n\}$. To see this, let $V$ be the $F$-vector subspace of $K$ spanned 
by $S$. If $V \subsetneq K$, then there exists a polynomial $p(x_1, \dots, x_n) 
\in F[x_1, \dots, x_n]$ such that $p(\alpha_1, \dots, \alpha_n) \notin V$. 
Pick such a $p$ of smallest degree lexicographically; that is, 
$x_1^{i_1} \cdots x_n^{i_n} <_{\textrm{lex}} x_1^{j_1} \cdots x_n^{j_n}$ if for some 
$1 \leq m \leq n$, we have $i_k = j_k$ for all $0 \leq k < m$, and 
$i_m < j_m$. Note that this is a total ordering. 

We claim that all monomials of $p$ have degree less than $n$. Suppose we had 
some $x_1^{i_1} \cdots x_s^{i_s} \cdots x_n^{i_n}$ with $i_s \geq n$. Recall 
that $\alpha_s$ is a root of $p$, so we have 
\[ \alpha_s^n + a_{n-1} \alpha_s^{n-1} + \cdots + a_1 \alpha_s + a_0 = 0. \] 
Rearranging the above equation and multiplying by $\alpha_s^{i_n-n}$ gives 
\[ \alpha_s^{i_n} = -a_{n-1} \alpha_s^{i_n-1} - \cdots - a_0 \alpha_s^{i_n-n}. \] 
From this, it follows that 
\[ \alpha_1^{i_1} \cdots \alpha_s^{i_s} \cdots \alpha_n^{i_n} 
= -a_{n-1} \alpha_1^{i_1} \cdots \alpha_s^{i_n-1} \cdots \alpha_n^{i_n} 
- \cdots - a_0 \alpha_1^{i_1} \cdots \alpha_s^{i_n-n} \cdots \alpha_n^{i_n}. \] 
We can repeat this with all other monomials to reach a contradiction. 

Note that $K$ is a field. Indeed, we have $K \subseteq \overline{F}$, so $K$ is 
an integral domain. Moreover, we showed that $\dim_F K < \infty$. Therefore, 
if $a \in K \setminus \{0\}$, then the map $L_a : K \to K$ given by 
$L_a(x) = ax$ is $K$-linear and onto. Since $L_a$ is onto, there exists $b \in K$ 
such that $L_a(b) = 1$, so $ab = ba = 1$. We call $K$ the splitting field
of $p(x) = x^n + a_{n-1} x^{n-1} + \cdots + a_1 x + a_0 = (x - \alpha_1) 
\cdots (x- \alpha_n)$. By construction, it is the smallest field extension of $F$
containing all the roots of $p(x) \in F[x]$. 

Recall that the Galois group $\Gal(K/F)$ is the set of all $F$-algebra 
automorphisms $\sigma : K \to K$ with $\sigma|_F = \id_F$ together with the 
operation of composition. 

\begin{remark}{}
    $\Gal(K/F)$ embeds into $S_n$. To see why, if $\sigma \in \Gal(K/F)$, then 
    $\sigma$ is uniquely determined by the values $\sigma(\alpha_1), \dots, 
    \sigma(\alpha_n)$. Now, we see that $\sigma$ permutes $\alpha_1, \dots, 
    \alpha_n$ because if $p(\alpha) = 0$ for some $\alpha \in K$, then 
    \[ \alpha^n + a_{n-1} \alpha^{n-1} + \cdots + a_1 \alpha + a_0 = 0. \] 
    Since $\sigma$ is $F$-linear, we obtain 
    \begin{align*}
        p(\sigma(\alpha)) 
        &= \sigma(\alpha)^n + a_{n-1} \sigma(\alpha)^{n-1} + \cdots + a_1 \sigma(\alpha) + a_0 \\
        &= \sigma(\alpha^n + a_{n-1} \alpha^{n-1} + \cdots + a_1 \alpha + a_0) \\
        &= \sigma(0) = 0. 
    \end{align*}
\end{remark}

Now, let $G = \Gal(K/F)$. The fundamental theorem of Galois theory tells us that 
if we are given a subgroup $\{\id\} \subseteq H \subseteq G$, then 
$K^H = \{a \in K : \tau(a) = a \text{ for all } \tau \in H\}$ forms a field, 
and the converse is also true. Moreover, we have $H_1 \subseteq H_2$ if and only if 
$K^{H_1} \supseteq K^{H_2}$; that is, this correspondence is inclusion-reversing. 
In particular, notice that $K^G = \{a \in K : \sigma(a) = a \text{ for all } 
\sigma \in \Gal(K/F)\} = F$. 

\begin{theo}[Kronecker]{}
    Consider the polynomial 
    \[ p(x) = x^n + a_{n-1} x^{n-1} + \cdots + a_1 x + a_0 = 
    (x - \alpha_1) \cdots (x - \alpha_n) \in \Z[x]. \] 
    If $|\alpha_i| \leq 1$ for all $i = 1, \dots, n$, then each $\alpha_i$ is 
    either $0$ or a root of unity. 
\end{theo}
\begin{pf}
    First, note that $\alpha_1, \dots, \alpha_n \in \A$. Let $K$ be the splitting 
    field of $p(x) \in F[x]$, and let $G = \Gal(K/\Q)$. For $j \in \Z^+$, define 
    the polynomial 
    \[ p_j(x) = (x - \alpha_1^j) \cdots (x - \alpha_n^j). \] 
    Note that $p_j(x) \in \Q[x]$. Indeed, if $\sigma \in \Gal(K/\Q)$, then 
    by $K$-linearity, we obtain 
    \[ \sigma(p_j(x)) = \sigma((x - \alpha_1^j) \cdots (x - \alpha_n^j)) 
    = (x - \sigma(\alpha_1^j)) \cdots (x - \sigma(\alpha_n^j)) = p_j(x). \] 
    This holds for all $\sigma \in \Gal(K/\Q)$. The coefficients of $p_j(x)$ 
    are all in $K^G = \Q$ (where this equality holds by the fundamental 
    theorem of Galois theory). It follows that $p_j(x) \in \Z[x]$ since 
    $\alpha_1, \dots, \alpha_n \in \A$, and so the coefficients are in 
    $\A \cap \Q = \Z$. 

    Now, write 
    \[ p_j(x) = x^n + a_{n-1,j} x^{n-1} + \cdots + a_{1,j} x + a_{0,j} \in \Z[x]. \] 
    We claim that $|a_{n,j}| \leq \binom{n}{k}$. Indeed, by expanding out 
    our original definition of $p_j(x)$, we have 
    \[ p_j(x) = x^n - (\alpha_1^j + \cdots + \alpha_n^j) x^{n-1} + \cdots 
    + (-1)^k \left( \sum_{1\leq i_1 < \cdots < i_k \leq n} \alpha_{i_1}^j \cdots 
    \alpha_{i_k}^j \right) x^{n-k} + \cdots. \]
    Then we have 
    \[ a_{n-k,j} = (-1)^k \sum_{1\leq i_1 < \cdots < i_k \leq n} \alpha_{i_1}^j \cdots 
    \alpha_{i_k}^j, \] 
    and since $|\alpha_i| \leq 1$ for all $i = 1, \dots, n$, it follows that 
    $|a_{n-k,j}| \leq \binom{n}{k}$. Now, observe that the set 
    $X = \{x^n + a_{n-1}x^{n-1} + \cdots + a_1x + a_0 \in \Z[x] : 
    |a_i| \leq \binom{n}{i} \text{ for all } i = 0, \dots, n-1\}$ is finite, 
    and $p_j(x) \in X$ for all $j \in \Z^+$. In particular, there exist 
    integers $m_1 < m_2$ such that $p_{m_1}(x) = p_{m_2}(x)$. Notice that 
    this implies that $\{\alpha_1^{m_1}, \dots, \alpha_n^{m_1}\} 
    = \{\alpha_1^{m_2}, \dots, \alpha_n^{m_2}\}$. Then we 
    either have $\alpha_i = 0$ or $\alpha_i^{m_2-m_1} = 1$ for each 
    $i = 1, \dots, n$, completing the proof. 
\end{pf}

\section{Lemma for Burnside's theorem, tensor products (11/08/2021)}
Now that we are equipped with Kronecker's theorem, we can prove the following 
lemma we used towards proving Burnside's theorem. In the interest of time, 
we will prove a slightly weaker version of it. 

\begin{lemma}{}
    Let $G$ be a non-abelian simple group. Let $\chi$ be an irreducible character 
    of degree $n \geq 2$. If $g \in G$ with $g \neq 1$ and $\chi(g)/n \in \A$, 
    then $\chi(g) = 0$. 
\end{lemma}
\begin{pf}
    Note that there exists an integer $d$ such that $g^d = 1$. 
    If $\chi$ is the character of some representation $\rho : G \to \GL_n(\C)$, 
    then 
    \[ \rho(g) \sim \begin{pmatrix}
        \omega_1 & & 0 \\ 
        & \ddots & \\ 
        0 & & \omega_n 
    \end{pmatrix}, \] 
    where $\omega_1, \dots, \omega_n$ are $d$-th roots of unity. Indeed, every 
    matrix over $\C$ is triangularizable. But $g^d = 1$ implies that 
    $\rho(g)^d - I = 0$, so the minimal polynomial of $\rho(g)$ divides 
    $x^d - 1$, and the roots of $x^d - 1$ are distinct. A matrix over 
    an algebraically closed field is diagonalizable if and only if its minimal 
    polynomial has distinct roots, so in particular, $\rho(g)$ is diagonalizable. 
    Now, observe that 
    \[ \frac{\chi(g)}{n} = \frac{\omega_1 + \cdots + \omega_n}{n}. \] 
    By the Cauchy-Schwarz inequality, we see that 
    \[ \left| \frac{\chi(g)}{n} \right| = 
    \left\langle (\omega_1, \dots, \omega_n), 
    \left( \frac1n, \dots, \frac1n \right) \right\rangle 
    \leq \sqrt n \cdot \sqrt{\frac1n} = 1, \] 
    with equality if and only if $\omega_1 = \cdots = \omega_n$. We now 
    consider two cases. 

    If $\omega_1 = \cdots = \omega_n =: \omega$, then 
    \[ \rho(g) \sim \begin{pmatrix}
        \omega & & 0 \\ 
        & \ddots & \\ 
        0 & & \omega
    \end{pmatrix} = \omega I. \] 
    Note that $\omega I$ commutes with everything in $\GL_n(\C)$. Hence, 
    we obtain $\rho(ghg^{-1}h^{-1}) = \rho(g)\rho(h)\rho(g)^{-1}\rho(h)^{-1} 
    = I$ for all $h \in G$. This means that $ghg^{-1}h^{-1} \in \ker\rho$ for 
    all $h \in G$. Note that $\ker\rho \trianglelefteq G$, and since $G$ 
    is simple, we either have $\ker\rho = G$ or $\ker\rho = \{1\}$. The 
    first case is impossible since $\rho$ is irreducible of degree $n \geq 2$. 
    Then $\ker\rho = \{1\}$, so $ghg^{-1}h^{-1} = 1$ for all $h \in G$. 
    This implies that $g \in Z(G) \trianglelefteq G$. But we assumed that 
    $g \neq 1$, so $G$ being simple shows that $Z(G) = G$. This is a 
    contradiction since $G$ is non-abelian. 

    Assume now that $\omega_1, \dots, \omega_n$ are not all equal. In this 
    case, we have 
    \[ \left| \frac{\chi(g)}{n} \right| = \left| \frac{\omega_1 + \cdots + 
    \omega_n}{n} \right| < 1. \] 
    Let $\alpha = \chi(g)/n$, and let $K = \Q(e^{2\pi i/d})$. Notice that 
    $\omega_1, \dots, \omega_n \in K$ as they are all $d$-th roots of unity. 
    A $\Q$-algebra automorphism sends $d$-th roots of unity to other $d$-th 
    roots of unity, since $\omega^d = 1$ implies that $\sigma(\omega)^d 
    = \sigma(\omega^d) = \sigma(1) = 1$. Now, if $\sigma \in \Gal(K/\Q)$, then 
    $\sigma$ is $\Q$-linear, so we get 
    \[ \sigma(\alpha) = \frac{\sigma(\omega_1) + \cdots + \sigma(\omega_n)}{n}. \] 
    Moreover, $\sigma(\omega_1), \dots, \sigma(\omega_n)$ are not all the same 
    since $\sigma$ is an automorphism, and hence injective. Therefore, 
    we have $|\sigma(\alpha)| < 1$ for all $\sigma \in \Gal(K/\Q)$. 

    Assume that $\alpha \in \A$, so $\sigma(\alpha) \in \A$ for all $\sigma 
    \in \Gal(K/\Q)$. Note that the polynomial 
    \[ \prod_{\sigma \in \Gal(K/\Q)} (x - \sigma(\alpha)) \] 
    lives in $\Q[x] \cap \A[x] = \Z[x]$. Note that $\alpha$ is a root of this 
    polynomial with $|\alpha| < 1$. By Kronecker, $\alpha$ is 
    either $0$ or a root of unity, with the latter case being impossible since 
    $|\alpha| \neq 1$. It follows that $\alpha = |\chi(g)|/n$ must be $0$, 
    and hence $\chi(g) = 0$. 
\end{pf}

We now introduce tensor products. Let $R$ be a ring, let $M = M_R$ be a 
right $R$-module, and let $N = {}_R N$ be a left $R$-module. We can form 
the tensor product $M \otimes_R N$ of $M$ and $N$. 

Given an abelian group $A$, we say that a map $f : M \times N \to A$ 
is {\bf bilinear} if 
\begin{enumerate}[(i)]
    \item $f(m_1 + m_2 \cdot a, n) = f(m_1, n) + a \cdot f(m_2, n)$ 
          for all $m_1, m_2 \in M$, $n \in N$, and $a \in \Z$; 
    \item $f(m, n_1 + an_2) = f(m, n_1) + a \cdot f(m, n_2)$ for all 
          $m \in M$, $n_1, n_2 \in N$, and $a \in \Z$; 
    \item $f(m \cdot r, n) = f(m, r \cdot n)$ for all $m \in M$, $n \in N$, and 
          $r \in R$. 
\end{enumerate}
We can think of bilinear maps as an abstraction of multiplication between modules. 

\begin{remark}{}
    Note that we need the left and right pairing of the modules to make this work 
    in general. Suppose that $M$ and $R$ are both right $R$-modules. We would like 
    to impose that $f(m \cdot r, n) = f(m, n \cdot r)$ for all $m \in M$, $n \in N$, and 
    $r \in R$. But this means that for all $r, s \in R$, we have $f(m(rs), n) 
    = f(m, n(rs))$, and also 
    \[ f(m(rs), n) = f(mr, ns) = f(m, n(sr)). \] 
    In particular, we require that $f(m, n(rs - sr)) = 0$ for all $m \in M$, 
    $n \in N$, and $r, s \in R$. This is not an issue for commutative rings, 
    but there are rings $R$ such that $1 \in [R, R]$, in which case the only 
    bilinear map is the zero map. The same issue occurs with two left $R$-modules. 
\end{remark}

We now formally define tensor products. 

\begin{defn}{}
    Let $T$ be the free abelian group on generators $e_{(m,n)}$ where $(m, n) 
    \in M \times N$. That is, we have 
    \[ T = \bigoplus_{(m, n) \in M \times N} \Z e_{(m,n)}. \] 
    Let $U \subseteq T$ be the $\Z$-submodule spanned by the relations 
    \begin{enumerate}[(i)]
        \item $e_{(m_1 + m_2, n)} - e_{(m_1, n)} - e_{(m_2, n)}$; 
        \item $e_{(m, n_1 + n_2)} - e_{(m, n_1)} - e_{(m, n_2)}$; and 
        \item $e_{(mr, n)} = e_{(m, rn)}$. 
    \end{enumerate}
    Then the {\bf tensor product} of $M$ and $N$ is defined to be 
    $M \otimes_R N := T/U$. 
\end{defn}