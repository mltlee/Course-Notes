\section{Burnside-Schur (12/01/2021)}
Now, we'll turn to proving the strengthening of Burnside's result 
due to Schur. First, we need to say a few words on transcendence bases. 

Let $K/F$ be a field extension. A subset $S \subseteq K$ is said to be 
{\bf algebraically independent} over $F$ if whenever $\{a_1, \dots, a_\ell\} 
\subseteq S$ is a finite subset and $P(t_1, \dots, t_\ell) \in 
F[t_1, \dots, t_\ell]$, then $P(a_1, \dots, a_\ell) = 0$ implies that 
$P \equiv 0$. 

For example, $\{\pi\} \subseteq \C$ is algebraically independent over $\Q$, 
whereas $\{\sqrt 2, e\}$ is not since we can take $P(t_1, t_2) = t_1^2 - 2$. 
An open problem: is $\{e, \pi\}$ algebraically independent over $\Q$? 

A {\bf transcendence basis} for $K/F$ is then a maximal algebraically 
independent subset $S \subseteq K$ over $F$, and these exist by Zorn's lemma. 

\begin{exercise}{}
    All transcendence bases for $K/F$ have the same cardinality.
\end{exercise}

We say that a field extension $K/F$ is {\bf finitely generated} if 
there exists $s \in \N$ and $a_1, \dots, a_s \in \K$ such that 
$K = F(a_1, \dots, a_s)$. 

Notice that $\Q(\sqrt 2, \sqrt 3)/\Q$ and $\Q(\pi, e, \sqrt 2)/\Q$ 
are finitely generated extensions, but $\C/\Q$ is not finitely generated since 
$\Q(a_1, \dots, a_s)$ is countable for any $a_1, \dots, a_s \in \C$. 

If $K/F$ is a finitely generated extension, there exists $r \in \N$ and 
$t_1, \dots, t_r \in K$ which are algebraically independent over $F$ 
such that $[K : F(t_1, \dots, t_r)] < \infty$. To see why, 
write $K = F(a_1, \dots, a_m)$. Pick $S \subseteq \{a_1, \dots, a_m\}$ 
to be a maximal algebraically independent subset. After relabelling, we 
may assume that $S = \{a_1, \dots, a_r\}$. Let $E = F(a_1, \dots, a_r) 
\cong F(t_1, \dots, t_r)$. Notice that by construction, each 
$a_i$ is algebraic over $E$ for $i = r+1, \dots, m$. Since 
$K = E(a_{r+1}, \dots, a_m)$, we have 
\[ [K : E] = [E(a_{r+1}, \dots, a_m) : E(a_{r+1}, \dots, a_{m-1})] 
\cdots [E(a_{r+1}) : E(a_r)] \leq \prod_{i>r} [E(a_i) : E] < \infty. \] 
Let's now state the Burnside-Schur theorem. 

\begin{theo}[Burnside-Schur]{}
    Let $G \leq \GL_n(\C)$ be a finitely generated torsion subgroup. Then 
    $G$ is finite. 
\end{theo}

Now, let's suppose $G \leq \GL_n(\C)$ is generated by $\{A_1^{\pm1}, \dots, 
A_r^{\pm1}\}$, where 
\begin{align*}
    A_i &= \begin{pmatrix}
        a_{11}^{(i)} & \cdots & a_{1n}^{(i)} \\ 
        \vdots & \ddots & \vdots \\ 
        a_{n1}^{(i)} & \cdots & a_{nn}^{(i)}
    \end{pmatrix}, & A_i^{-1} &= \begin{pmatrix}
        b_{11}^{(i)} & \cdots & b_{1n}^{(i)} \\ 
        \vdots & \ddots & \vdots \\ 
        b_{n1}^{(i)} & \cdots & b_{nn}^{(i)}
    \end{pmatrix}. 
\end{align*}
Let $K = \Q(\{a_{ij}^{(s)}, b_{ij}^{(s)} : 1 \leq i, j \leq n,\, 1 \leq s \leq r
\})$, which is a finitely generated extension of $\Q$ with $K \subseteq \C$. 
In fact, we have $G \leq \GL_n(K)$ since $\GL_n(K)$ is a group containing all 
generators of $G$. 

Suppose $A \in \GL_n(K)$ has finite order. We have seen that 
\[ A \sim \begin{pmatrix}
    \omega_1 & & 0 \\ 
    & \ddots & \\ 
    0 & & \omega_n
\end{pmatrix} \] 
where $\omega_1, \dots, \omega_n$ are the eigenvalues of $A$, and they are 
$n$-th roots of unity. 

We have showed that there are only finitely many roots of unity $\omega$ with 
$[\Q(\omega) : \Q] \leq n$. Schur proved that this extends to finitely 
generated extensions of $\Q$. 

\begin{theo}[Schur]{}
    Let $K$ be a finitely generated extension of $\Q$. For each $n \geq 1$, 
    there are only finitely many roots of unity $\omega$ with $[K(\omega) : K] \leq n$. 
\end{theo}

By applying the same argument as Burnside, the Burnside-Schur theorem follows. 

Now, let's prove Schur's result. Suppose that $K$ has transcendence degree 
$M < \infty$. It suffices to prove the result for a purely transcendental 
extension $K = \Q(t_1, \dots, t_s)$. To see why, notice that if we 
have a general finitely generated extension $K/\Q$, then 
\[ [K(\omega) : K] \leq [K(\omega) : \Q(t_1, \dots, t_s)] = 
[K(\omega) : K][K : \Q(t_1, \dots, t_s)] \leq n\cdot M. \] 

\begin{lemma}{}
    Let $t_1, \dots, t_s$ be algebraically independent over $\Q$. Then 
    \[ [\Q(t_1, \dots, t_s)(\omega)] = \Q(t_1, \dots, t_s)] = 
    [\Q(\omega) : \Q]. \] 
\end{lemma}
\begin{pf}
    Notice that $[\Q(t_1, \dots, t_s)(\omega)] = \Q(t_1, \dots, t_s)]$ 
    is the degree of the minimal polynomial of $\omega$ with coefficients 
    in $\Q(t_1, \dots, t_s)$, which is at most the degree of the minimal 
    polynomial of $\omega$ with coefficients in $\Q$. 

    For the reverse inequality, suppose $m$ is the degree of the minimal 
    polynomial of $\omega$ with coefficients in $\Q$. Then we can write 
    \[ P_m(t_1, \dots, t_s) \omega^m + \cdots + P_0(t_1, \dots, t_s) = 0 \] 
    for some $P_0, \dots, P_m \in \Q[t_1, \dots, t_s]$. We can rewrite this 
    in the form 
    \[ \sum_{i_1,\dots,i_s \geq 0} q_{i_1\cdots i_s}(\omega) t_1^{i_1} 
    \cdots t_s^{i_s} = 0 \] 
    where $\deg q_{i_1\cdots i_s} \leq m$. Notice that 
    $q_{i_1\cdots i_s}(\omega) = 0$ for all $i_1, \dots, i_s \geq 0$ since 
    \[ \sum_{i_1,\dots, i_s \geq 0} 
    \left( \sum_{\sigma \in \Gal(\Q(\omega)/\Q)} \sigma(\omega^p 
    q_{i_1\cdots i_s}(\omega)) \right) t_1^{i_1} \cdots t_s^{i_s} = 0. \] 
    Since $t_1, \dots, t_s$ is algebraically independent over $\Q$, the 
    above coefficients must be $0$. Hence, we have $\deg q_{i_1\cdots i_s} \leq
    [\Q(\omega) : \Q]$, so $m \leq [\Q(\omega) : \Q]$. 
\end{pf}
