\section{Opposite ring, spectrum of a ring, Nullstellensatz (09/22/2021)}

\begin{defn}{}
Let $S$ be a ring. The {\bf opposite ring} $S^{\op}$ of $S$ is defined to be another ring with 
the same elements and addition as $S$, but the multiplication $* : S^{\op} \times S^{\op} \to S^{\op}$ is given by 
\[ s_1 * s_2 := s_2 \cdot s_1, \]
where $\cdot$ denotes the multiplication in $S$. 
\end{defn}

\begin{remark}{}
\begin{enumerate}[(1)]
    \item If $S$ is commutative, then $S^{\op}$ is the same as $S$ (since $*$ is the same as $\cdot$).
    \item If $\Delta$ is a division ring, then $\Delta^{\op}$ is also a division ring. Indeed, 
    let $a \in \Delta^{\op}$ be non-zero. Then there exists $b \in \Delta$ such that 
    $a \cdot b = b \cdot a = 1_\Delta$, so we have $b * a = a * b = 1_{\Delta^{\op}}$. Hence, 
    $a$ is invertible. 
\end{enumerate}
\end{remark}

\begin{exercise}{}
Let $\Delta$ be a division ring. If $M$ is an $n$-dimensional left $\Delta$-vector space, then 
$\End_\Delta(M) \cong M_n(\Delta^{\op})$. Note that if $\Delta = k$ for a field $k$, then this is just saying that $\End_k(M) \cong M_n(k)$, which is a familiar fact.

Hint: Construct the map $\Psi : \End_\Delta(M) \to M_n(\Delta^{\op})$ as follows: pick a basis 
$\{m_1, \dots, m_n\}$ as a left $\Delta$-vector space. For $f \in \End_\Delta(M)$, we have 
\[ f(m_j) = \sum_{i=1}^n a_{ij} m_i \]
where $a_{ij} \in \Delta$ since $f$ is $\Delta$-linear. Define $\Psi(f) := (a_{ij})$, and 
show that $\Psi(f \circ g) = \Psi(g) \cdot \Psi(f) = \Psi(f) * \Psi(g)$. 
\end{exercise}

Last time, we were proving Theorem 6.5, which stated that if $R$ is a prime left Artinian ring, 
then $R \cong M_n(D)$ for some $n \geq 1$ and division ring $D$. We can now finish the proof. 

We can set $n := \dim_\Delta M$ as we showed that $M$ is a finite-dimensional left $\Delta$-vector space.
Then we have 
\[ R \cong \End_\Delta(M) \cong M_n(\Delta^{\op}) \]
by Exercise 7.3, and we are done since $D = \Delta^{\op}$ is a division ring. \qed

\begin{cor}{}
Let $k$ be an algebraically closed field, and let $R$ be a prime finite-dimensional $k$-algebra. Then 
$R \cong M_n(k)$. 
\end{cor}
\begin{pf}
Since $R$ is a finite-dimensional $k$-algebra, it is left Artinian (see Remark 5.7). Moreover,
Proposition 3.1 shows that $\Delta = \End_R(M) \cong k$ where $M$ is a simple left 
$R$-module, since $k$ is algebraically closed. 
\end{pf}

For the rest of the lecture, we will consider the connection between prime ideals and nil ideals. 

\begin{defn}{}
We define the {\bf spectrum} of a ring $R$ to be the set of all prime ideals of $R$,
denoted $\Spec(R)$. 
\end{defn}

\begin{exmp}{}
For $R = \Z$, we have $\Spec(\Z) = \{p\Z : p \text{ prime}\} \cup \{(0)\}$. 
\end{exmp}

\begin{exmp}{}
For $R = M_n(\Delta)$ for $\Delta$ a division ring, we have $\Spec(M_n(\Delta)) = \{(0)\}$ as we 
showed that $M_n(\Delta)$ is a simple ring in Proposition 6.3.
\end{exmp}

\begin{exmp}{}
For $R = \C[x, y]$, we have
\[ \Spec(\C[x, y]) = \{(0)\} \cup \{(f) : f \text{ irreducible}\} \cup \{(x-a, y-b) : a, b \in \C\}. \]
Note that $(0)$ is a prime ideal as $\C[x, y]$ is an integral domain. Every maximal ideal of 
$\C[x, y]$ is of the form $(x-a, y-b)$, and this is due to the Nullstellensatz which we will prove. 
The other prime ideals are generated by irreducible polynomials $f$. 
\end{exmp}

\begin{theo}[Nullstellensatz]{}
Let $\mathfrak{M}$ be a maximal ideal of $\C[x_1, \dots, x_n]$. Then there exist 
$a_1, \dots, a_n \in \C$ such that 
\[ \mathfrak M = (x_1-a_1, \dots, x_n-a_n). \]
\end{theo}
\begin{pf}
Let $F = \C[x_1, \dots, x_n]/\mathfrak M$, which is a field because $\mathfrak M$ is maximal. 
Note that $F \supseteq \C$. Moreover, we have $\dim_{\C} F \leq \aleph_0$ since 
$\C[x_1, \dots, x_n]$ has a basis $\{x_1^{i_1} \cdots x_n^{i_n} : i_1, \dots, i_n \geq 0\} 
\cong \N^n$, which is countable. 

We claim that $F$ is algebraic over $\C$. That is, if $t \in F$, then $F$ satisfies $p(t) = 0$ 
for some $p(x) \in \C[x] \setminus \{0\}$. Note that this implies $F = \C$ since $\C$ is 
algebraically closed. 

Let $t \in F \setminus \C$. Consider the set 
\[ S = \left\{ \frac1{t-\lambda} : \lambda \in \C\right\} \subseteq F. \] 
Then $S$ is linearly dependent since $\C$ is uncountable while $\dim_\C F \leq \aleph_0$. 
Thus, there exist distinct elements $\lambda_1, \dots, \lambda_n \in \C$ and $c_1, \dots, c_n 
\in \C$, not all zero, such that 
\[ \frac{c_1}{t - \lambda_1} + \cdots + \frac{c_n}{t - \lambda_n} = 0. \]
Multiplying by $\prod_{i=1}^n (t - \lambda_i)$ gives 
\[ \sum_{i=1}^n c_i \prod_{j\neq i} (t - \lambda_j) = p(t) = 0, \]
where $p(x)$ is a polynomial in $\C[x]$. Note that $p(x)$ is non-trivial since 
\[ p(\lambda_i) = c_i \prod_{j\neq i} (\lambda_i - \lambda_j) \neq 0. \]
This proves the claim, so $\C[x_1, \dots, x_n]/\mathfrak M \cong \C$. Hence, there is a homomorphism 
\[ \phi : \C[x_1, \dots, x_n] \to \C \] 
such that $\ker\phi = \mathfrak M$. Letting $a_i = \phi(x_i)$ for all $1 \leq i \leq n$, we have 
\[ \phi(x_1^{i_1} \cdots x_n^{i_n}) = a_1^{i_1} \cdots a_n^{i_n} \]
as $\phi$ is a homomorphism. Since the $x_1^{i_1} \cdots x_n^{i_n}$ form a basis for 
$\C[x_1, \dots, x_n]$, it then follows that $\phi$ is simply the evaluation at 
$(a_1, \dots, a_n)$ map. In particular, $\mathfrak M = (x_1 - a_1, \dots, x_n - a_n)$ as desired.
\end{pf}

\begin{theo}{}
Let $R$ be a ring. Then the intersection of all prime ideals 
\[ \bigcap_{P \in \Spec(R)} P \]
is a nil ideal of $R$. 
\end{theo}
\begin{pf}
Let $N = \bigcap_{P \in \Spec(R)} P$. Suppose that there exists some $x \in N$ which is not nilpotent,
and let 
\[ {\cal T} = \{1, x, x^2, x^3, \dots \}. \]
Notice that $0 \notin {\cal T}$ since $x$ is not nilpotent. Let 
\[ S = \{I \trianglelefteq R : I \cap {\cal T} = \varnothing\}. \]
We have $(0) \in S$, so $S$ is non-empty. We leave it as an exercise to show that $S$ has a maximal 
element $P$. 

We claim that $P$ is a prime ideal. Suppose otherwise, so that there exists $a, b \notin P$ 
such that $aRb \subseteq P$. Since $a \notin P$, we have $RaR + P \supsetneq P$, and similarly, 
$RbR + P \supsetneq P$ as $b \notin P$. As $P$ is maximal in $S$, this implies that 
$RaR + P \notin S$, so there exists $i \geq 1$ such that $x^i \in RaR + P$. 
Analogously, we have $RbR + P \notin S$, so there exists $j \geq 1$ such that $x^j \in RbR + P$. 
It follows that 
\[ x^{i+j} = x^i \cdot x^j \in (RaR + P)(RbR + P) \subseteq R(aRb)R + P \subseteq P. \]
But this is a contradiction since $P \in S$ implies that 
$P \cap {\cal T} = \varnothing$, but we have $x^{i+j} \in P \cap {\cal T}$. 
Thus, $P$ is a prime ideal, proving the claim. 

Now, we find that $x \in N \subseteq P$ since $P$ is prime and $N$ is the intersection of all 
the prime ideals. However, $x \in {\cal T}$, which again contradicts the fact that 
$P \cap {\cal T} = \varnothing$. We conclude that every $x \in N$ must be nilpotent, so 
$N$ is a nil ideal, as required.
\end{pf}

\section{Sun-tzu, the Artin-Wedderburn theorem (09/24/2021)}
We almost have all the tools we need to prove the Artin-Wedderburn theorem. 
First, we make a remark and prove a couple of results that we need. 

\begin{remark}{}
If $R$ is a left Artinian ring, then so is $R/P$ where $P$ is an ideal of $R$ by correspondence. 
Moreover, if $P$ is a prime ideal, then $R/P$ is a prime ring. The converse of this holds when 
$R$ is commutative. 
\end{remark}

\begin{lemma}{}
Let $R$ be a left Artinian ring. 
\begin{enumerate}[(1)]
    \item Every prime ideal of $R$ is a maximal ideal. 
    \item There are only finitely many prime ideals of $R$. 
    \item Let $P_1, \dots, P_s$ be all the prime ideals of $R$. Then for all $i = 1, \dots, s$, we have 
    \[ P_i + \bigcap_{j \neq i} P_j = R. \]
\end{enumerate}
\end{lemma}
\begin{pf}~
\begin{enumerate}[(1)]
    \item Let $P$ be a prime ideal of $R$. By Remark 8.1, $R/P$ is a prime left Artinian ring 
    and hence 
    \[ R/P \cong M_n(D) \]
    for a division ring $D$ and some $n \geq 1$ by Theorem 6.5. We know that $M_n(D)$ is simple by 
    Proposition 6.3, so its only ideals are $(0)$ and $M_n(D)$. In particular, $R/P$ is also simple. 
    By correspondence, there are only two ideals of $R$ that contain $P$. We already know that 
    $P$ and $R$ are ideals that contain $P$, so they are in fact all of them. Thus, $P$ is 
    maximal. 
    
    \item Suppose towards a contradiction that we have infinitely many distinct prime ideals 
    $P_1, P_2, \dots$ of $R$. Recall that if $I$ and $J$ are ideals of $R$, we define 
    \[ IJ = \left\{\sum_{k=1}^s i_kj_k : s \geq 1,\, i_k \in I,\, j_k \in J\right\}. \]
    We have a descending chain of ideals 
    \[ P_1 \supseteq P_1P_2 \supseteq P_1P_2P_3 \supseteq \cdots \]
    and since $R$ is left Artinian, this chain must terminate. Thus, there exists $n \geq 1$ such that 
    \[ P_1 \cdots P_n = P_1 \cdots P_n P_{n+1} \subseteq P_{n+1}. \]
    Since $P_1 \neq P_{n+1}$ and $P_1$ is a maximal ideal by (1), there exists some element 
    $a_1 \in P_1 \setminus P_{n+1}$. Similarly, for all $i = 1, \dots, n$, there exists 
    $a_i \in P_i \setminus P_{n+1}$. Then, we see that 
    \[ (a_1R)(a_2R)(a_3R) \cdots (a_{n-1}R) a_n \subseteq P_1P_2P_3 \cdots P_n \subseteq P_{n+1}, \]
    so we have 
    \begin{equation}
        a_1Ra_2R \cdots a_{n-1}Ra_n \subseteq P_{n+1}.
    \end{equation} 
    Since neither $a_1$ nor $a_2$ are in $P_{n+1}$, there exists $r_1 \in R$ such that 
    $a_1r_1a_2 \notin P_{n+1}$. This is because $P_{n+1}$ is a prime ideal of $R$, so $a_1, a_2 \notin P_{n+1}$ 
    implies that $a_1 R a_2 \nsubseteq P_{n+1}$. By (8.1), we find that 
    \[ (a_1Ra_2)Ra_3R \cdots a_{n-1}Ra_n \subseteq P_{n+1}. \]
    Now, $a_1ra_2$ and $a_3$ are both not in $P_{n+1}$, so there exists $r_2 \in R$ such that 
    $a_1r_1a_2r_2a_3 \notin P_{n+1}$ since $P_{n+1}$ is a prime ideal. Continuing in this manner, 
    we have elements 
    $r_1, r_2, \dots, r_{n-1} \in R$ such that 
    \[ a_1r_1a_2r_2 \cdots a_{n-1}r_{n-1}a_n \notin P_{n+1}, \]
    which contradicts (8.1) since 
    \[ a_1r_1a_2r_2 \cdots a_{n-1}r_{n-1}a_n \in a_1Ra_2R \cdots a_{n-1}Ra_n. \]
    We conclude that $R$ must have finitely many prime ideals.
    
    \item Let $P_1, \dots, P_s$ be the prime ideals of $R$, and fix $1 \leq i \leq s$. We claim that 
    \[ I_i := P_i + \bigcap_{j\neq i} P_j = R. \]
    Notice that $P_i \subseteq I_i \subseteq R$. Since $P_i$ is a maximal ideal by part (1), we either
    have $I_i = P_i$ or $I_i = R$. Suppose towards a contradiction that $I_i = P_i$, and recall the fact
    that $I + J = I$ if and only if $J \subseteq I$. Then we have 
    \[ \bigcap_{j\neq i} P_j \subseteq P_i, \]
    which implies that 
    \[ P_1P_2 \cdots P_{i-1}P_{i+1} \cdots P_s \subseteq \bigcap_{j\neq i} P_j \subseteq P_i. \]
    Using the same argument as the proof of (2), we can show that $P_1P_2 \cdots P_{i-1}P_{i+1} 
    \cdots P_s$ cannot be contained in the prime ideal $P_i$ as each ideal $P_j$ with $j \neq i$ is 
    prime. This contradicts our assumption that $I_i = P_i$, so we must have 
    $I_i = R$. \qedhere 
\end{enumerate}
\end{pf}

\begin{theo}[Sun-tzu]{}
Let $R$ be a ring, and let $I_1, \dots, I_s$ be two-sided ideals of $R$ such that 
\begin{enumerate}[(i)]
    \item $\bigcap_{j=1}^s I_j = (0)$, and 
    \item for all $i = 1, \dots, s$, we have $I_i + \bigcap_{j\neq i} I_j = R$. 
\end{enumerate}
Then we have 
\[ R \cong \prod_{i=1}^s R/I_i. \]
\end{theo}
\begin{pf}
For $i = 1, \dots, s$, we have a canonical surjection 
\begin{align*} \pi_i : R &\to R/I_i \\ r &\mapsto r + I_i, \end{align*}
which is a ring homomorphism. We define 
\begin{align*} \Psi : R &\to R/I_1 \times \cdots \times R/I_s \\ r &\mapsto (\pi_1(r), \dots, \pi_s(r)). 
\end{align*}
Note that since $\pi_1, \dots, \pi_s$ are ring homomorphisms, $\Psi$ is also a ring homomorphism. 
We now show that $\Psi$ is an isomorphism. Notice that 
\[ \ker\Psi = \{r \in R : \Psi(r) = 0\} = \{r \in R : \pi_1(r) = \cdots = \pi_s(r) = 0\} 
= \bigcap_{j=1}^s I_j = (0) \]
where the last equality follows from (i), so $\Psi$ is injective. Now, fix $1 \leq i \leq s$. 
By (ii), we have 
\[ I_i + \bigcap_{j\neq i} I_j = R, \]
so we can find $a_i \in I_i$ and $b_i \in \bigcap_{j\neq i} I_j$ such that $a_i + b_i = 1$. 
We can write $b_i = 1 - a_i$, so we find that 
\[ \pi_i(b_i) = \pi_i(1) - \pi_i(a_i) = (1 + I_i) - (0 + I_i) = 1 + I_i \]
since $a_i \in I_i$. 
On the other hand, when $n \neq i$, we have $b_i \in \bigcap_{j\neq i} I_j \subseteq I_n$, which 
gives $\pi_n(b_i) = 0 + I_n$. It follows that 
\begin{align*} \Psi(b_i) &= (\pi_1(b_i), \pi_2(b_i), \dots, \pi_i(b_i), \dots, \pi_s(b_i)) \\
&= (0 + I_1, 0 + I_2,\, \dots , 0 + I_{i-1}, 1 + I_i, 0 + I_{i+1},\, \dots, 0 + I_s). \end{align*}
Therefore, given $r_1, \dots, r_s \in R$, we have $r = r_1b_1 + \cdots + r_sb_s \in R$, and we see that
\[ \Psi(r) = \Psi(r_1b_1 + \cdots + r_sb_s) = \Psi(r_1 + I_1,\, \dots, r_s + I_s). \]
We conclude that $\Psi$ is surjective, so it is an isomorphism, as required. 
\end{pf}

\begin{theo}[Artin-Wedderburn]{}
Let $R$ be a left Artinian ring. If $R$ has no nonzero nil ideals, then there exists $s \geq 1$, 
division rings $D_1, \dots, D_s$, and integers $n_1, \dots, n_s \geq 1$ such that 
\[ R \cong \prod_{i=1}^s M_{n_i}(D_i). \]
\end{theo}
\begin{pf}
Let $R$ be a left Artinian ring with no nonzero nil ideals. We showed in Lemma 8.1 that $R$ 
has finitely many prime ideals; call them $P_1, \dots, P_s$. Moreover, observe that $\bigcap_{j=1}^s P_j$ is a nil ideal by Theorem 7.10, and since $R$ has no nonzero nil ideals, it must be that 
$\bigcap_{j=1}^s P_j = (0)$. We also showed that $P_i + \bigcap_{j\neq i} P_j = R$ for all 
$i = 1, \dots s$ in Lemma 8.2. It follows from Sun-tzu that 
\[ R \cong \prod_{i=1}^s R/P_i. \] 
We know that $R/P_i$ is a prime left Artinian ring by Remark 8.1, so Theorem 6.5 implies that 
$R/P_i \cong M_{n_i}(D_i)$ for some $n_i \geq 1$ and division ring $D_i$. Thus, we conclude that 
\[ R \cong \prod_{i=1}^s R/P_i \cong \prod_{i=1}^s M_{n_i}(D_i). \qedhere \]
\end{pf}

\section{Corollary of Sun-tzu, Maschke's theorem (09/27/2021)}

Let's first look at a corollary of Sun-tzu (Theorem 8.3).

\begin{remark}{}
Let $k$ be a field. If $R$ is a $k$-algebra, then each canonical surjection $\pi_i : R \to R/I_i$ 
in the proof of Sun-tzu is a $k$-algebra homomorphism, which means that the map
$\Psi : R \to R/I_1 \times \cdots \times R/I_s$ is also a $k$-algebra homomorphism. 
\end{remark}

\begin{cor}{}
Let $k$ be an algebraically closed field, and let $R$ be a finite-dimensional $k$-algebra with no nonzero
nil ideals. Then we have 
\[ R \cong \prod_{i=1}^s M_{n_i}(k) \]
for some $s \geq 1$ and integers $n_1, \dots, n_s \geq 1$. 
\end{cor}
\begin{pf}
Since $R$ is a finite-dimensional $k$-algebra, we see that $R$ is left Artinian (Remark 5.7).
Hence, $R$ has only finitely many prime ideals (Lemma 8.2), say $P_1, \dots, P_s$. Moreover, 
since $R$ has no nonzero nil ideals, we have $\bigcap_{j=1}^s P_j = (0)$ (Theorem 7.10), and 
$P_i + \bigcap_{j\neq i} P_j = R$ for all $i = 1, \dots, s$ since $R$ is left Artinian (Lemma 8.2). 
By Sun-tzu, we have 
\[ R \cong R/P_1 \times \cdots \times R/P_s. \]
Then, since $k$ is algebraically closed and each $R/P_i$ is a prime finite-dimensional $k$-algebra, 
Corollary 7.4 implies that 
\[ R \cong R/P_1 \times \cdots \times R/P_s \cong M_{n_1}(k) \times \cdots \times M_{n_s}(k). 
\qedhere \]
\end{pf}

Now that we have the Artin-Wedderburn theorem in our toolkit, we can finally get started with 
some representation theory. Let $G$ be a finite group with $|G| = n$, and let $k$ 
be an algebraically closed field. Recall that $V := k[G]$ is the group algebra of the group $G$ over 
the field $k$, with $\dim_k k[G] = |G| = n$. Moreover, we have the embedding 
\[ \Phi : k[G] \to \End_k(V) \cong M_n(k), \]
where we send each $g \in G$ to the left multiplication map 
\begin{align*}
    L_g : V &\to V \\ x &\mapsto g \cdot x
\end{align*}
and extend linearly over $k$. We call this the {\bf left regular representation} of $G$. 

Next, let's recall some linear algebra. Let $W$ be a vector space with basis ${\cal B} 
= \{w_1, \dots, w_n\}$, and let $T : W \to W$ be a linear map. Then we can define a matrix 
$[T]_{\cal B}$ by setting $[T]_{\cal B} = (c_{ij})$ where 
\[ Tw_j = \sum_{i=1}^n c_{ij} w_i. \] 
Observe that the trace of $T$ is given by 
\[ \Tr(T) = \sum_{j=1}^n c_{jj}, \]
which is the sum of the coefficients of $w_j$ in $Tw_j$ (and this is independent of the basis 
${\cal B}$). 

Now, for the map $L_g : V \to V$ with $g \in G$, what is $\Tr(L_g)$? First, note that the elements of 
$G$ form a basis over $V = k[G]$; that is, we have ${\cal B} = \{g_1, \dots, g_n\} = G$ in this case. 
Observe that 
\begin{equation} L_g(g_i) = g \cdot g_i = 1 \cdot g \cdot g_i + \sum_{h \in G \setminus \{gg_i\}} 0 \cdot h. \end{equation}
For $j = 1, \dots, n$, the coefficient of $g_j$ in $gg_j$ is $1$ if $g = 1$, and $0$ if 
$g \neq 1$. Indeed, if $g \neq 1$, then $g_j$ would be different than $gg_j$, so it would have 
coefficient $0$ in equation (9.1). This implies that 
\[ \Tr(L_g) = \sum_{j=1}^n \begin{cases} 1 & \text{if } g = 1 \\ 0 & \text{if } g \neq 1 \end{cases} 
= \begin{cases} n & \text{if } g = 1 \\ 0 & \text{if } g \neq 1. \end{cases} \]

\begin{theo}[Maschke]{}
Let $k$ be a field (not necessarily algebraically closed), and let $G$ be a finite group. If 
$\ch(k) \nmid |G|$, then $k[G]$ has no nonzero nil ideals. 
\end{theo}
\begin{pf}
Suppose towards a contradiction that $k[G]$ has a nonzero nil two-sided ideal $N$. Pick a nonzero 
element 
\[ x = \sum_{g \in G} \alpha_g g \in N. \]
Since $x$ is nonzero, there exists some $g_0 \in G$ such that $\alpha_{g_0} \neq 0$. As 
$N$ is an ideal of $k[G]$, we also have $x \cdot g_0^{-1} \in N$. The coefficient of $1$ in 
$x \cdot g_0^{-1}$ is $\alpha_{g_0} \neq 0$. Thus, we can assume without loss of generality that 
$x = \sum_{g \in G} \alpha_g g \in N$ has $\alpha_1 \neq 0$. 

Let $A \in M_n(k)$ be nilpotent so that $A^n = 0$. Let $v$ be an eigenvector of $A$ with eigenvalue $\lambda$. Then we have 
\[ Av = \lambda v \implies A^2v = \lambda^2v \implies \cdots \implies A^n v = \lambda^n v, \]
which implies that $\lambda = 0$ since $A^n = 0$. In particular, $\Tr(A)$ is the sum of the 
eigenvalues of $A$ (with multiplicity), and the above argument shows that all the eigenvalues of 
$A$ are $0$, so $\Tr(A) = 0$.

Now, consider the injective $k$-algebra homomorphism 
\begin{align*}
    \Phi : k[G] \to \End_k(V) \cong M_n(k) : \sum_{g \in G} \beta_g \cdot g \mapsto \sum_{g \in G} \beta_g \cdot L_g
\end{align*}
we discussed at the beginning of this lecture (the left regular representation of $G$). We had an 
element 
\[ x = \sum_{g \in G} \alpha_g g \in N \]
with $\alpha_1 \neq 0$. Note that $x$ is nilpotent since $N$ is a nil ideal; say
$x^j = 0$ for some $j \geq 1$. Then we have 
\[ \Phi(x)^j = \Phi(x^j) = \Phi(0) = 0, \]
so $\Phi(x)$ is also nilpotent. Our discussion above implies that $\Tr(\Phi(x)) = 0$. However, 
we also see that 
\[ \Tr(\Phi(x)) = \Tr\left( \sum_{g \in G} \alpha_g \cdot L_g \right)
= \sum_{g \in G} \alpha_g \cdot \Tr(L_g) = |G| \cdot \alpha_1 \neq 0 \]
since $\alpha_1 \neq 0$ and $\ch(k) \nmid |G|$. This is a contradiction, so $k[G]$ has 
no nonzero nil ideals. 
\end{pf}

\begin{remark}{}
Question 1 of Assignment 1 shows that the converse of Maschke's theorem also holds. Indeed, we 
proved that if $\ch(k) \mid |G|$, then $k[G]u$ is a nonzero nil ideal where $u = \sum_{g \in G} g$. 
\end{remark}

\begin{cor}{}
Let $k$ be an algebraically closed field, and let $G$ be a finite group. If $\ch(k) \nmid |G|$, then 
there exists $s \geq 1$ and integers $n_1, \dots, n_s \geq 1$ such that 
\[ k[G] \cong M_{n_1}(k) \times \cdots \times M_{n_s}(k). \]
\end{cor}
\begin{pf}
By Maschke's theorem, we know that $k[G]$ has no nonzero nil ideals, and it is left Artinian 
as it is a finite-dimensional $k$-algebra. Since $k$ is algebraically closed, it follows that 
\[ k[G] \cong M_{n_1}(k) \times \cdots \times M_{n_s}(k) \]
for some $s \geq 1$ and integers $n_1, \dots, n_s \geq 1$ by Corollary 9.2. 
\end{pf}

This corollary is important because this gives us a nice relationship between $k[G]$, which is 
something intrinsic to the group $G$, and the direct product of matrix rings, which is more in the 
realm of linear algebra.