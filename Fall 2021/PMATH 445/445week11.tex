\section{Some lemmas for Jordan's theorem (11/24/2021)}
Recall that we are trying to prove Jordan's theorem, which states that if 
$d \geq 1$ is an integer, then there exists an integer $N = N(d)$ such that 
if $G$ is a finite subgroup of $\GL_d(\C)$, then $G$ has an abelian subgroup 
of index at most $N(d)$. Notice that $G \leq {\cal U}$, where ${\cal U}$ is the
group of unitary matrices. We saw last time that ${\cal U}$ is compact. 

Let $U \in {\cal U}$ be unitary. Then for any $v$ with $|v| = 1$, we have 
\[ |Uv| = \ip{Uv}{Uv}^{1/2} = \ip{v}{U^{-1}Uv}^{1/2} = \ip{v}{v}^{1/2} = |v|. \]
Therefore, if $A \in M_d(\C)$, then $\|UA\| = \|AU\| = \|A\|$. Indeed, notice that 
\[ \|UA\| = \sup_{|v|=1} |UAv| = \sup_{|v|=1} |Av| = \|A\|, \] 
and the argument for $\|AU\| = \|A\|$ is analogous.  

Since ${\cal U} \subseteq M_d(\C)$ is compact, we know that for any $\eps > 0$, 
there exists $R = R(d, \eps)$ such that ${\cal U}$ can be covered by $R$ 
open balls of radius $\eps$ (this is a finite subcover). 

\begin{lemma}{}
    Let $G \leq {\cal U}$ and let $H \leq G$ be the subgroup generated by 
    $G \cap B(I, 2\eps)$. Then $[G : H] \leq R(d, \eps)$. 
\end{lemma}
\begin{pf}
    Suppose that $[G : H] = r > R(d, \eps)$. Then we can write $G = 
    A_1H \cup \cdots \cup A_rH$ as a disjoint union of cosets. Since ${\cal U}$ 
    is covered by $R(d, \eps)$ open balls and $r > R(d, \eps)$, there exist 
    integers $i \neq j$ such that $A_i, A_j \in B(X, \eps)$ for some 
    $X \in M_d(\C)$ by the pigeonhole principle. In particular, note that 
    \[ \|A_i - A_j\| = \|A_i - X + X - A_j\| \leq \|A_i - X\| + \|X - A_j\| 
    < 2\eps. \] 
    Note that $A_i^{-1} \in {\cal U}$, so we have 
    \[ \|A_i^{-1}(A_i - A_j)\| = \|I - A_i^{-1}A_j\| < 2\eps. \] 
    This implies that $A_i^{-1}A_j \in G \cap B(I, 2\eps) \subseteq H$, 
    and hence $A_j \in A_iH$, which is a contradiction. 
\end{pf}

It can be shown that we can choose $R(d, \eps) \leq (4/\eps)^{d^2}$. 

Next, we make an observation. If $A, B \in {\cal U} \cap B(I, \eps)$, then 
$BA \in {\cal U}$ as well, so we obtain 
\begin{align*}
    \|ABA^{-1}B^{-1} - I\| 
    &= \|(ABA^{-1}B^{-1} - I)BA\| \\
    &= \|AB - BA\| \\
    &= \|(A-I)(B-I) - (B-I)(A-I)\| \\
    &\leq \|A-I\|\|B-I\| + \|B-I\|\|A-I\| \\ 
    &< 2\eps^2. 
\end{align*}
By choosing $\eps < 1/2$, we see that the the commutator of $A$ and $B$ 
is closer to $I$ than $A$ and $B$. For our purposes, we will pick 
$\eps = \frac{1}{10d}$, which is certainly less than $1/2$. 

\begin{lemma}{}
    Let $A \in \SL_d(\C) \cap {\cal U}$. Suppose that $A = \lambda I$ for some 
    $\lambda \in \C$ and $A \in B(I, \eps)$. Then we have $A = I$. 
\end{lemma}
\begin{pf}
    Since $A \in {\cal U}$ and $A = \lambda I$ for $\lambda \in \C$, we have 
    \[ \|A - I\| = \|\lambda I - I\| = |\lambda - 1|\|I\| = |\lambda - 1| < \eps. \] 
    Moreover, we have $\det A = 1$ since $A \in \SL_d(\C)$, so $\lambda^d = 1$.
    Then $\lambda = e^{2\pi ij/d}$ for some $j = \{0, \dots, d-1\}$. Thus, 
    we obtain 
    \[ |\lambda - 1| = |e^{2\pi ij/d} - 1| 
    = \left| \cos\left(\frac{2\pi j}d\right) + i\sin\left(\frac{2\pi j}d\right) - 1 \right| 
    < \eps. \] 
    This means that $\lvert\sin(2\pi j/d)\rvert < \eps$. By some elementary calculus, 
    we obtain the inequalities $2x/\pi \leq \sin x \leq x$ for 
    $x \in [0, \pi/2]$. Now, if $\lvert\sin(2\pi j/d)\rvert \neq 0$, then 
    \[ \left| \sin\left( \frac{2\pi j}d \right) \right| 
    \geq \sin\left( \frac{\pi}{d} \right) \geq \frac{2}{\pi} \cdot 
    \frac{\pi}{d} = \frac{2}{d} > \frac{1}{10d} = \eps. \] 
    Thus, it must be the case that $\sin(2\pi j/d) = 0$, as 
    we had $\lvert\sin(2\pi j/d)\rvert < \eps$. It follows that 
    $\lambda = \cos(2\pi j/d) \in \{1, -1\}$. But $\lvert -1-1\rvert = 2 > \eps$, 
    so we must have $\lambda = 1$, and $A = I$ as required. 
\end{pf}

\begin{lemma}{}
    Let $G \leq {\cal U}$ be finite, and let $H$ be the subgroup of $G$
    generated by $G \cap B(I, \eps)$. Then either every element of $H$ 
    is a scalar multiple of $I$, or there exists $z \in H$ which is central 
    in $H$ and is not a scalar multiple of $I$. 
\end{lemma}
\begin{pf}
    Let ${\cal B} = \{B_1, \dots, B_m\} = G \cap B(I, \eps)$. If $B_1, 
    \dots, B_m$ are all scalar multiples of $I$, then we're already done. 
    So without loss of generality, pick $B \in \{B_1, \dots, B_m\}$ 
    such that $B$ is not a scalar multiple of $I$, and $\|B - I\| \leq 
    \|B_j - I\|$ whenever $B_j$ is not a scalar multiple of $I$. 

    We claim that $B$ is central in $H$. It suffices to show that 
    $BB_jB^{-1}B_j^{-1} = I$ for all $B_j \in {\cal B}$. First, since 
    $\eps < 1/2$, we obtain 
    \[ \|BB_jB^{-1}B_j^{-1} - I\| < 2\eps^2 < \eps. \] 
    In fact, we have 
    \[ \|BB_jB^{-1}B_j^{-1} - I\| \leq 2\|B - I\|\|B_j - I\| < 
    2\eps\|B - I\| < \|B - I\|. \] 
    Since $B$ is minimal with respect to the distance from $I$ for 
    non-scalar multiples of $I$ in ${\cal B}$, we see that 
    $BB_jB^{-1}B_j^{-1} = B_k$ for some $k \in \{1, \dots, m\}$, 
    and $B_k$ is a scalar multiple of $I$. Observe that $\det B_k = 1$ 
    so that $B_k \in \SL_d(\C)$. By Lemma 31.2, it follows that $B_k = I$. 
    Thus, $BB_j = B_jB$ for all $B_j \in {\cal B}$, so 
    we conclude that $B$ is central in $H$. 
\end{pf}

Now, we can start by giving a sketch of the proof of Jordan's theorem. We know 
that $G \leq {\cal U}$, and we can cover ${\cal U}$ with $R = R(d, \eps)$ balls
of radius $\frac12 \eps = \frac{1}{20d}$. 

Let $H$ be the subgroup of $G$ generated by $G \cap B(I, \eps)$. We showed in 
Lemma 31.1 that $[G : H] \leq R$. By Lemma 31.3, either everything in $H$ 
is of the form $\lambda I$ for some $\lambda \in \C$, in which case 
we're already done, or there exists $z \in H$ such that $z$ is not a 
scalar multiple of $I$ and $z$ is central in $H$. 

Notice that $\GL_d(\C)$ acts on $\C^d = V$. Pick an eigenvalue $\alpha$ of $z$. 
Let $W = \ker(z - \alpha I)$. Now, we have $(0) \subsetneq W \subsetneq V$.
Indeed, by definition of $\alpha$ being an eigenvalue, $z - \alpha I$ must 
have some kernel. On the other hand, we have $W \subsetneq V$ since 
$z$ is not a scalar multiple of $I$. 

Now, $W$ is invariant under $H$, because for $w \in W$ and $h \in H$, we have 
\[ (z - \alpha I)(h \cdot w) = (hz - \alpha h I) \cdot w = h(z - \alpha I) 
\cdot w = 0, \] 
which implies that $h \cdot w \in W$. 

By Maschke, we have $V = W \oplus W'$ where $W$ and $W'$ are both $H$-invariant. 
This means that if we take a new basis for $V = W \oplus W'$ where 
$\{w_1, \dots, w_e\}$ is a basis for $W$ and $\{w_{e+1}, \dots, w_d\}$ is a 
basis for $W'$, then the elements of $H$ with respect to this new basis 
are of the form 
\[ \begin{pmatrix}
    \rho_1(h) & 0 \\ 
    0 & \rho_2(h)
\end{pmatrix}, \] 
where $\rho_1(h)$ is an $e\times e$ matrix, and $\rho_2(h)$ is a 
$(d-e)\times (d-e)$ matrix. The idea here is that we can proceed by 
induction. We've broken it up into smaller blocks, and we can apply the 
induction hypothesis to these smaller blocks. If we get a big abelian 
subgroup in each block, then we can put these together to get a big 
abelian subgroup for $H$!

\section{Proof of Jordan's theorem (11/26/2021)}
We prove another lemma towards Jordan's theorem. 

\begin{lemma}{}
    Let $H_1, \dots, H_k$ be groups, and suppose that each $H_i$ has an 
    abelian subgroup $A_i$ of index at most $n_i$. If $H \leq H_1 \times 
    \cdots \times H_k$, then $H$ has an abelian subgroup of index at most 
    $n_1 \cdots n_k$. 
\end{lemma}
\begin{pf}
    Let $A = A_1 \times \cdots \times A_k \subseteq H_1 \times \cdots \times H_k$. 
    Note that $A$ is abelian as each $A_i$ is abelian. Notice that 
    \[ [H_1 \times \cdots \times H_k : A] = [H_1 : A_1] \cdots [H_k : A_k] 
    = n_1 \cdots n_k. \] 
    Now, let $B = A \cap H$, which is still abelian. It remains to show that 
    $[H : B] \leq [H_1 \times \cdots \times H_k : A] = n_1 \cdots n_k$. 
    Let $X$ be the set of left cosets of $B$ in $H$, and let $Y$ be the 
    set of left cosets of $A$ in $H_1 \times \cdots \times H_k$. Define 
    $f : X \to Y$ by 
    \[ f(hB) = hA. \] 
    To see that $f$ is well-defined, suppose that $h_1B = h_2B$. This tells us 
    that $h_2^{-1}h_1 \in B \subseteq A$, so $h_1 \in h_2A$, and hence 
    $h_1A = h_2A$. Now, we show that $f$ is injective. Suppose that 
    $f(h_1B) = f(h_2B)$. Then $h_1A = h_2A$, and hence $h_2^{-1}h_1 \in A 
    \cap H = B$. It follows that $h_1B = h_2B$, so $f$ is indeed injective. 
    Thus, we have $[H : B] = |X| \leq |Y| = [H_1 \times \cdots \times H_k 
    : A]$. 
\end{pf}

{\sc Proof of Jordan's theorem.} We proceed by induction on $d$. For 
$d = 1$, we have $\GL_1(\C) = \C^*$, which is abelian, so we can take 
$N(1) = 1$. Now, suppose that the result holds up to $d-1$. Define 
\[ N(d) = R\left(d, \frac{1}{20d}\right) \cdot \left[ \max_{1\leq e\leq d-1} 
N(e) N(d-e) \right], \] 
where $R(d, \eps)$ is the number of open $\eps$-balls needed to cover 
${\cal U} \subseteq \GL_d(\C)$. 

We claim that $G$ has an abelian subgroup of index at most $N(d)$. 
We showed in Lemma 31.1 that if $H \leq G$ is generated by $G 
\cap B(I, \frac{1}{10d})$, then $[G : H] \leq R(d, \frac{1}{20d})$. 
So it suffices to show that $H$ has an abelian subgroup $B$ of index 
at most $\max_{1\leq e\leq d-1} N(e)N(d-e)$. Indeed, notice that 
\[ [G : B] = [G : H][H : B] \leq R\left(d, \frac{1}{20d}\right) \cdot [H : B]. \] 
We have also showed that there exists $e \in \{1, \dots, d-1\}$ such that 
\[ h \in H \mapsto \begin{pmatrix}
    \rho_1(h) & 0 \\ 0 & \rho_2(h)
\end{pmatrix} \] 
where $\rho_1(h)$ is an $e \times e$ matrix and $\rho_2(h)$ is a 
$(d-e) \times (d-e)$ matrix. Let $H_1 = \{\rho_1(h) : h \in H\} \leq \GL_e(\C)$ 
and $H_2 = \{\rho_2(h) : h \in H\} \leq \GL_{d-e}(\C)$. By induction, 
there exists an abelian subgroup $A_1 \leq H_1$ of index at most $N(e)$, 
and an abelian subgroup $A_2 \leq H_2$ of index at most $N(d-e)$. Consider the 
map 
\begin{align*}
    H &\to H_1 \times H_2, \\ 
    h &\mapsto (\rho_1(h), \rho_2(h)). 
\end{align*}
By Lemma 32.1, there exists an abelian subgroup $B \leq H$ such that 
\[ [H : B] \leq N(e)N(d-e) \leq \max_{1\leq e\leq d-1} N(e)N(d-e), \] 
which completes the proof. \qed 

\begin{prop}{}
    If $H \leq G$ are groups and $[G : H] = n$, then there exists 
    $H_0 \leq H$ such that $H_0 \trianglelefteq G$ and $[G : H_0] \leq n!$. 
\end{prop}
\begin{pf}
    Let $X$ be the set of left cosets $\{x_1H, \dots, x_mH\}$. Without loss 
    of generality, suppose that $x_1 = 1$. Let $G$ act on $X$ by 
    $g \cdot (x_iH) = gx_iH = x_jH$ for some $j$. An action of $G$ on $X$
    induces a homomorphism 
    \[ \phi : G \to \Sym_X = \{f : X \to X \mid f \text{ is bijective}\} \] 
    given b $\phi(g) = \sigma$ where $g \cdot x_iH = x_{\sigma(i)}H$. Let 
    $H_0 = \ker\phi \trianglelefteq G$. Notice that $G/H_0$ embeds into 
    $\Sym_X$. In particular, if $|X| < \infty$, then $[G : H_0] \leq |X|!$. 
    In our case, we have $|X| = n$, so $[G : H_0] \leq n!$. Finally, 
    notice that $H_0 \leq H$ since 
    \begin{align*}
        \ker\phi &= \{g \in G : gx_iH = x_iH \text{ for all } i = 1, \dots, m\} \\ 
        &= \{g \in G : gx_1H = x_1H\} \\ 
        &= \{g \in G : gH = H\} = H. \qedhere 
    \end{align*}
\end{pf}

We give a nice application of this result. 

\begin{exmp}{}
    Let $n \geq 5$. If $H \leq S_n$ and $[S_n : H] < n$, then $H = A_n$ or 
    $H = S_n$. 
\end{exmp}
\begin{pf}
    Suppose that $[S_n : H] = r < n$. Then by Proposition 32.2, there exists 
    $H_0 \trianglelefteq S_n$ such that $[S_n : H_0] \leq r!$. Notice that 
    $[S_n : A_n] \leq 2$, which implies that $[A_n : A_n \cap H_0] \leq r!$ 
    with $A_n \cap H_0 \trianglelefteq A_n$. Now, $A_n$ is simple for 
    $n \geq 5$, so either $H_0 = A_n$ or $H_0 = \{1\}$. If $H_0 = \{1\}$, then 
    \[ [A_n : H_0] = |A_n| = \frac{n!}2 \leq r! \leq (n-1)!. \] 
    Now $n! \leq 2(n-1)!$, so $n \leq 2$, which is a contradiction. So we must 
    have $H_0 = A_n$, in which case $H \geq H_0$, so $H = A_n$ or $H = S_n$. 
\end{pf}

Proposition 32.2 also implies the following extension of Jordan's theorem; 
we leave the proof as an exercise. 

\begin{theo}{}
    If $G \leq \GL_d(\C)$ is finite, there exists $M = M(d)$ such that 
    $G$ has a normal abelian subgroup of index $M(d)$. In fact, we can 
    take $M(d) = N(d)!$. 
\end{theo}

Now, we'll start looking at another theorem by Burnside. In 1902, Burnside 
wondered that if every element of a group $G$ is finite, then is $G$ finite? 
It is clear that this is false in general by taking 
\[ G = \bigoplus_{i=1}^\infty \Z/2\Z. \] 
But what if $G$ is finitely generated? It turns out this is still false. 
In 1964, Golod and Schafarevich showed that there exist finitely generated 
groups that are infinite, but every element has finite order. However, 
we have the special case that if $G \leq \GL_d(\C)$ is finitely generated 
with every element having finite order, then $G$ is in fact finite. This is 
called the Burnside-Schur theorem, which we will work towards proving. 

\begin{prop}{}
    Let $R \subseteq M_n(\C)$ be a $\C$-submodule of $M_n(\C)$. Then either 
    $R = M_n(\C)$, or after a change of basis, elements of $R$ can be 
    put in block triangular form 
    \[ \begin{pmatrix}
        * & * \\ 0 & * 
    \end{pmatrix}, \] 
    where the $*$ on the upper left is $e \times e$ and the $*$ on the 
    bottom right is $(n-e) \times (n-e)$ for some $e \in \{1, \dots, n-1\}$. 
\end{prop}
\begin{pf}
    Let $V = \C^{n\times 1}$, which is a left $R$-module and right 
    $M_n(\C)$-module. Also, $V$ is faithful since it is faithful as an 
    $M_n(\C)$-module. If $V$ is simple, then the Jacobson density theorem 
    tells us that $R = \End_R(V) = M_n(\C)$. On the other hand, if $V$ 
    is not simple, then there exists an $R$-submodule $0 \subsetneq W 
    \subsetneq V$ of dimension $e$. Let $v_1, \dots, v_e$ be a basis for $W$. 
    We can extend this to a basis $v_1, \dots, v_n$ for $V$, in which 
    case we can write 
    \[ r \mapsto \begin{pmatrix}
        * & * \\ 0 & *
    \end{pmatrix} \] 
    with respect to this new basis. 
\end{pf}

\section{Another theorem by Burnside (11/29/2021)}
Let $R \subseteq M_d(\C)$ be a $\C$-submodule. Recall that either $R = 
M_d(\C)$, or after a change of basis, every element of $R$ can be 
put into block triangular form. 

\begin{lemma}{}
    Let $N \geq 2$. Let $G \leq \GL_d(\C)$ be a group in which every 
    element has order at most $N$. Then $|G| \leq N^{2d^3}$. 
\end{lemma}
\begin{pf}
   Let $A \in G \leq \GL_d(\C)$. Since every element in $G$ has order 
   at most $N$, the eigenvalues of $A$ belong to the set 
   \[ \Omega = \{\omega \in \C^* : \omega^j = 1 \text{ for some } 
   1 \leq j \leq N\}. \] 
   Notice that $|\Omega| = 1 + 2 + \cdots + N \leq N^2$. Next, notice that 
   if $A \in G$, then $\Tr(A) \in \mathbb{X}$, where 
   \[ \mathbb{X} := \{\omega_1 + \cdots + \omega_d : \omega_1, \dots, 
   \omega_d \in \Omega\}. \] 
   We see that $|\mathbb{X}| \leq |\Omega|^d \leq N^{2d}$. 

   Now, we'll proceed by induction on $d$. For $d = 1$, we have $G \leq 
   \C^*$, and in fact, $G \leq \Omega$, so $|G| \leq N^2$. Assume that the 
   result holds up to $d - 1$. Let $R$ be the $\C$-span of all the elements 
   in $G$; note that $R \subseteq M_d(\C)$. We can see that $R$ is a 
   $\C$-algebra since 
   \[ \left( \sum \lambda_A A \right) \left( \sum \gamma_B B \right) = 
   \sum \lambda_A \gamma_B AB. \] 
   Using Proposition 32.5, we can break this into two cases. 

   {\sc Case 1.} Suppose that $R = M_d(\C)$. Then there exist $Y_1, \dots, Y_d 
   \in G$ which form a basis for $M_d(\C)$. 

   {\sc Claim.} If $A, B \in M_d(\C)$ and $\Tr(AY_i) = \Tr(BY_i)$ for all 
   $i = 1, \dots, d$, then $A = B$.

   {\sc Proof of Claim.} Letting $C = A - B$ and using the fact that trace is 
   linear, it suffices to show that $\Tr(CY_i) = 0$ for all $i = 1, \dots, d$ 
   implies that $C = 0$. So let $C = (c_{ij}) \in M_d(\C)$, and let 
   $C^* = (\bar{c}_{ij})$. Then we have $\Tr(CC^*) = \sum |c_{ij}|^2$. 
   If we write $C^* = \sum_{i=1}^d \lambda_i Y_i$, we get 
   \[ \Tr(CC^*) = \Tr\left( C \left(\sum_{i=1}^d \lambda_iY_i\right) \right) 
   = \sum_{i=1}^d \lambda_i \Tr(CY_i) = 0, \] 
   where the last equality is by assumption, so $C = 0$. \hfill $\blacksquare$ 

   Now, consider the map $\Psi : G \to \mathbb{X}^{d^2}$ given by 
   \[ \Psi(A) = (\Tr(AY_1), \dots, \Tr(AY_d)). \] 
   By the claim, this map is injective, so we obtain $|G| \leq |\mathbb{X}|^{d^2} 
   \leq |\mathbb{X}|^{d^3}$. 

   {\sc Case 2.} Suppose that $R \subsetneq M_d(\C)$. Then after a change of 
   basis, we can assume that 
   \[ G \subseteq R \subseteq \left\{ \begin{pmatrix} *_1 & * \\ 0 & *_2 \end{pmatrix} :
    *_1 \in M_e(\C),\, *_2 \in M_{d-e}(\C) \right\} \] 
   for some $1 \leq e \leq d-1$. So we can write each $A \in G$ in the form 
   \[ A = \begin{pmatrix}
       A_{11} & A_{12} \\ 0 & A_{22} 
   \end{pmatrix} \] 
   where $A_{11} \in \GL_e(\C)$ and $A_{22} \in \GL_{d-e}(\C)$. Since $A$ has 
   order at most $N$, there exists $j \in \{1, \dots, n\}$ such that $A^j = I$. 
   Then we obtain 
   \[ A^j = \begin{pmatrix}
       A_{11}^j & * \\ 0 & A_{22}^j 
   \end{pmatrix} = \begin{pmatrix}
       I_e & 0 \\ 0 & I_{d-e}
   \end{pmatrix} = I. \] 
   Hence, we have $A_{11}^j = I_e$ and $A_{22}^j = I_{d-e}$. Define the groups 
   \begin{align*}
       G_1 &= \{A_{11} : A \in G\} \leq \GL_e(\C), \\ 
       G_2 &= \{A_{22} : A \in G\} \leq \GL_{d-e}(\C). 
   \end{align*}
   Every element of $G_1$ has order at most $N$, and every element of $G_2$
   also has order at most $N$. By the induction hypothesis, we have 
   $|G_1| \leq N^{2e^3}$ and $|G_2| \leq N^{2(d-e)^3}$. It follows that 
   \[ |G_1||G_2| \leq N^{2(e^3+(d-e)^3)} \leq N^{2d^3}. \] 
   To finish the proof, it remains to show that $|G| \leq |G_1||G_2|$.  
   Suppose otherwise that $|G| > |G_1||G_2|$. Then by the pigeonhole 
   principle, we can find $B_1 \in G_1$ and $B_2 \in G_2$ such that 
   there are distinct elements $A, A' \in G$ of the form 
   \begin{align*}
       A &= \begin{pmatrix}
           B_1 & C \\ 0 & B_2
       \end{pmatrix}, & 
       A' &= \begin{pmatrix}
           B_1 & C' \\ 0 & B_2
       \end{pmatrix},
   \end{align*}
   where $C \neq C'$. We see that 
   \[ I \neq A(A')^{-1} = \begin{pmatrix}
       B_1B_1^{-1} & * \\ 0 & B_2B_2^{-1} 
   \end{pmatrix} = \begin{pmatrix}
       I_e & D \\ 0 & I_{d-e}
   \end{pmatrix} \] 
   where $D \neq 0$. But $A(A')^{-1} \in G$, so there exists $j \in \{1, \dots, n\}$ 
   such that $[A(A')^{-1}]^j = I$. This implies that 
   \[ \begin{pmatrix}
       I_e & D \\ 0 & I_{d-e}
   \end{pmatrix}, \] 
   so we must have $D = 0$, which is a contradiction.
\end{pf}

We note that this result is false if we are working over a field of characteristic 
$p$, where $p$ is a prime. Indeed, take the group 
\[ G = \left\{ \begin{pmatrix}
    1 & a \\ 0 & 1 
\end{pmatrix} : a \in \F_p(t)\right\} \leq \GL_2(\F_p(t)). \] 
Then for any $a \in \F_p(t)$, we have 
\[ \begin{pmatrix}
    1 & a \\ 0 & 1 
\end{pmatrix}^{\!p} = \left[ \begin{pmatrix}
    1 & 0 \\ 0 & 1 
\end{pmatrix}^{\!p} + \begin{pmatrix}
    0 & a \\ 0 & 0 
\end{pmatrix}^{\!p} \right] = \begin{pmatrix}
    1 & 0 \\ 0 & 1 
\end{pmatrix}^{\!p} + \begin{pmatrix}
    0 & a \\ 0 & 0
\end{pmatrix}^{\!p} = I. \] 

Burnside used the above lemma to prove the following result. 

\begin{theo}[Burnside]{}
    Let $G \leq \GL_d(\Q)$ be a group in which element has finite order. 
    Then $G$ is finite. 
\end{theo}

Recall that for $n \in \N$, the Euler totient function is given by 
$\phi(n) = \#\{1 \leq m \leq n : \gcd(m, n) = 1\}$. Moreover, 
we have $\lim_{n\to\infty} \phi(n) = \infty$, and it satisfies 
\[ \phi(p_1^{a_1} \cdots p_s^{a_s}) = p_1^{a_1} \cdots p_s^{a_s} 
\left(1 - \frac1{p_1}\right) \cdots \left(1 - \frac1{p_s}\right). \] 
Note that for all $n \in \N$, we have $\phi(n) \geq \sqrt{n}/2$. 
To see this, let $p \geq 3$ be prime and $a \geq 1$ be an integer. 
Then 
\[ \phi(p^a) = p^a\left(1 - \frac{1}{p}\right) \geq p^{a/2}. \] 
Now, writing $n = 2^{b_1}p_2^{b_2} \cdots p_s^{b_s}$ where $p_2, \dots, p_s$ 
are distinct odd primes, we have 
\[ \phi(n) = \begin{cases}
    2^{b_1-1}\phi(p_2^{b_2}) \cdots \phi(p_s^{b_s}) & \text{if } b_1 \geq 1, \\ 
    \phi(p_2^{b_2}) \cdots \phi(p_s^{b_s}) & \text{if } b_1 = 0. 
\end{cases} \] 
In particular, we see that 
\[ \phi(n) \geq \frac{2^{b_1/2}}2 p_2^{b_2/2} \cdots p_s^{b_s/2} = \frac{\sqrt n}2. \] 
\begin{lemma}{}
    Let $d \geq 1$. Then there exists an integer $n = n(d)$ such that if 
    $A \in \GL_d(\C)$ has finite order, then the order of $A$ is at most $n$. 
\end{lemma}
\begin{pf}
    Let $P(x) = x^d + c_{d-1}x^{d-1} + \cdots + c_0 \in \Q[x]$ be the characteristic 
    polynomial of $A$. If $\omega$ is an eigenvalue of $A$, then $P(\omega) = 0$. This 
    means that 
    \[ [\Q(\omega) : \Q] \leq d \] 
    since $P(x)$ has degree at most $d$. Now, if $\omega = e^{2\pi ij/b}$ 
    where $\gcd(j, b) = 1$ (that is, $\omega$ is a $b$-th root of unity), then 
    it can be shown that 
    \[ [\Q(\omega) : \Q] = \phi(b). \] 
    Note that $\phi(b) \leq d$. Since $\phi(n) \to \infty$ as $n \to \infty$, 
    there exists $n = n(d)$ such that $\phi(j) > d$ for all $j \geq n$. 
    So every eigenvalue of $A$ is a $b$-th root of unity where $b \leq 
    n = n(d)$. Recall that 
    \[ A \sim \begin{pmatrix}
        \omega_1 & & 0 \\ 
        & \ddots & \\ 
        0 & & \omega_n
    \end{pmatrix}, \] 
    so $A^{n!} = I$. Thus, $A$ has order at most $n!$. 
\end{pf}

Notice that taking $N = \max(n(d), 2)$ and applying Lemma 33.1 shows that 
$G \leq \GL_d(\Q)$ is finite when every element of $G$ has finite order.  