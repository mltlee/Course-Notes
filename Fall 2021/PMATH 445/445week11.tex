\section{Some lemmas for Jordan's theorem (11/24/2021)}
Recall that we are trying to prove Jordan's theorem, which states that if 
$d \geq 1$ is an integer, then there exists an integer $N = N(d)$ such that 
if $G$ is a finite subgroup of $\GL_d(\C)$, then $G$ has an abelian subgroup 
of index at most $N(d)$. Notice that $G \leq {\cal U}$, where ${\cal U}$ is the
group of unitary matrices. We saw last time that ${\cal U}$ is compact. 

Let $U \in {\cal U}$ be unitary. Then for any $v$ with $|v| = 1$, we have 
\[ |Uv| = \ip{Uv}{Uv}^{1/2} = \ip{v}{U^{-1}Uv}^{1/2} = \ip{v}{v}^{1/2} = |v|. \]
Therefore, if $A \in M_d(\C)$, then $\|UA\| = \|AU\| = \|A\|$. Indeed, notice that 
\[ \|UA\| = \sup_{|v|=1} |UAv| = \sup_{|v|=1} |Av| = \|A\|, \] 
and the argument for $\|AU\| = \|A\|$ is analogous.  

Since ${\cal U} \subseteq M_d(\C)$ is compact, we know that for any $\eps > 0$, 
there exists $R = R(d, \eps)$ such that ${\cal U}$ can be covered by $R$ 
open balls of radius $\eps$ (this is a finite subcover). 

\begin{lemma}{}
    Let $G \leq {\cal U}$ and let $H \leq G$ be the subgroup generated by 
    $G \cap B(I, 2\eps)$. Then $[G : H] \leq R(d, \eps)$. 
\end{lemma}
\begin{pf}
    Suppose that $[G : H] = r > R(d, \eps)$. Then we can write $G = 
    A_1H \cup \cdots \cup A_rH$ as a disjoint union of cosets. Since ${\cal U}$ 
    is covered by $R(d, \eps)$ open balls and $r > R(d, \eps)$, there exist 
    integers $i \neq j$ such that $A_i, A_j \in B(X, \eps)$ for some 
    $X \in M_d(\C)$ by the pigeonhole principle. In particular, note that 
    \[ \|A_i - A_j\| = \|A_i - X + X - A_j\| \leq \|A_i - X\| + \|X - A_j\| 
    < 2\eps. \] 
    Note that $A_i^{-1} \in {\cal U}$, so we have 
    \[ \|A_i^{-1}(A_i - A_j)\| = \|I - A_i^{-1}A_j\| < 2\eps. \] 
    This implies that $A_i^{-1}A_j \in G \cap B(I, 2\eps) \subseteq H$, 
    and hence $A_j \in A_iH$, which is a contradiction. 
\end{pf}

It can be shown that we can choose $R(d, \eps) \leq (4/\eps)^{d^2}$. 

Next, we make an observation. If $A, B \in {\cal U} \cap B(I, \eps)$, then 
$BA \in {\cal U}$ as well, so we obtain 
\begin{align*}
    \|ABA^{-1}B^{-1} - I\| 
    &= \|(ABA^{-1}B^{-1} - I)BA\| \\
    &= \|AB - BA\| \\
    &= \|(A-I)(B-I) - (B-I)(A-I)\| \\
    &\leq \|A-I\|\|B-I\| + \|B-I\|\|A-I\| \\ 
    &< 2\eps^2. 
\end{align*}
By choosing $\eps < 1/2$, we see that the the commutator of $A$ and $B$ 
is closer to $I$ than $A$ and $B$. For our purposes, we will pick 
$\eps = \frac{1}{10d}$, which is certainly less than $1/2$. 

\begin{lemma}{}
    Let $A \in \SL_d(\C) \cap {\cal U}$. Suppose that $A = \lambda I$ for some 
    $\lambda \in \C$ and $A \in B(I, \eps)$. Then we have $A = I$. 
\end{lemma}
\begin{pf}
    Since $A \in {\cal U}$ and $A = \lambda I$ for $\lambda \in \C$, we have 
    \[ \|A - I\| = \|\lambda I - I\| = |\lambda - 1|\|I\| = |\lambda - 1| < \eps. \] 
    Moreover, we have $\det A = 1$ since $A \in \SL_d(\C)$, so $\lambda^d = 1$.
    Then $\lambda = e^{2\pi ij/d}$ for some $j = \{0, \dots, d-1\}$. Thus, 
    we obtain 
    \[ |\lambda - 1| = |e^{2\pi ij/d} - 1| 
    = \left| \cos\left(\frac{2\pi j}d\right) + i\sin\left(\frac{2\pi j}d\right) - 1 \right| 
    < \eps. \] 
    This means that $\lvert\sin(2\pi j/d)\rvert < \eps$. By some elementary calculus, 
    we obtain the inequalities $2x/\pi \leq \sin x \leq x$ for 
    $x \in [0, \pi/2]$. Now, if $\lvert\sin(2\pi j/d)\rvert \neq 0$, then 
    \[ \left| \sin\left( \frac{2\pi j}d \right) \right| 
    \geq \sin\left( \frac{\pi}{d} \right) \geq \frac{2}{\pi} \cdot 
    \frac{\pi}{d} = \frac{2}{d} > \frac{1}{10d} = \eps. \] 
    Thus, it must be the case that $\sin(2\pi j/d) = 0$, as 
    we had $\lvert\sin(2\pi j/d)\rvert < \eps$. It follows that 
    $\lambda = \cos(2\pi j/d) \in \{1, -1\}$. But $\lvert -1-1\rvert = 2 > \eps$, 
    so we must have $\lambda = 1$, and $A = I$ as required. 
\end{pf}

\begin{lemma}{}
    Let $G \leq {\cal U}$ be finite, and let $H$ be the subgroup of $G$
    generated by $G \cap B(I, \eps)$. Then either every element of $H$ 
    is a scalar multiple of $I$, or there exists $z \in H$ which is central 
    in $H$ and is not a scalar multiple of $I$. 
\end{lemma}
\begin{pf}
    Let ${\cal B} = \{B_1, \dots, B_m\} = G \cap B(I, \eps)$. If $B_1, 
    \dots, B_m$ are all scalar multiples of $I$, then we're already done. 
    So without loss of generality, pick $B \in \{B_1, \dots, B_m\}$ 
    such that $B$ is not a scalar multiple of $I$, and $\|B - I\| \leq 
    \|B_j - I\|$ whenever $B_j$ is not a scalar multiple of $I$. 

    We claim that $B$ is central in $H$. It suffices to show that 
    $BB_jB^{-1}B_j^{-1} = I$ for all $B_j \in {\cal B}$. First, since 
    $\eps < 1/2$, we obtain 
    \[ \|BB_jB^{-1}B_j^{-1} - I\| < 2\eps^2 < \eps. \] 
    In fact, we have 
    \[ \|BB_jB^{-1}B_j^{-1} - I\| \leq 2\|B - I\|\|B_j - I\| < 
    2\eps\|B - I\| < \|B - I\|. \] 
    Since $B$ is minimal with respect to the distance from $I$ for 
    non-scalar multiples of $I$ in ${\cal B}$, we see that 
    $BB_jB^{-1}B_j^{-1} = B_k$ for some $k \in \{1, \dots, m\}$, 
    and $B_k$ is a scalar multiple of $I$. Observe that $\det B_k = 1$ 
    so that $B_k \in \SL_d(\C)$. By Lemma 31.2, it follows that $B_k = I$. 
    Thus, $BB_j = B_jB$ for all $B_j \in {\cal B}$, so 
    we conclude that $B$ is central in $H$. 
\end{pf}

Now, we can start by giving a sketch of the proof of Jordan's theorem. We know 
that $G \leq {\cal U}$, and we can cover ${\cal U}$ with $R = R(d, \eps)$ balls
of radius $\frac12 \eps = \frac{1}{20d}$. 

Let $H$ be the subgroup of $G$ generated by $G \cap B(I, \eps)$. We showed in 
Lemma 31.1 that $[G : H] \leq R$. By Lemma 31.3, either everything in $H$ 
is of the form $\lambda I$ for some $\lambda \in \C$, in which case 
we're already done, or there exists $z \in H$ such that $z$ is not a 
scalar multiple of $I$ and $z$ is central in $H$. 

Notice that $\GL_d(\C)$ acts on $\C^d = V$. Pick an eigenvalue $\alpha$ of $z$. 
Let $W = \ker(z - \alpha I)$. Now, we have $(0) \subsetneq W \subsetneq V$.
Indeed, by definition of $\alpha$ being an eigenvalue, $z - \alpha I$ must 
have some kernel. On the other hand, we have $W \subsetneq V$ since 
$z$ is not a scalar multiple of $I$. 

Now, $W$ is invariant under $H$, because for $w \in W$ and $h \in H$, we have 
\[ (z - \alpha I)(h \cdot w) = (zh - \alpha h I) \cdot w = h(z - \alpha I) 
\cdot w = 0, \] 
which implies that $h \cdot w \in W$. 

By Maschke, we have $V = W \oplus W'$ where $W$ and $W'$ are both $H$-invariant. 
This means that if we take a new basis for $V = W \oplus W'$ where 
$\{w_1, \dots, w_e\}$ is a basis for $W$ and $\{w_{e+1}, \dots, w_d\}$ is a 
basis for $W'$, then the elements of $H$ with respect to this new basis 
are of the form 
\[ \begin{pmatrix}
    \rho_1(h) & 0 \\ 
    0 & \rho_2(h)
\end{pmatrix}, \] 
where $\rho_1(h)$ is an $e\times e$ matrix, and $\rho_2(h)$ is a 
$(d-e)\times (d-e)$ matrix. The idea here is that we can proceed by 
induction. We've broken it up into smaller blocks, and we can apply the 
induction hypothesis to these smaller blocks. If we get a big abelian 
subgroup in each block, then we can put these together to get a big 
abelian subgroup for $H$!