\section{Frobenius reciprocity (11/15/2021)}
Let $H$ be a subgroup, and let $W$ be a left $k[H]$-module. Then 
$W \uparrow_H^G\;= k[G] \otimes_{k[H]} W$ is a left $k[G]$-module. 
We saw in the previous lecture that 
\[ \chi_W \uparrow_H^G(g) = \frac{1}{|H|} \sum_{\{x\in G\,:\,x^{-1}gx \in H\}} \chi_W(x^{-1}gx). \] 
Also, if $G$ has disjoint coset decomposition $x_1H \cup \cdots \cup x_mH$, then 
\[ \chi_W \uparrow_H^G(g) = \sum_{\{i\,:\,x_i^{-1}gx_i \in H\}} \chi_W(x_i^{-1}gx_i). \] 

\begin{remark}{}
    If $H$ is a normal subgroup of $G$, then 
    \[ \chi_W \uparrow_H^G (g) = \begin{cases} 
        0 & \text{if } g \notin H, \\ 
        \frac{1}{|H|} \sum_{x\in G} \chi_W(x^{-1}gx) & \text{if } g \in H,
    \end{cases} \] 
    since $x^{-1}gx \in H$ if and only if $g \in xHx^{-1} = H$. 
\end{remark}

Let $V$ be a left $k[G]$-module. We have a left $k[H]$-module $V \downarrow_H^G$
by taking $V$ as a set, but we only remember the $k[H]$-module structure. 
That is, we have $V \downarrow_H^G\;= {}_{k[H]} V$. If $\chi_V$ is the 
character of $V$, then 
\[ \chi_V \downarrow_H^G (h) = \chi_V(h) \] 
for all $h \in H$. Indeed, we have $\chi_V \downarrow_H^G (h) = \Tr(\rho(h))$, 
where $\rho(h) : V \to V$ is given by $v \mapsto h \cdot v$. But $h \in H 
\subseteq G$, so this is just the trace of the particular element $\rho(h)$, 
which is $\chi_V(h)$. 

\begin{theo}[Frobenius reciprocity]{}
    Let $\phi : G \to C$ and $\psi : H \to \C$ be class functions, where 
    $H \leq G$ are finite groups. Then 
    \[ \langle \psi \uparrow_H^G, \phi \rangle_G = \langle \psi, \phi \downarrow_H^G \rangle_H. \] 
\end{theo}
\begin{pf}
    As with the transitivity of induction, this can also be done using a high
    level proof, but we will do this using a straightforward computation. We have 
    \begin{align*}
        \langle \psi \uparrow_H^G, \phi \rangle_G 
        &= \frac{1}{|G|} \sum_{g \in G} \psi \uparrow_H^G (g) \phi(g^{-1}) \\ 
        &= \frac{1}{|G|} \sum_{g \in G} \left( \frac{1}{|H|} 
        \sum_{\{x\in G\,:\,x^{-1}gx\in H\}} \psi(x^{-1}gx) \right) \phi(g^{-1}) \\
        &= \frac{1}{|G|} \frac{1}{|H|} \sum_{g\in G} \sum_{\{x\in G\,:\,x^{-1}gx\in H\}}
        \psi(x^{-1}gx) \phi(g^{-1}). 
    \end{align*}
    Now, substitute $h = x^{-1}gx$ so that $g = xhx^{-1}$ and $g^{-1} = 
    xh^{-1}x^{-1}$. Then, we obtain 
    \begin{align*}
        \langle \psi \uparrow_H^G, \phi \rangle_G 
        &= \frac{1}{|G|} \frac{1}{|H|} \sum_{h \in H} \sum_{x \in G} \psi(h) \phi(xh^{-1}x^{-1}) \\ 
        &= \frac{1}{|G|} \frac{1}{|H|} \sum_{h \in H} |G| \psi(h) \phi(h^{-1}) \\ 
        &= \frac{1}{|H|} \sum_{h\in H} \psi(h) \phi(h^{-1}) \\ 
        &= \langle \psi, \phi \downarrow_H^G \rangle_H, 
    \end{align*}
    where the second equality is because $\phi$ is a class function, which yields 
    $\phi(xh^{-1}x^{-1}) = \phi(h^{-1})$. 
\end{pf}

This result has a nice connection with tensor-hom adjunction. Let 
$W$ be a left $k[H]$-module and let $V$ be a left $k[G]$-module. Then we have 
a $k$-algebra isomorphism 
\[ \Hom_{k[G]}(k[G] \otimes_{k[H]} W, V) \cong \Hom_{k[H]}(W, \Hom_{k[G]}(k[G], V)). \] 
This essentially states that if we have a tensor on the left, then we can move it over. 

How is this the same as Frobenius reciprocity? We have the nice fact that 
if $V_1$ and $V_2$ are left $k[G]$-modules where $k$ is algebraically closed, then 
\[ \dim_k \Hom_{k[G]}(V_1, V_2) = \langle \chi_{V_1}, \chi_{V_2} \rangle. \] 
Somehow, the category of representations sees the character table. Let's 
consider the case where $V_1$ and $V_2$ are simple. If $V_1$ and $V_2$ are 
non-isomorphic, then both values are $0$. On the other hand, if $V_1$ and $V_2$
are isomorphic, then Schur's lemma tells us that $\Hom_{k[G]}(V_1, V_2)$ is 
just $k$, so these are both equal to $1$. Now, suppose that $V$ and $W$ 
are general left $k[G]$-modules with decompositions   
$V \cong V_1^{a_1} \oplus \cdots \oplus V_s^{a_s}$ and $W \cong V_1^{b_1} 
\oplus \cdots \oplus V_s^{b_s}$ where $V_1, \dots, V_s$ are simple. Then 
\[ \Hom_{k[G]}(V, W) \cong k^{a_1b_1 + \cdots + a_sb_s}, \] 
which is just $\langle \chi_V, \chi_W \rangle$. 

Now, let $W$ be a left $k[H]$-module and $V$ be a left $k[G]$-module. Then we have 
\[ \dim_k \Hom_{k[G]} (k[G] \otimes_{k[H]} W, V) = 
\dim_k \Hom_{k[G]} (W \uparrow_H^G, V) = \langle \chi_W \uparrow_H^G, \chi_V \rangle_G. \] 
On the other hand, we can see that 
\[ \dim_k \Hom_{k[H]} (W, {}_{k[H]} V) = \langle \chi_W, \chi_V \downarrow_H^G \rangle_H. \] 
Hence, the above isomorphism gives us Frobenius reciprocity. 

Let $\fix_n(\sigma)$ be the number of fixed points of $\sigma \in S_n$. 
Recall that we had 
\[ \frac{1}{n!} \sum_{\sigma \in S_n} \fix_n(\sigma) = 1 \] 
for all $n \geq 1$, and 
\[ \frac{1}{n!} \sum_{\sigma \in S_n} \fix_n(\sigma)^2 = 2 \] 
for all $n \geq 2$. First, we'll give an easy proof of the first result. 
Write $\fix_n = X_1 + \cdots + X_n$ where each $X_i$ is a random variable 
given by 
\[ X_i(\sigma) = \begin{cases} 
    1 & \text{if } \sigma(i) = i, \\ 
    0 & \text{otherwise.} 
\end{cases} \] 
We can think of the quantity on the left as the expected value of the number 
of fixed points for a given $\sigma \in S_n$. Then it follows that 
\[ \E(\fix_n) = \E(X_1) + \cdots + \E(X_n) = \frac1n + \cdots + \frac1n = 1. \] 
However, we can't extend this proof to the second result. Instead, we can use 
Frobenius reciprocity!

\begin{remark}{}
    Let $S_n$ act on $\{1, \dots, n\}$ as usual. Let $H$ be the subgroup of 
    $S_n$ consisting of elements that fix $n$, and note that $H \cong S_{n-1}$. 
    Then we will show on the assignment that  
    \[ \chi_{\triv} \uparrow_H^{S_n} (\sigma) = \fix_n(\sigma), \] 
    where $\chi_{\triv}$ is the trivial character on $H$. 
\end{remark}

Let's give another proof of the first fact. Let $\chi'_{\triv}$ be the trivial 
character on $S_n$. By Frobenius reciprocity, we obtain 
\[ \langle \chi_{\triv} \uparrow_H^{S_n}, \chi'_{\triv} \rangle_{S_n} 
= \langle \chi_{\triv}, \chi'_{\triv} \downarrow_H^{S_n} \rangle_H 
= \langle \chi_{\triv}, \chi_{\triv} \rangle_H = 1. \]
On the other hand, we see that 
\[ \langle \chi_{\triv} \uparrow_H^{S_n}, \chi'_{\triv} \rangle_{S_n} 
= \langle \fix_n, \chi'_{\triv} \rangle_{S_n} = \frac{1}{n!} \sum_{\sigma \in S_n} 
\fix(\sigma), \] 
and this gives the result. Now, let $n \geq 2$. Applying Frobenius reciprocity, 
we have 
\begin{align*}
    \langle \chi_{\triv} \uparrow_H^{S_n}, \fix_n \rangle_{S_n} 
    &= \langle \chi_{\triv}, \fix_n \downarrow_H^{S_4} \rangle_H \\ 
    &= \langle \chi_{\triv}, \chi_{\triv} + \fix_{n-1} \rangle_{S_{n-1}} \\ 
    &= \langle \chi_{\triv}, \chi_{\triv} \rangle_{S_{n-1}} 
    + \langle \chi_{\triv}, \fix_{n-1} \rangle_{S_{n-1}} \\ 
    &= 1 + 1 = 2. 
\end{align*}
The second equality is because $n$ is fixed in $H$, so this proves the second fact. 

In fact, we have the following more general result. 

\begin{prop}{}
    Let $m \leq n$ be positive integers. Then we have 
    \[ \frac{1}{n!} \sum_{\sigma \in S_n} \fix_n(\sigma)^m = B_m, \] 
    where $B_m$ is the number of partitions of $\{1, \dots, m\}$, called the 
    $m$-th Bell number. 
\end{prop}
\begin{pf}
    We proceed by induction on $m$. We have already proved this for $m = 1$ 
    and $m = 2$. Notice that 
    \[ \langle \chi_{\triv} \uparrow_H^{S_4}, \fix_n^{m-1} \rangle_{S_n} 
    = \langle \fix_n, \fix_n^{m-1} \rangle_{S_n} = \frac{1}{n!} 
    \sum_{\sigma \in S_n} \fix(\sigma) \fix(\sigma)^{m-1} 
    = \frac{1}{n!} \sum_{\sigma \in S_n} \fix(\sigma)^m. \] 
    On the other hand, Frobenius reciprocity gives us 
    \begin{align*}
        \langle \chi_{\triv} \uparrow_H^{S_4}, \fix_n^{m-1} \rangle_{S_n} 
        &= \langle \chi_{\triv}, \fix_n^{m-1} \downarrow_H^{S_n} \rangle_H \\
        &= \langle \chi_{\triv}, (1 + \fix_{n-1})^{m-1} \rangle_{S_{n-1}} \\
        &= \sum_{j=0}^{m-1} \binom{m-1}{j} \langle \chi_{\triv}, \fix_{n-1} \rangle_{S_{n-1}}. 
    \end{align*}
    We have $j \leq m-1 \leq n-1$, so by the induction hypothesis, we obtain 
    \[ \langle \chi_{\triv}, \fix_{n-1} \rangle_{S_{n-1}} 
    = \frac{1}{(n-1)!} \sum_{\sigma \in S_{n-1}} \fix(\sigma)^j = B_j. \] 
    In order to prove the result, it suffices to show that 
    \[ \sum_{j=0}^{m-1} \binom{m-1}{j} B_j = \sum_{j=0}^{m-1} \binom{m-1-j}{j} B_j = B_m. \] 
    To see this, note that $m$ has to be in one of the sets. Pick $m - 1 - j$ 
    other elements to be in the same set as $m$. Then the remaining $j$ 
    elements form a partition, which yields the desired identity. 
\end{pf}
