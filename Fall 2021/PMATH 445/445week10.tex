\section{Frobenius reciprocity (11/15/2021)}
Let $H$ be a subgroup, and let $W$ be a left $k[H]$-module. Then 
$W \uparrow_H^G\;= k[G] \otimes_{k[H]} W$ is a left $k[G]$-module. 
We saw in the previous lecture that 
\[ \chi_W \uparrow_H^G(g) = \frac{1}{|H|} \sum_{\{x\in G\,:\,x^{-1}gx \in H\}} \chi_W(x^{-1}gx). \] 
Also, if $G$ has disjoint coset decomposition $x_1H \cup \cdots \cup x_mH$, then 
\[ \chi_W \uparrow_H^G(g) = \sum_{\{i\,:\,x_i^{-1}gx_i \in H\}} \chi_W(x_i^{-1}gx_i). \] 

\begin{remark}{}
    If $H$ is a normal subgroup of $G$, then 
    \[ \chi_W \uparrow_H^G (g) = \begin{cases} 
        0 & \text{if } g \notin H, \\ 
        \frac{1}{|H|} \sum_{x\in G} \chi_W(x^{-1}gx) & \text{if } g \in H,
    \end{cases} \] 
    since $x^{-1}gx \in H$ if and only if $g \in xHx^{-1} = H$. 
\end{remark}

Let $V$ be a left $k[G]$-module. We have a left $k[H]$-module $V \downarrow_H^G$
by taking $V$ as a set, but we only remember the $k[H]$-module structure. 
That is, we have $V \downarrow_H^G\;= {}_{k[H]} V$. If $\chi_V$ is the 
character of $V$, then 
\[ \chi_V \downarrow_H^G (h) = \chi_V(h) \] 
for all $h \in H$. Indeed, we have $\chi_V \downarrow_H^G (h) = \Tr(\rho(h))$, 
where $\rho(h) : V \to V$ is given by $v \mapsto h \cdot v$. But $h \in H 
\subseteq G$, so this is just the trace of the particular element $\rho(h)$, 
which is $\chi_V(h)$. 

\begin{theo}[Frobenius reciprocity]{}
    Let $\phi : G \to C$ and $\psi : H \to \C$ be class functions, where 
    $H \leq G$ are finite groups. Then 
    \[ \langle \psi \uparrow_H^G, \phi \rangle_G = \langle \psi, \phi \downarrow_H^G \rangle_H. \] 
\end{theo}
\begin{pf}
    As with the transitivity of induction, this can also be done using a high
    level proof, but we will do this using a straightforward computation. We have 
    \begin{align*}
        \langle \psi \uparrow_H^G, \phi \rangle_G 
        &= \frac{1}{|G|} \sum_{g \in G} \psi \uparrow_H^G (g) \phi(g^{-1}) \\ 
        &= \frac{1}{|G|} \sum_{g \in G} \left( \frac{1}{|H|} 
        \sum_{\{x\in G\,:\,x^{-1}gx\in H\}} \psi(x^{-1}gx) \right) \phi(g^{-1}) \\
        &= \frac{1}{|G|} \frac{1}{|H|} \sum_{g\in G} \sum_{\{x\in G\,:\,x^{-1}gx\in H\}}
        \psi(x^{-1}gx) \phi(g^{-1}). 
    \end{align*}
    Now, substitute $h = x^{-1}gx$ so that $g = xhx^{-1}$ and $g^{-1} = 
    xh^{-1}x^{-1}$. Then, we obtain 
    \begin{align*}
        \langle \psi \uparrow_H^G, \phi \rangle_G 
        &= \frac{1}{|G|} \frac{1}{|H|} \sum_{h \in H} \sum_{x \in G} \psi(h) \phi(xh^{-1}x^{-1}) \\ 
        &= \frac{1}{|G|} \frac{1}{|H|} \sum_{h \in H} |G| \psi(h) \phi(h^{-1}) \\ 
        &= \frac{1}{|H|} \sum_{h\in H} \psi(h) \phi(h^{-1}) \\ 
        &= \langle \psi, \phi \downarrow_H^G \rangle_H, 
    \end{align*}
    where the second equality is because $\phi$ is a class function, which yields 
    $\phi(xh^{-1}x^{-1}) = \phi(h^{-1})$. 
\end{pf}

This result has a nice connection with tensor-hom adjunction. Let 
$W$ be a left $k[H]$-module and let $V$ be a left $k[G]$-module. Then we have 
a $k$-algebra isomorphism 
\[ \Hom_{k[G]}(k[G] \otimes_{k[H]} W, V) \cong \Hom_{k[H]}(W, \Hom_{k[G]}(k[G], V)). \] 
This essentially states that if we have a tensor on the left, then we can move it over. 

How is this the same as Frobenius reciprocity? We have the nice fact that 
if $V_1$ and $V_2$ are left $k[G]$-modules where $k$ is algebraically closed, then 
\[ \dim_k \Hom_{k[G]}(V_1, V_2) = \langle \chi_{V_1}, \chi_{V_2} \rangle. \] 
Somehow, the category of representations sees the character table. Let's 
consider the case where $V_1$ and $V_2$ are simple. If $V_1$ and $V_2$ are 
non-isomorphic, then both values are $0$. On the other hand, if $V_1$ and $V_2$
are isomorphic, then Schur's lemma tells us that $\Hom_{k[G]}(V_1, V_2)$ is 
just $k$, so these are both equal to $1$. Now, suppose that $V$ and $W$ 
are general left $k[G]$-modules with decompositions   
$V \cong V_1^{a_1} \oplus \cdots \oplus V_s^{a_s}$ and $W \cong V_1^{b_1} 
\oplus \cdots \oplus V_s^{b_s}$ where $V_1, \dots, V_s$ are simple. Then 
\[ \Hom_{k[G]}(V, W) \cong k^{a_1b_1 + \cdots + a_sb_s}, \] 
which is just $\langle \chi_V, \chi_W \rangle$. 

Now, let $W$ be a left $k[H]$-module and $V$ be a left $k[G]$-module. Then we have 
\[ \dim_k \Hom_{k[G]} (k[G] \otimes_{k[H]} W, V) = 
\dim_k \Hom_{k[G]} (W \uparrow_H^G, V) = \langle \chi_W \uparrow_H^G, \chi_V \rangle_G. \] 
On the other hand, we can see that 
\[ \dim_k \Hom_{k[H]} (W, {}_{k[H]} V) = \langle \chi_W, \chi_V \downarrow_H^G \rangle_H. \] 
Hence, the above isomorphism gives us Frobenius reciprocity. 

Let $\fix_n(\sigma)$ be the number of fixed points of $\sigma \in S_n$. 
Recall that we had 
\[ \frac{1}{n!} \sum_{\sigma \in S_n} \fix_n(\sigma) = 1 \] 
for all $n \geq 1$, and 
\[ \frac{1}{n!} \sum_{\sigma \in S_n} \fix_n(\sigma)^2 = 2 \] 
for all $n \geq 2$. First, we'll give an easy proof of the first result. 
Write $\fix_n = X_1 + \cdots + X_n$ where each $X_i$ is a random variable 
given by 
\[ X_i(\sigma) = \begin{cases} 
    1 & \text{if } \sigma(i) = i, \\ 
    0 & \text{otherwise.} 
\end{cases} \] 
We can think of the quantity on the left as the expected value of the number 
of fixed points for a given $\sigma \in S_n$. Then it follows that 
\[ \E(\fix_n) = \E(X_1) + \cdots + \E(X_n) = \frac1n + \cdots + \frac1n = 1. \] 
However, we can't extend this proof to the second result. Instead, we can use 
Frobenius reciprocity!

\begin{remark}{}
    Let $S_n$ act on $\{1, \dots, n\}$ as usual. Let $H$ be the subgroup of 
    $S_n$ consisting of elements that fix $n$, and note that $H \cong S_{n-1}$. 
    Then we will show on the assignment that  
    \[ \chi_{\triv} \uparrow_H^{S_n} (\sigma) = \fix_n(\sigma), \] 
    where $\chi_{\triv}$ is the trivial character on $H$. 
\end{remark}

Let's give another proof of the first fact. Let $\chi'_{\triv}$ be the trivial 
character on $S_n$. By Frobenius reciprocity, we obtain 
\[ \langle \chi_{\triv} \uparrow_H^{S_n}, \chi'_{\triv} \rangle_{S_n} 
= \langle \chi_{\triv}, \chi'_{\triv} \downarrow_H^{S_n} \rangle_H 
= \langle \chi_{\triv}, \chi_{\triv} \rangle_H = 1. \]
On the other hand, we see that 
\[ \langle \chi_{\triv} \uparrow_H^{S_n}, \chi'_{\triv} \rangle_{S_n} 
= \langle \fix_n, \chi'_{\triv} \rangle_{S_n} = \frac{1}{n!} \sum_{\sigma \in S_n} 
\fix(\sigma), \] 
and this gives the result. Now, let $n \geq 2$. Applying Frobenius reciprocity, 
we have 
\begin{align*}
    \langle \chi_{\triv} \uparrow_H^{S_n}, \fix_n \rangle_{S_n} 
    &= \langle \chi_{\triv}, \fix_n \downarrow_H^{S_4} \rangle_H \\ 
    &= \langle \chi_{\triv}, \chi_{\triv} + \fix_{n-1} \rangle_{S_{n-1}} \\ 
    &= \langle \chi_{\triv}, \chi_{\triv} \rangle_{S_{n-1}} 
    + \langle \chi_{\triv}, \fix_{n-1} \rangle_{S_{n-1}} \\ 
    &= 1 + 1 = 2. 
\end{align*}
The second equality is because $n$ is fixed in $H$, so this proves the second fact. 

In fact, we have the following more general result. 

\begin{prop}{}
    Let $m \leq n$ be positive integers. Then we have 
    \[ \frac{1}{n!} \sum_{\sigma \in S_n} \fix_n(\sigma)^m = B_m, \] 
    where $B_m$ is the number of partitions of $\{1, \dots, m\}$, called the 
    $m$-th Bell number. 
\end{prop}
\begin{pf}
    We proceed by induction on $m$. We have already proved this for $m = 1$ 
    and $m = 2$. Notice that 
    \[ \langle \chi_{\triv} \uparrow_H^{S_4}, \fix_n^{m-1} \rangle_{S_n} 
    = \langle \fix_n, \fix_n^{m-1} \rangle_{S_n} = \frac{1}{n!} 
    \sum_{\sigma \in S_n} \fix(\sigma) \fix(\sigma)^{m-1} 
    = \frac{1}{n!} \sum_{\sigma \in S_n} \fix(\sigma)^m. \] 
    On the other hand, Frobenius reciprocity gives us 
    \begin{align*}
        \langle \chi_{\triv} \uparrow_H^{S_4}, \fix_n^{m-1} \rangle_{S_n} 
        &= \langle \chi_{\triv}, \fix_n^{m-1} \downarrow_H^{S_n} \rangle_H \\
        &= \langle \chi_{\triv}, (1 + \fix_{n-1})^{m-1} \rangle_{S_{n-1}} \\
        &= \sum_{j=0}^{m-1} \binom{m-1}{j} \langle \chi_{\triv}, \fix_{n-1} \rangle_{S_{n-1}}. 
    \end{align*}
    We have $j \leq m-1 \leq n-1$, so by the induction hypothesis, we obtain 
    \[ \langle \chi_{\triv}, \fix_{n-1} \rangle_{S_{n-1}} 
    = \frac{1}{(n-1)!} \sum_{\sigma \in S_{n-1}} \fix(\sigma)^j = B_j. \] 
    In order to prove the result, it suffices to show that 
    \[ \sum_{j=0}^{m-1} \binom{m-1}{j} B_j = \sum_{j=0}^{m-1} \binom{m-1}{m-1-j} B_j = B_m. \] 
    To see this, note that $m$ has to be in one of the sets. Pick $m - 1 - j$ 
    other elements to be in the same set as $m$. Then the remaining $j$ 
    elements form a partition, which yields the desired identity. 
\end{pf}

\section{Inducing up and restricting down (11/19/2021)}
Let $H \leq G$ be groups, and let $\chi$ be a character on $H$. 
We'll be working towards Mackey's criterion, which gives us sufficient 
and necessary conditions for $\chi \uparrow_H^G$ to be irreducible. 

Recall that if $\chi$ is a character on $G$, then $\chi$ is irreducible 
if and only if $\langle \chi, \chi \rangle_G = 1$. So to start, we 
require that $\langle \chi \uparrow_H^G, \chi \uparrow_H^G \rangle_G = 1$
in order for $\chi \uparrow_H^G$ to be irreducible. By Frobenius reciprocity, 
we obtain 
\[ \langle \chi, \chi \uparrow_H^G \downarrow_H^G \rangle_H = 
\langle \chi \uparrow_H^G, \chi \uparrow_H^G \rangle_G = 1. \] 
To understand this, we need to know the effect of inducing up and then 
restricting down. Let $H, K \leq G$ be groups, and let $\chi$ be a character 
on $H$. What is the character $\chi \uparrow_H^G \downarrow_K^G$ on $K$? 

First, we'll review double cosets. For $H, K \leq G$, we put an equivalence 
relation on $G$ by setting $g_1 \sim g_2$ if and only if there exist elements
$k \in K$ and $h \in H$ such that $g_1 = kg_2h$. Let's verify that this is 
indeed an equivalence relation. Observe that we have reflexivity by 
taking $k = h = 1$. To see that $\sim$ is symmetric, notice that $g_1 = 
kg_2h$ implies that $g_2 = k^{-1}g_1h^{-1}$. Finally, $\sim$ is transitive 
since $g_1 = kg_2h$ and $g_2 = k'g_3h'$ implies that $g_1 = kk'g_3h'h$ with 
$kk' \in K$ and $h'h \in H$. It follows that $\sim$ partitions $G$ 
into disjoint sets called double cosets, and these are equivalence classes 
under $\sim$. For $g \in G$, we write 
\[ [g]_{\sim} = \{kgh : k\in K,\, h\in H\} = KgH. \] 
Write $G = Ks_1H \cup \cdots \cup Ks_rH$ where $s_1, \dots, s_r$ is a 
complete set of double coset representatives. Given a double coset 
representative $s \in G$, we let $H_s := K \cap sHs^{-1}$. If we have a 
character $\chi$ on $H$, then we get a new character $\chi_s$ on $H_s$ 
defined by $\chi_s(shs^{-1}) := \chi(h)$. To see that $\chi_s$ is a 
character, note that $\chi$ is a character by assumption, so there 
exists a representation $\rho : H \to \GL_d(\C)$ such that $\chi = \Tr(\rho)$. 
Define $\tilde\rho : sHs^{-1} \to \GL_d(\C)$ by $\tilde\rho(sHs^{-1}) = 
\rho(h)$. This is a group homomorphism since 
\[ \tilde\rho(sh_1s^{-1}sh_2s^{-1}) = \tilde\rho(sh_1h_2s^{-1}) = 
\rho(h_1h_2) = \rho(h_1)\rho(h_2) = \tilde\rho(sh_1s^{-1})\tilde\rho(sh_2s^{-1}). \] 
By restricting to $H_s$, we have the map 
\[ \tilde\rho|_{H_s} : K \cap sHs^{-1} \to \GL_d(\C), \] 
and we can take $\chi_s = \Tr(\tilde\rho|_{H_s})$. 

\begin{theo}{}
    Let $H, K \leq G$ be groups, and let $\chi$ be a character on $H$. Then 
    we have 
    \[ \chi \uparrow_H^G \downarrow_K^G \;= \sum_{i=1}^r \chi_{s_i} \uparrow_{H_{s_i}}^K, \] 
    where $G = Ks_1H \cup \cdots \cup Ks_rH$ and $H_{s_i} = K \cap s_iHs_i^{-1}$ for $i = 1, \dots, r$. 
\end{theo}

For $i = 1, \dots, r$, observe that we have 
\[ H_{s_i} = K \cap s_iHs_i^{-1} \subseteq K. \] 
Let $T_i \subseteq K$ be a set of left coset representatives for $K/H_{s_i}$ so that 
\[ K = \bigcup_{x \in T_i} xH_{s_i}. \] 
We claim that $\{xs_i : 1 \leq i \leq r,\, x \in T_i\}$ is a complete set 
of left coset representatives of $H$ in $G$. First, we need to show that 
the cosets $xs_iH$ are pairwise disjoint. Suppose that $x \in T_i$ and 
$y \in T_j$ with $x \neq y$ if $i = j$. We show that $xs_iH \neq ys_jH$. 

Suppose that $xs_iH = ys_jH$. Then $xs_iH \in Ks_iH$ since $x \in K$ and 
$ys_jH \in Ks_jH$ since $y \in K$. This gives us $Ks_iH \cap Ks_jH \neq 
\varnothing$, so $s_i = s_j$. Now, assume that we have $xs_iH = ys_iH$. 
Then we get $xs_iHs_i^{-1} = ys_iHs_i^{-1}$. In particular, notice that 
$y^{-1}x \in s_iHs_i^{-1} \cap K = H_{s_i}$, so $x = y$. Thus, the left cosets 
$xs_iH$ are all pairwise disjoint. 

Next, we show that each $g \in G$ is in one of the cosets $xs_iH$. 
First, there exists a unique $1 \leq i \leq r$ such that $g \in Ks_iH$. 
Write $g = ks_ih$ for some $k \in K$ and $h \in H$. Then, there exists a 
unique $x \in T_i$ such that $k \in xHs_i$. This gives $g \in ks_iH = 
x\theta s_iH$ for some $\theta \in H_{s_i} = K \cap s_iHs_i^{-1}$. 
It follows that $\theta = s_ih_0s_i^{-1}$ for some $h_0 \in H$. Finally, 
we obtain $g \in x\theta s_iH = x(s_ih_0s_i^{-1})s_i H = xs_i H$,
which proves the claim. 

Now that we have established that $\{xs_i : 1 \leq i \leq r,\,x\in T_i\}$ 
forms a complete set of left coset representatives of $H$ in $G$, 
we can prove Theorem 29.1. 

{\sc Proof of Theorem 29.1.} Let $k \in K$. Notice that 
$x^{-1}kx \in H_{s_i} = s_iHs_i^{-1} \cap K$, so we can write 
$x^{-1}kx = s_ihs_i^{-1}$ if $x^{-1}kx \in H_{s_i}$. Then $s_i^{-1}x^{-1}kxs_i 
= h$, so we get 
\begin{align*}
    \chi_{s_i} \uparrow_{H_{s_i}}^K (k) 
    &= \sum_{\{x\in T_i\,:\,x^{-1}kx\in H_{s_i}\}} \chi_{s_i}(x^{-1}kx) \\
    &= \sum_{\{x\in T_i\,:\,x^{-1}kx\in H_{s_i}\}} \chi_{s_i}(s_ihs_i^{-1}) \\  
    &= \sum_{\{x\in T_i\,:\,x^{-1}kx\in H_{s_i}\}} \chi(h) \\ 
    &= \sum_{\{x\in T_i\,:\,x^{-1}kx\in H_{s_i}\}} \chi(s_i^{-1}x^{-1}hxs_i).
\end{align*}
Next, we'll look at the left side of the equation. First, we immediately 
notice that $\chi \uparrow_H^G \downarrow_K^G (k) = \chi \uparrow_H^G (k)$. 
Since $G$ has left coset decomposition given by $G = \bigcup_{i=1}^r 
\bigcup_{x\in T_i} xs_i H$, it follows that 
\begin{align*}
    \chi \uparrow_H^G (k) 
    &= \sum_{i=1}^r \sum_{\{x\in T_i\,:\,(xs_i)^{-1}k(xs_i) \in H\}} \chi((xs_i)^{-1}k(xs_i)) \\
    &= \sum_{i=1}^r \sum_{\{x\in T_i\,:\,s_i^{-1}x^{-1}kxs_i \in H\}} \chi(s_i^{-1}x^{-1}kxs_i). 
\end{align*} 
Note that $s_i^{-1}x^{-1}kxs_i \in H$ if and only if $x^{-1}kx \in s_iHs_i^{-1}$. 
Moreover, we have $T_i \subseteq K$, so $x^{-1}kx \in K$ as well. 
Hence, we have $x^{-1}kx \in H_{s_i} = K \cap s_iHs_i^{-1}$, so we obtain 
\begin{align*}
    \chi \uparrow_H^G (k) 
    = \sum_{i=1}^r \sum_{\{x\in T_i\,:\,x^{-1}kx \in H_{s_i}\}} \chi(s_i^{-1}x^{-1}kxs_i) 
    = \sum_{i=1}^r \chi \uparrow_{H_{s_i}}^K (k),
\end{align*}
and this completes the proof. \qed

\section{Mackey's criterion, unitary operators (11/22/2021)}
Let $H \leq G$ be groups, and let $\chi$ be a character on $H$. 
When is $\chi \uparrow_H^G$ irreducible? Recall that we showed that 
$\chi \uparrow_H^G$ is irreducible if and only if 
\[ \langle \chi, \chi \uparrow_H^G \downarrow_H^G \rangle_H = 
\langle \chi \uparrow_H^G, \chi \uparrow_H^G \rangle_G = 1 \] 
by applying Frobenius reciprocity. Now, let $G = Hs_1H \cup \cdots 
\cup Hs_rH$ be the double coset decomposition of $G$. We may assume that 
$s_1 = 1$ as it must be in one of the double cosets. In particular, 
notice that $Hs_1 H = H \cdot 1 \cdot H = H$, so by Theorem 29.1, we obtain 
\[ \chi \uparrow_H^G \downarrow_H^G\; = \sum_{i=1}^r \chi_{s_i} \uparrow_{H_{s_i}}^H, \] 
where $H_{s_i} = H \cap s_i H s_i^{-1}$ and $\chi_{s_i}(s_ihs_i^{-1}) = \chi(h)$. 
Therefore, $\chi \uparrow_H^G$ being irreducible is equivalent to 
\[ \left\langle \chi,\,\sum_{i=1}^r \chi_{s_i} \uparrow_{H_{s_i}}^H \right\rangle_{\!\!H} = 1. \] 
Since $\chi_{s_1} = \chi \uparrow_H^H = \chi$, we obtain 
\[ \left\langle \chi,\, \chi + \sum_{i=2}^r \chi_{s_i} \uparrow_{H_{s_i}}^H \right\rangle_{\!\!H} 
= \langle \chi, \chi \rangle_H + \sum_{i=2}^r \langle \chi, \chi_{s_i} \uparrow_{H_{s_i}}^H \rangle_H = 1. \] 
Note that $\langle \chi, \chi \rangle_H = 1$ and $\langle \chi, \chi_{s_i} \uparrow_{H_{s_i}}^H \rangle_H \geq 0$ 
for all $i = 2, \dots, r$. Therefore, we see that $\chi \uparrow_H^G$ is 
irreducible if and only if $\chi$ is irreducible and $\langle \chi, 
\chi_{s_i} \uparrow_{H_{s_i}}^H \rangle_H = 0$ for all $i = 2, \dots, r$. 
Here, we notice that $s_i \notin H$ for all $i = 2, \dots, r$. This gives 
us Mackey's criterion. 

\begin{theo}[Mackey's criterion]{}
    Let $H \leq G$ be groups, and let $\chi$ be a character on $H$. Then 
    the induced character $\chi \uparrow_H^G$ is irreducible if and only if 
    $\chi$ is irreducible and $\langle \chi, \chi_{s_i} \uparrow_{H_{s_i}} \rangle_H = 0$
    for all $s_i \in G \setminus H$. 
\end{theo}

\begin{exmp}{}
    Let $G = \langle x, y \mid x^3 = y^7 = 1,\, xyx^{-1} = y^2 \rangle$. Let 
    $H = \langle y \rangle$. Define a character on $H$ by $\chi(y) = e^{2\pi i/7}$, 
    which is irreducible since it is of degree $1$. Notice that $H \trianglelefteq G$, 
    so we have $G = H \cup HxH \cup Hx^2H$ since 
    \[ HxH = x(x^{-1}Hx)H = xHH = xH, \] 
    and similarly, $Hx^2H = x^2H$. Moreover, we also have 
    \begin{align*}
        H_x &= H \cap xHx^{-1} = H, \\ 
        H_{x^2} &= H \cap x^2Hx^{-2} = H. 
    \end{align*}
    By definition, we have $\chi_x(xyx^{-1}) = \chi(y) = e^{2\pi i/7}$. 
    This yields $\chi_x(y^2) = e^{2\pi i/7}$ and hence 
    \[ \chi_x(y) = \chi_x(y^8) = \chi_x(y^2)^4 = e^{8\pi i/7}. \] 
    On the other hand, note that $\chi_{x^2}(y^4) = \chi_{x^2}(x^2yx^{-2})
    = e^{2\pi i/7}$, which implies that 
    \[ \chi_{x^2}(y) = \chi_{x^2}(y^8) = \chi_{x^2}(y^4)^2 = e^{4\pi i/7}. \] 
    Finally, notice that $H_x = H_{x^2} = H$, so $\chi_x \uparrow_{H_x}^H 
    = \chi_x$ and $\chi_{x^2} \uparrow_{H_{x^2}}^H = \chi_{x^2}$. These 
    are both distinct characters from $\chi$ by our above calculations, 
    so $\langle \chi, \chi_x \rangle_H = \langle \chi, \chi_{x^2} \rangle_H = 0$.
    Thus, Mackey's criterion holds in this case. 
\end{exmp}

We've now reached the end of the syllabus, but not the end of the lecture period. 
Let's now focus on some results about subgroups of $\GL_d(\C)$, since 
representation theory at its heart is about looking at images of groups 
inside $\GL_d(\C)$. Some of these are more useful in the context of 
representations of infinite groups, but Jordan's theorem, which we will soon 
give, is a very useful for representations of finite groups. In many of these 
theorems, we will give explicit bounds; these will not be the best bounds, 
but are done for the sake of proving that explicit bounds exist. 

\begin{theo}[Jordan's theorem]{}
    Let $d \geq 1$ be an integer. Then there exists an integer $N = N(d)$ 
    such that if $G$ is a finite subgroup of $\GL_d(\C)$, then $G$ has an 
    abelian subgroup of index at most $N(d)$. 
\end{theo}

When $G$ is small, this doesn't say much, as the trivial group has index 
at most $N(d)$ if $|G| \leq N(d)$. On the other hand, this is very useful 
when $G$ is large, as it says that $G$ is ``almost abelian''. As an 
application, we can use this to prove results about representations of $S_n$. 
The proof of Jordan's theorem requires some analysis, so we'll recall some 
of the results as we go. 

Let $G$ be a finite group, and let $V$ be a left $\C[G]$-module so that 
$V \cong \C^d$ as a $\C$-vector space. We can define an inner product 
between $v = (v_1, \dots, v_d)^T$ and $w = (w_1, \dots, w_d)^T$ by 
\[ \langle v, w \rangle = \sum_{i=1}^d v_i \overline{w_i}. \] 
Recall that an operator $U : V \to V$ is \textbf{unitary} if 
$\langle Uv, w \rangle = \langle v, U^{-1}w \rangle$ for all $v, w \in V$. 
For all $g \in G$, we have a linear map $L_g : V \to V$ defined by 
$L_g(v) = g \cdot v$, but note that $L_g$ is not unitary in general. 
To get around this, we can construct a new inner product by 
\[ \langle v, w \rangle_G := \frac{1}{|G|} \sum_{g \in G} \langle gv, gw \rangle, \] 
where $\langle \cdot, \cdot \rangle$ in the sum is our ordinary Euclidean inner 
product over $\C^d$. 

\begin{exercise}{}
    Verify that $\langle \cdot, \cdot \rangle_G$ defined as above is an inner product. 
\end{exercise}

\begin{prop}{}
    The operators $L_g : V \to V$ are unitary with respect to the inner product 
    $\langle \cdot, \cdot \rangle_G$. 
\end{prop}
\begin{pf}
    We need to show that $\langle L_gv, w \rangle_G = \langle v, L_{g^{-1}}w \rangle_G$, 
    since $L_{g^{-1}}$ is the inverse of $L_g$. First, we have 
    \[ \langle L_gv, w \rangle_G = \frac{1}{|G|} \sum_{x\in G} \langle xL_g v, xw \rangle 
    = \frac{1}{|G|} \sum_{x\in G} \langle xgv, xw \rangle. \] 
    By substituting $y = xg$, we obtain 
    \[ \langle L_gv, w \rangle_G = \frac{1}{|G|} \sum_{y \in G} \langle yv, yg^{-1}w \rangle 
    = \frac{1}{|G|} \sum_{y\in G} \langle yv, yL_{g^{-1}}w \rangle = \langle v, L_{g^{-1}}w \rangle_G. 
    \qedhere \] 
\end{pf}
This is very useful for computations. Let $V$ be a representation, and let
$W$ be a subrepresentation of $V$. If we define 
\[ W^{\perp} := \{v \in V : \langle v, w \rangle_G = 0 \text{ for all } w \in W\}, \] 
then $V = W \oplus W^{\perp}$. Note that $W^{\perp}$ is another representation, 
because if we take $u \in W^{\perp}$, $g \in G$, and $w \in W$, then we obtain 
\[ \langle g \cdot u, w \rangle_G = \langle u, g^{-1} \cdot w \rangle_G = 0 \] 
as $g^{-1} \cdot w \in W$. Then $u \in W^{\perp}$ implies that $g \cdot u \in W^{\perp}$. 

Note that $M_d(\C)$ has a norm, called the \textbf{operator norm}, which is defined by 
\[ \|A\| := \sup_{\|v\|_G = 1} \sqrt{\langle Av, Av \rangle_G}, \] 
where $\|v\|_G = \sqrt{\langle v, v \rangle_G}$. In particular, this 
satisfies $\|A + B\| \leq \|A\| + \|B\|$, and this turns $M_d(\C)$ into 
a metric space where the distance from $A$ to $B$ is given by $\|A - B\|$. 

Let $U$ be the set of all unitary matrices in $M_d(\C)$. Then $U$ is compact;
that is, every open cover has a finite subcover. Note that $M_d(\C)$ is 
finite-dimensional over $\C$, so by Heine-Borel, this is equivalent to 
saying that $U$ is closed and bounded. To see that $U$ is bounded, take 
$A \in U$. Then we have 
\[ \langle Av, Av \rangle_G = \langle v, A^{-1}Av \rangle_G = \langle v, v 
\rangle_G, \] 
which implies that $\|A\| = 1$, because if we are given $v$ on the unit ball
(any $v$ having inner product $1$ with itself), then $Av$ is also on the unit 
ball. This is called being an isometry; it preserves distance. To 
see that $U$ is closed, recall that for any $A \in M_d(\C)$, there exists 
$A^* \in M_d(\C)$, called the \textbf{adjoint}, such that $\langle Av, w 
\rangle = \langle v, A^*w \rangle$, with $A$ being unitary if and only if 
$A^{-1} = A^*$. Define the map $f : M_d(\C) \to M_d(\C)$ by 
\[ f(A) = AA^*. \] 
We leave it as an exercise to show that $f$ is continuous. Now, $\{I\}$ 
is closed, and the preimage of a closed set under a continuous map is 
closed, so it follows that $f^{-1}(\{I\}) = U$ is closed. Therefore, 
$U$ is compact. 
