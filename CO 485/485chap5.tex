\section{RSA Security}
As we previously discussed in Section 4.1, one way to attack the RSA encryption 
scheme would be to factor the public modulus. We have covered some special 
purpose and general purpose factoring algorithms. In fact, RSA-$129$, a 
$426$-bit RSA modulus, was factored using a variant of the quadratic sieve 
algorithm in 1994. More recently, in 2020, the 
\href{https://en.wikipedia.org/wiki/General_number_field_sieve}{number field 
sieve algorithm} was used to factor RSA-$250$, a $829$-bit RSA modulus. We will 
not cover the number field sieve algorithm in this course, but for the sake of 
completeness, we will state that its complexity is $L[1/3, \sqrt[3]{64/9}]$, 
which is asymptotically better than the $L[1/2, 1]$ complexity of the quadratic 
sieve algorithm. 

From a practical point of view, one can set the RSA key size so that it is 
infeasible to find a factor of the modulus using the best known factoring algorithms. 
For instance, if we assume that the number field sieve algorithm is the best 
option to factor an $\ell$-bit RSA public modulus $N$ and that it runs in time 
\[ T(\ell) = e^{\sqrt[3]{64/9} (\ln 2^\ell)^{1/3} (\ln\ln 2^\ell)^{2/3}}, \] 
we find that $T(2048) \approx 2^{117}$ and $T(3072) \approx 2^{139}$, 
both of which are considered an infeasible number of operations to carry out. 
Therefore, RSA keys of length at least $2048$-bits are believed to yield 
secure encryption schemes by today's standards. However, a $512$-bit modulus 
would instantiate an insecure RSA encryption scheme because $T(512) \approx 
2^{64}$, which is considered to be a feasible number of operations to carry out. 

In short, the intractability of factoring the RSA modulus is necessary for the 
security of RSA. However, we can ask another question. Can we break RSA 
without having to factor $N$? One strategy would be to compute the secret 
exponent $d$, which would result in breaking RSA since the knowledge of $d$ 
immediately allows decrypting ciphertexts. Computing $d$ is not harder than 
factoring $N$, because if we know the prime factors $p$ and $q$ of $N$, 
then we can easily compute $\phi = (p-1)(q-1)$ and recover $d$ as the 
multiplicative inverse of $e$ modulo $\phi$. It turns out that computing 
$d$ is not easier than factoring $N$; that is, these two problems are equivalent. 
Indeed, we will give a strategy for reducing the factoring $N$ problem 
to the computing $d$ problem in polynomial time. Let $(N, e, d)$ be the RSA 
parameters as before, and suppose that we have access to an algorithm that 
computes $d$. We first write $ed - 1 = 2^s t$, where $t$ is an odd integer. 
Select $a \in \Z_N^*$ at random and try to find an integer $1 \leq r \leq s$ 
such that $a^{2^{r-1}t} \not\equiv \pm 1 \pmod N$ and $a^{2^r t} \equiv 1 \pmod N$, 
in which case $\gcd(a^{2^{r-1}t} - 1, N)$ yields a nontrivial factor of $N$ 
(using similar reasoning as in the Miller-Rabin primality test arguments). 
If we cannot find such an $r$, then select another $a \in \Z_N^*$ and repeat the 
process. Using the Chinese remainder theorem, we can show that we expect to 
try two $a \in \Z_N^*$ on average before finding a factor of $N$. 

In fact, Weiner showed in 1990 \cite{54902} that if the secret exponent $d$ 
of RSA is bounded by $\frac13 N^{1/4}$, then one can efficiently recover $d$. 
This attack is known as the low private exponent attack. We refer the reader 
to \href{https://crypto.stanford.edu/~dabo/papers/RSA-survey.pdf}{Twenty 
Years of Attacks on the RSA Cryptosystem} for a proof of Weiner's low private 
exponent attack, as well as other strategies for breaking RSA. 