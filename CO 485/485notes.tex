\documentclass[10pt]{article}
\usepackage[T1]{fontenc}
\usepackage{amsmath,amssymb,amsthm}
\usepackage[shortlabels]{enumitem}
\usepackage[english]{babel}
\usepackage[utf8]{inputenc}
\usepackage{fancyhdr}
\usepackage{bold-extra}
\usepackage{color}   
\usepackage{tocloft}
\usepackage{graphicx}
\usepackage{lipsum}
\usepackage{wrapfig}
\usepackage{cutwin}
\usepackage{hyperref}
\usepackage{lastpage}
\usepackage{multicol}
\usepackage{tikz}
\usepackage{tikz-cd}
\usepackage{xcolor}
\usepackage{microtype}
\usepackage{algpseudocode}

% some useful math commands
\newcommand{\eps}{\varepsilon}
\newcommand{\R}{\mathbb{R}}
\newcommand{\C}{\mathbb{C}}
\newcommand{\N}{\mathbb{N}}
\newcommand{\Z}{\mathbb{Z}}
\newcommand{\Q}{\mathbb{Q}}
\newcommand{\K}{\mathbb{K}}
\newcommand{\F}{\mathbb{F}}

\newcommand{\PubKey}{{\sf PubKey}}
\newcommand{\SecKey}{{\sf SecKey}}
\newcommand{\Enc}{{\sf Enc}}
\newcommand{\Dec}{{\sf Dec}}
\newcommand{\ord}{{\sf ord}}
\newcommand{\w}{{\sf w}}
\newcommand{\QR}{{\sf QR}}
\newcommand{\QNR}{{\sf QNR}}
\newcommand{\RNG}{{\sf RNG}}
\newcommand{\IsPrime}{{\sf IsPrime}}
\newcommand{\True}{{\sf True}}
\newcommand{\False}{{\sf False}}
\newcommand{\counter}{{\sf counter}}
\newcommand{\FW}{{\sf FW}}
\newcommand{\EJW}{{\sf EJW}}
\newcommand{\MRW}{{\sf MRW}}

\DeclareMathOperator{\GL}{GL}
\DeclareMathOperator{\id}{id}
\DeclareMathOperator{\Arg}{Arg}
\DeclareMathOperator{\Log}{Log}
\DeclareMathOperator{\PV}{PV}
\DeclareMathOperator{\sech}{sech}
\DeclareMathOperator{\csch}{csch}
\DeclareMathOperator{\Res}{Res}

\newcommand{\suchthat}{\;\ifnum\currentgrouptype=16 \;\middle|\;\else\mid\fi\;}

% title formatting
\newcommand{\newtitle}[4]{
  \begin{center}
	\huge{\textbf{\textsc{#1 Course Notes}}}
    
	\large{\sc #2}
    
	{\sc #3 \textbullet\, #4 \textbullet\, University of Waterloo}
	\normalsize\vspace{1cm}\hrule
  \end{center}
}

% for theorems
\newtheoremstyle{newstyle}      
{} %Aboveskip 
{-0.25pt} %Below skip
{\mdseries} %Body font e.g.\mdseries,\bfseries,\scshape,\itshape
{} %Indent
{\scshape} %Head font e.g.\bfseries,\scshape,\itshape
{.} %Punctuation afer theorem header
{ } %Space after theorem header
{} %Heading

\theoremstyle{newstyle}

\newtheorem*{prop*}{Proposition}
\newtheorem*{cor*}{Corollary}
\newtheorem*{exercise*}{Exercise}
\newtheorem*{lemma*}{Lemma}
\newtheorem*{remark*}{Remark}
\newtheorem*{exmp*}{Example}
\newtheorem*{defn*}{Definition}
\newtheorem*{thm*}{Theorem}
\newtheorem*{notation*}{Notation}
\newtheorem{thm}{Theorem}[section]
\newtheorem{fact}[thm]{Fact}
\newtheorem{cor}[thm]{Corollary}
\newtheorem{lemma}[thm]{Lemma}
\newtheorem{remark}[thm]{Remark}
\newtheorem{prop}[thm]{Proposition}
\newtheorem{defn}[thm]{Definition}
\newtheorem{claim}[thm]{Claim}
\newtheorem{axiom}[thm]{Axiom}
\newtheorem{notation}[thm]{Notation}
\newtheorem{exercise}[thm]{Exercise}
\newtheorem{exmp}[thm]{Example}
\newtheorem{algo}[thm]{Algorithm}

% new proof environment
\makeatletter
\newenvironment{pf}[1][\proofname]{\par
  \pushQED{\qed}%
  \normalfont \topsep0\p@\relax
  \trivlist
  \item[\hskip\labelsep\scshape
  #1\@addpunct{.}]\ignorespaces
}{%
  \popQED\endtrivlist\@endpefalse
}
\makeatother

% 1-inch margins
\topmargin 0pt
\advance \topmargin by -\headheight
\advance \topmargin by -\headsep
\textheight 8.9in
\oddsidemargin 0pt
\evensidemargin \oddsidemargin
\marginparwidth 0.5in
\textwidth 6.5in

\parindent 0in
\parskip 1.5ex

\setlist[itemize]{topsep=0pt}
\setlist[enumerate]{topsep=0pt}

% hyperlinks
\hypersetup{
  colorlinks=true, 
  linktoc=all,     % table of contents is clickable  
  allcolors=blue  % all hyperlink colours
}

% table of contents
\addto\captionsenglish{
  \renewcommand{\contentsname}%
    {Table of Contents}%
}
\renewcommand{\cftsecfont}{\normalfont}
\renewcommand{\cftsecpagefont}{\normalfont}
\cftsetindents{section}{0em}{2em}

\fancypagestyle{plain}{%
\fancyhf{} % clear all header and footer fields
\lhead{CO 485: Fall 2021}
\fancyhead[R]{Table of Contents}
%\headrule
\fancyfoot[R]{{\small Page \thepage\ of \pageref*{LastPage}}}
}

% headers and footers
\pagestyle{fancy}
\renewcommand{\sectionmark}[1]{\markboth{#1}{#1}}
\lhead{CO 485: Fall 2021}
\cfoot{}
\setlength\headheight{14pt}

%\setcounter{section}{-1}

\begin{document}

\pagestyle{fancy}
\newtitle{CO 485}{The Mathematics of Public Key Cryptography}{Koray Karabina}{Fall 2021}
\rhead{Table of Contents}
\rfoot{{\small Page \thepage\ of \pageref*{LastPage}}}

\tableofcontents
\vspace{1cm}\hrule
\fancyhead[R]{\nouppercase\rightmark}
\newpage 
\fancyhead[R]{Week \thesection: \nouppercase\leftmark}

\section{Introduction}

\subsection{What is Cryptography?}
Cryptography is the science of securing information and communication in the presence of attackers. As an example, cryptography helps clients do online banking safely. In a typical online banking application, clients are connected to their banks through a wireless channel which can be observed or controlled by attackers. In particular, we assume that attackers can read, modify, delete exchanged messages, and inject new messages into the channel. How can we secure a channel between two parties if they have never met before?

This scenario motivates the fundamental goals of cryptography.
\begin{itemize}
    \item The first goal is {\bf confidentiality}. Confidentiality assures that only authorized parties can access or see the data. Attackers should not be able to extract the real content of the data even though they can read or steal packages exchanged in the channel. We would like to keep our banking passwords to ourselves.
    \item The second goal is {\bf message authentication}. Message authentication, also known as data origin authentication, assures that parties can verify the source of the received messages. When we receive a message, which is claimed to be sent from our bank, we have to make sure it has indeed been sent from our bank.
    \item The third goal is {\bf data integrity}. Data integrity assures that data cannot be altered by unauthorized or unknown means. When we are willing to pay 1,500 CAD for a new laptop, and commit to this transaction at a time, we have to make sure that this transaction cannot be modified as a 15,000 CAD worth transaction at a later time.
    \item Finally, the fourth goal is {\bf non-repudiation} which prevents communicating parties from falsely denying their actions. Once we commit to a 1,500 CAD transaction, then we should not be able to break our commitment.
\end{itemize}

There is a variety of cryptographic techniques that help achieve the fundamental goals of cryptography. As a high-level overview, encryption algorithms help achieve confidentiality; digital signature schemes help achieve authentication and non-repudiation; message authentication codes help achieve data integrity.

{\bf Public key cryptography.} Now, let’s go back to our online banking scenario. Before the communication between the client and the bank starts, the bank generates a public key, secret key pair $(\PubKey, \SecKey)$. The key $\PubKey$ is public in the sense that it is known to everyone including attackers. The key $\SecKey$ is secret in the sense the bank is the only party that knows it. After the bank generates its key pair $(\PubKey, \SecKey)$, the bank visits a certification authority. The certification authority issues a certificate to validate the public key $\PubKey$, and its ownership by the bank.

One can view the public key certificate of a website by clicking on the lock icon displayed on the 
web browser. This should display a certificate viewer, where one can click on the ``Details'' tab. 
The certificate includes information about the website, the certification authority (also known as the verifier), the validity period of the certificate, the website’s public key, and the certification authority’s signature. The public key and the signature are long sequences of hexadecimal characters.
Of course, we do not see any trace of the secret key in the certificate.

For a concrete example, the Bank of Canada is using the RSA public key cryptosystem, and its public key consists of two integers $N$ and $e$, which are called the {\bf public modulus} and the {\bf public exponent}, respectively. The certificate encodes these two integers (see \href{https://en.wikipedia.org/wiki/ASN.1}{ASN.1}), and displays them using hexadecimal (base-16) representation. For decoding, one can copy and paste the encoded public key to an ASN.1 decoder. For instance, this \href{https://lapo.it/asn1js/}{ASN.1 JavaScript decoder} can be used to decode the hexadecimal and integer values of $N$ and $e$.

\begin{exercise*}
Find the public key modulus and exponent of the \href{https://uwaterloo.ca/}{University of Waterloo} and \href{google.ca}{google.ca}.
\end{exercise*}

We should note two important properties of the Bank of Canada’s public key values. 
Notice that $N$ is a 2048-bit composite integer, which is supposed to be very hard to factor; 
moreover, $e=2^{16}+1$ is a relatively small integer with Hamming weight 2. The first property assures the security of the system, and the second property is for efficient implementation of the protocol. We will explain these in more detail when we cover the RSA cryptosystem.

Now that we know more about public key certificates, we can summarize the sequence of steps for turning a wireless {\bf insecure} communication channel into a secure channel between a client and her bank.
\begin{enumerate}
    \item {\bf Public key generation.} The bank generates a public key and secret key pair.

    \item {\bf Signature generation.} Certificate authority issues a certificate to the bank, validating the public key and its ownership by the bank.

    \item {\bf Signature verification.} The client obtains the bank’s certificate, and verifies the certification authority’s signature on the bank’s certificate. In other words, the client authenticates the bank.

    \item {\bf Random number generation.} The client creates a random secret session key $K$.

    \item {\bf Public key encryption.} The client encrypts $K$ using the bank’s public key $\PubKey$, and sends this encrypted key to the bank.

    \item {\bf Public key decryption.} The bank decrypts the client’s ciphertext using its private key $\SecKey$, and recovers the session key $K$.

    \item {\bf Symmetric key cryptography.} The client and the bank use the shared secret key $K$ to secure and authenticate their communication, using symmetric key cryptography.

    \item {\bf Efficiency and security.} Presumably, the steps above can be performed efficiently and that attackers cannot gather any useful information about the secret key $K$, and that the communication channel stays secure.
\end{enumerate}

In this course, we will cover these steps in detail with the exception of symmetric key cryptography. 
One can read Chapter 1.1 and Chapter 1.7 in \href{https://link-springer-com.proxy.lib.uwaterloo.ca/book/10.1007/978-1-4939-1711-2}{An Introduction to Mathematical Cryptography} and Chapter 1.5 in \href{https://www-taylorfrancis-com.proxy.lib.uwaterloo.ca/books/mono/10.1201/9780429466335/handbook-applied-cryptography-alfred-menezes-paul-van-oorschot-scott-vanstone}{Handbook of Applied Cryptography} 
for introductory level texts on symmetric key cryptography.

\subsection{The RSA Cryptosystem}

\href{https://en.wikipedia.org/wiki/RSA_(cryptosystem)}{RSA} is a public key cryptosystem invented by Ron Rivest, Adi Shamir, and Leonard Adleman, and published in 1978. The RSA cryptosystem offers a public key encryption scheme and a digital signature scheme.

{\bf The RSA encryption scheme.} The RSA public key encryption scheme consists of three algorithms.
\begin{enumerate}
    \item {\bf Key generation.} The purpose of this algorithm is to generate a public key and secret key pair. The public key is a pair of integers
    \[ \PubKey = [N, e], \]
    where $N = p\cdot q$ is a product of two randomly chosen distinct primes $p$ and $q$, and 
    $e \in (1, (p-1)(q-1))$ such that
    \[ \gcd(e, (p-1)(q-1)) = 1. \]
    For ease of notation, we define $\phi = (p-1)(q-1)$ so that $\gcd(e, \phi) = 1$. As mentioned in 
    Section 1.1, $N$ and $e$ are also called the {\bf public modulus} and the {\bf public 
    exponent}, respectively. The secret key is a tuple of integers 
    \[ \SecKey = [p, q, d], \]
    where $p$ and $q$ are as chosen before, and $d$ is the multiplicative inverse of 
    $e$ modulo $\phi$. That is, 
    \[ e \cdot d \equiv 1 \pmod \phi. \]
    The integer $d$ is also known as the {\bf secret exponent}.
    
    \newpage
    {\sc Example.} The bank chooses two distinct 8-bit random primes $p = 233$ and $q = 211$. 
    Therefore, $N = p \cdot q = 49163$ and $\phi = (p-1)(q-1) = 48720$. Next, the bank 
    chooses $e = 20771$ (one can verify that $\gcd(e, \phi) = 1$). This choice of $e$ fixes 
    $d = 36971$ since $e$ and $d$ must satisfy $e \cdot d \equiv 1 \pmod \phi$. Therefore, the bank has 
    a public key and secret pair given by 
    \begin{align*}
        \PubKey &= [N, e] = [49163, 20771], \\
        \SecKey &= [p, q, d] = [233, 211, 36971].
    \end{align*}
    
    {\sc Question.} How do you choose a fixed-length prime number at random? How do you compute modular multiplicative inverses? Are these methods efficient? What does efficient mean?
    
    \item {\bf Encryption algorithm.} Let $\Z_N$ denote the set of integers modulo $N$. 
    For a given public key $\PubKey = [N, e]$, the encryption algorithm takes as input a message 
    $m$ from the message space $\Z_N$, and outputs the ciphertext $c = m^e \pmod N$ in the 
    ciphertext space $\Z_N$. We can denote this process by 
    \begin{align*}
        \Enc_{N,e} : \Z_N &\to \Z_N \\
        m &\mapsto c = m^e \pmod N,
    \end{align*}
    or simply by $\Enc(m) = m^e \mod N$ when $N$ and $e$ are clear from the context.
    
    {\sc Example.} The client obtains the bank’s public key $\PubKey = [N,e] = [49163,20771]$, 
    and encrypts her \href{https://en.wikipedia.org/wiki/Card_security_code}{Card Security Code (CSC)} $m=123$ by 
    \[ c = m^e \pmod N = 123^{20771} \pmod {49163} = 37917. \]
    
    {\sc Question.} How do you (efficiently) perform modular exponentiation?
    
    \item {\bf Decryption algorithm.} For a given public modulus $N$ and the secret exponent $d$, the decryption algorithm takes as input a ciphertext $c$ from the ciphertext space $\Z_N$, 
    and outputs the message $m = c^d \pmod N \in \Z_N$. We can denote this process by 
    \begin{align*}
        \Dec_{N,d} : \Z_N &\to \Z_N \\
        c &\mapsto m = c^d \pmod N,
    \end{align*}
    or simply by $\Dec(c) = c^d \pmod N$ when $N$ and $d$ are clear from the context.
    
    {\sc Remark.} Note that $p$ and $q$ are implicit in the decryption algorithm above. However, as we will see later, one can explicitly use them for a more efficient decryption algorithm.
    
    It should now be clear why $N$, $e$, and $d$ are called the public modulus, public exponent, and secret exponent, respectively.
    
    {\sc Example.} The bank receives the ciphertext $c=37917$ from the client, and uses its secret key to decrypt with 
    \[ m = c^d \pmod N = 37917^{36971} \pmod {49163} = 123. \]
    Observe that the bank succesfully recovered the client's CSC. 
    
    {\sc Question.} Can you guarantee that the RSA decryption algorithm will always work correctly and recover the original message? Is the RSA encryption scheme secure? What does it mean for a public key encryption scheme to be secure?
\end{enumerate}\newpage 
\section{Some Number Theory}

\subsection{Modular Arithmetic}
We begin with some notation and definitions. The set of integers is denoted by $\Z$. For an 
integer $n \geq 2$, we define the sets 
\begin{align*}
    \Z_n &= \{a \in \Z : 0 \leq a < n\}, \\ \Z_n^* &= \{a \in \Z_n : \gcd(a, n) = 1\}. 
\end{align*}
For example, we have 
\begin{align*}
    \Z_{15} &= \{0, 1, 2, 3, 4, 5, 6, 7, 8, 9, 10, 11, 12, 13, 14\}, \\
    \Z_{15}^* &= \{1,2,4,7,8,11,13,14\},
\end{align*}
where we exclude all multiples of $3$ and $5$ in the latter set. 
For $a, b \in \Z_n$, we define the operations 
\begin{align*}
    a \oplus b &= a + b \pmod n, \\
    a \odot b &= a \cdot b \pmod n.
\end{align*}
For instance, given $n = 15$, one can check that $11 \oplus 13 = 9$ and $11 \odot 13 = 8$. 
Most of the algebraic operations we work with in this course are modular operations, so we will simply 
write $+$ and $\cdot$ instead of $\oplus$ and $\odot$ when it is clear from the context 
whether $+$ and $\cdot$ are regular or modular operations. 
Notice that when $a, b \in \Z_n$, we have $a + b \in \Z_n$ and $a \cdot b \in \Z_n$; moreover, 
for $a, b \in \Z_n^*$, we have $a \cdot b \in \Z_n^*$. In fact, elements in $\Z_n$ and 
$\Z_n^*$ satisfy a larger set of properties, known as the group axioms, which we will 
formally define in the following definition.

\begin{defn}
A {\bf group} $(G, *)$ is a non-empty set $G$ together with a binary operation $*$ satisfying the 
following properties:
\begin{enumerate}[(1)]
    \item $G$ is {\bf closed}: For all $a, b \in G$, we have $a * b \in G$. 
    \item $G$ is {\bf associative}: For all $a, b, c \in G$, we have $(a * b) * c = a * (b * c)$. 
    \item $G$ has an {\bf identity}: There exists $e \in G$ such that $a * e = e * a = a$ for all $a \in G$.
    \item Elements in $G$ are {\bf invertible}: For all $a \in G$, there exists $a^{-1} \in G$ 
    such that $a * a^{-1} = a^{-1} * a = e$. 
\end{enumerate}
\end{defn}

As we noted before, $(\Z_n, +)$ is a group with identity $0$ and $(\Z_n^*, \cdot)$ is a group 
with identity $1$. However, $\Z_n$ is not a group with respect to multiplication 
because $0$ is not invertible, and 
$\Z_n^*$ is not a group with respect to addition because there is no identity element $0$. 

Finally, for a finite group $G$, the {\bf order} of $G$ is the number of elements in 
$G$, and is denoted by $|G|$. 

\subsection{GCD and Modular Inverses}
Proving the existence of inverses is interesting, but what about finding the inverse of a particular
element in a group? Finding inverses of elements in $(\Z_n, +)$ is easy: simply take the 
additive inverse $-a \pmod n$ of $a \in \Z$. On the other hand, finding the multiplicative 
inverse of an element in $(\Z_n^*, \cdot)$ takes some more work. In particular, we care about 
efficient algorithms to compute modular inverses because the secret exponent $d$ in the RSA 
encryption scheme (as in Section 1.2) is the multiplicative inverse of $e$ modulo $\phi$, where 
$e$ is the public key exponent and $\phi$ is the secret modulus. Fortunately, modular multiplicative
inverses can be computed using extended Euclidean type algorithms, and we describe one below. 

\begin{algo}[Modular Multiplicative Inverses]~

{\bf Input:} $n \geq 2$ and $a \in \Z_n^*$.

{\bf Output:} $b \in \Z_n^*$ such that $a \cdot b \equiv 1 \pmod n$. \vspace{0.5em}

\begin{algorithmic}[1]
\State Set the initial state $t_a = a$, $t_n = n$, $u = [u_0, u_1] = [1, 0]$, $v = [v_0, v_1] = [0, 1]$. 
\If{$t_n > t_a$} 
    \State Use the division algorithm to write $t_n = qt_a + r$ for some $q \geq 0$ and $0 \leq r < t_a$.
    \State Update the state: $t_n \gets r$, $v \gets v - q \cdot u$. 
\ElsIf{$t_a > t_n$}
    \State Use the division algorithm to write $t_a = qt_n + r$ for some $q \geq 0$ and $0 \leq r < t_n$.
    \State Update the state: $t_a \gets r$, $u \gets u - q \cdot v$. 
\EndIf 
\If{$t_n = 1$}
    \State Set $b = v_0 \pmod n$ and output $b$.
\ElsIf{$t_a = 1$}
    \State Set $b = u_0 \pmod n$ and output $b$.
\Else
    \State Go back to line 2.
\EndIf
\end{algorithmic}
\end{algo}

Notice that the input $(a, n)$ of Algorithm 2.2 assumes that $\gcd(a, n) = 1$ since $a \in \Z_n^*$. 
Therefore, on a more general input $(a, n)$ with $a \in \Z_n$, one first has to check that 
$\gcd(a, n) = 1$ is satisfied before Algorithm 2.2 can be run. We can address this nuisance with 
some minor modifications to the algorithm.

\begin{algo}[GCD and Modular Multiplicative Inverses]~

{\bf Input:} $n \geq 2$ and $0 \neq a \in \Z_n$.

{\bf Output:} $\gcd = \gcd(a, n)$, and if $\gcd = 1$, then $b \in \Z_n^*$ such that $a \cdot b \equiv 1 \pmod n$. \vspace{0.5em}

\begin{algorithmic}[1]
\State Set the initial state $t_a = a$, $t_n = n$, $u = [u_0, u_1] = [1, 0]$, $v = [v_0, v_1] = [0, 1]$. 
\If{$t_n > t_a$} 
    \State Use the division algorithm to write $t_n = qt_a + r$ for some $q \geq 0$ and $0 \leq r < t_a$.
    \State Update the state: $t_n \gets r$, $v \gets v - q \cdot u$. 
\ElsIf{$t_a > t_n$}
    \State Use the division algorithm to write $t_a = qt_n + r$ for some $q \geq 0$ and $0 \leq r < t_n$.
    \State Update the state: $t_a \gets r$, $u \gets u - q \cdot v$. 
\EndIf 
\If{$t_a = 0$}
    \If{$t_n = 1$}
        \State Set $\gcd = 1$, $b = v_0 \pmod n$ and output $(\gcd, b)$.
    \Else
        \State Set $\gcd = t_n$ and output $\gcd$. 
    \EndIf
\ElsIf{$t_n = 0$}
    \If{$t_a = 1$}
        \State Set $\gcd = 1$, $b = u_0 \pmod n$ and output $(\gcd, b)$.
    \Else
        \State Set $\gcd = t_a$ and output $\gcd$. 
    \EndIf
\Else
    \State Go back to line 2.
\EndIf
\end{algorithmic}
\end{algo}

\newpage
\begin{lemma}[Facts about Algorithm 2.3]~
\begin{enumerate}[(1)]
    \item Algorithm 2.3 terminates with $t_a = 0$ or $t_n = 0$. 
    \item $\gcd(t_a, t_n)$ is invariant throughout Algorithm 2.3.
    \item We have $u \cdot (a, n) = u_0a + u_1n = t_a$ and $v \cdot (a, n) = v_0a + v_1n = t_n$
    throughout Algorithm 2.3. 
\end{enumerate}
\end{lemma}
\begin{pf}
We leave the proof as an exercise. 
\end{pf}

\begin{thm}[Correctness of Algorithm 2.3]
Algorithm 2.3 terminates and its output is correct.
\end{thm}
\begin{pf}
By Lemma 2.4, we know that Algorithm 2.3 terminates with $t_a = 0$ or $t_n = 0$. Assume without 
loss of generality that it terminates with $t_a = 0$. In the beginning of the algorithm,
we set $t_a = a$ and $t_n = n$. Since $\gcd(t_a, t_n)$ is invariant throughout the algorithm by 
Lemma 2.4, we see that 
\[ \gcd(a, n) = \gcd(t_a, t_n) = \gcd(0, t_n) = t_n, \]
which shows that the first output $\gcd = t_n$ is correct. Furthermore, if $t_n = 1$ at the 
end of the algorithm, then Lemma 2.4 shows that 
\[ v \cdot (a, n) = v_0a + v_1n = t_n = 1, \]
and this implies that
\[ v_0 \cdot a \equiv 1 \pmod n. \]
Therefore, the second output $b = v_0 \pmod n$ is also correct, completing the proof.
\end{pf}

Algorithm 2.3 can be used to calculate the public key and secret key pair $(e, d)$ in the RSA 
encryption scheme. For a given RSA public modulus $N$ and secret modulus $\phi$, choose 
$e \in \Z_N$ and call Algorithm 2.3 with input $a = e$ and $n = \phi$. Repeat this until 
$\gcd = \gcd(e, \phi) = 1$ and then set the secret exponent as $d = b$, where $\gcd$ and $b$ are the 
outputs given by Algorithm 2.3. 

\subsection{Modular Exponentiation}
Let $e$ and $n\geq 2$ be integers. Let $a \in \Z_n$. Given a base $a$ and an exponent $e$, 
a modular exponential algorithm computes $a^e \in \Z_n$. We have already seen an application of modular
exponential in the RSA encryption scheme; we compute $m^e \in \Z_N$ to encrypt a message $m$, 
and compute $c^d \in \Z_N$ to decrypt a ciphertext $c$, where $e$, $d$, and $N$ are the public 
exponent, secret exponent, and public modulus, respectively. Moreover, we will see later 
that modular exponentiation is the main operation in several other cryptographic schemes. 
These include Diffie-Hellman key exchange, the elliptic curve digital signature algorithm, 
and isogeny-based cryptosystems. This gives us a lot of motivation to design and implement 
efficient modular exponentiation algorithms. 

First, we introduce some notation. We assume that exponents are positive integers unless 
otherwise stated. 
\begin{itemize}
    \item We denote the {\bf binary representation} of an $\ell$-bit integer $e = \sum_{i=0}^{\ell-1} e_i 2^i$
    where each $e_i \in \{0, 1\}$ and $e_{\ell-1} = 1$ by $(e_{\ell-1}\,e_{\ell-2}\,\cdots\,e_0)_2$. 
    Moreover, for an $\ell$-bit integer $e$, we define 
    \[ e[i : j]_2 := (e_i\,e_{i-1}\,\cdots\,e_j)_2 = \sum_{k=j}^i e_k 2^{k-j} \]
    for $0 \leq j \leq i \leq \ell-1$. Note that $e[\ell-1 : 0]_2 = e$ and $e[i : i] = e_i$.
    \item In some of the algorithms, $e$ will be represented in a more general form using a base 
    $b \geq 2$. More specifically, we write $e = \sum_{i=0}^{\ell-1} e_i b^i$ where each 
    $0 \leq e_i < b$ and $e_{\ell-1} \neq 0$. We call this the 
    {\bf $b$-ary representation} of $e$, and denote it by $e = (e_{\ell-1}\,e_{\ell-2}\,\cdots\,e_0)_b$.
    Note that the binary representation of $e$ is obtained by setting $b = 2$. 
    \item One may further relax the condition $0 \leq e_i < b$ and allow for a more general 
    digit set $D$ for the $e_i$. To be more specific, if $e$ can be written as 
    \[ e = \sum_{i=0}^{\ell-1} e_i b^i \]
    where each $e_i \in D$ for a digit set $D$, then we will still denote $e = (e_{\ell-1}\,e_{\ell-2}\,\cdots\,e_0)_b$ and extend our notation $e[i : j]_2$ to 
    \[ e[i : j]_b = (e_i\,e_{i-1}\,\cdots\,e_j)_b = \sum_{k=j}^i e_k b^{k-j}. \]
    If the base $b$ is clear from the context, we can drop it and simply write $e[i : j]$. 
    \item The complexity of some algorithms will depend on the {\bf weight} of the $b$-ary 
    representation of $e$ (the number of indices $i$ with $e_i \neq 0$), which we will denote by 
    $\w_b(a)$. 
    \item Digit sets may contain negative digits, and it will be convenient to denote a negative 
    digit $-d$ by $\overline{d}$. 
\end{itemize}

\begin{exmp}
The $2^w$-ary representations of $20771$ for $w = 1, 2, 3, 4$ are 
\begin{align*}
    20771 &= (1\,0\,1\,0\,0\,0\,1\,0\,0\,1\,0\,0\,0\,1\,1)_2 \\
    &= (1\,1\,0\,1\,0\,2\,0\,3)_4 \\
    &= (5\,0\,4\,4\,3)_8 \\
    &= (5\,1\,2\,3)_{16}
\end{align*}
A binary representation of $20771$ using the digit set $D = \{0, 1, \overline1, 3, \overline3\}$ 
and a $4$-ary representation of $20771$ using the digit set $D = \{1, \overline1, 3, \overline3\}$ are given by 
\begin{align*}
    20771 &= (1\,0\,0\,\overline3\,0\,0\,0\,1\,0\,0\,1\,0\,0\,0\,0\,3)_2 \\
    &= (1\,1\,1\,\overline3\,1\,\overline1\,\overline3\,\overline1)_4.
\end{align*}
To obtain the binary representation, we notice that 
\[ 20771 = 2^{15} - 3 \cdot 2^{12} + 2^8 + 2^5 + 3 \cdot 2^5. \]
Observe that the last representation does not use any $0$ digit and maximizes the weight 
$\w_4(e)$, whereas the second last representation is sparse with a relatively low weight $\w_2(e) = 5$.
\end{exmp}

The most widely known efficient method to perform exponentiation dates back to 200 BC and is called the 
{\bf square and multiply method}. Brauer generalized the square and multiply method using $b = 2^w$
representations of integers, where a set of elements $T = \{a_i = a^i : 1 \leq i < 2^w\}$ is 
precomputed and stored. We present this method in the following algorithm. The parameter 
$w$ used in the algorithm is also called the {\bf window size} because we iterate through $w$ 
bits at a time and we imagine placing (and moving) a window of size $w$ on the binary 
representation of the exponent. 

\begin{algo}[$2^w$-ary Square and Multiply Method]~

{\bf Input:} $w \in \Z$, $w \geq 1$, $b = 2^w$, $e = (e_{\ell-1}\,e_{\ell-2}\,\cdots\,e_0)_b$, 
$0 \leq e_i < 2^w$, $a \in \Z_n$, $n \in \Z$, $n \geq 2$.

{\bf Output:} $a^e \in \Z_n$. \vspace{0.5em}

\begin{algorithmic}[1]
\State $a_1 \gets a$, $a_i \gets a_{i-1} \cdot a$ for $2 \leq i \leq 2^w$ \Comment{we can ignore this step when $w = 1$}
\State $t \gets 1$
\For{$i = \ell-1$ {\bf to} 0 {\bf by} $-1$}
    \State $t \gets t^{2^w}$ \Comment{this step requires $w$ successive squaring operations}
    \If{$e_i \neq 0$}
        \State $t \gets t \cdot a_{e_i}$ \Comment{the multiply step; if $w = 1$, then $a_{e_i}$ is always $a$}
    \EndIf
\EndFor
\State Output $t$. 
\end{algorithmic}
\end{algo}

\newpage
\begin{thm}[Correctness of Algorithm 2.7]
Algorithm 2.7 terminates and its output is correct.
\end{thm}
\begin{pf}
The proof follows from induction and observing that $e[\ell-1 : 0]_b = 2^w \cdot (e[\ell-1 : 1]_b) + e_0$. 
We leave the details as an exercise. 
\end{pf}

\subsection{Quadratic Residues}
We now introduce quadratic residues from number theory. This will prepare us for our next 
topics on primality testing and other cryptographic constructions such as random number generators 
and public key encryption algorithms. 

Let $p$ be an odd prime. Since the only integer $a \in \Z_p$ that does not satisfy $\gcd(a, p) = 1$ is 
$a = 0$, we have $\Z_p^* = \{1, 2, \dots, p-1\}$ and $|\Z_p^*| = p-1$. 

Recall that $\Z_p^*$ is a group under multiplication. An interesting fact in algebra states that 
there always exists an element $g \in \Z_p^*$ such that for all $c \in \Z_p^*$, there is a unique 
integer $1 \leq k \leq p-1$ such that $c = g^k$. In other words, for such an element $g$, we can write 
\[ \Z_p^* = \{g^k : 1 \leq k \leq p-1\}. \]
More generally, if a (multiplicative) group $G$ has an element such that 
\[ G = \{g^k : 1 \leq k \leq |G|-1\}, \]
then we call $G$ a {\bf cyclic group} generated by $g$, and we write $G = \langle g \rangle$. 
Equivalently, we say that $g$ is a {\bf generator} of $G$. We note that a cyclic group can 
have more than one generator, and that not every group is cyclic. Indeed, $\Z_5^* = \{1, 2, 3, 4\}$ 
and $\Z_8^* = \{1, 3, 5, 7\}$ are both groups of order $4$; we have that $\Z_5^* = \langle 2 
\rangle = \langle 3 \rangle$ is cyclic, whereas $\Z_8^*$ is not since $a^2 = 1$ for all $a \in \Z_8^*$. 

It can be shown that $\Z_n^*$ is a cyclic group if and only if $n$ is one of $2$, $4$, $p^k$, or 
$2p^k$ where $p$ is an odd prime and $k$ is a positive integer. 

For a given positive integer $n$, Euler's totient function, denoted $\phi$, counts the number of positive
integers $1 \leq a \leq n$ such that $\gcd(a, n) = 1$. That is, 
\[ \phi(n) = \#\{a \in \Z : \gcd(a, n) = 1,\, 1 \leq a \leq n\}. \]
By definition, we have $|\Z_n^*| = \phi(n)$, so it is of interest to us to compute $\phi(n)$. 
Notice that we have the following properties:
\begin{enumerate}[(1)]
    \item If $p$ is a prime and $k \geq 1$ is an integer, then $\phi(p^k) = p^k - p^{k-1}$. 
    \item If $m$ and $n$ are positive integers with $\gcd(m, n) = 1$, then 
    $\phi(m \cdot n) = \phi(m) \phi(n)$. 
\end{enumerate}
In particular, for two distinct primes $p$ and $q$, we know that $|\Z_p^*| = \phi(p) = p-1$ 
and $|\Z_q^*| = \phi(q) = q-1$. It follows that if $N = p \cdot q$, then 
\[ |\Z_N^*| = \phi(N) = \phi(p)\phi(q) = (p-1)(q-1). \]
This is exactly why we defined $\phi = (p-1)(q-1)$ in the RSA encryption scheme, where we have been 
practically working with $N$. 

We now recall some important definitions and facts about finite groups. 

\begin{prop}[Properties of Finite Groups]
Let $G$ be a finite group with identity $1$. 
\begin{itemize}
    \item For all $g \in G$, we have $g^{|G|} = 1$. 
    \item The {\bf order} of an element $g \in G$ is the smallest positive integer $s$ such that 
    $g^s = 1$, denoted by $\ord(g)$. 
    \item If $g^a = 1$ for some positive integer $a$, then $\ord(g) \mid a$. In particular, we see that 
    $\ord(g) \mid |G|$. 
    \item If $\ord(g) = s$, then 
    \[ \ord(g^a) = \frac{s}{\gcd(a, s)}. \]
    In particular, we have $\ord(g) = \ord(g^a)$ if and only if $\gcd(a, \ord(g)) = 1$. 
    \item A cyclic group $G = \langle g \rangle$ has $\phi(|G|)$ generators, and the set of 
    generators is given by 
    \[ \{g^a : \gcd(a, |G|) = 1\}. \]
    \item We have $g^a = g^b$ if and only if $a \equiv b \pmod{\ord(g)}$. 
\end{itemize}
\end{prop}

Let $p$ be a prime, and consider the linear equation 
\[ ax + b \equiv 0 \pmod p \]
for $a, b \in \Z_p$ with $a \neq 0$. This equation has a solution $x = -b \cdot a^{-1}$, and one can 
show that this solution is unique. Next, consider the quadratic equation 
\[ ax^2 + bx + c \equiv 0 \pmod p \]
where $a, b, c \in \Z_p$ with $a \neq 0$. Then we can consider two cases: either the equation 
has no solution in $\Z_p$ (for instance, take $p = 3$, $a = c = 1$, and $b = 0$), or 
it has at least one solution. Suppose we are in the second case, and let $s_1$ be a solution. Then 
we can write 
\begin{align*}
    ax^2 + bx + c &= (ax^2 + bx + c) - (as_1^2 + bs_1 + c) \\
    &= a(x^2 - s_1^2) + b(x - s_1) \\
    &= (x - s_1)(ax + as_1 + b). 
\end{align*}
This gives rise to another solution $s_2 = -(s_1 + ba^{-1}) \in \Z_p$ to the same equation. 
Therefore, we can either have $0$, $1$ (when $s_1 = s_2$), or $2$ (when $s_1 \neq s_2$) solutions in 
$\Z_p$. By setting $a = 1$, $b = 0$, and taking a non-zero $c \in \Z_p$, we conclude that 
\[ x^2 \equiv c \pmod p \]
has either no solution, or solution set $\{s, -s\}$ for some $s \in \Z_p^*$. Note that $s$ cannot 
be zero, and for $p$ an odd prime, $s$ and $-s$ are pairwise distinct. This motivates the 
definition of a quadratic residue modulo $p$. 

\begin{defn}[Quadratic Residues and Non-Residues]
Let $p$ be a prime. An element $c \in \Z_p^*$ is said to be a {\bf quadratic residue} modulo $p$ if 
the equation 
\[ x^2 \equiv c \pmod p \]
as a solution in $\Z_p^*$. If $c \in \Z_p^*$ is not a quadratic residue, we call $c$ a 
{\bf quadratic non-residue} modulo $p$. The set of all quadratic residues modulo $p$ is denoted by 
$\QR_p$, and the set of all quadratic non-residues modulo $p$ is denoted by $\QNR_p$. 
\end{defn}

The following theorem gives two characterizations of $\QR_p$. 

\begin{thm}[Characterizations of $\QR_p$]
Let $p$ be an odd prime and let $g$ be a generator of $\Z_p^*$. Let $c = g^k$ for some $1 \leq k 
\leq p-1$.
\begin{enumerate}[(1)]
    \item We have $c \in \QR_p$ if and only if $k$ is even. 
    \item We have $c \in \QR_p$ if and only if $c^{(p-1)/2} = 1$ (Euler's criterion). 
\end{enumerate}
\end{thm}
\begin{pf}
To prove (1), suppose that $c \in \QR_p$. By definition, there exists $x \in \Z_p^*$ such that 
$c = x^2 = g^{2\alpha}$, where the last equality follows because $g$ is a generator of 
$\Z_p^*$. Setting $k = 2\alpha$ proves the forward direction. For the converse, let $k$ be even. 
Then we can write $c = g^k = g^{2\alpha} = x^2$ for $x = g^\alpha$, finishing the proof of (1). 
The proof of (2) follows from (1) and Proposition 2.9, and we leave it as an exercise.
\end{pf}

Euler's criterion in Theorem 2.11 can be used to test whether an element $c \in \Z_p^*$ 
is in $\QR_p$ for an odd prime $p$. \newpage 
\section{Primality Testing}

\subsection{Motivation for Primality Testing}
The key generation algorithm in the RSA encryption scheme requires us to generate two large primes $p$
and $q$ to form the public modulus $N = pq$. How large should these primes be? We have seen that 
some of the currently used public key certificates have a $2048$-bit RSA modulus $N$, which would 
require us to generate two $1024$-bit primes; we want $p$ or $q$ to be of the same size for 
security reasons. 

Suppose that we would like to generate an $\ell$-bit prime $p$. Moreover, suppose that we 
have two magic boxes $\RNG$ and $\IsPrime$. Every time we use $\RNG$, it gives us a random bit, 
or a sequence of random bits. We may have to initiate $\RNG$ with some (short but random) seed. 
Therefore, using $\RNG$ nets us a random $\ell$-bit integer $p$. Next, we provide our random integer $p$
to $\IsPrime$, which tells us $\True$ if $p$ is prime and $\False$ otherwise. We repeat this 
process until the output of $\IsPrime$ is $\True$, as depicted in the following diagram. 

\begin{center}
% https://tikzcd.yichuanshen.de/#N4Igdg9gJgpgziAXAbVABwnAlgFyxMJZARgBoAGAXVJADcBDAGwFcYkQAdDuAM2C54AjCAA9gAJQByAcQC+skLNLpMufIRQAmCtTpNW7NIuUgM2PASIBmHTQYs2iTtz4DhYgJJwACgCcsALYw8sYq5upEACy2eg7sXLzAACq+rApKYWqWKDbEuvYGTgl8AGJMcMGhpqoWGsjkMQWOIIq6MFAA5vBEoDy+EAFIDSA4EEjEGSB9A+M0o0iak9ODiNojY4iRS-0ra-OIVtszm3Mb5EcrAKynQ3b6zVw4MCI4wBXt6ZSyQA
\begin{tikzcd}
{} \arrow[r, "\text{seed}"] & {\fbox{\RNG}} \arrow[r] & p \arrow[r] & {\fbox{\IsPrime}} \arrow[d] \arrow[r] & \sf{True} \\
                            &                           &             & \sf{False} \arrow[llu]                  &          
\end{tikzcd}
\end{center}

Defining the number of primes not exceeding $L$ as $\pi(L)$ and assuming that the primes are 
randomly distributed in any given interval, we would expect to query $\IsPrime$ about 
$2^{\ell-1}/(\pi(2^\ell) - \pi(2^{\ell-1}))$ times before we have an affirmative answer on the primality
of $p$. The Prime Number Theorem tells us that $\pi(L) \approx L/\ln L$, so we would expect to 
try approximately $\ln 2^\ell$ random integers before obtaining a prime $p$. In particular, 
this means that we would run $710$ primality tests before generating a $1024$-bit prime. We can 
reduce this number in half by trying only the odd numbers, but it is still critical that we find 
efficient implementations of $\RNG$ and $\IsPrime$. This week, we will focus on $\IsPrime$. 

\subsection{A Basic Primality Test}
Let $n$ be a positive integer. If $n$ is prime, then it only has the factors $1$ and $n$, which we call 
the trivial factors of $n$. On the other hand, if $n$ is composite, then it has factors 
$n_1 \leq n_2$ with $n_1 \leq \sqrt n$. In other words, a positive integer $n \geq 2$ is prime if and 
only if it is not divisible by any integer between $2$ and $\sqrt n$. 

This observation gives us a basic primality testing algorithm. For an integer $n \geq 2$ as input, 
compute $n \pmod k$ for all integers $2 \leq k \leq \sqrt n$. If no such $k$ exists or 
$n \pmod k$ is non-zero for all $k$, then output $\True$; otherwise, there is some $k$ such that 
$n \pmod k$ is zero, in which case we output $\False$. 

This algorithm can be very quick to identify $n$ as a composite number if $n$ has a small divisor. 
However, what if our input $n$ is composite with only two prime divisors $p \approx q$? 
What if $n$ is prime (which is exactly what we are trying to determine)? In such cases, 
we would have to compute $n \pmod k$ approximately $\sqrt n$ times. To generate a 
$1024$-bit prime, we would need to run the Euclidean division algorithm about $2^{512}$ times. 
If we run this algorithm on a $4.7$ GHz computer and assume that Euclidean division can be 
computed in $1$ clock cycle, then we would expect to wait $2^{512}/(4.7 \times 10^9)$ seconds, 
or about $10^{137}$ years. This is clearly infeasible for our purposes. Fortunately, 
more efficient primality testing algorithms exist. 

\subsection{Fermat Primality Test}
Let $n \geq 2$ be a positive integer. If we can find an integer $1 \leq a \leq n-1$ such that 
$\gcd(a, n) > 1$, then we are guaranteed that $n$ is composite. However, what if the integer 
$1 \leq a \leq n-1$ satisfies $\gcd(a, n) = 1$? Then we either have $a^{n-1} \not\equiv 1 \pmod n$ 
or $a^{n-1} \equiv 1 \pmod n$. In the first case, we can conclude that $n$ is composite because 
if $n$ were prime, then $a^{n-1} \equiv 1 \pmod n$ for all $1 \leq a \leq n-1$ by Fermat's Little 
Theorem. However, in the second case, we cannot deduce whether $n$ is composite or prime. 
For example, we have $2^{340} \equiv 1 \pmod{341}$, but $n = 341 = 11 \cdot 31$ is not prime. 
This discussion motivates the definition of a Fermat witness and Fermat liar for the compositeness of $n$.

\begin{defn}
Let $n \geq 2$ and $1 \leq a \leq n-1$ be positive integers.
\begin{itemize}
    \item The integer $a$ is said to be a {\bf Fermat witness} for the compositeness of $n$ if it satisfies
    $\gcd(a, n) = 1$ and $a^{n-1} \not\equiv 1 \pmod n$. 
    \item The integer $a$ is said to be a {\bf Fermat liar} for the compositeness of $n$ if it satisfies 
    $\gcd(a, n) = 1$ and $a^{n-1} \equiv 1 \pmod n$. 
\end{itemize}
\end{defn}

We saw that $a = 2$ is a Fermat liar for the compositeness of $n = 341$ since $\gcd(2, 341) = 1$
and $2^{340} \equiv 1 \pmod{341}$. On the other hand, $a = 3$ is a Fermat witness for the compositeness
of $n = 341$ since $\gcd(3, 341) = 1$ and $3^{340} \equiv 56 \not\equiv 1 \pmod{341}$. 

We are now ready to present the Fermat primality test, where we hope that the number of Fermat liars 
is relatively low so that we can quickly find a Fermat witness for a composite integer $n$. 
If we cannot find a Fermat witness for the compositeness of $n$ after $r$ iterations, 
then we will conclude that $n$ is {\it probably} prime. We cannot say for sure that $n$ is prime 
since it is possible that $n$ is composite but no Fermat witness is found within $r$ iterations.

\begin{algo}[Fermat Primality Test]~

{\bf Input:} A positive integer $n \geq 2$ and a positive integer $r$. 

{\bf Output:} One of ``$n$ is composite'' or ``$n$ is probably prime''; the success probability is controlled by the input parameter $r$, the number of iterations. \vspace{0.5em}

\begin{algorithmic}[1]
\State $\counter \gets 1$. 
\State Pick a random integer $1 < a < n-1$. 
\State Compute $\gcd = \gcd(a, n)$. 
\If{$\gcd > 1$}
    \State Output ``$n$ is composite'' and exit.
\Else
    \State Compute $a^{n-1} \in \Z_n$. 
    \If{$a^{n-1} \neq 1$} \Comment{$a$ is a Fermat witness}
        \State Output ``$n$ is composite'' and exit. 
    \Else \Comment{$a$ is a Fermat liar when $n$ is composite}
        \State $\counter \gets \counter + 1$. 
        \If{$\counter = r$} \Comment{after $r$ iterations, bet that $n$ is prime}
            \State Output ``$n$ is probably prime'' and exit.
        \Else 
            \State Go back to line 2. 
        \EndIf 
    \EndIf
\EndIf
\end{algorithmic}
\end{algo}

It is clear that if the input $n$ to the Fermat primality test is prime, then the output 
will always be ``$n$ is probably prime'', as expected. Moreover, if the Fermat primality test 
outputs ``$n$ is composite'', then $n$ is always composite. However, as we noted above, the Fermat 
primality test can output ``$n$ is probably prime'' even when $n$ is composite. Ideally, we want 
to ensure that when the input $n$ is composite, then the probability that the algorithm will 
output ``$n$ is probably prime'' is small. This probably heavily depends on the density of Fermat 
witnesses for a composite number $n$. The good news is that the existence of a Fermat witness for a 
composite $n$ implies that at least half of the elements in $\Z_n^*$ are Fermat witnesses, as we 
shall see in the following theorem.

\begin{thm}
Let $n$ be a positive integer for which there is at least one Fermat witness. Then at least 
half of the elements in $\Z_n^* = \{a \in \Z_n : \gcd(a, n) = 1\}$ are Fermat witnesses. 
\end{thm}
\begin{pf}
Let $a \in \Z_n^*$ be a Fermat witness for the compositeness of $n$. If there is no Fermat liar in 
$\Z_n^*$, then the result holds trivially. Now, let $b_i \in \Z_n^*$ be pairwise distinct 
Fermat liars for $1 \leq i \leq s$. It suffices to show that $ab_i$ are pairwise distinct 
Fermat witnesses. First, note that $ab_i \in \Z_n^*$ since $\Z_n^*$ is a multiplicative group. 
Moreover, the $ab_i$ are pairwise distinct because $ab_i \equiv ab_j \pmod n$ would imply that 
$b_i \equiv b_j \pmod n$ by multiplying both sides by $a^{-1} \in \Z_n$. Finally, we have 
\[ (ab_i)^{n-1} \equiv a^{n-1} b_i^{n-1} \equiv a^{n-1} \not\equiv 1 \pmod n, \]
so $ab_i$ is a Fermat witness, finishing the proof. 
\end{pf}

The above theorem implies that if there is at least one Fermat witness for a composite $n$, then 
the Fermat primality test would fail with probability at most $1/2^r$. This is because for 
composite $n$, the probability that $\gcd(a, n) = 1$ and that $a$ is not a Fermat witness is at 
most $1/2$. 

Unfortunately, there are composite numbers $n$ with no Fermat witnesses. 
In other words, $a^{n-1} \equiv 1 \pmod n$ holds for all $a \in \Z_n^*$. Such an integer $n$ 
is known as a \href{https://en.wikipedia.org/wiki/Carmichael_number}{Carmichael number}, with the smallest one being $561$. There are 
infinitely many Carmichael numbers, and the number of Carmichael numbers less than $n$ is 
estimated to be about $n^{2/7}$ for sufficiently large $n$. In particular, the Fermat primality 
test will fail to output the correct answer unless we happen to choose $1 < a < n-1$ with 
$\gcd(a, n) > 1$ in one of the iterations. 

\subsection{Solovay-Strassen Primality Test}
The Fermat primality test utilizes the key equality $a^{n-1} \equiv \pmod n$ which holds for all $a \in 
\Z_n^*$ when $n$ is prime. We observed a similar property, namely Euler's criterion in Proposition 2.11,
which states that if $n$ is an odd prime, then $a \in \QR_n$ if and only if $a^{(n-1)/2} = 1$. 
As one might guess, the Solovay-Strassen primality test will follow a similar approach to the 
Fermat primality test; repetitive computations of $a^{(n-2)} = 1$ for randomly chosen 
$a \in \Z_n^*$ will help determine if $n$ is prime or composite. Before describing the 
Solovay-Strassen primality test, we will first introduce some more definitions and algorithms 
building on our previous discussion of quadratic residues. 

\begin{defn}[Legendre symbol]
Let $p$ be an odd prime and let $a \geq 0$ be an integer. The {\bf Legendre symbol} 
$(\frac ap)$ is defined as 
\[ \left( \frac ap \right) = \begin{cases} 0 & \text{if $a \equiv 0 \pmod p$,} \\ 1 & \text{if $a \in \QR_p$,} \\ -1 & \text{if $a \in \QNR_p$.} \end{cases} \]
\end{defn}

\begin{thm}
If $p$ is an odd prime, then for all integers $a \geq 0$, we have 
\[ \left( \frac ap \right) \equiv a^{(p-1)/2} \pmod p. \]
\end{thm}
\begin{pf}
If $p$ divides $a$, then $a \equiv 0 \pmod p$, so $a^{(p-1)/2} \equiv 0 \pmod p$ and 
$(\frac ap) = 0$ by the definition of the Legendre symbol. Next, assume that $\gcd(a, p) = 1$. 
If $a \in \QR_p$, then $(\frac ap) = 1$ by definition, and $a^{(p-1)/2} \equiv 1 \pmod p$ 
by Euler's criterion. Suppose now that $a \in \QNR_p$ so that $(\frac ap) = -1$. By Proposition 
2.9, we can write $a^{p-1} \equiv 1 \pmod p$, so $a^{(p-1)/2}$ is a solution to $x^2 \equiv 1 
\pmod p$. The only two solutions to this equation are $\pm1$ as we discussed in Section 2.4. 
Since $a \in \QNR_p$, it follows from Euler's criterion that $a^{(p-1)/2} \equiv -1 \pmod p$,
completing the proof.
\end{pf}

Note that the above theorem gives us an efficient algorithm to compute the Legendre symbol. 
Now, we generalize the Legendre symbol to the Jacobi symbol $\left(\frac an\right)$ for all odd positive integers 
$n$, and present efficient algorithms to compute it. This is critical for the Solovay-Strassen 
primality test as we will declare $n$ to be composite if we can find $a \in \Z_n$ such that 
$\gcd(a, n) > 1$ or $\left(\frac an\right) \not\equiv a^{(n-1)/2} \pmod n$. 

\begin{defn}[Jacobi symbol]
Let $n$ be an odd positive integer with unique prime factorization given by 
\[ n = \prod_{i=1}^k p_i^{e_i} \]
where $p_i$ is a prime and $e_i \geq 1$ is an integer for all $1 \leq i \leq k$. 
For an integer $a \geq 0$, the {\bf Jacobi symbol} $\left(\frac an\right)$ is defined by 
\[ \left( \frac an \right) = \prod_{i=1}^k \left( \frac a{p_i} \right)^{\!e_i}, \]
where $(\frac a{p_i})$ denotes the Legendre symbol as defined above. 
\end{defn}

We noted earlier that we will declare $n$ to be composite if we can find $a \in \Z_n$ such that 
$\left(\frac an\right) \not\equiv a^{(n-1)/2} \pmod n$ in the Solovay-Strassen primality test. We already 
know how to compute $a^{(n-1)/2} \pmod p$ and $(\frac ap)$ efficiently for a prime $p$, and 
the definition of the Jacobi symbol offers an efficient way to compute $\left(\frac an\right)$. 
However, this method is not useful to us because we are trying to determine if $n$ is prime, 
whence we do not know the prime factorization of $n$. Computation number theory comes in 
handy here, and some facts about the Jacobi symbol will help us to compute the Jacobi symbol
$\left(\frac an\right)$ efficiently without needing to factor $n$.

\begin{prop}
Let $m$ and $n$ be odd positive integers. For all integers $a, b \geq 0$, we have the following properties:
\begin{enumerate}[(1)]
    \item $\left(\frac an\right) = 0$ if and only if $\gcd(a, n) > 1$. 
    \item $\left(\frac {ab}n\right) = \left(\frac an\right)\left(\frac bn\right)$. 
    \item If $a \equiv b \pmod n$, then $\left(\frac an\right) = \left(\frac bn\right)$. 
    \item $\left(\frac1n\right) = 1$. 
    \item $\left(\frac2n\right) = (-1)^{(n^2-1)/8} = \begin{cases} 1 & \text{if } n \equiv \pm1 \pmod 8, \\ -1 & \text{if } n \equiv \pm3 \pmod 8. \end{cases}$
    \item $\left(\frac mn\right) = \left(\frac nm\right) (-1)^{((m-1)/2)((n-1)/2)} = \begin{cases} -\left(\frac nm\right) & \text{if } m\equiv n \equiv 3 \pmod 4, \\ \left(\frac nm\right) & \text{otherwise.} \end{cases}$
\end{enumerate}
\end{prop}

\begin{exmp}
We can use the above facts to compute $\left(\frac{123}{5472940991761}\right)$. Indeed, we have 
\begin{align*}
    \left( \frac{123}{5472940991761} \right) &= 
    \left( \frac{5472940991761}{123} \right) = \left( \frac{70}{123} \right) 
    = \left( \frac{2}{123} \right) \left( \frac{35}{123} \right) = -\left( \frac{35}{123} \right) \\
    &= \left( \frac{123}{35} \right) = \left( \frac{18}{35} \right) = \left( \frac{2}{35} \right)
    \left( \frac{9}{35} \right) = -\left( \frac{9}{35} \right) \\
    &= -\left( \frac{35}{9} \right) = -\left( \frac{8}{9} \right) = -\left( \frac{2}{9} \right)^{\!3}
    \left( \frac{1}{9} \right) = -\left( \frac{1}{9} \right) = -1. 
\end{align*}
\end{exmp}

We now have all the computational tools we need for presenting the Solovay-Strassen primality test. 
Due to Theorem 3.5, we hope to quickly find an integer $a$ such that $\left(\frac an \right) 
\not\equiv a^{(n-1)/2} \pmod n$ for composite $n$. This is because if $n$ were prime, then
for all $a \geq 0$, we would have $\left( \frac an \right) \equiv a^{(n-1)/2} \pmod n$.
We call such an integer $a$ an Euler witness for 
the compositeness of $n$, as it certifies that $n$ is composite. If we do not find an Euler 
witness for the compositeness of $n$ after $r$ iterations, then we will conclude that 
$n$ is {\it probably} prime, as with the Fermat primality test. 

\newpage 
\begin{algo}[Solovay-Strassen Primality Test]~

{\bf Input:} A positive odd integer $n \geq 3$ and a positive integer $r$. 

{\bf Output:} One of ``$n$ is composite'' or ``$n$ is probably prime''; the success probability is controlled by the input parameter $r$, the number of iterations. \vspace{0.5em}

\begin{algorithmic}[1]
\State $\counter \gets 1$. 
\State Pick a random integer $1 < a < n-1$. 
\State Compute $\gcd = \gcd(a, n)$. 
\If{$\gcd > 1$}
    \State Output ``$n$ is composite'' and exit.
\Else
    \State Compute $\left( \frac an \right)$ and $a^{(n-1)/2} \in \Z_n$. 
    \If{$\left(\frac an \right) \not\equiv a^{(n-1)/2} \neq 1 \pmod n$} \Comment{$a$ is an Euler witness}
        \State Output ``$n$ is composite'' and exit. 
    \Else \Comment{$a$ is an Euler liar when $n$ is composite}
        \State $\counter \gets \counter + 1$. 
        \If{$\counter = r$} \Comment{after $r$ iterations, bet that $n$ is prime}
            \State Output ``$n$ is probably prime'' and exit.
        \Else 
            \State Go back to line 2. 
        \EndIf 
    \EndIf
\EndIf
\end{algorithmic}
\end{algo}

\begin{defn}
Let $n \geq 3$ be an odd positive composite number. An integer $1 \leq a \leq n-1$ is called an 
{\bf Euler-Jacobi witness} for the compositeness of $n$ if it satisfies $\gcd(a, n) = 1$ and 
$\left( \frac an \right) \not\equiv a^{(n-1)/2} \pmod n$. Otherwise, $a$ is called an 
{\bf Euler-Jacobi liar} if it satisfies $\gcd(a, n) = 1$ and $\left( \frac an \right) 
\equiv a^{(n-1)/2} \pmod n$. 
\end{defn}

\begin{thm}[Density of Euler-Jacobi witnesses]
Let $n$ be an odd positive composite integer. Then at least half of the elements in $\Z_n^* 
= \{a \in \Z_n : \gcd(a, n) = 1\}$ are Euler-Jacobi witnesses.
\end{thm}
\begin{pf}
We leave the proof as an exercise. Show that the Euler-Jacobi liars
form a proper subgroup of $\Z_n^*$, and use the fact that the order of a subgroup divides the order 
of the group. 
\end{pf}

The above theorem implies that the Solovay-Strassen primality test will fail to output the 
correct result with probability at most $1/2^r$ since the probability that $a \in \Z_n^*$ 
is not an Euler-Jacobi witness per iteration is at most $1/2$ for composite $n$. 

We note that Theorem 3.11 is stronger than its analogous statement in Theorem 3.3 in the sense that
the number of Euler-Jacobi witnesses is bounded by $|\Z_n^*|/2$ without needing to assume the 
existence of an Euler-Jacobi witness. 

\subsection{Miller-Rabin Primality Test}

In this section, we present the Miller-Rabin primality test, where we introduce the notions of a Miller-Rabin witness and a Miller-Rabin liar. The Miller-Rabin primality test is more advantageous than the Solovay-Strassen primality test because the number of Miller-Rabin liars in $\Z_n^*$ is bounded by $(n-1)/4$, and the set of Euler-Jacobi witnesses is a subset of 
the Miller-Rabin witnesses. In particular, the wrong output is provided with probability at most 
$1/4^r$ in the case where the input $n$ is composite, where $r$ is the number of iterations. 
The Miller-Rabin primality test is widely used in practice as it is one of the simplest and 
fastest tests, while being more accurate than the Fermat and Solovay-Strassen primality tests. 

\begin{thm}
Let $p$ be an odd prime. Let $s, d \in \Z_n^*$ be elements such that $d$ is odd and 
$p - 1 = 2^s d$ with $s \geq 1$. For every $a \in \Z_p^*$, we either have 
\begin{enumerate}[(i)]
    \item $a^d \equiv 1 \pmod p$, or 
    \item $a^{2^td} \equiv -1 \pmod p$ for some $0 \leq t \leq s-1$. 
\end{enumerate}
\end{thm}
\begin{pf}
By Fermat's little theorem, we have $a^{p-1} \equiv a^{2^sd} \equiv 1 \pmod p$. Therefore, 
$a^{2^{s-1}d}$ satisfies the equation $x^2 \equiv 1 \pmod p$, which has exactly two roots 
$\pm1 \pmod p$. In other words, we either have $a^{2^{s-1}d} \equiv -1 \pmod p$ or 
$a^{2^{s-1}d} \equiv 1 \pmod p$. In the first case, we are done as (ii) holds. 
In the second case, we have $a^{2^{s-1}d} \equiv 1 \pmod p$, and if $s = 1$, then we are done as 
(i) holds. If $s > 1$, we can repeat the same process and either obtain $a^{2^td} \equiv -1 \pmod p$
for some $0 \leq t \leq s-1$ along the way, or we would have to have $a^d \equiv 1 \pmod p$, 
which completes the proof. 
\end{pf}

\begin{defn}
Let $n$ be an odd positive composite integer such that $n-1=2^sd$ for some positive integer $s$ and an odd integer $d$. An element $a\in \Z_n^*$ is called a {\bf Miller-Rabin witness} for the compositeness of $n$ if it satisfies
\begin{enumerate}[(i)]
    \item $a^d \not\equiv 1 \pmod n$, and 
    \item $a^{2^td} \not\equiv -1 \pmod n$ for all $0 \leq t \leq s-1$. 
\end{enumerate}
Otherwise, $a$ is called a {\bf Miller-Rabin liar} for the compositeness of $n$. 
\end{defn}

We now present the Miller-Rabin primality test, where we hope to quickly find a Miller-Rabin 
witness for a composite input $n$. 

\begin{algo}[Miller-Rabin Primality Test]~

{\bf Input:} A positive odd integer $n \geq 3$ and a positive integer $r$. 

{\bf Output:} One of ``$n$ is composite'' or ``$n$ is probably prime''; the success probability is controlled by the input parameter $r$, the number of iterations. \vspace{0.5em}

\begin{algorithmic}[1]
\For{$\counter = 0$ \textbf{to} $r - 1$} \Comment{at most $r$ iterations}
    \State Pick a random integer $1 < a < n-1$. 
    \State Compute $\gcd = \gcd(a, n)$. 
    \If{$\gcd > 1$}
        \State Output ``$n$ is composite'' and exit.
    \Else
        \State Write $n - 1 = 2^s d$ such that $s \geq 1$ and $d$ is odd.
        \State Compute $b \equiv a^d \pmod n$. 
        \If{$b \not\equiv 1 \pmod n$ {\bf and} $b^{2^t} \not\equiv -1 \pmod n$ {\bf for all} $0 \leq t \leq s-1$}
            \State Output ``$n$ is composite'' and exit. 
        \EndIf 
    \EndIf
\EndFor 
\State Output ``$n$ is probably prime''. 
\end{algorithmic}
\end{algo}

\end{document}