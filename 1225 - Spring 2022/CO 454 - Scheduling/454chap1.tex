\section{Introduction to Scheduling}\label{sec:1}

\subsection{Examples of Scheduling Problems}\label{subsec:1.1}
To begin, we'll first introduce some examples of scheduling problems. 

\begin{exmp}{exmp:1.1}
    Suppose there are $n$ students who need to consult an advisor AI machine 
    $M$ about their weekly timetables at the start of the semester. The 
    amount of time needed by $M$ to advise student $j$, where 
    $j \in \{1, \dots, n\}$, is denoted by $p_j$. 

    If the students meet with $M$ in the order $1, 2, \dots, n$, then student 
    $1$ completes their meeting with $M$ at time $C_1 = p_1$, student $2$ 
    completes their meeting with $M$ at time $C_2 = p_1 + p_2$, and in general, 
    student $j$ completes their meeting with $M$ at time $C_j = p_1 
    + \cdots + p_j$. We call $C_j$ the {\bf completion time} of student $j$. 

    Now, suppose that there are $n = 3$ students. We can associate each 
    student with a job. Assume that the processing times are 
    $p_1 = 10$, $p_2 = 5$, and $p_3 = 2$. Then for the schedule $1, 2, 3$, 
    the completion times are $C_1 = 10$, $C_2 = 15$, and $C_3 = 17$, for 
    an average completion time of $\frac{10+15+17}{3} = 14$. On the 
    other hand, the ordering $2, 3, 1$ has average completion time 
    $\frac{5+7+17}{3} = \frac{29}{3}$. 

    Our objective in this case is to minimize the average completion time 
    $\frac1n \sum_{j=1}^n C_j$. Notice that this is equivalent to 
    just minimizing the sum of the completion times $\sum_{j=1}^n C_j$. 

    We pose scheduling problems as a triplet $(\alpha \mid \beta \mid \gamma)$, 
    as we will detail in Section \ref{subsec:1.2}. This example can be denoted by 
    $(1\;\|\;\sum C_j)$.
\end{exmp}

\begin{exmp}{exmp:1.2}
    Alice is preparing to write the graduation exams at the KW School of 
    Magic (KWSM). The exam is based on $n$ books $B_1, \dots, B_n$ that 
    can be borrowed from the library of KWSM, and these books are not 
    available anywhere else. Alice estimates that she needs $p_j$ days of 
    preparation for the book $B_j$, but unfortunately, there is a due 
    date $d_j$ for returning $B_j$ to the library. The library charges 
    a late fee of \$$1$ per day for each overdue book. The goal is to 
    find a sequence for returning the books (after completing the 
    preparation for each book) that minimizes her late fees. 

    Suppose that Alice picks the sequence $B_1, \dots, B_n$ for 
    returning the books. For this particular sequence, we let 
    $T_j$ denote the late fees for $B_j$, and let $C_j$ denote the day 
    in which Alice completes her studies from $B_j$. Notice that 
    $T_j = \max(0, C_j - d_j)$, so $T_j = 0$ if Alice completes $B_j$ 
    by day $d_j$, and $T_j = C_j - d_j$ otherwise. 

    More concretely, suppose that there are $n = 4$ books with the 
    following preparation times and due dates. 
    \begin{align*}
        \begin{array}{c|cccc}
            j   & 1  & 2  & 3 & 4  \\ \hline
            p_j & 4  & 6  & 8 & 12 \\
            d_j & 10 & 12 & 9 & 15 \\
        \end{array}
    \end{align*}
    For the sequence $B_1, B_2, B_3, B_4$, we find that $T_1 = 0$, 
    $T_2 = 0$, $T_3 = 18 - 9 = 9$, and $T_4 = 30 - 15 = 15$, which gives 
    $\sum T_j = 24$. On the other hand, Alice could return the books 
    in the sequence $B_3, B_1, B_2, B_4$, and we can see that 
    $T_3 = 0$, $T_1 = 12 - 10 = 2$, $T_2 = 18 - 12 = 6$, and $T_4
    = 30 - 15 = 15$. In this case, we obtain $\sum T_j = 23$. 

    For the triplet notation $(\alpha \mid \beta \mid \gamma)$, we can 
    denote this problem by $(1\;\|\;\sum T_j)$. 
\end{exmp}

\begin{exmp}{exmp:1.3}
    A bus containing $n$ UW students has arrived at the entry point of 
    Michitania. There is a sequence of three automatic checks for each 
    visitor, which are 
    \begin{itemize}
        \item a passport scan ($M_1$),
        \item a temperature scan ($M_2$), 
        \item a facial scan and photo ($M_3$).
    \end{itemize}
    There are three well-separated machines $M_1, M_2, M_3$ located in a 
    broad lane, and machine $M_i$ applies the $i$-th scan. Each student 
    $S_j$ who gets off the bus is required to visit the machines in the 
    sequence $M_1, M_2, M_3$. Viewing each student $S_j$ as a job $j$, 
    observe that $j$ consists of the three operations $(1, j)$, $(2, j)$, 
    and $(3, j)$, with a chain of precedence constraints among the 
    operations given by $(1, j) \to (2, j) \to (3, j)$. In particular, 
    the operation $(2, j)$ cannot start until $(1, j)$ is completed, 
    and $(3, j)$ cannot start until $(2, j)$ is completed. 

    We denote the time required for machine $M_i$ to process student 
    $S_j$ by $p_{ij}$. 

    The goal is to find a schedule such that the checks for all students 
    are completed as soon as possible. Consider a fixed schedule which 
    has the same sequence of students $S_1, S_2, \dots, S_n$ for all 
    three machines. Let $C_{ij}$ denote the time when the scan of $S_j$ 
    is completed on $M_i$. Let $C_{\max}$ denote the maximum completion 
    time of the students on $M_3$; that is, $C_{\max} = 
    \max_{j \in \{1, \dots, n\}} C_{3j}$. We often refer to 
    $C_{\max}$ as the {\bf makespan} of the schedule. The goal is to 
    find a schedule which minimizes $C_{\max}$ among all feasible schedules.
    
    For instance, suppose that there are $n = 4$ students with the 
    following processing times. 
    \begin{align*}
        \begin{array}{c|cccc}
            & j=1 & j=2 & j=3 & j=4 \\ \hline
        i=1 & 4   & 6   & 8   & 12  \\
        i=2 & 10  & 9   & 4   & 6   \\
        i=3 & 11  & 7   & 5   & 3   \\
        \end{array}
    \end{align*}
    Using the schedule $S_1, S_2, S_3, S_4$ for all three machines, we 
    can determine that $C_{\max} = \max(C_{24}, C_{33}) + p_{34} = 
    \max(36, 37) + 3 = 40$. We can visualize these values via a Gantt chart, 
    like below. 
    \begin{verbatim}
        M1 |-S1-|--S2--|---S3---|-----S4-----| 
           0    4      10       18           30 

        M2      |----S1----|---S2---|-S3-|idle|--S4--|
                4          14       23   26   30     36

        M3                 |----S1----|---S2---|--S3--|-S4-|
                           14         25       32     37   40 
    \end{verbatim} 
    \vspace{-1em}
    Notice that $M_2$ had to idle because $M_1$ did not finish processing 
    $S_4$ yet. 
    
    Using the triplet notation $(\alpha \mid \beta \mid \gamma)$, 
    this problem is denoted by $(F3\;\|\;C_{\max})$, where $3$ denotes 
    the number of machines and $F$ refers to a ``flow shop''.

    Note that this problem does not require the students to visit $M_1, 
    M_2, M_3$ in the same sequence. For example, the students could 
    visit $M_1$ in the sequence $S_1, S_2, S_3, S_4, \dots, S_n$, while visiting 
    $M_2$ in the sequence $S_2, S_1, S_4, S_3, \dots, S_n$. It may seem 
    ``obvious'' that there exists an optimal schedule such that the jobs 
    visit each of the machines in the same sequence; this statement is 
    in fact false when there are at least $4$ machines, but is true when 
    there are at most $3$ machines. 
\end{exmp}

\subsection{Notation and Framework}\label{subsec:1.2}
The examples above are examples of scheduling problems and illustrate the 
issue of allocating limited resources over time in order to optimize an 
objective function. We remark here that different objectives can lead to 
different solutions and so a ``universally best'' schedule may not exist. 
It should also be clear from the above examples that scheduling problems
appear in a wide range of fields, and ideally, we should have a unified theory 
for studying them. We now describe a general framework and notation that 
captures most (but not all) scheduling problems.

Any scheduling problem is associated with a finite set of tasks or jobs and 
a finite set of resources or machines. The set of jobs is denoted by $J$, 
and we use $n$ to denote $|J|$; moreover, we write $J = \{1, 2, \dots, n\}$. 
Similarly, the set of machines is denoted by $M$, and we use $m$ to denote $|M|$.

At any point in time, a single machine can process at most one job. 

Most scheduling problems can be described with a triplet $(\alpha \mid 
\beta \mid \gamma)$. The first term $\alpha$ is the {\bf machine environment}
and contains a single entry. This field describes the resources that are 
available for the completion of various tasks. The second term $\beta$ 
denotes the various constraints on the machines and the jobs that must be 
respected by the schedule. The third term $\gamma$ is the objective function 
for the scheduling problem to be minimized; a particular feasible schedule 
is optimal if it has the smallest value for $\gamma$ among all feasible 
schedules. 

Each job $j \in J$ may have one or more of the following pieces of data 
associated with it. 

\begin{itemize}

    \item {\bf Processing Time} ($p_{ij}$) The time taken by machine $i$ 
    to process job $j$ is denoted by $p_{ij}$. In many scheduling problems, 
    the processing time of job $j$ is independent of the machine, and in 
    such cases the processing time (of job $j$ on any machine) is denoted
    simply by $p_j$.

    \item {\bf Release Time} ($r_j$) In several scheduling problems, a job 
    $j$ is only available for processing after time $r_j$. This is called 
    the release time of the job.

    \item {\bf Due Date} ($d_j$) This is the planned time when a job $j$ 
    should be completed. In several scheduling problems, a job is allowed 
    to complete after its due date, but such jobs incur a penalty for the 
    violation.

    \item {\bf Weight} ($w_j$) The weight of a job $j$ denotes the relative 
    worth of $j$ with respect to the other jobs. Usually, $w_j$ is a 
    coefficient in the objective function $\gamma$, for instance 
    $\sum w_j C_j$. As we will see later, introducing weights can often lead 
    to added complexity in the scheduling problem.

\end{itemize}

{\bf Machine Environment} ($\alpha$) The possible machine environments 
in our course are as follows.

\begin{itemize}

    \item {\bf Single Machine Environment} ($\alpha = 1$) In this case, we 
    have only one machine. Although this appears to be a rather special case, 
    the study of single-machine problems leads to many techniques useful for 
    more general cases.

    \item {\bf Identical Parallel Machines} ($\alpha = P$) In this case, we 
    have $m$ identical machines and any job can run on any machine. Each job 
    has the same processing time on each machine. When the number of machines 
    is constant, the number of machines is appended after the letter $P$.
    For example, if there are $m = 2$ machines, then we write $\alpha = P2$.

    \item {\bf Uniform Speed Parallel Machines} ($\alpha = Q$) In this case, 
    we have $m$ machines and any job can run on any machine. Each machine 
    $i$ has a speed $s_i$. The time taken to process job $j$ on machine $i$ 
    is then $p_j/s_i$.

    \item {\bf Unrelated Parallel Machines} ($\alpha = R$) In this case, we 
    have $m$ machines and any job can run on any machine. However, each 
    job $j$ takes time $p_{ij}$ on machine $i$, and the $p_{ij}$'s are 
    unrelated to each other. For instance, machine $i$ could have 
    $p_{ij} > p_{ij'}$, but machine $\ell$ could have $p_{\ell j} < 
    p_{\ell j'}$. 

    \item {\bf Open Shop} ($\alpha = O$) The following three machine 
    environments fall in the shop scheduling framework. In this framework, 
    each job $j$ consists of $m$ operations and a job is said to be completed 
    if and only if all the operations of the job are completed. Furthermore, 
    each operation takes place on a dedicated machine. Thus, each job
    needs to visit each machine before completion. Some of the operations may 
    have zero processing times. 
    
    In the open shop environment, the jobs can 
    visit the $m$ machines in any order. There are no restrictions with regard 
    to the routing of each job; that is, the ``scheduler'' is allowed to 
    determine a route for each job and different jobs may have different routes.

    \item {\bf Job Shop} ($\alpha = J$) In a job shop, each job comes with a 
    specified order in which the $m$ operations of the job are processed by 
    the $m$ machines. In other words, each job has a specified routing 
    (a sequence for the $m$ machines), and it follows this routing to visit 
    the $m$ machines.

    \item {\bf Flow Shop} ($\alpha = F$) In a flow shop, each job visits the 
    $m$ machines in the same fixed order, which is assumed to be 
    $\{1, 2, \dots, m\}$; that is, all jobs have the same routing.
    
    We can view a flow shop in a different perspective. Each job $j$ 
    visits the $m$ machines in the same sequence, and after the operation of 
    $j$ on one machine is completed, the job enters the ``queue'' of the next 
    machine in the sequence. Moreover, whenever a machine completes an 
    operation, the machine picks another operation from its queue, and
    processes that. 

    In some applications, each machine is required to process the jobs in
    the order the jobs enter the machine's queue. Such schedules are called 
    {\bf FIFO schedules}, and such flow shops are called {\bf permutation
    flow shops}; these additional constraints are indicated in the 
    $\beta$ field with $\beta = prmu$. 

\end{itemize}

{\bf Side Constraints} ($\beta$) The side constraints capture the various 
restrictions on the scheduling problem. We note that there could be more than 
one side constraint. Some of the side constraints relevant to our course are 
as follows.

\begin{itemize}

    \item {\bf Release Dates} ($\beta = r_j$) Unless specified, we assume 
    that all jobs are available from the beginning.

    \item {\bf Setup Times} ($\beta = s_{jk}(i)$) In some applications, a 
    setup time is required on machine $i$ after the completion of job $j$ 
    and before the start of the next job $k$. For example, the machine $i$ 
    may need to be cleaned and recalibrated between jobs $j$ and $k$. Unless
    mentioned otherwise, these setup times are assumed to be zero.

    \item {\bf Precedence Constraints} ($\beta = prec$) In some applications, 
    job $k$ cannot be processed until another job $j$ is completed. Such 
    constraints are called precedence constraints. We assume that these 
    constraints are not cyclical; that is, we do not have a situation
    where job $j$ precedes job $k$, job $k$ precedes job $\ell$, and job 
    $\ell$ precedes job $j$. One represents the precedence constraints via a 
    directed acyclic graph (DAG) where the nodes represent the various jobs, 
    and an arc from node $j$ to node $k$ represents the constraint ``$j$ 
    precedes $k$'' (that is, $k$ cannot start until $j$ is completed).

    \item {\bf Preemption} ($\beta = prmp$) Informally speaking, if a job 
    allows preemption, then the ``scheduler'' is allowed to interrupt the 
    processing of the job (preempt) at any point in time and put a different 
    job on the machine instead. The amount of processing a preempted job 
    already has received is not lost.

    Consider a job $j$ that consists of a single operation. Job $j$ does 
    not allow preemption if all of the processing of $j$ must occur on one 
    machine in one contiguous time period; otherwise, the job allows preemption 
    (in which case the job could be processed by two or more distinct machines,
    or it could be processed by one machine over two or more non-contiguous 
    time periods). By default, we assume that preemption is not allowed. 

\end{itemize}

{\bf Objective} ($\gamma$) The objective function specifies the optimality 
criterion for choosing among several feasible schedules. There is a large list
of possible objective functions, depending on the applications. Before we get 
into the objective functions, we will introduce some definitions.

\begin{itemize}

    \item Given a job $j$, the {\bf completion time} of job $j$ in a schedule 
    $S$, denoted $C_j^S$, is the time when all operations of the job have 
    been completed. We drop the superscript $S$ when the schedule is clear 
    from the context.

    \item When jobs have due dates, the {\bf lateness} of a job $j$, 
    denoted $L_j$, is the difference between the completion time and the 
    due date. That is, we have 
    \[ L_j = C_j - d_j. \] 
    Note that if $L_j < 0$, then the job $j$ is not late. 

    \item A closely related measure is the {\bf tardiness} of a job, 
    which is the maximum of $0$ and the lateness of the job. Therefore, 
    we have 
    \[ T_j = \max(0, L_j) = \max(0, C_j - d_j). \] 
    
    \item Finally, we use $U_j$ to indicate whether a job $j$ is completed 
    after the due date. If a job $j$ has $C_j > d_j$, then we say that 
    the job is {\bf tardy} and we have $U_j = 1$; otherwise, we have $U_j = 0$. 

\end{itemize}  

We now discuss a few fundamental objective functions that are most relevant 
to the course.

\begin{itemize}

    \item {\bf Makespan} ($\gamma = C_{\max}$) Find a schedule which minimizes 
    the maximum completion time $C_{\max} := \max_{j\in J} C_j$.

    \item {\bf Total Weighted Completion Time} ($\gamma = \sum w_j C_j$)
    Find a schedule which minimizes the weighted average time taken for a 
    job to complete. When all weights are $1$, we instead write 
    $\sum C_j$ in the field. This measure is also called the {\bf flow time}
    or the {\bf weighted flow time}.

    \item {\bf Maximum Lateness} ($\gamma = L_{\max}$) Find a schedule 
    which minimizes the maximum lateness of a job; that is, we want to 
    minimize $L_{\max} := \max_{j\in J} L_j$. 

    \item {\bf Weighted number of tardy jobs} ($\gamma = \sum w_j U_j$)
    Find a schedule that minimizes the weighted number of tardy jobs.
    When all weights are $1$, we instead write $\sum U_j$ in the field. 

\end{itemize}

Observe that all of the above objective functions are non-decreasing in 
$C_1, \dots, C_n$; in order words, if we take two schedules $S$ and $S'$, 
and the completion times are such that $C_j^S \leq C_j^{S'}$ for all 
$j$, then the objective value of $S$ is at most the objective value of $S'$. 
Such objective functions or performance measures are said to be {\bf regular}.

\begin{prop}{prop:1.4}
    The performance measure $\sum U_j$ is regular. 
\end{prop}
\begin{pf}
    Consider two schedules $S$ and $S'$ with $C_j^S \leq C_j^{S'}$ for all 
    $j$. Note that if $C_j^S > d_j$, then $C_j^{S'} > d_j$. 
    Hence, if $U_j^S = 1$ for a job $j$, then $U_j^{S'} = 1$ as well. It follows 
    that $\sum U_j^S \leq \sum U_j^{S'}$, so $\sum U_j$ is regular. 
\end{pf}

\begin{exercise}{exercise:1.5}
    Prove that all of the above performance measures are regular.
\end{exercise}

Not all performance measures are regular. For instance, there are scheduling 
problems where each job comes $j$ with a time window $[a_j, b_j]$ and the 
job needs to processed in that particular time window. Let $X_j = 0$ if the 
job is processed in that time window, and $X_j = 1$ otherwise. 
Is the performance measure $\sum X_j$ regular?
