\section{Additional Topics} \label{sec:10}

\subsection{Weighted Vertex Cover} \label{subsec:10.1}
Given a graph $G = (V, E)$ with vertex weights $w_i \geq 0$, the 
weighted vertex cover problem is to find a minimum weight subset of 
vertices $S \subseteq V$ such that every edge is incident to at least 
one vertex in $S$. 

Let $x_i = 1$ if vertex $i$ is chosen to be in the vertex cover, and 
$x_i = 0$ otherwise. Then an integer linear programming formulation for the 
weighted vertex cover is given as follows. 
\begin{align*}
    \min\quad & \sum_{i\in V} w_i x_i \\ 
    \text{s.t.}\quad & x_i + x_j \geq 1, && \text{for all } (i, j) \in E, \\  
    & x_i \geq 0 \text{ and integer}, && \text{for all } i \in V. 
\end{align*}
Note that there is no need for an upper bound on $x_i$ because this is 
a minimization problem, which implicitly forces the restriction $x_i \leq 1$.
Then the LP relaxation simply removes the integrality constraint above. 

\begin{lemma}{lemma:10.1}
    If $x^*$ is optimal for the LP relaxation, then $S = \{i \in V : x_i^* \geq 1/2\}$ 
    is a vertex cover whose weight is at most twice the minimum possible weight. 
\end{lemma}
\begin{pf}
    To see that $S$ is a vertex cover, let $(i, j)$ be an edge in $G$. Then the 
    first constraint tells us that $x_i^* + x_j^* \geq 1$. We must have at least 
    one of $x_i^* \geq 1/2$ or $x_j^* \geq 1/2$ for otherwise $x_i^* < 1/2$ 
    and $x_j^* < 1/2$ would imply $x_i^* + x_j^* < 1$. Thus, the edge 
    $(i, j)$ is covered. 

    Next, we check that $S$ has the desired weight. If $S^*$ is an optimal 
    vertex cover, then 
    \[ \sum_{i\in S^*} w_i \geq \sum_{i\in S} w_i x_i^* \geq \frac{1}{2} \sum_{i\in S} w_i, \] 
    where the first inequality is because the $x_i^*$ is an optimal solution to the 
    LP relaxation whereas the first sum corresponds to an optimal solution 
    to the IP, and the second inequality is because 
    $x_i^* \geq 1/2$ for all $i \in S$. 
\end{pf}

Since the LP relaxation can be solved in polynomial time, this gives us a 
$2$-approximation algorithm for weighted vertex cover. 

\subsection{LP-based Approximation Algorithm for Makespan on Unrelated Machines}
\label{subsec:10.2}
The $(R~||~C_{\max})$ problem is $\NP$-hard, but the approach we used for the 
weighted vertex cover problem gives us some inspiration to find a similar 
LP-based approximation algorithm. In this section, we will discuss a 
$2$-approximation algorithm for $(R~||~C_{\max})$. 

First, we make an attempt at formulating the problem as an integer linear program. 
Call it $(\text{IP}_1)$. 
\begin{align*}
    \min\quad & C \\ 
    \text{s.t.}\quad & \sum_{j=1}^n p_{ij} x_{ij} \leq C, && \text{for all } i = 1, \dots, m, \\ 
    & \sum_{i=1}^m x_{ij} = 1, && \text{for all } j = 1, \dots, n, \\ 
    & x_{ij} \geq 0 \text{ and integer}, && \text{for all } i = 1, \dots, m,\; j = 1, \dots, n. 
\end{align*}
We make a quick aside on designing LP-based approximation algorithms for 
minimization problems. Note that $\text{OPT}(\text{LP}) \leq 
\text{OPT}(\text{IP})$ in general, so the goal is to find a schedule 
$S$ such that 
\[ C_{\max}(S) \leq \alpha \cdot \text{OPT}(\text{LP}) \leq 
\alpha \cdot \text{OPT}(\text{IP}) \] 
for some factor $\alpha = 1 + \eps$ where $\eps > 0$. 

The {\bf integrality gap} (or {\bf integrality ratio}) of an LP relaxation is 
given by 
\[ \text{IR}(\text{LP}) = \max_I
\frac{\text{OPT}(\text{IP})}{\text{OPT}(\text{LP})}, \] 
where the maximum runs all over all instances $I$ of the problem. In particular, 
if $\text{IR}(\text{LP}) > \alpha$, then we could not hope to design an 
$\alpha$-approximation algorithm using the LP relaxation because we 
cannot prove the approximation guarantee of $\alpha$. 

Let us return to our original integer linear program $(\text{IP}_1)$. 
The LP relaxation of $(\text{IP}_1)$, say $(\text{LP}_1)$, replaces the 
last constraint with $x_{ij} \geq 0$.

It turns out that $(\text{LP}_1)$ is not enough for a $2$-approximation 
algorithm because the integrality gap is unbounded. Indeed, consider the instance 
of $(R~||~C_{\max})$ where there are $m \geq 3$ machines and $1$ job 
which has $p_{i1} = m$ for all $i = 1, \dots, m$. Then the optimal solution 
to the LP relaxation takes $x_{i1} = 1/m$ for all $i = 1, \dots, m$ 
and has value $1$, whereas the optimal solution to the IP has optimal 
value $m$ by taking $x_{i1} = 1$ for some $i = 1, \dots, m$. In particular, 
$(\text{LP}_1)$ has integrality gap at least $m \geq 3$. 

Why does $(\text{IP}_1)$ have such a large integrality ratio? A handwavy 
explanation is that there are large values of $p_{ij}$ that cannot be 
used for solving the IP, but can be exploited by the LP relaxation. 

Therefore, we need a new plan. Fix $\alpha = 2$, and suppose that we 
know the optimal makespan $C$ (which is $\NP$-hard to compute). 
We give a subroutine that takes an instance of $(R~||~C_{\max})$ 
and parameter $C$ as input, which either 
\begin{enumerate}[(i)]
    \item finds a schedule $S$ with $C_{\max}(S) \leq \alpha C$, or 
    \item certifies that the optimal makespan is greater than $C$. 
\end{enumerate}
Given the parameter $C$, we define the set $S_C = \{(i, j) : p_{ij} \leq C\}$.
The idea is that we are throwing away all the irrelevant jobs! We then 
give an integer linear program $(\text{IP}_C)$ as follows.
\begin{align*}
    \min\quad & 0 \\ 
    \text{s.t.}\quad & \sum_{j=1}^n p_{ij} x_{ij} \leq C, && \text{for all } i = 1, \dots, m, \\ 
    & \sum_{(i, j) \in S_C} x_{ij} = 1, && \text{for all } j = 1, \dots, n, \\ 
    & x_{ij} \geq 0 \text{ and integer}, && \text{for all } i = 1, \dots, m,\; j = 1, \dots, n. 
\end{align*}
In particular, in the second constraint, we are only summing over the 
relevant jobs in $S_C$. Moreover, the objective does not matter now 
since we are only looking for a feasible solution. Let $(\text{LP}_C)$ denote 
the LP relaxation of $(\text{IP}_C)$.

\begin{theo}{theo:10.2}
    Suppose that $(\text{LP}_C)$ has a feasible solution for some $C$.
    Then there is an efficient algorithm that finds a schedule $S$ 
    with $C_{\max}(S) \leq 2C$. 
\end{theo}
\begin{pf}
    Let $x^*$ be a basic feasible solution (BFS) of $(\text{LP}_C)$. 

    Recall from CO 250 that for an LP with constraints $\{Ax \geq b,\; 
    x \geq 0\}$ where $A$ is a $p \times q$ matrix, a BFS $x^*$ has at most 
    $p$ non-zero entries. (The proof of this can be derived by writing 
    the LP in standard equality form.)

    Let $G = (V, E^*)$ be the bipartite graph which has a node corresponding 
    to each machine $i$ and a node corresponding to each job $j$. 
    There is an edge $(i, j) \in E^*$ if and only if $x_{ij}^* > 0$. 

    Note that $G$ has $m + n$ nodes and $(\text{LP}_C)$ has $m + n$ 
    constraints. Since $x^*$ is a BFS, we see from the fact from CO 250 
    that $G$ has at most $m + n$ edges. 

    We may assume without loss of generality that $G$ is connected. 
    Indeed, if $G$ were not connected, then we could consider the connected 
    components of $G$ one by one and apply the rounding algorithm to each. 

    Then a result from graph theory says that a connected graph $(U, F)$ 
    with $|F| \leq |U|$ has at most one cycle, so in particular, $G$ 
    has at most one cycle. 

    We are now ready to state the rounding algorithm. 
    \begin{enumerate}
        \item Call a job node $j$ a {\bf leaf} if $\deg_G(j) = 1$. 
        Assign each leaf job to the unique machine $i$ adjacent to $j$ 
        in $G$. (Due to the second set of constraints, note that the unique edge 
        $(i, j)$ incident to a leaf job $j$ has $x_{ij}^* = 1$.) 
        
        Delete all leaf jobs from $G$ to obtain a graph $G'$ where $\deg_{G'}(j) \geq 2$ 
        for all job nodes $j$, and the number of edges is at most the number of 
        nodes. We'll maintain this inequality throughout because whenever 
        we delete a node, we also delete at least one edge. 

        \item If $G'$ has a cycle $Q$ (and there is at most one), then pick a 
        maximum matching $\hat M$ of $Q$. Assign jobs to the machines according 
        to this matching, and delete the job nodes afterwards. 

        \item For any remaining machine node $i$ with $\deg_{G'}(i) = 1$, 
        assign its unique job node neighbour $j$ to machine $i$. Delete 
        the job node together with the machine node in this case. 
    \end{enumerate}
    At termination, the algorithm has assigned each job to some machine. 

    Note that after Step 1, each machine is assigned to at most one job. 
    This can be seen by handling the unique cycle of $G'$ 
    (if it exists) at the very last step. 

    Moreover, the makespan of the resulting schedule is at most $2C$. 
    Indeed, at Step 1, the load of any machine $i$ is at most the 
    load assigned to machine $i$ by $(\text{LP}_C)$, which is at most $C$. 
    The later stages put at most one job on machine $i$. Since 
    $p_{ij} \leq C$ for all considered jobs in $(\text{LP}_C)$, our 
    claim follows. 
\end{pf}

Based on Theorem~\ref{theo:10.2}, we get an algorithm that uses 
binary search to handle the parameter $C$. Assuming that the values of 
$p_{ij}$ are integral, perform the following steps: 
\begin{enumerate}
    \item Set $L = 1$ and $U = np_{\max}$ where $p_{\max} = \max\{p_{ij}\}$. 
    \item While $U - L > 1$, do the following: 
    \begin{enumerate}[(a)]
        \item Let $C = (L + U)/2$. 
        \item If $(\text{LP}_C)$ is feasible, set $U = C$; otherwise, set $L = C$. 
    \end{enumerate}
\end{enumerate}
Let $C^*$ be the value of $C$ at the end of the while loop. Note that 
$C^*$ is the minimum value of $C$ such that $(\text{LP}_C)$ is feasible. 
By Theorem~\ref{theo:10.2}, we can find a schedule $S$ with makespan 
at most $2C^*$ efficiently. 