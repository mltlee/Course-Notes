\documentclass[10pt]{article}
\usepackage[T1]{fontenc}
\usepackage{amsmath,amssymb,amsthm}
\usepackage{mathtools}
\usepackage[shortlabels]{enumitem}
\usepackage[english]{babel}
\usepackage[utf8]{inputenc}
\usepackage{fancyhdr}
\usepackage{bold-extra}
\usepackage{color}   
\usepackage{tocloft}
\usepackage{graphicx}
\usepackage{lipsum}
\usepackage{wrapfig}
\usepackage{cutwin}
\usepackage{hyperref}
\usepackage{lastpage}
\usepackage{multicol}
\usepackage{tikz}
\usepackage{xcolor}
\usepackage{microtype}
\usepackage{verbatim}
\usepackage{listings}
\usepackage[framemethod=TikZ]{mdframed}

\lstset{
  basicstyle=\normalfont\ttfamily,
  columns=flexible,
  mathescape
}

% some useful math commands
\newcommand{\eps}{\varepsilon}
\newcommand{\R}{\mathbb{R}}
\newcommand{\C}{\mathbb{C}}
\newcommand{\N}{\mathbb{N}}
\newcommand{\Z}{\mathbb{Z}}
\newcommand{\Q}{\mathbb{Q}}
\newcommand{\K}{\mathbb{K}}
\newcommand{\F}{\mathbb{F}}

\numberwithin{equation}{section}

\newcommand{\dd}{\,\mathrm{d}}
\newcommand{\ddz}{\frac{\rm d}{{\rm d}z}}
\newcommand{\pv}{\text{p.v.}}

\renewcommand{\Re}{{\rm Re}}

\DeclareMathOperator{\GL}{GL}
\DeclareMathOperator{\id}{id}
\DeclareMathOperator{\Arg}{Arg}
\DeclareMathOperator{\Log}{Log}
\DeclareMathOperator{\PV}{PV}
\DeclareMathOperator{\sech}{sech}
\DeclareMathOperator{\csch}{csch}
\DeclareMathOperator{\Res}{Res}
\DeclareMathOperator{\Li}{Li}
\DeclareMathOperator{\QR}{QR}
\DeclareMathOperator{\NR}{NR}
\DeclareMathOperator{\lcm}{lcm}
\DeclareMathOperator{\OPT}{OPT}

\DeclarePairedDelimiter\ceil{\lceil}{\rceil}
\DeclarePairedDelimiter\floor{\lfloor}{\rfloor}

\newcommand{\suchthat}{\;\ifnum\currentgrouptype=16 \;\middle|\;\else\mid\fi\;}

% title formatting
\newcommand{\newtitle}[4]{
  \begin{center}
	\huge{\textbf{\textsc{#1 Course Notes}}}
    
	\large{\sc #2}
    
	{\sc #3 \textbullet\, #4 \textbullet\, University of Waterloo}
	\normalsize\vspace{1cm}\hrule
  \end{center}
}

\newcounter{theo}[section]\setcounter{theo}{0}
\renewcommand{\thetheo}{\arabic{section}.\arabic{theo}}
\newenvironment{theo}[2][]{%
\refstepcounter{theo}%
\ifstrempty{#1}%
{\mdfsetup{%
frametitle={%
\tikz[baseline=(current bounding box.east),outer sep=0pt]
\node[anchor=east,rectangle,fill=blue!20]
{\strut {\sc Theorem~\thetheo}};}}
}%
{\mdfsetup{%
frametitle={%
\tikz[baseline=(current bounding box.east),outer sep=0pt]
\node[anchor=east,rectangle,fill=blue!20]
{\strut {\sc Theorem~\thetheo:~#1}};}}%
}%
\mdfsetup{innertopmargin=10pt,linecolor=blue!20,%
linewidth=2pt,topline=true,%
frametitleaboveskip=\dimexpr-\ht\strutbox\relax
}
\begin{mdframed}[nobreak=true]\relax%
\label{#2}}{\end{mdframed}}

%%%%%%%%%%%%%%%%%%%%%%%%%%%%%%
%Definition
\newenvironment{defn}[2][]{%
\refstepcounter{theo}%
\ifstrempty{#1}%
{\mdfsetup{%
frametitle={%
\tikz[baseline=(current bounding box.east),outer sep=0pt]
\node[anchor=east,rectangle,fill=yellow!20]
{\strut {\sc Definition~\thetheo}};}}
}%
{\mdfsetup{%
frametitle={%
\tikz[baseline=(current bounding box.east),outer sep=0pt]
\node[anchor=east,rectangle,fill=yellow!20]
{\strut {\sc Definition~\thetheo:~#1}};}}%
}%
\mdfsetup{innertopmargin=10pt,linecolor=yellow!20,%
linewidth=2pt,topline=true,%
frametitleaboveskip=\dimexpr-\ht\strutbox\relax
}
\begin{mdframed}[nobreak=true]\relax%
\label{#2}}{\end{mdframed}}

%%%%%%%%%%%%%%%%%%%%%%%%%%%%%%
%Example
\newenvironment{exmp}[2][]{%
\refstepcounter{theo}%
\ifstrempty{#1}%
{\mdfsetup{%
frametitle={%
\tikz[baseline=(current bounding box.east),outer sep=0pt]
\node[anchor=east,rectangle,fill=cyan!20]
{\strut {\sc Example~\thetheo}};}}
}%
{\mdfsetup{%
frametitle={%
\tikz[baseline=(current bounding box.east),outer sep=0pt]
\node[anchor=east,rectangle,fill=cyan!20]
{\strut {\sc Example~\thetheo:~#1}};}}%
}%
\mdfsetup{innertopmargin=10pt,linecolor=cyan!20,%
linewidth=2pt,topline=true,%
frametitleaboveskip=\dimexpr-\ht\strutbox\relax
}
\begin{mdframed}[nobreak=true]\relax%
\label{#2}}{\end{mdframed}}

%%%%%%%%%%%%%%%%%%%%%%%%%%%%%%
%Corollary
\newenvironment{cor}[2][]{%
\refstepcounter{theo}%
\ifstrempty{#1}%
{\mdfsetup{%
frametitle={%
\tikz[baseline=(current bounding box.east),outer sep=0pt]
\node[anchor=east,rectangle,fill=lime!20]
{\strut {\sc Corollary~\thetheo}};}}
}%
{\mdfsetup{%
frametitle={%
\tikz[baseline=(current bounding box.east),outer sep=0pt]
\node[anchor=east,rectangle,fill=lime!20]
{\strut {\sc Corollary~\thetheo:~#1}};}}%
}%
\mdfsetup{innertopmargin=10pt,linecolor=lime!20,%
linewidth=2pt,topline=true,%
frametitleaboveskip=\dimexpr-\ht\strutbox\relax
}
\begin{mdframed}[nobreak=true]\relax%
\label{#2}}{\end{mdframed}}

%%%%%%%%%%%%%%%%%%%%%%%%%%%%%%
%Remark
\newenvironment{remark}[2][]{%
\refstepcounter{theo}%
\ifstrempty{#1}%
{\mdfsetup{%
frametitle={%
\tikz[baseline=(current bounding box.east),outer sep=0pt]
\node[anchor=east,rectangle,fill=orange!20]
{\strut {\sc Remark~\thetheo}};}}
}%
{\mdfsetup{%
frametitle={%
\tikz[baseline=(current bounding box.east),outer sep=0pt]
\node[anchor=east,rectangle,fill=orange!20]
{\strut {\sc Remark~\thetheo:~#1}};}}%
}%
\mdfsetup{innertopmargin=10pt,linecolor=orange!20,%
linewidth=2pt,topline=true,%
frametitleaboveskip=\dimexpr-\ht\strutbox\relax
}
\begin{mdframed}[nobreak=true]\relax%
\label{#2}}{\end{mdframed}}

%%%%%%%%%%%%%%%%%%%%%%%%%%%%%%
%Exercise
\newenvironment{exercise}[2][]{%
\refstepcounter{theo}%
\ifstrempty{#1}%
{\mdfsetup{%
frametitle={%
\tikz[baseline=(current bounding box.east),outer sep=0pt]
\node[anchor=east,rectangle,fill=pink!20]
{\strut {\sc Exercise~\thetheo}};}}
}%
{\mdfsetup{%
frametitle={%
\tikz[baseline=(current bounding box.east),outer sep=0pt]
\node[anchor=east,rectangle,fill=pink!20]
{\strut {\sc Exercise~\thetheo:~#1}};}}%
}%
\mdfsetup{innertopmargin=10pt,linecolor=pink!20,%
linewidth=2pt,topline=true,%
frametitleaboveskip=\dimexpr-\ht\strutbox\relax
}
\begin{mdframed}[nobreak=true]\relax%
\label{#2}}{\end{mdframed}}

%%%%%%%%%%%%%%%%%%%%%%%%%%%%%%
%Lemma
\newenvironment{lemma}[2][]{%
\refstepcounter{theo}%
\ifstrempty{#1}%
{\mdfsetup{%
frametitle={%
\tikz[baseline=(current bounding box.east),outer sep=0pt]
\node[anchor=east,rectangle,fill=green!20]
{\strut {\sc Lemma~\thetheo}};}}
}%
{\mdfsetup{%
frametitle={%
\tikz[baseline=(current bounding box.east),outer sep=0pt]
\node[anchor=east,rectangle,fill=green!20]
{\strut {\sc Lemma~\thetheo:~#1}};}}%
}%
\mdfsetup{innertopmargin=10pt,linecolor=green!20,%
linewidth=2pt,topline=true,%
frametitleaboveskip=\dimexpr-\ht\strutbox\relax
}
\begin{mdframed}[nobreak=true]\relax%
\label{#2}}{\end{mdframed}}

%%%%%%%%%%%%%%%%%%%%%%%%%%%%%%
%Proposition
\newenvironment{prop}[2][]{%
\refstepcounter{theo}%
\ifstrempty{#1}%
{\mdfsetup{%
frametitle={%
\tikz[baseline=(current bounding box.east),outer sep=0pt]
\node[anchor=east,rectangle,fill=purple!20]
{\strut {\sc Proposition~\thetheo}};}}
}%
{\mdfsetup{%
frametitle={%
\tikz[baseline=(current bounding box.east),outer sep=0pt]
\node[anchor=east,rectangle,fill=purple!20]
{\strut {\sc Proposition~\thetheo:~#1}};}}%
}%
\mdfsetup{innertopmargin=10pt,linecolor=purple!20,%
linewidth=2pt,topline=true,%
frametitleaboveskip=\dimexpr-\ht\strutbox\relax
}
\begin{mdframed}[nobreak=true]\relax%
\label{#2}}{\end{mdframed}}

%%%%%%%%%%%%%%%%%%%%%%%%%%%%%%
%Algorithm
\newenvironment{algo}[2][]{%
\refstepcounter{theo}%
\ifstrempty{#1}%
{\mdfsetup{%
frametitle={%
\tikz[baseline=(current bounding box.east),outer sep=0pt]
\node[anchor=east,rectangle,fill=pink!20]
{\strut {\sc Algorithm~\thetheo}};}}
}%
{\mdfsetup{%
frametitle={%
\tikz[baseline=(current bounding box.east),outer sep=0pt]
\node[anchor=east,rectangle,fill=pink!20]
{\strut {\sc Algorithm~\thetheo:~#1}};}}%
}%
\mdfsetup{innertopmargin=10pt,linecolor=pink!20,%
linewidth=2pt,topline=true,%
frametitleaboveskip=\dimexpr-\ht\strutbox\relax
}
\begin{mdframed}[nobreak=true]\relax%
\label{#2}}{\end{mdframed}}

% new proof environment
\makeatletter
\newenvironment{pf}[1][\proofname]{\par
  \pushQED{\qed}%
  \normalfont \topsep0\p@\relax
  \trivlist
  \item[\hskip\labelsep\scshape
  #1\@addpunct{.}]\ignorespaces
}{%
  \popQED\endtrivlist\@endpefalse
}
\makeatother

% 1-inch margins
\topmargin 0pt
\advance \topmargin by -\headheight
\advance \topmargin by -\headsep
\textheight 8.9in
\oddsidemargin 0pt
\evensidemargin \oddsidemargin
\marginparwidth 0.5in
\textwidth 6.5in

\parindent 0in
\parskip 1.5ex

\setlist[itemize]{topsep=0pt}
\setlist[enumerate]{topsep=0pt}

\newcommand{\pushright}[1]{\ifmeasuring@#1\else\omit\hfill$\displaystyle#1$\fi\ignorespaces}

% hyperlinks
\hypersetup{
  colorlinks=true, 
  linktoc=all,     % table of contents is clickable  
  allcolors=red    % all hyperlink colours
}

% table of contents
\addto\captionsenglish{
  \renewcommand{\contentsname}%
    {Table of Contents}%
}
\renewcommand{\cftsecfont}{\normalfont}
\renewcommand{\cftsecpagefont}{\normalfont}
\cftsetindents{section}{0em}{2em}

\fancypagestyle{plain}{%
\fancyhf{} % clear all header and footer fields
\lhead{CO 454: Spring 2022}
\fancyhead[R]{Table of Contents}
%\headrule
\fancyfoot[R]{{\small Page \thepage\ of \pageref*{LastPage}}}
}

% headers and footers
\pagestyle{fancy}
\renewcommand{\sectionmark}[1]{\markboth{#1}{#1}}
\lhead{CO 454: Spring 2022}
\cfoot{}
\setlength\headheight{14pt}

%\setcounter{section}{-1}

\begin{document}

\pagestyle{fancy}
\newtitle{CO 454}{Scheduling}{Joseph Cheriyan}{Spring 2022}
\rhead{Table of Contents}
\rfoot{{\small Page \thepage\ of \pageref*{LastPage}}}

\tableofcontents
\vspace{1cm}\hrule
\fancyhead[R]{\nouppercase\rightmark}
\newpage 
\fancyhead[R]{Section \thesection: \nouppercase\leftmark}

\section{Placeholder section}\label{sec:1}

\subsection{Placeholder subsection}\label{subsec:1.1}
\newpage
\section{Single Machine Models}\label{sec:2}
Single machine models are very important, as they are relatively simple 
and can be viewed as a special case of all other environments. We 
will analyze various single machine models in detail, such as 
the total weighted completion time, as well as some due date related 
objectives in the assignments. One observation that we can make for 
single machine models is that when a problem is non-preemptive and the 
objective is regular, finding an optimal schedule boils down to finding 
a sequence of jobs.

\subsection{Total Weighted Completion Time}\label{subsec:2.1}
Before we begin, we should say something about interchange arguments, which 
are commonplace in scheduling. Suppose we have two different sequences 
for the same set of jobs, say 
\begin{enumerate}
    \item a reference sequence $R = r_1, r_2, \dots, r_n$, and 
    \item an adversary sequence $A = a_1, a_2, \dots, a_n$,
\end{enumerate}
satisfying $\{r_1, \dots, r_n\} = \{a_1, \dots, a_n\}$ but $R \neq A$. 

\begin{prop}{prop:2.1}
    There exists an adjacent pair of items in $A$, say $a_i$ and $a_{i+1}$, 
    such that $a_{i+1}$ precedes $a_i$ in $R$. 
\end{prop}
\begin{pf}
    Assume no such pair exists, so every adjacent pair $a_i$ and $a_{i+1}$
    of items in $A$ is such that $a_i$ precedes $a_{i+1}$ in $R$ as well. 
    Then the only way for $R$ to have exactly $n$ jobs with 
    $a_i$ preceding $a_{i+1}$ for all $i \in \{1, \dots, n-1\}$ is 
    for $R$ to be the sequence $a_1, a_2, \dots, a_n$. This is a contradiction 
    with our assumption that $R \neq A$. 
\end{pf}

We can now give an example of an interchange argument. A so-called {\bf adjacent 
pairwise interchange} uses Proposition \ref{prop:2.1} to obtain 
two adjacent items which can be swapped. 

Consider the problem $(1\;\|\;\sum C_j)$, where there are $n$ jobs with 
processing times $p_1, \dots, p_n$. The {\bf Shortest Processing Time 
first (SPT) rule} says to put the shortest processing times first.

\begin{theo}{theo:2.2}
    The SPT rule is optimal for $(1\;\|\;\sum C_j)$. 
\end{theo}
\begin{pf}
    Assume for simplicity that the processing times $p_1, \dots, p_n$ are 
    distinct. Suppose there is a schedule $S$ that does not satisfy the 
    SPT rule which is optimal. There exist two adjacent jobs, say $k$ followed 
    by $\ell$, such that $p_k > p_\ell$, and using adjacent pairwise 
    interchange, we can obtain a new schedule $S'$ by swapping $k$ and $\ell$. 

    Note that all completion times are the same in $S$ and $S'$ except for 
    $C_k$ and $C_\ell$. Suppose that $t$ is the starting time of job $k$ 
    in $S$. Then in $S$, we have $C_k^S = t + p_k$ and $C_\ell^S = t + p_k + 
    p_\ell$. On the other hand, in $S'$, we have $C_k^{S'} = t + p_k + p_\ell$ 
    and $C_\ell^{S'} = t + p_\ell$. We see that $C_\ell^S = C_k^{S'}$, 
    so subtracting the objectives yields 
    \[ \sum C_j^S - \sum C_j^{S'} = C_k^S - C_\ell^{S'} = p_k - p_\ell > 0. \] 
    This means that $S'$ has a better objective value than $S$, contradicting 
    the optimality of $S$. 
\end{pf}

More generally, we can consider the total weighted completion time 
$(1\;\|\;\sum w_j C_j)$. This problem gives rise to the {\bf Weighted 
Shortest Processing Time first (WSPT) rule}, and according to this rule, 
the jobs are placed in decreasing order of $w_j/p_j$. 

\begin{theo}{theo:2.3}
    The WSPT rule is optimal for $(1\;\|\;\sum w_j C_j)$. 
\end{theo}
\begin{pf}
    Again, we apply an interchange argument. Suppose that there is a optimal 
    schedule $S$ that is not WSPT. Then there must exist two adjacent 
    jobs, say job $k$ followed by job $\ell$, such that 
    \[ \frac{w_k}{p_k} < \frac{w_\ell}{p_\ell}. \] 
    Using adjacent pairwise interchange, obtain a new schedule $S'$ by 
    swapping the jobs $k$ and $\ell$. As before, all completion times 
    are the same in $S$ and $S'$ except for $C_k$ and $C_\ell$. 
    Suppose that job $k$ starts processing at time $t$ in $S$. Under $S$, 
    the total weighted completion time for jobs $k$ and $\ell$ is 
    \[ w_k(t + p_k) + w_\ell(t + p_k + p_\ell), \] 
    whereas under $S'$, it is equal to 
    \[ w_k(t + p_\ell + p_k) + w_\ell(t + p_\ell). \] 
    Then subtracting the objective of $S$ from the objective of $S'$ yields 
    the quantity 
    \[ w_\ell p_k - w_k p_\ell, \] 
    which is positive due to the assumption that $w_k/p_k < w_\ell/p_\ell$. 
    This contradicts the optimality of $S$. 
\end{pf}

The computation time needed to order the jobs according to the WSPT rule 
is the time required to sort the jobs according to the ratio of the 
two parameters. This takes $O(n\log n)$ time since this is the time it takes
to perform a simple sort. Since the SPT rule is a special case of the WSPT 
rule with all weights equal to $1$, it also requires $O(n\log n)$ time. 

How is the minimization of the total weighted completion time affected by
precedence constraints?  Consider the simplest form of precedence constraints
which take the form of parallel chains. This problem can still be solved by a 
relatively simple and very efficient (polynomial time) algorithm. This 
algorithm is based on some fundamental properties of scheduling with 
precedence constraints.

Consider two chains of jobs. The first chain consists of jobs $1, \dots, k$, 
and the second chain consists of jobs $k+1, \dots, n$. The precedence 
constraints are then $1 \to 2 \to \cdots \to k$ and $k+1 \to k+2 \to 
\cdots \to n$. 

The next lemma is based on the assumption that if the scheduler decides to
start processing jobs of one chain, then they have to complete the entire 
chain before they are allowed to work on jobs of the other chain.
Which of the two chains should be processed first? 

\begin{lemma}{lemma:2.4}
    If we have 
    \[ \frac{\sum_{j=1}^k w_j}{\sum_{j=1}^k p_j} > 
    \frac{\sum_{j=k+1}^n w_j}{\sum_{j=k+1}^n p_j}, \] 
    then it is optimal to process the chain of jobs $1, \dots, k$ before 
    the chain of jobs $k+1, \dots, n$. 
\end{lemma}
\begin{pf}
    We proceed by contradiction. Under the sequence $1, \dots, k, k+1, 
    \dots, n$, the total completion time is 
    \[ w_1p_1 + \cdots + w_k \sum_{j=1}^k p_j + w_{k+1} \sum_{j=1}^{k+1} 
    p_j + \cdots + w_n \sum_{j=1}^n p_j, \] 
    while under the sequence $k+1, \dots, n, 1, \dots, k$, it is 
    \[ w_{k+1}p_{k+1} + \cdots + w_n \sum_{j=k+1}^n p_j + w_1 
    \left( \sum_{j=k+1}^n p_j + p_1 \right) + \cdots + w_k \sum_{j=1}^n p_j. \] 
    Using the inequality 
    \[ \frac{\sum_{j=1}^k w_j}{\sum_{j=1}^k p_j} > 
    \frac{\sum_{j=k+1}^n w_j}{\sum_{j=k+1}^n p_j}, \] 
    the total weighted completion time of the first sequence 
    is less than that of the second. The result follows. 
\end{pf}

An interchange between two adjacent chains of jobs is usually referred to as
an {\bf adjacent sequence interchange}. Such an interchange is a generalization of
an adjacent pairwise interchange.

\begin{defn}{defn:2.5}
    Let $1 \to 2 \to \cdots \to k$ be a chain. Let $\ell^*$ satisfy 
    \[ \frac{\sum_{j=1}^{\ell^*} w_j}{\sum_{j=1}^{\ell^*} p_j} 
    = \max_{\ell\in\{1, \dots, k\}} \left( \frac{\sum_{j=1}^\ell w_j}
    {\sum_{j=1}^\ell p_j} \right). \] 
    The ratio on the left-hand side is called the {\bf $\rho$-factor} 
    of the chain $1, \dots, k$ and is denoted by $\rho(1, \dots, k)$. 
    The job $\ell^*$ is referred to as the {\bf determining job} of the chain.
\end{defn}

More generally, we now assume that the scheduler does not have to fully 
complete chains immediately; they can process some jobs of one chain 
(while adhering to the precedence constraints), switch over to another 
chain, and revisit the original chain later. If the total weighted 
completion time is the objective function, then the following result holds. 

\begin{lemma}{lemma:2.6}
    For a chain of jobs $1 \to 2 \to \cdots \to k$, suppose $\ell^*$ is 
    a determining job. Then there exists an optimal schedule that processes 
    the jobs $1, \dots, \ell^*$ consecutively, without processing any jobs 
    of any other chains. 
\end{lemma}
\begin{pf}
    We proceed by contradiction. Suppose that under the optimal sequence, 
    the processing of the subsequence $1, \dots, \ell^*$ is interrupted by a 
    job, say $v$, from another chain. That is, the optimal sequence 
    contains the subsequence $1, \dots, u, v, u+1, \dots, \ell^*$; call 
    this subsequence $S$. It suffices to show that either with subsequence 
    $v, u+1, \dots, \ell^*$ which we denote $S'$, or subsequence 
    $1, \dots, u, v$, which we denote $S''$, the total weighted completion 
    time is less than with subsequence $S$. If it is not less with the
    first subsequence, then it has to be less with the second and vice versa.
    From Lemma~\ref{lemma:2.4}, it follows that if the total weighted 
    completion time with $S$ is less than with $S'$, then
    \[ \frac{w_v}{p_v} < \frac{w_1 + w_2 + \cdots + w_u}{p_1 + p_2 + \cdots 
    + p_u}. \] 
    Lemma~\ref{lemma:2.4} also tells us that if the total weighted completion 
    time with $S$ is less than with $S''$, then
    \[ \frac{w_v}{p_v} > \frac{w_{u+1} + w_{u+2} + \cdots + w_{\ell^*}}
    {p_{u+1} + p_{u+2} + \cdots + p_{\ell^*}}. \] 
    If job $\ell^*$ is the determining job for the chain $1, \dots, k$, then 
    by definition of $\ell^*$, we have 
    \[ \frac{w_1 + \cdots + w_u + w_{u+1} + \cdots + w_{\ell^*}}{p_1 
    + \cdots + p_u + p_{u+1} + \cdots + p_{\ell^*}} > 
    \frac{w_{u+1} + w_{u+2} + \cdots + w_{\ell^*}}
    {p_{u+1} + p_{u+2} + \cdots + p_{\ell^*}}. \] 
    Noting that $(a+c)/(b+d) > a/b$ implies $c/d > a/b$, we obtain 
    \[ \frac{w_{u+1} + w_{u+2} + \cdots + w_{\ell^*}}
    {p_{u+1} + p_{u+2} + \cdots + p_{\ell^*}} > \frac{w_1 + w_2 + \cdots + w_u}
    {p_1 + p_2 + \cdots + p_u}. \] 
    If $S$ is better than $S''$, this means that 
    \[ \frac{w_v}{p_v} > \frac{w_{u+1} + w_{u+2} + \cdots + w_{\ell^*}}
    {p_{u+1} + p_{u+2} + \cdots + p_{\ell^*}} > \frac{w_1 + w_2 + \cdots + w_u}
    {p_1 + p_2 + \cdots + p_u}. \] 
    Therefore, $S'$ is better than $S$. The same argument applies if the 
    interruption of the chain is caused by more than one job. 
\end{pf}

Intuitively, Lemma~\ref{lemma:2.6} makes sense. The condition of the lemma 
implies that the ratios of the weight divided by the processing time of the 
jobs in the string $1, \dots, \ell^*$ must be increasing in some sense. 
If one had already decided to start processing a string of jobs, it makes 
sense to continue processing the string until job $\ell^*$ is completed 
without processing any other job in between. Our two lemmas contain the 
basis for a simple algorithm that minimizes the total weighted completion 
time when the precedence constraints take the form of chains.

\begin{algo}[Total Weighted Completion Time and Chains]{algo:2.7}
    Whenever the machine is freed, select among the remaining chains the one 
    with the highest $\rho$-factor. Process this chain without interruption 
    up to and including the job that determines its $\rho$-factor.
\end{algo}

We illustrate the use of this algorithm with an example. 

\begin{exmp}{exmp:2.8}
    Consider the two chains $1 \to 2 \to 3 \to 4$ and $5 \to 6 \to 7$. 
    The weights and processing times of the jobs are as follows. 
    \begin{align*}
        \begin{array}{c|ccccccc}
        j   & 1 & 2  & 3  & 4 & 5 & 6  & 7  \\ \hline
        w_j & 6 & 18 & 12 & 8 & 8 & 17 & 18 \\
        p_j & 3 & 6  & 6  & 5 & 4 & 8  & 10
        \end{array}
    \end{align*}
    The $\rho$-factor of the first chain is $(6+18)/(3+6)$ and is determined 
    by job $2$. The $\rho$-factor of the second chain is $(8+17)/(4+8)$ and 
    is determined by job $6$. Since $24/9$ is larger than $25/12$, jobs 
    $1$ and $2$ are processed first. The $\rho$-factor of the remaining 
    part of the first chain is $12/6$ and is determined by job $3$. This is 
    less than $25/12$, so we process jobs $5$ and $6$ next. The $\rho$-factor 
    of the remaining part of the second chain is $18/10$ and is determined by 
    job $7$. Hence, job $3$ follows job $6$. The $w_j/p_j$ ratio of job $7$ 
    is higher than that of job $4$, so job $7$ follows job $3$ and job $4$ 
    goes last. 
\end{exmp}\newpage

\end{document}
