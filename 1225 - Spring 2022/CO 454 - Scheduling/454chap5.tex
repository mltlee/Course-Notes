\section{Advanced Single Machine Models} \label{sec:5}

\subsection{Total Earliness and Tardiness} \label{subsec:5.1}
One could imagine a situation where an earliness penalty is incurred. For 
example, if one needs to go to convocation in a different city a week from now, 
they wouldn't want to go immediately as they would need to pay for rent 
or parking.  We define the earliness of a job $j$ by $E_j = \max(0, d_j - C_j)$.
Note that unlike the performance measures we have discussed so far, 
earliness is non-regular because it is decreasing with respect to 
completion time. 

In this section, we cover a generalization of the total tardiness 
problem. We will focus on the objective 
\[ \sum_{j=1}^n E_j + \sum_{j=1}^n T_j. \] 
Intuitively, this problem is harder than total tardiness, which is known to be 
$\NP$-hard. Therefore, we will focus on the special case where all 
jobs have the same due date $d_j = d$. 

An optimal schedule for this special case has some useful properties. For 
example, it is easily shown that there is no idleness between any two jobs 
for an optimal schedule, as one could close the gap by shifting jobs 
later if they are early, or shifting jobs earlier if they are late. 
Moreover, it is possible that an optimal schedule does not start processing 
the jobs immediately at time $0$. Indeed, by the previous property, 
an increase of $d$ would increase the starting time of the first job if 
$d$ is sufficiently large. 

We also observe that we can divide any sequence of jobs into 
two disjoint sets and possibly one additional job. One set consists of the jobs 
that are completed early so that $C_j \leq d$, and we will denote it by $J_1$. 
The other set consists of the jobs that are started late, and we will denote it 
by $J_2$. There may be another job which is started early but 
completed late. 

\begin{prop}{prop:5.1}
    There is an optimal sequence in which one job is completed exactly 
    at time $d$. 
\end{prop}
\begin{pf}
    We proceed by contradiction. Suppose there is no such schedule. Then
    there is always one job that starts its processing before time $d$ and 
    completes its processing after time $d$. Call this job $j^*$. 
    Let $|J_1|$ denote the number of jobs that are early and $|J_2|$ 
    denote the number of jobs that are late. If $|J_1| < |J_2|$, then we 
    can shift the entire schedule to the left in such a way that job $j^*$
    completes its processing exactly at time $d$. This implies that the 
    total tardiness decreases by $|J_2|$ times the length of the shift, 
    while the total earliness increases by $|J_1|$ times the length of 
    the shift. Clearly, the objective value is reduced. The case 
    where $|J_1| > |J_2|$ can be treated in a similar way. 

    The case where $|J_1| = |J_2|$ is somewhat special. In this case, there 
    are many optimal schedules, of which only two satisfy the property 
    given in the lemma. 
\end{pf}

Due to Proposition~\ref{prop:5.1}, we can avoid the situation where we have the 
additional job in the middle altogether. In particular, there is an optimal 
schedule such that the first job starts at time $0$ or later, the jobs in $J_1$ 
scheduled first according to Longest Processing Time (LPT), and the jobs in $J_2$
are scheduled last according to Shortest Processing Time (SPT). In this 
sense, the schedule is shaped like a ``V''. Note that for a schedule of the 
above form, say $1, \dots, n$, the total earliness is 
\[ \sum_{j\in J_1} E_j = 0 \cdot p_1 + 1 \cdot p_2 + 2 \cdot 
p_3 + \cdots + (|J_1| - 1) \cdot p_{|J_1|}, \] 
while the total tardiness is 
\[ \sum_{j\in J_2} T_j = 1 \cdot p_{|J_1|+|J_2|} + 2 \cdot 
p_{|J_1|+|J_2|-1} + \cdots + |J_2| \cdot p_{|J_1|+1}. \] 
For an instance in which all optimal schedules start processing the first job
some time after $t = 0$ (that is, the due date $d$ is sufficiently large), 
the following algorithm yields the optimal allocations of jobs to sets 
$J_1$ and $J_2$. We assume that the jobs are ordered such that $p_1 \geq 
p_2 \geq \cdots \geq p_n$. 

\begin{algo}[Minimizing Total Earliness and Tardiness with Loose Due Date]{algo:5.2}
    {\sc Step 1.} Assign job $1$ to set $J_1$. Initialize $k = 2$. 

    {\sc Step 2.} Assign job $k$ to set $J_1$ and job $k+1$ to set $J_2$, 
    or vice versa. 

    {\sc Step 3.} If $k+2 \leq n-1$, then increase $k$ by $2$ and go back to 
    Step 2. If $k+2 = n$, assign job $n$ to either set $J_1$ or set $J_2$ 
    and stop. If $k+2 = n+1$, then all jobs have been assigned, so we stop. 
\end{algo}

This algorithm is somewhat flexible in its assignment of jobs to sets $J_1$ and 
$J_2$. It can be implemented in a way that in the optimal assignment, the 
total processing time of the jobs assigned to $J_1$ is minimized. Given the 
total processing time of the jobs in $J_1$ and the due date $d$, it can be 
verified easily whether the machine indeed must remain idle before it starts 
processing its first job.

If the due date $d$ is tight and it is necessary to start processing 
a job at time $0$, then the problem is $\NP$-hard. The following heuristic 
which assigns the $n$ jobs to the $n$ positions in the sequence can be 
very effective, but does not always yield an optimal sequence. Again, we 
assume that $p_1 \geq p_2 \geq \cdots \geq p_n$.  

\begin{algo}[Minimizing Total Earliness and Tardiness with Tight Due Date]{algo:5.3}
    {\sc Step 1.} Initialize $\tau_1 = d$ and $\tau_2 = \sum_{j=1}^n p_j - d$. 
    Set $k = 1$.  

    {\sc Step 2.} If $\tau_1 > \tau_2$, then assign job $k$ to the first 
    unfilled position in the sequence and decrease $\tau_1$ by $p_k$. 
    If $\tau_1 < \tau_2$, then assign job $k$ to the last unfilled position 
    in the sequence and decrease $\tau_2$ by $p_k$. 

    {\sc Step 3.} If $k < n$, then increase $k$ by $1$ and go back to Step 2. 
    If $k = n$, then stop. 
\end{algo}

We give an example of this heuristic in action. 

\begin{exmp}{exmp:5.4}
    Consider the following example with $6$ jobs and common due date $d = 180$. 
    \begin{align*}
        \begin{array}{c|cccccc}
            j & 1 & 2 & 3 & 4 & 5 & 6 \\ \hline 
            p_j & 106 & 100 & 96 & 22 & 20 & 2
        \end{array}
    \end{align*}
    Applying the heuristic yields the following results. 
    \begin{align*}
        \begin{array}{c|c|c|c}
            \tau_1 & \tau_2 & \text{Assignment} & \text{Sequence} \\ \hline 
            180 & 166 & \text{Job $1$ placed first} & 1, *, *, *, *, * \\ 
            74 & 166 & \text{Job $2$ placed last} & 1, *, *, *, *, 2 \\ 
            74 & 66 & \text{Job $3$ placed first} & 1, 3, *, *, *, 2 \\ 
            -22 & 66 & \text{Job $4$ placed last} & 1, 3, *, *, 4, 2 \\ 
            -22 & 44 & \text{Job $5$ placed last} & 1, 3, *, 5, 4, 2 \\
            -22 & 24 & \text{Job $6$ placed last} & 1, 3, 6, 5, 4, 2 \\  
        \end{array}
    \end{align*}
\end{exmp}

To end this section, we discuss some other versions of total 
earliness and tardiness problems. 
\begin{itemize}

    \item Consider the objective $\sum w'E_j + \sum w''T_j$ where all 
    due dates are equal with $d_j = d$. All jobs have exactly the same 
    cost function, but the earliness penalty $w'$ and the tardiness 
    penalty $w''$ are not the same. All the previous properties and
    algorithms can be generalized relatively easily to take the 
    difference between $w'$ and $w''$ into account. 

    \item Consider the more general objective $\sum w'_j E_j + \sum w''_j T_j$ 
    with $d_j = d$ for all jobs $j$. In this case, the shapes of the cost 
    functions of the jobs are different. The LPT-SPT sequence described above 
    is not necessarily optimal. The first part of the sequence must now be 
    ordered according to Weighted Longest Processing Time first (WLPT), 
    which is in increasing order of $w_j/p_j$. The second part of the 
    sequence is ordered by Weighted Shortest Processing Time first (WSPT), 
    which we are already familiar with. 

    \item Consider the model with the objective function $\sum w'E_j + \sum w''T_j$
    where we do not assume that the due dates are equal. It is clear that this 
    case is $\NP$-hard as it is a more general model of the one we discussed in
    Section~\ref{subsec:4.3}. This problem has an additional level of complexity. Because of
    the different due dates, it may not necessarily be optimal to process the 
    jobs one after another without interruption; it may be necessary to have 
    idle times between the processing of consecutive jobs. Therefore, this 
    problem has two aspects: one concerning the sequencing of the jobs, 
    and the other concerning the idle time between the jobs. Approaches 
    for this problem are typically based either on dynamic programming 
    or branch and bound. However, given a predetermined sequence, the timing of 
    the processing of the jobs (and therefore also the idle times) can be 
    determined in polynomial time. This can also be applicable in a more 
    general setting, which we will describe next. 

    \item The most general setting has objective $\sum w'_j E_j + 
    \sum w''_j T_j$ where the jobs have different due dates and weights. 
    This problem is strongly $\NP$-hard since it is harder than 
    $(1~||~\sum w_j T_j)$, which is also known to be strongly $\NP$-hard.
    But again, given a predetermined sequence of the jobs, the timing 
    of the jobs can be determined in polynomial time. However, we will not 
    go into detail about this. 

\end{itemize} 

\subsection{Primary and Secondary Objectives} \label{subsec:5.2}
In practice, a scheduler is often concerned with more than one objective. 
For example, one might want to minimize inventory costs and meet due dates. 
It would then be of interest, for instance, to find a schedule that minimizes a 
combination of $\sum C_j$ and $L_{\max}$. 

In many cases, there is more than one optimal schedule that minimizes a 
given objective. A scheduler may then wish to consider the set of all 
schedules that are optimal with respect to one objective (which 
we will call the primary objective), and then search within this set of 
schedules for the schedule that is best with regard to a secondary objective. 
If the primary objective is $\gamma_1$ and the secondary objective is 
$\gamma_2$, then such a problem is denoted by 
$(\alpha~|~\beta~|~\gamma_1^{(1)}, \gamma_2^{(2)})$.

We begin with the simple example $(1~||~\sum C_j^{(1)}, L_{\max}^{(2)})$ 
where the primary objective is the total completion time and the secondary 
objective is the maximum lateness. If there are no jobs with identical 
processing times, then there is exactly one schedule that minimizes the 
total completion time (namely the one obtained by SPT), so there is 
no freedom to minimize $L_{\max}$. If there are jobs with identical processing 
times, then there are multiple schedules that minimize the total completion 
time. A set of jobs with identical processing times is preceded by a job with a 
strictly shorter processing time and followed by a job with a strictly longer
processing time. Jobs with identical processing times have to be processed 
one after another, but the key is that they may be done in any order. 
Then to minimize $L_{\max}$, we can put these jobs with identical processing 
times in EDD order. We can refer to this rule as SPT/EDD, since the jobs are 
first scheduled according to SPT and ties are broken according to EDD. This 
rule can generalized to the case where the primary objective is 
the total weighted completion time $\sum w_j C_j$. 

Consider now the problem $(1~||~L_{\max}^{(1)}, \sum C_j^{(2)})$, which 
has the same two objectives with reversed priorities. We know that 
the EDD rule minimizes $L_{\max}$; suppose that the value of this minimum 
$L_{\max}$ is $z$. Then the original problem can be transformed to 
another equivalent problem. By creating a new set of due dates 
$\overline{d_j} = d_j + z$, we see that these due dates are now deadlines. 
The problem is to find a schedule which minimizes $\sum C_j$ subject to 
the constraint that every job is completed by its deadline. We denote 
this problem by $(1~|~\text{hard } d_j~|~\sum C_j)$. The algorithm 
for finding the optimal schedule is based on the following result. 

\begin{lemma}{lemma:5.5}
    There exists a schedule that minimizes $\sum C_j$ in which job $k$ 
    is scheduled last if and only if 
    \begin{enumerate}[(1)]
        \item $\overline{d_k} \geq \sum_{j=1}^n p_j$, and 
        \item $p_k \geq p_\ell$ for all jobs $\ell$ such that 
        $\overline{d_\ell} \geq \sum_{j=1}^n p_j$. 
    \end{enumerate}
\end{lemma}
\begin{pf} 
    Assume towards a contradiction that job $k$ is not scheduled last. 
    There is a set of jobs that is scheduled after job $k$, and 
    some job $\ell$ that is scheduled last. Note that job $\ell$ 
    must satisfy condition (1) for otherwise it would not meet its due date. 
    This means that job $\ell$ does not satisfy condition (2) and 
    so $p_\ell < p_k$. After swapping jobs $k$ and $\ell$, the sum of the 
    completion times of jobs $k$ and $\ell$ decreases, and the sum of the 
    completion times of jobs scheduled between $k$ and $\ell$ also decreases. 
    Hence, the original schedule that positioned job $\ell$ last could not have 
    minimized $\sum C_j$. 
\end{pf}

In the following algorithm, $J^c$ represents the set of jobs that have 
not been scheduled yet. 

\begin{algo}[Minimizing Total Completion Time with Deadlines]{algo:5.6}
    {\sc Step 1.} Set $k = n$, $\tau = \sum_{j=1}^n p_j$, and $J^c = \{1, \dots, n\}$. 

    {\sc Step 2.} Find a job $k^* \in J^c$ such that $\overline{d_{k^*}} \geq \tau$ 
    and $p_{k^*} \geq p_\ell$ for all jobs $\ell \in J^c$ such that 
    $\overline{d_\ell} \geq \tau$. Put job $k^*$ in position $k$ of the sequence. 

    {\sc Step 3.} Decrease $k$ by $1$, decrease $\tau$ by $p_{k^*}$, and 
    delete job $k^*$ from $J^c$. 

    {\sc Step 4.} If $k \geq 1$, then go back to Step 2. Otherwise, stop. 
\end{algo}

We give a simple example of the use of this backward algorithm. 

\begin{exmp}{exmp:5.7}
    Consider the following instance of $5$ jobs. 
    \begin{align*}
        \begin{array}{c|ccccc}
            j & 1 & 2 & 3 & 4 & 5 \\ \hline 
            p_j & 4 & 6 & 2 & 4 & 2 \\ 
            d_j & 10 & 12 & 14 & 18 & 18
        \end{array}
    \end{align*}
    For ease of notation, we will let $\hat J(\tau) = \{\ell \in J^c : 
    \overline{d_\ell} \geq \tau\}$ be the set of eligible jobs that have 
    not been scheduled yet. We start with $\tau = \sum_{j=1}^5 p_j = 18$, 
    and applying the algorithm yields the following results. 
    \begin{align*}
        \begin{array}{c|c|c|c} 
            \text{Iteration} & \tau & \hat J(\tau) & k^* \\ \hline 
            1 & 18 & \{4, 5\} & 4 \\ 
            2 & 14 & \{3, 5\} & 3 \text{ or } 5 \\ 
            3 & 12 & \{2, 5\} \text{ or } \{2, 3\} & 2 \\ 
            4 & 6 & \{1, 5\} \text{ or } \{1, 3\} & 1 \\ 
            5 & 2 & \{5\} \text{ or } \{3\} & 5 \text{ or } 3            
        \end{array}
    \end{align*}
    Thus, we obtain two optimal schedules, which are $5, 1, 2, 3, 4$ and 
    $3, 1, 2, 5, 4$. 
\end{exmp}

It can be shown that even when preemptions are allowed, the optimal schedules 
are non-preemptive for the $(1~|~\text{hard } d_j~|~\sum C_j)$ problem. We can 
also assume feasibility of the problem in general as it can be checked easily 
using the EDD rule. 

Finally, we note that a fairly large number of problems of the form 
$(1~|~\beta~|~\gamma_1^{(1)}, \gamma_2^{(2)})$ have been studied in the 
literature. Very few of them can be solved in polynomial time. However, 
problems of the type $(1~|~\beta~|~\sum w_j C_j^{(1)}, \gamma_2^{(2)})$ 
tend to be easy based on the analysis of the $(1~||~\sum C_j^{(1)}, 
L_{\max}^{(2)})$ problem we discussed at the beginning of this section. 
We will see one such example on Assignment 6.

\subsection{Parametric Analysis of Multiple Objectives} \label{subsec:5.3}
Suppose that we have the two objectives $\gamma_1$ and $\gamma_2$. If 
the overall objective is $\theta_1 \gamma_1 + \theta_2 \gamma_2$ where 
$\theta_1$ and $\theta_2$ are the weights of the two objectives, then 
we can denote the scheduling problem by $(1~|~\beta~|~\theta_1 \gamma_1 
+ \theta_2 \gamma_2)$. Multiplying both weights by the same constant 
does not change the problem, so we may assume without loss of generality 
that $\theta_1 + \theta_2 = 1$. For this section, we will focus on a 
specific class of schedules. 

\begin{defn}{defn:5.8}
    A schedule is called {\bf Pareto optimal} if it is not possible to 
    decrease the value of one objective without increasing the value of 
    the other. 
\end{defn}

All Pareto optimal schedules can be represented by a set of points in the 
$(\gamma_1, \gamma_2)$ plane. This set of points illustrates the tradeoffs 
between the two objectives. Consider the two objectives we analyzed in the 
previous section, namely $\sum C_j$ and $L_{\max}$. The two cases 
we covered are the two extreme points of the tradeoff curve. Note that 
if $\theta_1 \to 0$ and $\theta_2 \to 1$, then 
\[ (1~|~\beta~|~\theta_1\gamma_1 + \theta_2\gamma_2) \to 
(1~|~\beta~|~\gamma_2^{(1)}, \gamma_1^{(2)}), \] 
while if $\theta_2 \to 0$ and $\theta_1 \to 1$, then 
\[ (1~|~\beta~|~\theta_1\gamma_1 + \theta_2\gamma_2) \to 
(1~|~\beta~|~\gamma_1^{(1)}, \gamma_2^{(2)}). \] 
We will now focus on generating all Pareto optimal schedules for the 
problem $(1~||~\theta_1 \sum C_j + \theta_2 L_{\max})$. 

On one end of the extremes is the $(1~||~\sum L_{\max}^{(1)}, \sum C_j^{(2)})$ 
problem, which could be solved using a more complicated backward algorithm
where we have $L_{\max} = L_{\max}(\text{EDD})$. 

On the other end, we have the $(1~||~\sum C_j^{(1)}, L_{\max}^{(2)})$ problem, 
where the total completion time was minimized by the SPT rule and ties were 
broken by EDD. We denoted this strategy by SPT/EDD, and we easily see that 
$L_{\max} = L_{\max}(\text{SPT/EDD})$ in this case. It is clear that 
\[ L_{\max}(\text{EDD}) \leq L_{\max}(\text{SPT/EDD}). \] 
The algorithm that generates all Pareto optimal solutions in the tradeoff 
curve contains two loops. One series of steps in the algorithm (the inner 
loop) is an adaptation of Algorithm~\ref{algo:5.6}. In addition to the optimal 
schedule with a maximum allowable $L_{\max}$, these steps determine 
the minimum increment $\delta$ in the $L_{\max}$ that would allow for a 
decrease in the minimum $\sum C_j$. The second (outer) loop of the algorithm 
contains the structure that generates all the Pareto optimal points.
The outer loop calls the inner loop at each Pareto optimal point to generate a 
schedule at that point and also to determine how to move to the next efficient 
point. The algorithm starts out with the EDD schedule that generates the first
Pareto optimal point in the upper left part of the trade-off curve. It 
determines the minimum increment in the $L_{\max}$ needed to achieve a 
reduction in $\sum C_j$. Given this new value of $L_{\max}$, it uses 
Algorithm~\ref{algo:5.6} to determine the schedule which minimizes 
$\sum C_j$, and proceeds to determine the next increment. This goes on 
until the algorithm reaches $L_{\max}(\text{SPT/EDD})$. 

\begin{algo}[Tradeoffs between Total Completion Time and 
    Maximum Lateness]{algo:5.9}
    {\sc Step 1.} Set $r = 1$. Set $L_{\max} = L_{\max}(\text{EDD})$ and 
    $\overline{d_j} = d_j + L_{\max}$. 

    {\sc Step 2.} Set $k = n$ and $J^c = \{1, \dots, n\}$. Set 
    $\tau = \sum_{j=1}^n p_j$ and $\delta = \tau$. 

    {\sc Step 3.} Find a job $j^* \in J^c$ such that $\overline{d_{j^*}} \geq \tau$ 
    and $p_{j^*} \geq p_\ell$ for all jobs $\ell \in J^c$ such that 
    $\overline{d_\ell} \geq \tau$. Put job $j^*$ in position $k$ of the 
    sequence. 

    {\sc Step 4.} If there is no job $\ell$ such that $\overline{d_\ell} < 
    \tau$ and $p_\ell > p_{j^*}$, then go to Step 5. Otherwise, let 
    \[ \delta^{**} = \min\{\tau - \overline{d_\ell} : 
    \text{jobs $\ell$ such that $\overline{d_\ell} < \tau$ and 
    $p_\ell > p_{j^*}$}\} \] 
    and set $\delta = \min\{\delta, \delta^{**}\}$. 

    {\sc Step 5.} Decrease $k$ by $1$, decrease $\tau$ by $p_{j^*}$, 
    and delete job $j^*$ from $J^c$. If $k \geq 1$, then go to Step 3. 
    If $k = 0$, then go to Step 6. 

    {\sc Step 6.} Set $L_{\max} = L_{\max} + \delta$. If $L_{\max} 
    > L_{\max}(\text{SPT/EDD})$, then stop. Otherwise, set 
    $r = r+1$, $\overline{d_j} = \overline{d_j} + \delta$, and 
    go to Step 2. 
\end{algo}

The outer loop consists of Steps 1 and 6, while the inner loop consists of 
Steps 2, 3, 4, and 5. Steps 2, 3, and 5 represent an adaptation of 
Algorithm~\ref{algo:5.6}, while Step 4 computes the minimum increment 
in the $L_{\max}$ needed to achieve a subsequent reduction in $\sum C_j$. 

It can be shown that the number of Pareto optimal solutions is bounded 
above by $\binom{n}{2} + O(1)$, which is $O(n^2)$. Generating one Pareto optimal 
solution can be done in $O(n\log n)$ time. Therefore, the total computation 
time of Algorithm~\ref{algo:5.9} is $O(n^3\log n)$. 

We now give an example of the use of the above algorithm. 

\begin{exmp}{exmp:5.10}
    Consider the following set of $5$ jobs. 
    \begin{align*}
        \begin{array}{c|ccccc}
            j & 1 & 2 & 3 & 4 & 5 \\ \hline 
            p_j & 1 & 3 & 6 & 7 & 9 \\ 
            d_j & 30 & 27 & 20 & 15 & 12
        \end{array}
    \end{align*}
    The EDD sequence is $5, 4, 3, 2, 1$ with $L_{\max}(\text{EDD}) = 2$ 
    and total completion time $98$. The SPT/EDD sequence is $1, 2, 3, 4, 5$ with 
    $L_{\max}(\text{SPT/EDD}) = 14$ and total completion time $58$. 

    We give an example of one full iteration of the inner loop.
    \begin{itemize}
        \item Initially, we have $k = 5$, $\tau = \sum_{j=1}^5 p_j = 26$, and 
        so the set of eligible jobs to put at the end of the sequence is $\{1, 2\}$.
        Since job $2$ has the largest processing time, we place it last. 
        Then we have $\delta^{**} = \tau - \overline{d_3} = 26 - 22 = 4$ and so
        $\delta = 4$. 
        \item Next, we have $k = 4$, $\tau = 23$, and job $1$ is the 
        only eligible job. Then $\delta^{**} = 23 - 22 = 1$ and $\delta = 1$. 
        \item Then, we have $k = 3$, $\tau = 22$, and the only eligible job is 
        job $3$. We then have $\delta^{**} = \tau - \overline{d_4} = 22 - 17 = 5$.
        \item For the next iteration of the inner loop, we have $k = 2$, $\tau = 16$, 
        and job $4$ is the only eligible job. Then $\delta^{**} = \tau 
        - \overline{d_5} = 16 - 12 = 4$.
        \item Finally, we pick job $5$ at the end, and obtain the Pareto optimal 
        schedule $5, 4, 3, 1, 2$ with $\delta = 1$.
    \end{itemize} 
    The full results are given in the following table. 
    \begin{align*}
        \begin{array}{c|c|c|c|c|c}
            \text{Iteration} & \sum C_j & L_{\max} & \text{Pareto optimal schedule} 
            & \text{current } \tau + \delta & \delta \\ \hline 
            1 & 96 & 2 & 5, 4, 3, 1, 2 & 32~29~22~17~14 & 1 \\ 
            2 & 77 & 3 & 1, 5, 4, 3, 2 & 33~30~23~18~15 & 2 \\ 
            3 & 75 & 5 & 1, 4, 5, 3, 2 & 35~32~25~20~17 & 1 \\ 
            4 & 64 & 6 & 1, 2, 5, 4, 3 & 36~33~26~21~18 & 2 \\ 
            5 & 62 & 8 & 1, 2, 4, 5, 3 & 38~35~28~23~20 & 3 \\ 
            6 & 60 & 11 & 1, 2, 3, 5, 4 & 41~38~31~26~23 & 3 \\ 
            7 & 58 & 14 & 1, 2, 3, 4, 5 & 44~41~34~29~26 & \text{STOP} 
        \end{array}
    \end{align*} 
\end{exmp}

However, when one would consider the objective $\theta_1 \sum C_j + \theta_2 
L_{\max}$, certain Pareto optimal schedules may never be optimal, 
no matter what the weights are. 

Consider the generalization $(1~||~\theta_1 \sum w_j C_j + \theta_2 L_{\max})$. 
It is clear that the two extreme points of the tradeoff curve can be determined 
in polynomial time using WSPT/EDD and EDD. However, even though the two 
endpoints of the tradeoff curve can be analyzed in polynomial time, the 
problem with arbitrary weights $\theta_1$ and $\theta_2$ is $\NP$-hard. 

The tradeoff curve that corresponds to the example in this section has the
shape of a staircase. This shape is fairly common in a single machine 
environment with multiple objectives, especially when preemptions are not 
allowed. However, in other machine environments such as parallel machines, 
smoother curves may occur, especially when preemptions are allowed.
We will not cover more parametric analysis in this course though. 
