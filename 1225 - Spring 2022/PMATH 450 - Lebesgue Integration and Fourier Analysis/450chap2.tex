\section{Lebesgue Measure and Integration}\label{sec:2}

\subsection{Riemann Integration}\label{subsec:2.1}
Recall that in Riemann's theory of integration, we start with a bounded function 
$f : [a, b] \to \R$. We could then obtain $\int_a^b f(x)\dd x$ via 
approximations of Riemann sums. More specifically, we take a partition 
\[ P = \{a = t_0 < t_1 < \cdots < t_n = b\} \] 
of the interval $[a, b]$. For each $1 \leq i \leq n$, we set 
$m_i = \inf_{x \in [t_{i-1}, t_i)} f(x)$ and 
$M_i = \sup_{x \in [t_{i-1}, t_i)} f(x)$. We define the {\bf lower 
Riemann sum} by 
\[ L(f, P) = \sum_{i=1}^n m_i(t_i - t_{i-1}), \] 
and similarly, the {\bf upper Riemann sum} by 
\[ U(f, P) = \sum_{i=1}^n M_i(t_i - t_{i-1}). \] 
It is clear for all partitions $P$ of $[a, b]$ that $L(f, P) \leq U(f, P)$. 
Moreover, suppose $P$ and $Q$ are both partitions of $[a, b]$, and set 
$P \vee Q$ to be the partition consisting of all points in $P$ and $Q$. 
Then recall that $P \vee Q$ refines both $P$ and $Q$, and we have 
\[ L(f, P) \leq L(f, P \vee Q) \leq U(f, P \vee Q) \leq U(f, Q). \] 
Interchanging $P$ and $Q$ above gives us $L(f, Q) \leq U(f, P)$, so we can 
deduce that 
\[ \sup_P L(f, P) \leq \inf_P U(f, P). \] 
That is, any lower Riemann sum of a given partition will always be at most 
the upper Riemann sum of any other partition. 

\begin{defn}{defn:2.1}
    We say that $f : [a, b] \to \R$ is {\bf Riemann integrable} if 
    \[ \sup_P L(f, P) = \inf_P U(f, P). \] 
    In this case, we write 
    \[ \int_a^b f(x)\dd x = \sup_P L(f, P) = \inf_P U(f, P). \] 
    We write $R[a, b]$ to denote the vector space of Riemann integrable 
    functions $f : [a, b] \to \R$. 
\end{defn}

Riemann's theory is good for many purposes, such as for the Fundamental 
Theorem of Calculus or analysis over smooth manifolds. But there are also 
many deficiencies. 
\begin{itemize}
    \item It forces $f$ to be bounded and ``almost continuous''. 
    \item It doesn't generalize to integration over sets that are not ``like''
    $\R$ or $\R^N$. Sometimes, one wants to integrate functions over 
    irregular sets, such as fractals. 
    \item There are no good limit theorems! If we have a 
    sequence of Riemann integrable functions $(f_n)_{n=1}^\infty \subseteq 
    R[a, b]$ and $f(x) = \lim_{n\to\infty} f_n(x)$ exists, then we would also 
    want $f$ to be Riemann integrable with 
    \[ \int_a^b f(x)\dd x = \lim_{n\to\infty} \int_a^b f_n(x)\dd x. \] 
    Unfortunately, this result is false! For example, one can take 
    $f_n(x) = n^2 x^n (1-x)$ for each $n \in \N$. Then $f_n \to 0$ 
    pointwise on $[0, 1]$, but 
    \[ \int_0^1 n^2 x^n (1-x)\dd x = 1. \] 
\end{itemize}