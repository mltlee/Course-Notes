\section{Lebesgue Measure and Integration}\label{sec:2}

\subsection{Riemann Integration}\label{subsec:2.1}
Recall that in Riemann's theory of integration, we start with a bounded function 
$f : [a, b] \to \R$. We could then obtain $\int_a^b f(x)\dd x$ via 
approximations of Riemann sums. More specifically, we take a partition 
\[ P = \{a = t_0 < t_1 < \cdots < t_n = b\} \] 
of the interval $[a, b]$. For each $1 \leq i \leq n$, we set 
$m_i = \inf_{x \in [t_{i-1}, t_i)} f(x)$ and 
$M_i = \sup_{x \in [t_{i-1}, t_i)} f(x)$. We define the {\bf lower 
Riemann sum} by 
\[ L(f, P) = \sum_{i=1}^n m_i(t_i - t_{i-1}), \] 
and similarly, the {\bf upper Riemann sum} by 
\[ U(f, P) = \sum_{i=1}^n M_i(t_i - t_{i-1}). \] 
It is clear for all partitions $P$ of $[a, b]$ that $L(f, P) \leq U(f, P)$. 
Moreover, suppose $P$ and $Q$ are both partitions of $[a, b]$, and set 
$P \vee Q$ to be the partition consisting of all points in $P$ and $Q$. 
Then recall that $P \vee Q$ refines both $P$ and $Q$, and we have 
\[ L(f, P) \leq L(f, P \vee Q) \leq U(f, P \vee Q) \leq U(f, Q). \] 
Interchanging $P$ and $Q$ above gives us $L(f, Q) \leq U(f, P)$, so we can 
deduce that 
\[ \sup_P L(f, P) \leq \inf_P U(f, P). \] 
That is, any lower Riemann sum of a given partition will always be at most 
the upper Riemann sum of any other partition. 

\begin{defn}{defn:2.1}
    We say that $f : [a, b] \to \R$ is {\bf Riemann integrable} if 
    \[ \sup_P L(f, P) = \inf_P U(f, P). \] 
    In this case, we write 
    \[ \int_a^b f(x)\dd x = \sup_P L(f, P) = \inf_P U(f, P). \] 
    We write $R[a, b]$ to denote the vector space of Riemann integrable 
    functions $f : [a, b] \to \R$. 
\end{defn}

Riemann's theory is good for many purposes, such as for the Fundamental 
Theorem of Calculus or analysis over smooth manifolds. But there are also 
many deficiencies. 

For one, it forces $f$ to be bounded and ``almost continuous''. It also doesn't 
generalize to integration over sets that are not ``like'' $\R$ or $\R^N$. 
Sometimes, one wants to integrate functions over irregular sets, such as fractals. 

Worst of all, there are no good limit theorems! Ideally, we want some kind of 
``monotone convergence theorem'' which says that if we have a sequence of 
Riemann integrable functions $(f_n)_{n=1}^\infty \subseteq R[a, b]$ satisfying 
$f_1 \leq f_2 \leq \cdots$ and $f(x) = 
\lim_{n\to\infty} f_n(x)$ exists, then $f$ is also Riemann integrable with 
\[ \int_a^b f(x)\dd x = \lim_{n\to\infty} \int_a^b f_n(x)\dd x. \] 
Unfortunately, this result is false! Note that the pointwise limit of Riemann 
integrable functions might not even be Riemann integrable. Let 
$\{r_n\}_{n=1}^\infty$ be an enumeration of $\Q \cap [0, 1]$, and for each 
$n \in \N$, define the function $f_n : [0, 1] \to \R$ by 
\[ f_n(x) = \begin{cases}
    1, & \text{if } x \in \{r_1, \dots, r_n\}, \\ 
    0, & \text{otherwise.} 
\end{cases} \] 
Then for all $n \in \N$, we have $f_n \in R[0, 1]$ with 
\[ \int_0^1 f_n(x)\dd x = 0. \] 
Moreover, we see that $(f_n)_{n=1}^\infty$ converges pointwise to 
\[ f(x) = \begin{cases} 
    1, & \text{if } x \in \Q \cap [0, 1], \\ 
    0, & \text{otherwise,} 
\end{cases} \] 
the indicator function of $\Q$ over $[0, 1]$. Notice that this is a nice 
monotone limit since $f_1 \leq f_2 \leq \cdots$. Even still, $f \notin R[0, 1]$
since the rationals and irrationals are dense in $\R$, so given any 
partition $P$ of $[0, 1]$, the upper Riemann sum is $U(f, P) = 1$ and 
the lower Riemann sum is $L(f, P) = 0$. 

\subsection{Lebesgue Outer Measure}\label{subsec:2.2}
In the previous section, we saw that Riemann's theory of integration had 
some flaws. Lebesgue had the idea that we could do Riemann sums over 
partitions of the $y$-axis, instead of partitions over the $x$-axis. It turns 
out that this idea allows many more functions to be integrable. 

Take a function $f : [a, b] \to \R$, and suppose that we have a partition 
$P = \{y_0 < y_1 < \cdots < y_n\}$ of the $y$-axis. Then we will take sums of 
terms of the form 
\[ y_i \cdot \ell(\{x \in [a, b] : f(x) \in (y_{i-1}, y_i]\}), \] 
where $\ell(E)$ denotes the ``length'' of $E$. Notice that the set above 
is the preimage of the half-open interval $(y_{i-1}, y_i]$, and so we have an 
approximation 
\[ \int_a^b f(x)\dd x \approx \sum_{i=1}^n y_i \cdot \ell(f^{-1}(y_{i-1}, y_i]). \] 
Now, consider the case where $f$ is the indicator function of $\Q$ over 
$[0, 1]$ as we discussed above. Then we would have 
\[ \int_0^1 f(x)\dd x \approx 1 \cdot \ell(\Q \cap [0, 1]) + 0 \cdot 
\ell(\Q^c \cap [0, 1]). \] 
We would expect this to equal $0$ if we want our ``monotone convergence theorem''
to hold, and also because $\Q$ is a countable set. As such, a desired property 
of our ``length'' function would be to have $\ell(\Q \cap [0, 1]) = 0$. 

We have now packaged the problem into generalizing the notion of length 
from nice sets (such as unions of intervals) to more complicated sets 
of the form $f^{-1}(y_{i-1}, y_i]$. This turns out to be a difficult task 
even for a continuous function $f$. We wish to have a function 
\[ m : {\cal P}(\R) \to [0, \infty] \] 
satisfying the following properties: 
\begin{enumerate}[(1)]
    \item For any interval $I$ from $a$ to $b$ (which could be open, closed, 
    or half-open), we have 
    \[ m(I) = b-a. \] 
    \item {\bf Translation invariance.} For all $x \in \R$, we have 
    \[ m(E + x) = m(E), \] 
    where $E + x = \{y + x : y \in E\}$. 
    \item {\bf Countable additivity.} If $E_n \subseteq \R$ are disjoint 
    for all $n \in \N$ and $E = \bigsqcup_{n=1}^\infty E_n$, then 
    \[ m(E) = \sum_{n=1}^\infty m(E_n). \] 
\end{enumerate}
Bad news: there is no function $m : {\cal P}(\R) \to [0, \infty]$ 
satisfying all of (1) to (3)! 

To see this, we will assume there is such a function $m$ satisfying all $3$ 
properties, and find a subset $E \subseteq [0, 1)$ such that $m(E)$ is not 
well-defined. Define an equivalence relation on $[0, 1)$ by $x \sim y$ 
if and only if $x - y \in \Q$. We leave it as an exercise to verify that 
this is indeed an equivalence relation. As usual, let $[x]$ be the 
equivalence class of each $x \in [0, 1)$. 

Choose a set $E$ of representatives of all the equivalence classes $[x]$ for 
$x \in [0, 1)$. Note that this is possible by the Axiom of Choice, and we have 
$E \subseteq [0, 1)$ with $|E \cap [x]| = 1$ for all $x \in [0, 1)$. 

We claim that $m(E)$ is not well-defined. Let $\{r_n\}_{n=1}^\infty$ be 
an enumeration of $\Q \cap [0, 1)$, and for each $n \in \N$, set 
\begin{align*}
    E_n = E + r_n \text{ (mod $1$)} = ((E + r_n) \cap [0, 1)) \sqcup 
    ((E + r_n - 1) \cap [0, 1)). 
\end{align*}
Since $m$ satisfies properties (1) to (3), we have that 
\begin{align*}
    m(E_n) &= m((E + r_n) \cap [0, 1)) + m((E + r_n - 1) \cap [0, 1)) & 
    \text{by (3)} \\ 
    &= m((E + r_n) \cap [0, 1)) + m((E + r_n) \cap [1, 2)) & \text{by (2)} \\ 
    &= m(E + r_n) & \text{by (3)} \\ 
    &= m(E) & \text{by (2)}
\end{align*}
for all $n \in \N$. We leave it as an exercise to check that $[0, 1) = 
\bigsqcup_{n=1}^\infty E_n$. It follows that 
\[ 1 = m([0, 1)) = \sum_{m=1}^\infty m(E_n) = \sum_{n=1}^\infty m(E) \] 
where the second equality comes from (3), so there is no admissible value for 
$m(E)$. 

In light of this result, we might be asking too much for $m$ 
to satisfy all $3$ properties. What can we do instead? 
\begin{enumerate}[(a)]
    \item We can restrict the domain of the function $m$ to a 
    more ``tractable'' family of subsets of $\R$. Naturally, we would want to 
    allow all intervals, as well as open and closed sets, to be in this family.
    \item One other approach is to take a function $m$ that works for all 
    subsets $E \subseteq \R$ and agrees with our intuitive notion of length 
    for intervals, but in doing so, sacrificing some of the desirable properties. 
\end{enumerate}

The standard approach is the first one, and we shall adopt it. We now 
introduce a candidate function that could be used for $m$. 

\begin{defn}{defn:2.2}
    The {\bf Lebesgue outer measure} of a subset $E \subseteq \R$ is defined to be 
    \[ m^*(E) = \inf\left\{ \sum_{n=1}^\infty \ell(I_n) : \{I_n\}_{n=1}^\infty 
    \text{ a cover of $E$ by intervals}\right\}. \] 
\end{defn}

We look at some properties of the Lebesgue outer measure. 

\begin{prop}{prop:2.3}
    \begin{enumerate}[(a)]
        \item We have $m^*(\varnothing) = 0$ and $m^*(E) \geq 0$ 
        for all $E \subseteq \R$. 
        \item {\bf Translation invariance.} For all $E \subseteq \R$ and 
        $x \in \R$, we have $m^*(E+x) = m^*(E)$. 
        \item {\bf Monotonicity.} If $E \subseteq F \subseteq \R$, then 
        $m^*(E) \leq m^*(F)$. 
        \item {\bf Countable subadditivity.} Suppose that $E = 
        \bigcup_{n=1}^\infty E_n$ where the subsets $E_n \subseteq \R$ are not 
        necessarily disjoint. Then 
        \[ m^*(E) \leq \sum_{n=1}^\infty m^*(E_n). \] 
    \end{enumerate}
\end{prop}
\begin{pf}~
    \begin{enumerate}[(a)]
        \item This is clear from definition. 
        \item Coverings of $E$ by countable families of intervals are in 
        bijection with those of $E + x$. In particular, we have 
        $E \subseteq \bigcup_{n=1}^\infty I_n$ if and only if 
        $E + x \subseteq \bigcup_{n=1}^\infty (I_n + x)$, with 
        \[ \sum_{n=1}^\infty \ell(I_n) = \sum_{n=1}^\infty \ell(I_n + x). \] 
        Taking infima, we obtain $m^*(E) = m^*(E + x)$. 
        \item Any covering $\bigcup_{n=1}^\infty I_n$ of $F$ also gives a 
        covering of $E$. On the other hand, coverings of $E$ are not 
        necessarily coverings of $F$. Then the infimum in $m^*(E)$ is 
        taken over a larger collection than with $m^*(F)$, so we get 
        $m^*(E) \leq m^*(F)$. 
        \item Without loss of generality, suppose that $\sum_{n=1}^\infty 
        m^*(E_n) < \infty$. Let $\eps > 0$. Since $m^*(E_n) < \infty$ 
        for all $n \in \N$, there exists a covering $\bigcup_{k=1}^\infty 
        I_{k,n}$ of $E_n$ such that 
        \[ \sum_{k=1}^\infty \ell(I_{k,n}) < m^*(E_n) + \frac{\eps}{2^n}. \] 
        Then we have $E = \bigcup_{n=1}^\infty E_n \subseteq 
        \bigcup_{n,k=1}^\infty I_{k,n}$, so it follows that 
        \begin{align*}
            m^*(E) &\leq \sum_{n=1}^\infty \sum_{k=1}^\infty \ell(I_{k,n}) \\ 
            &\leq \sum_{n=1}^\infty \left( m^*(E_n) + \frac{\eps}{2^n} \right) \\
            &= \sum_{n=1}^\infty m^*(E_n) + \eps. 
        \end{align*}
        Since $\eps > 0$ was arbitrary, the result follows. \qedhere 
    \end{enumerate}
\end{pf}

The following result tells us that we can compute $m^*(E)$ using ``small''
open intervals $I_n$ in our covers. 

\begin{prop}{prop:2.4}
    Let $E \subseteq \R$ and $\delta > 0$. Then 
    \[ m^*(E) = \inf\left\{ \sum_{n=1}^\infty \ell(I_n) : 
    \{I_n\}_{n=1}^\infty \text{ a cover of $E$ by open intervals such that }
    \ell(I_n) < \delta \right\}. \] 
\end{prop}
\begin{pf}
    It is clear that $m^*(E)$ is at most the right hand side, because 
    the collection we are taking the infimum over in the definition of 
    $m^*$ is more general than the one in this proposition, which forces 
    the intervals to be open and have length less than $\delta$. 

    So, we turn to proving the other direction. Without loss of generality, 
    we may assume that $m^*(E) < \infty$. Let $\eps > 0$. We can 
    find intervals $\{J_n\}_{n=1}^\infty$ such that $E \subseteq 
    \bigcup_{n=1}^\infty J_n$ and 
    \[ \sum_{n=1}^\infty \ell(J_n) \leq m^*(E) + \eps. \] 
    Without loss of generality, we can also partition these intervals 
    $\{J_n\}_{n=1}^\infty$ into subintervals so that $\ell(J_n) < \delta$
    for all $n \in \N$. Now, choose open intervals $I_n \supseteq J_n$ 
    such that $\ell(I_n) \leq \max\{\delta, \ell(J_n) + \eps/2^n\}$. 
    We see that $E \subseteq \bigcup_{n=1}^\infty I_n$ with 
    \[ \sum_{n=1}^\infty \ell(I_n) \leq \sum_{n=1}^\infty 
    \left( \ell(J_n) + \frac{\eps}{2^n} \right) \leq (m^*(E) + \eps) + \eps 
    = m^*(E) + 2\eps. \] 
    Taking infima, we see that $m^*(E)$ is at least the right hand side 
    and we are done. 
\end{pf}

Next, we show that Lebesgue outer measure really generalizes length. 

\begin{theo}{theo:2.5}
    Let $I$ be an interval with left endpoint $a$ and right endpoint $b$, where 
    $a < b \in \R$. Then 
    \[ m^*(I) = \ell(I) = b-a. \]  
\end{theo}
\begin{pf}
    We first prove this for the case where $I = [a, b]$ is a compact interval. 
    By taking $I_1 = I$ and $I_2 = \varnothing$ for $n \geq 2$, the 
    collection $\{I_n\}_{n=1}^\infty$ is a cover of $[a, b]$ by intervals, 
    which implies that 
    \[ m^*(I) \leq \sum_{n=1}^\infty \ell(I_n) = \ell(I_1) = b-a. \] 
    Thus, we have $m^*(I) \leq b-a$. We now turn to showing that 
    $m^*(I) \geq b-a$. Let $\eps > 0$. By Proposition~\ref{prop:2.4}, we can 
    find an open cover $\bigcup_{n=1}^\infty (a_n, b_n)$ of $I$ such that 
    \[ \sum_{n=1}^\infty (b_n - a_n) \leq m^*(I) + \eps. \] 
    But $I$ is compact, so we can find a finite subcover; that is, 
    there exists $N \in \N$ such that $I \subseteq \bigcup_{n=1}^N (a_n, b_n)$. 
    Without loss of generality, we may toss away any intervals $I_n$ such that 
    $I_n \cap I = \varnothing$, and reorder the $I_n$'s if necessary to get 
    $a_1 < a_2 < \cdots < a_N$. Since $I$ is an interval, it is connected. 
    Thus, the intervals $(a_n, b_n)$ must overlap, and we obtain 
    \[ \ell(I) = b-a \leq \sum_{n=1}^N (b_n - a_n) \leq 
    \sum_{n=1}^\infty (b_n - a_n) \leq m^*(I) + \eps. \] 
    But $\eps > 0$ was arbitrary, so we have $m^*(I) \geq \ell(I) = b-a$, as 
    desired. 

    We now prove the result in the case that $I = (a, b]$. For all 
    $0 < \eps < b-a$, we have $[a + \eps, b] \subseteq (a, b] \subseteq 
    [a, b]$. By the monotonicity of Lebesgue outer measure 
    (Proposition~\ref{prop:2.3}), we see that 
    \[ (b-a) - \eps = m^*[a + \eps, b] \leq m^*(a, b] \leq m^*[a, b] = b-a. \] 
    Since $\eps$ is arbitrary (subject to $0 < \eps < b-a$), we deduce that 
    $m^*(a, b] = b-a$. The cases where $I = [a, b)$ and $I = (a, b)$ are 
    proved similarly. 
\end{pf}

We have now shown that the Lebesgue outer measure is translation invariant 
and is a ``good'' generalization of length. Does countable additivity hold 
for $m^*$? The answer is of course no, because we showed that there is 
no function $m : {\cal P}(\R) \to [0, \infty]$ which simultaneously 
extends length, is translation invariant, and countably additive. 

So when can countable or even just finite additivity hold? 
We first consider some special cases. 

\begin{defn}{defn:2.6}
    A set $E \subseteq \R$ is said to have {\bf Lebesgue (outer) measure 
    zero} if $m^*(E) = 0$.
\end{defn}

The following lemma says that when the sets have Lebesgue measure zero, 
then $m^*$ is countably additive. 

\begin{lemma}{lemma:2.7}
    If $(E_n)_{n=1}^\infty$ are not necessarily disjoint sets with 
    $m^*(E_n) = 0$ for all $n \in \N$, then 
    \[ m^*\!\left( \bigcup_{n=1}^\infty E_n \right) = 
    \sum_{n=1}^\infty m^*(E_n) = 0. \] 
\end{lemma}
\begin{pf}
    By subadditivity of $m^*$, we obtain 
    \[ 0 \leq m^*\!\left( \bigcup_{n=1}^\infty E_n \right) \leq 
    \sum_{n=1}^\infty m^*(E_n) = 0. \qedhere \] 
\end{pf}

Next, we consider the notion of distance between two non-empty subsets of $\R$.

\begin{defn}{defn:2.8}
    Let $E, F \subseteq \R$ be non-empty. We define 
    \[ d(E, F) = \inf\{|x - y| : x\in E, y\in F\} \] 
    to be the {\bf distance} between $E$ and $F$. 
\end{defn}

Notice that if $E, F \subseteq \R$ satisfy $d(E, F) > 0$, then they are 
certainly disjoint. In this case, it turns out that finite additivity holds
for $m^*$.

\begin{prop}{prop:2.9}
    If $E, F \subseteq \R$ are such that $d(E, F) > 0$, then 
    \[ m^*(E \sqcup F) = m^*(E) + m^*(F). \] 
\end{prop}
\begin{pf}
    We will assume that $m^*(E), m^*(F) < \infty$. Otherwise, we get equality 
    for free by observing that $E \sqcup F \supseteq E$ and $E \sqcup F 
    \supseteq F$ and using the monotonicity of $m^*$. 
    
    By the countable subadditivity of $m^*$, we have $m^*(E \sqcup F) \leq 
    m^*(E) + m^*(F)$, so we only need to prove the other direction. 
    Let $\delta = d(E, F) > 0$, and let $\eps > 0$. Then there is a 
    covering of $E \sqcup F$ by intervals $\bigcup_{n=1}^\infty I_n$ 
    such that $\ell(I_n) < \delta$ for all $n \in \N$ and 
    \[ \sum_{n=1}^\infty \ell(I_n) < m^*(E \sqcup F) + \eps. \] 
    Without loss of generality, we can toss away any intervals $I_n$ 
    such that $I_n \cap (E \sqcup F) = \varnothing$. Moreover, the 
    restriction that $\ell(I_n) < \delta$ for all $n \in \N$ means that 
    each $I_n$ touches one of $E$ or $F$, but not both. So we can 
    partition $\{I_n\}_{n=1}^\infty$ into $\{I'_n\}_{n=1}^\infty 
    \cup \{I''_n\}_{n=1}^\infty$, where the intervals $I'_n$ only touch 
    $E$ and the intervals $I''_n$ only touch $F$. Observe now that 
    $\{I'_n\}_{n=1}^\infty$ is a covering of $E$ and 
    $\{I''_n\}_{n=1}^\infty$ is a covering of $F$, so we obtain 
    \[ m^*(E) + m^*(F) \leq \sum_{n=1}^\infty \ell(I'_n) + \sum_{n=1}^\infty 
    \ell(I''_n) \leq \sum_{n=1}^\infty \ell(I_n) < m^*(E \sqcup F) + \eps. \] 
    Since $\eps > 0$ was arbitrary, this completes the proof. 
\end{pf}

\begin{cor}{cor:2.10}
    If $K_1, \dots, K_n \subseteq \R$ are pairwise disjoint compact sets, then 
    \[ m^*\!\left( \bigsqcup_{i=1}^n K_i \right) = \sum_{i=1}^n m^*(K_i). \] 
\end{cor}
\begin{pf}
    Observe that $d(E, F) > 0$ when $E$ and $F$ are compact with $E \cap F 
    = \varnothing$. So this result follows by induction and applying 
    Proposition~\ref{prop:2.9}. 
\end{pf}

\subsection{Lebesgue Measure and Lebesgue Measurable Sets}\label{subsec:2.3}
Our goal is to find a {\bf large} class of subsets ${\cal L} \subseteq 
{\cal P}(\R)$ so that countable additivity of $m^*$ holds for ${\cal L}$. 
We want ${\cal L}$ to contain all intervals, closed sets, open sets, and 
anything else that can be built from them by countable unions and intersections. 
In other words, we ideally want ${\cal L}$ to be a so-called $\sigma$-algebra, 
which we define below.

\begin{defn}{defn:2.11}
    Let $X$ be a non-empty set. A family ${\cal M}$ of subsets of $X$ is 
    called a {\bf $\sigma$-algebra} if the following three properties hold: 
    \begin{enumerate}[(1)]
        \item $\varnothing \in {\cal M}$ and $X \in {\cal M}$; 
        \item $E \in {\cal M}$ if and only if $E^c \in {\cal M}$; and 
        \item if $\{E_n\}_{n=1}^\infty$ is a countable sequence in ${\cal M}$, 
        then $\bigcup_{n=1}^\infty E_n \in {\cal M}$. 
    \end{enumerate}
\end{defn}

\begin{remark}{remark:2.12}
    It follows immediately from the definition that $\sigma$-algebras 
    are closed under countable intersections as well. Indeed, for 
    a $\sigma$-algebra ${\cal M}$ and a countable sequence $\{E_n\}_{n=1}^\infty$ 
    in ${\cal M}$, we have 
    \[ \bigcap_{n=1}^\infty E_n = \left( \bigcup_{n=1}^\infty E_n^c \right)^{\!c} 
    \in {\cal M} \] 
    by De Morgan's law. In fact, we could use countable intersection 
    in the definition of a $\sigma$-algebra and derive countable unions from it.
\end{remark}

\begin{exmp}{exmp:2.13}
    \begin{enumerate}[(a)]

        \item Some examples of $\sigma$-algebras are ${\cal P}(X)$ and 
        ${\cal M}_E = \{\varnothing, X, E, E^c\}$ for a subset $E \subseteq X$.

        \item The $\sigma$-algebras from part (a) are not very interesting; 
        ${\cal P}(X)$ is too big and ${\cal M}_E$ is too small to work with. 
        We give a slightly more interesting example. Let ${\cal F} \subseteq 
        {\cal P}(X)$, and define 
        \[ {\cal M}_{\cal F} = \bigcap_{\substack{
            \text{$\sigma$-algebras ${\cal M}$} \\ 
            \text{on $X$ such} \\ 
            \text{that } {\cal F} \subseteq {\cal M}}} {\cal M}. \] 
        Then ${\cal M}_{\cal F}$ is also a $\sigma$-algebra on $X$. 
        In fact, it is the smallest $\sigma$-algebra containing 
        ${\cal F}$. We call ${\cal M}_{\cal F}$ the {\bf $\sigma$-algebra 
        generated by ${\cal F}$}.

        \item Let $(X, \tau)$ be a topological space. Then 
        \[ {\cal B}_X = {\cal M}_\tau \subseteq {\cal P}(X) \] 
        is called the {\bf Borel $\sigma$-algebra}, and is the $\sigma$-algebra 
        generated by open sets in $X$. 
    \end{enumerate}
\end{exmp}

With the definition of a $\sigma$-algebra out of the way, we can now
discuss what a measure on one should look like. 

\begin{defn}{defn:2.14}
    Let $X$ be a non-empty set, and let ${\cal M}$ be a $\sigma$-algebra on $X$. 
    We say that a function $\mu : {\cal M} \to [0, \infty]$ is a 
    {\bf measure on ${\cal M}$} if we have 
    \begin{enumerate}[(1)]
        \item $\mu(\varnothing) = 0$, and 
        \item if $(E_n)_{n=1}^\infty$ is a sequence of pairwise disjoint sets
        in ${\cal M}$, then 
        \[ \mu\!\left( \bigsqcup_{n=1}^\infty E_n \right) = \sum_{n=1}^\infty 
        \mu(E_n). \] 
    \end{enumerate}
\end{defn}

In particular, the second property means that the measure $\mu$ is 
countably additive on the $\sigma$-algebra ${\cal M}$. Recall that we wanted 
our large set ${\cal L}$ above to be a $\sigma$-algebra on $\R$. Therefore, 
our hope is that $m = m^*|_{\cal L}$ is a measure on ${\cal L}$.

Let us now actually construct this large set ${\cal L}$. We first begin 
with a definition. 

\begin{defn}{defn:2.15}
    A set $E \subseteq \R$ is said to satisfy {\bf Carathéodory's condition} 
    if for any $A \subseteq \R$, we have 
    \[ m^*(A) = m^*(A \cap E) + m^*(A \cap E^c). \] 
    That is, $m^*$ is additive when partitioning any set $A$ with 
    $\{E, E^c\}$. 
\end{defn}

Then, we set ${\cal L} := \{E \subseteq \R : E \text { satisfies 
Carathéodory's condition}\}$. It turns out that this choice of ${\cal L}$ 
gives us exactly what we want. Of course, we will need to check that 
${\cal L}$ is actually a rich collection of sets, but we will leave 
that for later. 

\begin{theo}[Carathéodory]{theo:2.16}
    The set ${\cal L}$ defined as above is a $\sigma$-algebra over $\R$, 
    and $m = m^*|_{\cal L}$ is a measure on ${\cal L}$. We call 
    $m$ the {\bf Lebesgue measure} on $\R$. 
\end{theo}
\begin{pf}
    We will prove both of these claims simultaneously by beginning with 
    the finite case, then lifting it to the infinite case.

    {\sc Step 0.} Recall that an {\bf algebra} of sets ${\cal A} 
    \subseteq {\cal P}(X)$ has the following properties: 
    \begin{enumerate}[(1)]
        \item $\varnothing \in {\cal A}$ and $X \in {\cal A}$;
        \item $E \in {\cal A}$ if and only if $E^c \in {\cal A}$; and 
        \item if $E, F \in {\cal A}$, then $E \cup F \in {\cal A}$.
    \end{enumerate}
    In particular, this is weaker than being a $\sigma$-algebra, 
    only being closed under finite unions (and by extension, finite 
    intersections). We show that ${\cal L}$ is an algebra of sets in $\R$.

    {\sc Proof of Step 0.}~
    \begin{enumerate}[(1)]

        \item Observe that for any $A \subseteq \R$, we have 
        \[ m^*(A) = m^*(A \cap \R) + m^*(A \cap \varnothing). \] 
        Then $\R \in {\cal L}$ and by symmetry, $\varnothing \in {\cal L}$ 
        as well.

        \item If $E \in {\cal L}$, then for any $A \subseteq \R$, we have
        \[ m^*(A) = m^*(A \cap E) + m^*(A \cap E^c), \] 
        so we see that $E^c \in {\cal L}$. 

        \item Let $E, F \in {\cal L}$ and $A \subseteq \R$. We will 
        show that $E \cap F \in {\cal L}$, and so $E \cup F = 
        (E^c \cap F^c)^c \in {\cal L}$ by De Morgan's laws. First, 
        observe that 
        \begin{align*}
            m^*(A) &= m^*(A \cap E) + m^*(A \cap E^c) \\ 
            &= m^*(A \cap E \cap F) + m^*(A \cap E \cap F^c) 
            + m^*(A \cap E^c),
        \end{align*}
        where the first equality uses the fact that $E \in {\cal L}$, 
        and the second equality comes from the fact that $F \in {\cal L}$ 
        and applying it to $m^*(A \cap E)$. The idea now is to combine 
        some terms to obtain a corresponding term $m^*(A \cap (E \cap F)^c)$. 
        Indeed, we have 
        \begin{align*}
            m^*(A \cap (E \cap F)^c) 
            &= m^*(A \cap (E^c \cup F^c)) \\ 
            &= m^*(A \cap (E^c \cup F^c) \cap E) + m^*(A \cap (E^c \cup F^c) \cap E^c) \\ 
            &= m^*(A \cap F^c \cap E) + m^*(A \cap E^c). 
        \end{align*}
        These last two terms are exactly what we had before, so combining
        everything gives us 
        \[ m^*(A) = m^*(A \cap (E \cap F)) + m^*(A \cap (E \cap F)^c). \] 
        Thus, $E \cap F \in {\cal L}$, and we conclude that ${\cal L}$ is 
        an algebra over $\R$. \hfill $\blacksquare$

    \end{enumerate}

    {\sc Step 1.} The Lebesgue outer measure $m^*$ is finitely additive 
    on ${\cal L}$. That is, if $E_1, \dots, E_n \in {\cal L}$ are pairwise 
    disjoint, then 
    \[ m^*\!\left( \bigsqcup_{i=1}^n E_i \right) = \sum_{i=1}^n m^*(E_i). \] 
    {\sc Proof of Step 1.} Let $E_1, \dots, E_n \in {\cal L}$ be 
    pairwise disjoint, and let $A \subseteq \R$. We know that 
    $\bigsqcup_{i=1}^n E_i \in {\cal L}$ since we have already shown that 
    ${\cal L}$ is an algebra. Thus, noting that $E_n \in {\cal L}$, we get
    \[ m^*\!\left(A \cap \bigsqcup_{i=1}^n E_i \right) = 
    m^*\!\left(A \cap \bigsqcup_{i=1}^n E_i \cap E_n \right) 
    + m^*\!\left( A \cap \bigsqcup_{i=1}^n E_i \cap E_n^c \right) 
    = m^*(A \cap E_n) + m^*\!\left( A \cap \bigsqcup_{i=1}^{n-1} E_i \right), \] 
    where the final equality is because the sets are pairwise disjoint. 
    Repeating this argument inductively yields
    \[ m^*\left( A \cap \bigsqcup_{i=1}^n E_i \right) = 
    \sum_{i=1}^n m^*(A \cap E_i). \] 
    Taking $A = \R$ completes the proof of this step. \hfill $\blacksquare$

    {\sc Step 2.} If $\{E_i\}_{i=1}^\infty$ is a countable family of 
    disjoint sets in ${\cal L}$, then $E = \bigsqcup_{i=1}^\infty E_i \in 
    {\cal L}$ and 
    \[ m^*(E) = \sum_{i=1}^\infty m^*(E_i). \] 
    {\sc Proof of Step 2.} It suffices to show that 
    \[ m^*(A) \geq m^*(A \cap E) + m^*(A \cap E^c) \] 
    for any $A \subseteq \R$ with $m^*(A) < \infty$. Indeed, by the 
    subadditivity of $m^*$, we always have $m^*(A) \leq 
    m^*(A \cap E) + m^*(A \cap E^c)$. Moreover, if $m^*(A) = \infty$, 
    then there is certainly equality. 

    Let $A \subseteq \R$ be such that $m^*(A) < \infty$. Let 
    $F_n = \bigsqcup_{i=1}^n E_i$, which is in ${\cal L}$ since 
    it is an algebra. Arguing as in Step 1, we have 
    \[ m^*(A) = m^*(A \cap F_n) + m^*(A \cap F_n^c) 
    = \sum_{i=1}^n m^*(A \cap E_i) + m^*(A \cap F_n^c). \] 
    Notice that $F_n \subseteq E$, so $E^c \subseteq F_n^c$, which in 
    turn implies $A \cap E^c \subseteq A \cap F_n^c$. By the 
    monotonicity of $m^*$, we get 
    \[ m^*(A \cap E^c) \leq m^*(A \cap F_n^c). \] 
    Applying this inequality to our above equation gives 
    \[ m^*(A) \geq \sum_{i=1}^n m^*(A \cap E_i) + m^*(A \cap E^c). \] 
    Since this holds for all $n \in \N$, we can take $n \to \infty$ 
    and the inequality will still hold. This means that 
    \begin{align*}
        m^*(A) &\geq \sum_{i=1}^\infty m^*(A \cap E_i) + m^*(A \cap E^c) \\ 
        &\geq m^*\!\left( \bigsqcup_{i=1}^\infty (A \cap E_i) \right) 
        + m^*(A \cap E^c) \\ 
        &= m^*(A \cap E) + m^*(A \cap E^c), 
    \end{align*}
    where the second inequality follows from the countable subadditivity of 
    $m^*$. Letting $A = E$, we deduce that 
    \[ m^*(E) \geq \sum_{i=1}^\infty m^*(E_i) + 0 \geq m^*(E), \] 
    so we have the desired equality. \hfill $\blacksquare$ 

    {\sc Step 3.} We claim that ${\cal L}$ is closed under countable unions. 
    
    {\sc Proof of Step 3.} Let $\{E_i\}_{i=1}^\infty$ be a countable 
    family of not necessarily pairwise disjoint sets in ${\cal L}$. 
    Let $E'_1 = E_1$, and for each $i \geq 2$, set $E'_i = E_i \setminus 
    (E_1 \cup \cdots \cup E_{i-1})$. Then $\{E'_i\}_{i=1}^\infty$ is a 
    countable family of disjoint sets in ${\cal L}$ with 
    \[ \bigcup_{i=1}^\infty E_i = \bigsqcup_{i=1}^\infty E'_i, \] 
    and this is in ${\cal L}$ since we showed that countable unions of 
    pairwise disjoint sets in ${\cal L}$ are also in ${\cal L}$. 
    \hfill $\blacksquare$  

    This last step shows that ${\cal L}$ is a $\sigma$-algebra over $\R$
    and Step 2 establishes that $m = m^*|_{\cal L}$ is a measure on 
    ${\cal L}$, completing the proof. 
\end{pf}

In light of this result, we say that a set satisfying Carathéodory's condition 
is a {\bf Lebesgue measurable set}. Now, let's determine which sets belong 
to ${\cal L}$. Below, we see that sets of Lebesgue measure zero are an 
easy example of Lebesgue measurable sets. 

\begin{prop}{prop:2.17}
    If $E \subseteq \R$ with $m^*(E) = 0$, then $E \in {\cal L}$.     
\end{prop}
\begin{pf}
    Let $A \subseteq \R$. Notice that $m^*(A \cap E) \leq m^*(E) = 0$
    by the monotonicity of $m^*$, so $m^*(A \cap E) = 0$. 
    Similarly, we have $m^*(A \cap E^c) \leq m^*(A)$. Therefore, we obtain  
    \[ m^*(A) \leq m^*(A \cap E) + m^*(A \cap E^c) = m^*(A \cap E^c) 
    \leq m^*(A) \] 
    and so we have the equality $m^*(A) = m^*(A \cap E) + m^*(A \cap E^c)$. 
\end{pf}

In fact, it turns out that ${\cal B}_{\R}$, the Borel $\sigma$-algebra 
over $\R$, is contained in ${\cal L}$. 

\begin{theo}{theo:2.18}
    We have ${\cal B}_{\R} \subseteq {\cal L}$. 
\end{theo}
\begin{pf}
    Note that every open set can be written as the countable union of open 
    intervals. Since ${\cal L}$ is a $\sigma$-algebra, it suffices to 
    show that any open interval $I = (a, b)$ is in ${\cal L}$, where 
    $a < b \in \R$ are finite. 

    Let $I = (a, b)$, and fix $A \subseteq \R$ such that $m^*(A) < \infty$. 
    It is enough to show that 
    \[ m^*(A) \geq m^*(A \cap I) + m^*(A \cap I^c). \] 
    Pick $n \in \N$ large enough so that $I_n = [a + \frac1n, b - \frac1n]
    \subseteq I$. Notice that 
    \[ A = (A \cap I) \sqcup (A \cap I^c) \supseteq 
    (A \cap I_n) \sqcup (A \cap I^c), \] 
    with $d(A \cap I_n, A \cap I^c) \geq \frac1n > 0$. By Proposition~\ref{prop:2.9},
    we know that $m^*$ is additive on sets with positive separation, so we have 
    \[ m^*(A) \geq m^*((A \cap I_n) \sqcup (A \cap I^c)) 
    = m^*(A \cap I_n) + m^*(A \cap I^c). \] 
    Now, we are close to the desired inequality. Note that $A \cap I = 
    (A \cap I_n) \sqcup (A \cap I \setminus I_n)$, so by the subadditivity
    and monotonicity of $m^*$, we obtain 
    \begin{align*}
        m^*(A) &\leq m^*(A \cap I_n) + m^*(A \cap I \setminus I_n) \\ 
        &\leq m^*(A \cap I_n) + m^*(I \setminus I_n) \\ 
        &= m^*(A \cap I_n) + \tfrac2n, 
    \end{align*}
    where $m^*(I \setminus I_n) = m^*((a, a+\frac1n)) + m^*((b-\frac1n, b)) 
    = \frac2n$ since the intervals are disjoint. It follows that 
    \[ \lim_{n\to\infty} m^*(A \cap I_n) = m^*(A \cap I), \] 
    and we deduce that $m^*(A) \geq m^*(A \cap I) + m^*(A \cap I^c)$ 
    as desired. 
\end{pf}

Is it the case that ${\cal L}$ is actually equal to ${\cal B}_{\R}$? The 
answer is no, but it is difficult to explicitly write down a Lebesgue 
measurable set that is not in ${\cal B}_{\R}$. Instead, we can argue using the 
cardinalities of the sets. 

We first consider the cardinality of ${\cal B}_{\R}$. Recall the following 
definitions from PMATH 351. 

\begin{defn}{defn:2.19}
    Let $E \subseteq \R$ be a set. 
    \begin{itemize}
        \item We say that $E$ is a {\bf $G_\delta$-set} if $E$ is a 
        countable intersection of open sets. 
        \item We say that $E$ is an {\bf $F_\sigma$-set} if $E$ is a 
        countable union of closed sets. 
    \end{itemize}
\end{defn}

Notice that every $G_\delta$-set and $F_\sigma$-set is also a Borel set 
since ${\cal B}_{\R}$ is a $\sigma$-algebra, which is generated by the 
open sets in $\R$ and is closed under complements, countable unions and 
intersections. Moreover, we can also iterate these operations, so for 
instance, $G_{\delta\sigma}$ contains all sets that are countable unions of 
$G_\delta$-sets.  

Every open set $U \subseteq \R$ is a countable union of open intervals 
with rational endpoints. Let ${\cal F}$ be the set of all open intervals with 
rational endpoints, and observe that ${\cal F}$ is countable. Then 
any Borel set $E \subseteq \R$ can be generated from ${\cal F}$ by iterating 
the operations 
\[ G_\delta \to G_{\delta\sigma} \to G_{\delta\sigma\delta} \to \cdots \] 
countably often. This means that $|{\cal B}_{\R}| = |\N|^{|\N|} = |\R|$. 

We now consider the cardinality of ${\cal L}$. On Assignment 2, we will 
show that there is a set $\Delta \subseteq [0, 1] \subseteq \R$ called 
the {\bf Cantor middle thirds set} such that $\Delta$ is uncountable 
and $m^*(\Delta) = 0$. By Proposition~\ref{prop:2.17}, this means that 
$\Delta \in {\cal L}$. Then monotonicity tells us that any subset 
$E \subseteq \Delta$ is also in ${\cal L}$. In particular, we have 
${\cal L} \supseteq {\cal P}(\Delta)$ and so 
\[ |{\cal L}| \geq |{\cal P}(\Delta)| = |{\cal P}(\R)| = |\R|^{|\R|} 
> |\R| = |{\cal B}_{\R}|. \] 
Thus, we get the following result. 

\begin{theo}{theo:2.20}
    We have ${\cal B}_{\R} \subsetneq {\cal L}$. 
\end{theo}

Next, we show that $m = m^*|_{\cal L}$ is the unique measure on ${\cal L}$
extending length. Before we do that, we note that every measure $\mu$ is 
subadditive as a consequence of additivity. Indeed, let $E, F \subseteq \R$. 
We first show that
\[ \mu(E \cup F) + \mu(E \cap F) = \mu(E) + \mu(F). \] 
To begin, observe that we have $\mu(E) = \mu(E \cap F) + \mu(E \cap F^c)$ and 
$\mu(F) = \mu(F \cap E) + \mu(F \cap E^c)$ with $(E \cap F^c) 
\cap (F \cap E^c) = \varnothing$. Then additivity of $\mu$ gives us 
\begin{align*}
    \mu(E) + \mu(F) &= \mu(E \cap F^c) + 2\mu(E \cap F) + \mu(F \cap E^c) \\ 
    &= \mu((E \cap F^c) \sqcup (E \cap F) \sqcup (F \cap E^c)) + \mu(E \cap F) \\ 
    &= \mu(E \cup F) + \mu(E \cap F).
\end{align*}
It easily follows from this that 
\[ \mu(E \cup F) = \mu(E) + \mu(F) - \mu(E \cap F) \leq \mu(E) + \mu(F). \] 
Next, we show that $\mu$ is monotone. Suppose that $E \subseteq F 
\subseteq \R$. Then we can write $F = (F \cap E) \sqcup (F \cap E^c) = E 
\sqcup (F \cap E^c)$, and since $\mu(E \cap F^c) \geq 0$, it follows from 
additivity that 
\[ \mu(E) \leq \mu(F \cap E^c) + \mu(E) = \mu(F). \] 

\begin{theo}[Uniqueness of Lebesgue measure]{theo:2.21}
    If $\mu : {\cal L} \to [0, \infty]$ is a measure on ${\cal L}$ such that 
    $\mu(I) = m(I) = \ell(I)$ for all open intervals $I \subseteq \R$, then 
    $\mu(E) = m(E)$ for all $E \in {\cal L}$. 
\end{theo}
\begin{pf}
    First, we deal with the case where $E \in {\cal L}$ is bounded. Then 
    there exists some $n \in \N$ such that $E \subseteq (-n, n)$. 
    By the subadditivity of $\mu$ and $m$, we have $\mu(E) \leq \mu((-n, n)) 
    = 2n$ and similarly, $m(E) \leq 2n$. 

    Let $\eps > 0$. There exist open intervals $(I_k)_{k=1}^\infty$ such that 
    $E \subseteq \bigcup_{k=1}^\infty I_k$ and 
    \[ \sum_{k=1}^\infty \ell(I_k) < m(E) + \eps. \] 
    By the monotonicity and subadditivity of $\mu$, we see that 
    \[ \mu(E) \leq \mu\!\left( \bigcup_{k=1}^\infty I_k \right) 
    \leq \sum_{k=1}^\infty \mu(I_k) = \sum_{k=1}^\infty \ell(I_k) 
    < m(E) + \eps. \] 
    Since $\eps > 0$ was arbitrary, we have $\mu(E) \leq m(E)$ for all 
    bounded sets $E \in {\cal L}$. By replacing $E$ above 
    with $(-n, n) \setminus E$ and repeating the argument, we can deduce that 
    $\mu((-n, n) \setminus E) \leq m((-n, n) \setminus E)$, so 
    \[ 2n = \mu((-n, n)) = \mu((-n, n) \setminus E) + \mu(E) 
    \leq m((-n, n) \setminus E) + m(E) = m((-n, n)) = 2n \] 
    and we have equality throughout. Hence, we have $\mu(E) = m(E)$ for all 
    bounded $E \in {\cal L}$. 

    Suppose now that $E \in {\cal L}$ is arbitrary. We can write 
    $E = \bigsqcup_{n=1}^\infty E_n$ where each $E_n \in {\cal L}$ is bounded 
    via the construction $E_n = E \cap (-n-1, n+1) \setminus (-n, n)$ for all 
    $n \in \N$. Then countable additivity of the measures assures that 
    \[ \mu(E) = \sum_{n=1}^\infty \mu(E_n) = \sum_{n=1}^\infty m(E_n) = m(E). 
    \qedhere \] 
\end{pf}

We have seen that not all Borel measurable sets are Lebesgue measurable. 
But the following structure theorem for ${\cal L}$ tells us that every 
$E \in {\cal L}$ is ``almost'' Borel measurable, up to throwing away 
sets of measure zero. 

\begin{theo}[Structure of Lebesgue measurable sets]{theo:2.22}
    Let $E \subseteq \R$. The following are equivalent: 
    \begin{enumerate}[(1)]
        \item $E$ is Lebesgue measurable. 
        \item For all $\eps > 0$, there exists an open set $V$ and closed 
        set $F$ such that $F \subseteq E \subseteq V$ and 
        $m(V \setminus F) < \eps$. 
        \item There exists an $F_\sigma$-set $A$ and a $G_\delta$-set 
        $B$ such that $A \subseteq E \subseteq B$ and $m(B \setminus A) = 0$. 
    \end{enumerate}
\end{theo}
\begin{pf}
    (1) $\Rightarrow$ (2). Let $K_n = [-n, n]$ for each $n \in \N$, and note that 
    $\R = \bigcup_{n=1}^\infty K_n$. Let $E$ be a Lebesgue measurable set 
    and let $\eps > 0$. Notice that $m(E \cap K_n) \leq m(K_n) = 2n < \infty$ 
    for all $n \in \N$. In particular, there exists an open set $V_n \subseteq \R$ 
    such that 
    \[ m(V_n \setminus (E \cap K_n)) < \frac{\eps}{2^{n+1}}. \] 
    Indeed, just take $V_n = \bigcup_{k=1}^\infty (a_k, b_k) \supseteq 
    E \cap K_n$ such that 
    \[ m(V_n) < m(E \cap K_n) + \frac{\eps}{2^{n+1}}, \] 
    then use monotonicity to get $m(V_n \setminus (E \cap K_n)) \leq m(V_n)$.

    Now, put $V = \bigcup_{n=1}^\infty V_n$. Observe that $V$ is open with 
    $E \subseteq V$, and 
    \[ m(V \setminus E) = m\!\left( \bigcup_{n=1}^\infty V_n \setminus E \right) 
    \leq m\!\left( \bigcup_{n=1}^\infty V_n \setminus (E \cap K_n) \right) 
    \leq \sum_{n=1}^\infty m(V_n \setminus (E \cap K_n)) 
    < \sum_{n=1}^\infty \frac{\eps}{2^{n+1}} = \frac{\eps}{2}. \] 
    We can repeat the same argument with $E^c$ replacing $E$. This gives us 
    an open set $W \supseteq E^c$ such that $m(W \setminus E^c) < \eps/2$. 
    Letting $F = W^c$, we see that $F$ is closed and $F \subseteq E \subseteq V$. 
    Using the additivity of Lebesgue measure gives 
    \[ m(V \setminus F) = m(V \setminus E) + m(E \setminus F) 
    = m(V \setminus E) + m(W \setminus E^c) < \frac{\eps}{2} + \frac{\eps}{2} 
    = \eps. \] 
    (2) $\Rightarrow$ (3). Applying (2) with $\eps_k = 1/k$ for all $k \in \N$, 
    we obtain closed sets $\{F_k\}_{k=1}^\infty$ and open sets $\{V_k\}_{k=1}^\infty$ 
    such that $F_k \subseteq E \subseteq V_k$ and $m(V_k \setminus F_k) 
    < 1/k$ for all $k \in \N$. Then $A = \bigcup_{k=1}^\infty F_k \subseteq E$ 
    is an $F_\sigma$-set and $B = \bigcap_{k=1}^\infty V_k \supseteq E$ is a 
    $G_\sigma$-set. We see that 
    \[ B \setminus A \subseteq V_k \setminus F_k \] 
    for all $k \in \N$. Note that $m(B \setminus A)$ is a well-defined quantity
    since $A$ and $B$ are both Borel and hence Lebesgue measurable. Then 
    monotonicity of $m$ implies that 
    \[ m(B \setminus A) \leq m(V_k \setminus F_k) < \frac1k. \] 
    Since this holds for all $k \in \N$, we in fact have $m(B \setminus A) = 0$. 

    (3) $\Rightarrow$ (1). Recall from Proposition~\ref{prop:2.17} that if 
    $W \subseteq \R$ with $m^*(W) = 0$, then $W \in {\cal L}$, and 
    monotonicity shows that every subset $W_0 \subseteq W$ is also in ${\cal L}$. 
    From (3), there exists an $F_\sigma$-set $A$ and a $G_\delta$-set $B$ 
    such that $A \subseteq E \subseteq B$. Writing $E = A \cup (E \setminus A)$, 
    notice that $A \in {\cal L}$ since it is Borel. On the other hand, 
    we have $E \setminus A \subseteq B \setminus A$ with $m(B \setminus A) = 0$, 
    so $E \setminus A \in {\cal L}$. Thus, $E$ is Lebesgue measurable as 
    ${\cal L}$ is a $\sigma$-algebra and so closed under unions. \qedhere 
\end{pf}

\subsection{Measurable Functions}\label{subsec:2.4}
Recall that when integrating a function $f : \R \to \R$ using Lebesgue's 
approach of partitioning the $y$-axis, we needed to measure the length 
of sets of the form 
\[ f^{-1}(a, b) = \{x \in \R : f(x) \in (a, b)\}. \] 
Therefore, we would like $m(f^{-1}(a, b))$ to be a well-defined quantity, or 
equivalently, we want $f^{-1}(a, b)$ to be a Lebesgue measurable set for 
any open interval $(a, b)$. 

More generally, let $f : X \to Y$ be an arbitrary function. Let $B$ and $\{B_i\}_{i=1}^\infty$ 
be subsets of $Y$. Then we have $f^{-1}(B^c) = (f^{-1}(B))^c$ and 
\[ f^{-1}\!\left( \bigcup_{i=1}^\infty B_i \right) = \bigcup_{i=1}^\infty 
f^{-1}(B_i), \] 
so the inverse image operation $f^{-1}$ respects all the $\sigma$-algebra operations. 

\newpage
In particular, every open set $U \subseteq \R$ is of the form 
\[ U = \bigcup_{n=1}^\infty (a_n, b_n). \] 
So our above condition where we want $f^{-1}(a, b) \in {\cal L}$ for any 
open interval $(a, b)$ is equivalent to having 
\[ f^{-1}(U) = \bigcup_{i=1}^\infty f^{-1}(a_n, b_n) \in {\cal L} \] 
for any open set $U \subseteq \R$. This leads us to the definition of a 
Lebesgue measurable function. 

\begin{defn}{defn:2.23}
    Let $D$ be a Lebesgue measurable set. A function $f : D \to \R$ is 
    called a {\bf Lebesgue measurable function} if for any open set 
    $U \subseteq \R$, we have 
    \[ f^{-1}(U) \in {\cal L}. \] 
\end{defn}

\begin{remark}{remark:2.24}
    This definition is reminiscent of the characterization of continuous
    functions via open sets; in fact, it is a generalization of it. 
    Suppose that $f : D \to \R$ is continuous where $D \in {\cal L}$. Then for 
    every open subset $U \subseteq \R$, we know that $f^{-1}(U)$ is relatively 
    open in $D$; that is, it is of the form $D \cap V$ where $V \subseteq \R$ 
    is open, so $f^{-1}(U) \in {\cal L}$. 
\end{remark}

Thinking more abstractly, we can generalize the above definition where the 
domain has a $\sigma$-algebra associated with it, and the codomain 
is a topological space. 

\begin{defn}{defn:2.25}
    Let $X$ be a set, and let ${\cal M}$ be a $\sigma$-algebra on $X$. 
    Let $Y$ be a topological space with the topology $\tau_Y$ of open sets. 
    Then $f : X \to Y$ is {\bf measurable} with respect to ${\cal M}$ 
    if for all open sets $U \in \tau_Y$, we have 
    \[ f^{-1}(U) \in {\cal M}. \] 
\end{defn}

\begin{prop}{prop:2.26}
    Let $f : X \to Y$ be as in the above definition. The following are equivalent:
    \begin{enumerate}[(1)]
        \item $f$ is measurable with respect to ${\cal M}$. 
        \item For all Borel sets $E \in {\cal B}_Y$, we have $f^{-1}(E) \in {\cal M}$. 
        \item If ${\cal F}$ is any collection of Borel sets in $Y$ that generates 
        ${\cal B}_Y$ (that is, ${\cal B}_Y = {\cal M}_{\cal F})$, then 
        $f^{-1}(E) \in {\cal M}$ for all $E \in {\cal F}$. 
    \end{enumerate}
\end{prop}
\begin{pf}
    We leave the proof as an exercise. Hint: Use the observation above where 
    the preimage is closed under all the operations of a $\sigma$-algebra. 
\end{pf}

\begin{prop}{prop:2.27}
    Let $D \in {\cal L}$, and let $f : D \to \R$ be a Lebesgue measurable 
    function. The following are equivalent: 
    \begin{enumerate}[(1)]
        \item $f$ is Lebesgue measurable. 
        \item $f^{-1}((\alpha, \infty)) \in {\cal L}$ for all $\alpha \in \R$. 
        \item $f^{-1}([\alpha, \infty)) \in {\cal L}$ for all $\alpha \in \R$. 
        \item $f^{-1}((-\infty, \alpha)) \in {\cal L}$ for all $\alpha \in \R$. 
        \item $f^{-1}((-\infty, \alpha]) \in {\cal L}$ for all $\alpha \in \R$. 
    \end{enumerate}
\end{prop}
\begin{pf}
    It is clear that (1) implies (2) since $(\alpha, \infty)$ is open
    for all $\alpha \in \R$. For (2) implies (3), observe that 
    \[ [\alpha, \infty) = \bigcap_{n=1}^\infty (\alpha - 1/n, \infty), \] 
    and since the inverse image preserves $\sigma$-algebra operations, 
    we obtain 
    \[ f^{-1}([\alpha, \infty)) = f^{-1}\!\left( \bigcap_{n=1}^\infty 
    (\alpha - 1/n, \infty) \right) = \bigcap_{n=1}^\infty 
    f^{-1}((\alpha - 1/n, \infty)) \in {\cal L}. \] 
    For (3) implies (4), we have $(-\infty, \alpha) = [\alpha, \infty)^c$ 
    and so 
    \[ f^{-1}((-\infty, \alpha)) = f^{-1}([\alpha, \infty)^c) 
    = (f^{-1}([\alpha, \infty)))^c \in {\cal L}. \] 
    For (4) implies (5), we can use a similar argument to (2) implies (3), 
    noting that 
    \[ (-\infty, \alpha] = \bigcap_{n=1}^\infty (-\infty, \alpha + 1/n). \] 
    Finally, for (5) implies (1), it suffices to show that $f^{-1}((a, b)) 
    \in {\cal L}$ for every bounded open interval $(a, b)$. Indeed, we 
    see that 
    \[ (a, b) = (-\infty, a]^c \cap \bigcup_{n=1}^\infty (-\infty, 
    b - 1/n], \] 
    and taking the inverse image completes the proof. 
\end{pf}

Using this characterization, we can see that there are many more examples 
of measurable functions other than continuous functions. 

\begin{exmp}{exmp:2.28}
    \begin{enumerate}[(1)]
        \item For a set $E \in {\cal L}$, the characteristic function 
        $\chi_E : \R \to \{0, 1\}$ on $E$ given by 
        \[ \chi_E(x) = \begin{cases}
            1, & \text{if } x \in E, \\ 
            0, & \text{if } x \notin E 
        \end{cases} \]
        is measurable, because given $\alpha \in \R$, we have 
        \[ f^{-1}((\alpha, \infty)) = \begin{cases}
            E, & \text{if } \alpha \in (0, 1), \\ 
            \varnothing, & \text{if } \alpha \geq 1, \\ 
            \R, & \text{if } \alpha \leq 1, 
        \end{cases} \] 
        all of which are Lebesgue measurable sets. 

        \item Let $f : \R \to \R$ be a non-decreasing function. 
        Then $f$ is Lebesgue measurable since given $\alpha \in \R$, we have 
        \[ f^{-1}((\alpha, \infty)) = \begin{cases}
            \varnothing, & \text{if $f(x) \leq \alpha$ for all $x \in \R$,} \\ 
            \R, & \text{if $f(x) > \alpha$ for all $x \in \R$,} \\ 
            [a, \infty), & \text{if $f(x) > \alpha$ for all $x \geq a$,} \\ 
            (a, \infty), & \text{if $f(x) > \alpha$ for all $x > a$.}
        \end{cases} \] 
        A similar computation shows that non-increasing functions are 
        Lebesgue measurable, and thus all monotone functions are Lebesgue 
        measurable. 

        \item Let $D \in {\cal L}$. Let $Y$ and $Z$ be topological spaces. 
        Let $g : Y \to Z$ be a continuous function, and let $f : D \to Y$ 
        be Lebesgue measurable. Then $g \circ f : D \to Z$ is Lebesgue 
        measurable. Indeed, if $U \subseteq Z$ is open, then 
        $g^{-1}(U)$ is open in $Y$. From this, we see that 
        \[ f^{-1}(g^{-1}(U)) = (g \circ f)^{-1}(U) \in {\cal L}. \] 
        More generally, we can allow for $g : Y \to Z$ to be {\bf Borel 
        measurable} where $g^{-1}(U)$ is a Borel set (rather than having 
        to be open) for any open subset $U \subseteq Z$, and the same proof 
        follows. 
    \end{enumerate}
\end{exmp}

It would be nice to extend Lebesgue measurability to complex-valued functions 
$f : D \to \C$ such that by taking the typical decomposition 
$f = u + iv$ of real-valued functions $u, v : D \to \R$, we have that $f$ 
is Lebesgue measurable if and only if $u$ and $v$ are. This is true, 
but first, we will prove a more general result. 

\begin{theo}{theo:2.29}
    Let $D \in {\cal L}$, and let $u, v : D \to \R$ be Lebesgue measurable 
    functions. Let $Y$ be a topological space and let $\Phi : \R^2 
    \to Y$ be a continuous function. Then $h : D \to Y$ given by 
    \[ h(x) = \Phi(u(x), v(x)) \] 
    is also Lebesgue measurable. 
\end{theo}
\begin{pf}
    Let $f : D \to \R^2$ be defined by $f(x) = (u(x), v(x))$. Then we have 
    $h = \Phi \circ f$. Since $\Phi$ is continuous, it is enough to 
    show that $f$ is measurable due to part (3) of Example~\ref{exmp:2.28}.
    
    Recall that the topology on $\R^2$ is generated by bounded open rectangles 
    \[ U = (a, b) \times (c, d), \] 
    so it suffices to show that $f^{-1}(U) \in {\cal L}$. We see that 
    \[ f^{-1}(U) = u^{-1}((a, b)) \cap v^{-1}((c, d)). \] 
    Since $u$ and $v$ are Lebesgue measurable, both the above preimages 
    are in ${\cal L}$, so we deduce that $f^{-1}(U) \in {\cal L}$. 
\end{pf}

\begin{cor}{cor:2.30}
    Let $D \in {\cal L}$. 
    \begin{enumerate}[(a)]
        \item If $u, v : D \to \R$ is measurable, then 
        $f = u+iv : D \to \C$ is also measurable. 
        \item If $f : D \to \C$ is measurable, then $u = \text{Re}(f)$,
        $v = \text{Im}(f)$, and $|f|$ are all measurable. 
        \item If $f, g : D \to \C$ are measurable, then $\alpha f$, 
        $f + g$, and $fg$ are measurable for all $\alpha \in \C$. 
    \end{enumerate}
\end{cor}
\begin{pf}~
    \begin{enumerate}[(a)]
        \item We can consider the topological isomorphism for $\R^2 \cong \C$. 
        Taking $\Phi(u, v) = u + iv$ in Theorem~\ref{theo:2.29}, we obtain the 
        result. 
        \item Notice that $\text{Re}$, $\text{Im}$, and $|\cdot|$ are all 
        continuous. Taking the composition of $f$ with these functions and 
        applying part (3) of Example~\ref{exmp:2.28} gives the result. 
        \item For any $\alpha \in \C$, it is clear that $\alpha f$ is 
        measurable. Assume for the moment that $f$ and $g$ are real-valued 
        functions. Applying Theorem~\ref{theo:2.29} with $\Phi(s, t) 
        = s + t$ and $\Phi(s, t) = st$ which are both continuous, we 
        find that $f + g$ and $fg$ are measurable in this special case. 

        Now, suppose that $f$ and $g$ are complex-valued. Suppose that 
        $f = u + iv$ and $g = w + iy$ where $u, v, w, y$ are real-valued. 
        By (1), we see that $u, v, w, y$ are measurable. Then 
        $f + g = (u + w) + i(v + y)$ and $fg = (uw - vy) + i(uy + vw)$ 
        are also measurable by what we have just shown. \qedhere 
    \end{enumerate}
\end{pf}

\begin{defn}{defn:2.31}
    Let $f : D \to \C$ and $g : E \to \C$ be functions where $D, E \in {\cal L}$. 
    We say that $f = g$ {\bf almost everywhere (a.e.)} if the set 
    \[ \{f \neq g\} := (E\,\Delta\,D) \cap \{x \in E \cap D : f(x) \neq g(x)\} \] 
    has Lebesgue measure zero, where $E\,\Delta\,D$ denotes the symmetric 
    difference of $E$ and $D$. 
\end{defn}

We can think of $f$ and $g$ as essentially the ``same'' function, even if 
$D$ and $E$ have trivial intersection. 

\begin{prop}{prop:2.32}
    Let $D \in {\cal L}$, and let $f : D \to \C$ be a measurable function.
    Suppose that $g : E \to \C$. If $f = g$ almost everywhere, then 
    $E \in {\cal L}$ and $g$ is a measurable function satisfying 
    \[ m(f^{-1}(U)) = m(g^{-1}(U)) \] 
    for all open sets $U \subseteq \C$. 
\end{prop}
\begin{pf}
    Observe that we can write 
    \begin{align*}
        E &= \{x \in E \cap D : f(x) = g(x)\} \cup (\{f \neq g\} \cap E) \\ 
        &= (D \setminus \{f \neq g\}) \cup (\{f \neq g\} \cap E). 
    \end{align*}
    The first set is the measurable set $D$ minus the set of measure zero
    $\{f \neq g\}$ whereas the second has measure zero, so $E \in {\cal L}$. 
    Now suppose that $U \subseteq \C$ is open. We have 
    \[ g^{-1}(U) = (f^{-1}(U) \setminus \{f \neq g\}) \cup 
    (g^{-1}(U) \cap \{f \neq g\}), \] 
    where the first set above is the set of elements in the preimage such that 
    $f$ and $g$ agree, and the second set consists of the elements where 
    either $f$ and $g$ disagree or $f$ is not defined where $g$ is. 
    Noting that $f^{-1}(U) \in {\cal L}$, it follows from an analogous 
    argument to above that $g^{-1}(U) \in {\cal L}$. Since $\{f \neq g\}$
    is null, it is easy to see that $m(g^{-1}(U)) = m(f^{-1}(U))$. 
\end{pf}

The previous proposition tells us that when $f = g$ almost everywhere, then 
$f = g$ in the eyes of Lebesgue measure. When $f$ is measurable, we are 
free to redefine $f$ on measure zero sets, or extend the domain by $f$ 
up to sets of measure zero. These operations do not ``materially'' change $f$. 

To end this section, we say some words on extended real-valued functions. 
We let $[-\infty, \infty] = \R \cup \{\pm\infty\}$ be the {\bf extended 
real numbers}, and we want to consider functions $f : D \to [-\infty, 
\infty]$ for some $D \in {\cal L}$. We will call $f$ {\bf Lebesgue 
measurable} if $f^{-1}((\alpha, \infty]) \in {\cal L}$ for all 
$\alpha \in \R$. In this case, we have $f^{-1}(\{\infty\}) \in {\cal L}$ 
and $f^{-1}(\{-\infty\}) \in {\cal L}$. Note that this notion only makes 
sense for real-valued functions, and not in $\C$. 

\subsection{Limits of Measurable Functions} \label{subsec:2.5}
Let $D \subseteq \R$, and let $(f_n)_{n=1}^\infty$ be a sequence of 
continuous functions defined on $D$. If $f_n \to f$ pointwise, recall 
that there is no guarantee that $f$ is continuous. For example, 
letting $f_n(x) = x^n$ on $[0, 1]$, then $f_n \to f$ where 
$f(x) = 0$ for $x \in [0, 1)$ and $f(1) = 1$. We can only guarantee 
that $f = \lim_{n\to\infty} f_n$ is continuous when the convergence 
is uniform (which is equivalent to convergence in the supremum norm). 

It turns out that measurability is a little nicer in this regard
and does not require uniform convergence. Recall that 
$\sup_{n\in\N} f_n$ and $\inf_{n\in\N} f_n$ are defined such that 
$(\sup_{n\in\N} f_n)(x) = \sup_{n\in\N} (f_n(x))$ and 
$(\inf_{n\in\N} f_n)(x) = \inf_{n\in\N} (f_n(x))$.

\begin{theo}{theo:2.33}
    Let $(f_n)_{n=1}^\infty$ be a sequence of measurable extended real-valued 
    functions defined on some $D \in {\cal L}$. Then $\sup_{n\in\N} f_n$ 
    and $\inf_{n\in\N} f_n$ are both measurable functions. 
\end{theo}
\begin{pf}
    Let $g = \sup_{n\in\N} f_n$ and take $\alpha \in \R$. Notice that 
    $g(x) > \alpha$ if and only if there exists some $n \in \N$ such that 
    $f_n(x) > \alpha$. Since each $f_n$ is measurable, we obtain
    \[ g^{-1}((\alpha, \infty]) = \bigcup_{n=1}^\infty f_n^{-1}((\alpha, 
    \infty]) \in {\cal L}, \] 
    and thus $g$ is measurable. It follows that $\inf_{n\in\N} f_n = 
    -\sup_{n\in\N} (-f_n)$ is also measurable. 
\end{pf}

\begin{cor}{cor:2.34}
    Let $D \in {\cal L}$. 
    \begin{enumerate}[(a)]
        \item Let $f, g : D \to [-\infty, \infty]$ be measurable 
        functions. Then $\max\{f, g\}$ and $\min\{f, g\}$ are 
        both measurable. 
        \item Let $f : D \to [-\infty, \infty]$ be a function. Define 
        $f^+ = \max\{f, 0\}$ and $f^- = -\min\{f, 0\}$, and observe that 
        $f = f^+ - f^-$. Then $f$ is measurable if and only if $f^+$ and 
        $f^-$ are both measurable. 
        \item Let $(f_n)_{n=1}^\infty$ be a sequence of real or complex
        valued functions on $D$. Suppose that $\lim_{n\to\infty} 
        f_n(x) = f(x)$ exists for all $x \in D$. Then $f$ is measurable. 
    \end{enumerate}
\end{cor}
\begin{pf}~
    \begin{enumerate}[(a)]
        \item Take the sequence $f_1 = f$ and $f_n = g$ for all $n \geq 2$, 
        then apply Theorem~\ref{theo:2.33} by observing that 
        $\max\{f, g\} = \sup_{n\in\N} f_n$ and $\min\{f, g\} = 
        \inf_{n\in\N} f_n$. 
        \item The forward direction follows from (a). The backwards 
        direction follows from part (c) of Corollary~\ref{cor:2.30}. 
        \item Let $(f_n)_{n=1}^\infty$ be a sequence of real valued 
        functions on $D$. Observe that 
        \[ \limsup_{n\to\infty} f_n = \inf_{n\in\N} \left( \sup_{k\geq n}
        f_k \right) \] 
        is measurable by applying Theorem~\ref{theo:2.33}, and so is 
        $\liminf_{n\to\infty} f_n = \sup_{n\in\N} (\inf_{k\geq n} f_k)$. 
        Since $f_n \to f$ pointwise, we see that  $f = \limsup_{n\to\infty} f_n 
        = \liminf_{n\to\infty} f_n$ is measurable. 
        
        For the complex valued case, we can write $f_n = u_n + iv_n$ 
        where $u_n$ and $v_n$ are real valued for each $n \in \N$. 
        Then $u_n \to u$ and $v_n \to v$ with $u$ and $v$ being measurable, 
        and thus $f = u + iv$ is measurable by Corollary~\ref{cor:2.30}. \qedhere 
    \end{enumerate}
\end{pf}

Corollary~\ref{cor:2.34} tells us that measurable functions are closed under 
pointwise limits. We now discuss some other modes of convergence. 

\begin{defn}{defn:2.35}
    Let $D \in {\cal L}$. A sequence of measurable functions $(f_n)_{n=1}^\infty$
    defined on $D$ converges {\bf pointwise almost everywhere} to a function 
    $f$ on $D$ if 
    \[ \lim_{n\to\infty} f_n(x) = f(x) \] 
    for almost every $x \in D$. We write $f_n \to f$ almost everywhere (a.e.). 
\end{defn}

Note that if we do not have pointwise convergence at a point, it 
either converges to the wrong value or the sequence doesn't converge 
at the point. The following lemma tells us that if $f_n \to f$ a.e., 
then it doesn't really matter what happens for points $x \in D$ where
$f(x) \neq \lim_{n\to\infty} f_n(x)$. 

\begin{lemma}{lemma:2.36}
    Let $D \in {\cal L}$. If $(f_n)_{n=1}^\infty$ is a sequence of measurable 
    functions on $D$ and $f_n \to f$ a.e., then $f$ is measurable. 
\end{lemma}
\begin{pf}
    Define the function 
    \[ g(x) = \begin{cases}
        \lim_{n\to\infty} f_n(x), & \text{if the limit exists,} \\ 
        f(x), & \text{otherwise.} 
    \end{cases} \] 
    Then by construction, we have $g = f$ almost everywhere. By 
    Proposition~\ref{prop:2.32}, $f$ is measurable. 
\end{pf}

Next, we talk about uniform convergence and almost uniform convergence. 

\begin{defn}{defn:2.37}
    Suppose that $(f_n)_{n=1}^\infty$ and $f$ are functions from $D$ 
    to $\C$, $\R$, or $[-\infty, \infty]$. 
    \begin{itemize}
        \item We say that $f_n \to f$ {\bf uniformly} if for all $\eps > 0$, 
        there exists $N \in \N$ such that $n \geq N$ implies
        \[ |f_n(x) - f(x)| < \eps. \] 
        \item We say that $f_n \to f$ {\bf almost uniformly} if for all 
        $\eps > 0$, there exists a measurable subset $E \subseteq D$ such that 
        $m(E) < \eps$ and $f_n \to f$ uniformly on $D \setminus E$. 
    \end{itemize}
\end{defn}

It is clear that uniform convergence implies almost uniform convergence 
and pointwise convergence, while pointwise convergence implies 
pointwise almost everywhere convergence. Our goal is to determine the 
relationship between almost uniform convergence and pointwise almost 
everywhere convergence. 

\begin{lemma}{lemma:2.38}
    If $f_n \to f$ almost uniformly, then $f_n \to f$ a.e. pointwise. 
\end{lemma}
\begin{pf}
    For each $m \in \N$, there exists a measurable subset $E_m \subseteq D$ 
    with $m(E_m) < 1/m$ such that $f_m \to f$ uniformly on $D \setminus E_m$. 
    Let $E = \bigcap_{m=1}^\infty E_m$. Using the continuity from above 
    property of the Lebesgue measure from Assignment 2, we obtain 
    \[ m(E) = \lim_{N\to\infty} m\!\left( \bigcap_{m=1}^N E_m \right) 
    \leq \lim_{N\to\infty} m(E_N) = 0 \] 
    since $(\bigcap_{m=1}^N E_m)_{N=1}^\infty$ is a decreasing sequence of sets. 
    Therefore, $E$ is a null set. 

    To show that $f_n \to f$ a.e. pointwise, it only remains to show that 
    $f_n \to f$ for all $x \in D \setminus E$. Observe that 
    \[ D \setminus E = D \setminus \bigcap_{m=1}^\infty E_m = 
    \bigcup_{m=1}^\infty (D \setminus E_m). \] 
    In particular, if $x \in D \setminus E$, then there exists some $m \in \N$ 
    such that $x \in D \setminus E_m$. Then $f_n \to f$ uniformly there, 
    which implies pointwise convergence. 
\end{pf}

What about the converse? Does almost everywhere convergence imply 
almost uniform convergence? It turns out that this is true under the 
assumption that $m(D) < \infty$. This is known as Egorov's Theorem. 

\begin{theo}[Egorov's Theorem]{theo:2.39}
    Suppose that $D \in {\cal L}$ is such that $m(D) < \infty$. 
    Let $(f_n)_{n=1}^\infty$ be a sequence of measurable functions 
    $f_n : D \to \C$, and let $f : D \to \C$. If $f_n \to f$ pointwise 
    a.e. on $D$, then $f_n \to f$ almost uniformly on $D$. 
\end{theo}
\begin{pf}
    We may assume without loss of generality that $f_n \to f$ pointwise 
    by replacing $D$ with the set $\{x \in D : \lim_{n\to\infty} f_n(x) =
    f(x)\}$; we are only dropping a set of measure zero. Given $n, k \in \N$, 
    we let 
    \[ E(n, k) = \bigcup_{m\geq n} \{x \in D : |f_m(x) - f(x)| \geq 1/k\}. \] 
    We can think of this as the set of points were $f_m$ is doing a bad 
    job of converging to $f$. 
    
    It is clear that $E(n, k)$ is measurable for all $n, k \in \N$. 
    Now, keep $k \in \N$ fixed. Then $(E(n, k))_{n=1}^\infty$ is a decreasing 
    sequence of sets with $\bigcap_{n=1}^\infty E(n, k) = \varnothing$. 
    Indeed, if there existed an element $x \in \bigcap_{n=1}^\infty E(n, k)$, 
    then for all $n \in \N$, there exists $m \in \N$ such that 
    $|f_m(x) - f(x)| \geq 1/k$. This contradicts our assumption that 
    $f_n \to f$ for all $x \in D$. By the continuity from above 
    of Lebesgue measure from Assignment 2, we have 
    \[ \lim_{n\to\infty} m(E(n, k)) = m\!\left( \bigcap_{n=1}^\infty E(n, k) 
    \right) = 0. \] 
    Note that we are using the fact that $m(E(1, n)) \leq m(D) < \infty$ here! 
    
    Let $\eps > 0$. Choose a subsequence $(n_k)_{k=1}^\infty$ such that 
    \[ m(E(n_k, k)) < \frac{\eps}{2^k}, \] 
    and set $E = \bigcup_{k=1}^\infty E(n_k, k)$. Then we have 
    \[ m(E) \leq \sum_{k=1}^\infty m(E(n_k, k)) < \sum_{k=1}^\infty 
    \frac{\eps}{2^k} = \eps. \] 
    For all $x \in D \setminus E$, we see that $x \in D \setminus E(n_k, k)$
    for all $k \in \N$. Then $|f_m(x) - f(x)| < 1/k$ for all $k \in \N$ 
    and $m \geq n_k$, implying that $f_m \to f$ uniformly on $D \setminus E$. 
\end{pf}

\begin{exmp}{exmp:2.40}
    For a simple example of Egorov's Theorem, consider the sequence of functions 
    $f_n(x) = x^n$ on $[0, 1]$. Recall that $f_n \to f$ pointwise on $[0, 1]$ where 
    $f(x) = 0$ on $[0, 1)$ and $f(1) = 1$. But for all $\eps > 0$, we see that 
    $f_n \to f$ uniformly on $[0, 1-\eps]$. In particular, we have $f_n \to f$ 
    almost uniformly on $[0, 1]$. 
\end{exmp}

\begin{remark}{remark:2.41}
    The assumption that $m(D) < \infty$ is necessary for Egorov's Theorem. 
    Take $D = \R$, and define the sequence of functions $f_n = \chi_{[-n, n]}$. 
    Then $\lim_{n\to\infty} f_n(x) = 1$ for all $x \in \R$. 

    We claim that $f_n \nrightarrow 1$ almost uniformly. Indeed, if 
    $E \subseteq \R$ satisfies $m(E) < \eps < \infty$, then $m(E^c) = \infty$. 
    That is, $E^c$ is unbounded. So there exists a sequence of points 
    $x_k \in E^c$ such that $x_k > k$ for all $k \in \N$. This gives us 
    \[ \sup_{x\in E^c} |f_k(x) - 1| \geq \sup_{k\in\N} |f_k(x_k) - 1| = 1, \] 
    and thus 
    \[ \limsup_{k\to\infty} \|f_k - 1\|_\infty = 1. \] 
    We conclude that $f_n \nrightarrow 1$ uniformly on $E^c = \R \setminus E$.
\end{remark}

We now consider simple functions, and see how measurable functions are 
well approximated by them. This will allow us to jump into Lebesgue integration.

\begin{defn}{defn:2.42}
    Let $D \in {\cal L}$. A function $f : D \to \C$ is {\bf simple} if 
    its range is finite. That is, we have 
    \[ \text{range}(f) = \{\alpha_1, \dots, \alpha_n\} \subseteq \C \] 
    where the $\alpha_i$ are distinct elements. 
\end{defn}

Notice that if we let $E_i = f^{-1}(\{\alpha_i\})$, then $D = 
\bigsqcup_{i=1}^n E_i$ and $f = \sum_{i=1}^n \alpha_i \chi_{E_i}$. 
Moreover, $f$ is measurable if and only if $E_i$ is measurable for all 
$1 \leq i \leq n$. 

In general, there are many ways to write a simple function as a linear 
combination of characteristic functions $\chi_{A_i}$, where $A_i \in 
{\cal L}$. But the form $f = \sum_{i=1}^n \alpha \chi_{E_i}$ above is really the 
most natural representation of $f$ as a linear combination of 
characteristic functions; we call it the {\bf standard representation} of $f$. 

The following theorem tells us that every measurable function is a limit 
of simple measurable functions. 

\begin{theo}{theo:2.43}
    \begin{enumerate}[(a)]
        \item Let $D \in {\cal L}$, and let $f : D \to [0, \infty]$ be a 
        measurable function. Then there exists a sequence $(f_n)_{n=1}^\infty$ 
        of non-negative simple measurable functions on $D$ such that 
        $0 \leq f_1 \leq f_2 \leq \cdots \leq f$ and for all $x \in D$, we have 
        \[ \lim_{n\to\infty} f_n(x) = \sup_{n\in\N} f_n(x) = f(x). \] 
        Moreover, for any $R > 0$, we have $f_n \to f$ uniformly on the 
        set $E_R = \{x \in D : f(x) \leq R\} \subseteq D$. 
        \item Let $D \in {\cal L}$ and let $f : D \to \C$ be measurable. 
        Then there exists a sequence of simple measurable functions 
        $f_n : D \to \C$ such that 
        \[ f(x) = \lim_{n\to\infty} f_n(x) \] 
        for all $x \in D$ and $0 \leq |f_1| \leq |f_2| \leq \cdots \leq |f|$. 
        Moreover, for all $R > 0$, we have $f_n \to f$ uniformly on the set 
        $E_R = \{x \in D : |f_n(x)| \leq R\} \subseteq D$. 
    \end{enumerate}
\end{theo}
\begin{pf}
    For part (a), we first prove a special case. Let $g : [0, \infty] 
    \to [0, \infty]$ be defined by $g(t) = t$. We want to approximate 
    $g$ by non-decreasing piecewise constant functions $g_n$. 
    
    Fix $n \in \N$. For $0 \leq t < \infty$, there is a unique $k \in \N$ 
    such that $t \in [\frac{k}{2^n}, \frac{k+1}{2^n}]$. We write 
    $k = k(t, n)$. Then, we define $g_n : [0, \infty] \to [0, \infty)$ by 
    \[ g_n(t) = \begin{cases}
        k(n, t)/2^n, & \text{if } t < n, \\ 
        n, & \text{if } n \leq t \leq \infty. 
    \end{cases} \]
    Observe that for all $n \in \N$, $g_n$ is simple and non-decreasing. 
    Moreover, for all $t \in [0, \infty]$, we have 
    \[ g_n(t) \leq g_{n+1}(t) \leq g(t) = t. \] 
    One more thing to note is that for $t \in [0, n]$, we have 
    \[ g_n(t) = \frac{k}{2^n} \leq t \leq \frac{k}{2^n} + \frac{1}{2^n}
    = g_n(t) + \frac{1}{2^n}. \]  
    This implies that $|g_n(t) - g(t)| \leq 2^{-n}$, so by picking 
    $n$ large enough so that $n > R$, we have $g_n \to g$ uniformly on 
    the bounded interval $[0, R]$. Finally, we see that $g_n(t) \to g(t)$ 
    for all $t \in [0, \infty]$. 

    We now prove the general case of part (a). Let $f : D \to [0, \infty]$ 
    be a measurable function and define $f_n = g_n \circ f$ for all 
    $n \in \N$. Then each $f_n$ is simple as the $g_n$'s are simple. 
    Moreover, for all $x \in D$, we have 
    \[ f_n(x) = g_n(f(x)) \leq g_{n+1}(f(x)) = f_{n+1}(x), \] 
    so the $f_n$'s are increasing. We see that $f_n(x) \to f(x)$ for all 
    $x \in D$ since $g_n(t) \to t$. 

    Let $R > 0$ and suppose that $x \in E_R = \{x \in D : f(x) \leq R\}$. 
    Let $t = f(x)$. Then 
    \[ |f(x) - f_n(x)| = |t - g_n(t)| < 2^{-n} \] 
    provided that $t \leq R < n$. Thus, $f_n \to f$ uniformly on $E_R$. 

    Finally, we claim that $f_n$ is measurable for all $n \in \N$. 
    Indeed, we saw earlier that the $g_n$ are non-decreasing functions for all 
    $n \in \N$, so $g_n^{-1}((\alpha, \infty])$ is an interval for all 
    $\alpha \in \R$, say $I$. Then 
    \[ f_n^{-1}((\alpha, \infty]) = f^{-1}(g_n^{-1}((\alpha, \infty])) 
    = f^{-1}(I) \in {\cal L} \] 
    since $f$ is measurable. This completes the proof of part (a). 

    To prove part (b), we can break $f : D \to \C$ into real and imaginary parts 
    $u, v : D \to \R$ so that $f = u + iv$. We can then write these as 
    $u = u^+ - u^-$ and $v = v^+ - v^-$, which are non-negative functions on 
    $D$. We leave it as an exercise to verify the remaining details. 
\end{pf}

\subsection{Lebesgue Integration} \label{subsec:2.6}
We will iteratively define the Lebesgue integral for measurable functions $f$
on a measurable set $E \in {\cal L}$. First, we consider the Lebesgue integral 
of non-negative simple measurable functions. We then use this to define 
the integral of $f$ where $f : E \to [0, \infty]$ is measurable, and 
derive a number of consequences of the definition. 

\begin{defn}{defn:2.44}
    Let $\varphi : \R \to [0, \infty)$ be a simple and measurable function
    with range $\{\alpha_1, \dots, \alpha_n\}$. Let $\varphi = \sum_{i=1}^n 
    \alpha_i \chi_{E_i}$ be its standard form with 
    $E_i = \varphi^{-1}(\{\alpha_i\}) \in {\cal L}$. Then the 
    {\bf Lebesgue integral} of $\varphi$ is 
    \[ \int_{\R} \varphi\dd m = \int_{\R} \varphi = \int \varphi 
    := \sum_{i=1}^n \alpha_i m(E_i). \] 
\end{defn}

By convention, we will define $0 \cdot \infty = \infty \cdot 0 = 0$. For example, 
if $\alpha_i = 0$ and $m(E_i) = \infty$, then $\alpha_i m(E_i) = 0$. 

Let us now extend this definition to arbitrary non-negative measurable 
extended real-valued functions. 

\begin{defn}{defn:2.45}
    Let $f : \R \to [0, \infty]$ be a measurable function. Then we 
    define the {\bf Lebesgue integral} of $f$ to be 
    \[ \int_{\R} f\dd m = \int_{\R} f = \int f 
    := \sup\left\{ \int_{\R} \varphi \dd m : 0 \leq \varphi \leq f, 
    \text{ where $\varphi$ is simple and measurable} \right\}. \] 
    For a measurable set $E \in {\cal L}$, we define 
    \[ \int_E f\dd m = \int_{\R} f\chi_E \dd m. \] 
\end{defn}

We now list some basic properties of the Lebesgue integral which follow 
easily from the definition.

\begin{prop}{prop:2.46}
    \begin{enumerate}[(a)]
        \item If $E \in {\cal L}$ and $0 \leq f \leq g$ are measurable 
        functions, then 
        \[ \int_E f \dd m \leq \int_E g \dd m. \] 
        \item If $A \subseteq B$ are measurable sets and $f \geq 0$ is a 
        measurable function, then 
        \[ \int_A f \dd m \leq \int_B f \dd m. \] 
        \item Let $E \in {\cal L}$. If $f \geq 0$ is a measurable function 
        and $c \in [0, \infty)$, then 
        \[ \int_E cf\dd m = c \int_E f\dd m. \] 
        \item Let $E \in {\cal L}$, and suppose that $f \geq 0$ is a measurable 
        function such that $f(x) = 0$ for all $x \in E$. Then we have 
        \[ \int_E f\dd m = 0. \] 
        \item If $E \in {\cal L}$ with $m(E) = 0$, then for any measurable 
        function $f \geq 0$, we have 
        \[ \int_E f\dd m = 0. \] 
    \end{enumerate}
\end{prop}

We want to have additivity of the Lebesgue integral. This will follow from 
the Lebesgue Monotone Convergence Theorem. This is a very powerful theorem, 
and it actually says a lot more: we actually have countable additivity 
of the Lebesgue integral! 

Towards this, we will first prove some useful properties concerning 
non-negative simple measurable functions. It turns out that we can 
actually construct a measure by applying the Lebesgue integral to a 
non-negative simple measurable function, and that we have additivity 
of the Lebesgue integral when the integrands are simple functions. 

\begin{prop}{prop:2.47}
    Let $\varphi$ and $\psi$ be non-negative simple measurable functions. 
    \begin{enumerate}[(1)]
        \item For any $E \in {\cal L}$, define 
        \[ \mu(E) = \int_E \varphi\dd m = \int_{\R} \varphi\chi_E \dd m. \] 
        Then $\mu$ is a measure. 
        \item We have 
        \[ \int_{\R} (\varphi + \psi)\dd m = \int_{\R} \varphi\dd m 
        + \int_{\R} \psi\dd m. \] 
    \end{enumerate}
\end{prop}
\begin{pf}
    Let $\varphi = \sum_{i=1}^n \alpha_i \chi_{E_i}$ and $\psi = 
    \sum_{j=1}^m \beta_j \chi_{F_j}$ be the standard forms of 
    $\varphi$ and $\psi$ respectively. 
    \begin{enumerate}[(1)]
        \item For $E \in {\cal L}$, observe that 
        \[ \mu(E) = \int_{\R} \varphi\chi_E \dd m = \int_{\R} 
        \sum_{i=1}^n \alpha_i \chi_{E_i \cap E}\dd m 
        = \sum_{i=1}^m \alpha_i m(E_i \cap E). \] 
        It is clear that $\mu(\varnothing) = 0$ and $\mu(E) \geq 0$ 
        since $\varphi \geq 0$. For countable additivity, suppose that 
        $E = \bigsqcup_{k=1}^\infty A_k$ for some $A_k \in {\cal L}$. 
        Then we obtain 
        \begin{align*}
            \mu(E) &= \sum_{i=1}^n \alpha_i m\!\left( E \cap \bigsqcup_{k=1}^\infty A_k \right) 
            = \sum_{i=1}^n \alpha_i \sum_{k=1}^\infty m(E_i \cap A_k) \\
            &= \sum_{k=1}^\infty \sum_{i=1}^n \alpha_i m(E_i \cap A_k)  
            = \sum_{k=1}^\infty \mu(A_k). 
        \end{align*}
        Thus, we conclude that $\mu$ is a measure. 

        \item Let $E_{ij} = E_i \cap F_j$ for $1 \leq i \leq n$ and $1 \leq j \leq m$. 
        We see that $\R = \bigsqcup_{i,j} E_{ij}$. Note that $\varphi + \psi$ 
        is a non-negative simple measurable function, so 
        $\mu(E) = \int_E (\varphi + \psi)\dd m$ is a measure by (1). 
        This gives us 
        \[ \int_{\R} (\varphi + \psi)\dd m = \mu(\R) = 
        \sum_{i,j} \mu(E_{ij}) = \sum_{i,j} \int_{E_{ij}} (\varphi + \psi)\dd m. \] 
        Now, observe that $\varphi + \psi$ takes on value $\alpha_i + \beta_j$ 
        on each $E_{ij} = E_i \cap F_j$. It follows that 
        \begin{align*}
            \int_{\R} (\varphi + \psi)\dd m = \mu(\R) 
            &= \sum_{i,j} \int_{E_{ij}} (\varphi + \psi)\dd m \\ 
            &= \sum_{i,j} \int_{\R} (\alpha_i + \beta_j) \chi_{E_i \cap F_j}\dd m \\ 
            &= \sum_{i,j} (\alpha_i + \beta_j) m(E_i \cap F_j) \\ 
            &= \sum_{i,j} \alpha_i m(E_i \cap F_j) + \sum_{i,j} \beta_j m(E_i \cap F_j) \\ 
            &= \sum_{i,j} \int_{E_{ij}} \varphi\dd m + \sum_{i,j} \int_{E_{ij}} \psi\dd m \\ 
            &= \int_{\R} \varphi\dd m + \int_{\R} \psi\dd m, 
        \end{align*}
        where the final equality is obtained by applying (1) to $\varphi$ and 
        $\psi$ individually. \qedhere 
    \end{enumerate}
\end{pf}

We are now ready to state and prove the Lebesgue Monotone Convergence Theorem, 
which is our first big limit theorem. It tells us that the 
Lebesgue integral is much better behaved than the Riemann integral. 

\begin{theo}[Lebesgue Monotone Convergence Theorem]{theo:2.48}
    Let $(f_n)_{n=1}^\infty$ be a sequence of non-negative extended 
    real-valued measurable functions on $E \in {\cal L}$. Suppose that 
    $0 \leq f_1 \leq f_2 \leq \cdots$ and for all $x \in E$, let 
    \[ f(x) = \lim_{n\to\infty} f_n(x) = \sup_{n\in\N} f_n(x). \] 
    Then $f$ is measurable and we have 
    \[ \int_E f\dd m = \lim_{n\to\infty} \int_E f_n\dd m. \] 
\end{theo}

We state an immediate corollary of Lebesgue's Monotone Convergence Theorem, 
which gives us a concrete way of defining the Lebesgue integral of $f$. 

\begin{cor}{cor:2.49}
    Let $E \in {\cal L}$. Let $f \geq 0$ be measurable. For any sequence 
    $(\varphi_n)_{n=1}^\infty$ of non-negative simple measurable functions 
    with $0 \leq \varphi_1 \leq \varphi_2 \leq \cdots \leq f$ and 
    $\lim_{n\to\infty} \varphi_n(x) = f(x)$ for all $x \in E$, we have 
    \[ \int_E f\dd m = \lim_{n\to\infty} \int_E \varphi_n \dd m. \] 
\end{cor}

Let us now prove the theorem. 

\begin{pf}
    We already know that $f$ is measurable by Theorem~\ref{theo:2.33}.
    By assumption, we have $0 \leq f_1 \leq f_2 \leq \cdots \leq f = 
    \sup_{n\in\N} f_n$. Then for all $n \in \N$, using part (a) of 
    Proposition~\ref{prop:2.46} gives us 
    \[ 0 \leq \int_E f_n\dd m \leq \int_E f_{n+1}\dd m \leq \int_E 
    f\dd m. \] 
    From this, we obtain 
    \[ I(f) = \lim_{n\to\infty} \int_E f_n\dd m = \sup_{n\in\N} \int_E 
    f_n\dd m \leq \int_E f\dd m. \] 
    This is actually one direction of the desired equality, so 
    it only remains to show that 
    \begin{equation}\label{eq:2.1}
        I(f) \geq \int_E f_n\dd m.
    \end{equation} 
    Fix $\eps \in (0, 1)$ and a simple and measurable function $\varphi$ 
    such that $0 \leq \varphi \leq f$. We claim it suffices to show that 
    \begin{equation}\label{eq:2.2}
        I(f) \geq \eps \int_E \varphi\dd m. 
    \end{equation}
    Why is this enough? If we take $\eps \to 1$ in \eqref{eq:2.2}, 
    then this implies that $I(f) \geq \int_E \varphi\dd m$ for all 
    simple measurable functions $0 \leq \varphi \leq f$. Then taking 
    the supremum over $\varphi$ on the right hand side implies \eqref{eq:2.1}. 

    Set $E_n = \{x \in \R : f_n(x) \geq \eps\varphi(x)\}$. By definition, we 
    have $E_n \in {\cal L}$. Moreover, since $(f_n)_{n=1}^\infty$ is an 
    increasing sequence, we have $E_1 \subseteq E_2 \subseteq \cdots$ 
    with $E = \bigcup_{n=1}^\infty E_n$. Applying Proposition~\ref{prop:2.47}, 
    we see that $\mu : {\cal L} \to [0, \infty]$ defined by 
    \[ \mu(F) = \int_F \eps\varphi\dd m \] 
    is a measure. Then by the continuity of measure from below, we get 
    \[ \int_E \eps\varphi\dd m = \mu(E) = \lim_{n\to\infty} \mu(E_n) 
    = \lim_{n\to\infty} \int_{E_n} \eps\varphi\dd m. \] 
    But $\eps\varphi \leq f_n$ on $E_n$ for all $n \in \N$ by definition, 
    and thus 
    \[ \int_{E_n} \eps\varphi\dd m \leq \int_{E_n} f_n\dd m 
    \leq \int_E f_n\dd m \leq \lim_{n\to\infty} \int_E f_n\dd m = I(f). \] 
    Taking $n \to \infty$ gives us $\eqref{eq:2.2}$, which completes the proof. 
\end{pf}
