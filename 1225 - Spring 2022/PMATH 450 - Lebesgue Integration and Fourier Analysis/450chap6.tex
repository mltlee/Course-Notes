\section{Fourier Transforms on the Line} \label{sec:6}

\subsection{The Fourier Transform and the Riemann-Lebesgue Lemma} \label{subsec:6.1}
So far, we have looked at functions $f \in L^1(\T)$, which are 
$2\pi$-periodic, measurable, and integrable on $\R$. We studied their 
Fourier coefficients $(\hat f(n))_{n\in\Z} \in c_0(\Z)$, and the 
``reconstruction'' problem of recovering $f$ from ${\cal F}f = 
(\hat f(n))_{n\in\Z}$.

But what happens if $f$ is not periodic? For example, what if 
$f \in L^1(\R)$ or $f \in L^2(\R)$? Can we still represent $f$ as a 
``sum'' of complex exponentials $x \mapsto e^{inx}$ for $x \in \R$? 

There is now a big issue: the functions $x \mapsto e^{inx}$ no longer 
belong to $L^1(\R)$ or $L^2(\R)$. In particular, the set 
$\{x \mapsto e^{inx} : n \in \Z\}$ no longer forms an orthonormal 
basis for $L^2(\R)$! 

Nonetheless, for all $t \in \R$, the function $x \mapsto e^{itx}$ 
defines a bounded continuous function. So for all $f \in L^1(\R)$, the 
integrals 
\[ \int_{\R} f(x) e^{-itx}\dd x \] 
are well-defined. This leads us to the following definition. 

\begin{defn}{defn:6.1}
    Given $f \in L^1(\R)$, its {\bf Fourier transform} is the function 
    $\hat f : \R \to \C$ given by 
    \[ \hat f(t) = \int_{\R} f(x) e^{-itx}\dd x. \] 
\end{defn}

Up to a scaling factor of $2\pi$, we see that $(\hat f(t))_{t\in\R}$ 
is a continuous analogue of the collection of Fourier coefficients of $f$. 

\begin{exmp}{exmp:6.2}
    Fix $A \in \R$ and let $f = \chi_{[-A, A]}$. Then the Fourier transform 
    of $f$ is 
    \[ \hat f(t) = \int_{-A}^A e^{-itx}\dd x = \frac{1}{-it} e^{-itx} 
    \bigg|_{-A}^A = \frac{e^{iAt} - e^{-iAt}}{it} = \frac{2\sin(At)}{t}. \] 
    Observe that 
    \[ \hat f \in C_0(\R) = \{f : \R \to \C \mid f \text{ continuous and } 
    \textstyle\lim_{t\to\pm\infty} f(t) = 0\}, \] 
    but $\hat f \notin L^1(\R)$. On the other hand, we have $\hat f \in 
    L^2(\R)$ with $\|\hat f\|_2 = 2\pi\|f\|_2$. 
\end{exmp}

Later on, we'll see that these observed properties between $f$ and $\hat f$ 
are in fact generic. We now discuss some basic properties of the Fourier 
transform on $\R$. 

\begin{theo}{theo:6.3}
    Let $f \in L^1(\R)$. Let $\alpha \in \R$ and $\lambda > 0$.
    \begin{enumerate}[(a)]
        \item If $g(x) = f(x)e^{i\alpha x}$, then $\hat g(t) = \hat f(t-\alpha)$. 
        \item If $g(x) = f(x-\alpha)$, then $\hat g(t) = e^{-i\alpha t} \hat f(t)$. 
        \item If $g(x) = \overline{f(-x)}$, then $\hat g(t) = \overline{\hat f(t)}$. 
        \item If $g(x) = f(x/\lambda)$, then $\hat g(t) = \lambda \hat f(\lambda t)$. 
        \item If $g(x) = -ixf(x)$ and $g \in L^1(\R)$, then $\hat f$ is differentiable 
        and $\hat f'(t) = \hat g(t)$. 
    \end{enumerate}    
\end{theo}
\begin{pf}~
    \begin{enumerate}[(a)]
        \item We have 
        \[ \hat g(t) = \int_{\R} f(x) e^{i\alpha x} e^{-itx}\dd x = 
        \int_{\R} f(x) e^{-i(t-\alpha)x}\dd x = \hat f(t - \alpha). \] 
        \item Using the substitution $x \mapsto x + \alpha$, we get 
        \[ \hat g(t) = \int_{\R} f(x-\alpha) e^{-itx}\dd x 
        = \int_{\R} f(x) e^{-it(x+\alpha)}\dd x = e^{-i\alpha t} \hat f(t). \] 
        \item Make the substitution $x \mapsto -x$ to obtain 
        \[ \hat g(t) = \int_{\R} \overline{f(-x)} e^{-itx}\dd x 
        = \overline{\int_{\R} f(-x) e^{itx}\dd x} = 
        \overline{\int_{\R} f(x) e^{-itx}\dd x} = \overline{\hat f(t)}. \] 
        \item Letting $z = x/\lambda$, we get 
        \[ \hat g(t) = \int_{\R} f(x/\lambda) e^{-itx}\dd x 
        = \int_{\R} f(z) e^{-it\lambda z} \dd z 
        = \lambda \int_{\R} f(z) e^{-i(\lambda t)z}\dd z 
        = \lambda \hat f(\lambda t). \] 
        \item Let $(h_n)_{n=1}^\infty \subseteq \R \setminus \{0\}$ be a 
        sequence such that $h_n \to 0$. Then we have 
        \[ \frac{\hat f(t + h_n) - \hat f(t)}{h_n} 
        = \int_{\R} f(x) \left( \frac{e^{-i(t+h_n)x} - e^{-itx}}{h_n} \right) \textrm{d}x 
        = \int_{\R} f(x) e^{-itx} \left( \frac{e^{-ih_nx} - 1}{h_n} \right) \textrm{d}x. \] 
        Set $\varphi_n(x) = (e^{-ih_nx} - 1)/h_n$. It can be seen that 
        for all $x \in \R$, we have $|\varphi_n(x)| \leq |x|$ and 
        $\varphi_n(x) \to -ix$ as $n \to \infty$. This means that 
        \[ |f(x) e^{-itx} \varphi_n(x)| \leq |xf(x)| = |g(x)| \in L^1(\R), \] 
        so by the Dominated Convergence Theorem (Theorem~\ref{theo:2.58}), we get 
        \begin{align*}
            \hat f'(t) 
            &= \lim_{n\to\infty} \frac{\hat f(t + h_n) - \hat f(t)}{h_n} \\
            &= \lim_{n\to\infty} \int_{\R} f(x) e^{-itx} \varphi_n(x)\dd x \\ 
            &= \int_{\R} f(x) e^{-itx} (-ix)\dd x \\ 
            &= \int_{\R} g(x) e^{-itx}\dd x = \hat g(t). \qedhere 
        \end{align*}
    \end{enumerate}
\end{pf}

\begin{theo}[Riemann-Lebesgue Lemma]{theo:6.4}
    Let $f \in L^1(\R)$. Then $\hat f \in C_0(\R)$ and $\|\hat f\|_\infty 
    \leq \|f\|_1$. In particular, if ${\cal F} : L^1(\R) \to C_0(\R)$ 
    denotes the Fourier transform, then ${\cal F}$ is a continuous linear map. 
\end{theo}
\begin{pf}
    Let $f \in L^1(\R)$. Then for all $t \in \R$, we have 
    \[ |\hat f(t)| = \left| \int_{\R} f(x) e^{-itx}\dd x \right| \leq 
    \int_{\R} |f(x)|\dd x = \|f\|_1, \] 
    which tells us that $\|\hat f\|_\infty \leq \|f\|_1$. To see that 
    $\hat f$ is continuous, fix $t \in \R$ and take a sequence $t_n \to t$. 
    Then
    \[ |\hat f(t_n) - \hat f(t)| = \left| \int_{\R} f(x) (e^{-it_nx} - e^{-itx})\dd x \right|. \] 
    Note that $f(x)(e^{-it_nx} - e^{-itx}) \to 0$ pointwise in $x \in \R$ 
    and is dominated by $2|f| \in L^1(\R)$. It follows from the Dominated 
    Convergence Theorem that $|\hat f(t_n) - \hat f(t)| \to 0$ as $n \to \infty$. 

    Finally, to see that $\hat f \in C_0(\R)$, fix $t \neq 0$. We have 
    \begin{align*}
        \hat f(t) &= \int_{\R} f(x) e^{-itx}\dd x \tag{A} \label{eq:5.3} \\ 
        &= -\int_{\R} f(x) e^{i\pi-itx}\dd x \\ 
        &= -\int_{\R} f(x) e^{-it(x-\pi/t)}\dd x \\ 
        &= -\int_{\R} f(z + \pi/t) e^{-itz}\dd z, \tag{B} \label{eq:5.4}
    \end{align*}
    where the last equality was obtained by making the substitution $z = 
    x - \pi/t$. We see that $2\hat f(t)$ can be obtained by 
    adding the results from $\eqref{eq:5.3}$ and $\eqref{eq:5.4}$ together, 
    so we get 
    \[ 2\hat f(t) = \int_{\R} (f(x) - f(x + \pi/t)) e^{-itx}\dd x. \] 
    Note that as $|t| \to \infty$, we have $\pi/t \to 0$. It follows from 
    Exercise 57 of Chapter 18 in Carothers (which we proved on Assignment 4)
    that 
    \[ |2\hat f(t)| \leq \int_{\R} |f(x) - f(x + \pi/t)|\dd x \to 0. \qedhere \] 
\end{pf}

\subsection{Convolutions in $L^1(\R)$} \label{subsec:6.2}
The story here is essentially identical to that of $L^1(\T)$ and our 
study of convolution there. For the moment, let $f, g \in C_c(\R)$, 
and note that $C_c(\R) \subseteq L^1(\R) \cap L^2(\R) \cap L^\infty(\R)$.
We define $f \star g : \R \to \C$ by 
\begin{align*}
    (f \star g)(x) &= \int_{\R} f(y)g(x - y)\dd y 
    = \int_{\R} f(-y) g(x+y) \dd y 
    = \int_{\R} f(x-y) g(y) \dd y = (g \star f)(x), 
\end{align*}
where we made the substitutions $y \mapsto -y$ and $y \mapsto x+y$ above. 
Note that $f \star g = g \star f$ is also well-defined in the case where 
$f \in L^\infty(\R)$ and $g \in L^1(\R)$, or when both $f, g \in L^2(\R)$. 

\begin{prop}{prop:6.5}
    Let $f, g \in C_c(\R)$. Then 
    \begin{enumerate}[(1)]
        \item $f \star g = g \star f \in C_c(\R)$; 
        \item $\|f \star g\|_p \leq \|f\|_p \|g\|_1$ for $p = 1, 2, \infty$; and 
        \item $\|f \star g\|_\infty \leq \|f\|_2 \|g\|_2$. 
    \end{enumerate}
\end{prop}

We'll omit the proof of (2) as it is identical to the proofs of the 
statements for $L^1(\T)$ from Proposition~\ref{prop:5.1}. The proof 
was based on the uniform continuity of $f \in C(\T)$, and one can 
repeat the argument for $f \in C_c(\R)$ as well because $f$ is then 
uniformly continuous on $\R$. 

To prove (1) and (3), we'll first need a lemma. 

\begin{lemma}{lemma:6.6}
    Given $f : \R \to \C$ and $y \in \R$, let $f_y(x) = f(x-y)$. 
    If $f \in L^p(\R)$, then the map $y \mapsto f_y$ is uniformly 
    continuous as a map $\R \to L^p(\R)$ for all $p \in [1, \infty)$. 
    That is, for all $\eps > 0$, there exists $\delta > 0$ such that 
    $|s - t| < \delta$ implies $\|f_s - f_t\|_p < \eps$. 
\end{lemma}
\begin{pf}
    Let $\eps > 0$. Let $f \in L^p(\R)$ for $p \in [1, \infty)$, and choose 
    $g \in C_c(\R)$ such that $\|f - g\|_p < \eps$. Assume that 
    $\text{supp}(g) \subseteq [-A, A]$ for some $A > 0$. 
    By the uniform continuity of $g$, there exists $0 < \delta < A$ 
    such that $|g(s) - g(t)| < \eps$ whenever $|s - t| < \delta$. Then we 
    have 
    \[ \|g_s - g_t\|_p^p = \int_{\R} |g(x-s) - g(x-t)|^p\dd x 
    < \eps^p \int_{\text{supp}(g_s-g_t)} 1\dd x \leq \eps^p(2A + \delta) 
    < 3A \eps^p. \] 
    Since $\|\cdot\|_p$ is translation invariant, it follows that 
    \begin{align*}
        \|f_s - f_t\|_p &\leq \|f_s - g_s\|_p + \|g_s - g_t\|_p 
        + \|g_t - f_t\|_p \\ 
        &= \|f - g\|_p + \|g_s - g_t\|_p + \|f - g\|_p \\ 
        &< 2\eps + (3A)^{1/p} \eps. \qedhere 
    \end{align*} 
\end{pf}

{\sc Proof of Proposition~\ref{prop:6.5}.} Let $f, g \in C_c(\R)$. 
Note that when $|x|$ is large, we have 
\[ (f \star g)(x) = \int_{\R} f(y) g(x-y)\dd y = 0, \] 
because eventually the support of $f$ and the support of $y \mapsto g(x-y)$ 
will not overlap. That is, $f \star g$ is compactly supported. So if we 
can establish continuity, then $f \star g \in C_c(\R)$. 

Put $\check g(y) = g(-y)$. Then we see that 
\[ (f \star g)(x) = \int_{\R} f(y) g(x-y)\dd y = \int_{\R} 
f(y) \check g(y-x)\dd y = \langle \check g_x, \overline f \rangle_{L^2(\R)}. \] 
Now, suppose that $x \to x_0 \in \R$. Using Cauchy-Schwarz and 
Lemma~\ref{lemma:6.6}, we get 
\begin{align*}
    |(f \star g)(x) - (f \star g)(x_0)| 
    &= |\langle \check g_x - \check g_{x_0}, \overline f \rangle| \\
    &\leq \| \check g_x - \check g_{x_0} \|_2 \| \overline f \|_2 \to 0. 
\end{align*}
Thus, $f \star g$ is continuous and $f \star g \in C_c(\R)$. 
Moreover, for all $x \in \R$, we see from Cauchy-Schwarz that
\[ |(f \star g)(x)| = |\langle \check g_x, \overline f \rangle_{L^2(\R)}|
\leq \|\check g_x\|_2 \|\overline f\|_2 = \|g\|_2 \|f\|_2, \] 
so $\|f \star g\|_\infty \leq \|g\|_2 \|f\|_2$. \qed 

We know that $C_c(\R)$ is dense in $L^p(\R)$ for $p \in [1, \infty)$. 
Then the inequality $\|f \star g\|_1 \leq \|f\|_1 \|g\|_1$ together 
with density tells us that $f \star g$ makes sense for $f, g \in L^1(\R)$. 
In particular, we can continuously extend convolution to $L^1(\R)$, 
and $L^1(\R)$ is an algebra under convolution!  

Moreover, the inequality $\|f \star g\|_\infty \leq \|f\|_2 \|g\|_2$ 
implies that $f \star g$ is well-defined for $f, g \in L^2(\R)$ and 
$f \star g \in C_0(\R)$. Indeed, one can take sequences $f_n \to f$ 
and $g_n \to g$, then show that $f \star g$ is the uniform limit of 
$f_n \star g_n$ with an argument similar to that of Theorem~\ref{theo:5.3}.

\begin{prop}{prop:6.7}
    The Fourier transform ${\cal F} : L^1(\R) \to C_0(\R)$ which 
    maps $f \in L^1(\R)$ to ${\cal F}f = (\hat f(t))_{t\in\R}$ given by 
    \[ \hat f(t) = \int_{\R} f(x) e^{-itx}\dd x \] 
    is an algebra homomorphism. That is, for any $f, g \in L^1(\R)$, 
    we have ${\cal F}(f \star g) = {\cal F}(f) {\cal F}(g) \in C_0(\R)$. 
\end{prop}
\begin{pf}
    By the density of $C_c(\R)$ in $L^1(\R)$ and the continuity of ${\cal F} : 
    L^1(\R) \to C_0(\R)$, it suffices to prove this for $f, g \in C_c(\R)$. 
    In this case, we have for all $t \in \R$ that 
    \begin{align*}
        (\widehat{f \star g})(t) 
        &= \int_{\R} \left( \int_{\R} f(y)g(x-y)\dd y \right) e^{-itx}\dd x \\ 
        &= \int_{\R} \int_{\R} f(y) g(x-y) e^{-itx}\dd x \dd y \\ 
        &= \int_{\R} f(y) e^{-ity} \hat g(t) \dd y = \hat f(t) \hat g(t),
    \end{align*}
    where we applied Fubini for the second equality. 
\end{pf}

\subsection{Fourier Inversion Theorem} \label{subsec:6.3}
Let $f \in L^1(\R)$. 
\begin{enumerate}[(1)]
    \item Is $f$ uniquely determined by its Fourier transform 
    ${\cal F}f = (\hat f(t))_{t\in\R}$? That is, if $\hat f(t) = \hat g(t)$
    for all $t \in \R$, then do we have $f = g$ almost everywhere? 
    Equivalently, is ${\cal F} : L^1(\R) \to C_0(\R)$ injective? 
    \item If the answer to the previous question is yes, then how do we reconstruct 
    $f$ from $\hat f$? 
\end{enumerate}
Recall that if $f \in L^1(\T)$ and we assume moreover that $\hat f = 
(\hat f(n))_{n\in\Z} \in \ell^1(\Z)$ (a strong assumption), then we know 
from our study of Fourier series that 
\[ g(x) = \sum_{n\in\Z} \hat f(n) e^{inx} \] 
converges absolutely for $x \in [-\pi, \pi]$ and $f = g$ almost everywhere. 
In this special situation, $f$ is recovered almost everywhere by simply 
summing the Fourier series of $f$!

For the Fourier transform on $\R$, the ``continuous'' version of the above holds. 
This is the Fourier inversion theorem on $L^1(\R)$. 

\begin{theo}[Fourier Inversion Theorem on $L^1(\R)$]{theo:6.8}
    Let $f \in L^1(\R)$, and assume that $\hat f \in L^1(\R)$ as well. Define 
    \[ g(x) = \frac{1}{2\pi} \int_{\R} \hat f(t) e^{itx}\dd t. \] 
    Then $g \in C_0(\R)$ and $g = f$ almost everywhere. That is, for 
    almost every $x \in \R$, we have 
    \[ f(x) = \frac{1}{2\pi} \int_{\R} \hat f(t) e^{itx}\dd t. \]
\end{theo}

As an immediate consequence, we get an affirmative answer to our first 
question above. 

\begin{cor}{cor:6.9}
    The Fourier transform ${\cal F} : L^1(\R) \to C_0(\R)$ is injective.
\end{cor}
\begin{pf}
    If $f, g \in L^1(\R)$ satisfies $\hat f = \hat g$, then 
    $\widehat{f-g} = 0 \in L^1(\R)$. It follows from the Fourier inversion 
    theorem (Theorem~\ref{theo:6.8}) that 
    \[ (f - g)(x) = \frac{1}{2\pi} \int_{\R} 0 e^{itx}\dd x = 0 \] 
    for almost every $x \in \R$, so $f = g$ almost everywhere. 
\end{pf}

The second question is partially answered by the Fourier inversion theorem. 
During the proof, we will develop a technique involving Ces\`aro summation 
and summability kernels that will fully answer the second question. 

Before we get to the proof, let's see why the theorem should hold heuristically 
by giving an engineer's ``proof''. Assume that $f \in L^1(\R)$ is bounded 
and continuous, and for $A > 0$, set $f_A = f\chi_{[-A, A]}$. Then 
we have $\lim_{A\to\infty} f_A = f$ in $L^1(\R)$. If we regard $f_A$ as a
$2\pi A$-periodic function, then $f_A$ has Fourier series 
\[ \sum_{n\in\Z} c_n e^{i(n/A)x}, \] 
where the coefficients $c_n$ are 
\[ c_n = \frac{1}{2\pi A} \int_{-\pi A}^{\pi A} f_A(x) e^{-i(n/A)x} \dd x 
= \frac{1}{2\pi A} \hat f_A(n/A). \] 
Moreover, if $f$ is sufficiently nice (for example, $f \in C^1(\R)$), then 
\[ f_A(x) = \sum_{n\in\Z} c_n e^{i(n/A)x} = \frac{1}{2\pi} 
\sum_{n\in\Z} \frac{1}{A} \hat f_A(n/A) e^{i(n/A)x}. \] 
This is a Riemann sum with respect to an infinite partition of $\R$ 
with common width $1/A$ and boundary points $n/A$ for $n \in \Z$. 
Then as $A \to \infty$, we have 
\[ \frac{1}{2\pi} \sum_{n\in\Z} \frac{1}{A} \hat f_A(n/A) e^{i(n/A)x} 
\to \frac{1}{2\pi} \int_{\R} \hat f(t) e^{itx}\dd t. \] 
We saw above that $f_A \to f$, so we get 
\[ f(x) = \frac{1}{2\pi} \int_{\R} \hat f(t) e^{itx}\dd t. \] 
Of course, this is all mathematical nonsense, and we are assuming far 
too many conditions on our function $f \in L^1(\R)$. To make this rigorous, 
we will need to take a more devious approach. 

We start with a pair of functions where the Fourier inversion theorem 
can be verified directly. Let $H(t) = e^{-|t|}$. For $\lambda > 0$, 
set $H_\lambda(t) = H(\lambda t) = e^{-\lambda|t|}$, and define 
\[ h_\lambda(x) = \frac{1}{2\pi} \int_{\R} H_\lambda(t) e^{itx}\dd x. \] 

\begin{lemma}{lemma:6.10}
    For all $x \in \R$, we have 
    \[ h_\lambda(x) = \frac{\lambda}{\pi} \cdot \frac{1}{x^2 + \lambda^2}. \] 
\end{lemma}
\begin{pf}
    By directly computing the integral, we get 
    \begin{align*}
        h_{\lambda}(x) &= \frac{1}{2\pi} \int_{\R} e^{-\lambda|t|+itx}\dd t \\ 
        &= \frac{1}{2\pi} \left( \int_0^\infty e^{(-\lambda+ix)t}\dd t 
        + \int_{-\infty}^0 e^{(\lambda+ix)t} \dd t \right) \\ 
        &= \frac{1}{2\pi} \left( \frac{1}{-\lambda+ix} (-1) + \frac{1}{\lambda+ix} \right) \\ 
        &= \frac{1}{2\pi} \left( \frac{-(\lambda + ix) + (-\lambda+ix)}{-x^2-\lambda^2} \right) \\ 
        &= \frac{1}{2\pi} \cdot \frac{-2\lambda}{-x^2-\lambda^2} \\
        &= \frac{\lambda}{\pi} \cdot \frac{1}{x^2+\lambda^2}. \qedhere 
    \end{align*}
\end{pf}
Here, we see that $0 \leq H_\lambda(t) \leq 1$ and $H_\lambda(t) \to 1$ 
as $t \to 0$. Moreover, we have $h_\lambda \geq 0$ and 
\[ \|h_\lambda\|_1 = \int_{\R} h_\lambda(x) \dd x = 
\frac{1}{\pi} \lim_{R\to0} \int_{-R}^R \frac{1}{\lambda} \cdot 
\frac{1}{1 + (x/\lambda)^2}\dd x = \frac{1}{\pi} \lim_{R\to0} 
\arctan(x/\lambda) \bigg|_{-R}^R = 1. \] 
The following theorem shows us that $(h_\lambda)_{\lambda > 0}$ acts like a 
summability kernel for the Fourier transform. 

\begin{theo}{theo:6.11}
    Let $p \in [1, \infty)$. If $f \in L^p(\R)$, then 
    $\|f \star h_\lambda - f\|_p \to 0$ as $\lambda \to 0$. Moreover, if 
    $f$ is continuous at $x \in \R$ and $\|f\|_\infty < \infty$, then 
    \[ f(x) = \lim_{\lambda\to 0} (f \star h_\lambda)(x). \] 
\end{theo}
\begin{pf}
    Suppose that $f$ is continuous at $x \in \R$, and $\|f\|_\infty < \infty$. 
    Using the fact that $\|h_\lambda\|_1 = 1$ and the substitution 
    $s = y/\lambda$, we have  
    \begin{align*}
        |(f \star h_\lambda)(x) - f(x)| 
        &= \left| \int_{\R} (f(x-y) - f(x)) h_\lambda(y)\dd y \right| \\
        &\leq \int_{\R} |f(x-y) - f(x)| h_\lambda(y) \dd y \\ 
        &= \int_{\R} |f(x-y) - f(x)| \lambda h_1(y/\lambda) \dd y \\ 
        &= \int_{\R} |f(x-\lambda s) - f(x)| h_1(s)\dd s. 
    \end{align*}
    We have a bound on the integrand given by 
    \[ |f(x-\lambda s) - f(x)| h_1(s) \leq 2\|f\|_\infty h_1(s). \] 
    Moreover, by taking a sequence $\lambda_n \to 0$, we see that 
    \[ |f(x - \lambda_n s) - f(x)| h_1(s) \to 0 \] 
    since $f$ is continuous. It follows from the Dominated Convergence 
    Theorem that 
    \[ |(f \star h_\lambda)(x) - f(x)| \to 0. \] 
    We now prove the first part of the theorem. Let $\eps > 0$, 
    and choose $f_0 \in C_c(\R)$ such that $\|f - f_0\|_p < \eps$. Then 
    \[ \|f \star h_\lambda - f\|_p \leq 
    \|f \star h_\lambda - f_0 \star h_\lambda\|_p + \|f_0 \star h_\lambda 
    - f_0\|_p + \|f_0 - f\|_p < 2\eps + \|f_0 \star h_\lambda - f_0\|_p, \] 
    so it suffices to prove this for $f = f_0 \in C_c(\R)$ by density. 
    First, we have 
    \begin{align*}
        \|f \star h_\lambda - f\|_p^p 
        &= \int_{\R} \left| \int_{\R} (f(x-y) - f(x)) h_\lambda(y) \dd y 
        \right|^p \textrm{d}x \\ 
        &\leq \int_{\R} \int_{\R} |f(x-y) - f(x)|^p h_\lambda(y)\dd y \dd x. 
    \end{align*}
    The inequality is clear for $p = 1$. For $p = 2$, this is obtained by 
    applying Cauchy-Schwarz to the pair $y \mapsto (f(x-y) - f(x))(h_\lambda(y))^{1/2}$ 
    and $(h_\lambda(y))^{1/2}$, and using the fact that $\|h_\lambda\|_1 = 1$. 
    Next, by Fubini and the substitution $s = y/\lambda$, we have 
    \begin{align*}
        \|f \star h_\lambda - f\|_p^p 
        &\leq \int_{\R} \int_{\R} |f(x-y) - f(x)|^p h_\lambda(y)\dd y \dd x \\ 
        &= \int_{\R} \left( \int_{\R} |f(x-y) - f(x)|^p \dd x \right) h_\lambda(y) \dd y \\ 
        &= \int_{\R} \|f_y - f\|_p^p h_\lambda(y)\dd y \\ 
        &= \int_{\R} \|f_{\lambda s} - f\|_p^p h_1(s)\dd s. 
    \end{align*}
    Similar to the argument above, we have 
    \[ \|f_{\lambda s} - f\|_p^p h_1(s) \leq 2^p \|f\|_p^p h_1(s) \] 
    and taking a sequence $\lambda_n \to 0$ gives us
    \[ \|f_{\lambda_n s} - f\|_p^p h_1(s) \to 0, \] 
    so it follows from the Dominated Convergence Theorem that 
    \[ \|f \star h_\lambda - f\|_p^p \to 0. \qedhere \] 
\end{pf}

The following lemma is an approximate version of the Fourier inversion theorem 
in view of Theorem~\ref{theo:6.11}, noting that $H_\lambda \approx 1$ and 
$f \star h_\lambda \approx f$. 

\begin{lemma}{lemma:6.12}
    If $f \in L^1(\R)$, then for all $\lambda > 0$ and $x \in \R$, we have 
    \[ (f \star h_\lambda)(x) = \frac{1}{2\pi} \int_{\R} H_\lambda(t) 
    \hat f(t) e^{itx}\dd t. \] 
\end{lemma}
\begin{pf}
    Let $f \in L^1(\R)$. Then we have 
    \begin{align*}
        (f \star h_\lambda)(x) 
        &= \frac{1}{2\pi} \int_{\R} f(x-y) \int_{\R} H_\lambda(t) e^{ity} \dd t \dd y \\ 
        &= \frac{1}{2\pi} \int_{\R} \int_{\R} f(x-y) H_\lambda(t) e^{ity} \dd y \dd t \\ 
        &= \frac{1}{2\pi} \int_{\R} \int_{\R} f(z) H_\lambda(t) e^{it(x-z)} \dd z \dd t \\ 
        &= \frac{1}{2\pi} \int_{\R} \hat f(t) H_\lambda(t) e^{itx} \dd t, 
    \end{align*}
    where we used Fubini for the second equality and the substitution 
    $z = x - y$ for the third equality. 
    
    \newpage 
    However, we note that the use of Fubini here was not fully rigorous, 
    as Fubini applies to integrals over compact intervals. To fix this, 
    we can take a sequence $f_n \in C_c(\R)$ with $f_n \to f$ in $L^1(\R)$. 
    Then we have $(f \star h_\lambda)(x) = \lim_{n\to\infty} (f_n \star 
    h_\lambda)(x)$. Also, we see that 
    \begin{align*}
        (f_n \star h_\lambda)(x) 
        &= \frac{1}{2\pi} \int_{\R} f_n(x-y) \lim_{N\to\infty} \int_{-N}^N H_\lambda(t) e^{ity} \dd t \dd y \\ 
        &= \lim_{N\to\infty} \frac{1}{2\pi} \int_{\R} f_n(x-y) \int_{-N}^N H_\lambda(t) e^{ity} \dd t \dd y \\
        &= \lim_{N\to\infty} \int_{-N}^N H_\lambda(t) \frac{1}{2\pi} \int_{\R} f_n(x-y) e^{ity} \dd y \dd t \\ 
        &= \lim_{N\to\infty} \frac{1}{2\pi} \int_{-N}^N H_\lambda(t) \hat f_n(t) e^{itx} \dd t \\ 
        &= \frac{1}{2\pi} \int_{\R} H_\lambda(t) \hat f_n(t) e^{itx} \dd t,
    \end{align*}
    where the second and final equalities were obtained by the Dominated Convergence 
    Theorem. Then the continuity of ${\cal F} : L^1(\R) \to C_0(\R)$ tells us that 
    as $n \to \infty$, we have 
    \[ (f_n \star h_\lambda)(x) 
    = \frac{1}{2\pi} \int_{\R} H_\lambda(t) \hat f_n(t) e^{itx} \dd t
    \to \frac{1}{2\pi} \int_{\R} H_\lambda(t) \hat f(t) e^{itx} \dd t. \qedhere \] 
\end{pf}

With all these tools, the Fourier inversion theorem follows quickly. 

{\sc Proof of Theorem~\ref{theo:6.8}.} Suppose that $f \in L^1(\R)$ with 
$\hat f \in L^1(\R)$ as well. Take a sequence $\lambda_n \to 0$. 
Note that $H_{\lambda_n}(t) \hat f(t) e^{itx} \to \hat f(t) e^{itx}$ 
as $\lambda_n \to 0$, and that this is bounded above by $|\hat f| \in L^1(\R)$. 
Then by Lemma~\ref{lemma:6.12} and the Dominated Convergence Theorem, we have 
\[ (f \star h_{\lambda_n})(x) = \frac{1}{2\pi} \int_{\R} H_{\lambda_n}(t) 
\hat f(t) e^{itx} \dd x \to \frac{1}{2\pi} \int_{\R} \hat f(t) e^{itx} \dd x = g(x). \]
We also know that $f = \lim_{n\to\infty} f \star h_{\lambda_n}$, so 
we have $f = \lim_{n\to\infty} f \star h_{\lambda_n} = g$ pointwise 
almost everywhere. \qed 

\subsection{Plancherel's Theorem on the Line} \label{subsec:6.4}
Recall from Corollary~\ref{cor:4.8} that if $f \in L^2(\T) \subseteq L^1(\T)$, 
then ${\cal F}f = (\hat f(n))_{n\in\Z} \in \ell^2(\Z)$ and we have 
\[ \|f\|_{L^2(\T)}^2 = \frac{1}{2\pi} \int_{-\pi}^\pi |f(t)|^2\dd t 
= \sum_{n\in\Z} |\hat f(n)|^2 = \|\hat f\|_{\ell^2(\Z)}. \] 
In fact, ${\cal F}_{\T} : L^2(\T) \to \ell^2(\Z)$ is a unitary isomorphism. 

Does this also hold true for ${\cal F}_{\R}$, the Fourier transform on $\R$? 
One big problem: we don't have $L^2(\R) \subseteq L^1(\R)$ as in the case of 
$\T$ (nor do we have $L^1(\R) \subseteq L^2(\R)$). In particular, 
the integral 
\[ \int_{\R} f(x) e^{-itx}\dd x \] 
might not be well-defined if $f \in L^2(\R) \setminus L^1(\R)$. To fix this, 
we will restrict our attention to the dense subspace $L^1(\R) \cap L^2(\R) 
\subseteq L^2(\R)$, where the Fourier transform is well-defined. 

\begin{prop}{prop:6.13}
    Let $f \in L^1(\R) \cap L^2(\R)$, and let $\hat f \in C_0(\R)$ be its 
    Fourier transform. Then $\hat f \in L^2(\R)$ and 
    \[ \frac{1}{2\pi} \int_{\R} |\hat f(t)|^2\dd t = \int_{\R} |f(x)|^2\dd x. \] 
\end{prop}

Before we prove this, we'll first fix some notation to improve clarity, 
because unlike the case of $\T$, the Fourier transform is not a sequence 
indexed by the integers, so it is harder to distinguish. 

Given a function $x \mapsto f(x) \in L^1(\R)$, let $\hat \R$ be the copy of 
$\R$ associated with the Fourier transform $\hat f$. That is, we have 
$\hat f \in C_0(\hat \R)$, and $t \in \hat \R$ gets mapped to $\hat f(t)$.
As usual, $(L^2(\R), \|\cdot\|_{L^2(\R)})$ has norm 
\[ \|f\|_{L^2(\R)}^2 = \int_{\R} |f(x)|^2\dd x. \] 
On the other hand, $(L^2(\hat \R), \|\cdot\|_{L^2(\hat\R)})$ has norm 
\[ \|g\|_{L^2(\hat \R)}^2 = \frac{1}{2\pi} \int_{\hat \R} |g(t)|^2 \dd t. \]   
Then Proposition~\ref{prop:6.13} tells us that ${\cal F} : 
(L^1(\R) \cap L^2(\R), \|\cdot\|_{L^2(\R)}) \to (L^2(\hat \R), 
\|\cdot\|_{L^2(\hat\R)})$ is an isometry! 

\textsc{Proof of Proposition~\ref{prop:6.13}.} Fix $f \in L^1(\R) 
\cap L^2(\R)$, and define $\tilde f \in L^1(\R) \cap L^2(\R)$ by 
\[ \tilde f(x) = \overline{f(-x)}. \] 
Note that convolving functions in $L^1(\R)$ yields a function in $L^1(\R)$, 
and convolving functions in $L^2(\R)$ gives a function in $C_0(\R)$. 
In particular, if we set $g = f \star \tilde f$, then $g \in 
L^1(\R) \cap C_0(\R)$. 

Recall the functions $H_\lambda(t) = e^{-\lambda|t|}$ and 
$h_\lambda = \frac{\lambda}{\pi} \frac{1}{x^2 + \lambda^2}$ we defined 
in Section~\ref{subsec:6.3}. By Lemma~\ref{lemma:6.12}, we have 
\[ (g \star h_\lambda)(x) = \frac{1}{2\pi} \int_{\hat \R} H_\lambda(t)
\hat g(t) e^{itx} \dd t. \]
Now, set $x = 0$ and take $\lambda \to 0$. By Theorem~\ref{theo:6.11}, 
it follows that 
\[ (g \star h_\lambda)(0) \to g(0) = (f \star \tilde f)(0) 
= \int_{\R} f(y) \tilde f(-y) \dd y = \int_{\R} f(y) \overline{f(y)} \dd y 
= \|f\|_{L^2(\R)}^2. \] 
On the other hand, observe that 
\[ \hat g(t) = (\widehat{f \star \tilde f})(t) = \hat f(t) \hat{\tilde f}(t) 
= \hat f(t) \overline{\hat f(t)} = |\hat f(t)|^2, \] 
so we obtain 
\begin{align*}
    (g \star h_\lambda)(0) 
    &= \frac{1}{2\pi} \int_{\hat \R} H_\lambda(t) \hat g(t) \dd t 
    = \frac{1}{2\pi} \int_{\hat \R} H_\lambda(t) |\hat f(t)|^2\dd t. 
\end{align*}
But $|\hat f(t)|^2 \in L^1(\R)$ and $H_\lambda(t) \to 1$ pointwise as
$\lambda \to 0$, so by the Monotone Convergence Theorem, we obtain 
\[ (g \star h_\lambda)(0) = 
\frac{1}{2\pi} \int_{\hat \R} H_\lambda(t) |\hat f(t)|^2\dd t 
\to \frac{1}{2\pi} \int_{\hat \R} |\hat f(t)|^2\dd t 
= \|\hat f\|_{L^2(\hat\R)}^2. \] 
Thus, we have $\|f\|_{L^2(\R)}^2 = \|\hat f\|_{L^2(\hat\R)}^2$, as desired. \qed

Similar to our study with convolutions where we extended $\star$ 
to an operation on all of $L^1(\T)$, we now extend the Fourier transform 
to all of $L^2(\R)$. 

\begin{theo}[Plancherel's Theorem on $\R$]{theo:6.14}
    The Fourier transform ${\cal F} : L^1(\R) \cap L^2(\R) \to L^2(\R)$ 
    extends uniquely to a unitary transformation ${\cal F}_2 
    : L^2(\R) \to L^2(\R)$. Moreover, for each $f \in L^2(\R)$, we have 
    $\hat f = \lim_{A\to\infty} \varphi_A$ where 
    \[ \varphi_A(t) = \int_{-A}^A f(x) e^{-itx}\dd x, \] 
    and $f = \lim_{A\to\infty} \psi_A$ where 
    \[ \psi_A(x) = \frac{1}{2\pi} \int_{-A}^A \hat f(t) e^{itx}\dd t. \] 
    (This is the concrete way to interpret the extension of ${\cal F}$ 
    as a limit.)
\end{theo}
\begin{pf}
    Let $Y = {\cal F}(L^1(\R) \cap L^2(\R)) \subseteq L^2(\hat \R)$. 
    We already know from Proposition~\ref{prop:6.13} that 
    ${\cal F} : L^1(\R) \cap L^2(\R) \to Y$ is linear and isometric. 
    Since $\overline{L^1(\R) \cap L^2(\R)} = L^2(\R)$, we can define 
    ${\cal F}_2 : L^2(\R) \to \overline{Y}$ by setting 
    \[ {\cal F}_2(f) = \lim_{n\to\infty} {\cal F}(f_n) \] 
    for any sequence $(f_n)_{n=1}^\infty \subseteq L^1(\R) \cap L^2(\R)$ 
    such that $f_n \to f$ in $\|\cdot\|_{L^2(\R)}$. We leave it 
    as an exercise to check that ${\cal F}_2$ is well-defined and isometric. 

    Next, we check that $\overline{Y} = L^2(\hat \R)$; that is, 
    ${\cal F}(L^1(\R) \cap L^2(\R))$ is dense in $L^2(\hat \R)$. 
    Since $L^1(\R) \cap L^2(\R) \subseteq L^2(\R)$ is dense, it suffices 
    to show that any $f \in L^1(\hat \R) \cap L^2(\hat \R)$ is in $\overline{Y}$. 
    For any such $f$, we have by Lemma~\ref{lemma:6.12} that 
    \[ (f \star h_\lambda)(x) = \frac{1}{2\pi} \int_{\R} \hat f(t) e^{itx} H_\lambda(t)\dd t. \] 
    If we put $g(t) = \frac{1}{2\pi} f(-t)H_\lambda(t)$, then we have $g \in 
    L^1(\R) \cap L^2(\R)$ because $\hat f$ is bounded and continuous, 
    and $H_\lambda \in L^1(\R) \cap L^2(\R)$. Moreover, note that 
    \[ \hat g(x) = \int_{\R} g(t) e^{-itx}\dd t = 
    \frac{1}{2\pi} \int_{\R} \hat f(-t) H_\lambda(t) e^{-itx}\dd t = 
    (f \star h_\lambda)(x), \] 
    so $f \star h_\lambda \in Y$ for all $\lambda > 0$. By Theorem~\ref{theo:6.11},
    we have $\|f - f \star h_\lambda\|_2 \to 0$ as $\lambda \to 0$, 
    so $f \in \overline{Y}$. 

    For $A > 0$, let $f \in L^2(\R)$ and note that 
    \[ \varphi_A(t) = \int_{-A}^A f(x) e^{-itx}\dd x = 
    (\widehat{f\chi_{[-A, A]}})(t) \] 
    since $f\chi_{[-A, A]} \in L^1(\R) \cap L^2(\R)$ and 
    $f\chi_{[-A, A]} \to f$ in $L^2(\R)$ as $A \to \infty$. Then we get 
    \[ \|\hat f - \varphi_A\|_{L^2(\hat\R)} = 
    \|\hat f - \widehat{f\chi_{[-A, A]}}\|_{L^2(\hat\R)} = 
    \|f - f\chi_A\|_{L^2(\R)} \to 0. \] 
    Similarly, we have 
    \[ \psi_A(x) = \frac{1}{2\pi} \int_{-A}^A \hat f(t) e^{itx} \dd t = 
    \widehat{\hat f \chi_{[-A, A]}}(-x). \] 
    Since $\hat f\chi_{[-A, A]} \to \hat f$ in $L^2(\hat \R)$ as $A \to \infty$, 
    we have by symmetry that 
    \[ \psi_A \to \hat{\hat{f}}(-x) = f. \qedhere \] 
\end{pf}

\subsection{Applications of Plancherel's Theorem} \label{subsec:6.5}
Suppose that we want to evaluate the integral 
\[ I = \int_{\R} \left( \frac{\sin t}{t} \right)^{\!2}\textrm{d}t. \] 
This can be done using methods from complex analysis: the integrand has a 
pole at $0$, so we just need to compute the residue there. But we can 
also evaluate this as an application of Plancherel's Theorem. 

In Example~\ref{exmp:6.2}, we showed that for $A > 0$ and 
$f = \chi_{[-A, A]} \in L^1(\R) \cap L^2(\R)$, we have 
\[ \hat f(t) = \frac{2\sin(At)}{t}. \] 
By Plancherel's Theorem, we know that 
\[ \int_{\R} |f(x)|^2 \dd x = \frac{1}{2\pi} |\hat f(t)|^2 \dd t. \] 
Note that $f(x)^2 = \chi_{[-A, A]}(x)^2 = \chi_{[-A, A]}(x) = f(x)$, 
so the left-hand side is equal to 
\[ \int_{\R} |f(x)|^2 \dd x = \int_{-A}^A 1\dd x = 2A, \] 
while the right-hand side is 
\[ \frac{1}{2\pi} |\hat f(t)|^2 \dd t 
= \frac{1}{2\pi} \int_{\R} \left( \frac{2\sin(At)}{t} \right)^{\!2} \textrm{d}t 
= \frac{2}{\pi} \int_{\R} \left( \frac{\sin(At)}{t} \right)^{\!2} \textrm{d}t. \] 
By plugging in $A = 1$, we find that $I = \pi$. 

We also give a nice application of Plancherel's Theorem in representation 
theory of groups. Recall that for $\alpha \in \R$ and $f \in L^2(\R)$, 
we write $f_\alpha(x) = f(x - \alpha)$ and we have $f_\alpha \in L^2(\R)$. 
In the language of representation theory, we say that $\R$ 
acts on $L^2(\R)$ by translation. 

\begin{defn}{defn:6.15}
    Let $M \subseteq L^2(\R)$ be a closed subspace. We say that $M$ is
    {\bf translation invariant} if for all $\alpha \in \R$ and $f \in M$, 
    we also have $f_\alpha \in M$. 
\end{defn}

We would like to classify all the closed, translation invariant subspaces 
$M \subseteq L^2(\R)$. We have trivial examples $M = L^2(\R)$ and $M = \{0\}$. 
But it turns out there are many more, and we can describe all of them 
with the help of Plancherel's Theorem! 

\begin{theo}{theo:6.16}
    Let $E \subseteq \hat \R$ be a measurable set, and let 
    \[ M_E = \{f \in L^2(\R) \mid \hat f \chi_E = 0\}. \] 
    Then $M_E$ is a closed, translation invariant subspace. Moreover, 
    every closed, translation invariant subspace $M \subseteq L^2(\R)$ 
    is of the form $M = M_E$ for some measurable set $E \subseteq \hat \R$. 
    Also, for measurable sets $E, F \subseteq \hat \R$, we have 
    $M_E = M_F$ if and only if $m(E\,\Delta\,F) = 0$, if and only if 
    $\chi_E = \chi_F$ almost everywhere. 
\end{theo}
\begin{pf}
    First, suppose that $E \subseteq \hat \R$ is measurable. Let 
    $\alpha \in \R$, and let $e_\alpha(t) = e^{i\alpha t}$. Recall that 
    $\hat f_\alpha(t) = e^{-i\alpha t} \hat f(t)$, so $\hat f_\alpha 
    = e_{-\alpha} \hat f$. Now, consider 
    \[ M_E = \{f \in L^2(\R) \mid \hat f \chi_E = 0\}
    = \{f \in L^2(\R) \mid \hat f = 0 \text{ almost everywhere on $E$}\}. \] 
    It is clear that $M_E = \overline{M_E}$ is a closed subspace. Also, 
    we have $\hat f = 0$ almost everywhere on $E$ if and only if 
    $e_\alpha \hat f = 0$ everywhere on $E$, if and only if 
    $\hat f_\alpha = 0$ almost everywhere on $E$ (using the fact that 
    $|e_\alpha(t)| = 1$). Thus, $M_E$ is translation invariant. 

    Conversely, let $M \subseteq L^2(\R)$ be a closed, translation invariant
    subspace. We want to show that $M = M_E$ for some measurable set 
    $E \subseteq \hat \R$.  

    Let $\hat M = {\cal F}(M) \subseteq L^2(\R)$ be the image of $M$ under the 
    Fourier transform. Since ${\cal F}$ is unitary, we see that $\hat M$ 
    is a closed subspace. Also, since $\hat f_\alpha = e_{-\alpha} \hat f$, 
    we have that $\hat M$ is invariant under multiplication by $e_\alpha$. 
    
    We now use a standard fact about Hilbert spaces. Let $K \subseteq H$ 
    be a closed subspace of a Hilbert space $H$. Then for each 
    $f \in H$, there exists a unique $Pf \in K$ such that $f - Pf 
    \perp K$; that is, $H \cong K \oplus K^\perp$ where each $f \in H$ 
    has the unique decomposition 
    \[ f = Pf + (1 - P)f. \] 
    The map $P : H \to K$ is called the {\bf orthogonal projection} of $H$ 
    onto $K$. 

    Now, consider the orthogonal projection $P : L^2(\hat\R) \to \hat M$.
    Then for all $f, g \in L^2(\hat \R)$ and $\alpha \in \R$, we have 
    \[ \langle f - Pf, e_\alpha P_g \rangle_{L^2(\hat\R)} = 
    \frac{1}{2\pi} \int_{-\pi}^\pi (f(t) - Pf(t)) e^{-i\alpha t} 
    \overline{P_g(t)}\dd t = 0. \] 
    This means that $(f - Pf) \overline{Pg} \in L^1(\hat \R)$ has $0$ 
    as its Fourier transform. Then $(f - Pf) \overline{Pg} = 0$ almost 
    everywhere, and hence $(f - Pf) Pg = 0$ almost everywhere as well. 
    So for all $f, g \in L^2(\hat \R)$, we have 
    \[ f(Pg) = PfPg = (Pf)g, \] 
    where the last equality was obtained by exchanging the roles of $f$ and $g$. 

    Take $g \in L^2(\hat \R)$ such that $g(t) > 0$ for all $t \in \hat \R$. 
    Put $\varphi = (Pg)/g$, which is a measurable function. Then 
    for all $f \in L^2(\hat \R)$, we have 
    \[ Pf = \frac{Pg}{g} f = \varphi f \] 
    by using the relation $f(Pg) = (Pf)g$ above. Since $P^2 f = Pf$, 
    we get $\varphi^2 f = \varphi f$ for all $f \in L^2(\hat\R)$, and thus 
    $\varphi = \varphi^2$. Consider the measurable set 
    \[ E = \{ t \in \hat\R \mid \varphi(t) = 0 \}. \] 
    Then $f \in \hat M$ if and only if $f = \varphi f$, if and only if 
    $(1 - \varphi)f = \chi_E f = 0$. This means that 
    \[ M = \{f \in L^2(\hat \R) \mid \hat f\chi_E = 0\} = M_E, \] 
    which completes the proof. 
\end{pf}
