\documentclass[10pt]{article}
\usepackage[T1]{fontenc}
\usepackage{amsmath,amssymb,amsthm}
\usepackage{mathtools}
\usepackage[shortlabels]{enumitem}
\usepackage[english]{babel}
\usepackage[utf8]{inputenc}
\usepackage{fancyhdr}
\usepackage{bold-extra}
\usepackage{color}   
\usepackage{tocloft}
\usepackage{graphicx}
\usepackage{lipsum}
\usepackage{wrapfig}
\usepackage{cutwin}
\usepackage{hyperref}
\usepackage{lastpage}
\usepackage{multicol}
\usepackage{tikz}
\usepackage{xcolor}
\usepackage{microtype}
\usepackage[framemethod=TikZ]{mdframed}

% some useful math commands
\newcommand{\eps}{\varepsilon}
\newcommand{\R}{\mathbb{R}}
\newcommand{\C}{\mathbb{C}}
\newcommand{\N}{\mathbb{N}}
\newcommand{\Z}{\mathbb{Z}}
\newcommand{\Q}{\mathbb{Q}}
\newcommand{\K}{\mathbb{K}}
\newcommand{\F}{\mathbb{F}}
\newcommand{\T}{\mathbb{T}}

\numberwithin{equation}{section}

\newcommand{\dd}{\,\mathrm{d}}
\newcommand{\ddz}{\frac{\rm d}{{\rm d}z}}
\newcommand{\pv}{\text{p.v.}}

\renewcommand{\Re}{{\rm Re}}

\DeclareMathOperator{\GL}{GL}
\DeclareMathOperator{\id}{id}
\DeclareMathOperator{\Arg}{Arg}
\DeclareMathOperator{\Log}{Log}
\DeclareMathOperator{\PV}{PV}
\DeclareMathOperator{\sech}{sech}
\DeclareMathOperator{\csch}{csch}
\DeclareMathOperator{\Res}{Res}
\DeclareMathOperator{\Li}{Li}
\DeclareMathOperator{\QR}{QR}
\DeclareMathOperator{\NR}{NR}
\DeclareMathOperator{\lcm}{lcm}
\DeclareMathOperator{\divergence}{div}
\DeclareMathOperator*{\esssup}{ess\,sup}
\DeclareMathOperator{\Span}{span}
\DeclareMathOperator{\Pol}{Pol}

\DeclarePairedDelimiter\ceil{\lceil}{\rceil}
\DeclarePairedDelimiter\floor{\lfloor}{\rfloor}

\newcommand{\suchthat}{\;\ifnum\currentgrouptype=16 \;\middle|\;\else\mid\fi\;}

% title formatting
\newcommand{\newtitle}[4]{
  \begin{center}
	\huge{\textbf{\textsc{#1 Course Notes}}}
    
	\large{\sc #2}
    
	{\sc #3 \textbullet\, #4 \textbullet\, University of Waterloo}
	\normalsize\vspace{1cm}\hrule
  \end{center}
}

\newcounter{theo}[section]\setcounter{theo}{0}
\renewcommand{\thetheo}{\arabic{section}.\arabic{theo}}
\newenvironment{theo}[2][]{%
\refstepcounter{theo}%
\ifstrempty{#1}%
{\mdfsetup{%
frametitle={%
\tikz[baseline=(current bounding box.east),outer sep=0pt]
\node[anchor=east,rectangle,fill=blue!20]
{\strut {\sc Theorem~\thetheo}};}}
}%
{\mdfsetup{%
frametitle={%
\tikz[baseline=(current bounding box.east),outer sep=0pt]
\node[anchor=east,rectangle,fill=blue!20]
{\strut {\sc Theorem~\thetheo:~#1}};}}%
}%
\mdfsetup{innertopmargin=10pt,linecolor=blue!20,%
linewidth=2pt,topline=true,%
frametitleaboveskip=\dimexpr-\ht\strutbox\relax
}
\begin{mdframed}[nobreak=false]\relax%
\label{#2}}{\end{mdframed}}

%%%%%%%%%%%%%%%%%%%%%%%%%%%%%%
%Definition
\newenvironment{defn}[2][]{%
\refstepcounter{theo}%
\ifstrempty{#1}%
{\mdfsetup{%
frametitle={%
\tikz[baseline=(current bounding box.east),outer sep=0pt]
\node[anchor=east,rectangle,fill=yellow!20]
{\strut {\sc Definition~\thetheo}};}}
}%
{\mdfsetup{%
frametitle={%
\tikz[baseline=(current bounding box.east),outer sep=0pt]
\node[anchor=east,rectangle,fill=yellow!20]
{\strut {\sc Definition~\thetheo:~#1}};}}%
}%
\mdfsetup{innertopmargin=10pt,linecolor=yellow!20,%
linewidth=2pt,topline=true,%
frametitleaboveskip=\dimexpr-\ht\strutbox\relax
}
\begin{mdframed}[nobreak=true]\relax%
\label{#2}}{\end{mdframed}}

%%%%%%%%%%%%%%%%%%%%%%%%%%%%%%
%Example
\newenvironment{exmp}[2][]{%
\refstepcounter{theo}%
\ifstrempty{#1}%
{\mdfsetup{%
frametitle={%
\tikz[baseline=(current bounding box.east),outer sep=0pt]
\node[anchor=east,rectangle,fill=cyan!20]
{\strut {\sc Example~\thetheo}};}}
}%
{\mdfsetup{%
frametitle={%
\tikz[baseline=(current bounding box.east),outer sep=0pt]
\node[anchor=east,rectangle,fill=cyan!20]
{\strut {\sc Example~\thetheo:~#1}};}}%
}%
\mdfsetup{innertopmargin=10pt,linecolor=cyan!20,%
linewidth=2pt,topline=true,%
frametitleaboveskip=\dimexpr-\ht\strutbox\relax
}
\begin{mdframed}[nobreak=false]\relax%
\label{#2}}{\end{mdframed}}

%%%%%%%%%%%%%%%%%%%%%%%%%%%%%%
%Corollary
\newenvironment{cor}[2][]{%
\refstepcounter{theo}%
\ifstrempty{#1}%
{\mdfsetup{%
frametitle={%
\tikz[baseline=(current bounding box.east),outer sep=0pt]
\node[anchor=east,rectangle,fill=lime!20]
{\strut {\sc Corollary~\thetheo}};}}
}%
{\mdfsetup{%
frametitle={%
\tikz[baseline=(current bounding box.east),outer sep=0pt]
\node[anchor=east,rectangle,fill=lime!20]
{\strut {\sc Corollary~\thetheo:~#1}};}}%
}%
\mdfsetup{innertopmargin=10pt,linecolor=lime!20,%
linewidth=2pt,topline=true,%
frametitleaboveskip=\dimexpr-\ht\strutbox\relax
}
\begin{mdframed}[nobreak=true]\relax%
\label{#2}}{\end{mdframed}}

%%%%%%%%%%%%%%%%%%%%%%%%%%%%%%
%Remark
\newenvironment{remark}[2][]{%
\refstepcounter{theo}%
\ifstrempty{#1}%
{\mdfsetup{%
frametitle={%
\tikz[baseline=(current bounding box.east),outer sep=0pt]
\node[anchor=east,rectangle,fill=orange!20]
{\strut {\sc Remark~\thetheo}};}}
}%
{\mdfsetup{%
frametitle={%
\tikz[baseline=(current bounding box.east),outer sep=0pt]
\node[anchor=east,rectangle,fill=orange!20]
{\strut {\sc Remark~\thetheo:~#1}};}}%
}%
\mdfsetup{innertopmargin=10pt,linecolor=orange!20,%
linewidth=2pt,topline=true,%
frametitleaboveskip=\dimexpr-\ht\strutbox\relax
}
\begin{mdframed}[nobreak=true]\relax%
\label{#2}}{\end{mdframed}}

%%%%%%%%%%%%%%%%%%%%%%%%%%%%%%
%Exercise
\newenvironment{exercise}[2][]{%
\refstepcounter{theo}%
\ifstrempty{#1}%
{\mdfsetup{%
frametitle={%
\tikz[baseline=(current bounding box.east),outer sep=0pt]
\node[anchor=east,rectangle,fill=pink!20]
{\strut {\sc Exercise~\thetheo}};}}
}%
{\mdfsetup{%
frametitle={%
\tikz[baseline=(current bounding box.east),outer sep=0pt]
\node[anchor=east,rectangle,fill=pink!20]
{\strut {\sc Exercise~\thetheo:~#1}};}}%
}%
\mdfsetup{innertopmargin=10pt,linecolor=pink!20,%
linewidth=2pt,topline=true,%
frametitleaboveskip=\dimexpr-\ht\strutbox\relax
}
\begin{mdframed}[nobreak=true]\relax%
\label{#2}}{\end{mdframed}}

%%%%%%%%%%%%%%%%%%%%%%%%%%%%%%
%Lemma
\newenvironment{lemma}[2][]{%
\refstepcounter{theo}%
\ifstrempty{#1}%
{\mdfsetup{%
frametitle={%
\tikz[baseline=(current bounding box.east),outer sep=0pt]
\node[anchor=east,rectangle,fill=green!20]
{\strut {\sc Lemma~\thetheo}};}}
}%
{\mdfsetup{%
frametitle={%
\tikz[baseline=(current bounding box.east),outer sep=0pt]
\node[anchor=east,rectangle,fill=green!20]
{\strut {\sc Lemma~\thetheo:~#1}};}}%
}%
\mdfsetup{innertopmargin=10pt,linecolor=green!20,%
linewidth=2pt,topline=true,%
frametitleaboveskip=\dimexpr-\ht\strutbox\relax
}
\begin{mdframed}[nobreak=true]\relax%
\label{#2}}{\end{mdframed}}

%%%%%%%%%%%%%%%%%%%%%%%%%%%%%%
%Proposition
\newenvironment{prop}[2][]{%
\refstepcounter{theo}%
\ifstrempty{#1}%
{\mdfsetup{%
frametitle={%
\tikz[baseline=(current bounding box.east),outer sep=0pt]
\node[anchor=east,rectangle,fill=purple!20]
{\strut {\sc Proposition~\thetheo}};}}
}%
{\mdfsetup{%
frametitle={%
\tikz[baseline=(current bounding box.east),outer sep=0pt]
\node[anchor=east,rectangle,fill=purple!20]
{\strut {\sc Proposition~\thetheo:~#1}};}}%
}%
\mdfsetup{innertopmargin=10pt,linecolor=purple!20,%
linewidth=2pt,topline=true,%
frametitleaboveskip=\dimexpr-\ht\strutbox\relax
}
\begin{mdframed}[nobreak=true]\relax%
\label{#2}}{\end{mdframed}}

%%%%%%%%%%%%%%%%%%%%%%%%%%%%%%
%Fact
\newenvironment{fact}[2][]{%
\refstepcounter{theo}%
\ifstrempty{#1}%
{\mdfsetup{%
frametitle={%
\tikz[baseline=(current bounding box.east),outer sep=0pt]
\node[anchor=east,rectangle,fill=gray!20]
{\strut {\sc Fact~\thetheo}};}}
}%
{\mdfsetup{%
frametitle={%
\tikz[baseline=(current bounding box.east),outer sep=0pt]
\node[anchor=east,rectangle,fill=gray!20]
{\strut {\sc Fact~\thetheo:~#1}};}}%
}%
\mdfsetup{innertopmargin=10pt,linecolor=gray!20,%
linewidth=2pt,topline=true,%
frametitleaboveskip=\dimexpr-\ht\strutbox\relax
}
\begin{mdframed}[nobreak=true]\relax%
\label{#2}}{\end{mdframed}}

% new proof environment
\makeatletter
\newenvironment{pf}[1][\proofname]{\par
  \pushQED{\qed}%
  \normalfont \topsep0\p@\relax
  \trivlist
  \item[\hskip\labelsep\scshape
  #1\@addpunct{.}]\ignorespaces
}{%
  \popQED\endtrivlist\@endpefalse
}
\makeatother

% 1-inch margins
\topmargin 0pt
\advance \topmargin by -\headheight
\advance \topmargin by -\headsep
\textheight 8.9in
\oddsidemargin 0pt
\evensidemargin \oddsidemargin
\marginparwidth 0.5in
\textwidth 6.5in

\parindent 0in
\parskip 1.5ex

\setlist[itemize]{topsep=0pt}
\setlist[enumerate]{topsep=0pt}

\newcommand{\pushright}[1]{\ifmeasuring@#1\else\omit\hfill$\displaystyle#1$\fi\ignorespaces}

% hyperlinks
\hypersetup{
  colorlinks=true, 
  linktoc=all,     % table of contents is clickable  
  allcolors=red    % all hyperlink colours
}

% table of contents
\addto\captionsenglish{
  \renewcommand{\contentsname}%
    {Table of Contents}%
}
\renewcommand{\cftsecfont}{\normalfont}
\renewcommand{\cftsecpagefont}{\normalfont}
\cftsetindents{section}{0em}{2em}

\fancypagestyle{plain}{%
\fancyhf{} % clear all header and footer fields
\lhead{PMATH 450: Spring 2022}
\fancyhead[R]{Table of Contents}
%\headrule
\fancyfoot[R]{{\small Page \thepage\ of \pageref*{LastPage}}}
}

% headers and footers
\pagestyle{fancy}
\renewcommand{\sectionmark}[1]{\markboth{#1}{#1}}
\lhead{PMATH 450: Spring 2022}
\cfoot{}
\setlength\headheight{14pt}

%\setcounter{section}{-1}

\begin{document}

\pagestyle{fancy}
\newtitle{PMATH 450}{Lebesgue Integration and Fourier Analysis}{Michael Brannan}{Spring 2022}
\rhead{Table of Contents}
\rfoot{{\small Page \thepage\ of \pageref*{LastPage}}}

\tableofcontents
\vspace{1cm}\hrule
\fancyhead[R]{\nouppercase\rightmark}
\newpage 
\fancyhead[R]{Section \thesection: \nouppercase\leftmark}

\section{Placeholder section}\label{sec:1}

\subsection{Placeholder subsection}\label{subsec:1.1}
\newpage
\section{Lebesgue Measure and Integration}\label{sec:2}

\subsection{Riemann Integration}\label{subsec:2.1}
Recall that in Riemann's theory of integration, we start with a bounded function 
$f : [a, b] \to \R$. We could then obtain $\int_a^b f(x)\dd x$ via 
approximations of Riemann sums. More specifically, we take a partition 
\[ P = \{a = t_0 < t_1 < \cdots < t_n = b\} \] 
of the interval $[a, b]$. For each $1 \leq i \leq n$, we set 
$m_i = \inf_{x \in [t_{i-1}, t_i)} f(x)$ and 
$M_i = \sup_{x \in [t_{i-1}, t_i)} f(x)$. We define the {\bf lower 
Riemann sum} by 
\[ L(f, P) = \sum_{i=1}^n m_i(t_i - t_{i-1}), \] 
and similarly, the {\bf upper Riemann sum} by 
\[ U(f, P) = \sum_{i=1}^n M_i(t_i - t_{i-1}). \] 
It is clear for all partitions $P$ of $[a, b]$ that $L(f, P) \leq U(f, P)$. 
Moreover, suppose $P$ and $Q$ are both partitions of $[a, b]$, and set 
$P \vee Q$ to be the partition consisting of all points in $P$ and $Q$. 
Then recall that $P \vee Q$ refines both $P$ and $Q$, and we have 
\[ L(f, P) \leq L(f, P \vee Q) \leq U(f, P \vee Q) \leq U(f, Q). \] 
Interchanging $P$ and $Q$ above gives us $L(f, Q) \leq U(f, P)$, so we can 
deduce that 
\[ \sup_P L(f, P) \leq \inf_P U(f, P). \] 
That is, any lower Riemann sum of a given partition will always be at most 
the upper Riemann sum of any other partition. 

\begin{defn}{defn:2.1}
    We say that $f : [a, b] \to \R$ is {\bf Riemann integrable} if 
    \[ \sup_P L(f, P) = \inf_P U(f, P). \] 
    In this case, we write 
    \[ \int_a^b f(x)\dd x = \sup_P L(f, P) = \inf_P U(f, P). \] 
    We write $R[a, b]$ to denote the vector space of Riemann integrable 
    functions $f : [a, b] \to \R$. 
\end{defn}

Riemann's theory is good for many purposes, such as for the Fundamental 
Theorem of Calculus or analysis over smooth manifolds. But there are also 
many deficiencies. 

For one, it forces $f$ to be bounded and ``almost continuous''. It also doesn't 
generalize to integration over sets that are not ``like'' $\R$ or $\R^N$. 
Sometimes, one wants to integrate functions over irregular sets, such as fractals. 

Worst of all, there are no good limit theorems! Ideally, we want some kind of 
``monotone convergence theorem'' which says that if we have a sequence of 
Riemann integrable functions $(f_n)_{n=1}^\infty \subseteq R[a, b]$ satisfying 
$f_1 \leq f_2 \leq \cdots$ and $f(x) = 
\lim_{n\to\infty} f_n(x)$ exists, then $f$ is also Riemann integrable with 
\[ \int_a^b f(x)\dd x = \lim_{n\to\infty} \int_a^b f_n(x)\dd x. \] 
Unfortunately, this result is false! Note that the pointwise limit of Riemann 
integrable functions might not even be Riemann integrable. Let 
$\{r_n\}_{n=1}^\infty$ be an enumeration of $\Q \cap [0, 1]$, and for each 
$n \in \N$, define the function $f_n : [0, 1] \to \R$ by 
\[ f_n(x) = \begin{cases}
    1, & \text{if } x \in \{r_1, \dots, r_n\}, \\ 
    0, & \text{otherwise.} 
\end{cases} \] 
Then for all $n \in \N$, we have $f_n \in R[0, 1]$ with 
\[ \int_0^1 f_n(x)\dd x = 0. \] 
Moreover, we see that $(f_n)_{n=1}^\infty$ converges pointwise to 
\[ f(x) = \begin{cases} 
    1, & \text{if } x \in \Q \cap [0, 1], \\ 
    0, & \text{otherwise,} 
\end{cases} \] 
the indicator function of $\Q$ over $[0, 1]$. Notice that this is a nice 
monotone limit since $f_1 \leq f_2 \leq \cdots$. Even still, $f \notin R[0, 1]$
since the rationals and irrationals are dense in $\R$, so given any 
partition $P$ of $[0, 1]$, the upper Riemann sum is $U(f, P) = 1$ and 
the lower Riemann sum is $L(f, P) = 0$. 

\subsection{Lebesgue Outer Measure}\label{subsec:2.2}
In the previous section, we saw that Riemann's theory of integration had 
some flaws. Lebesgue had the idea that we could do Riemann sums over 
partitions of the $y$-axis, instead of partitions over the $x$-axis. It turns 
out that this idea allows many more functions to be integrable. 

Take a function $f : [a, b] \to \R$, and suppose that we have a partition 
$P = \{y_0 < y_1 < \cdots < y_n\}$ of the $y$-axis. Then we will take sums of 
terms of the form 
\[ y_i \cdot \ell(\{x \in [a, b] : f(x) \in (y_{i-1}, y_i]\}), \] 
where $\ell(E)$ denotes the ``length'' of $E$. Notice that the set above 
is the preimage of the half-open interval $(y_{i-1}, y_i]$, and so we have an 
approximation 
\[ \int_a^b f(x)\dd x \approx \sum_{i=1}^n y_i \cdot \ell(f^{-1}(y_{i-1}, y_i]). \] 
Now, consider the case where $f$ is the indicator function of $\Q$ over 
$[0, 1]$ as we discussed above. Then we would have 
\[ \int_0^1 f(x)\dd x \approx 1 \cdot \ell(\Q \cap [0, 1]) + 0 \cdot 
\ell(\Q^c \cap [0, 1]). \] 
We would expect this to equal $0$ if we want our ``monotone convergence theorem''
to hold, and also because $\Q$ is a countable set. As such, a desired property 
of our ``length'' function would be to have $\ell(\Q \cap [0, 1]) = 0$. 

We have now packaged the problem into generalizing the notion of length 
from nice sets (such as unions of intervals) to more complicated sets 
of the form $f^{-1}(y_{i-1}, y_i]$. This turns out to be a difficult task 
even for a continuous function $f$. We wish to have a function 
\[ m : {\cal P}(\R) \to [0, \infty] \] 
satisfying the following properties: 
\begin{enumerate}[(1)]
    \item For any interval $I$ from $a$ to $b$ (which could be open, closed, 
    or half-open), we have 
    \[ m(I) = b-a. \] 
    \item {\bf Translation invariance.} For all $x \in \R$, we have 
    \[ m(E + x) = m(E), \] 
    where $E + x = \{y + x : y \in E\}$. 
    \item {\bf Countable additivity.} If $E_n \subseteq \R$ are disjoint 
    for all $n \in \N$ and $E = \bigsqcup_{n=1}^\infty E_n$, then 
    \[ m(E) = \sum_{n=1}^\infty m(E_n). \] 
\end{enumerate}
Bad news: there is no function $m : {\cal P}(\R) \to [0, \infty]$ 
satisfying all of (1) to (3)! 

To see this, we will assume there is such a function $m$ satisfying all $3$ 
properties, and find a subset $E \subseteq [0, 1)$ such that $m(E)$ is not 
well-defined. Define an equivalence relation on $[0, 1)$ by $x \sim y$ 
if and only if $x - y \in \Q$. We leave it as an exercise to verify that 
this is indeed an equivalence relation. As usual, let $[x]$ be the 
equivalence class of each $x \in [0, 1)$. 

Choose a set $E$ of representatives of all the equivalence classes $[x]$ for 
$x \in [0, 1)$. Note that this is possible by the Axiom of Choice, and we have 
$E \subseteq [0, 1)$ with $|E \cap [x]| = 1$ for all $x \in [0, 1)$. 

We claim that $m(E)$ is not well-defined. Let $\{r_n\}_{n=1}^\infty$ be 
an enumeration of $\Q \cap [0, 1)$, and for each $n \in \N$, set 
\begin{align*}
    E_n = E + r_n \text{ (mod $1$)} = ((E + r_n) \cap [0, 1)) \sqcup 
    ((E + r_n - 1) \cap [0, 1)). 
\end{align*}
Since $m$ satisfies properties (1) to (3), we have that 
\begin{align*}
    m(E_n) &= m((E + r_n) \cap [0, 1)) + m((E + r_n - 1) \cap [0, 1)) & 
    \text{by (3)} \\ 
    &= m((E + r_n) \cap [0, 1)) + m((E + r_n) \cap [1, 2)) & \text{by (2)} \\ 
    &= m(E + r_n) & \text{by (3)} \\ 
    &= m(E) & \text{by (2)}
\end{align*}
for all $n \in \N$. We leave it as an exercise to check that $[0, 1) = 
\bigsqcup_{n=1}^\infty E_n$. It follows that 
\[ 1 = m([0, 1)) = \sum_{m=1}^\infty m(E_n) = \sum_{n=1}^\infty m(E) \] 
where the second equality comes from (3), so there is no admissible value for 
$m(E)$. 

In light of this result, we might be asking too much for $m$ 
to satisfy all $3$ properties. What can we do instead? 
\begin{enumerate}[(a)]
    \item We can restrict the domain of the function $m$ to a 
    more ``tractable'' family of subsets of $\R$. Naturally, we would want to 
    allow all intervals, as well as open and closed sets, to be in this family.
    \item One other approach is to take a function $m$ that works for all 
    subsets $E \subseteq \R$ and agrees with our intuitive notion of length 
    for intervals, but in doing so, sacrificing some of the desirable properties. 
\end{enumerate}

The standard approach is the first one, and we shall adopt it. We now 
introduce a candidate function that could be used for $m$. 

\begin{defn}{defn:2.2}
    The {\bf Lebesgue outer measure} of a subset $E \subseteq \R$ is defined to be 
    \[ m^*(E) = \inf\left\{ \sum_{n=1}^\infty \ell(I_n) : \{I_n\}_{n=1}^\infty 
    \text{ a cover of $E$ by intervals}\right\}. \] 
\end{defn}

We look at some properties of the Lebesgue outer measure. 

\begin{prop}{prop:2.3}
    \begin{enumerate}[(a)]
        \item We have $m^*(\varnothing) = 0$ and $m^*(E) \geq 0$ 
        for all $E \subseteq \R$. 
        \item {\bf Translation invariance.} For all $E \subseteq \R$ and 
        $x \in \R$, we have $m^*(E+x) = m^*(E)$. 
        \item {\bf Monotonicity.} If $E \subseteq F \subseteq \R$, then 
        $m^*(E) \leq m^*(F)$. 
        \item {\bf Countable subadditivity.} Suppose that $E = 
        \bigcup_{n=1}^\infty E_n$ where the subsets $E_n \subseteq \R$ are not 
        necessarily disjoint. Then 
        \[ m^*(E) \leq \sum_{n=1}^\infty m^*(E_n). \] 
    \end{enumerate}
\end{prop}
\begin{pf}~
    \begin{enumerate}[(a)]
        \item This is clear from definition. 
        \item Coverings of $E$ by countable families of intervals are in 
        bijection with those of $E + x$. In particular, we have 
        $E \subseteq \bigcup_{n=1}^\infty I_n$ if and only if 
        $E + x \subseteq \bigcup_{n=1}^\infty (I_n + x)$, with 
        \[ \sum_{n=1}^\infty \ell(I_n) = \sum_{n=1}^\infty \ell(I_n + x). \] 
        Taking infima, we obtain $m^*(E) = m^*(E + x)$. 
        \item Any covering $\bigcup_{n=1}^\infty I_n$ of $F$ also gives a 
        covering of $E$. On the other hand, coverings of $E$ are not 
        necessarily coverings of $F$. Then the infimum in $m^*(E)$ is 
        taken over a larger collection than with $m^*(F)$, so we get 
        $m^*(E) \leq m^*(F)$. 
        \item Without loss of generality, suppose that $\sum_{n=1}^\infty 
        m^*(E_n) < \infty$. Let $\eps > 0$. Since $m^*(E_n) < \infty$ 
        for all $n \in \N$, there exists a covering $\bigcup_{k=1}^\infty 
        I_{k,n}$ of $E_n$ such that 
        \[ \sum_{k=1}^\infty \ell(I_{k,n}) < m^*(E_n) + \frac{\eps}{2^n}. \] 
        Then we have $E = \bigcup_{n=1}^\infty E_n \subseteq 
        \bigcup_{n,k=1}^\infty I_{k,n}$, so it follows that 
        \begin{align*}
            m^*(E) &\leq \sum_{n=1}^\infty \sum_{k=1}^\infty \ell(I_{k,n}) \\ 
            &\leq \sum_{n=1}^\infty \left( m^*(E_n) + \frac{\eps}{2^n} \right) \\
            &= \sum_{n=1}^\infty m^*(E_n) + \eps. 
        \end{align*}
        Since $\eps > 0$ was arbitrary, the result follows. \qedhere 
    \end{enumerate}
\end{pf}

The following result tells us that we can compute $m^*(E)$ using ``small''
open intervals $I_n$ in our covers. 

\begin{prop}{prop:2.4}
    Let $E \subseteq \R$ and $\delta > 0$. Then 
    \[ m^*(E) = \inf\left\{ \sum_{n=1}^\infty \ell(I_n) : 
    \{I_n\}_{n=1}^\infty \text{ a cover of $E$ by open intervals such that }
    \ell(I_n) < \delta \right\}. \] 
\end{prop}
\begin{pf}
    It is clear that $m^*(E)$ is at most the right hand side, because 
    the collection we are taking the infimum over in the definition of 
    $m^*$ is more general than the one in this proposition, which forces 
    the intervals to be open and have length less than $\delta$. 

    So, we turn to proving the other direction. Without loss of generality, 
    we may assume that $m^*(E) < \infty$. Let $\eps > 0$. We can 
    find intervals $\{J_n\}_{n=1}^\infty$ such that $E \subseteq 
    \bigcup_{n=1}^\infty J_n$ and 
    \[ \sum_{n=1}^\infty \ell(J_n) \leq m^*(E) + \eps. \] 
    Without loss of generality, we can also partition these intervals 
    $\{J_n\}_{n=1}^\infty$ into subintervals so that $\ell(J_n) < \delta$
    for all $n \in \N$. Now, choose open intervals $I_n \supseteq J_n$ 
    such that $\ell(I_n) \leq \max\{\delta, \ell(J_n) + \eps/2^n\}$. 
    We see that $E \subseteq \bigcup_{n=1}^\infty I_n$ with 
    \[ \sum_{n=1}^\infty \ell(I_n) \leq \sum_{n=1}^\infty 
    \left( \ell(J_n) + \frac{\eps}{2^n} \right) \leq (m^*(E) + \eps) + \eps 
    = m^*(E) + 2\eps. \] 
    Taking infima, we see that $m^*(E)$ is at least the right hand side 
    and we are done. 
\end{pf}

Next, we show that Lebesgue outer measure really generalizes length. 

\begin{theo}{theo:2.5}
    Let $I$ be an interval with left endpoint $a$ and right endpoint $b$, where 
    $a < b \in \R$. Then 
    \[ m^*(I) = \ell(I) = b-a. \]  
\end{theo}
\begin{pf}
    We first prove this for the case where $I = [a, b]$ is a compact interval. 
    By taking $I_1 = I$ and $I_2 = \varnothing$ for $n \geq 2$, the 
    collection $\{I_n\}_{n=1}^\infty$ is a cover of $[a, b]$ by intervals, 
    which implies that 
    \[ m^*(I) \leq \sum_{n=1}^\infty \ell(I_n) = \ell(I_1) = b-a. \] 
    Thus, we have $m^*(I) \leq b-a$. We now turn to showing that 
    $m^*(I) \geq b-a$. Let $\eps > 0$. By Proposition~\ref{prop:2.4}, we can 
    find an open cover $\bigcup_{n=1}^\infty (a_n, b_n)$ of $I$ such that 
    \[ \sum_{n=1}^\infty (b_n - a_n) \leq m^*(I) + \eps. \] 
    But $I$ is compact, so we can find a finite subcover; that is, 
    there exists $N \in \N$ such that $I \subseteq \bigcup_{n=1}^N (a_n, b_n)$. 
    Without loss of generality, we may toss away any intervals $I_n$ such that 
    $I_n \cap I = \varnothing$, and reorder the $I_n$'s if necessary to get 
    $a_1 < a_2 < \cdots < a_N$. Since $I$ is an interval, it is connected. 
    Thus, the intervals $(a_n, b_n)$ must overlap, and we obtain 
    \[ \ell(I) = b-a \leq \sum_{n=1}^N (b_n - a_n) \leq 
    \sum_{n=1}^\infty (b_n - a_n) \leq m^*(I) + \eps. \] 
    But $\eps > 0$ was arbitrary, so we have $m^*(I) \geq \ell(I) = b-a$, as 
    desired. 

    We now prove the result in the case that $I = (a, b]$. For all 
    $0 < \eps < b-a$, we have $[a + \eps, b] \subseteq (a, b] \subseteq 
    [a, b]$. By the monotonicity of Lebesgue outer measure 
    (Proposition~\ref{prop:2.3}), we see that 
    \[ (b-a) - \eps = m^*[a + \eps, b] \leq m^*(a, b] \leq m^*[a, b] = b-a. \] 
    Since $\eps$ is arbitrary (subject to $0 < \eps < b-a$), we deduce that 
    $m^*(a, b] = b-a$. The cases where $I = [a, b)$ and $I = (a, b)$ are 
    proved similarly. 
\end{pf}

We have now shown that the Lebesgue outer measure is translation invariant 
and is a ``good'' generalization of length. Does countable additivity hold 
for $m^*$? The answer is of course no, because we showed that there is 
no function $m : {\cal P}(\R) \to [0, \infty]$ which simultaneously 
extends length, is translation invariant, and countably additive. 

So when can countable or even just finite additivity hold? 
We first consider some special cases. 

\begin{defn}{defn:2.6}
    A set $E \subseteq \R$ is said to have {\bf Lebesgue (outer) measure 
    zero} if $m^*(E) = 0$.
\end{defn}

The following lemma says that when the sets have Lebesgue measure zero, 
then $m^*$ is countably additive. 

\begin{lemma}{lemma:2.7}
    If $(E_n)_{n=1}^\infty$ are not necessarily disjoint sets with 
    $m^*(E_n) = 0$ for all $n \in \N$, then 
    \[ m^*\!\left( \bigcup_{n=1}^\infty E_n \right) = 
    \sum_{n=1}^\infty m^*(E_n) = 0. \] 
\end{lemma}
\begin{pf}
    By subadditivity of $m^*$, we obtain 
    \[ 0 \leq m^*\!\left( \bigcup_{n=1}^\infty E_n \right) \leq 
    \sum_{n=1}^\infty m^*(E_n) = 0. \qedhere \] 
\end{pf}

Next, we consider the notion of distance between two non-empty subsets of $\R$.

\begin{defn}{defn:2.8}
    Let $E, F \subseteq \R$ be non-empty. We define 
    \[ d(E, F) = \inf\{|x - y| : x\in E, y\in F\} \] 
    to be the {\bf distance} between $E$ and $F$. 
\end{defn}

Notice that if $E, F \subseteq \R$ satisfy $d(E, F) > 0$, then they are 
certainly disjoint. In this case, it turns out that finite additivity holds
for $m^*$.

\begin{prop}{prop:2.9}
    If $E, F \subseteq \R$ are such that $d(E, F) > 0$, then 
    \[ m^*(E \sqcup F) = m^*(E) + m^*(F). \] 
\end{prop}
\begin{pf}
    We will assume that $m^*(E), m^*(F) < \infty$. Otherwise, we get equality 
    for free by observing that $E \sqcup F \supseteq E$ and $E \sqcup F 
    \supseteq F$ and using the monotonicity of $m^*$. 
    
    By the countable subadditivity of $m^*$, we have $m^*(E \sqcup F) \leq 
    m^*(E) + m^*(F)$, so we only need to prove the other direction. 
    Let $\delta = d(E, F) > 0$, and let $\eps > 0$. Then there is a 
    covering of $E \sqcup F$ by intervals $\bigcup_{n=1}^\infty I_n$ 
    such that $\ell(I_n) < \delta$ for all $n \in \N$ and 
    \[ \sum_{n=1}^\infty \ell(I_n) < m^*(E \sqcup F) + \eps. \] 
    Without loss of generality, we can toss away any intervals $I_n$ 
    such that $I_n \cap (E \sqcup F) = \varnothing$. Moreover, the 
    restriction that $\ell(I_n) < \delta$ for all $n \in \N$ means that 
    each $I_n$ touches one of $E$ or $F$, but not both. So we can 
    partition $\{I_n\}_{n=1}^\infty$ into $\{I'_n\}_{n=1}^\infty 
    \cup \{I''_n\}_{n=1}^\infty$, where the intervals $I'_n$ only touch 
    $E$ and the intervals $I''_n$ only touch $F$. Observe now that 
    $\{I'_n\}_{n=1}^\infty$ is a covering of $E$ and 
    $\{I''_n\}_{n=1}^\infty$ is a covering of $F$, so we obtain 
    \[ m^*(E) + m^*(F) \leq \sum_{n=1}^\infty \ell(I'_n) + \sum_{n=1}^\infty 
    \ell(I''_n) \leq \sum_{n=1}^\infty \ell(I_n) < m^*(E \sqcup F) + \eps. \] 
    Since $\eps > 0$ was arbitrary, this completes the proof. 
\end{pf}

\begin{cor}{cor:2.10}
    If $K_1, \dots, K_n \subseteq \R$ are pairwise disjoint compact sets, then 
    \[ m^*\!\left( \bigsqcup_{i=1}^n K_i \right) = \sum_{i=1}^n m^*(K_i). \] 
\end{cor}
\begin{pf}
    Observe that $d(E, F) > 0$ when $E$ and $F$ are compact with $E \cap F 
    = \varnothing$. So this result follows by induction and applying 
    Proposition~\ref{prop:2.9}. 
\end{pf}

\subsection{Lebesgue Measure and Lebesgue Measurable Sets}\label{subsec:2.3}
Our goal is to find a {\bf large} class of subsets ${\cal L} \subseteq 
{\cal P}(\R)$ so that countable additivity of $m^*$ holds for ${\cal L}$. 
We want ${\cal L}$ to contain all intervals, closed sets, open sets, and 
anything else that can be built from them by countable unions and intersections. 
In other words, we ideally want ${\cal L}$ to be a so-called $\sigma$-algebra, 
which we define below.

\begin{defn}{defn:2.11}
    Let $X$ be a non-empty set. A family ${\cal M}$ of subsets of $X$ is 
    called a {\bf $\sigma$-algebra} if the following three properties hold: 
    \begin{enumerate}[(1)]
        \item $\varnothing \in {\cal M}$ and $X \in {\cal M}$; 
        \item $E \in {\cal M}$ if and only if $E^c \in {\cal M}$; and 
        \item if $\{E_n\}_{n=1}^\infty$ is a countable sequence in ${\cal M}$, 
        then $\bigcup_{n=1}^\infty E_n \in {\cal M}$. 
    \end{enumerate}
\end{defn}

\begin{remark}{remark:2.12}
    It follows immediately from the definition that $\sigma$-algebras 
    are closed under countable intersections as well. Indeed, for 
    a $\sigma$-algebra ${\cal M}$ and a countable sequence $\{E_n\}_{n=1}^\infty$ 
    in ${\cal M}$, we have 
    \[ \bigcap_{n=1}^\infty E_n = \left( \bigcup_{n=1}^\infty E_n^c \right)^{\!c} 
    \in {\cal M} \] 
    by De Morgan's law. In fact, we could use countable intersection 
    in the definition of a $\sigma$-algebra and derive countable unions from it.
\end{remark}

\begin{exmp}{exmp:2.13}
    \begin{enumerate}[(a)]

        \item Some examples of $\sigma$-algebras are ${\cal P}(X)$ and 
        ${\cal M}_E = \{\varnothing, X, E, E^c\}$ for a subset $E \subseteq X$.

        \item The $\sigma$-algebras from part (a) are not very interesting; 
        ${\cal P}(X)$ is too big and ${\cal M}_E$ is too small to work with. 
        We give a slightly more interesting example. Let ${\cal F} \subseteq 
        {\cal P}(X)$, and define 
        \[ {\cal M}_{\cal F} = \bigcap_{\substack{
            \text{$\sigma$-algebras ${\cal M}$} \\ 
            \text{on $X$ such} \\ 
            \text{that } {\cal F} \subseteq {\cal M}}} {\cal M}. \] 
        Then ${\cal M}_{\cal F}$ is also a $\sigma$-algebra on $X$. 
        In fact, it is the smallest $\sigma$-algebra containing 
        ${\cal F}$. We call ${\cal M}_{\cal F}$ the {\bf $\sigma$-algebra 
        generated by ${\cal F}$}.

        \item Let $(X, \tau)$ be a topological space. Then 
        \[ {\cal B}_X = {\cal M}_\tau \subseteq {\cal P}(X) \] 
        is called the {\bf Borel $\sigma$-algebra}, and is the $\sigma$-algebra 
        generated by open sets in $X$. 
    \end{enumerate}
\end{exmp}

With the definition of a $\sigma$-algebra out of the way, we can now
discuss what a measure on one should look like. 

\begin{defn}{defn:2.14}
    Let $X$ be a non-empty set, and let ${\cal M}$ be a $\sigma$-algebra on $X$. 
    We say that a function $\mu : {\cal M} \to [0, \infty]$ is a 
    {\bf measure on ${\cal M}$} if we have 
    \begin{enumerate}[(1)]
        \item $\mu(\varnothing) = 0$, and 
        \item if $(E_n)_{n=1}^\infty$ is a sequence of pairwise disjoint sets
        in ${\cal M}$, then 
        \[ \mu\!\left( \bigsqcup_{n=1}^\infty E_n \right) = \sum_{n=1}^\infty 
        \mu(E_n). \] 
    \end{enumerate}
\end{defn}

In particular, the second property means that the measure $\mu$ is 
countably additive on the $\sigma$-algebra ${\cal M}$. Recall that we wanted 
our large set ${\cal L}$ above to be a $\sigma$-algebra on $\R$. Therefore, 
our hope is that $m = m^*|_{\cal L}$ is a measure on ${\cal L}$.

Let us now actually construct this large set ${\cal L}$. We first begin 
with a definition. 

\begin{defn}{defn:2.15}
    A set $E \subseteq \R$ is said to satisfy {\bf Carathéodory's condition} 
    if for any $A \subseteq \R$, we have 
    \[ m^*(A) = m^*(A \cap E) + m^*(A \cap E^c). \] 
    That is, $m^*$ is additive when partitioning any set $A$ with 
    $\{E, E^c\}$. 
\end{defn}

Then, we set ${\cal L} := \{E \subseteq \R : E \text { satisfies 
Carathéodory's condition}\}$. It turns out that this choice of ${\cal L}$ 
gives us exactly what we want. Of course, we will need to check that 
${\cal L}$ is actually a rich collection of sets, but we will leave 
that for later. 

\begin{theo}[Carathéodory]{theo:2.16}
    The set ${\cal L}$ defined as above is a $\sigma$-algebra over $\R$, 
    and $m = m^*|_{\cal L}$ is a measure on ${\cal L}$. We call 
    $m$ the {\bf Lebesgue measure} on $\R$. 
\end{theo}
\begin{pf}
    We will prove both of these claims simultaneously by beginning with 
    the finite case, then lifting it to the infinite case.

    {\sc Step 0.} Recall that an {\bf algebra} of sets ${\cal A} 
    \subseteq {\cal P}(X)$ has the following properties: 
    \begin{enumerate}[(1)]
        \item $\varnothing \in {\cal A}$ and $X \in {\cal A}$;
        \item $E \in {\cal A}$ if and only if $E^c \in {\cal A}$; and 
        \item if $E, F \in {\cal A}$, then $E \cup F \in {\cal A}$.
    \end{enumerate}
    In particular, this is weaker than being a $\sigma$-algebra, 
    only being closed under finite unions (and by extension, finite 
    intersections). We show that ${\cal L}$ is an algebra of sets in $\R$.

    {\sc Proof of Step 0.}~
    \begin{enumerate}[(1)]

        \item Observe that for any $A \subseteq \R$, we have 
        \[ m^*(A) = m^*(A \cap \R) + m^*(A \cap \varnothing). \] 
        Then $\R \in {\cal L}$ and by symmetry, $\varnothing \in {\cal L}$ 
        as well.

        \item If $E \in {\cal L}$, then for any $A \subseteq \R$, we have
        \[ m^*(A) = m^*(A \cap E) + m^*(A \cap E^c), \] 
        so we see that $E^c \in {\cal L}$. 

        \item Let $E, F \in {\cal L}$ and $A \subseteq \R$. We will 
        show that $E \cap F \in {\cal L}$, and so $E \cup F = 
        (E^c \cap F^c)^c \in {\cal L}$ by De Morgan's laws. First, 
        observe that 
        \begin{align*}
            m^*(A) &= m^*(A \cap E) + m^*(A \cap E^c) \\ 
            &= m^*(A \cap E \cap F) + m^*(A \cap E \cap F^c) 
            + m^*(A \cap E^c),
        \end{align*}
        where the first equality uses the fact that $E \in {\cal L}$, 
        and the second equality comes from the fact that $F \in {\cal L}$ 
        and applying it to $m^*(A \cap E)$. The idea now is to combine 
        some terms to obtain a corresponding term $m^*(A \cap (E \cap F)^c)$. 
        Indeed, we have 
        \begin{align*}
            m^*(A \cap (E \cap F)^c) 
            &= m^*(A \cap (E^c \cup F^c)) \\ 
            &= m^*(A \cap (E^c \cup F^c) \cap E) + m^*(A \cap (E^c \cup F^c) \cap E^c) \\ 
            &= m^*(A \cap F^c \cap E) + m^*(A \cap E^c). 
        \end{align*}
        These last two terms are exactly what we had before, so combining
        everything gives us 
        \[ m^*(A) = m^*(A \cap (E \cap F)) + m^*(A \cap (E \cap F)^c). \] 
        Thus, $E \cap F \in {\cal L}$, and we conclude that ${\cal L}$ is 
        an algebra over $\R$. \hfill $\blacksquare$

    \end{enumerate}

    {\sc Step 1.} The Lebesgue outer measure $m^*$ is finitely additive 
    on ${\cal L}$. That is, if $E_1, \dots, E_n \in {\cal L}$ are pairwise 
    disjoint, then 
    \[ m^*\!\left( \bigsqcup_{i=1}^n E_i \right) = \sum_{i=1}^n m^*(E_i). \] 
    {\sc Proof of Step 1.} Let $E_1, \dots, E_n \in {\cal L}$ be 
    pairwise disjoint, and let $A \subseteq \R$. We know that 
    $\bigsqcup_{i=1}^n E_i \in {\cal L}$ since we have already shown that 
    ${\cal L}$ is an algebra. Thus, noting that $E_n \in {\cal L}$, we get
    \[ m^*\!\left(A \cap \bigsqcup_{i=1}^n E_i \right) = 
    m^*\!\left(A \cap \bigsqcup_{i=1}^n E_i \cap E_n \right) 
    + m^*\!\left( A \cap \bigsqcup_{i=1}^n E_i \cap E_n^c \right) 
    = m^*(A \cap E_n) + m^*\!\left( A \cap \bigsqcup_{i=1}^{n-1} E_i \right), \] 
    where the final equality is because the sets are pairwise disjoint. 
    Repeating this argument inductively yields
    \[ m^*\left( A \cap \bigsqcup_{i=1}^n E_i \right) = 
    \sum_{i=1}^n m^*(A \cap E_i). \] 
    Taking $A = \R$ completes the proof of this step. \hfill $\blacksquare$

    {\sc Step 2.} If $\{E_i\}_{i=1}^\infty$ is a countable family of 
    disjoint sets in ${\cal L}$, then $E = \bigsqcup_{i=1}^\infty E_i \in 
    {\cal L}$ and 
    \[ m^*(E) = \sum_{i=1}^\infty m^*(E_i). \] 
    {\sc Proof of Step 2.} It suffices to show that 
    \[ m^*(A) \geq m^*(A \cap E) + m^*(A \cap E^c) \] 
    for any $A \subseteq \R$ with $m^*(A) < \infty$. Indeed, by the 
    subadditivity of $m^*$, we always have $m^*(A) \leq 
    m^*(A \cap E) + m^*(A \cap E^c)$. Moreover, if $m^*(A) = \infty$, 
    then there is certainly equality. 

    Let $A \subseteq \R$ be such that $m^*(A) < \infty$. Let 
    $F_n = \bigsqcup_{i=1}^n E_i$, which is in ${\cal L}$ since 
    it is an algebra. Arguing as in Step 1, we have 
    \[ m^*(A) = m^*(A \cap F_n) + m^*(A \cap F_n^c) 
    = \sum_{i=1}^n m^*(A \cap E_i) + m^*(A \cap F_n^c). \] 
    Notice that $F_n \subseteq E$, so $E^c \subseteq F_n^c$, which in 
    turn implies $A \cap E^c \subseteq A \cap F_n^c$. By the 
    monotonicity of $m^*$, we get 
    \[ m^*(A \cap E^c) \leq m^*(A \cap F_n^c). \] 
    Applying this inequality to our above equation gives 
    \[ m^*(A) \geq \sum_{i=1}^n m^*(A \cap E_i) + m^*(A \cap E^c). \] 
    Since this holds for all $n \in \N$, we can take $n \to \infty$ 
    and the inequality will still hold. This means that 
    \begin{align*}
        m^*(A) &\geq \sum_{i=1}^\infty m^*(A \cap E_i) + m^*(A \cap E^c) \\ 
        &\geq m^*\!\left( \bigsqcup_{i=1}^\infty (A \cap E_i) \right) 
        + m^*(A \cap E^c) \\ 
        &= m^*(A \cap E) + m^*(A \cap E^c), 
    \end{align*}
    where the second inequality follows from the countable subadditivity of 
    $m^*$. Letting $A = E$, we deduce that 
    \[ m^*(E) \geq \sum_{i=1}^\infty m^*(E_i) + 0 \geq m^*(E), \] 
    so we have the desired equality. \hfill $\blacksquare$ 

    {\sc Step 3.} We claim that ${\cal L}$ is closed under countable unions. 
    
    {\sc Proof of Step 3.} Let $\{E_i\}_{i=1}^\infty$ be a countable 
    family of not necessarily pairwise disjoint sets in ${\cal L}$. 
    Let $E'_1 = E_1$, and for each $i \geq 2$, set $E'_i = E_i \setminus 
    (E_1 \cup \cdots \cup E_{i-1})$. Then $\{E'_i\}_{i=1}^\infty$ is a 
    countable family of disjoint sets in ${\cal L}$ with 
    \[ \bigcup_{i=1}^\infty E_i = \bigsqcup_{i=1}^\infty E'_i, \] 
    and this is in ${\cal L}$ since we showed that countable unions of 
    pairwise disjoint sets in ${\cal L}$ are also in ${\cal L}$. 
    \hfill $\blacksquare$  

    This last step shows that ${\cal L}$ is a $\sigma$-algebra over $\R$
    and Step 2 establishes that $m = m^*|_{\cal L}$ is a measure on 
    ${\cal L}$, completing the proof. 
\end{pf}

In light of this result, we say that a set satisfying Carathéodory's condition 
is a {\bf Lebesgue measurable set}. Now, let's determine which sets belong 
to ${\cal L}$. Below, we see that sets of Lebesgue measure zero are an 
easy example of Lebesgue measurable sets. 

\begin{prop}{prop:2.17}
    If $E \subseteq \R$ with $m^*(E) = 0$, then $E \in {\cal L}$.     
\end{prop}
\begin{pf}
    Let $A \subseteq \R$. Notice that $m^*(A \cap E) \leq m^*(E) = 0$
    by the monotonicity of $m^*$, so $m^*(A \cap E) = 0$. 
    Similarly, we have $m^*(A \cap E^c) \leq m^*(A)$. Therefore, we obtain  
    \[ m^*(A) \leq m^*(A \cap E) + m^*(A \cap E^c) = m^*(A \cap E^c) 
    \leq m^*(A) \] 
    and so we have the equality $m^*(A) = m^*(A \cap E) + m^*(A \cap E^c)$. 
\end{pf}

In fact, it turns out that ${\cal B}_{\R}$, the Borel $\sigma$-algebra 
over $\R$, is contained in ${\cal L}$. 

\begin{theo}{theo:2.18}
    We have ${\cal B}_{\R} \subseteq {\cal L}$. 
\end{theo}
\begin{pf}
    Note that every open set can be written as the countable union of open 
    intervals. Since ${\cal L}$ is a $\sigma$-algebra, it suffices to 
    show that any open interval $I = (a, b)$ is in ${\cal L}$, where 
    $a < b \in \R$ are finite. 

    Let $I = (a, b)$, and fix $A \subseteq \R$ such that $m^*(A) < \infty$. 
    It is enough to show that 
    \[ m^*(A) \geq m^*(A \cap I) + m^*(A \cap I^c). \] 
    Pick $n \in \N$ large enough so that $I_n = [a + \frac1n, b - \frac1n]
    \subseteq I$. Notice that 
    \[ A = (A \cap I) \sqcup (A \cap I^c) \supseteq 
    (A \cap I_n) \sqcup (A \cap I^c), \] 
    with $d(A \cap I_n, A \cap I^c) \geq \frac1n > 0$. By Proposition~\ref{prop:2.9},
    we know that $m^*$ is additive on sets with positive separation, so we have 
    \[ m^*(A) \geq m^*((A \cap I_n) \sqcup (A \cap I^c)) 
    = m^*(A \cap I_n) + m^*(A \cap I^c). \] 
    Now, we are close to the desired inequality. Note that $A \cap I = 
    (A \cap I_n) \sqcup (A \cap I \setminus I_n)$, so by the subadditivity
    and monotonicity of $m^*$, we obtain 
    \begin{align*}
        m^*(A) &\leq m^*(A \cap I_n) + m^*(A \cap I \setminus I_n) \\ 
        &\leq m^*(A \cap I_n) + m^*(I \setminus I_n) \\ 
        &= m^*(A \cap I_n) + \tfrac2n, 
    \end{align*}
    where $m^*(I \setminus I_n) = m^*((a, a+\frac1n)) + m^*((b-\frac1n, b)) 
    = \frac2n$ since the intervals are disjoint. It follows that 
    \[ \lim_{n\to\infty} m^*(A \cap I_n) = m^*(A \cap I), \] 
    and we deduce that $m^*(A) \geq m^*(A \cap I) + m^*(A \cap I^c)$ 
    as desired. 
\end{pf}

Is it the case that ${\cal L}$ is actually equal to ${\cal B}_{\R}$? The 
answer is no, but it is difficult to explicitly write down a Lebesgue 
measurable set that is not in ${\cal B}_{\R}$. Instead, we can argue using the 
cardinalities of the sets. 

We first consider the cardinality of ${\cal B}_{\R}$. Recall the following 
definitions from PMATH 351. 

\begin{defn}{defn:2.19}
    Let $E \subseteq \R$ be a set. 
    \begin{itemize}
        \item We say that $E$ is a {\bf $G_\delta$-set} if $E$ is a 
        countable intersection of open sets. 
        \item We say that $E$ is an {\bf $F_\sigma$-set} if $E$ is a 
        countable union of closed sets. 
    \end{itemize}
\end{defn}

Notice that every $G_\delta$-set and $F_\sigma$-set is also a Borel set 
since ${\cal B}_{\R}$ is a $\sigma$-algebra, which is generated by the 
open sets in $\R$ and is closed under complements, countable unions and 
intersections. Moreover, we can also iterate these operations, so for 
instance, $G_{\delta\sigma}$ contains all sets that are countable unions of 
$G_\delta$-sets.  

Every open set $U \subseteq \R$ is a countable union of open intervals 
with rational endpoints. Let ${\cal F}$ be the set of all open intervals with 
rational endpoints, and observe that ${\cal F}$ is countable. Then 
any Borel set $E \subseteq \R$ can be generated from ${\cal F}$ by iterating 
the operations 
\[ G_\delta \to G_{\delta\sigma} \to G_{\delta\sigma\delta} \to \cdots \] 
countably often. This means that $|{\cal B}_{\R}| = |\N|^{|\N|} = |\R|$. 

We now consider the cardinality of ${\cal L}$. On Assignment 2, we will 
show that there is a set $\Delta \subseteq [0, 1] \subseteq \R$ called 
the {\bf Cantor middle thirds set} such that $\Delta$ is uncountable 
and $m^*(\Delta) = 0$. By Proposition~\ref{prop:2.17}, this means that 
$\Delta \in {\cal L}$. Then monotonicity tells us that any subset 
$E \subseteq \Delta$ is also in ${\cal L}$. In particular, we have 
${\cal L} \supseteq {\cal P}(\Delta)$ and so 
\[ |{\cal L}| \geq |{\cal P}(\Delta)| = |{\cal P}(\R)| = |\R|^{|\R|} 
> |\R| = |{\cal B}_{\R}|. \] 
Thus, we get the following result. 

\begin{theo}{theo:2.20}
    We have ${\cal B}_{\R} \subsetneq {\cal L}$. 
\end{theo}

Next, we show that $m = m^*|_{\cal L}$ is the unique measure on ${\cal L}$
extending length. Before we do that, we note that every measure $\mu$ is 
subadditive as a consequence of additivity. Indeed, let $E, F \subseteq \R$. 
We first show that
\[ \mu(E \cup F) + \mu(E \cap F) = \mu(E) + \mu(F). \] 
To begin, observe that we have $\mu(E) = \mu(E \cap F) + \mu(E \cap F^c)$ and 
$\mu(F) = \mu(F \cap E) + \mu(F \cap E^c)$ with $(E \cap F^c) 
\cap (F \cap E^c) = \varnothing$. Then additivity of $\mu$ gives us 
\begin{align*}
    \mu(E) + \mu(F) &= \mu(E \cap F^c) + 2\mu(E \cap F) + \mu(F \cap E^c) \\ 
    &= \mu((E \cap F^c) \sqcup (E \cap F) \sqcup (F \cap E^c)) + \mu(E \cap F) \\ 
    &= \mu(E \cup F) + \mu(E \cap F).
\end{align*}
It easily follows from this that 
\[ \mu(E \cup F) = \mu(E) + \mu(F) - \mu(E \cap F) \leq \mu(E) + \mu(F). \] 
Next, we show that $\mu$ is monotone. Suppose that $E \subseteq F 
\subseteq \R$. Then we can write $F = (F \cap E) \sqcup (F \cap E^c) = E 
\sqcup (F \cap E^c)$, and since $\mu(E \cap F^c) \geq 0$, it follows from 
additivity that 
\[ \mu(E) \leq \mu(F \cap E^c) + \mu(E) = \mu(F). \] 

\begin{theo}[Uniqueness of Lebesgue measure]{theo:2.21}
    If $\mu : {\cal L} \to [0, \infty]$ is a measure on ${\cal L}$ such that 
    $\mu(I) = m(I) = \ell(I)$ for all open intervals $I \subseteq \R$, then 
    $\mu(E) = m(E)$ for all $E \in {\cal L}$. 
\end{theo}
\begin{pf}
    First, we deal with the case where $E \in {\cal L}$ is bounded. Then 
    there exists some $n \in \N$ such that $E \subseteq (-n, n)$. 
    By the subadditivity of $\mu$ and $m$, we have $\mu(E) \leq \mu((-n, n)) 
    = 2n$ and similarly, $m(E) \leq 2n$. 

    Let $\eps > 0$. There exist open intervals $(I_k)_{k=1}^\infty$ such that 
    $E \subseteq \bigcup_{k=1}^\infty I_k$ and 
    \[ \sum_{k=1}^\infty \ell(I_k) < m(E) + \eps. \] 
    By the monotonicity and subadditivity of $\mu$, we see that 
    \[ \mu(E) \leq \mu\!\left( \bigcup_{k=1}^\infty I_k \right) 
    \leq \sum_{k=1}^\infty \mu(I_k) = \sum_{k=1}^\infty \ell(I_k) 
    < m(E) + \eps. \] 
    Since $\eps > 0$ was arbitrary, we have $\mu(E) \leq m(E)$ for all 
    bounded sets $E \in {\cal L}$. By replacing $E$ above 
    with $(-n, n) \setminus E$ and repeating the argument, we can deduce that 
    $\mu((-n, n) \setminus E) \leq m((-n, n) \setminus E)$, so 
    \[ 2n = \mu((-n, n)) = \mu((-n, n) \setminus E) + \mu(E) 
    \leq m((-n, n) \setminus E) + m(E) = m((-n, n)) = 2n \] 
    and we have equality throughout. Hence, we have $\mu(E) = m(E)$ for all 
    bounded $E \in {\cal L}$. 

    Suppose now that $E \in {\cal L}$ is arbitrary. We can write 
    $E = \bigsqcup_{n=1}^\infty E_n$ where each $E_n \in {\cal L}$ is bounded 
    via the construction $E_n = E \cap (-n-1, n+1) \setminus (-n, n)$ for all 
    $n \in \N$. Then countable additivity of the measures assures that 
    \[ \mu(E) = \sum_{n=1}^\infty \mu(E_n) = \sum_{n=1}^\infty m(E_n) = m(E). 
    \qedhere \] 
\end{pf}

We have seen that not all Borel measurable sets are Lebesgue measurable. 
But the following structure theorem for ${\cal L}$ tells us that every 
$E \in {\cal L}$ is ``almost'' Borel measurable, up to throwing away 
sets of measure zero. 

\begin{theo}[Structure of Lebesgue measurable sets]{theo:2.22}
    Let $E \subseteq \R$. The following are equivalent: 
    \begin{enumerate}[(1)]
        \item $E$ is Lebesgue measurable. 
        \item For all $\eps > 0$, there exists an open set $V$ and closed 
        set $F$ such that $F \subseteq E \subseteq V$ and 
        $m(V \setminus F) < \eps$. 
        \item There exists an $F_\sigma$-set $A$ and a $G_\delta$-set 
        $B$ such that $A \subseteq E \subseteq B$ and $m(B \setminus A) = 0$. 
    \end{enumerate}
\end{theo}
\begin{pf}
    (1) $\Rightarrow$ (2). Let $K_n = [-n, n]$ for each $n \in \N$, and note that 
    $\R = \bigcup_{n=1}^\infty K_n$. Let $E$ be a Lebesgue measurable set 
    and let $\eps > 0$. Notice that $m(E \cap K_n) \leq m(K_n) = 2n < \infty$ 
    for all $n \in \N$. In particular, there exists an open set $V_n \subseteq \R$ 
    such that 
    \[ m(V_n \setminus (E \cap K_n)) < \frac{\eps}{2^{n+1}}. \] 
    Indeed, just take $V_n = \bigcup_{k=1}^\infty (a_k, b_k) \supseteq 
    E \cap K_n$ such that 
    \[ m(V_n) < m(E \cap K_n) + \frac{\eps}{2^{n+1}}, \] 
    then use monotonicity to get $m(V_n \setminus (E \cap K_n)) \leq m(V_n)$.

    Now, put $V = \bigcup_{n=1}^\infty V_n$. Observe that $V$ is open with 
    $E \subseteq V$, and 
    \[ m(V \setminus E) = m\!\left( \bigcup_{n=1}^\infty V_n \setminus E \right) 
    \leq m\!\left( \bigcup_{n=1}^\infty V_n \setminus (E \cap K_n) \right) 
    \leq \sum_{n=1}^\infty m(V_n \setminus (E \cap K_n)) 
    < \sum_{n=1}^\infty \frac{\eps}{2^{n+1}} = \frac{\eps}{2}. \] 
    We can repeat the same argument with $E^c$ replacing $E$. This gives us 
    an open set $W \supseteq E^c$ such that $m(W \setminus E^c) < \eps/2$. 
    Letting $F = W^c$, we see that $F$ is closed and $F \subseteq E \subseteq V$. 
    Using the additivity of Lebesgue measure gives 
    \[ m(V \setminus F) = m(V \setminus E) + m(E \setminus F) 
    = m(V \setminus E) + m(W \setminus E^c) < \frac{\eps}{2} + \frac{\eps}{2} 
    = \eps. \] 
    (2) $\Rightarrow$ (3). Applying (2) with $\eps_k = 1/k$ for all $k \in \N$, 
    we obtain closed sets $\{F_k\}_{k=1}^\infty$ and open sets $\{V_k\}_{k=1}^\infty$ 
    such that $F_k \subseteq E \subseteq V_k$ and $m(V_k \setminus F_k) 
    < 1/k$ for all $k \in \N$. Then $A = \bigcup_{k=1}^\infty F_k \subseteq E$ 
    is an $F_\sigma$-set and $B = \bigcap_{k=1}^\infty V_k \supseteq E$ is a 
    $G_\sigma$-set. We see that 
    \[ B \setminus A \subseteq V_k \setminus F_k \] 
    for all $k \in \N$. Note that $m(B \setminus A)$ is a well-defined quantity
    since $A$ and $B$ are both Borel and hence Lebesgue measurable. Then 
    monotonicity of $m$ implies that 
    \[ m(B \setminus A) \leq m(V_k \setminus F_k) < \frac1k. \] 
    Since this holds for all $k \in \N$, we in fact have $m(B \setminus A) = 0$. 

    (3) $\Rightarrow$ (1). Recall from Proposition~\ref{prop:2.17} that if 
    $W \subseteq \R$ with $m^*(W) = 0$, then $W \in {\cal L}$, and 
    monotonicity shows that every subset $W_0 \subseteq W$ is also in ${\cal L}$. 
    From (3), there exists an $F_\sigma$-set $A$ and a $G_\delta$-set $B$ 
    such that $A \subseteq E \subseteq B$. Writing $E = A \cup (E \setminus A)$, 
    notice that $A \in {\cal L}$ since it is Borel. On the other hand, 
    we have $E \setminus A \subseteq B \setminus A$ with $m(B \setminus A) = 0$, 
    so $E \setminus A \in {\cal L}$. Thus, $E$ is Lebesgue measurable as 
    ${\cal L}$ is a $\sigma$-algebra and so closed under unions. \qedhere 
\end{pf}

\subsection{Measurable Functions}\label{subsec:2.4}
Recall that when integrating a function $f : \R \to \R$ using Lebesgue's 
approach of partitioning the $y$-axis, we needed to measure the length 
of sets of the form 
\[ f^{-1}(a, b) = \{x \in \R : f(x) \in (a, b)\}. \] 
Therefore, we would like $m(f^{-1}(a, b))$ to be a well-defined quantity, or 
equivalently, we want $f^{-1}(a, b)$ to be a Lebesgue measurable set for 
any open interval $(a, b)$. 

More generally, let $f : X \to Y$ be an arbitrary function. Let $B$ and $\{B_i\}_{i=1}^\infty$ 
be subsets of $Y$. Then we have $f^{-1}(B^c) = (f^{-1}(B))^c$ and 
\[ f^{-1}\!\left( \bigcup_{i=1}^\infty B_i \right) = \bigcup_{i=1}^\infty 
f^{-1}(B_i), \] 
so the inverse image operation $f^{-1}$ respects all the $\sigma$-algebra operations. 

\newpage
In particular, every open set $U \subseteq \R$ is of the form 
\[ U = \bigcup_{n=1}^\infty (a_n, b_n). \] 
So our above condition where we want $f^{-1}(a, b) \in {\cal L}$ for any 
open interval $(a, b)$ is equivalent to having 
\[ f^{-1}(U) = \bigcup_{i=1}^\infty f^{-1}(a_n, b_n) \in {\cal L} \] 
for any open set $U \subseteq \R$. This leads us to the definition of a 
Lebesgue measurable function. 

\begin{defn}{defn:2.23}
    Let $D$ be a Lebesgue measurable set. A function $f : D \to \R$ is 
    called a {\bf Lebesgue measurable function} if for any open set 
    $U \subseteq \R$, we have 
    \[ f^{-1}(U) \in {\cal L}. \] 
\end{defn}

\begin{remark}{remark:2.24}
    This definition is reminiscent of the characterization of continuous
    functions via open sets; in fact, it is a generalization of it. 
    Suppose that $f : D \to \R$ is continuous where $D \in {\cal L}$. Then for 
    every open subset $U \subseteq \R$, we know that $f^{-1}(U)$ is relatively 
    open in $D$; that is, it is of the form $D \cap V$ where $V \subseteq \R$ 
    is open, so $f^{-1}(U) \in {\cal L}$. 
\end{remark}

Thinking more abstractly, we can generalize the above definition where the 
domain has a $\sigma$-algebra associated with it, and the codomain 
is a topological space. 

\begin{defn}{defn:2.25}
    Let $X$ be a set, and let ${\cal M}$ be a $\sigma$-algebra on $X$. 
    Let $Y$ be a topological space with the topology $\tau_Y$ of open sets. 
    Then $f : X \to Y$ is {\bf measurable} with respect to ${\cal M}$ 
    if for all open sets $U \in \tau_Y$, we have 
    \[ f^{-1}(U) \in {\cal M}. \] 
\end{defn}

\begin{prop}{prop:2.26}
    Let $f : X \to Y$ be as in the above definition. The following are equivalent:
    \begin{enumerate}[(1)]
        \item $f$ is measurable with respect to ${\cal M}$. 
        \item For all Borel sets $E \in {\cal B}_Y$, we have $f^{-1}(E) \in {\cal M}$. 
        \item If ${\cal F}$ is any collection of Borel sets in $Y$ that generates 
        ${\cal B}_Y$ (that is, ${\cal B}_Y = {\cal M}_{\cal F})$, then 
        $f^{-1}(E) \in {\cal M}$ for all $E \in {\cal F}$. 
    \end{enumerate}
\end{prop}
\begin{pf}
    We leave the proof as an exercise. Hint: Use the observation above where 
    the preimage is closed under all the operations of a $\sigma$-algebra. 
\end{pf}

\begin{prop}{prop:2.27}
    Let $D \in {\cal L}$, and let $f : D \to \R$ be a Lebesgue measurable 
    function. The following are equivalent: 
    \begin{enumerate}[(1)]
        \item $f$ is Lebesgue measurable. 
        \item $f^{-1}((\alpha, \infty)) \in {\cal L}$ for all $\alpha \in \R$. 
        \item $f^{-1}([\alpha, \infty)) \in {\cal L}$ for all $\alpha \in \R$. 
        \item $f^{-1}((-\infty, \alpha)) \in {\cal L}$ for all $\alpha \in \R$. 
        \item $f^{-1}((-\infty, \alpha]) \in {\cal L}$ for all $\alpha \in \R$. 
    \end{enumerate}
\end{prop}
\begin{pf}
    It is clear that (1) implies (2) since $(\alpha, \infty)$ is open
    for all $\alpha \in \R$. For (2) implies (3), observe that 
    \[ [\alpha, \infty) = \bigcap_{n=1}^\infty (\alpha - 1/n, \infty), \] 
    and since the inverse image preserves $\sigma$-algebra operations, 
    we obtain 
    \[ f^{-1}([\alpha, \infty)) = f^{-1}\!\left( \bigcap_{n=1}^\infty 
    (\alpha - 1/n, \infty) \right) = \bigcap_{n=1}^\infty 
    f^{-1}((\alpha - 1/n, \infty)) \in {\cal L}. \] 
    For (3) implies (4), we have $(-\infty, \alpha) = [\alpha, \infty)^c$ 
    and so 
    \[ f^{-1}((-\infty, \alpha)) = f^{-1}([\alpha, \infty)^c) 
    = (f^{-1}([\alpha, \infty)))^c \in {\cal L}. \] 
    For (4) implies (5), we can use a similar argument to (2) implies (3), 
    noting that 
    \[ (-\infty, \alpha] = \bigcap_{n=1}^\infty (-\infty, \alpha + 1/n). \] 
    Finally, for (5) implies (1), it suffices to show that $f^{-1}((a, b)) 
    \in {\cal L}$ for every bounded open interval $(a, b)$. Indeed, we 
    see that 
    \[ (a, b) = (-\infty, a]^c \cap \bigcup_{n=1}^\infty (-\infty, 
    b - 1/n], \] 
    and taking the inverse image completes the proof. 
\end{pf}

Using this characterization, we can see that there are many more examples 
of measurable functions other than continuous functions. 

\begin{exmp}{exmp:2.28}
    \begin{enumerate}[(1)]
        \item For a set $E \in {\cal L}$, the characteristic function 
        $\chi_E : \R \to \{0, 1\}$ on $E$ given by 
        \[ \chi_E(x) = \begin{cases}
            1, & \text{if } x \in E, \\ 
            0, & \text{if } x \notin E 
        \end{cases} \]
        is measurable, because given $\alpha \in \R$, we have 
        \[ f^{-1}((\alpha, \infty)) = \begin{cases}
            E, & \text{if } \alpha \in (0, 1), \\ 
            \varnothing, & \text{if } \alpha \geq 1, \\ 
            \R, & \text{if } \alpha \leq 1, 
        \end{cases} \] 
        all of which are Lebesgue measurable sets. 

        \item Let $f : \R \to \R$ be a non-decreasing function. 
        Then $f$ is Lebesgue measurable since given $\alpha \in \R$, we have 
        \[ f^{-1}((\alpha, \infty)) = \begin{cases}
            \varnothing, & \text{if $f(x) \leq \alpha$ for all $x \in \R$,} \\ 
            \R, & \text{if $f(x) > \alpha$ for all $x \in \R$,} \\ 
            [a, \infty), & \text{if $f(x) > \alpha$ for all $x \geq a$,} \\ 
            (a, \infty), & \text{if $f(x) > \alpha$ for all $x > a$.}
        \end{cases} \] 
        A similar computation shows that non-increasing functions are 
        Lebesgue measurable, and thus all monotone functions are Lebesgue 
        measurable. 

        \item Let $D \in {\cal L}$. Let $Y$ and $Z$ be topological spaces. 
        Let $g : Y \to Z$ be a continuous function, and let $f : D \to Y$ 
        be Lebesgue measurable. Then $g \circ f : D \to Z$ is Lebesgue 
        measurable. Indeed, if $U \subseteq Z$ is open, then 
        $g^{-1}(U)$ is open in $Y$. From this, we see that 
        \[ f^{-1}(g^{-1}(U)) = (g \circ f)^{-1}(U) \in {\cal L}. \] 
        More generally, we can allow for $g : Y \to Z$ to be {\bf Borel 
        measurable} where $g^{-1}(U)$ is a Borel set (rather than having 
        to be open) for any open subset $U \subseteq Z$, and the same proof 
        follows. 
    \end{enumerate}
\end{exmp}

It would be nice to extend Lebesgue measurability to complex-valued functions 
$f : D \to \C$ such that by taking the typical decomposition 
$f = u + iv$ of real-valued functions $u, v : D \to \R$, we have that $f$ 
is Lebesgue measurable if and only if $u$ and $v$ are. This is true, 
but first, we will prove a more general result. 

\begin{theo}{theo:2.29}
    Let $D \in {\cal L}$, and let $u, v : D \to \R$ be Lebesgue measurable 
    functions. Let $Y$ be a topological space and let $\Phi : \R^2 
    \to Y$ be a continuous function. Then $h : D \to Y$ given by 
    \[ h(x) = \Phi(u(x), v(x)) \] 
    is also Lebesgue measurable. 
\end{theo}
\begin{pf}
    Let $f : D \to \R^2$ be defined by $f(x) = (u(x), v(x))$. Then we have 
    $h = \Phi \circ f$. Since $\Phi$ is continuous, it is enough to 
    show that $f$ is measurable due to part (3) of Example~\ref{exmp:2.28}.
    
    Recall that the topology on $\R^2$ is generated by bounded open rectangles 
    \[ U = (a, b) \times (c, d), \] 
    so it suffices to show that $f^{-1}(U) \in {\cal L}$. We see that 
    \[ f^{-1}(U) = u^{-1}((a, b)) \cap v^{-1}((c, d)). \] 
    Since $u$ and $v$ are Lebesgue measurable, both the above preimages 
    are in ${\cal L}$, so we deduce that $f^{-1}(U) \in {\cal L}$. 
\end{pf}

\begin{cor}{cor:2.30}
    Let $D \in {\cal L}$. 
    \begin{enumerate}[(a)]
        \item If $u, v : D \to \R$ is measurable, then 
        $f = u+iv : D \to \C$ is also measurable. 
        \item If $f : D \to \C$ is measurable, then $u = \text{Re}(f)$,
        $v = \text{Im}(f)$, and $|f|$ are all measurable. 
        \item If $f, g : D \to \C$ are measurable, then $\alpha f$, 
        $f + g$, and $fg$ are measurable for all $\alpha \in \C$. 
    \end{enumerate}
\end{cor}
\begin{pf}~
    \begin{enumerate}[(a)]
        \item We can consider the topological isomorphism for $\R^2 \cong \C$. 
        Taking $\Phi(u, v) = u + iv$ in Theorem~\ref{theo:2.29}, we obtain the 
        result. 
        \item Notice that $\text{Re}$, $\text{Im}$, and $|\cdot|$ are all 
        continuous. Taking the composition of $f$ with these functions and 
        applying part (3) of Example~\ref{exmp:2.28} gives the result. 
        \item For any $\alpha \in \C$, it is clear that $\alpha f$ is 
        measurable. Assume for the moment that $f$ and $g$ are real-valued 
        functions. Applying Theorem~\ref{theo:2.29} with $\Phi(s, t) 
        = s + t$ and $\Phi(s, t) = st$ which are both continuous, we 
        find that $f + g$ and $fg$ are measurable in this special case. 

        Now, suppose that $f$ and $g$ are complex-valued. Suppose that 
        $f = u + iv$ and $g = w + iy$ where $u, v, w, y$ are real-valued. 
        By (1), we see that $u, v, w, y$ are measurable. Then 
        $f + g = (u + w) + i(v + y)$ and $fg = (uw - vy) + i(uy + vw)$ 
        are also measurable by what we have just shown. \qedhere 
    \end{enumerate}
\end{pf}

\begin{defn}{defn:2.31}
    Let $f : D \to \C$ and $g : E \to \C$ be functions where $D, E \in {\cal L}$. 
    We say that $f = g$ {\bf almost everywhere (a.e.)} if the set 
    \[ \{f \neq g\} := (E\,\Delta\,D) \cap \{x \in E \cap D : f(x) \neq g(x)\} \] 
    has Lebesgue measure zero, where $E\,\Delta\,D$ denotes the symmetric 
    difference of $E$ and $D$. 
\end{defn}

We can think of $f$ and $g$ as essentially the ``same'' function, even if 
$D$ and $E$ have trivial intersection. 

\begin{prop}{prop:2.32}
    Let $D \in {\cal L}$, and let $f : D \to \C$ be a measurable function.
    Suppose that $g : E \to \C$. If $f = g$ almost everywhere, then 
    $E \in {\cal L}$ and $g$ is a measurable function satisfying 
    \[ m(f^{-1}(U)) = m(g^{-1}(U)) \] 
    for all open sets $U \subseteq \C$. 
\end{prop}
\begin{pf}
    Observe that we can write 
    \begin{align*}
        E &= \{x \in E \cap D : f(x) = g(x)\} \cup (\{f \neq g\} \cap E) \\ 
        &= (D \setminus \{f \neq g\}) \cup (\{f \neq g\} \cap E). 
    \end{align*}
    The first set is the measurable set $D$ minus the set of measure zero
    $\{f \neq g\}$ whereas the second has measure zero, so $E \in {\cal L}$. 
    Now suppose that $U \subseteq \C$ is open. We have 
    \[ g^{-1}(U) = (f^{-1}(U) \setminus \{f \neq g\}) \cup 
    (g^{-1}(U) \cap \{f \neq g\}), \] 
    where the first set above is the set of elements in the preimage such that 
    $f$ and $g$ agree, and the second set consists of the elements where 
    either $f$ and $g$ disagree or $f$ is not defined where $g$ is. 
    Noting that $f^{-1}(U) \in {\cal L}$, it follows from an analogous 
    argument to above that $g^{-1}(U) \in {\cal L}$. Since $\{f \neq g\}$
    is null, it is easy to see that $m(g^{-1}(U)) = m(f^{-1}(U))$. 
\end{pf}

The previous proposition tells us that when $f = g$ almost everywhere, then 
$f = g$ in the eyes of Lebesgue measure. When $f$ is measurable, we are 
free to redefine $f$ on measure zero sets, or extend the domain by $f$ 
up to sets of measure zero. These operations do not ``materially'' change $f$. 

To end this section, we say some words on extended real-valued functions. 
We let $[-\infty, \infty] = \R \cup \{\pm\infty\}$ be the {\bf extended 
real numbers}, and we want to consider functions $f : D \to [-\infty, 
\infty]$ for some $D \in {\cal L}$. We will call $f$ {\bf Lebesgue 
measurable} if $f^{-1}((\alpha, \infty]) \in {\cal L}$ for all 
$\alpha \in \R$. In this case, we have $f^{-1}(\{\infty\}) \in {\cal L}$ 
and $f^{-1}(\{-\infty\}) \in {\cal L}$. Note that this notion only makes 
sense for real-valued functions, and not in $\C$. 

\subsection{Limits of Measurable Functions} \label{subsec:2.5}
Let $D \subseteq \R$, and let $(f_n)_{n=1}^\infty$ be a sequence of 
continuous functions defined on $D$. If $f_n \to f$ pointwise, recall 
that there is no guarantee that $f$ is continuous. For example, 
letting $f_n(x) = x^n$ on $[0, 1]$, then $f_n \to f$ where 
$f(x) = 0$ for $x \in [0, 1)$ and $f(1) = 1$. We can only guarantee 
that $f = \lim_{n\to\infty} f_n$ is continuous when the convergence 
is uniform (which is equivalent to convergence in the supremum norm). 

It turns out that measurability is a little nicer in this regard
and does not require uniform convergence. Recall that 
$\sup_{n\in\N} f_n$ and $\inf_{n\in\N} f_n$ are defined such that 
$(\sup_{n\in\N} f_n)(x) = \sup_{n\in\N} (f_n(x))$ and 
$(\inf_{n\in\N} f_n)(x) = \inf_{n\in\N} (f_n(x))$.

\begin{theo}{theo:2.33}
    Let $(f_n)_{n=1}^\infty$ be a sequence of measurable extended real-valued 
    functions defined on some $D \in {\cal L}$. Then $\sup_{n\in\N} f_n$ 
    and $\inf_{n\in\N} f_n$ are both measurable functions. 
\end{theo}
\begin{pf}
    Let $g = \sup_{n\in\N} f_n$ and take $\alpha \in \R$. Notice that 
    $g(x) > \alpha$ if and only if there exists some $n \in \N$ such that 
    $f_n(x) > \alpha$. Since each $f_n$ is measurable, we obtain
    \[ g^{-1}((\alpha, \infty]) = \bigcup_{n=1}^\infty f_n^{-1}((\alpha, 
    \infty]) \in {\cal L}, \] 
    and thus $g$ is measurable. It follows that $\inf_{n\in\N} f_n = 
    -\sup_{n\in\N} (-f_n)$ is also measurable. 
\end{pf}

\begin{cor}{cor:2.34}
    Let $D \in {\cal L}$. 
    \begin{enumerate}[(a)]
        \item Let $f, g : D \to [-\infty, \infty]$ be measurable 
        functions. Then $\max\{f, g\}$ and $\min\{f, g\}$ are 
        both measurable. 
        \item Let $f : D \to [-\infty, \infty]$ be a function. Define 
        $f^+ = \max\{f, 0\}$ and $f^- = -\min\{f, 0\}$, and observe that 
        $f = f^+ - f^-$. Then $f$ is measurable if and only if $f^+$ and 
        $f^-$ are both measurable. 
        \item Let $(f_n)_{n=1}^\infty$ be a sequence of real or complex
        valued functions on $D$. Suppose that $\lim_{n\to\infty} 
        f_n(x) = f(x)$ exists for all $x \in D$. Then $f$ is measurable. 
    \end{enumerate}
\end{cor}
\begin{pf}~
    \begin{enumerate}[(a)]
        \item Take the sequence $f_1 = f$ and $f_n = g$ for all $n \geq 2$, 
        then apply Theorem~\ref{theo:2.33} by observing that 
        $\max\{f, g\} = \sup_{n\in\N} f_n$ and $\min\{f, g\} = 
        \inf_{n\in\N} f_n$. 
        \item The forward direction follows from (a). The backwards 
        direction follows from part (c) of Corollary~\ref{cor:2.30}. 
        \item Let $(f_n)_{n=1}^\infty$ be a sequence of real valued 
        functions on $D$. Observe that 
        \[ \limsup_{n\to\infty} f_n = \inf_{n\in\N} \left( \sup_{k\geq n}
        f_k \right) \] 
        is measurable by applying Theorem~\ref{theo:2.33}, and so is 
        $\liminf_{n\to\infty} f_n = \sup_{n\in\N} (\inf_{k\geq n} f_k)$. 
        Since $f_n \to f$ pointwise, we see that  $f = \limsup_{n\to\infty} f_n 
        = \liminf_{n\to\infty} f_n$ is measurable. 
        
        For the complex valued case, we can write $f_n = u_n + iv_n$ 
        where $u_n$ and $v_n$ are real valued for each $n \in \N$. 
        Then $u_n \to u$ and $v_n \to v$ with $u$ and $v$ being measurable, 
        and thus $f = u + iv$ is measurable by Corollary~\ref{cor:2.30}. \qedhere 
    \end{enumerate}
\end{pf}

Corollary~\ref{cor:2.34} tells us that measurable functions are closed under 
pointwise limits. We now discuss some other modes of convergence. 

\begin{defn}{defn:2.35}
    Let $D \in {\cal L}$. A sequence of measurable functions $(f_n)_{n=1}^\infty$
    defined on $D$ converges {\bf pointwise almost everywhere} to a function 
    $f$ on $D$ if 
    \[ \lim_{n\to\infty} f_n(x) = f(x) \] 
    for almost every $x \in D$. We write $f_n \to f$ almost everywhere (a.e.). 
\end{defn}

Note that if we do not have pointwise convergence at a point, it 
either converges to the wrong value or the sequence doesn't converge 
at the point. The following lemma tells us that if $f_n \to f$ a.e., 
then it doesn't really matter what happens for points $x \in D$ where
$f(x) \neq \lim_{n\to\infty} f_n(x)$. 

\begin{lemma}{lemma:2.36}
    Let $D \in {\cal L}$. If $(f_n)_{n=1}^\infty$ is a sequence of measurable 
    functions on $D$ and $f_n \to f$ a.e., then $f$ is measurable. 
\end{lemma}
\begin{pf}
    Define the function 
    \[ g(x) = \begin{cases}
        \lim_{n\to\infty} f_n(x), & \text{if the limit exists,} \\ 
        f(x), & \text{otherwise.} 
    \end{cases} \] 
    Then by construction, we have $g = f$ almost everywhere. By 
    Proposition~\ref{prop:2.32}, $f$ is measurable. 
\end{pf}

Next, we talk about uniform convergence and almost uniform convergence. 

\begin{defn}{defn:2.37}
    Suppose that $(f_n)_{n=1}^\infty$ and $f$ are functions from $D$ 
    to $\C$, $\R$, or $[-\infty, \infty]$. 
    \begin{itemize}
        \item We say that $f_n \to f$ {\bf uniformly} if for all $\eps > 0$, 
        there exists $N \in \N$ such that $n \geq N$ implies
        \[ |f_n(x) - f(x)| < \eps. \] 
        \item We say that $f_n \to f$ {\bf almost uniformly} if for all 
        $\eps > 0$, there exists a measurable subset $E \subseteq D$ such that 
        $m(E) < \eps$ and $f_n \to f$ uniformly on $D \setminus E$. 
    \end{itemize}
\end{defn}

It is clear that uniform convergence implies almost uniform convergence 
and pointwise convergence, while pointwise convergence implies 
pointwise almost everywhere convergence. Our goal is to determine the 
relationship between almost uniform convergence and pointwise almost 
everywhere convergence. 

\begin{lemma}{lemma:2.38}
    If $f_n \to f$ almost uniformly, then $f_n \to f$ a.e. pointwise. 
\end{lemma}
\begin{pf}
    For each $m \in \N$, there exists a measurable subset $E_m \subseteq D$ 
    with $m(E_m) < 1/m$ such that $f_m \to f$ uniformly on $D \setminus E_m$. 
    Let $E = \bigcap_{m=1}^\infty E_m$. Using the continuity from above 
    property of the Lebesgue measure from Assignment 2, we obtain 
    \[ m(E) = \lim_{N\to\infty} m\!\left( \bigcap_{m=1}^N E_m \right) 
    \leq \lim_{N\to\infty} m(E_N) = 0 \] 
    since $(\bigcap_{m=1}^N E_m)_{N=1}^\infty$ is a decreasing sequence of sets. 
    Therefore, $E$ is a null set. 

    To show that $f_n \to f$ a.e. pointwise, it only remains to show that 
    $f_n \to f$ for all $x \in D \setminus E$. Observe that 
    \[ D \setminus E = D \setminus \bigcap_{m=1}^\infty E_m = 
    \bigcup_{m=1}^\infty (D \setminus E_m). \] 
    In particular, if $x \in D \setminus E$, then there exists some $m \in \N$ 
    such that $x \in D \setminus E_m$. Then $f_n \to f$ uniformly there, 
    which implies pointwise convergence. 
\end{pf}

What about the converse? Does almost everywhere convergence imply 
almost uniform convergence? It turns out that this is true under the 
assumption that $m(D) < \infty$. This is known as Egorov's Theorem. 

\begin{theo}[Egorov's Theorem]{theo:2.39}
    Suppose that $D \in {\cal L}$ is such that $m(D) < \infty$. 
    Let $(f_n)_{n=1}^\infty$ be a sequence of measurable functions 
    $f_n : D \to \C$, and let $f : D \to \C$. If $f_n \to f$ pointwise 
    a.e. on $D$, then $f_n \to f$ almost uniformly on $D$. 
\end{theo}
\begin{pf}
    We may assume without loss of generality that $f_n \to f$ pointwise 
    by replacing $D$ with the set $\{x \in D : \lim_{n\to\infty} f_n(x) =
    f(x)\}$; we are only dropping a set of measure zero. Given $n, k \in \N$, 
    we let 
    \[ E(n, k) = \bigcup_{m\geq n} \{x \in D : |f_m(x) - f(x)| \geq 1/k\}. \] 
    We can think of this as the set of points were $f_m$ is doing a bad 
    job of converging to $f$. 
    
    It is clear that $E(n, k)$ is measurable for all $n, k \in \N$. 
    Now, keep $k \in \N$ fixed. Then $(E(n, k))_{n=1}^\infty$ is a decreasing 
    sequence of sets with $\bigcap_{n=1}^\infty E(n, k) = \varnothing$. 
    Indeed, if there existed an element $x \in \bigcap_{n=1}^\infty E(n, k)$, 
    then for all $n \in \N$, there exists $m \in \N$ such that 
    $|f_m(x) - f(x)| \geq 1/k$. This contradicts our assumption that 
    $f_n \to f$ for all $x \in D$. By the continuity from above 
    of Lebesgue measure from Assignment 2, we have 
    \[ \lim_{n\to\infty} m(E(n, k)) = m\!\left( \bigcap_{n=1}^\infty E(n, k) 
    \right) = 0. \] 
    Note that we are using the fact that $m(E(1, n)) \leq m(D) < \infty$ here! 
    
    Let $\eps > 0$. Choose a subsequence $(n_k)_{k=1}^\infty$ such that 
    \[ m(E(n_k, k)) < \frac{\eps}{2^k}, \] 
    and set $E = \bigcup_{k=1}^\infty E(n_k, k)$. Then we have 
    \[ m(E) \leq \sum_{k=1}^\infty m(E(n_k, k)) < \sum_{k=1}^\infty 
    \frac{\eps}{2^k} = \eps. \] 
    For all $x \in D \setminus E$, we see that $x \in D \setminus E(n_k, k)$
    for all $k \in \N$. Then $|f_m(x) - f(x)| < 1/k$ for all $k \in \N$ 
    and $m \geq n_k$, implying that $f_m \to f$ uniformly on $D \setminus E$. 
\end{pf}

\begin{exmp}{exmp:2.40}
    For a simple example of Egorov's Theorem, consider the sequence of functions 
    $f_n(x) = x^n$ on $[0, 1]$. Recall that $f_n \to f$ pointwise on $[0, 1]$ where 
    $f(x) = 0$ on $[0, 1)$ and $f(1) = 1$. But for all $\eps > 0$, we see that 
    $f_n \to f$ uniformly on $[0, 1-\eps]$. In particular, we have $f_n \to f$ 
    almost uniformly on $[0, 1]$. 
\end{exmp}

\begin{remark}{remark:2.41}
    The assumption that $m(D) < \infty$ is necessary for Egorov's Theorem. 
    Take $D = \R$, and define the sequence of functions $f_n = \chi_{[-n, n]}$. 
    Then $\lim_{n\to\infty} f_n(x) = 1$ for all $x \in \R$. 

    We claim that $f_n \nrightarrow 1$ almost uniformly. Indeed, if 
    $E \subseteq \R$ satisfies $m(E) < \eps < \infty$, then $m(E^c) = \infty$. 
    That is, $E^c$ is unbounded. So there exists a sequence of points 
    $x_k \in E^c$ such that $x_k > k$ for all $k \in \N$. This gives us 
    \[ \sup_{x\in E^c} |f_k(x) - 1| \geq \sup_{k\in\N} |f_k(x_k) - 1| = 1, \] 
    and thus 
    \[ \limsup_{k\to\infty} \|f_k - 1\|_\infty = 1. \] 
    We conclude that $f_n \nrightarrow 1$ uniformly on $E^c = \R \setminus E$.
\end{remark}

We now consider simple functions, and see how measurable functions are 
well approximated by them. This will allow us to jump into Lebesgue integration.

\begin{defn}{defn:2.42}
    Let $D \in {\cal L}$. A function $f : D \to \C$ is {\bf simple} if 
    its range is finite. That is, we have 
    \[ \text{range}(f) = \{\alpha_1, \dots, \alpha_n\} \subseteq \C \] 
    where the $\alpha_i$ are distinct elements. 
\end{defn}

Notice that if we let $E_i = f^{-1}(\{\alpha_i\})$, then $D = 
\bigsqcup_{i=1}^n E_i$ and $f = \sum_{i=1}^n \alpha_i \chi_{E_i}$. 
Moreover, $f$ is measurable if and only if $E_i$ is measurable for all 
$1 \leq i \leq n$. 

In general, there are many ways to write a simple function as a linear 
combination of characteristic functions $\chi_{A_i}$, where $A_i \in 
{\cal L}$. But the form $f = \sum_{i=1}^n \alpha \chi_{E_i}$ above is really the 
most natural representation of $f$ as a linear combination of 
characteristic functions; we call it the {\bf standard representation} of $f$. 

The following theorem tells us that every measurable function is a limit 
of simple measurable functions. 

\begin{theo}{theo:2.43}
    \begin{enumerate}[(a)]
        \item Let $D \in {\cal L}$, and let $f : D \to [0, \infty]$ be a 
        measurable function. Then there exists a sequence $(f_n)_{n=1}^\infty$ 
        of non-negative simple measurable functions on $D$ such that 
        $0 \leq f_1 \leq f_2 \leq \cdots \leq f$ and for all $x \in D$, we have 
        \[ \lim_{n\to\infty} f_n(x) = \sup_{n\in\N} f_n(x) = f(x). \] 
        Moreover, for any $R > 0$, we have $f_n \to f$ uniformly on the 
        set $E_R = \{x \in D : f(x) \leq R\} \subseteq D$. 
        \item Let $D \in {\cal L}$ and let $f : D \to \C$ be measurable. 
        Then there exists a sequence of simple measurable functions 
        $f_n : D \to \C$ such that 
        \[ f(x) = \lim_{n\to\infty} f_n(x) \] 
        for all $x \in D$ and $0 \leq |f_1| \leq |f_2| \leq \cdots \leq |f|$. 
        Moreover, for all $R > 0$, we have $f_n \to f$ uniformly on the set 
        $E_R = \{x \in D : |f_n(x)| \leq R\} \subseteq D$. 
    \end{enumerate}
\end{theo}
\begin{pf}
    For part (a), we first prove a special case. Let $g : [0, \infty] 
    \to [0, \infty]$ be defined by $g(t) = t$. We want to approximate 
    $g$ by non-decreasing piecewise constant functions $g_n$. 
    
    Fix $n \in \N$. For $0 \leq t < \infty$, there is a unique $k \in \N$ 
    such that $t \in [\frac{k}{2^n}, \frac{k+1}{2^n}]$. We write 
    $k = k(t, n)$. Then, we define $g_n : [0, \infty] \to [0, \infty)$ by 
    \[ g_n(t) = \begin{cases}
        k(n, t)/2^n, & \text{if } t < n, \\ 
        n, & \text{if } n \leq t \leq \infty. 
    \end{cases} \]
    Observe that for all $n \in \N$, $g_n$ is simple and non-decreasing. 
    Moreover, for all $t \in [0, \infty]$, we have 
    \[ g_n(t) \leq g_{n+1}(t) \leq g(t) = t. \] 
    One more thing to note is that for $t \in [0, n]$, we have 
    \[ g_n(t) = \frac{k}{2^n} \leq t \leq \frac{k}{2^n} + \frac{1}{2^n}
    = g_n(t) + \frac{1}{2^n}. \]  
    This implies that $|g_n(t) - g(t)| \leq 2^{-n}$, so by picking 
    $n$ large enough so that $n > R$, we have $g_n \to g$ uniformly on 
    the bounded interval $[0, R]$. Finally, we see that $g_n(t) \to g(t)$ 
    for all $t \in [0, \infty]$. 

    We now prove the general case of part (a). Let $f : D \to [0, \infty]$ 
    be a measurable function and define $f_n = g_n \circ f$ for all 
    $n \in \N$. Then each $f_n$ is simple as the $g_n$'s are simple. 
    Moreover, for all $x \in D$, we have 
    \[ f_n(x) = g_n(f(x)) \leq g_{n+1}(f(x)) = f_{n+1}(x), \] 
    so the $f_n$'s are increasing. We see that $f_n(x) \to f(x)$ for all 
    $x \in D$ since $g_n(t) \to t$. 

    Let $R > 0$ and suppose that $x \in E_R = \{x \in D : f(x) \leq R\}$. 
    Let $t = f(x)$. Then 
    \[ |f(x) - f_n(x)| = |t - g_n(t)| < 2^{-n} \] 
    provided that $t \leq R < n$. Thus, $f_n \to f$ uniformly on $E_R$. 

    Finally, we claim that $f_n$ is measurable for all $n \in \N$. 
    Indeed, we saw earlier that the $g_n$ are non-decreasing functions for all 
    $n \in \N$, so $g_n^{-1}((\alpha, \infty])$ is an interval for all 
    $\alpha \in \R$, say $I$. Then 
    \[ f_n^{-1}((\alpha, \infty]) = f^{-1}(g_n^{-1}((\alpha, \infty])) 
    = f^{-1}(I) \in {\cal L} \] 
    since $f$ is measurable. This completes the proof of part (a). 

    To prove part (b), we can break $f : D \to \C$ into real and imaginary parts 
    $u, v : D \to \R$ so that $f = u + iv$. We can then write these as 
    $u = u^+ - u^-$ and $v = v^+ - v^-$, which are non-negative functions on 
    $D$. We leave it as an exercise to verify the remaining details. 
\end{pf}

\subsection{Lebesgue Integration} \label{subsec:2.6}
We will iteratively define the Lebesgue integral for measurable functions $f$
on a measurable set $E \in {\cal L}$. First, we consider the Lebesgue integral 
of non-negative simple measurable functions. We then use this to define 
the integral of $f$ where $f : E \to [0, \infty]$ is measurable, and 
derive a number of consequences of the definition. 

\begin{defn}{defn:2.44}
    Let $\varphi : \R \to [0, \infty)$ be a simple and measurable function
    with range $\{\alpha_1, \dots, \alpha_n\}$. Let $\varphi = \sum_{i=1}^n 
    \alpha_i \chi_{E_i}$ be its standard form with 
    $E_i = \varphi^{-1}(\{\alpha_i\}) \in {\cal L}$. Then the 
    {\bf Lebesgue integral} of $\varphi$ is 
    \[ \int_{\R} \varphi\dd m = \int_{\R} \varphi = \int \varphi 
    := \sum_{i=1}^n \alpha_i m(E_i). \] 
\end{defn}

By convention, we will define $0 \cdot \infty = \infty \cdot 0 = 0$. For example, 
if $\alpha_i = 0$ and $m(E_i) = \infty$, then $\alpha_i m(E_i) = 0$. 

Let us now extend this definition to arbitrary non-negative measurable 
extended real-valued functions. 

\begin{defn}{defn:2.45}
    Let $f : \R \to [0, \infty]$ be a measurable function. Then we 
    define the {\bf Lebesgue integral} of $f$ to be 
    \[ \int_{\R} f\dd m = \int_{\R} f = \int f 
    := \sup\left\{ \int_{\R} \varphi \dd m : 0 \leq \varphi \leq f, 
    \text{ where $\varphi$ is simple and measurable} \right\}. \] 
    For a measurable set $E \in {\cal L}$, we define 
    \[ \int_E f\dd m = \int_{\R} f\chi_E \dd m. \] 
\end{defn}

We now list some basic properties of the Lebesgue integral which follow 
easily from the definition.

\begin{prop}{prop:2.46}
    \begin{enumerate}[(a)]
        \item If $E \in {\cal L}$ and $0 \leq f \leq g$ are measurable 
        functions, then 
        \[ \int_E f \dd m \leq \int_E g \dd m. \] 
        \item If $A \subseteq B$ are measurable sets and $f \geq 0$ is a 
        measurable function, then 
        \[ \int_A f \dd m \leq \int_B f \dd m. \] 
        \item Let $E \in {\cal L}$. If $f \geq 0$ is a measurable function 
        and $c \in [0, \infty)$, then 
        \[ \int_E cf\dd m = c \int_E f\dd m. \] 
        \item Let $E \in {\cal L}$, and suppose that $f \geq 0$ is a measurable 
        function such that $f(x) = 0$ for all $x \in E$. Then we have 
        \[ \int_E f\dd m = 0. \] 
        \item If $E \in {\cal L}$ with $m(E) = 0$, then for any measurable 
        function $f \geq 0$, we have 
        \[ \int_E f\dd m = 0. \] 
    \end{enumerate}
\end{prop}

We want to have additivity of the Lebesgue integral. This will follow from 
the Lebesgue Monotone Convergence Theorem. This is a very powerful theorem, 
and it actually says a lot more: we actually have countable additivity 
of the Lebesgue integral! 

Towards this, we will first prove some useful properties concerning 
non-negative simple measurable functions. It turns out that we can 
actually construct a measure by applying the Lebesgue integral to a 
non-negative simple measurable function, and that we have additivity 
of the Lebesgue integral when the integrands are simple functions. 

\begin{prop}{prop:2.47}
    Let $\varphi$ and $\psi$ be non-negative simple measurable functions. 
    \begin{enumerate}[(1)]
        \item For any $E \in {\cal L}$, define 
        \[ \mu(E) = \int_E \varphi\dd m = \int_{\R} \varphi\chi_E \dd m. \] 
        Then $\mu$ is a measure. 
        \item We have 
        \[ \int_{\R} (\varphi + \psi)\dd m = \int_{\R} \varphi\dd m 
        + \int_{\R} \psi\dd m. \] 
    \end{enumerate}
\end{prop}
\begin{pf}
    Let $\varphi = \sum_{i=1}^n \alpha_i \chi_{E_i}$ and $\psi = 
    \sum_{j=1}^m \beta_j \chi_{F_j}$ be the standard forms of 
    $\varphi$ and $\psi$ respectively. 
    \begin{enumerate}[(1)]
        \item For $E \in {\cal L}$, observe that 
        \[ \mu(E) = \int_{\R} \varphi\chi_E \dd m = \int_{\R} 
        \sum_{i=1}^n \alpha_i \chi_{E_i \cap E}\dd m 
        = \sum_{i=1}^m \alpha_i m(E_i \cap E). \] 
        It is clear that $\mu(\varnothing) = 0$ and $\mu(E) \geq 0$ 
        since $\varphi \geq 0$. For countable additivity, suppose that 
        $E = \bigsqcup_{k=1}^\infty A_k$ for some $A_k \in {\cal L}$. 
        Then we obtain 
        \begin{align*}
            \mu(E) &= \sum_{i=1}^n \alpha_i m\!\left( E \cap \bigsqcup_{k=1}^\infty A_k \right) 
            = \sum_{i=1}^n \alpha_i \sum_{k=1}^\infty m(E_i \cap A_k) \\
            &= \sum_{k=1}^\infty \sum_{i=1}^n \alpha_i m(E_i \cap A_k)  
            = \sum_{k=1}^\infty \mu(A_k). 
        \end{align*}
        Thus, we conclude that $\mu$ is a measure. 

        \item Let $E_{ij} = E_i \cap F_j$ for $1 \leq i \leq n$ and $1 \leq j \leq m$. 
        We see that $\R = \bigsqcup_{i,j} E_{ij}$. Note that $\varphi + \psi$ 
        is a non-negative simple measurable function, so 
        $\mu(E) = \int_E (\varphi + \psi)\dd m$ is a measure by (1). 
        This gives us 
        \[ \int_{\R} (\varphi + \psi)\dd m = \mu(\R) = 
        \sum_{i,j} \mu(E_{ij}) = \sum_{i,j} \int_{E_{ij}} (\varphi + \psi)\dd m. \] 
        Now, observe that $\varphi + \psi$ takes on value $\alpha_i + \beta_j$ 
        on each $E_{ij} = E_i \cap F_j$. It follows that 
        \begin{align*}
            \int_{\R} (\varphi + \psi)\dd m = \mu(\R) 
            &= \sum_{i,j} \int_{E_{ij}} (\varphi + \psi)\dd m \\ 
            &= \sum_{i,j} \int_{\R} (\alpha_i + \beta_j) \chi_{E_i \cap F_j}\dd m \\ 
            &= \sum_{i,j} (\alpha_i + \beta_j) m(E_i \cap F_j) \\ 
            &= \sum_{i,j} \alpha_i m(E_i \cap F_j) + \sum_{i,j} \beta_j m(E_i \cap F_j) \\ 
            &= \sum_{i,j} \int_{E_{ij}} \varphi\dd m + \sum_{i,j} \int_{E_{ij}} \psi\dd m \\ 
            &= \int_{\R} \varphi\dd m + \int_{\R} \psi\dd m, 
        \end{align*}
        where the final equality is obtained by applying (1) to $\varphi$ and 
        $\psi$ individually. \qedhere 
    \end{enumerate}
\end{pf}

We are now ready to state and prove the Lebesgue Monotone Convergence Theorem, 
which is our first big limit theorem. It tells us that the 
Lebesgue integral is much better behaved than the Riemann integral. 

\begin{theo}[Lebesgue Monotone Convergence Theorem]{theo:2.48}
    Let $(f_n)_{n=1}^\infty$ be a sequence of non-negative extended 
    real-valued measurable functions on $E \in {\cal L}$. Suppose that 
    $0 \leq f_1 \leq f_2 \leq \cdots$ and for all $x \in E$, let 
    \[ f(x) = \lim_{n\to\infty} f_n(x) = \sup_{n\in\N} f_n(x). \] 
    Then $f$ is measurable and we have 
    \[ \int_E f\dd m = \lim_{n\to\infty} \int_E f_n\dd m. \] 
\end{theo}

We state an immediate corollary of Lebesgue's Monotone Convergence Theorem, 
which gives us a concrete way of defining the Lebesgue integral of $f$. 

\begin{cor}{cor:2.49}
    Let $E \in {\cal L}$. Let $f \geq 0$ be measurable. For any sequence 
    $(\varphi_n)_{n=1}^\infty$ of non-negative simple measurable functions 
    with $0 \leq \varphi_1 \leq \varphi_2 \leq \cdots \leq f$ and 
    $\lim_{n\to\infty} \varphi_n(x) = f(x)$ for all $x \in E$, we have 
    \[ \int_E f\dd m = \lim_{n\to\infty} \int_E \varphi_n \dd m. \] 
\end{cor}

Let us now prove the theorem. 

\begin{pf}
    We already know that $f$ is measurable by Theorem~\ref{theo:2.33}.
    By assumption, we have $0 \leq f_1 \leq f_2 \leq \cdots \leq f = 
    \sup_{n\in\N} f_n$. Then for all $n \in \N$, using part (a) of 
    Proposition~\ref{prop:2.46} gives us 
    \[ 0 \leq \int_E f_n\dd m \leq \int_E f_{n+1}\dd m \leq \int_E 
    f\dd m. \] 
    From this, we obtain 
    \[ I(f) = \lim_{n\to\infty} \int_E f_n\dd m = \sup_{n\in\N} \int_E 
    f_n\dd m \leq \int_E f\dd m. \] 
    This is actually one direction of the desired equality, so 
    it only remains to show that 
    \begin{equation}\label{eq:2.1}
        I(f) \geq \int_E f_n\dd m.
    \end{equation} 
    Fix $\eps \in (0, 1)$ and a simple and measurable function $\varphi$ 
    such that $0 \leq \varphi \leq f$. We claim it suffices to show that 
    \begin{equation}\label{eq:2.2}
        I(f) \geq \eps \int_E \varphi\dd m. 
    \end{equation}
    Why is this enough? If we take $\eps \to 1$ in \eqref{eq:2.2}, 
    then this implies that $I(f) \geq \int_E \varphi\dd m$ for all 
    simple measurable functions $0 \leq \varphi \leq f$. Then taking 
    the supremum over $\varphi$ on the right hand side implies \eqref{eq:2.1}. 

    Set $E_n = \{x \in \R : f_n(x) \geq \eps\varphi(x)\}$. By definition, we 
    have $E_n \in {\cal L}$. Moreover, since $(f_n)_{n=1}^\infty$ is an 
    increasing sequence, we have $E_1 \subseteq E_2 \subseteq \cdots$ 
    with $E = \bigcup_{n=1}^\infty E_n$. Applying Proposition~\ref{prop:2.47}, 
    we see that $\mu : {\cal L} \to [0, \infty]$ defined by 
    \[ \mu(F) = \int_F \eps\varphi\dd m \] 
    is a measure. Then by the continuity of measure from below, we get 
    \[ \int_E \eps\varphi\dd m = \mu(E) = \lim_{n\to\infty} \mu(E_n) 
    = \lim_{n\to\infty} \int_{E_n} \eps\varphi\dd m. \] 
    But $\eps\varphi \leq f_n$ on $E_n$ for all $n \in \N$ by definition, 
    and thus 
    \[ \int_{E_n} \eps\varphi\dd m \leq \int_{E_n} f_n\dd m 
    \leq \int_E f_n\dd m \leq \lim_{n\to\infty} \int_E f_n\dd m = I(f). \] 
    Taking $n \to \infty$ gives us $\eqref{eq:2.2}$, which completes the proof. 
\end{pf}

We now discuss many consequences of Lebesgue's Monotone Convergence Theorem. 
The first one we will prove is the countable additivity of the Lebesgue 
integral for general non-negative measurable functions. 

\begin{cor}{cor:2.50}
    Let $(f_n)_{n=1}^\infty$ be a sequence of non-negative measurable functions
    on $E \in {\cal L}$. Then $\sum_{n=1}^\infty f_n$ is measurable and we have 
    \[ \int_E \left( \sum_{n=1}^\infty f_n \right) \textrm{d}m 
    = \sum_{n=1}^\infty \int_E f_n\dd m. \] 
\end{cor}
\begin{pf}
    We first prove finite additivity. Suppose that $(\varphi_n)_{n=1}^\infty$ 
    and $(\psi_n)_{n=1}^\infty$ are non-negative simple measurable 
    functions such that $\varphi_n \nearrow f_1$ and $\psi_n \nearrow f_2$
    (where $\varphi_n \nearrow f_1$ means that $\varphi_1 \leq 
    \varphi_2 \leq \cdots$ and $\varphi_n \to f_1$ pointwise). Then 
    we see that $\varphi_n + \psi_n \nearrow f_1 + f_2$. It follows that 
    \begin{align*}
        \int_E (f_1 + f_2)\dd m 
        &= \lim_{n\to\infty} \int_E (\varphi_n + \psi_n)\dd m \\ 
        &= \lim_{n\to\infty} \left( \int_E \varphi_n\dd m + \int_E 
        \psi_n\dd m \right) \\ 
        &= \lim_{n\to\infty} \int_E \varphi_n\dd m + \lim_{n\to\infty} 
        \int_E \psi_n\dd m \\ 
        &= \int_E f_1\dd m + \int_E f_2\dd m, 
    \end{align*}
    where the first and last equality follow from Lebesgue's Monotone Convergence 
    Theorem, and the second equality follows from Proposition~\ref{prop:2.47}
    since $\varphi_n$ and $\psi_n$ are simple for all $n \in \N$. 
    By induction, we obtain finite additivity. 

    Next, observe that $f_1 + \cdots + f_N \nearrow \sum_{n=1}^\infty f_n$. 
    Applying Lebesgue's Monotone Convergence Theorem gives 
    \[ \int_E \left( \sum_{n=1}^\infty f_n \right) \textrm{d}m 
    = \lim_{N\to\infty} \int_E \left( \sum_{n=1}^N f_n \right) 
    \textrm{d}m = \lim_{N\to\infty} \sum_{n=1}^N \int_E f_n\dd m 
    = \sum_{n=1}^\infty \int_E f_n\dd m. \qedhere \] 
\end{pf}

The next corollary tells us we can construct a measure from general 
non-negative measurable functions as well, not just simple ones. 

\begin{cor}{cor:2.51}
    Suppose that $f : \R \to [0, \infty]$ is measurable. Define 
    $\mu : {\cal L} \to [0, \infty]$ by 
    \[ \mu(E) = \int_E f\dd m. \] 
    Then $\mu$ is a measure. 
\end{cor}
\begin{pf}
    It is clear that $\mu(\varnothing) = 0$ and $\mu(E) \geq 0$ for all 
    $E \in {\cal L}$. Towards countable additivity, suppose that 
    $E = \bigsqcup_{n=1}^\infty E_n$. Let $f_n = f\chi_{E_n}$ for each 
    $n \in \N$. Observe that 
    \[ f\chi_E = \sum_{n=1}^\infty f\chi_{E_n} = \sum_{n=1}^\infty f_n, \] 
    so applying Corollary~\ref{cor:2.50} gives us 
    \[ \mu(E) = \int_E f\dd m 
    = \int_{\R} f\chi_E \dd m 
    = \int_{\R} \left( \sum_{n=1}^\infty f_n \right)\textrm{d}m 
    = \sum_{n=1}^\infty \int_{\R} f_n\dd m 
    = \sum_{n=1}^\infty \int_{E_n} f\dd m 
    = \sum_{n=1}^\infty \mu(E_n). \qedhere \] 
\end{pf}

We can think of $f$ as the ``density of $\mu$ relative to $m$''. 
In the literature, this is often written $f = \frac{{\rm d}\mu}{{\rm d}m}$, 
and we call this the Radon-Nikodym derivative. This topic is focused on 
more heavily in PMATH 451. 

The next result makes intuitive sense for how the integral should behave 
when $f = 0$ almost everywhere. 

\begin{cor}{cor:2.52}
    If $f \geq 0$ is measurable and $E \in {\cal L}$, then $f = 0$ 
    almost everywhere on $E$ if and only if 
    \[ \int_E f\dd m = 0. \] 
\end{cor}
\begin{pf}
    Suppose that $f = 0$ almost everywhere on $E$. Then we have 
    \[ \int_E f\dd m = \int_{\{x\in E\,:\,f(x) = 0\}} f\dd m 
    + \int_{\{x\in E\,:\,f(x) > 0\}} f\dd m = 0, \] 
    where the first integral is $0$ because $f(x) = 0$ everywhere on that set 
    and the second integral is $0$ because $m(\{x\in E : f(x) > 0\}) = 0$. 

    Conversely, suppose that $\int_E f\dd m = 0$. For all $n \in \N$, define 
    \[ E_n = \{x \in E : f(x) \geq 1/n\} \subseteq E. \] 
    We see that $E_n \nearrow \{x \in E : f(x) > 0\}$. For all $n \in \N$, 
    the monotonicity of the Lebesgue integral implies that 
    \[ 0 = \int_E f\dd m \geq \int_{E_n} f\dd m \geq \int_{E_n} 
    \frac1n\dd m \geq 0. \] 
    This means that $\frac1n m(E_n) = 0$ and hence $m(E_n) = 0$ for all 
    $n \in \N$. By the continuity of measure, we get 
    \[ m(\{x \in E : f(x) > 0\}) = \lim_{n\to\infty} m(E_n) = 0. \] 
    Thus, $f = 0$ almost everywhere. 
\end{pf}

\begin{remark}{remark:2.53}
    The monotonicity of the sequence $(f_n)_{n=1}^\infty$ in Lebesgue's 
    Monotone Convergence Theorem is essential. For example, let 
    $f_n = n\chi_{[0, 1/n]}$ for all $n \in \N$. Then $f_n \to f$ pointwise 
    where $f(0) = \infty$ and $f(x) = 0$ for all $x \neq 0$. 
    This is certainly not a monotone limit. By Corollary~\ref{cor:2.52}, we have 
    $\int_{\R} f\dd m = 0$. But each $f_n$ is simple and we see that 
    \[ \int_{\R} f_n\dd m = n \cdot \frac1n = 1 \] 
    for all $n \in \N$. In particular, the conclusion of Lebesgue's 
    Monotone Convergence Theorem fails, since 
    \[ 0 = \int_{\R} f\dd m < \lim_{n\to\infty} \int_{\R} f_n\dd m = 1. \] 
\end{remark}

Does the above inequality always hold? That is, do we have 
\[ \int_{\R} \left( \lim_{n\to\infty} f_n \right) \textrm{d}m 
\leq \lim_{n\to\infty} \int_{\R} f_n\dd m \] 
given a sequence of non-negative measurable functions $(f_n)_{n=1}^\infty$? 
The problem is that these limits may not make sense. But by making a slight 
relaxation, it turns out that this does hold in general. This is known 
as Fatou's Lemma. 

\begin{theo}[Fatou's Lemma]{theo:2.54}
    If $(f_n)_{n=1}^\infty$ be a sequence of non-negative measurable functions, then
    \[ \int_{\R} \left( \liminf_{n\to\infty} f_n \right) \textrm{d}m 
    \leq \liminf_{n\to\infty} \int_{\R} f_n\dd m. \] 
\end{theo}
\begin{pf}
    By definition, we have 
    \[ \liminf_{n\to\infty} f_n = \sup_{n\in\N} \left( \inf_{k\geq n} f_k \right). \] 
    Setting $h_n = \inf_{k\geq n} f_k$, we see that $h_n \leq h_{n+1} \leq 
    \cdots$ and $h_n = \inf_{k\geq n} f_k \leq f_k$ for all $k \geq n$. This 
    gives us 
    \[ \int_{\R} h_n\dd m = \int_{\R} \left( \inf_{k\geq n} f_k \right)\dd m 
    \leq \int_{\R} f_k\dd m \] 
    for all $k \geq n$, and thus 
    \[ \int_{\R} h_n\dd m \leq \inf_{k\geq n} \int_{\R} f_k\dd m. \] 
    Finally, applying Lebesgue's Monotone Theorem implies that 
    \[ \int_{\R} \left( \liminf_{n\to\infty} f_n \right) \textrm{d}m 
    = \lim_{n\to\infty} \int_{\R} h_n\dd m 
    = \sup_{n\in\N} \int_{\R} h_n\dd m 
    \leq \sup_{n\in\N} \left( \inf_{k\geq n} \int_{\R} f_k\dd m \right) 
    = \liminf_{n\to\infty} \int_{\R} f_n\dd m. \qedhere \] 
\end{pf}
\newpage
\section{Banach and Hilbert Spaces}\label{sec:3}

\subsection{Banach and Hilbert Spaces}\label{subsec:3.1}
Functional analysis is the study of normed vector spaces and the continuous 
linear maps between them. Some of the most important examples of complete 
metric spaces are Banach spaces. Recall from PMATH 351 that a metric space is 
complete if every Cauchy sequence is convergent. We first give the definition 
of a normed vector space and a Banach space, and then illustrate these concepts 
with some examples. 

\begin{defn}{defn:3.1}
    A {\bf normed vector space} is a pair $(V, \|\cdot\|)$ where $V$ is a 
    vector space over $\C$ and $\|\cdot\| : V \to [0, \infty)$ is a 
    {\bf norm} on $V$. That is, $\|\cdot\|$ satisfies the following properties: 
    \begin{enumerate}[(1)]
        \item $\|v\| \geq 0$ for all $v \in V$, and $\|v\| = 0$ 
        if and only if $v = 0$; 
        \item $\|\alpha v\| = |\alpha|\|v\|$ for all $\alpha \in \C$ and 
        $v \in V$; and 
        \item $\|v + w\| \leq \|v\| + \|w\|$ for all $v, w \in V$. 
    \end{enumerate}
\end{defn}

Note that the property that $\|v\| = 0$ if and only if $v = 0$ says that 
$\|\cdot\|$ is non-degenerate; without this requirement, $\|\cdot\|$ is 
called a {\bf semi-norm}. Moreover, given $v, w \in V$, the above properties 
together give rise to a canonical metric defined by 
\[ d(v, w) = \|v - w\|. \]
Thus, a normed vector space is a special case of a metric space. 

\begin{defn}{defn:3.2}
    A {\bf Banach space} is a complete normed vector space. 
\end{defn}

We begin with the prototypical example of a normed vector space. 

\begin{exmp}{exmp:3.3} 
    Let $v = (v_1, v_2, \dots, v_n) \in \C^n$. We define the $1$-norm on $\C^n$ by 
    \[ \|v\|_1 = |v_1| + |v_2| + \cdots + |v_n|. \] 
    The $2$-norm on $\C^n$ is given by 
    \[ \|v\|_2 = (|v_1|^2 + |v_2|^2 + \cdots + |v_n|^2)^{1/2}. \] 
    More generally, for $1 \leq p < \infty$, the $p$-norm on $\C^n$ is 
    \[ \|v\|_p = (|v_1|^p + |v_2|^p + \cdots + |v_n|^p)^{1/p}. \] 
    The $\infty$-norm or supremum norm on $\C^n$ is 
    \[ \|v\|_\infty = \max\{|v_1|, |v_2|, \dots, |v_n|\}. \] 
    In fact, $\C^n$ equipped with each of these norms is a finite-dimensional 
    Banach space. 
\end{exmp}

We can also consider the continuous analogues for the space $C[a, b]$ 
of continuous functions on $[a, b]$.

\begin{exmp}{exmp:3.4}
    Let $f \in C[a, b]$. The $1$-norm on $C[a, b]$ is 
    \[ \|f\|_1 = \int_a^b |f(x)|\dd x. \] 
    The $2$-norm on $C[a, b]$ is given by 
    \[ \|f\|_2 = \left( \int_a^b |f(x)|^2\dd x \right)^{\!1/2}. \] 
    Finally, the $\infty$-norm on $C[a, b]$ is 
    \[ \|f\|_\infty = \sup_{x\in [a, b]} |f(x)|. \] 
    Note that $C[a, b]$ is a Banach space equipped with the supremum norm, 
    but it is not complete with respect to the $1$-norm and $2$-norm. 
\end{exmp}

Next, we consider inner product spaces and Hilbert spaces. The latter 
is a special case of a Banach space. 

\begin{defn}{defn:3.5} 
    An {\bf inner product space} is a pair $(H, \langle \cdot, \cdot \rangle)$ 
    where $H$ is a vector space over $\C$ and $\langle \cdot, \cdot \rangle
    : H \times H \to \C$ satisfies the following properties.
    \begin{enumerate}[(1)]
        \item $\langle x, y \rangle = \overline{\langle y, x \rangle}$
        for all $x, y \in H$. 
        \item For fixed $y \in H$, the map $x \mapsto \langle x, y \rangle$
        is a linear functional.
        \item {\bf Positive semidefiniteness:} $\langle x, x \rangle \geq 0$ 
        for all $x \in H$.
        \item {\bf Non-degeneracy:} $\langle x, x \rangle = 0$ if and only if 
        $x = 0$. 
    \end{enumerate} 
\end{defn}

Note that properties (1) and (2) imply that $\langle \cdot, \cdot \rangle$ 
is a {\bf sesquilinear form} on $H$. That is, $\langle \cdot, \cdot \rangle$ 
is linear in the left variable, and conjugately linear in the right 
variable. Moreover, we can define a norm on $H$ by 
\[ \|x\| = \langle x, x \rangle^{1/2}. \] 
We verify this fact in the following proposition. 

\begin{prop}{prop:3.6}
    Let $(H, \langle \cdot, \cdot \rangle)$ be an inner product space. 
    \begin{enumerate}[(1)]
        \item {\bf Cauchy-Schwarz inequality:} For all $x, y \in H$, 
        we have $|\langle x, y \rangle| \leq \|x\|\|y\|$. 
        \item $(H, \langle \cdot, \cdot \rangle)$ is a normed vector space. 
    \end{enumerate}
\end{prop}
\begin{pf}~
    \begin{enumerate}[(1)]

        \item Fix $x, y \in H$. The result is clear if $\|x\| = 0$ or 
        $\|y\| = 0$. Assume now that $x \neq 0$ and $y \neq 0$. For 
        $t \in \R$, define $p(t) = \|x - ty\|^2 \geq 0$. Observe that 
        \[ p(t) = \langle x - ty, x - ty \rangle = \|x\|^2 
        + t^2 \|y\|^2 + 2t\,\text{Re}\langle x, y \rangle. \] 
        By setting $a = \|y\|^2$, $b = 2\,\text{Re}\langle x, y \rangle$, 
        and $c = \|x\|^2$, we have $p(t) = at^2 + bt + c \geq 0$. 
        This means that $p(t)$ has at most one distinct real root, 
        and thus $b^2 - 4ac \leq 0$. In particular, we have 
        \[ 4(\text{Re}\langle x, y \rangle)^2 \leq 4\|x\|^2\|y\|^2, \] 
        or equivalently, $|\text{Re}\langle x, y \rangle| \leq \|x\|\|y\|$. 
        Choose $\alpha \in \C$ such that $|\alpha| = 1$ and 
        $\langle \alpha x, y \rangle = |\langle x, y \rangle|$. Then 
        we deduce that 
        \[ |\langle x, y \rangle| = \langle \alpha x, y \rangle
        = |\text{Re}\langle \alpha x, y \rangle| \leq \|\alpha x\|\|y\| 
        = \|x\|\|y\|. \] 

        \item We will show that $\|\cdot\|$ satisfies the triangle 
        inequality; the other properties are obvious. Given $x, y \in H$, 
        we have 
        \[ \|x + y\|^2 = \|x\|^2 + \|y\|^2 + 2\,\text{Re}\langle x, y \rangle 
        \leq \|x\|^2 + \|y\|^2 + 2\|x\|\|y\| = (\|x\| + \|y\|)^2. \qedhere \] 
    \end{enumerate}
\end{pf}

Since we have shown that $\|x\| = \langle x, x \rangle^{1/2}$ is indeed a 
norm on an inner product space $(H, \langle \cdot, \cdot \rangle)$, it 
makes sense to make the following definition. 

\begin{defn}{defn:3.7}
    A {\bf Hilbert space} is a complete inner product space. 
\end{defn}

\subsection{Banach Spaces of Measurable Functions} \label{subsec:3.2}
Recall that for $E \in {\cal L}$, we defined 
\[ L^1(E) = \left\{ f : E \to \C : f \text{ is measurable and } 
\int_E |f|\dd m < \infty \right\}, \] 
and we showed in Theorem~\ref{theo:2.57} that $L^1(E)$ is a vector space 
over $\C$. Given $f \in L^1(E)$, we will define the \textbf{$L^1$-norm} of 
$f$ by the value of the integral above, so 
\[ \|f\|_1 = \int_E |f|\dd m. \] 
Note that $\|\cdot\|_1$ is close to being a norm, since it is 
easy to see that $\|\alpha f\|_1 = |\alpha|\|f\|_1$ and 
\[ \|f + g\|_1 = \int_E |f+g|\dd m \leq \int_E (|f| + |g|)\dd m 
= \|f\|_1 + \|g\|_1 \] 
for any $f, g \in L^1(E)$ and $\alpha \in \C$. However, we are not quite 
there. Notice that 
\[ \|f\|_1 = \int_E |f|\dd m = 0 \] 
only implies that $f = 0$ almost everywhere, so we are not guaranteed 
non-degeneracy. 

To repair this, we can define an equivalence relation on $L^1(E)$ by 
$f \sim g$ if and only if $f - g = 0$ almost everywhere. Then $L^1(E)/\!\sim$ 
is a quotient space where $f \in L^1(E)$ is associated with the 
equivalence class $[f] \in L^1(E)/\!\sim$. It can be verified that 
$L^1(E)/\!\sim$ is a vector space over $\C$ with $\alpha [f] + \beta [g] 
= [\alpha f + \beta g]$ for all $\alpha, \beta \in \C$ and $f, g \in L^1(E)$. 
Moreover, we can define a norm on it by $\|[f]\|_1 = \|f\|_1$.

From now on, we will abuse notation and identify $L^1(E)$ with 
$L^1(E)/\!\sim$. That is, we only identify functions $f \in L^1(E)$ up to 
almost everywhere equality. Then $(L^1(E), \|\cdot\|_1)$ is a normed 
vector space. 

\begin{defn}{defn:3.8}
    Let $E \in {\cal L}$. We define the set of {\bf square-integrable 
    functions} over $E$ by 
    \[ L^2(E) = \left\{ f : E \to \C : f \text{ is measurable and }
    \int_E |f|^2 \dd m < \infty \right\}. \] 
    Equivalently, we have 
    \[ L^2(E) = \{f : E \to \C \mid f \text{ is measurable and } 
    f^2 \in L^1(E)\}. \] 
\end{defn}

\begin{prop}{prop:3.9}
    Let $E \in {\cal L}$ and $f, g \in L^2(E)$. Then $L^2(E)$ is an inner 
    product space with inner product given by 
    \[ \langle f, g \rangle = \int_E f\overline g\dd m. \] 
    In particular, this induces a norm on $L^2(E)$ via 
    \[ \|f\|_2 = \langle f, f \rangle^{1/2} = \left( \int_E |f|^2 \dd m 
    \right)^{\!1/2}, \] 
    which we call the {\bf $L^2$-norm}. 
\end{prop}
\begin{pf}
    Let $f, g \in L^2(E)$ and $\alpha \in \C$. It is clear that $\alpha f 
    \in L^2(E)$. To see that $f + g \in L^2(E)$, note that 
    $|f(x) + g(x)| \leq 2\max\{|f(x)|, |g(x)|\}$, and thus 
    \[ |f(x) + g(x)|^2 \leq 4\max\{|f(x)|^2, |g(x)|^2\} 
    \leq 4(|f(x)|^2 + |g(x)|^2). \] 
    It follows that 
    \[ \int_E |f + g|^2\dd m \leq \int_E 4(|f|^2 + |g|^2)\dd m < \infty, \] 
    so $f + g \in L^2(E)$ and we conclude that $L^2(E)$ is a $\C$-vector space. 

    Next, we verify that $\langle f, g \rangle$ is well-defined by 
    showing that $f\overline g \in L^1(E)$ for any $f, g \in L^2(E)$. 
    Note that if $a, b \geq 0$, then $ab \leq \frac12(a^2 + b^2)$. 
    By taking $a = |f(x)|$ and $b = |g(x)|$, we obtain 
    \[ |f(x) \overline g(x)| = |f(x)||g(x)| \leq \frac12(|f(x)|^2 + 
    |g(x)|^2). \] 
    Hence, we see that 
    \[ \int_E f\overline g\dd m \leq \frac12 \int_E |f|^2\dd m 
    + \frac12 \int_E |g|^2\dd m < \infty. \] 
    It is easily checked that $\langle \alpha f + \beta g, h \rangle 
    = \alpha \langle f, h \rangle + \beta \langle g, h \rangle$ 
    and $\langle f, g \rangle = \overline{\langle g, f \rangle}$ 
    for any $f, g, h \in L^2(E)$ and $\alpha, \beta \in \C$, so 
    $\langle \cdot, \cdot \rangle$ is a sesquilinear form on $L^2(E)$. 
    Finally, we have 
    \[ \langle f, f \rangle = \int_E |f|^2 \dd m \geq 0 \] 
    and $\langle f, f \rangle = 0$ if and only if $|f|^2 = 0$ almost 
    everywhere, which is equivalent to saying that $f = 0$ almost everywhere. 
    By identifying $L^2(E)$ with $L^2(E)/\!\sim$ where $\sim$ is the same 
    equivalence relation as above, this is enough to show that 
    $(L^2(E), \langle \cdot, \cdot \rangle)$ is an inner product space. 
\end{pf}

\begin{remark}{remark:3.10}
    Let $1 \leq p < \infty$ and $E \in {\cal L}$. Analogous to 
    what we have done so far, we can define 
    \[ L^p(E) = \left\{ f : E \to \C : f \text{ is measurable and }
    \int_E |f|^p \dd m < \infty \right\}. \] 
    and define the {\bf $L^p$-norm} by 
    \[ \|f\|_p = \left( \int_E |f|^p \dd m \right)^{\!1/p}. \] 
    It can be verified that this is indeed a norm; one possible approach 
    is by using H\"older's inequality which we will not prove. 
    This makes $(L^p(E), \|\cdot\|_p)$ a normed vector space. 
\end{remark}

Let $1 \leq p < \infty$. We now show that $L^p(E)$ is complete with 
respect to the norm $\|\cdot\|_p$ defined as above. In particular, 
this means that $(L^1(E), \|\cdot\|_1)$ is a Banach space and 
$(L^2(E), \langle \cdot, \cdot \rangle)$ is a Hilbert space. 
We recall that if a Cauchy sequence has a convergent subsequence in a 
metric space, then the Cauchy sequence also converges. 

\begin{theo}{theo:3.11}
    Let $1 \leq p < \infty$ and $E \in {\cal L}$. Then $(L^p(E), \|\cdot\|_p)$ 
    is complete. 
\end{theo}
\begin{pf}
    Fix $1 \leq p < \infty$, and take a Cauchy sequence $(f_n)_{n=1}^\infty$ 
    in $L^p(E)$ with respect to $\|\cdot\|_p$. Choose a subsequence 
    $n_1 < n_2 < n_3 < \cdots$ such that 
    \[ \|f_{n_{k+1}} - f_{n_k}\|_p < 2^{-k} \] 
    for all $k \in \N$. By the above remark, it suffices to find a function 
    $f \in L^p(E)$ such that $\|f - f_{n_k}\|_p \to 0$ as $k \to \infty$. 
    We define 
    \[ g_m(x) = |f_{n_1}(x)| + \sum_{k=1}^m |f_{n_{k+1}}(x) - f_{n_k}(x)|. \] 
    Then we see that each $g_m$ is measurable with $0 \leq g_1 
    \leq g_2 \leq \cdots$ and  
    \[ \|g_m\|_p \leq \|f_{n_1}\|_p + \sum_{k=1}^m \|f_{n_{k+1}} - 
    f_{n_k}\|_p \leq \|f_{n_1}\|_p + 1 = M. \]     
    Thus, we have $0 \leq g_1^p \leq g_2^p \leq \cdots$ and 
    \[ \int_E g_m^p \dd m \leq M^p \] 
    for all $m \in \N$. Now, define 
    \[ g(x) = \lim_{m\to\infty} g_m(x) = |f_{n_1}(x)| + 
    \sum_{k=1}^\infty |f_{n_{k+1}}(x) - f_{n_k}(x)|, \] 
    and observe that $g$ is measurable. By Lebesgue's Monotone Convergence 
    Theorem (Theorem~\ref{theo:2.48}), we have 
    \[ \int_E g^p\dd m = \lim_{m\to\infty} \int_E g_m^p\dd m \leq M^p 
    < \infty. \] 
    Thus, we have $g \in L^p(E)$ and in particular, $g(x) < \infty$ almost 
    everywhere. If $g(x) < \infty$, then we know that the telescoping series 
    \[ f_{n_1}(x) + \sum_{k=1}^\infty (f_{n_{k+1}}(x) - f_{n_k}(x)) \] 
    converges absolutely. This means that $\lim_{m\to\infty} f_{n_m}(x)$ 
    exists for almost every $x \in E$. Finally, define 
    \[ f(x) = \begin{cases}
        \lim_{m\to\infty} f_{n_m}(x), & \text{if } g(x) < \infty, \\ 
        0, & \text{otherwise.}
    \end{cases} \] 
    Then $f_{n_k} \to f$ almost everywhere, so $f$ is measurable. Moreover, 
    $|f_{n_k}| \leq g_{k-1} \leq g$ implies that $|f| \leq g$. This means that 
    \[ \int_E |f|^p \dd m \leq \int_E g^p \dd m < \infty, \] 
    so $f \in L^p(E)$. Finally, observe that 
    \[ \|f_{n_m} - f\|_p = \left\| \sum_{k=m}^\infty (f_{n_{k+1}} 
    - f_{n_k}) \right\|_p \leq \sum_{k=m}^\infty \|f_{n_{k+1}} - f_{n_k}\|_p 
    \to 0. \] 
    Then applying Lebesgue's Dominated Convergence Theorem to 
    $f_{n_{m+1}}^p \to f^p$ with $|f_{n_{m+1}}|^p \leq g^p \in L^1(E)$ implies 
    that $f^p \in L^1(E)$, which is equivalent to saying that 
    $f \in L^p(E)$. This completes the proof. 
\end{pf}

\begin{cor}{cor:3.12}
    Let $1 \leq p < \infty$ and let $(f_n)_{n=1}^\infty \subseteq L^p(E)$ 
    be a Cauchy sequence with limit $f \in L^p(E)$. Then there exists a 
    subsequence $(f_{n_k})_{k=1}^\infty$ such that $f_{n_k}(x) \to f(x)$ 
    for almost every $x \in E$. 
\end{cor}
\begin{pf}
    Take the subsequence $(f_{n_k})_{k=1}^\infty$ from the proof of the 
    previous theorem. 
\end{pf}

\begin{remark}{remark:3.13}
    We know that if $f_n \to f$ with respect to $\|\cdot\|_p$ for 
    $1 \leq p < \infty$, then $f_{n_k}(x) \to f(x)$ for almost every 
    $x \in E$ for some subsequence $(f_{n_k})_{k=1}^\infty$. But 
    does this imply that $f_n \to f$ for the original sequence? 

    The answer is no. Define $(f_n)_{n=1}^\infty$ by $f_1 = \chi_{[0, 1]}$,
    $f_2 = \chi_{[0, 1/2]}$, $f_3 = \chi_{[1/2, 1]}$, $f_4 = \chi_{[0, 1/3]}$,
    $f_5 = \chi_{[1/3, 2/3]}$, $f_6 = \chi_{[2/3, 1]}$, and so on. Then 
    $\|f_n\|_p \to 0$ for $1 \leq p < \infty$, but $f_n$ does not converge 
    for any $x \in [0, 1]$. Therefore, we really need a subsequence 
    $(f_{n_k})_{k=1}^\infty$ in general. 
\end{remark}

Let $E \in {\cal L}$ with $m(E) > 0$. We now consider the space $L^\infty(E)$.

\begin{defn}{defn:3.14}
    A measurable function $f : E \to \C$ is called {\bf essentially bounded} 
    if there exists $A > 0$ such that $|f| \leq A$ almost everywhere. 
    That is, we have $m\{x \in E : |f(x)| > A\} = 0$. 

    When $f : E \to \C$ is essentially bounded, the {\bf essential supremum} 
    of $|f|$ is defined to be 
    \[ \|f\|_\infty = \esssup_{x\in E} |f(x)| = 
    \inf\{A \geq 0 \mid m\{x \in E : |f(x)| > A\} = 0\}. \] 
    Then the space $L^\infty(E)$ is given by 
    \[ L^\infty(E) = \{f : E \to \C \mid f \text{ is measurable and } 
    \|f\|_\infty < \infty\}. \] 
\end{defn}

Similar to $(L^p(E), \|\cdot\|_p)$ for $1 \leq p < \infty$, we have that 
$L^\infty(E)$ equipped with the essential supremum is a normed vector space. 
In fact, it is a Banach space. 

\begin{prop}{prop:3.15}
    $(L^\infty(E), \|\cdot\|_\infty)$ is a normed vector space. 
\end{prop}
\begin{pf}
    Let $\alpha \in \C$ and $f, g \in L^\infty(E)$. Note that $|f| > A$ 
    if and only if $|\alpha f| \geq |\alpha|A$ for all $\alpha \in 
    \C \setminus \{0\}$, so $\|\alpha f\|_\infty = |\alpha|\|f\|_\infty$ 
    and thus $\alpha f \in L^\infty(E)$. To see that $f + g \in L^\infty(E)$, 
    there exist $A, B > 0$ such that $|f(x)| \leq A$ and $|g(x)| \leq B$
    for almost every $x \in E$. Then for almost every $x \in E$, we get 
    \[ |f(x) + g(x)| \leq |f(x)| + |g(x)| \leq A + B < \infty, \] 
    which implies that $f + g \in L^\infty(E)$. 

    Now, we check that $\|\cdot\|_\infty$ is a norm. It is clear that 
    $\|f\|_\infty \geq 0$. We claim that $|f| \leq \|f\|_\infty$ and 
    $|g| \leq \|g\|_\infty$ almost everywhere. Indeed, observe that 
    \[ m\{x \in E : |f(x)| > \|f\|_\infty + 1/n\} = 0 \] 
    by the definition of $\|f\|_\infty$, and thus 
    \[ m\{x \in E : |f(x)| > \|f\|_\infty\} = m\!\left( \bigcup_{n=1}^\infty 
    \{x \in E : |f(x)| > \|f\|_\infty + 1/n\} \right) = 0. \] 
    In particular, if $\|f\|_\infty = 0$, then $|f| = 0$ almost everywhere 
    and hence $f = 0$ almost everywhere. Moreover, we have  
    \[ |f + g| \leq |f| + |g| \leq \|f\|_\infty + \|g\|_\infty \] 
    almost everywhere, which implies that 
    \[ m\{x \in E : |f(x) + g(x)| > \|f\|_\infty + \|g\|_\infty\} = 0. \] 
    This means that $\|f + g\|_\infty \leq \|f\|_\infty + \|g\|_\infty$, 
    so $\|\cdot\|_\infty$ is a norm as desired. 
\end{pf}

\begin{theo}{theo:3.16}
    $(L^\infty(E), \|\cdot\|_\infty)$ is a Banach space. 
\end{theo}
\begin{pf}
    Let $(f_n)_{n=1}^\infty$ be a Cauchy sequence in $(L^\infty(E), 
    \|\cdot\|_\infty)$. Pick a subsequence $(f_{n_k})_{k=1}^\infty$ such that 
    \[ \sum_{k=1}^\infty \|f_{n_{k+1}} - f_{n_k}\|_\infty < \infty. \] 
    Then we have 
    \[ g(x) = |f_{n_1}(x)| + \sum_{k=1}^\infty |f_{n_{k+1}}(x) - f_{n_k}(x)|
    < \infty \] 
    for almost every $x \in E$. This means that 
    \[ \lim_{m\to\infty} f_{n_m}(x) = \lim_{m\to\infty} 
    \left( f_{n_1}(x) + \sum_{k=1}^{m-1} (f_{n_{k+1}}(x) - f_{n_k}(x)) \right) \] 
    exists for almost every $x \in E$. Define $f : E \to \C$ by 
    \[ f(x) = \begin{cases}
        \lim_{m\to\infty} f_{n_m}(x), & \text{if it converges,} \\ 
        0, & \text{otherwise.} 
    \end{cases} \] 
    Then we have $\|f - f_{n_m}\|_\infty \to 0$ because for 
    $x \in \{y \in E | \lim_{m\to\infty} f_{n_m}(y) = f(y)\} =: A$, 
    we get 
    \[ |f_{n_m}(x) - f(x)| \leq \sum_{k=m}^\infty |f_{n_{k+1}}(x) - f_{n_k}(x)|, \] 
    which converges uniformly for $x \in A$. Since $m(E \setminus A) = 0$, 
    this gives as $\|f - f_{n_m}\|_\infty \to 0$ as claimed. Finally, we obtain 
    \[ 0 \leq \lim_{n\to\infty} \|f_n - f\|_\infty \leq 
    \lim_{n\to\infty} (\|f_n - f_{n_m}\|_\infty + \|f_{n_m} - f\|_\infty) 
    = 0. \qedhere \] 
\end{pf}

\begin{remark}{remark:3.17}
    If $m(E) < \infty$, then $L^\infty(E) \subseteq L^2(E) \subseteq L^1(E)$. 
    Indeed, if $f \in L^\infty(E)$, then 
    \[ \|f\|_2^2 = \int_E |f|^2\dd m \leq \|f\|_\infty^2 \int_E 1\dd m 
    = \|f\|_\infty^2 m(E), \] 
    and similarly, we have $\|f\|_1 \leq \|f\|_\infty m(E)$. 

    This result is not true without the assumption that $m(E) < \infty$. 
    For example, let $E = \R$, and define $f(x) = 1$ for all $x \in \R$, 
    $g(x) = 1/x$ for $x \geq 1$ and $g(x) = 0$ otherwise, and 
    $h(x) = x^{-1/2}$ on $(0, 1)$ and $h(x) = 0$ otherwise. Then 
    $f \in L^\infty(\R) \setminus (L^1(\R) \cup L^2(\R))$, 
    $g \in L^2(\R) \setminus L^1(\R)$, and $h \in L^1(\R) \setminus 
    (L^2(\R) \cup L^\infty(\R))$. 
\end{remark}

\subsection{Density and Approximation Results} \label{subsec:3.3}
We now consider the density of certain spaces in $L^p(\R)$, where 
$1 \leq p \leq \infty$. 

\begin{defn}{defn:3.18}
    The space of compactly supported continuous functions is given by 
    \[ C_c(\R) = \{f : \R \to \C \mid \text{$f$ is continuous and 
    there exists a compact subset $K \subseteq \R$ with $f|_{\R \setminus K}
    = 0$}\}. \] 
    The space of simple and measurable functions is given by 
    \[ S(\R) = \{f : \R \to \C \mid \text{$f$ is simple and measurable}\}. \] 
    The space of simple and measurable functions which are also Lebesgue 
    integrable is defined by 
    \[ S_1(\R) = S(\R) \cap L^1(\R) = \left\{ \varphi = \sum_{i=1}^n 
    \alpha_i \chi_{E_i} : \alpha_i \in \C,\, m(E_i) < \infty \right\}. \] 
\end{defn}

Note that for $1 \leq p \leq \infty$, we have $C_c(\R) \subseteq L^p(\R)$ and 
$S_1(\R) \subseteq L^p(\R)$ as subspaces. Moreover, we have $S(\R) \subseteq 
L^\infty(\R)$ as a subspace. None of these are closed subspaces though. 

\begin{theo}{theo:3.19}
    $S(\R)$ is dense in $L^p(\R)$ for $1 \leq p \leq \infty$. That is, 
    for $f \in L^p(\R)$, there exists a sequence $(\varphi_n)_{n=1}^\infty$ 
    of simple measurable functions in $L^p(\R)$ such that 
    $\varphi_n \to f$ with respect to $\|\cdot\|_p$. 
\end{theo}
\begin{pf}
    Let $f \in L^p(\R)$. Since $f$ is measurable, we know from Theorem~\ref{theo:2.43} 
    that we can approximate $f$ using simple measurable functions. In particular, 
    there exists a sequence $(\varphi_n)_{n=1}^\infty$ of simple measurable 
    functions such that $\varphi_n \to f$ pointwise and 
    \[ |\varphi_1| \leq |\varphi_2| \leq \cdots \leq |f|. \] 
    When $p = \infty$, we have $\varphi_n \to f$ almost uniformly. 
    This means that $\|\varphi_n - f\|_\infty \to 0$. On the other hand, 
    if $1 \leq p < \infty$, then we have $|f - \varphi_n|^p \to 0$ 
    pointwise, and we know that 
    \[ |f - \varphi_n|^p \leq (2|f|)^p \in L^1(\R) \] 
    since $f \in L^p(\R)$. This means that $\varphi_n \in L^p(\R)$, 
    and applying Lebesgue's Dominated Convergence Theorem (Theorem~\ref{theo:2.58})
    implies that 
    \[ \|f - \varphi_n\|_p^p = \int_{\R} |f - \varphi_n|^p\dd m \to 0. \qedhere \] 
\end{pf}

\begin{theo}{theo:3.20}
    $C_c(\R)$ is dense in $L^p(\R)$ for $1 \leq p < \infty$. 
\end{theo}
\begin{pf}
    Let $\eps > 0$ and $f \in L^p(\R)$ where $1 \leq p < \infty$. We need 
    to find $g \in C_c(\R)$ such that $\|f - g\|_p < \eps$. 

    Since $L_p(\R)$ is the closure of the simple integrable functions, 
    it suffices to assume that $f = \varphi$ is simple. Indeed, if we 
    can find a simple function $\varphi$ such that $\|f - \varphi\|_p < \eps/2$
    and a function $g \in C_c(\R)$ such that $\|\varphi - g\|_p < \eps/2$, 
    then the triangle inequality gives us the result. So we have 
    \[ \varphi = \sum_{i=1}^n \alpha_i \chi_{E_i} \] 
    where $\alpha \in \C$ and $E_i \in {\cal L}$ with $m(E_i) < \infty$. 

    We can further reduce this problem to approximating $\varphi = \chi_E$ 
    for $E \in {\cal L}$ with $m(E) < \infty$, because the triangle inequality 
    again assures us that 
    \[ \left\| \sum_{i=1}^n \alpha_i \chi_{E_i} - \sum_{i=1}^n 
    \alpha_i g_i \right\|_p \leq \sum_{i=1}^n |\alpha_i| \|\chi_{E_i} 
    - g_i\|_p \] 
    for some $g_i \in C_c(\R)$. 

    Finally, fix $E \in {\cal L}$ with $m(E) < \infty$. We can write 
    $E = \bigcup_{n\in\N} E \cap (-n, n)$, and observe that 
    $m(E) = \lim_{n\to\infty} m(E_n)$ by the continuity of measure. 
    We leave it as an exercise to show that 
    \[ \chi_E = \lim_{n\to\infty} \chi_{E_n} \] 
    with respect to $\|\cdot\|_p$. So without loss of generality, 
    we can assume that $E \in {\cal L}$ is bounded, say $E \subseteq 
    (-n, n)$ for some $n \in \N$. We will find $g \in C_c(\R)$ 
    such that $\|\chi_E - g\|$ is bounded above by $\eps$ multiplied 
    by a constant. 

    Recall from the structure theorem of Lebesgue measurable sets 
    (Theorem~\ref{theo:2.22}) that there exists a closed set $F$ and 
    an open set $G$ such that $F \subseteq E \subseteq G$ and 
    $m(G \setminus F) < \eps$. We may assume without loss of generality 
    that $G \subseteq (-n, n)$ because we can chop off elements outside 
    of this interval otherwise. 

    We now design a $g \in C_c(\R)$ such that 
    \begin{enumerate}[(i)]
        \item $g(x) \in [0, 1]$ for all $x \in \R$; 
        \item $g(x) = 1$ for all $x \in F$; and 
        \item $g(x) = 0$ for all $x \in G^c$. 
    \end{enumerate}
    Given such a function $g \in C_c(\R)$, we are done because 
    \[ \|\chi_E - g\|_p^p = \int_G |\chi_E - g|^p \dd m 
    = \int_F 0\dd m + \int_{G\setminus F} |\chi_E - g|^p\dd m 
    \leq 0 + m(G \setminus F) 2^p < 2^p\eps. \] 
    To construct this function, recall that for $S \subseteq \R$ with 
    $S \neq \varnothing$, we define 
    \[ d(x, S) = \inf\{|x - y| : y \in S\} \] 
    for any fixed $x \in \R$. We leave it as an exercise to show that 
    $d(\cdot, S) : \R \to [0, \infty)$ is continuous. Then define 
    \[ g(x) = \frac{d(x, G^c)}{d(x, G^c) + d(x, F)}. \] 
    This is well-defined since $F \cap G^c = \varnothing$ 
    and so $d(x, G^c) + d(x, F) \neq 0$. Moreover, it is continuous 
    because it is the composition of continuous functions. Finally, 
    it is easily seen that $g$ satisfies all three properties above, 
    which completes the proof.
\end{pf}

We note that $C_c(\R)$ is \emph{not} dense in $L^\infty(\R)$; this can be 
seen by taking the function $f(x) = 1$ for all $x \in \R$. 

\begin{cor}{cor:3.21}
    Let $a < b$ be finite. Then $C[a, b]$ is dense in $L^p[a, b]$ for 
    $1 \leq p \leq \infty$. 
\end{cor}
\begin{pf}
    Repeat the same argument as Theorem~\ref{theo:3.20} by reducing it to 
    the case where $f = \chi_E$ with $E \subseteq (a + 1/n, b - 1/n)$. 
    The case where $p = \infty$ is not an issue here because we have 
    $L^1[a, b] \supseteq L^p[a, b] \supseteq L^\infty[a, b] \supseteq 
    C[a, b]$, and being dense in $L^1[a, b]$ implies being dense in 
    $L^\infty[a, b]$ because $L^1[a, b]$ is a larger space. 
\end{pf}\newpage
\section{Fourier Analysis on the Circle} \label{sec:4}

\subsection{General Problem of Fourier Analysis} \label{subsec:4.1}
Recall that we defined $\T = \{z \in \C : |z| = 1\}$, which we can 
identify with the interval $[-\pi, \pi]$ via the map $\theta \mapsto 
e^{i\theta}$. We also identify the endpoints so that $\pi = -\pi$. 
We are interested in studying $(L^p(\T), \|\cdot\|_p)$, where
\[ \|f\|_p = \left( \frac{1}{2\pi} \int_{-\pi}^\pi |f(\theta)|^p \dd\theta 
\right)^{\!1/p} \]
for $1 \leq p < \infty$, and $(L^\infty(\T), \|\cdot\|_\infty)$ is defined 
as usual. Note that we have introduced a new normalization here, which is 
merely for cosmetic purposes to ensure that the identity function 
has norm $1$. 

\begin{exercise}{exercise:4.1}
    Show that for all $p \in (1, \infty)$, we have 
    \[ C(\T) \subseteq L^\infty(\T) \subseteq L^p(\T) \subseteq L^1(\T), \] 
    and that each of these inclusions are dense. Moreover, prove that 
    \[ \|f\|_\infty \geq \|f\|_p \geq \|f\|_1. \]  
\end{exercise}

Recall that at the beginning of the course in Section~\ref{subsec:1.3}, we 
discussed Fourier coefficients of continuous functions. We now write down a 
slightly more rigorous definition, allowing for Lebesgue integrable 
functions instead of only continuous ones.

\begin{defn}{defn:4.2}
    Let $f \in L^1(\T)$. For each $n \in \Z$, we define the {\bf $n$-th 
    Fourier coefficient} of $f$ by 
    \[ \hat f(n) = \frac1{2\pi} \int_{-\pi}^\pi f(\theta) e^{-in\theta} \dd\theta. \] 
    The {\bf Fourier series} of $f$ is the series $\sum_{n\in\Z} \hat f(n) 
    e^{in\theta}$, and we write 
    \[ f \sim \sum_{n\in\Z} \hat f(n) e^{in\theta}. \] 
\end{defn}

The general problem of Fourier analysis is to answer the following questions: 
\begin{itemize}
    \item Given $f \in L^p(\T)$ or $f \in C(\T)$, to what extent does the 
    sequence $(\hat f(n))_{n\in\Z}$ determine $f$? 
    \item Given a sequence $(a_n)_{n\in\Z}$, are there necessary or sufficient 
    conditions in which there exists a (possibly unique) function $f \in 
    L^p(\T)$ such that $a_n = \hat f(n)$ for all $n \in \Z$? 
    \item To what extent does the Fourier series represent $f$? Do we have 
    \[ f(\theta) = \sum_{n\in\Z} \hat f(n) e^{in\theta} \] 
    either pointwise, pointwise almost everywhere, or convergence with 
    respect to $\|\cdot\|_p$? 
\end{itemize}

\subsection{$L^2$-convergence of Fourier Series} \label{subsec:4.2}
It turns out that the theory for functions in $L^2(\T)$ is the most 
``beautiful'' and ``clean''. This is because $L^2(\T)$ is a Hilbert space! 
In fact, most of the theory for $L^2(\T)$ follows from general Hilbert 
space theory. 

\begin{defn}{defn:4.3}
    Let $(H, \langle \cdot, \cdot \rangle)$ be a Hilbert space. 
    \begin{itemize}
        \item Let $x, y \in H$. We say that $x$ and $y$ are {\bf orthogonal}
        if $\langle x, y \rangle = 0$, and we write $x \perp y$. 
        Note that this implies that $\|x + y\|^2 = \|x\|^2 + \|y\|^2$.

        Given a subset $S \subseteq H$, the {\bf orthogonal complement} of $S$ 
        is defined to be 
        \[ S^\perp = \{x \in H \mid \langle x, y \rangle = 0 \text{ for all }
        y \in S\}. \] 
        Note that $S^\perp$ is always a closed subspace. 

        \item An {\bf orthonormal system} in $H$ is a family $\{e_n\}_{n\in S} 
        \subseteq H$ such that for all $n, m \in S$, we have 
        \[ \langle e_n, e_m \rangle = \delta_{nm} = \begin{cases} 
            1, & \text{if } n = m, \\ 
            0, & \text{if } n \neq m. 
        \end{cases} \] 
        In particular, this means that $\|e_n\| = 1$ for all $n \in S$ 
        and $e_n \perp e_m$ whenever $n \neq m$. 

        \item An {\bf orthonormal basis} in $H$ is an orthonormal system 
        $\{e_n\}_{n\in S}$ such that $\Span\{e_n \mid n \in S\}$ is 
        dense in $H$ with respect to the norm $\|\cdot\|$ induced by the 
        inner product. 
    \end{itemize}
\end{defn}

\begin{exmp}{exmp:4.4}
    Let $S \neq \varnothing$ be a countable set. Then the space of 
    square-summable sequences, denoted 
    \[ \ell^2(S) = \left\{ a = (a_n)_{n\in S} : a_n \in \C,\; 
    \sum_{n\in S} |a_n|^2 < \infty \right\}, \] 
    is a Hilbert space with the inner product 
    \[ \langle a, b \rangle = \sum_{n\in S} a_n \overline{b_n}. \] 
    The norm induced by the inner product is given by 
    \[ \|a\| = \langle a, a \rangle^{1/2} = \left( \sum_{n\in S} |a_n|^2 
    \right)^{\!1/2}. \] 
    For each $n \in S$, let $\delta_n$ be the sequence with 
    $1$ in the $n$-th entry and $0$ in every other entry. Then 
    it is easy to see that $(\delta_n)_{n\in S}$ is an orthonormal 
    system in $\ell^2(S)$. In fact, its span is dense in $\ell^2(S)$, 
    so it forms an orthonormal basis. 
\end{exmp}

In this section, our interest is $(L^2(\T), \|\cdot\|_2)$. 
Note that the sequence of functions $\{e_n\}_{n\in \Z}$ defined by 
$e_n(\theta) = e^{in\theta}$ is an orthonormal system in $L^2(\T)$. We will 
show later that this is in fact an orthonormal basis for $L^2(\T)$. First, 
we will prove Parseval's Theorem. Recall that a topological space is 
called separable if it contains a countable dense subset. 

\begin{theo}[Parseval's Theorem]{theo:4.5}
    Let $(H, \langle \cdot, \cdot \rangle)$ be a separable Hilbert space. 
    Let $\{e_n\}_{n\in S}$ be an orthonormal system in $H$, and let 
    \[ K = \overline{\Span\{e_n \mid n \in S\}}^{\|\cdot\|} \subseteq H. \] 
    For $n \in S$ and $x \in H$, set $\hat x(n) = \langle x, e_n \rangle$. 
    \begin{enumerate}[(1)]
        \item For any $x \in H$, we have 
        \[ \sum_{n\in S} |\hat x(n)|^2 \leq \|x\|^2 < \infty. \] 
        \item The map $P : H \to H$ defined by 
        \[ Px = \sum_{n\in S} \hat x(n) e_n = \sum_{n\in S} 
        \langle x, e_n \rangle e_n \] 
        satisfies the following properties: 
        \begin{enumerate}[(a)]
            \item For any $P \in B(H, H)$, we have $\|P\| \leq 1$, 
            and $\|P\| = 1$ if $S \neq \varnothing$. 
            \item We have $PH = K$ and $P^2 = P$ (so $P$ is idempotent). 
            \item For all $x \in H$, we have $(x - Px) \perp K$. 
            \item We have $\ker(P) = K^\perp = \{x \in H \mid \langle x, k 
            \rangle = 0 \text{ for all } k \in K\}$. 
        \end{enumerate}
        Note that properties (b) and (c) imply that $P$ is the orthogonal 
        projection of $H$ onto $K$. 
        \item For all $x \in H$, $Px$ is the closest point in $K$ to $x$. 
        That is, we have 
        \[ \|x - Px\| = \inf_{k\in K} \|x - k\|. \] 
    \end{enumerate}
\end{theo}
\begin{pf}
    We first assume that $|S| < \infty$. In this case, we have 
    $K = \Span\{e_n \mid n \in S\}$ (meaning that there is no need to 
    take the closure with respect to $\|\cdot\|$), and for all $x \in H$, 
    the map $P : H \to H$ in the theorem is well-defined and linear. 
    To see that $P^2 = P$, note that 
    \[ P^2x = P(Px) = P\left( \sum_{n\in S} \hat x(n) e_n \right) 
    = \sum_{n\in S} \hat x(n) e_n = Px \] 
    by pushing the map $P$ to each $e_n$. Moreover, it is clear that $PH = K$. 

    Next, we see that 
    \[ \|Px\|^2 = \langle Px, Px \rangle = \sum_{n\in S} |\hat x(n)|^2 
    = \sum_{n\in S} \langle x, e_n \rangle \overline{\langle x, e_n \rangle} 
    = \left\langle x, \sum_{n\in S} \langle x, e_n \rangle e_n \right\rangle 
    = \langle x, Px \rangle \leq \|x\|\|Px\|, \] 
    where the final inequality follows from Cauchy-Schwarz. Rearranging 
    this yields $\|Px\|(\|x\| - \|Px\|) \geq 0$, and hence $\|Px\| \leq \|x\|$. 
    This means that $\|P\| \leq 1$, and $\|Pe_n\| = \|e_n\| = 1$ implies that 
    $\|P\| = 1$ when $S \neq \varnothing$. Our work above also shows that 
    \[ \sum_{n\in S} |\hat x(n)|^2 \leq \|x\|\|Px\| \leq \|x\|^2 < \infty. \] 
    For all $n_0 \in S$, notice that we have 
    \[ \langle x - Px, e_{n_0} \rangle = \hat x(n_0) - \hat x(n_0) = 0, \] 
    so $(x - Px) \perp \Span\{e_n \mid n \in S\} = K$ for all $x \in H$. 
    Note that $x \in \ker(P)$ if and only if $\hat x(n) = \langle x, e_n 
    \rangle = 0$ for all $n \in S$. This is equivalent to saying that 
    $x \perp K$; that is, $x \in K^\perp$. 

    Finally, for any $k \in K$, we have 
    \[ \|x - k\|^2 = \|(x - Px) + (Px - k)\|^2 = \|x - Px\|^2 + 
    \|Px - k\|^2 \geq \|x - Px\|^2, \] 
    which implies that $\|x - Px\| = \inf_{k\in K} \|x - k\|$. 

    Suppose now that $S$ is an infinite set. We leave it as an exercise 
    to show that the separability of $H$ implies that $|S| = |\N|$. 
    Without loss of generality, we will assume that $S = \N$, 
    so our orthonormal system is $\{e_n\}_{n=1}^\infty$. For each 
    $N \in \N$, let $K_N = \Span\{e_1, \dots, e_n\} \subseteq K$. 
    Let $P_N : H \to K_N$ be the corresponding projection 
    \[ P_Nx = \sum_{n=1}^N \hat x(n) e_n. \] 
    For all $N \in \N$, our above work shows that 
    \[ \sum_{n=1}^N |\hat x(n)|^2 \leq \|x\|^2 < \infty, \] 
    and taking $N \to \infty$ gives us 
    \[ \sum_{n=1}^\infty |\hat x(n)|^2 \leq \|x\|^2 < \infty. \] 
    For all $x \in H$, we claim that $(P_Nx)_{N=1}^\infty$ is a Cauchy 
    sequence. Taking $M < N$, we see that 
    \[ \|P_nx - P_mx\|^2 = \left\| \sum_{n=M+1}^N \hat x(n) e_n \right\|^{2}
    = \sum_{n=M+1}^N |\hat x(n)|^2, \] 
    which converges to $0$ as $M, N \to \infty$. Then we see that 
    \[ \lim_{N\to\infty} P_Nx = \sum_{n=1}^\infty \hat x(n) e_n \] 
    exists in $K$. We define the map $P : H \to K$ by 
    \[ Px = \lim_{N\to\infty} P_Nx = \sum_{n=1}^\infty \hat x(n) e_n, \] 
    which is linear. Moreover, we have $\|Px\| = \lim_{N\to\infty} 
    \|P_Nx\| \leq \|x\|$, so $P$ is bounded. 

    Note that $Px = x$ for all $x \in \Span\{e_1, \dots, e_N\}$ and $N \in \N$, 
    so taking $N \to \infty$ gives $Px = x$ for all $x \in K$. Also, we have 
    \[ \langle x - Px, e_n \rangle = \langle x, e_n \rangle - \langle x, 
    e_n \rangle = 0 \] 
    as before, so $(x - Px) \perp \overline{\Span\{e_n \mid n \in \N\}}^{\|\cdot\|} = K$. 
    Since $Px \in K$, this gives us $x \perp Px$, and thus 
    \[ \|x\|^2 = \|x - Px\|^2 + \|Px\|^2. \] 
    The rest of the proof is the same as the $|S| < \infty$ case. 
\end{pf}

\begin{cor}{cor:4.6}
    If $\{e_n\}_{n=1}^\infty$ is an orthonormal basis for a separable 
    Hilbert space $(H, \langle \cdot, \cdot \rangle)$, then for all $x \in H$, 
    we have 
    \[ \|x\|^2 = \sum_{n\in S} |\hat x(n)|^2. \] 
\end{cor}
\begin{pf}
    Note that $(x - Px) \perp K$ for all $x \in H$ where $P$ and $K$ 
    are as in Parseval's Theorem (Theorem~\ref{theo:4.5}). But 
    $\{e_n\}_{n=1}^\infty$ is an orthonormal basis, which means that 
    $H = K$. Then $x - Px = 0$ and hence $\|x - Px\| = 0$
    for all $x \in H$. From this, we obtain 
    \[ \|x\|^2 = \|Px\|^2 + \|x - Px\|^2 = \|Px\|^2 = \sum_{n\in S} |\hat x(n)|^2. 
    \qedhere \] 
\end{pf}

Now, we return to the case $(L^2(\T), \|\cdot\|_2)$ where 
\[ \|f\|_2^2 = \frac{1}{2\pi} \int_{-\pi}^\pi |f(\theta)|^2\dd \theta. \] 
We had an orthonormal system $\{e_n\}_{n\in\Z}$ in $L^2(\T)$ given by 
$e_n(\theta) = e^{in\theta}$. For $f \in L^2(\T)$, we have 
\[ \hat f(n) = \frac{1}{2\pi} \int_{-\pi}^\pi f(\theta) e^{-in\theta}\dd\theta 
= \langle f, e_n \rangle. \] 
We define the $N$-th partial sum of the Fourier series of $f$ as 
\[ S_N f(\theta) = \sum_{n=-N}^N \hat f(n) e^{in\theta}. \] 
In particular, denoting $\Pol(\T) = \Span\{e_n \mid n \in \Z\}$ as the 
``trigonometric functions on the circle'', we have $\Pol(\T) 
\subseteq C(\T) \subseteq L^2(\T)$ and $S_Nf \in \Pol(\T)$ for all 
$N \in \N$. From Parseval's Theorem, we see that 
\[ S_N : L^2(\T) \to \Span\{e_n \mid n \in \Z,\, |n| \leq N\} \subseteq 
\Pol(\T) \subseteq L^2(\T) \] 
is the orthogonal projection of $L^2(\T)$ onto $\Span\{e_n \mid n \in \Z,\, 
|n| \leq N\}$.

\begin{theo}{theo:4.7}
    The family $\{e_n\}_{n\in\Z}$ is an orthonormal basis for $L^2(\T)$. 
    In particular, if $f \in L^2(\T)$, then $f$ is the $\|\cdot\|_2$-limit 
    of its partial sums $(S_Nf)_{N=1}^\infty$. That is, with respect to 
    $\|\cdot\|_2$, we have 
    \[ f = \sum_{n\in\Z} \hat f(n) e_n = \lim_{N\to\infty} S_N f. \] 
\end{theo}
\begin{pf}
    We already know that $\{e_n\}_{n\in\Z}$ is an orthonormal system for 
    $L^2(\T)$. Consider the set 
    \[ K = \overline{\Span\{e_n \mid n \in \Z\}}^{\|\cdot\|_2} 
    = \overline{\Pol(\T)}^{\|\cdot\|_2} \subseteq L^2(\T). \] 
    To show that $\{e_n\}_{n\in\Z}$ is an orthonormal basis for $L^2(\T)$, 
    our aim is to show that $K = L^2(\T)$. 
    
    It suffices to prove that $\Pol(\T) \subseteq C(\T)$ is dense 
    in $C(\T)$ with respect to $\|\cdot\|_\infty$. Indeed, suppose that 
    $\Pol(\T)$ is dense in $C(\T)$ with respect to $\|\cdot\|_\infty$. 
    Let $\eps > 0$ and let $f \in L^2(\T)$. From Corollary~\ref{cor:3.21}, 
    we know that $C[-\pi, \pi]$ is dense in $L^2[-\pi, \pi]$ with respect 
    to $\|\cdot\|_2$. Hence, there exists $g \in C(\T)$ such that 
    $\|g - f\|_2 < \eps$. Choose $p \in \Pol(\T)$ such that $\|p - g\|_\infty 
    < \eps$, which exists by our above assumption. Then we obtain 
    \[ \|p - g\|_2 \leq \|p - g\|_\infty < \eps, \] 
    and the triangle inequality implies that $\|f - p\|_2 < 2\eps$. 

    To show that $\Pol(\T)$ is dense in $C(\T)$ with respect to 
    $\|\cdot\|_\infty$, we recall the Stone-Weierstrass Theorem. 
    It states that if $X$ is a compact metric space and ${\cal A} \subseteq C(X)$
    is an algebra which is self-adjoint, separates points, and contains a 
    non-zero constant function, then ${\cal A}$ is dense in $C(X)$ with respect 
    to $\|\cdot\|_\infty$. Recall that ${\cal A} \subseteq C(X)$ separates 
    points if for any distinct points $x, y \in X$, there exists a 
    function $f \in {\cal A}$ such that $f(x) \neq f(y)$. 

    In our case, we have $X = \T \subseteq \C$ and ${\cal A} = \Pol(\T)$. 
    It is clear that $\Pol(\T)$ is an algebra, and it is self-adjoint 
    since $\overline{e_n} = e_{-n}$. It also contains $1 = e_0 \in \Pol(\T)$. 
    Finally, we see that $\Pol(\T)$ separates points because if 
    $\theta_1, \theta_2 \in \T$ are distinct, then 
    \[ e_1(\theta_1) = e^{i\theta_1} \neq e^{i\theta_2} = e_1(\theta_2). \] 
    It follows from Stone-Weierstrass that $\Pol(\T)$ is dense in $C(\T)$ 
    with respect to $\|\cdot\|_\infty$, which shows that $\{e_n\}_{n\in\Z}$
    is an orthonormal basis for $L^2(\T)$. 

    Now, let $f \in L^2(\T)$. Our goal is to show that $\|f - S_Nf\|_2 \to 0$. 
    Let $\eps > 0$. From our work above, we can find $p \in \Pol(\T)$ 
    such that $\|f - p\|_2 < \eps$. By choosing $N \geq \deg(p)$, we have 
    \begin{align*}
        \|f - S_Nf\| &\leq \|f - p\|_2 + \|p - S_Nf\|_2 \\ 
        &= \|f - p\|_2 + \|S_Np - S_Nf\|_2 \\ 
        &\leq \|f - p\|_2 + \|S_N\| \|f - p\|_2 \\ 
        &< 2\eps 
    \end{align*}
    since $\|S_N\| \leq 1$. It follows that $f = \lim_{N\to\infty} S_Nf$ 
    with respect to $\|\cdot\|_2$. 
\end{pf}

\begin{cor}[Plancherel Identity]{cor:4.8}
    For any $f \in L^2(\T)$, we have 
    \[ \|f\|_2^2 = \sum_{n\in\Z} |\hat f(n)|^2. \] 
\end{cor}
\begin{pf}
    We know that $S_Nf \to f$ with respect to $\|\cdot\|_2$. Applying 
    Parseval's Theorem gives us 
    \[ \|f\|_2 = \lim_{N\to\infty} \|S_N f\|_2 
    = \lim_{N\to\infty} \sum_{n=-N}^N |\hat f(n)|^2 = 
    \sum_{n\in\Z} |\hat f(n)|^2. \qedhere \] 
\end{pf}

\begin{defn}{defn:4.9}
    Let $(H_1, \|\cdot\|_{H_1})$ and $(H_2, \|\cdot\|_{H_2})$ be Hilbert 
    spaces. A {\bf unitary isomorphism} from $H_1$ to $H_2$ is a 
    linear map $U : H_1 \to H_2$ which is surjective and isometric. That is, 
    for all $x \in H_1$, we have 
    \[ \|Ux\|_{H_2} = \|x\|_{H_1}. \] 
    Note that isometric implies injective, so $U$ is a bijection. 
\end{defn}

In particular, Plancherel's Theorem (Corollary~\ref{cor:4.8}) says that 
the function $U : L^2(\T) \to \ell^2(\Z)$ defined by $f \mapsto 
(\hat f(n))_{n\in\Z}$ is a unitary isomorphism. Then for 
$f, g \in L^2(\T)$, we have 
\[ \langle Uf, Ug \rangle = \sum_{n\in\Z} \hat f(n) \overline{\hat g(n)} 
= \lim_{N\to\infty} \sum_{n=-N}^N \hat f(n) \overline{\hat g(n)} 
= \lim_{N\to\infty} \langle S_n f, g \rangle 
= \langle f, g \rangle 
= \frac{1}{2\pi} \int_{-\pi}^\pi f(\theta) \overline{g(\theta)} \dd\theta, \] 
so $U$ preserves inner products. In fact, this property is equivalent to being 
surjective and isometric. 

\subsection{Pointwise and Uniform Convergence of Fourier Series} \label{subsec:4.3}
So far, we know that if $f \in L^2(\T)$, then the Fourier coefficients 
$(\hat f(n))_{n\in\Z}$ allow us to completely recover $f$ in terms of 
$L^2$-convergence. Namely, we have 
\[ S_Nf(\theta) = \sum_{n=-N}^N \hat f(n) e^{in\theta} \] 
where $\lim_{N\to\infty} \|f - S_Nf\|_2 = 0$. Moreover, by abstract 
measure theory, we know that for all $f \in L^2(\T)$, there exists a subsequence 
$(N_k)_{k=1}^\infty$ with $N_k < N_{k+1}$ for all $k \in \N$ such that 
\[ f(\theta) = \lim_{k\to\infty} S_{N_k}f(\theta) = \lim_{k\to\infty} 
\sum_{n=-N_k}^{N_k} \hat f(n) e^{in\theta} \] 
for almost every $\theta \in [-\pi, \pi]$. 

Our fundamental problem is the following: for all $f \in L^2(\T)$, do we have 
$S_Nf \to f$ almost everywhere (that is, without passing to a subsequence)? 
It turns out the answer is yes! However, this result is far beyond the 
scope of the course. We state it to give some context. 

\begin{theo}[Carleson's Theorem]{theo:4.10}
    Let $f \in L^p(\T)$ where $p \in (1, \infty]$. Then $S_n f \to f$ 
    almost everywhere, and $\|S_N f - f\|_p \to 0$. 
\end{theo}

In other words, there is no need to pass to a subsequence to get convergence 
almost everywhere. However, notice that in the above theorem, we had $p > 1$. 
The bad news is that everything falls apart when $p = 1$. 

\begin{theo}[Kolmogorov's Counterexample]{theo:4.11}
    There exists $f \in L^1(\T)$ such that $S_N f(\theta)$ diverges for 
    all $\theta \in [-\pi, \pi]$, and $S_n f \nrightarrow f$ in $\|\cdot\|_1$. 
\end{theo}

This result is again beyond the scope of the course, but it tells us that 
the topic of pointwise convergence is a very delicate and difficult one. 

Now, we will investigate the pointwise convergence of Fourier series for 
more ``regular'' functions. For example, if $f \in C(\T)$ or $f \in C^1(\T)$ 
(so that $f'$ exists and is continuous), is it true that $S_N f(\theta) \to 
f(\theta)$ for all $\theta \in [-\pi, \pi]$? 

\begin{lemma}{lemma:4.12}
    Let $f \in L^2(\T)$, and suppose that $(S_N f)_{N=1}^\infty 
    \subseteq \Pol(\T)$ is Cauchy with respect to $\|\cdot\|_\infty$. 
    Then there exists $g \in C(\T)$ such that $f = g$ almost everywhere 
    and $\|f - S_N f\|_\infty \to 0$. 
\end{lemma}
\begin{pf}
    Since $(S_N f)_{N=1}^\infty \subseteq \Pol(\T) \subseteq C(\T)$ is 
    Cauchy in $\|\cdot\|_\infty$, there exists a function $g \in C(\T)$ such 
    that $\|g - S_N f\|_\infty \to 0$. Moreover, by Theorem~\ref{theo:4.7}, 
    we know that $\|f - S_N f\|_2 \to 0$. To show that $f = g$
    almost everywhere, note that $f, g \in L^2(\T)$, so it is enough 
    to show that $\|g - f\|_2 \to 0$. Indeed, we see that 
    \[ \|g - f\|_2 \leq \|g - S_N f\|_2 + \|S_N f - f\|_2 
    \leq \|g - S_N f\|_\infty + \|S_N f - f\|_2 \to 0. \]
    This means that $f = g$ almost everywhere and $f \in L^\infty(\T)$, 
    with $\|f - S_N f\|_\infty = \|g - S_N f\|_\infty \to 0$. 
\end{pf}

\begin{cor}{cor:4.13}
    Let $f \in C(\T)$, and suppose that its Fourier coefficients satisfy 
    $\hat f \in \ell^1(\Z)$ (that is, $\sum_{n\in\Z} |\hat f(n)| < \infty$). 
    Then $S_N f \to f$ uniformly. 
\end{cor}
\begin{pf}
    Note that $|\hat f(n) e^{in\theta}| = |\hat f(n)|$, so the Fourier series 
    \[ \sum_{n\in\Z} \hat f(n) e^{in\theta} \] 
    converges absolutely with respect to $\|\cdot\|_\infty$. This 
    means that $(S_N f)_{N=1}^\infty$ is Cauchy in $\|\cdot\|_\infty$, 
    and applying Lemma~\ref{lemma:4.12} gives the result. 
\end{pf}

Now, we consider the convergence of a function under some differentiability 
and ``smoothness'' conditions. Here, we do not mean smooth in the sense 
that the function is infinitely differentiable, but more so how nicely 
it behaves. More specifically, we will discuss pointwise convergence 
when the function is continuously differentiable everywhere, 
and when the function is locally Lipschitz at a point. 

\begin{prop}{prop:4.14}
    Let $f \in C^1(\T)$. Then $S_N f \to f$ uniformly. 
\end{prop}
\begin{pf}
    By Corollary~\ref{cor:4.13}, it suffices to show that $\hat f \in 
    \ell^1(\Z)$. We will first compute the Fourier coefficients of 
    $f'$, and see how they relate with the Fourier coefficients of $f$. 
    Observe that 
    \begin{align*}
        \widehat{f'}(n) 
        &= \frac1{2\pi} \int_{-\pi}^\pi f'(\theta) e^{-in\theta}\dd\theta \\
        &= \frac1{2\pi} f(\theta) e^{-in\theta} \bigg|_{-\pi}^\pi 
        + \frac{in}{2\pi} \int_{-\pi}^\pi f(\theta) e^{-in\theta}\dd\theta \\ 
        &= 0 + in \hat f(n).
    \end{align*}
    Note that by assumption, we have $f' \in C(\T) \subseteq L^2(\T)$, which 
    means that $\widehat{f'} \in \ell^2(\Z)$. This tells us that 
    \[ \sum_{n\in\Z} |\widehat{f'}(n)|^2 = \sum_{n\in\Z} n^2 |\hat f(n)|^2 < \infty. \] 
    Applying the Cauchy-Schwarz inequality, we obtain 
    \[ \sum_{n\in\Z} |\hat f(n)| 
    = |\hat f(0)| + \sum_{n\neq 0} |n| |\hat f(n)| \left| \frac1n \right| 
    \leq |\hat f(0)| + \left( \sum_{n\neq 0} |n^2| |\hat f(n)|^2 \right)^{\!1/2}
    \left( \sum_{n\neq 0} \frac{1}{n^2} \right)^{\!1/2} < \infty, \] 
    and thus $\hat f \in \ell^1(\Z)$. The result follows. 
\end{pf}

What happens if $f'$ exists but is not continuous? What if $f$ is not even
differentiable everywhere? The previous arguments will no longer apply. 

\begin{theo}{theo:4.15}
    Let $f \in L^1(\T)$, and assume that $f$ is differentiable at a point 
    $\theta_0 \in [-\pi, \pi]$. Then $S_N f(\theta_0) \to f(\theta_0)$. 
\end{theo}
\begin{pf}
    We will prove something stronger: if $f \in L^1(\T)$ and 
    $f$ is locally Lipschitz at $\theta_0 \in [-\pi, \pi]$, then 
    \[ f(\theta_0) = \lim_{N\to\infty} S_N f(\theta_0). \] 
    Recall that $f$ is {\bf locally Lipschitz} at $\theta_0 \in [-\pi, \pi]$ 
    if there exists $L > 0$ and $\delta > 0$ such that $0 < 
    |\theta - \theta_0| < \delta$ implies $|f(\theta) - f(\theta_0)| 
    \leq L|\theta - \theta_0|$. Note that the existence of $f'(\theta_0)$ 
    implies that $f$ is locally Lipschitz at $\theta_0$, which then 
    implies that $f$ is continuous at $\theta_0$. 

    We will assume without loss of generality that $\theta_0 = 0$ 
    and $f(\theta_0) = 0$. This is because we can replace $f$ with 
    $\tilde f(\theta) = f(\theta + \theta_0) - f(\theta_0)$, and on 
    Assignment 4, we will show that the Fourier coefficients of 
    $\tilde f$ can be computed in terms of those of $f$. 

    Our goal now is to show that 
    \[ S_N f(0) = \sum_{n=-N}^N \hat f(n) \to f(0) = 0. \] 
    We will show this using a clever proof due to P. R. Chernoff. 
    Define the function 
    \[ g(\theta) = \begin{cases}
        f(\theta)/(e^{i\theta} - 1), & \text{if } \theta \in [-\pi, \pi] 
        \setminus \{0\}, \\ 
        0, & \text{if } \theta = 0. 
    \end{cases} \] 
    Note that $g$ is measurable. The value $g(0)$ is irrelevant, but 
    we note that if $f$ is differentiable, then it is natural to take 
    $g(0) = f'(0)/i$ since $g$ would then be continuous at $0$. 

    We claim that $g \in L^1(\T)$. The hard part here is the behaviour 
    of $g$ when $\theta$ is close to $0$. But since $f$ is locally Lipschitz 
    at $0$, there exists $L > 0$ and $\delta > 0$ such that 
    $0 < |\theta| < \delta$ implies $|f(\theta)| \leq L|\theta|$. 
    Then we get 
    \[ \|g\|_1 = \frac1{2\pi} \int_{|\theta|\leq\delta} |g(\theta)|\dd\theta 
    + \frac1{2\pi} \int_{|\theta|>\delta} |g(\theta)|\dd\theta. \] 
    The second integral is finite because $f \in L^1(\T)$ and 
    $1/|e^{i\theta}-1| \leq C$ for some constant $C$ when $|\theta| > \delta$. 
    We shall denote the value of the integral by $C'$. Now, we obtain 
    \begin{align*}
        \|g\|_1 &= \frac1{2\pi} \int_{|\theta|\leq\delta} |g(\theta)|\dd\theta + C' \\ 
        &\leq \frac{1}{2\pi} \int_{|\theta|\leq\delta} \frac{|f(\theta)|}{|e^{i\theta} - 1|}\dd\theta + C' \\ 
        &\leq \frac{1}{2\pi} \int_{|\theta|\leq\delta} \frac{L|\theta|}{|e^{i\theta} - 1|}\dd\theta + C' < \infty 
    \end{align*}
    since $L|\theta|/|e^{i\theta} - 1|$ is continuous at $\theta$, so we indeed 
    have $g \in L^1(\T)$. 

    Note that $f(\theta) = (e^{i\theta} - 1)g(\theta)$ for all $\theta \in 
    [-\pi, \pi]$, so we can write $f = (e_1 - 1)g$. Moreover, we can compute 
    \[ \hat f(n) = \widehat{e_1g}(n) - \hat g(n) = \hat g(n-1) - \hat g(n), \] 
    which implies that 
    \[ S_N f(0) = \sum_{n=-N}^N \hat f(n) = 
    \sum_{n=-N}^N (\hat g(n-1) - \hat g(n)) = \hat g(-N-1) - \hat g(N). \] 
    On Assignment 4, we will show that $g \in L^1(\T)$ implies that 
    $\hat g \in c_0(\Z)$ where $c_0(\Z) \subseteq \ell^\infty(\Z)$ is the 
    closed subspace of sequences $a = (a_n)_{n\in\Z}$ which vanish at infinity.
    That is, $a = (a_n)_{n\in\Z}$ satisfies $\|a\|_\infty = \sup_{n\in\Z} 
    |a_n| < \infty$ and 
    \[ \lim_{n\to\infty} a_n = \lim_{n\to\infty} a_{-n} = 0. \]  
    In particular, we get $\hat g(N) \to 0$ and $\hat g(-N) \to 0$ 
    as $N \to \infty$, and thus $S_N f \to 0$, as desired. 
\end{pf}

\begin{exmp}{exmp:4.16}
    Consider the function $f$ on $\T$ defined by 
    \[ f(\theta) = \begin{cases}
        1, & \text{if } \theta \in [0, \pi], \\ 
        0, & \text{if } \theta \in (-\pi, \theta). 
    \end{cases} \] 
    We will find its Fourier coefficients and investigate the pointwise 
    convergence of the Fourier series of $f$. First, we see that 
    \[ \hat f(0) = \frac{1}{2\pi} \int_{-\pi}^\pi f(\theta)\dd\theta 
    = \frac{1}{2\pi} \int_0^\pi 1\dd\theta = \frac12. \] 
    When $n \neq 0$, we obtain 
    \begin{align*}
        \hat f(n) 
        &= \frac{1}{2\pi} \int_{-\pi}^\pi f(\theta) e^{-in\theta}\dd\theta \\
        &= \frac{1}{2\pi} \int_0^\pi e^{-in\theta}\dd\theta \\
        &= -\frac{1}{2\pi in} e^{-in\theta} \bigg|_0^\pi \\
        &= \frac{1}{2\pi in} (1 - (-1)^n).
    \end{align*}
    In particular, notice that $\hat f(n) = -\hat f(-n)$ for $n \neq 0$. Then 
    the Fourier series of $f$ is given by 
    \begin{align*}
        \sum_{n\in\Z} \hat f(n) e^{in\theta} 
        &= \frac12 + \frac{1}{2\pi i} \sum_{n\in\Z\setminus\{0\}} \frac{1 - (-1)^n}{n} e^{in\theta} \\ 
        &= \frac12 + \frac1\pi \sum_{n=1}^\infty \frac{1 + (-1)^n}n \sin(n\theta) \\ 
        &= \frac12 + \frac1\pi \sum_{k=1}^\infty \frac{2}{2k-1} \sin((2k-1)\theta).  
    \end{align*}
    By Theorem~\ref{theo:4.15}, we know that $S_Nf(\theta) \to f(\theta)$ 
    whenever $\theta \notin \{0, \pi, -\pi\}$ because $f$ is differentiable 
    away from the above points. Taking $\theta = \pi/2$, we obtain 
    \[ 1 = f(\pi/2) = \lim_{n\to\infty} S_Nf(\pi/2) = 
    \frac12 + \frac1\pi \sum_{k=1}^\infty \frac{2}{2k-1} (-1)^{k+1}, \] 
    which yields the familiar formula 
    \[ \frac{\pi}{4} = 1 - \frac13 + \frac15 - \frac17 + \cdots 
    = \sum_{k=0}^\infty \frac{(-1)^{k+1}}{2k+1}. \] 
    However, our interest is the behaviour at $\theta_0 \in \{0, \pi, -\pi\}$, 
    where $f$ has jump discontinuities. At these points, we find that 
    \[ S_N f(\theta_0) = \frac12 = \frac{f(\theta_0^+) + f(\theta_0^-)}{2}, \] 
    where $f(\theta_0^+) = \lim_{\theta\to\theta_0^+} f(\theta)$ 
    and $f(\theta_0^-) = \lim_{\theta\to\theta_0^-} f(\theta)$. So the 
    Fourier series does not converge to $f$ at these values, but is 
    equal to the average of the left and right values! 
\end{exmp}

We will see in the following theorem that this sort of behaviour at jump 
discontinuities is generic. 

\begin{theo}{theo:4.17}
    Let $f \in L^1(\T)$ and $\theta_0 \in [-\pi, \pi]$. Assume that 
    \begin{enumerate}[(1)]
        \item $f(\theta_0^+) = \lim_{\theta\to\theta_0^+} f(\theta)$ 
        and $f(\theta_0^-) = \lim_{\theta\to\theta_0^-} f(\theta)$ exist, and 
        \item $f'(\theta_0^+) = \lim_{h\to0^+} (f(\theta_0+h) - f(\theta_0^+))/h$ 
        and $f'(\theta_0^-) = \lim_{h\to0^-} (f(\theta_0+h) - f(\theta_0^-))/h$ exist. 
    \end{enumerate} 
    Then we have 
    \[ \lim_{N\to\infty} S_N f(\theta_0) = \frac{f(\theta_0^+) + f(\theta_0^-)}{2}. \]
\end{theo}
\begin{pf}
    First, we define the step function 
    \[ h(\theta) = \begin{cases}
        f(\theta_0^-), & \text{if } \theta < \theta_0, \\ 
        f(\theta_0), & \text{if } \theta = \theta_0, \\ 
        f(\theta_0^+), & \text{if } \theta > \theta_0.
    \end{cases} \] 
    Let $\tilde h(\theta) = h(\theta + \theta_0) - f(\theta_0^-)$. 
    By our computations in Example~\ref{exmp:4.16}, we find that 
    \[ S_N \tilde h(\theta) \to \frac{f(\theta_0^+) - f(\theta_0^-)}{2}. \] 
    In particular, it follows that 
    \[ S_N h(\theta_0) = S_N \tilde h(0) + f(\theta_0^-) \to
    \frac{f(\theta_0^+) + f(\theta_0^-)}{2}. \] 
    Now, let $k = f - h$. We see that $k(\theta_0) = f(\theta_0) - h(\theta_0) = 0$ 
    and $k(\theta_0^+) = k(\theta_0^-) = 0$. Moreover, 
    $k'(\theta_0^+) = f'(\theta_0^+)$ and $k'(\theta_0^-) = f'(\theta_0^-)$ 
    both exist, implying that $k$ is locally Lipschitz at $\theta_0$. 
    By Theorem~\ref{theo:4.15}, we have $S_N k(\theta_0) \to k(\theta_0) = 0$ 
    and hence $S_N f(\theta_0) - S_N h(\theta_0) \to 0$. Since 
    $S_N h(\theta_0) \to (f(\theta_0^+) + f(\theta_0^-))/2$, we obtain 
    the desired result.  
\end{pf}

Next, we discuss term-by-term integration of Fourier series. 

\begin{theo}{theo:4.18}
    Let $f \in L^2(\T)$. For $\theta \in [-\pi, \pi]$, define 
    \[ g(\theta) = \int_{-\pi}^\theta f(t)\dd t. \] 
    Then $g \in C[-\pi, \pi]$, and we have 
    \begin{align*}
        g(\theta) = \sum_{n\in\Z} \hat f(n) \int_{-\pi}^\theta  e^{int}\dd t 
        = \hat f(0) (\theta + \pi) + \sum_{n\in\Z\setminus\{0\}} 
        \frac{\hat f(n)}{in} (e^{in\theta} - (-1)^n). 
    \end{align*}
    Moreover, the convergence is uniform. (Note that in general, 
    we do not have $g \in C(\T)$. This is only the case if 
    $g(\pi) = g(-\pi) = 0$, which is equivalent to saying that 
    $\hat f(0) = 0$. In the case that $\hat f(0) = 0$, it turns out 
    that the above is the Fourier series for $g$.)
\end{theo}
\begin{pf}
    Since $f \in L^2(\T)$, we have $S_N f \to f$ with respect to $\|\cdot\|_2$. 
    For $N \in \N$, let 
    \[ g_N(\theta) = \int_{-\pi}^\theta S_N f(t)\dd t = 
    \sum_{n=-N}^N \hat f(n) \int_{-\pi}^\theta e^{int}\dd t. \] 
    Then we obtain 
    \[ |g(\theta) - g_N(\theta)|^2 
    = \left| \int_{-\pi}^\theta (f(t) - S_Nf(t))\dd t \right|^2 
    \leq \left( \int_{-\pi}^\theta |f(t) - S_Nf(t)|\dd t \right)^{\!2} 
    \leq |\theta + \pi| \|f - S_N f\|_2^2  \to 0 \] 
    uniformly in $\theta$, where the last inequality was due 
    to Cauchy-Schwarz. 
\end{pf}
\newpage

\end{document}
