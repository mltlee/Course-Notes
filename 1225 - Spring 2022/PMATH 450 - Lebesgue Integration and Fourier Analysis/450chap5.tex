\section{Convolution and Summability Kernels} \label{sec:5}

\subsection{Convolution} \label{subsec:5.1}
Recall that the general problem of Fourier analysis is to try 
and recover a function $f$ from $L^1(\T)$, $L^2(\T)$, or $C(\T)$ 
given its Fourier transform ${\cal F}f = (\hat f(n))_{n\in\Z}$. 
We already know the following: 
\begin{enumerate}[(1)]
    \item If $f \in L^2(\T)$, then we can recover $f$ as 
    $\lim_{N\to\infty} S_N f$ in the sense of $L^2$-convergence. 
    \item If $f$ is locally Lipschitz at a point $\theta$ (that is, 
    $f$ is ``regular''), then $f(\theta) = \lim_{N\to\infty} S_N f(\theta)$. 
    \item If $f \in C^1(\T)$, then $\|f - S_N f\|_\infty \to 0$. 
\end{enumerate}
But what if we merely assume that $f \in C(\T)$ or $f \in L^1(\T)$? 
In the case of $L^1(\T)$, it might not seem possible to recover 
$f$ from ${\cal F}f$ in light of Kolmogorov's counterexample 
(Theorem~\ref{theo:4.11}). We will see that it actually is possible.
We first take a look at the convolution of two functions. 

\begin{prop}{prop:5.1}
    Let $f \in C(\T)$ and $g \in L^1(\T)$. Define a function 
    \[ h(\theta) = \frac{1}{2\pi} \int_{-\pi}^\pi f(\theta - t) g(t)\dd t. \] 
    Note that this is well-defined because the function 
    $t \mapsto f(\theta - t) g(t)$ is in $L^1(\T)$. We have 
    \begin{enumerate}[(1)]
        \item $h \in C(\T)$ and $\|h\|_\infty \leq \|f\|_\infty \|g\|_1$; 
        \item $\|h\|_1 \leq \|f\|_1 \|g\|_1$ and $\|h\|_2 \leq \|f\|_2 \|g\|_1$. 
    \end{enumerate}
    We write $h = f \star g$, and call it the {\bf convolution} of $f$ and $g$. 
\end{prop}
\begin{pf}~
    \begin{enumerate}[(1)]
        \item Since $f \in C(\T)$ and $\T$ is compact, we know that $f$ 
        is uniformly continuous. So for all $\eps > 0$, there exists 
        $\delta > 0$ such that $|s - t| < \delta$ implies 
        $|f(s) - f(t)| < \eps$. Suppose that $\theta, \theta_0 \in \T$ with 
        $|\theta - \theta_0| < \delta$. We get 
        \begin{align*}
            |h(\theta) - h(\theta_0)| 
            &= \frac{1}{2\pi} \left| \int_{-\pi}^\pi (f(\theta - t) - f(\theta_0 - t)) g(t)\dd t \right| \\ 
            &\leq \frac{1}{2\pi} \int_{-\pi}^\pi |f(\theta - t) - f(\theta_0 - t)| |g(t)|\dd t \\ 
            &< \frac{1}{2\pi} \int_{-\pi}^\pi \eps |g(t)|\dd t \leq \eps\|g\|_1, 
        \end{align*}
        where the second last inequality is because $|\theta - t - (\theta_0 - t)| < \delta$. 
        So $h \in C(\T)$, and it is clear that $\|h\|_\infty \leq 
        \|f\|_\infty \|g\|_1$ by a similar computation to above. 

        \item First suppose that $g \in C(\T)$. Then we have 
        \begin{align*}
            \|h\|_1 &= \frac{1}{2\pi} \int_{-\pi}^\pi \left| \frac{1}{2\pi} \int_{-\pi}^\pi f(\theta - t) g(t)\dd t \right|\textrm{d}\theta \\ 
            &\leq \frac{1}{(2\pi)^2} \int_{-\pi}^\pi \int_{-\pi}^\pi |f(\theta - t)| |g(t)|\dd t \dd\theta \\ 
            &= \frac{1}{(2\pi)^2} \int_{-\pi}^\pi |g(t)| \left( \int_{-\pi}^\pi |f(\theta - t)|\dd\theta \right)\textrm{d}t
            = \|f\|_1 \|g\|_1,
        \end{align*}
        where the second last equality was obtained by applying Fubini's theorem
        to interchange $\textrm{d}t$ and $\textrm{d}\theta$, 
        noting that this is a Riemann integral since $f, g \in C(\T)$. 
        Similarly, we have 
        \begin{align*}
            \|h\|_2^2 
            &= \frac{1}{2\pi} \int_{-\pi}^\pi \left| \frac{1}{2\pi} \int_{-\pi}^\pi f(\theta - t) g(t)\dd t \right|^2 \textrm{d}\theta \\ 
            &\leq \frac{1}{2\pi} \int_{-\pi}^\pi \left( \frac{1}{2\pi} \int_{-\pi}^\pi |f(\theta - t)| |g(t)|\dd t \right)^{\!2} \textrm{d}\theta \\ 
            &\leq \frac{1}{2\pi} \int_{-\pi}^\pi \left( \frac{1}{2\pi} \int_{-\pi}^\pi |f(\theta - t)|^2 |g(t)|\dd t \right) \left( \frac{1}{2\pi} \int_{-\pi}^\pi |g(t)|\dd t \right)\textrm{d}\theta \\ 
            &= \|g\|_1 \cdot \frac{1}{2\pi} \int_{-\pi}^\pi |f(\theta - t)|^2 |g(t)|\dd t \dd \theta = \|g\|_1 \|f\|_2^2 \|g\|_1, 
        \end{align*}
        where the last equality was due to Fubini, and the second inequality 
        was from applying Cauchy-Schwarz to $|f(\theta - t)||g(t)| 
        = |f(\theta - t)||g(t)|^{1/2} |g(t)|^{1/2}$. 

        Suppose now that $g \in L^1(\T)$. Take a sequence $(g_n)_{n=1}^\infty 
        \subseteq C(\T)$ such that $g_n \to g$ in $\|\cdot\|_1$. 
        Then we know for $p \in \{1, 2\}$ that 
        $\|f \star g_n\|_p \leq \|f\|_p \|g_n\|_1$.
        We claim that $(f \star g_n)_{n=1}^\infty$ is Cauchy in $\|\cdot\|_p$ 
        for $p \in \{1, 2\}$. Indeed, we see that 
        \[ \|f \star g_n - f \star g_m\|_p = \|f \star (g_n - g_m)\|_p 
        \leq \|f\|_p \|g_n - g_m\|_1 \to 0 \] 
        since $g_n \to g$ in $\|\cdot\|_1$, and thus $\|g_n - g_m\|_1 \to 0$. 
        This means that $h = f \star g = \lim_{n\to\infty} f \star g_n$, 
        since the sequence $(f \star g_n)_{n=1}^\infty$ being Cauchy 
        implies that the limit exists, and 
        \[ \|f \star g - f \star g_n\|_p = \|f \star (g - g_n)\|_p 
        \leq \|f\|_p \|g - g_n\|_1 \to 0. \] 
        It follows that 
        $\|f \star g\|_p = \lim_{n\to\infty} \|f \star g_n\|_p 
        \leq \lim_{n\to\infty} \|f\|_p \|g_n\|_1 = \|f\|_p \|g\|_1$. \qedhere  
    \end{enumerate}
\end{pf}

\begin{remark}{remark:5.2}
    If $f, g \in C(\T)$, then $\star : C(\T) \times C(\T) \to C(\T)$ 
    turns $C(\T)$ into a {\bf commutative algebra} over $\C$. That is, 
    for all $f, g, h \in C(\T)$ and $\alpha \in \C$, we have 
    \begin{itemize}
        \item $f \star g = g \star f$; 
        \item $f \star (g + h) = f \star g + f \star h$; 
        \item $(f \star g) \star h = f \star (g \star h)$; and 
        \item $\alpha(f \star g) = (\alpha f) \star g$. 
    \end{itemize}
    It can be seen that $f \star g = g \star f$ by making a substitution 
    $s = \theta - t$. 
\end{remark}

\begin{theo}{theo:5.3}
    The convolution product $\star$ extends continuously to a multiplication 
    $\star : L^1(\T) \times L^1(\T) \to L^1(\T)$ given by $(f, g) \mapsto 
    g \star g = g \star f$, and it satisfies $\|f \star g\|_1 \leq \|f\|_1 \|g\|_1$. 
    Moreover, for $f, g \in L^1(\T)$, the function $t \mapsto f(\theta - t) g(t)$
    is in $L^1(\T)$ for almost every $\theta \in [-\pi, \pi]$, and for such 
    $\theta$, we have 
    \[ (f \star g)(\theta) = \frac{1}{2\pi} \int_{-\pi}^\pi f(\theta - t)g(t)\dd t. \] 
\end{theo}
\begin{pf}
    It turns out that the integrability of the function $t \mapsto f(\theta - t) g(t)$ 
    is highly non-trivial and requires a measure theoretic version of 
    Fubini, which is typically seen in PMATH 451. We will therefore 
    content ourselves to abstractly define $f \star g$ for 
    $f, g \in L^1(\T)$ via a limiting process. 

    Let $(f_n)_{n=1}^\infty, (g_n)_{n=1}^\infty \subseteq C(\T)$ 
    be sequences such that $f_n \to f$ and $g_n \to g$ in $\|\cdot\|_1$. 
    Note that $(f_n \star g_n)_{n=1}^\infty$ is Cauchy in $\|\cdot\|_1$. 
    Indeed, we see that  
    \begin{align*}
        \|f_n \star g_n - f_m \star g_m\|_1 
        &= \|(f_n - f_m) \star g_n + f_m \star (g_n - g_m)\|_1 \\ 
        &\leq \|(f_n - f_m) \star g_n\|_1 + \|f_m \star (g_n - g_m)\|_1 \\ 
        &\leq \|f_n - f_m\|_1 \|g_n\|_1 + \|f_m\|_1 \|g_n - g_m\|_1 \to 0 
    \end{align*}
    as $m, n \to \infty$ because $(f_n)_{n=1}^\infty$ and $(g_n)_{n=1}^\infty$ 
    are Cauchy. We define $f \star g = \lim_{n\to\infty} 
    f_n \star g_n \in L^1(\T)$. To see this is well-defined, we show that 
    $f \star g$ is independent of the choice of $(f_n)_{n=1}^\infty$ 
    and $(g_n)_{n=1}^\infty$. Suppose that $(\tilde f_n)_{n=1}^\infty$ 
    and $(\tilde g_n)_{n=1}^\infty$ also satisfy $\tilde f_n \to f$ and 
    $\tilde g_n \to g$ in $\|\cdot\|_1$. Then we obtain 
    \[ \|f_n \star g_n - \tilde f_n \star \tilde g_n\|_1 
    \leq \|f_n - \tilde f_n\|_1 \|g_n\|_1 + \|\tilde f_n\|_1 
    \|g_n - \tilde g_n\|_1 \to 0, \] 
    and thus $f \star g = \lim_{n\to\infty} f_n \star g_n = 
    \lim_{n\to\infty} \tilde f_n \star \tilde g_n$. 
\end{pf}

\subsection{Properties of Summability Kernels} \label{subsec:5.2}
From Theorem~\ref{theo:5.3}, we see that $L^1(\T)$ is an algebra under convolution. 
Note that $L^1(\T)$ is not unital; that is, there does not exist $k \in L^1(\T)$ 
such that $f \star k = k \star f = f$ for all $f \in L^1(\T)$. Indeed, assume 
towards such a function $k \in L^1(\T)$ existed. Then for all $f \in L^1(\T)$, 
we would have 
\[ f(\theta) = (f \star k)(\theta) = \frac{1}{2\pi} \int_{-\pi}^\pi f(\theta - t)
k(t)\dd t. \] 
Informally, this would force $k$ to be the Dirac delta function at $0$, 
which is given by 
\[ \delta(t) = \begin{cases}
    \infty, & \text{if } t = 0, \\ 
    0, & \text{if } t \neq 0.
\end{cases} \] 
It is clear that this is not integrable, which is a contradiction.

But $L^1(\T)$ has ``approximate'' units with respect to convolution; they are 
called summability kernels. 

\begin{defn}{defn:5.4}
    A sequence $(k_n)_{n=1}^\infty \subseteq L^1(\T)$ is called a {\bf summability
    kernel} if it satisfies the following properties: 
    \begin{enumerate}[(1)]
        \item For all $n \in \N$, we have 
        \[ \frac{1}{2\pi} \int_{-\pi}^\pi k_n(\theta)\dd\theta = 1. \] 
        \item The sequence $(k_n)_{n=1}^\infty$ is uniformly bounded; that is, 
        $\sup_{n\in\N} \|k_n\|_1 = C < \infty$ for some constant $C$. 
        \item For all $\delta > 0$, we have 
        \[ \lim_{n\to\infty} \int_{|\theta|\geq\delta} |k_n(\theta)|\dd\theta = 0. \]
    \end{enumerate}
    We say that a summability kernel is {\bf even} or {\bf symmetric} if $k_n(\theta) 
    = k_n(-\theta)$ for almost every $\theta \in [-\pi, \pi]$, and {\bf positive} 
    if $k_n \geq 0$ for all $n \in \N$. 
\end{defn}

Pictorially, there will be very large values near $0$, but small values away from $0$. 
Note that when $(k_n)_{n=1}^\infty$ is a positive summability kernel, then 
condition (1) implies condition (2) with the constant $C = 1$. 
An example of a symmetric and positive summability kernel is $(k_n)_{n=1}^\infty$ given by 
$k_n = \pi n\chi_{[-1/n, 1/n]}$. 

The following theorem tells us that summability kernels play the role of an approximate 
unit with respect to convolution. 

\begin{theo}{theo:5.5}
    Let $(k_n)_{n=1}^\infty$ be a summability kernel. 
    \begin{enumerate}[(1)]
        \item If $f \in C(\T)$, then as $n \to \infty$, we have 
        \[ \|f - f \star k_n\|_\infty \to 0. \] 
        \item Let $p \in \{1, 2\}$ and $f \in L^p(\T)$. Then as $n \to \infty$, we have 
        \[ \|f - f \star k_n\|_p \to 0. \] 
    \end{enumerate}
\end{theo}
\begin{pf}~
    \begin{enumerate}[(1)]
        \item Let $f \in C(\T)$ and $\eps > 0$. By the uniform continuity of $f$, there exists 
        $\delta > 0$ such that $|f(s) - f(t)| < \eps$ whenever $|s - t| < \delta$. Then we have 
        \begin{align*}
            |f(\theta) - (f \star k_n)(\theta)| 
            &= \frac{1}{2\pi} \left| \int_{-\pi}^\pi (f(\theta) - f(\theta - t)) k_n(t)\dd t \right| \\ 
            &\leq \frac{1}{2\pi} \int_{-\pi}^\pi |f(\theta) - f(\theta - t)| |k_n(t)|\dd t \\ 
            &= \frac{1}{2\pi} \int_{|t|\leq\delta} |f(\theta) - f(\theta - t)||k_n(t)|\dd t 
            + \frac{1}{2\pi} \int_{|t|>\delta} |f(\theta) - f(\theta - t)||k_n(t)|\dd t \\ 
            &< \eps\|k_n(t)\|_1 + 2\|f\|_\infty \int_{|t|>\delta} |k_n(t)|\dd t, 
        \end{align*}
        where the first equality is due to property (1) of being a summability kernel, which 
        states that 
        \[ \frac{1}{2\pi} \int_{-\pi}^\pi k_n(t)\dd t = 1. \] 
        Taking $n \to \infty$, property (3) of being a summability kernel tells us that 
        \[ \int_{|t|>\delta} |k_n(t)|\dd t \to 0, \] 
        and thus $|f(\theta) - (f \star k_n)(\theta)| \leq C\eps$ uniformly in $\theta$. 
        Hence, we have $\|f - f \star k_n\|_\infty \to 0$. 

        \item Let $p \in \{1, 2\}$, and let $f \in L^p(\T)$. Let $\eps > 0$. Then by 
        Corollary~\ref{cor:3.21}, we can find $g \in C(\T)$ such that $\|f - g\|_p < \eps$. 
        Choose sufficiently large $n \in \N$ such that 
        \[ \|g - g \star k_n\|_\infty < \eps, \] 
        which exists since $\|g - g \star k_n\|_\infty \to 0$ by part (1). Then we obtain 
        \begin{align*}
            \|f - f \star k_n\|_p 
            &\leq \|f - g\|_p + \|g - g \star k_n\|_p + \|g \star k_n - f \star k_n\|_p \\ 
            &< \eps + \|g - g \star k_n\|_\infty + \|(g - f) \star k_n\|_p \\ 
            &< \eps + \eps + \|g - f\|_p \|k_n\|_1 \\ 
            &< 2\eps + C\eps \\ 
            &= (C + 2)\eps. \qedhere 
        \end{align*}
    \end{enumerate}
\end{pf}

\subsection{The Fej\'er Kernel and Dirichlet Kernel} \label{subsec:5.3}
We now look at some applications of summability kernels. First, we consider 
Ces\`aro summation of Fourier series. Given a series $\sum_{n=0}^\infty a_n$
where $(a_n)_{n=0}^\infty \subseteq \C$, we know that it may or may not converge. 
That is, the $N$-th partial sum $S_N = \sum_{n=0}^N a_n$ may or may not converge 
as a sequence in $\C$. A simple example of this is $a_n = (-1)^n$, where it is 
clear that $\sum_{n=0}^\infty (-1)^n$ does not converge. 

\begin{defn}{defn:5.6}
    Let $\sum_{n=0}^\infty a_n$ be a series with $(a_n)_{n=0}^\infty \subseteq \C$, 
    and suppose that it has partial sums $(S_N)_{N=0}^\infty$. The 
    {\bf $N$-th Cesa\`ro sum mean} of the series is given by 
    \[ \sigma_N = \frac{1}{N+1} \sum_{k=0}^N S_k. \] 
\end{defn}

How are the convergence properties of the partial sums $(S_N)_{N=0}^\infty$ 
and the Ces\`aro means $(\sigma_N)_{N=0}^\infty$ related? It turns out 
that if the partial sums converge, then the Ces\`aro means converge to the 
same limit. But the converse fails, because taking our example $a_n = (-1)^n$ 
above, we find that 
\[ \sigma_N = \frac{1}{N+1}(1 + 0 + 1 + 0 + \cdots) \to \frac12. \] 
In some sense, the Ces\`aro means allow us to make sense of convergence of 
typically non-convergent series. 

\begin{lemma}{lemma:5.7}
    Let $(a_n)_{n=0}^\infty$ be a sequence. Let $(S_N)_{N=0}^\infty$ denote the 
    partial sums and $(\sigma_N)_{N=0}^\infty$ denote the Ces\`aro means. 
    If $\lim_{N\to\infty} S_N = L$, then $\lim_{N\to\infty} \sigma_N = L$. 
\end{lemma}
\begin{pf}
    Let $\eps > 0$. Since $S_N \to L$, there exists $N \in \N$ such that 
    $|S_k - L| < \eps$ for all $k \geq N$. Let $M = \sup_{k\in\N} |S_k| < \infty$. 
    Then we have 
    \begin{align*}
        |\sigma_n - L| &= \left| \frac{1}{n+1} \sum_{k=0}^n (S_k - L) \right| \\
        &\leq \frac{1}{n+1} \sum_{k=0}^n |S_k - L| \\ 
        &= \frac{1}{n+1} \sum_{k=0}^{N-1} |S_k - L| + \frac{1}{n+1} \sum_{k=N}^n |S_k - L| \\ 
        &\leq \frac{1}{n+1} N(M + |L|) + \frac{1}{n+1} (n - N + 1)\eps . 
    \end{align*}
    Taking $n \to \infty$, we obtain $|\sigma_n - L| \to 0 + \eps = \eps$. 
\end{pf}

\begin{defn}{defn:5.8}
    Let $f \in L^1(\T)$, and suppose that its Fourier series is $\sum_{n\in\Z}
    \hat f(n) e^{in\theta}$. The {\bf $N$-th Ces\`aro mean} of the Fourier 
    series is 
    \[ \sigma_N f(\theta) = \frac{1}{N+1} \sum_{k=0}^N S_k f(\theta). \] 
\end{defn}

Observe that 
\begin{align*}
    \sigma_N f(\theta) 
    &= \frac{1}{N+1} \sum_{k=0}^N \sum_{n=-k}^k \hat f(n) e^{in\theta} \\ 
    &= \frac{1}{N+1} \sum_{n=-N}^N \sum_{k=|n|}^N \hat f(n) e^{in\theta} \\ 
    &= \frac{1}{N+1} \sum_{n=-N}^N (N + 1 - |n|) \hat f(n) e^{in\theta} \\ 
    &= \sum_{n=-N}^N \left( 1 - \frac{|n|}{N+1} \right) \hat f(n) e^{in\theta}. 
\end{align*}
In particular, by looking at the Ces\`aro means $\sigma_N f$ instead of 
the partial sums $S_N f$, we are only slightly adjusting the Fourier 
coefficients via the map 
\[ \hat f(n) \mapsto \left( 1 + \frac{|n|}{N+1} \right) \hat f(n), \] 
where $|n| \leq N$. 

\begin{theo}[Fej\'er's Theorem]{theo:5.9}
    Let $f \in L^1(\T)$. 
    \begin{enumerate}[(1)]
        \item We have $\|\sigma_N f - f\|_1 \to 0$ as $N \to \infty$. (The same 
        is true if we replace $p = 1$ with $p = 2$.)
        \item If we know furthermore that $f \in C(\T)$, then $\|\sigma_N f - f\|_\infty
        \to 0$ as $N \to \infty$. 
    \end{enumerate}
\end{theo}

We should contrast part (1) of Fej\'er's Theorem with Kolmogorov's counterexample 
(Theorem~\ref{theo:4.11}). The behaviour seems to be much better with Ces\`aro means 
than with partial sums. Before we prove Fej\'er's Theorem, we will look at some 
powerful consequences of it. 

\begin{cor}{cor:5.10}
    Any function $f \in L^1(\T)$ is uniquely determined by its Fourier transform 
    ${\cal F}f = (\hat f(n))_{n\in\Z} \in c_0(\Z)$. More precisely, the 
    map ${\cal F} : L^1(\T) \to c_0(\Z)$ is an injective algebra homomorphism. 
\end{cor}
\begin{pf}
    We will show that the kernel of ${\cal F} : L^1(\T) \to c_0(\Z)$ is the zero 
    function, which is equivalent to the fact that ${\cal F}$ is injective. 
    Indeed, note that for $f \in L^1(\T)$, we have $\hat f(n) = 0$ for all $n \in \Z$ 
    if and only if 
    \[ \left( 1 - \frac{|n|}{N+1} \right) \hat f(n) = 0 \] 
    for all $n \in \Z$ and $N \in \N$. This is equivalent to the fact that 
    $\sigma_N f = 0$, and we have $f = \lim_{N\to\infty} \sigma_N f = 0$ by 
    Fej\'er's Theorem. 
\end{pf}

Another consequence is a more constructive proof of the Stone-Weierstrass Theorem. 

\begin{cor}[Stone-Weierstrass]{cor:5.11}
    The trigonometric polynomials $\Pol(\T)$ are uniformly dense in $C(\T)$. 
\end{cor}
\begin{pf}
    Let $f \in C(\T)$. By Fej\'er's Theorem, we have $\|f - \sigma_N f\|_\infty \to 0$, 
    where $\sigma_N f \in \Pol(\T)$. 
\end{pf}

To prove Fej\'er's Theorem, we first introduce the Fej\'er kernel. 

\begin{defn}{defn:5.12}
    The Fej\'er kernel $(K_N)_{N=0}^\infty \subseteq \Pol(\T) \subseteq C(\T)$ is 
    defined to be 
    \[ K_N(t) = \sum_{n=-N}^N \left( 1 - \frac{|n|}{N+1} \right) e^{int}. \] 
\end{defn}

Now, given $f \in L^1(\T)$, we can use the equation we derived above to see that 
\begin{align*}
    \sigma_N f (\theta) 
    &= \sum_{n=-N}^N \left( 1 - \frac{|n|}{N+1} \right) \hat f(n) e^{in\theta} \\ 
    &= \sum_{n=-N}^N \left( 1 - \frac{|n|}{N+1} \right) \frac{1}{2\pi} 
    \int_{-\pi}^\pi f(t) e^{in(\theta - t)}\dd t \\ 
    &= \frac{1}{2\pi} \int_{-\pi}^\pi f(t) \sum_{n=-N}^N \left( 1 - 
    \frac{|n|}{N+1} \right) e^{in(\theta - t)}\dd t \\ 
    &= \frac{1}{2\pi} \int_{-\pi}^\pi f(t) K_N(\theta - t) = (f \star K_N)(\theta). 
\end{align*}
If $(K_N)_{N=0}^\infty$ is a summability kernel, then Fej\'er's Theorem will follow 
immediately from Theorem~\ref{theo:5.5}. This is indeed the case. 
We first introduce an auxiliary kernel, known as the Dirichlet kernel, which 
we will need later on. 

\begin{defn}{defn:5.13}
    The {\bf Dirichlet kernel} $(D_N)_{N=0}^\infty$ is defined to be 
    \[ D_N(\theta) = \sum_{k=-N}^N e^{ik\theta} = 1 + \sum_{k=1}^N 2\cos(k\theta). \] 
\end{defn}

Using the identity $\sin(A+B) - \sin(A-B) = 2\sin(B)\cos(A)$, we find that 
\begin{align*}
    \sin(\theta/2) D_N(\theta) 
    &= \sin(\theta/2) + \sum_{k=1}^N 2\sin(\theta/2) \cos(k\theta) \\ 
    &= \sin(\theta/2) + \sum_{k=1}^N \sin((k+1/2)\theta) - \sin((k-1/2)\theta) \\ 
    &= \sin((N + 1/2)\theta).     
\end{align*}
Thus, we obtain the formula 
\[ D_N(\theta) = \frac{\sin((N + 1/2)\theta)}{\sin(\theta/2)}. \] 
Moreover, it can be seen with a direct computation that 
\[ S_N f(\theta) = \frac{1}{2\pi} \int_{-\pi}^\pi f(t) D_N(\theta - t)\dd t 
= (f \star D_N)(\theta). \] 
Using this fact, we obtain 
\[ f \star K_N = \sigma_N f = \frac{1}{N+1} \sum_{k=0}^N S_k f 
= \frac{1}{N+1} \sum_{k=0}^N f \star D_k. \] 
It follows from this that 
\[ K_N = \frac{1}{N+1} \sum_{k=0}^N D_k. \] 
In other words, $K_N$ is the $N$-th Ces\`aro mean of the $D_k$. Finally, 
we can use this fact to show that the Fej\'er kernel is indeed a 
summability kernel. 

\begin{prop}{prop:5.14} 
    The Fej\'er kernel $(K_N)_{N=0}^\infty$ is a positive and symmetric summability kernel. 
\end{prop}
\begin{pf}
    We first prove a nice formula for $(K_N)_{N=0}^\infty$. 

    {\sc Claim.} For all $N \geq 0$, we have 
    \[ K_N(\theta) = \frac{\sin^2((N+1)\theta/2)}{(N+1)\sin^2(\theta/2)}. \] 
    {\sc Proof of Claim.} Recall that 
    \[ K_N(\theta) = \frac{1}{N+1} \sum_{k=0}^N D_k(\theta) 
    = \frac{1}{N+1} \sum_{k=0}^N \frac{\sin((k+1/2)\theta)}{\sin(\theta/2)}. \] 
    Multiplying both sides by $(N+1) \sin^2(\theta/2)$ and using the  
    identity $\sin(A) \sin(B) = \frac12(\cos(B-A) - \cos(B+A))$, we get 
    \begin{align*}
        (N+1) \sin^2(\theta/2) K_N(\theta) 
        &= \sum_{k=0}^N \sin(\theta/2) \sin((k+1/2)\theta) \\ 
        &= \sum_{k=0}^N \frac12 \cos(k\theta) - \cos((k+1)\theta) \\ 
        &= \frac12(1 - \cos(N+1)\theta) = \sin^2((N+1)\theta). 
    \end{align*}
    Rearranging this gives us the result. \hfill $\blacksquare$
    
    With this identity, we can now verify that $(K_N)_{N=0}^\infty$ is a 
    positive and symmetric summability kernel. Just by looking at the 
    formula, we have $K_N(\theta) \geq 0$ and $K_N(\theta) = K_N(-\theta)$. 
    For all $N \geq 0$, it can be shown through a computation that 
    \[ \frac{1}{2\pi} \int_{-\pi}^\pi K_N(t)\dd t = 1. \] 
    Pick $\delta > 0$, and observe that 
    \[ \sup_{\delta \leq |\theta| \leq \pi} K_N(\theta) 
    = \sup_{\delta \leq |\theta| \leq \pi} \frac{1}{N+1} \cdot 
    \frac{\sin^2((N+1)\theta/2)}{\sin^2(\theta/2)} \leq 
    \frac{1}{N+1} \cdot \frac{1}{\sin^2(\delta/2)} \to 0 \] 
    as $N \to \infty$. That is, we have $K_N \to 0$ uniformly away from 
    the interval $[-\delta, \delta]$, which implies that 
    \[ \lim_{N\to\infty} \int_{|\theta|\geq\delta} K_N(\theta)\dd\theta = 0. \] 
    We conclude that $(K_N)_{N=0}^\infty$ is a summability kernel. 
\end{pf}

Can we play the same game with the Dirichlet kernel $(D_N)_{N=0}^\infty$? 
Using the fact that $S_N = f \star D_N$ and applying Theorem~\ref{theo:5.5} would 
be very nice for us. Unfortunately, the Dirichlet kernel $(D_N)_{N=0}^\infty$ is 
\emph{not} a summability kernel. This can be seen from the following theorem, 
which states that $D_N$ grows like the logarithm. 

\begin{prop}{prop:5.15}
    \begin{enumerate}[(1)]
        \item $D_N$ is symmetric. 
        \item We have $\|D_N\|_\infty = D_N(0) = 2N+1$. 
        \item For all $N \geq 0$, we have 
        \[ \frac{1}{2\pi} \int_{-\pi}^\pi D_N(t)\dd t = 1. \] 
        \item For all $N \geq 0$, we have 
        \[ \|D_N\|_1 \geq \frac{4}{\pi^2} \log(2N+2). \] 
        (The values $L_N = \|D_N\|_1$ are known as the {\bf Lebesgue numbers}; 
        it is known that $L_N = O(\log N)$.)
    \end{enumerate}
\end{prop}
\begin{pf}
    Property (1) is clear by looking at any of the formulas. Property (2) can be 
    seen by applying the triangle inequality to the formula 
    \[ D_N(\theta) = 1 + \sum_{k=1}^N 2\cos(k\theta), \] 
    and property (3) can be seen by a direct computation. 

    To prove (4), note that $D_N$ is symmetric and $\lvert \sin(t/2) \rvert \leq t/2$, 
    so we obtain 
    \[ \|D_N\|_1 = \frac1\pi \int_0^\pi |D_N(t)|\dd t 
    = \frac1\pi \int_0^\pi \frac{\lvert\sin((N+1/2)t)\rvert}{\lvert\sin(t/2)\rvert}\dd t 
    \geq \frac{2}{\pi} \int_0^\pi \frac{\lvert\sin((N+1/2)t)\rvert}{t}\dd t. \] 
    Next, we break up this integral into $2N+1$ different ones over the intervals 
    $[\frac{k\pi}{2N+1}, \frac{(k+1)\pi}{2N+1}]$. We also note that 
    $\frac1t \geq \frac{2N+1}{(k+1)\pi}$ on each of these intervals, which implies that 
    \begin{align*}
        \|D_N\|_1 
        &\geq \frac{2}{\pi} \int_0^\pi \frac{\lvert\sin((N+1/2)t)\rvert}{t}\dd t \\ 
        &= \frac{2}{\pi} \sum_{k=0}^{2N} \int_{\frac{k\pi}{2N+1}}^{\frac{(k+1)\pi}{2N+1}} 
        \frac{\lvert\sin((N+1/2)t)\rvert}{t}\dd t \\ 
        &\geq \frac{2}{\pi} \sum_{k=0}^{2N} \frac{2N+1}{(k+1)\pi} 
        \int_{\frac{k\pi}{2N+1}}^{\frac{(k+1)\pi}{2N+1}} \lvert\sin((N+1/2)t)\rvert\dd t. 
    \end{align*}
    The period of each of the sine functions in the integrand is equal to 
    $\frac{4\pi}{2N+1}$. This means that all of these integrals are equal to 
    \[ \int_0^{\frac{\pi}{2N+1}} \sin((N+1/2)t)\dd t = \frac{2}{2N+1}. \] 
    It follows that 
    \begin{align*}
        \|D_N\|_1 \geq \frac{2}{\pi} \sum_{k=0}^{2N} \frac{2N+1}{(k+1)\pi} \cdot \frac{2}{2N+1} 
        = \frac{4}{\pi^2} \sum_{k=0}^{2N} \frac{1}{k+1} 
        \geq \frac{4}{\pi^2} \int_1^{2N+2} \frac1x\dd x 
        = \frac{4}{\pi^2} \log(2N+2), 
    \end{align*}
    where the final inequality was from applying the integral test. 
\end{pf}

Let $f \in C(\T)$. We consider the convergence of the Fourier series of $f$ at $0$. 
We have 
\[ S_N f(0) = (f \star D_N)(0) = \frac{1}{2\pi} \int_{-\pi}^\pi f(t) D_N(-t)\dd t 
= \frac{1}{2\pi} \int_{-\pi}^\pi f(t) D_N(t)\dd t, \] 
where the last equality is because $D_N$ is even. This implies that 
\[ |S_N f(0)| = \left| \frac{1}{2\pi} \int_{-\pi}^\pi f(t) D_N(t)\dd t \right| 
\leq \|f\|_\infty \|D_N\|_1 = \|f\|_\infty L_N. \] 
In particular, the map $\varphi_N$ sending $f$ to $S_N f(0)$ is a bounded 
linear functional for all $N \in \N$ with 
\[ \|\varphi_N\|_{C(\T)^*} \leq L_N \sim \log N. \] 
We claim that $L_N = \|\varphi_N\|$. To see this, let $g_N = \text{sign}(D_N) \in L^\infty(\T)$. 
Note that $g_N \notin C(\T)$, but we have 
\[ S_N g_N(0) = \frac{1}{2\pi} \int_{-\pi}^\pi \text{sign}(D_N(t))D_N(t) \dd t
= \frac{1}{2\pi} \int_{-\pi}^\pi |D_N(t)|\dd t = \|D_N\|_1. \] 
Fix $\eps > 0$. We can find a continuous approximation $g \in C(\T)$ such that 
$\|g\|_\infty \leq 1$ and $g = \text{sign}(D_N)$ except on a set of measure 
$< \eps/\|D_N\|_1$ using a similar construction to the one in Theorem~\ref{theo:3.20}.
Using the reverse triangle inequality, we have 
\[ |\varphi_N(g)| = |S_N g(0)| = |S_N g_N(0) + S_N (g - g_N)(0)| 
\geq \|D_N\|_1 - C\eps. \] 
This proves the claim. In particular, given any $N \in \N$, we can always find a 
function $g_N \in C(\T)$ such that $\|g_N\|_\infty = 1$ and $S_N g_N(0) \sim \log N$.
