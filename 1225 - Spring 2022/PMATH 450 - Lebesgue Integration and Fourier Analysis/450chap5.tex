\section{Convolution and Summability Kernels} \label{sec:5}

\subsection{Convolution} \label{subsec:5.1}
Recall that the general problem of Fourier analysis is to try 
and recover a function $f$ from $L^1(\T)$, $L^2(\T)$, or $C(\T)$ 
given its Fourier transform ${\cal F}f = (\hat f(n))_{n\in\Z}$. 
We already know the following: 
\begin{enumerate}[(1)]
    \item If $f \in L^2(\T)$, then we can recover $f$ as 
    $\lim_{N\to\infty} S_N f$ in the sense of $L^2$-convergence. 
    \item If $f$ is locally Lipschitz at a point $\theta$ (that is, 
    $f$ is ``regular''), then $f(\theta) = \lim_{N\to\infty} S_N f(\theta)$. 
    \item If $f \in C^1(\T)$, then $\|f - S_N f\|_\infty \to 0$. 
\end{enumerate}
But what if we merely assume that $f \in C(\T)$ or $f \in L^1(\T)$? 
In the case of $L^1(\T)$, it might not seem possible to recover 
$f$ from ${\cal F}f$ in light of Kolmogorov's counterexample 
(Theorem~\ref{theo:4.11}). We will see that it actually is possible.
We first take a look at the convolution of two functions. 

\begin{prop}{prop:5.1}
    Let $f \in C(\T)$ and $g \in L^1(\T)$. Define a function 
    \[ h(\theta) = \frac{1}{2\pi} \int_{-\pi}^\pi f(\theta - t) g(t)\dd t. \] 
    Note that this is well-defined because the function 
    $t \mapsto f(\theta - t) g(t)$ is in $L^1(\T)$. We have 
    \begin{enumerate}[(1)]
        \item $h \in C(\T)$ and $\|h\|_\infty \leq \|f\|_\infty \|g\|_1$; 
        \item $\|h\|_1 \leq \|f\|_1 \|g\|_1$ and $\|h\|_2 \leq \|f\|_2 \|g\|_1$. 
    \end{enumerate}
    We write $h = f \star g$, and call it the {\bf convolution} of $f$ and $g$. 
\end{prop}
\begin{pf}~
    \begin{enumerate}[(1)]
        \item Since $f \in C(\T)$ and $\T$ is compact, we know that $f$ 
        is uniformly continuous. So for all $\eps > 0$, there exists 
        $\delta > 0$ such that $|s - t| < \delta$ implies 
        $|f(s) - f(t)| < \eps$. Suppose that $\theta, \theta_0 \in \T$ with 
        $|\theta - \theta_0| < \delta$. We get 
        \begin{align*}
            |h(\theta) - h(\theta_0)| 
            &= \frac{1}{2\pi} \left| \int_{-\pi}^\pi (f(\theta - t) - f(\theta_0 - t)) g(t)\dd t \right| \\ 
            &\leq \frac{1}{2\pi} \int_{-\pi}^\pi |f(\theta - t) - f(\theta_0 - t)| |g(t)|\dd t \\ 
            &< \frac{1}{2\pi} \int_{-\pi}^\pi \eps |g(t)|\dd t \leq \eps\|g\|_1, 
        \end{align*}
        where the second last inequality is because $|\theta - t - (\theta_0 - t)| < \delta$. 
        So $h \in C(\T)$, and it is clear that $\|h\|_\infty \leq 
        \|f\|_\infty \|g\|_1$ by a similar computation to above. 

        \item First suppose that $g \in C(\T)$. Then we have 
        \begin{align*}
            \|h\|_1 &= \frac{1}{2\pi} \int_{-\pi}^\pi \left| \frac{1}{2\pi} \int_{-\pi}^\pi f(\theta - t) g(t)\dd t \right|\textrm{d}\theta \\ 
            &\leq \frac{1}{(2\pi)^2} \int_{-\pi}^\pi \int_{-\pi}^\pi |f(\theta - t)| |g(t)|\dd t \dd\theta \\ 
            &= \frac{1}{(2\pi)^2} \int_{-\pi}^\pi |g(t)| \left( \int_{-\pi}^\pi |f(\theta - t)|\dd\theta \right)\textrm{d}t
            = \|f\|_1 \|g\|_1,
        \end{align*}
        where the second last equality was obtained by applying Fubini's theorem
        to interchange $\textrm{d}t$ and $\textrm{d}\theta$, 
        noting that this is a Riemann integral since $f, g \in C(\T)$. 
        Similarly, we have 
        \begin{align*}
            \|h\|_2^2 
            &= \frac{1}{2\pi} \int_{-\pi}^\pi \left| \frac{1}{2\pi} \int_{-\pi}^\pi f(\theta - t) g(t)\dd t \right|^2 \textrm{d}\theta \\ 
            &\leq \frac{1}{2\pi} \int_{-\pi}^\pi \left( \frac{1}{2\pi} \int_{-\pi}^\pi |f(\theta - t)| |g(t)|\dd t \right)^{\!2} \textrm{d}\theta \\ 
            &\leq \frac{1}{2\pi} \int_{-\pi}^\pi \left( \frac{1}{2\pi} \int_{-\pi}^\pi |f(\theta - t)|^2 |g(t)|\dd t \right) \left( \frac{1}{2\pi} \int_{-\pi}^\pi |g(t)|\dd t \right)\textrm{d}\theta \\ 
            &= \|g\|_1 \cdot \frac{1}{2\pi} \int_{-\pi}^\pi |f(\theta - t)|^2 |g(t)|\dd t \dd \theta = \|g\|_1 \|f\|_2^2 \|g\|_1, 
        \end{align*}
        where the last equality was due to Fubini, and the second inequality 
        was from applying Cauchy-Schwarz to $|f(\theta - t)||g(t)| 
        = |f(\theta - t)||g(t)|^{1/2} |g(t)|^{1/2}$. 

        Suppose now that $g \in L^1(\T)$. Take a sequence $(g_n)_{n=1}^\infty 
        \subseteq C(\T)$ such that $g_n \to g$ in $\|\cdot\|_1$. 
        Then we know for $p \in \{1, 2\}$ that 
        $\|f \star g_n\|_p \leq \|f\|_p \|g_n\|_1$.
        We claim that $(f \star g_n)_{n=1}^\infty$ is Cauchy in $\|\cdot\|_p$ 
        for $p \in \{1, 2\}$. Indeed, we see that 
        \[ \|f \star g_n - f \star g_m\|_p = \|f \star (g_n - g_m)\|_p 
        \leq \|f\|_p \|g_n - g_m\|_1 \to 0 \] 
        since $g_n \to g$ in $\|\cdot\|_1$, and thus $\|g_n - g_m\|_1 \to 0$. 
        This means that $h = f \star g = \lim_{n\to\infty} f \star g_n$, 
        since the sequence $(f \star g_n)_{n=1}^\infty$ being Cauchy 
        implies that the limit exists, and 
        \[ \|f \star g - f \star g_n\|_p = \|f \star (g - g_n)\|_p 
        \leq \|f\|_p \|g - g_n\|_1 \to 0. \] 
        It follows that 
        $\|f \star g\|_p = \lim_{n\to\infty} \|f \star g_n\|_p 
        \leq \lim_{n\to\infty} \|f\|_p \|g_n\|_1 = \|f\|_p \|g\|_1$. \qedhere  
    \end{enumerate}
\end{pf}

\begin{remark}{remark:5.2}
    If $f, g \in C(\T)$, then $\star : C(\T) \times C(\T) \to C(\T)$ 
    turns $C(\T)$ into a {\bf commutative algebra} over $\C$. That is, 
    for all $f, g, h \in C(\T)$ and $\alpha \in \C$, we have 
    \begin{itemize}
        \item $f \star g = g \star f$; 
        \item $f \star (g + h) = f \star g + f \star h$; 
        \item $(f \star g) \star h = f \star (g \star h)$; and 
        \item $\alpha(f \star g) = (\alpha f) \star g$. 
    \end{itemize}
    It can be seen that $f \star g = g \star f$ by making a substitution 
    $s = \theta - t$. 
\end{remark}

\begin{theo}{theo:5.3}
    The convolution product $\star$ extends continuously to a multiplication 
    $\star : L^1(\T) \times L^1(\T) \to L^1(\T)$ given by $(f, g) \mapsto 
    g \star g = g \star f$, and it satisfies $\|f \star g\|_1 \leq \|f\|_1 \|g\|_1$. 
    Moreover, for $f, g \in L^1(\T)$, the function $t \mapsto f(\theta - t) g(t)$
    is in $L^1(\T)$ for almost every $\theta \in [-\pi, \pi]$, and for such 
    $\theta$, we have 
    \[ (f \star g)(\theta) = \frac{1}{2\pi} \int_{-\pi}^\pi f(\theta - t)g(t)\dd t. \] 
\end{theo}
\begin{pf}
    It turns out that the integrability of the function $t \mapsto f(\theta - t) g(t)$ 
    is highly non-trivial and requires a measure theoretic version of 
    Fubini, which is typically seen in PMATH 451. We will therefore 
    content ourselves to abstractly define $f \star g$ for 
    $f, g \in L^1(\T)$ via a limiting process. 

    Let $(f_n)_{n=1}^\infty, (g_n)_{n=1}^\infty \subseteq C(\T)$ 
    be sequences such that $f_n \to f$ and $g_n \to g$ in $\|\cdot\|_1$. 
    Note that $(f_n \star g_n)_{n=1}^\infty$ is Cauchy in $\|\cdot\|_1$. 
    Indeed, we see that  
    \begin{align*}
        \|f_n \star g_n - f_m \star g_m\|_1 
        &= \|(f_n - f_m) \star g_n + f_m \star (g_n - g_m)\|_1 \\ 
        &\leq \|(f_n - f_m) \star g_n\|_1 + \|f_m \star (g_n - g_m)\|_1 \\ 
        &\leq \|f_n - f_m\|_1 \|g_n\|_1 + \|f_m\|_1 \|g_n - g_m\|_1 \to 0 
    \end{align*}
    as $m, n \to \infty$ because $(f_n)_{n=1}^\infty$ and $(g_n)_{n=1}^\infty$ 
    are Cauchy. We define $f \star g = \lim_{n\to\infty} 
    f_n \star g_n \in L^1(\T)$. To see this is well-defined, we show that 
    $f \star g$ is independent of the choice of $(f_n)_{n=1}^\infty$ 
    and $(g_n)_{n=1}^\infty$. Suppose that $(\tilde f_n)_{n=1}^\infty$ 
    and $(\tilde g_n)_{n=1}^\infty$ also satisfy $\tilde f_n \to f$ and 
    $\tilde g_n \to g$ in $\|\cdot\|_1$. Then we obtain 
    \[ \|f_n \star g_n - \tilde f_n \star \tilde g_n\|_1 
    \leq \|f_n - \tilde f_n\|_1 \|g_n\|_1 + \|\tilde f_n\|_1 
    \|g_n - \tilde g_n\|_1 \to 0, \] 
    and thus $f \star g = \lim_{n\to\infty} f_n \star g_n = 
    \lim_{n\to\infty} \tilde f_n \star \tilde g_n$. 
\end{pf}

\subsection{Properties of Summability Kernels} \label{subsec:5.2}
From Theorem~\ref{theo:5.3}, we see that $L^1(\T)$ is an algebra under convolution. 
Note that $L^1(\T)$ is not unital; that is, there does not exist $k \in L^1(\T)$ 
such that $f \star k = k \star f = f$ for all $f \in L^1(\T)$. Indeed, assume 
towards such a function $k \in L^1(\T)$ existed. Then for all $f \in L^1(\T)$, 
we would have 
\[ f(\theta) = (f \star k)(\theta) = \frac{1}{2\pi} \int_{-\pi}^\pi f(\theta - t)
k(t)\dd t. \] 
Informally, this would force $k$ to be the Dirac delta function at $0$, 
which is given by 
\[ \delta(t) = \begin{cases}
    \infty, & \text{if } t = 0, \\ 
    0, & \text{if } t \neq 0.
\end{cases} \] 
It is clear that this is not integrable, which is a contradiction.

But $L^1(\T)$ has ``approximate'' units with respect to convolution; they are 
called summability kernels. 

\begin{defn}{defn:5.4}
    A sequence $(k_n)_{n=1}^\infty \subseteq L^1(\T)$ is called a {\bf summability
    kernel} if it satisfies the following properties: 
    \begin{enumerate}[(1)]
        \item For all $n \in \N$, we have 
        \[ \frac{1}{2\pi} \int_{-\pi}^\pi k_n(\theta)\dd\theta = 1. \] 
        \item The sequence $(k_n)_{n=1}^\infty$ is uniformly bounded; that is, 
        $\sup_{n\in\N} \|k_n\|_1 = C < \infty$ for some constant $C$. 
        \item For all $\delta > 0$, we have 
        \[ \lim_{n\to\infty} \int_{|\theta|\geq\delta} |k_n(\theta)|\dd\theta = 0. \]
    \end{enumerate}
    We say that a summability kernel is {\bf even} or {\bf symmetric} if $k_n(\theta) 
    = k_n(-\theta)$ for almost every $\theta \in [-\pi, \pi]$, and {\bf positive} 
    if $k_n \geq 0$ for all $n \in \N$. 
\end{defn}

Pictorially, there will be very large values near $0$, but small values away from $0$. 
Note that when $(k_n)_{n=1}^\infty$ is a positive summability kernel, then 
condition (1) implies condition (2) with the constant $C = 1$. 
An example of a symmetric and positive summability kernel is $(k_n)_{n=1}^\infty$ given by 
$k_n = \pi n\chi_{[-1/n, 1/n]}$. We leave it as an exercise to verify this. 

The following theorem tells us that summability kernels play the role of an approximate 
unit with respect to convolution. This will allow us to use summability 
kernels to try and answer our main question, which is to recover a function 
from its Fourier transform. 

\begin{theo}{theo:5.5}
    Let $(k_n)_{n=1}^\infty$ be a summability kernel. 
    \begin{enumerate}[(1)]
        \item If $f \in C(\T)$, then as $n \to \infty$, we have 
        \[ \|f - f \star k_n\|_\infty \to 0. \] 
        \item Let $p \in \{1, 2\}$ and $f \in L^p(\T)$. Then as $n \to \infty$, we have 
        \[ \|f - f \star k_n\|_p \to 0. \] 
    \end{enumerate}
\end{theo}
\begin{pf}~
    \begin{enumerate}[(1)]
        \item Let $f \in C(\T)$ and $\eps > 0$. By the uniform continuity of $f$, there exists 
        $\delta > 0$ such that $|f(s) - f(t)| < \eps$ whenever $|s - t| < \delta$. Then we have 
        \begin{align*}
            |f(\theta) - (f \star k_n)(\theta)| 
            &= \frac{1}{2\pi} \left| \int_{-\pi}^\pi (f(\theta) - f(\theta - t)) k_n(t)\dd t \right| \\ 
            &\leq \frac{1}{2\pi} \int_{-\pi}^\pi |f(\theta) - f(\theta - t)| |k_n(t)|\dd t \\ 
            &= \frac{1}{2\pi} \int_{|t|\leq\delta} |f(\theta) - f(\theta - t)||k_n(t)|\dd t 
            + \frac{1}{2\pi} \int_{|t|>\delta} |f(\theta) - f(\theta - t)||k_n(t)|\dd t \\ 
            &< \eps\|k_n\|_1 + 2\|f\|_\infty \int_{|t|>\delta} |k_n(t)|\dd t, 
        \end{align*}
        where the first equality is due to property (1) of being a summability kernel, which 
        states that 
        \[ \frac{1}{2\pi} \int_{-\pi}^\pi k_n(t)\dd t = 1. \] 
        Taking $n \to \infty$, property (3) of being a summability kernel tells us that 
        \[ \int_{|t|>\delta} |k_n(t)|\dd t \to 0, \] 
        and thus $|f(\theta) - (f \star k_n)(\theta)| \leq C\eps$ uniformly in $\theta$. 
        Hence, we have $\|f - f \star k_n\|_\infty \to 0$. 

        \item Let $p \in \{1, 2\}$, and let $f \in L^p(\T)$. Let $\eps > 0$. Then by 
        Corollary~\ref{cor:3.21}, we can find $g \in C(\T)$ such that $\|f - g\|_p < \eps$. 
        Choose sufficiently large $n \in \N$ such that 
        \[ \|g - g \star k_n\|_\infty < \eps, \] 
        which exists since $\|g - g \star k_n\|_\infty \to 0$ by part (1). Then we obtain 
        \begin{align*}
            \|f - f \star k_n\|_p 
            &\leq \|f - g\|_p + \|g - g \star k_n\|_p + \|g \star k_n - f \star k_n\|_p \\ 
            &< \eps + \|g - g \star k_n\|_\infty + \|(g - f) \star k_n\|_p \\ 
            &< \eps + \eps + \|g - f\|_p \|k_n\|_1 \\ 
            &< 2\eps + C\eps \\ 
            &= (C + 2)\eps. \qedhere 
        \end{align*}
    \end{enumerate}
\end{pf}

\subsection{The Fej\'er Kernel and Dirichlet Kernel} \label{subsec:5.3}
We now look at some applications of summability kernels. First, we consider 
Ces\`aro summation of Fourier series. Given a series $\sum_{n=0}^\infty a_n$
where $(a_n)_{n=0}^\infty \subseteq \C$, we know that it may or may not converge. 
That is, the $N$-th partial sum $S_N = \sum_{n=0}^N a_n$ may or may not converge 
as a sequence in $\C$. A simple example of this is $a_n = (-1)^n$, where it is 
clear that $\sum_{n=0}^\infty (-1)^n$ does not converge. 

\begin{defn}{defn:5.6}
    Let $\sum_{n=0}^\infty a_n$ be a series with $(a_n)_{n=0}^\infty \subseteq \C$, 
    and suppose that it has partial sums $(S_N)_{N=0}^\infty$. The 
    {\bf $N$-th Cesa\`ro sum mean} of the series is given by 
    \[ \sigma_N = \frac{1}{N+1} \sum_{k=0}^N S_k. \] 
\end{defn}

How are the convergence properties of the partial sums $(S_N)_{N=0}^\infty$ 
and the Ces\`aro means $(\sigma_N)_{N=0}^\infty$ related? It turns out 
that if the partial sums converge, then the Ces\`aro means converge to the 
same limit. But the converse fails, because taking our example $a_n = (-1)^n$ 
above, we find that 
\[ \sigma_N = \frac{1}{N+1}(1 + 0 + 1 + 0 + \cdots) \to \frac12. \] 
In some sense, the Ces\`aro means allow us to make sense of convergence of 
typically non-convergent series. 

\begin{lemma}{lemma:5.7}
    Let $(a_n)_{n=0}^\infty$ be a sequence. Let $(S_N)_{N=0}^\infty$ denote the 
    partial sums and $(\sigma_N)_{N=0}^\infty$ denote the Ces\`aro means. 
    If $\lim_{N\to\infty} S_N = L$, then $\lim_{N\to\infty} \sigma_N = L$. 
\end{lemma}
\begin{pf}
    Let $\eps > 0$. Since $S_N \to L$, there exists $N \in \N$ such that 
    $|S_k - L| < \eps$ for all $k \geq N$. Let $M = \sup_{k\in\N} |S_k| < \infty$. 
    Then we have 
    \begin{align*}
        |\sigma_n - L| &= \left| \frac{1}{n+1} \sum_{k=0}^n (S_k - L) \right| \\
        &\leq \frac{1}{n+1} \sum_{k=0}^n |S_k - L| \\ 
        &= \frac{1}{n+1} \sum_{k=0}^{N-1} |S_k - L| + \frac{1}{n+1} \sum_{k=N}^n |S_k - L| \\ 
        &\leq \frac{1}{n+1} N(M + |L|) + \frac{1}{n+1} (n - N + 1)\eps . 
    \end{align*}
    Taking $n \to \infty$, we obtain $|\sigma_n - L| \to 0 + \eps = \eps$. 
\end{pf}

\begin{defn}{defn:5.8}
    Let $f \in L^1(\T)$, and suppose that its Fourier series is $\sum_{n\in\Z}
    \hat f(n) e^{in\theta}$. The {\bf $N$-th Ces\`aro mean} of the Fourier 
    series is 
    \[ \sigma_N f(\theta) = \frac{1}{N+1} \sum_{k=0}^N S_k f(\theta). \] 
\end{defn}

Observe that 
\begin{align*}
    \sigma_N f(\theta) 
    &= \frac{1}{N+1} \sum_{k=0}^N \sum_{n=-k}^k \hat f(n) e^{in\theta} \\ 
    &= \frac{1}{N+1} \sum_{n=-N}^N \sum_{k=|n|}^N \hat f(n) e^{in\theta} \\ 
    &= \frac{1}{N+1} \sum_{n=-N}^N (N + 1 - |n|) \hat f(n) e^{in\theta} \\ 
    &= \sum_{n=-N}^N \left( 1 - \frac{|n|}{N+1} \right) \hat f(n) e^{in\theta}. 
\end{align*}
In particular, by looking at the Ces\`aro means $\sigma_N f$ instead of 
the partial sums $S_N f$, we are only slightly adjusting the Fourier 
coefficients via the map 
\[ \hat f(n) \mapsto \left( 1 + \frac{|n|}{N+1} \right) \hat f(n), \] 
where $|n| \leq N$. 

\begin{theo}[Fej\'er's Theorem]{theo:5.9}
    Let $f \in L^1(\T)$. 
    \begin{enumerate}[(1)]
        \item We have $\|\sigma_N f - f\|_1 \to 0$ as $N \to \infty$. (The same 
        is true if we replace $p = 1$ with $p = 2$.)
        \item If we know furthermore that $f \in C(\T)$, then $\|\sigma_N f - f\|_\infty
        \to 0$ as $N \to \infty$. 
    \end{enumerate}
\end{theo}

We should contrast part (1) of Fej\'er's Theorem with Kolmogorov's counterexample 
(Theorem~\ref{theo:4.11}). The behaviour seems to be much better with Ces\`aro means 
than with partial sums, because this tells us that Ces\`aro means always 
converge with respect to $\|\cdot\|_1$ even when this fails for the partial 
sums. The case where $p = 2$ is less interesting because we already have this 
for partial sums by Theorem~\ref{theo:4.7}, and we know that the limit of the 
Ces\`aro means coincides with the limit of the partial sums by Lemma~\ref{lemma:5.7}.

Before we prove Fej\'er's Theorem, we will look at some powerful consequences of it.

\begin{cor}{cor:5.10}
    Any function $f \in L^1(\T)$ is uniquely determined by its Fourier transform 
    ${\cal F}f = (\hat f(n))_{n\in\Z} \in c_0(\Z)$. More precisely, the 
    map ${\cal F} : L^1(\T) \to c_0(\Z)$ is an injective algebra homomorphism. 
\end{cor}
\begin{pf}
    We will show that the kernel of ${\cal F} : L^1(\T) \to c_0(\Z)$ is the zero 
    function, which is equivalent to the fact that ${\cal F}$ is injective. 
    Indeed, note that for $f \in L^1(\T)$, we have $\hat f(n) = 0$ for all $n \in \Z$ 
    if and only if 
    \[ \left( 1 - \frac{|n|}{N+1} \right) \hat f(n) = 0 \] 
    for all $n \in \Z$ and $N \in \N$. This is equivalent to the fact that 
    $\sigma_N f = 0$, and we have $f = \lim_{N\to\infty} \sigma_N f = 0$ by 
    Fej\'er's Theorem. 
\end{pf}

Another consequence is a more constructive proof of the Stone-Weierstrass 
Theorem, which follows immediately from Fej\'er's Theorem because 
given a function $f \in C(\T)$, the Ces\`aro means $(\sigma_N f)_{N=0}^\infty$
are trigonometric polynomials and we have $\|f - \sigma_N f\|_\infty \to 0$ 
by part (2). 

\begin{cor}[Stone-Weierstrass]{cor:5.11}
    The trigonometric polynomials $\Pol(\T)$ are uniformly dense in $C(\T)$. 
\end{cor}

To prove Fej\'er's Theorem, we first introduce the Fej\'er kernel. 

\begin{defn}{defn:5.12}
    The Fej\'er kernel $(K_N)_{N=0}^\infty \subseteq \Pol(\T) \subseteq C(\T)$ is 
    defined to be 
    \[ K_N(t) = \sum_{n=-N}^N \left( 1 - \frac{|n|}{N+1} \right) e^{int}. \] 
\end{defn}

Now, given $f \in L^1(\T)$, we can use the equation we derived above to see that 
\begin{align*}
    \sigma_N f (\theta) 
    &= \sum_{n=-N}^N \left( 1 - \frac{|n|}{N+1} \right) \hat f(n) e^{in\theta} \\ 
    &= \sum_{n=-N}^N \left( 1 - \frac{|n|}{N+1} \right) \frac{1}{2\pi} 
    \int_{-\pi}^\pi f(t) e^{in(\theta - t)}\dd t \\ 
    &= \frac{1}{2\pi} \int_{-\pi}^\pi f(t) \sum_{n=-N}^N \left( 1 - 
    \frac{|n|}{N+1} \right) e^{in(\theta - t)}\dd t \\ 
    &= \frac{1}{2\pi} \int_{-\pi}^\pi f(t) K_N(\theta - t) = (f \star K_N)(\theta). 
\end{align*}
If $(K_N)_{N=0}^\infty$ is a summability kernel, then Fej\'er's Theorem will follow 
immediately from Theorem~\ref{theo:5.5}. This is indeed the case. 
We first introduce an auxiliary kernel, known as the Dirichlet kernel, which 
we will need later on. 

\begin{defn}{defn:5.13}
    The {\bf Dirichlet kernel} $(D_N)_{N=0}^\infty$ is defined to be 
    \[ D_N(\theta) = \sum_{k=-N}^N e^{ik\theta} = 1 + \sum_{k=1}^N 2\cos(k\theta). \] 
\end{defn}

Using the identity $\sin(A+B) - \sin(A-B) = 2\sin(B)\cos(A)$, we find that 
\begin{align*}
    \sin(\theta/2) D_N(\theta) 
    &= \sin(\theta/2) + \sum_{k=1}^N 2\sin(\theta/2) \cos(k\theta) \\ 
    &= \sin(\theta/2) + \sum_{k=1}^N \sin((k+1/2)\theta) - \sin((k-1/2)\theta) \\ 
    &= \sin((N + 1/2)\theta).     
\end{align*}
Thus, we obtain the formula 
\[ D_N(\theta) = \frac{\sin((N + 1/2)\theta)}{\sin(\theta/2)}. \] 
Moreover, it can be seen with a direct computation that 
\[ S_N f(\theta) = \frac{1}{2\pi} \int_{-\pi}^\pi f(t) D_N(\theta - t)\dd t 
= (f \star D_N)(\theta). \] 
Using this fact, we obtain 
\[ f \star K_N = \sigma_N f = \frac{1}{N+1} \sum_{k=0}^N S_k f 
= \frac{1}{N+1} \sum_{k=0}^N f \star D_k. \] 
It follows from this that 
\[ K_N = \frac{1}{N+1} \sum_{k=0}^N D_k. \] 
In other words, $K_N$ is the $N$-th Ces\`aro mean of the $D_k$. Finally, 
we can use this fact to show that the Fej\'er kernel is indeed a 
summability kernel. 

\begin{prop}{prop:5.14} 
    The Fej\'er kernel $(K_N)_{N=0}^\infty$ is a positive and symmetric summability kernel. 
\end{prop}
\begin{pf}
    We first prove a nice formula for $(K_N)_{N=0}^\infty$. 

    {\sc Claim.} For all $N \geq 0$, we have 
    \[ K_N(\theta) = \frac{\sin^2((N+1)\theta/2)}{(N+1)\sin^2(\theta/2)}. \] 
    {\sc Proof of Claim.} Recall that 
    \[ K_N(\theta) = \frac{1}{N+1} \sum_{k=0}^N D_k(\theta) 
    = \frac{1}{N+1} \sum_{k=0}^N \frac{\sin((k+1/2)\theta)}{\sin(\theta/2)}. \] 
    Multiplying both sides by $(N+1) \sin^2(\theta/2)$ and using the  
    identity $\sin(A) \sin(B) = \frac12(\cos(B-A) - \cos(B+A))$, we get 
    \begin{align*}
        (N+1) \sin^2(\theta/2) K_N(\theta) 
        &= \sum_{k=0}^N \sin(\theta/2) \sin((k+1/2)\theta) \\ 
        &= \sum_{k=0}^N \frac12 \cos(k\theta) - \cos((k+1)\theta) \\ 
        &= \frac12(1 - \cos(N+1)\theta) = \sin^2((N+1)\theta/2). 
    \end{align*}
    Rearranging this gives us the result. \hfill $\blacksquare$
    
    With this identity, we can now verify that $(K_N)_{N=0}^\infty$ is a 
    positive and symmetric summability kernel. Just by looking at the 
    formula, we have $K_N(\theta) \geq 0$ and $K_N(\theta) = K_N(-\theta)$. 
    For all $N \geq 0$, it can be shown through a computation that 
    \[ \frac{1}{2\pi} \int_{-\pi}^\pi K_N(t)\dd t = 1. \] 
    Pick $\delta > 0$, and observe that 
    \[ \sup_{\delta \leq |\theta| \leq \pi} K_N(\theta) 
    = \sup_{\delta \leq |\theta| \leq \pi} \frac{1}{N+1} \cdot 
    \frac{\sin^2((N+1)\theta/2)}{\sin^2(\theta/2)} \leq 
    \frac{1}{N+1} \cdot \frac{1}{\sin^2(\delta/2)} \to 0 \] 
    as $N \to \infty$. That is, we have $K_N \to 0$ uniformly away from 
    the interval $[-\delta, \delta]$, which implies that 
    \[ \lim_{N\to\infty} \int_{|\theta|\geq\delta} K_N(\theta)\dd\theta = 0. \] 
    We conclude that $(K_N)_{N=0}^\infty$ is a summability kernel. 
\end{pf}

Can we play the same game with the Dirichlet kernel $(D_N)_{N=0}^\infty$? 
If $(D_N)_{N=0}^\infty$ were a summability kernel, then using the fact that 
$S_N f = f \star D_N$ and applying Theorem~\ref{theo:5.5} would 
be very nice for us. Unfortunately, this is not the case. This can be seen 
from the following theorem, which states that $D_N$ grows like the logarithm. 

\begin{prop}{prop:5.15}
    \begin{enumerate}[(1)]
        \item $D_N$ is symmetric. 
        \item We have $\|D_N\|_\infty = D_N(0) = 2N+1$. 
        \item For all $N \geq 0$, we have 
        \[ \frac{1}{2\pi} \int_{-\pi}^\pi D_N(t)\dd t = 1. \] 
        \item For all $N \geq 0$, we have 
        \[ \|D_N\|_1 \geq \frac{4}{\pi^2} \log(2N+2). \] 
        (The values $L_N = \|D_N\|_1$ are known as the {\bf Lebesgue numbers}; 
        it is known that $L_N \leq 2 + \log N$ and hence $L_N = O(\log N)$.)
    \end{enumerate}
\end{prop}
\begin{pf}
    Property (1) is clear by looking at any of the formulas. Property (2) can be 
    seen by applying the triangle inequality to the formula 
    \[ D_N(\theta) = 1 + \sum_{k=1}^N 2\cos(k\theta), \] 
    and property (3) can be seen by a direct computation. 

    To prove (4), note that $D_N$ is symmetric and $\lvert \sin(t/2) \rvert \leq t/2$, 
    so we obtain 
    \[ \|D_N\|_1 = \frac1\pi \int_0^\pi |D_N(t)|\dd t 
    = \frac1\pi \int_0^\pi \frac{\lvert\sin((N+1/2)t)\rvert}{\lvert\sin(t/2)\rvert}\dd t 
    \geq \frac{2}{\pi} \int_0^\pi \frac{\lvert\sin((N+1/2)t)\rvert}{t}\dd t. \] 
    Next, we break up this integral into $2N+1$ different ones over the intervals 
    $[\frac{k\pi}{2N+1}, \frac{(k+1)\pi}{2N+1}]$. We also note that 
    $\frac1t \geq \frac{2N+1}{(k+1)\pi}$ on each of these intervals, which implies that 
    \begin{align*}
        \|D_N\|_1 
        &\geq \frac{2}{\pi} \int_0^\pi \frac{\lvert\sin((N+1/2)t)\rvert}{t}\dd t \\ 
        &= \frac{2}{\pi} \sum_{k=0}^{2N} \int_{\frac{k\pi}{2N+1}}^{\frac{(k+1)\pi}{2N+1}} 
        \frac{\lvert\sin((N+1/2)t)\rvert}{t}\dd t \\ 
        &\geq \frac{2}{\pi} \sum_{k=0}^{2N} \frac{2N+1}{(k+1)\pi} 
        \int_{\frac{k\pi}{2N+1}}^{\frac{(k+1)\pi}{2N+1}} \lvert\sin((N+1/2)t)\rvert\dd t. 
    \end{align*}
    The period of each of the sine functions in the integrand is equal to 
    $\frac{4\pi}{2N+1}$. This means that all of these integrals are equal to 
    \[ \int_0^{\frac{\pi}{2N+1}} \sin((N+1/2)t)\dd t = \frac{2}{2N+1}. \] 
    It follows that 
    \begin{align*}
        \|D_N\|_1 \geq \frac{2}{\pi} \sum_{k=0}^{2N} \frac{2N+1}{(k+1)\pi} \cdot \frac{2}{2N+1} 
        = \frac{4}{\pi^2} \sum_{k=0}^{2N} \frac{1}{k+1} 
        \geq \frac{4}{\pi^2} \int_1^{2N+2} \frac1x\dd x 
        = \frac{4}{\pi^2} \log(2N+2), 
    \end{align*}
    where the final inequality was from applying the integral test. 
\end{pf}

Let $f \in C(\T)$. We consider the convergence of the Fourier series of $f$ at $0$. 
We have 
\[ S_N f(0) = (f \star D_N)(0) = \frac{1}{2\pi} \int_{-\pi}^\pi f(t) D_N(-t)\dd t 
= \frac{1}{2\pi} \int_{-\pi}^\pi f(t) D_N(t)\dd t, \] 
where the last equality is because $D_N$ is even. This implies that 
\[ |S_N f(0)| = \left| \frac{1}{2\pi} \int_{-\pi}^\pi f(t) D_N(t)\dd t \right| 
\leq \|f\|_\infty \|D_N\|_1 = \|f\|_\infty L_N. \] 
In particular, the map $\varphi_N$ sending $f$ to $S_N f(0)$ is a bounded 
linear functional for all $N \in \N$ with 
\[ \|\varphi_N\|_{C(\T)^*} \leq L_N \sim \log N. \] 
We claim that $L_N = \|\varphi_N\|$. To see this, let $g_N = \text{sign}(D_N) \in L^\infty(\T)$. 
Note that $g_N \notin C(\T)$, but we have 
\[ S_N g_N(0) = \frac{1}{2\pi} \int_{-\pi}^\pi \text{sign}(D_N(t))D_N(t) \dd t
= \frac{1}{2\pi} \int_{-\pi}^\pi |D_N(t)|\dd t = \|D_N\|_1. \] 
Fix $\eps > 0$. We can find a continuous approximation $g \in C(\T)$ such that 
$\|g\|_\infty \leq 1$ and $g = \text{sign}(D_N)$ except on a set of measure 
$< \eps/\|D_N\|_1$ using a similar construction to the one in Theorem~\ref{theo:3.20}.
Since $S_N(g - g_N)(0) \in (-\eps, 0)$, we obtain 
\[ \varphi_N(g) = S_N g(0) = S_N g_N(0) + S_N (g - g_N)(0) 
\geq \|D_N\|_1 - \eps. \] 
This proves the claim. In particular, given any $N \in \N$, we can always find a 
function $g_N \in C(\T)$ such that $\|g_N\|_\infty = 1$ and $S_N g_N(0) \sim \log N$.

\subsection{Failure of Pointwise Convergence for Continuous Functions} \label{subsec:5.4}
We are now ready to discuss the pointwise converge of the Fourier series of a 
continuous function. In the previous section, we showed that 
the family of bounded linear functionals $\varphi_N : C(\T) \to \C$ given by 
\[ \varphi_N(f) = S_N f(0) = (f \star D_N)(0) \] 
is unbounded in $C(\T)^* = B(C(\T), \C)$ with 
$\|\varphi_N\| = \|D_N\|_1 = L_N \to \infty$ as $N \to \infty$. This 
suggests that the pointwise convergence of the Fourier series might not 
behave too well! 

We would like to find a single function $g \in C(\T)$ such that 
$\varphi_N(g) = S_N g(0)$ diverges as $N \to \infty$. This is hard to explicitly
cook up because our previous construction used a continuous approximation for 
$\text{sign}(D_N)$, whose periodicity varies with $N$. 
First, we recall the Baire Category Theorem from PMATH 351. 

\begin{theo}[Baire Category Theorem]{theo:5.16}
    Let $(X, d)$ be a complete metric space. Then the intersection of 
    every countable family of dense open subsets of $X$ is open in $X$. 
    That is, if we have $(V_n)_{n=1}^\infty$ where $V_n \subseteq X$ is 
    open and $\overline{V_n} = X$ for all $n \in \N$, then 
    $\bigcap_{n=1}^\infty V_n$ is a dense $G_\delta$-set. 
\end{theo}

This leads us to the Principle of Uniform Boundedness, which is 
also known as the Banach-Steinhaus Principle. 

\begin{theo}[Principle of Uniform Boundedness]{theo:5.17}
    Let $X$ and $Y$ be Banach spaces. Let $(T_\alpha)_{\alpha\in A} 
    \subseteq B(X, Y)$ be a family of bounded linear maps. Then either: 
    \begin{enumerate}[(1)]
        \item $(T_\alpha)_{\alpha \in A}$ is uniformly bounded (that is, 
        $\sup_{\alpha \in A} \|T_\alpha\| = M < \infty$); or 
        \item there exists a dense $G_\delta$-set $E \subseteq X$ such that 
        $\sup_{\alpha \in A} \|T_\alpha x\| = \infty$ for all $x \in E$. 
    \end{enumerate}
\end{theo}
\begin{pf}
    Define $f : X \to [0, \infty]$ by 
    \[ f(x) = \sup_{\alpha\in A} \|T_\alpha x\|. \] 
    Note that $x \mapsto \|T_\alpha x\|$ is continuous for all $\alpha \in A$. 
    Then $f$ is the supremum of continuous maps, and hence is 
    lower semicontinuous (exercise). This implies that $f^{-1}(\beta, \infty]$ 
    is open for all $\beta \geq 0$. 

    For each $n \in \N$, define the open set $V_n \subseteq X$ by 
    \[ V_n = f^{-1}(n, \infty] = \{x \in X : f(x) > n\}. \] 
    {\sc Case 1.} Suppose that there exists $N \in \N$ such that 
    $V_N$ is not dense in $X$. 
    Then there exists a point $x_0 \in X$ and $r > 0$ such that $B_r(x_0) 
    \cap V_N = \varnothing$. Then $\varphi(x) \leq N$ for all $x \in 
    B_r(x_0)$, and hence $\|T_\alpha x\| \leq N$ for all $x \in B_r(x_0)$
    and $\alpha \in A$. Now, given $z \in B_r(0)$, we have 
    for all $\alpha \in A$ that 
    \[ \|T_\alpha z\| \leq \|T_\alpha (x_0 + z)\| + \|T_\alpha x_0\| 
    \leq N + N = 2N. \] 
    It follows that $\|T_\alpha z\| \leq 2N/r$ for all $z$ with 
    $\|z\| \leq 1$, so $(T_\alpha)_{\alpha \in A}$ is uniformly bounded. 

    {\sc Case 2.} Suppose that $\overline{V_n} = X$ for all $n \in \N$. 
    By the Baire Category Theorem, we have that $E = \bigcap_{n=1}^\infty V_n$ 
    is dense in $X$. In particular, for all $x \in E$, we know by 
    construction that 
    \[ f(x) = \sup_{\alpha \in A} \|T_\alpha x\| > n \] 
    for all $n \in \N$, and thus $f(x) = \infty$. 
\end{pf}

\begin{remark}{remark:5.18}
    In order to check that $(T_\alpha)_{\alpha \in A} \subseteq 
    B(X, Y)$ is uniformly bounded, it suffices to prove that 
    $(T_\alpha x)_{\alpha \in A}$ is uniformly bounded for all $x \in X$. 
    This is because (2) of the Principle of Uniform Boundedness would fail, 
    so we must have (1). 
\end{remark}

We can now apply the Principle of Uniform Boundedness to prove the 
non-convergence of Fourier series of general continuous functions. 

\begin{cor}{cor:5.19}
    There exists a dense $G_\delta$-set $E \subseteq C(\T)$ such that 
    $(S_N f(0))_{N=0}^\infty$ diverges for all $f \in E$. 
\end{cor}
\begin{pf}
    From our previous discussion, we know that the maps 
    $\varphi_N : C(\T) \to \C$ defined by 
    $\varphi_N(f) = S_N f(0)$ are bounded linear functionals satisfying 
    $\|\varphi_N\| = \|D_N\|_1$. Since $\sup_{N\geq 0} \|\varphi_N\| = \infty$, the 
    Banach-Steinhaus Principle implies that 
    \[ \sup_{N\geq 0} |\varphi_N(f)| = \sup_{N\geq 0} |S_N f(0)| = +\infty \] 
    for all $f \in E$, where $E \subseteq C(\T)$ is some dense 
    $G_\delta$-set. 
\end{pf}

Can we find an explicit example witnessing Corollary~\ref{cor:5.19}? This 
is tricky because almost every function $f \in C(\T)$ one can write down 
is piecewise smooth, and the convergence is well understood for those. 

An explicit example is due to Fej\'er. First, we will need a simple lemma. 

\begin{lemma}{lemma:5.20}
    For all $n \in \N$, put $q_n \in C(\T)$ defined by 
    \[ q_n(t) = \sum_{k=1}^n \frac{\sin(kt)}{k}. \] 
    Then $(q_n)_{n=1}^\infty$ is uniformly bounded on $[-\pi, \pi]$. 
\end{lemma}
\begin{pf}
    Note that the naive approach using 
    the triangle inequality will fail because it will yield the harmonic 
    series. The main idea is to verify that $q_n(t) = S_N f(t)$ where 
    \[ f(t) = \begin{cases}
        i(\pi - x), & \text{if } x \in (0, \pi), \\ 
        -i(\pi + x), & \text{if } x \in (-\pi, 0),
    \end{cases} \] 
    and $f/(2i)$ is the ``sawtooth'' function. Observe that $f$ is 
    piecewise continuously differentiable, and has left and right 
    derivatives at its jump discontinuities. This means that 
    $(q_n(t))_{n=1}^\infty$ is bounded for all $t \in [-\pi, \pi]$. 
    It only remains to show that it is bounded uniformly in $t \in [-\pi, \pi]$ 
    and $n \in \N$, which we will achieve on Assignment 5. 
\end{pf}

Next, we define some parameters. For each $j \in \N$, choose 
$\lambda_j > 0$ and integers $n_j < N_j$ satisfying 
\begin{enumerate}[(1)]
    \item $\sum_{j=1}^\infty \lambda_j < \infty$, 
    \item $\lim_{j\to\infty} \lambda_j \log(n_j) = +\infty$, and 
    \item $N_j + n_j < N_{j+1} - n_{j+1}$ for all $j \in \N$. 
\end{enumerate}
For example, we could pick $\lambda_j = 1/j^2$, $n_j = 2^{j^3}$, and 
$N_j = 2^{j^3+1}$ for all $j \in \N$. 

Then for all $n, N \in \N$, define 
\[ p(N, n)(t) = 2ie^{iNt} \sum_{k=1}^n \frac{\sin(kt)}{k}. \] 
In particular, we have $p(N, n)(t) = 2ie^{iNt} q_n(t)$ where $q_n(t)$ is 
uniformly bounded in $n$ and $t$ by Lemma~\ref{lemma:5.20}, so 
there exists $C > 0$ such that $\|q_n\|_\infty \leq C$. Note that 
$p(N, n) \in \Pol(\T)$ whose Fourier coefficients lie in 
$[N - n, N + n] \cap \Z$. Then, set 
\[ f = \sum_{j=1}^\infty \lambda_j p(N_j, n_j). \] 
Observe that 
\[ \sum_{j=1}^\infty \lambda_j \|p(N_j, n_j)\|_\infty 
= 2 \sum_{j=1}^\infty \lambda_j \|q_n\|_\infty \leq 
2C \sum_{j=1}^\infty \lambda_j < \infty \] 
by property (1) above. It follows from the Weierstrass $M$-test that 
$f \in C(\T)$. 

Finally, we claim that $(S_N f(0))_{N=0}^\infty$ diverges. To see 
this, note that the Fourier coefficients of the functions $p(N_j, n_j)$ 
do not overlap. Recall that by property (3), we have $N_j + n_j < 
N_{j+1} - n_{j+1}$, and hence 
\[ [N_j - n_j, N_j + n_j] \cap [N_{j+1} - n_{j+1}, N_{j+1} + n_{j+1}] = 
\varnothing. \] 
Next, we consider $S_{N_j+n_j} f(0) - S_{N_j} f(0)$. This is equal 
to the sum of $\hat f(n)$ from $n = N_j+1$ to $N_j + n_j$, so we get 
\[ S_{N_j+n_j} f(0) - S_{N_j} f(0) = 2i \lambda_j \sum_{k=1}^{n_j} \frac1k. \] 
But by the integral test, we obtain 
\[ \sum_{k=1}^{n_j} \frac1k \geq \int_1^{n_j+1} \frac1x\dd x = 
\log(n_j+1), \]
and thus $|S_{N_j+n_j} f(0) - S_{N_j} f(0)| \geq 2\lambda_j \log(n_j) 
\to +\infty$ by assumption (2). This means $(S_N f(0))_{N=0}^\infty$ 
is not Cauchy, so it does not converge! 

\subsection{Range of the Fourier Transform} \label{subsec:5.5} 
Let ${\cal F} : L^1(\T) \to c_0(\Z)$ denote the Fourier transform 
sending $f \in L^1(\T)$ to its Fourier coefficients $(\hat f(n))_{n\in\Z}$. 
We know that $(c_0(\Z), \|\cdot\|_\infty)$ is a Banach space.
Recall from Assignment 4 that ${\cal F}$ is continuous with 
$\|{\cal F}f\|_\infty \leq \|f\|_1$ and ${\cal F}(fg) = 
{\cal F}(f) {\cal F}(g)$ for all $f, g \in L^1(\T)$. By Corollary~\ref{cor:5.10},
we also know that ${\cal F}$ is an injective algebra homomorphism.

But is ${\cal F}$ surjective? That is, given $(a_n)_{n\in\Z} \in c_0(\Z)$, 
does there exist a function $f \in L^1(\T)$ such that $a_n = \hat f(n)$ for all 
$n \in \Z$? 

\begin{prop}{prop:5.21}
    The Fourier transform ${\cal F} : L^1(\T) \to c_0(\Z)$ has dense range. 
\end{prop}
\begin{pf}
    Recall that $\Pol(\T) \subseteq L^1(\T)$. Observe that 
    \[ {\cal F}(\Pol(\T)) = c_{00}(\Z) = 
    \{a = (a_n)_{n\in\Z} \mid a_n \neq 0 \text{ only finitely often}\} 
    \subseteq c_0(\Z). \] 
    But we have $\overline{c_{00}(\Z)}^{\|\cdot\|_\infty} = c_0(\Z)$ 
    since every sequence in $c_0(\Z)$ can be approximated by 
    members of $c_{00}(\Z)$ with respect to $\|\cdot\|_\infty$. 
\end{pf}

Due to Proposition~\ref{prop:5.21}, it might seem plausible to expect 
that ${\cal F} : L^1(\T) \to c_0(\Z)$ is surjective. However, it turns 
out this is not the case. 

\begin{theo}{theo:5.22}
    Let $f \in L^1(\T)$ be such that $\hat f(n) = -\hat f(n) \geq 0$ for all 
    $n > 0$. Then 
    \[ \sum_{n=1}^\infty \frac{\hat f(n)}{n} < \infty. \] 
\end{theo}
\begin{pf}
    Let $f \in L^1(\T)$ satisfy $\hat f(n) = -\hat f(n) \geq 0$ for all 
    $n > 0$, and put 
    \[ F(t) = \int_{-\pi}^t f(\tau)\dd\tau. \] 
    We claim that $F \in C(\T)$ and $\hat F(n) = \frac{1}{in} \hat f(n)$
    for all $n > 0$. 
    In the case where $f \in C(\T)$, this is easily checked by 
    computing $\hat F(n)$ using integration by parts. For the 
    general case where $f \in L^1(\T)$, take a sequence $(f_i)_{i=1}^\infty$
    such that $f_i \to f$ with respect to $\|\cdot\|_1$, and set 
    \[ F_i(t) = \int_{-\pi}^t f_i(\tau)\dd\tau. \] 
    Then we get 
    \[ |F_i(t) - F(t)| \leq \int_{-\pi}^t |f_i(\tau) - f(\tau)|\dd\tau 
    \leq \|f_i - f\|_1, \] 
    so $\|F_i - F\|_\infty \leq \|f_i - f\| \to 0$. Then 
    $\|F_i - F\|_1 \to 0$, and so 
    \[ \hat F(n) = \lim_{i\to\infty} \hat F_i(n) = \lim_{n\to\infty} 
    \frac{\hat f_i(n)}{in} = \frac{\hat f(n)}{in}, \] 
    which proves the claim. Now, since $F \in C(\T)$, we see from 
    Fej\'er's Theorem (Theorem~\ref{theo:5.9}) that 
    \begin{align*}
        F(0) = \lim_{N\to\infty} (F \star K_N)(0) 
        &= \lim_{N\to\infty} \sum_{n=-N}^N \left( 1 - \frac{|n|}{N+1} \right) \hat F(n) \\ 
        &= \hat F(0) + \lim_{N\to\infty} \sum_{\substack{n=-N \\ n\neq0}}^N 
        \left( 1 - \frac{|n|}{N+1} \right) \frac{\hat f(n)}{in}.
    \end{align*}
    Using our assumption that $\hat f(n) = -\hat f(n) \geq 0$ for all $n > 0$, 
    we have 
    \[ C := i(F(0) - \hat F(0)) = \lim_{N\to\infty} 2 
    \sum_{n=1}^N \left( 1 - \frac{n}{N+1} \right) \frac{\hat f(n)}{n} 
    \geq 0. \] 
    Then for all $M \in \N$, we get 
    \[ \sum_{n=1}^M \frac{\hat f(n)}{n} 
    = \lim_{N\to\infty} \sum_{n=1}^M \left(1 - \frac{n}{N+1} \right)
    \frac{\hat f(n)}{n} \leq \lim_{N\to\infty} \sum_{n=1}^N
    \left(1 - \frac{n}{N+1} \right) \frac{\hat f(n)}{n} \leq \frac{C}{2}. \] 
    Taking $M \to \infty$ gives us the result. 
\end{pf}

Theorem~\ref{theo:5.22} tells us that an odd function has a required decay 
weight on the Fourier coefficients. This leads us to the non-surjectivity 
of the Fourier transform.

\begin{cor}{cor:5.23}
    Let $(a_n)_{n=1}^\infty$ be a sequence satisfying 
    \begin{enumerate}[(1)]
        \item $a_n \geq 0$ for all $n \in \N$, 
        \item $\lim_{n\to\infty} a_n = 0$, and 
        \item $\sum_{n=1}^\infty a_n/n = +\infty$. 
    \end{enumerate}
    Extend this sequence to $(a_n)_{n\in\Z}$ by setting 
    $a_0 = 0$ and $a_{-n} = -a_n$ for all $n > 0$. Then 
    $(a_n)_{n\in\Z} \in c_0(\Z) \setminus {\cal F}(L^1(\T))$, so 
    ${\cal F} : L^1(\T) \to c_0(\Z)$ is not surjective. 
\end{cor}

For example, we see from Corollary~\ref{cor:5.23} that 
\[ \sum_{n=2}^\infty \frac{\sin(nt)}{\log n} \] 
is not the Fourier series of a function in $L^1(\T)$. Indeed, by setting 
$a_n = 1/\log n$ and $a_{-n} = -1/\log n$ for $n \geq 2$, we have 
\[ \sum_{n=2}^\infty \frac{a_n}{n} = \sum_{n=2}^\infty \frac{1}{n\log n} 
= +\infty. \] 
There is another ``abstract'' proof of the non-surjectivity of 
${\cal F} : L^1(\T) \to c_0(\Z)$. For that, one needs some 
more functional analysis typically covered in PMATH 453. 

\begin{theo}{theo:5.24}
    Let $X$ and $Y$ be Banach spaces, and let $T \in B(X, Y)$ be 
    bijective. Then $T^{-1} \in B(Y, X)$.
\end{theo}

We omit the proof of Theorem~\ref{theo:5.24}. Note that $T^{-1}$ always 
exists as a linear map when $T$ is bijective. But the crucial point here is 
that $T^{-1}$ is bounded with $\|T^{-1}y\| \leq C\|y\|$ for all $y \in Y$. 

\begin{cor}{cor:5.25}
    The Fourier transform ${\cal F} : L^1(\T) \to c_0(\Z)$ is not surjective.
\end{cor}
\begin{pf}
    Suppose towards a contradiction that ${\cal F}$ were surjective. 
    Combined with Corollary~\ref{cor:5.10}, this means that 
    ${\cal F}$ is bijective. So there exists $C > 0$ such that 
    $\|{\cal F}^{-1}\| \leq C$.

    For each $N \in \N$, let $d_N = \chi_{[-N,N]\cap \Z} \in c_0(\Z)$. 
    One can compute that ${\cal F}(D_N) = d_N$. Note that 
    $\|d_N\|_\infty = 1$, which implies that 
    \[ \|D_N\|_1 = \|{\cal F}^{-1}(d_N)\|_1 \leq 
    \|{\cal F}^{-1}\| \|d_N\|_\infty \leq C \] 
    for all $N \in \N$. This contradicts the fact that $\|D_N\|_1 
    \sim \log(2N+2)$ from Proposition~\ref{prop:5.15}.
\end{pf}

For general functions $f \in L^1(\T)$, is there a required decay rate on
the Fourier coefficients? 

\begin{theo}{theo:5.26}
    Let $(a_n)_{n\in\Z} \in c_0(\Z)$ satisfy the properties 
    \begin{enumerate}[(1)]
        \item $a_n = a_{-n} \geq 0$ for all $n \geq 0$, 
        \item $a_n \leq \frac12(a_{n+1} + a_{n-1})$ (which can be thought of 
        as convexity).
    \end{enumerate}
    Then there exists a function $f \in L^1(\T)$ such that $f \geq 0$ 
    almost everywhere and $\hat f(n) = a_n$ for all $n \in \Z$. 
\end{theo}

To construct such a sequence in Theorem~\ref{theo:5.26}, take a convex 
function $g : [0, \infty) \to [0, \infty)$ such that $\lim_{x\to\infty}
g(x) = 0$. Then define $a_n = g(n)$. For example, given $c > 0$, one can take 
$g(x) = 1/(x+1)^c$ and hence $a_n = 1/(n+1)^c$. This shows that there is 
no specific decay rate for sequences of Fourier coefficients, because 
we can take $c > 0$ as small as we like. 

To prove Theorem~\ref{theo:5.26}, we first need a simple lemma. 

\begin{lemma}{lemma:5.27}
    Let $(b_n)_{n=1}^\infty$ be a sequence such that $b_1 \geq b_2 \geq 
    \cdots \geq 0$ and $\sum_{n=1}^\infty b_n < \infty$. Then 
    $\lim_{n\to\infty} nb_n = 0$. That is, $b_n$ must decay faster than 
    $1/n$. 
\end{lemma}
\begin{pf}
    Let $\eps > 0$. Then there exists $M \in \N$ such that 
    \[ \sum_{k=M+1}^\infty b_k < \eps. \] 
    Taking $n > M$, we have 
    \[ nb_n = (n-M)b_n + Mb_n = \sum_{k=M+1}^n b_n + Mb_n 
    \leq \sum_{k=M+1}^n b_k + Mb_n < \eps + Mb_n. \] 
    This implies that $\limsup_{n\to\infty} nb_n < \eps$. 
\end{pf}

Now, we are ready to prove the theorem. 

\textsc{Proof of Theorem~\ref{theo:5.26}.} Suppose that $(a_n)_{n\in\Z}$ 
is given as in the theorem. From assumption (2), we have 
\[ a_n - a_{n+1} \leq a_{n-1} - a_n. \] 
Letting $b_n = a_n - a_{n+1}$, we see that $(b_n)_{n\geq0}$ is a 
non-increasing sequence with $\lim_{n\to\infty} b_n = 0$ so that 
$b_n \geq 0$ for all $n \geq 0$. Moreover, observe that 
\[ \sum_{n=0}^\infty b_n = \sum_{n=0}^\infty (a_n - a_{n+1}) 
= \lim_{N\to\infty} (a_0 - a_{N+1}) = a_0. \]  
So by Lemma~\ref{lemma:5.27}, we have $\lim_{n\to\infty} n(a_n - a_{n+1}) = 0$.

{\sc Claim 1.} We have 
\[ \sum_{n=1}^\infty n(a_{n-1} + a_{n+1} - 2a_n) < \infty. \] 
{\sc Proof of Claim 1.} We will show that for all $N \geq 1$, we have 
\begin{equation}\label{eq:5.1} 
    \sum_{n=1}^N n(a_{n-1} + a_{n+1} - 2a_n) = a_0 - a_N - N(a_N - a_{N+1}). \tag{$\star$} 
\end{equation}  
We proceed by induction. For $N = 1$, this is clear. Assume that $\eqref{eq:5.1}$
holds for some $N \geq 1$. Then 
\begin{align*}
    \sum_{n=1}^{N+1} n(a_{n-1} + a_{n+1} - 2a_n) 
    &= a_0 - a_N - Na_N + Na_{N+1} + (N+1)(a_N + a_{N+2} - 2a_{N+1}) \\ 
    &= a_0 - (N+2)a_{N+1} + (N+1)a_{N+2} \\ 
    &= a_0 - a_{N+1} - (N+1)(a_{N+1} - a_{N+2}). 
\end{align*} 
So $\eqref{eq:5.1}$ holds by induction, so in particular, taking $N \to \infty$ 
shows that 
\[ \sum_{n=1}^\infty n(a_{n-1} + a_{n+1} - 2a_n) = a_0 < \infty, \] 
which proves the claim. \hfill$\blacksquare$

Given $j \geq 0$ and $N \geq 1$, observe that $\eqref{eq:5.1}$ also gives us 
the useful identity 
\begin{align*}
    \sum_{n=1}^N n(a_{n-1} + a_{n+1} - 2a_n) 
    &= \sum_{n=1}^j n(a_{n-1} + a_{n+1} - 2a_n) - 
    \sum_{n=j+1}^N n(a_{n-1} + a_{n+1} - 2a_n) \\ 
    &= a_j - a_N - N(a_N - a_{N+1}) + j(a_j - a_{j+1}). 
\end{align*}
Then taking $N \to \infty$ shows that 
\begin{equation}\label{eq:5.2}
    \sum_{n=j+1}^\infty n(a_{n-1} + a_{n+1} - 2a_n) 
    = a_j + j(a_j - a_{j+1}). \tag{$\star\star$}
\end{equation}
To finish off the proof, consider the function $f \in L^1(\T)$ defined by 
\[ f = \sum_{n=1}^\infty n(a_{n-1} + a_{n+1} - 2a_n)K_{n-1}. \] 
We claim that $f$ is well-defined with $f \geq 0$ almost everywhere 
and $\hat f(j) = a_{|j|}$ for all $j \in \Z$. 

Recall that the Fej\'er kernel satisfies $\|K_n\|_1 = 1$ for all $n \geq 0$, 
so by Claim 1, we get  
\[ \sum_{n=1}^\infty n(a_{n-1} + a_{n+1} - 2a_n)\|K_{n-1}\|_1 < \infty. \] 
By the Weierstrass $M$-test, the limit $f \in L^1(\T)$ exists. 
Moreover, we have $f \geq 0$ since $a_{n-1} + a_{n+1} - 2a_n \geq 0$ by 
assumption (2) and $K_{n-1} \geq 0$ for all $n \geq 1$. Finally, 
fix $j \in \Z$. Note that we have 
\[ \widehat{K}_{n-1} = \begin{cases}
    1 - \frac{|j|}{n}, & \text{if } n \geq |j| + 1, \\ 
    0, & \text{otherwise.} 
\end{cases} \] 
From this, we obtain 
\begin{align*}
    \hat f(j) &= \sum_{n=1}^\infty n(a_{n-1} + a_{n+1} - 2a_n) \widehat{K}_{n-1}(j) \\ 
    &= \sum_{n=|j|+1}^\infty n(a_{n-1} + a_{n+1} - 2a_n) \left(1 - \frac{|j|}{n} \right) \\ 
    &= a_{|j|} + |j|(a_{|j|} - a_{|j|+1}) - |j| \sum_{n=|j|+1}^\infty 
    (a_{n-1} + a_{n+1} - 2a_n) \\ 
    &= a_{|j|} + |j|(a_{|j|} - a_{|j|+1}) - |j|(a_{|j|} - a_{|j|+1}) \\ 
    &= a_{|j|}, 
\end{align*}
where the second equality followed from equation $\eqref{eq:5.2}$. 
This completes the proof of the theorem. \qed 
