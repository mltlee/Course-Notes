\section{Banach and Hilbert Spaces}\label{sec:3}

\subsection{Banach and Hilbert Spaces}\label{subsec:3.1}
Functional analysis is the study of normed vector spaces and the continuous 
linear maps between them. Some of the most important examples of complete 
metric spaces are Banach spaces. Recall from PMATH 351 that a metric space is 
complete if every Cauchy sequence is convergent. We first give the definition 
of a normed vector space and a Banach space, and then illustrate these concepts 
with some examples. 

\begin{defn}{defn:3.1}
    A {\bf normed vector space} is a pair $(V, \|\cdot\|)$ where $V$ is a 
    vector space over $\C$ and $\|\cdot\| : V \to [0, \infty)$ is a 
    {\bf norm} on $V$. That is, $\|\cdot\|$ satisfies the following properties: 
    \begin{enumerate}[(1)]
        \item $\|v\| \geq 0$ for all $v \in V$, and $\|v\| = 0$ 
        if and only if $v = 0$; 
        \item $\|\alpha v\| = |\alpha|\|v\|$ for all $\alpha \in C$ and 
        $v \in V$; and 
        \item $\|v + w\| \leq \|v\| + \|w\|$ for all $v, w \in V$. 
    \end{enumerate}
\end{defn}

Note that the property that $\|v\| = 0$ if and only if $v = 0$ says that 
$\|\cdot\|$ is non-degenerate; without this requirement, $\|\cdot\|$ is 
called a {\bf semi-norm}. Moreover, given $v, w \in V$, the above properties 
together give rise to a canonical metric defined by 
\[ d(v, w) = \|v - w\|. \]
Thus, a normed vector space is a special case of a metric space. 

\begin{defn}{defn:3.2}
    A {\bf Banach space} is a complete normed vector space. 
\end{defn}

We begin with the prototypical example of a normed vector space. 

\begin{exmp}{exmp:3.3} 
    Let $v = (v_1, v_2, \dots, v_n) \in \C^n$. We define the $1$-norm on $\C^n$ by 
    \[ \|v\|_1 = |v_1| + |v_2| + \cdots + |v_n|. \] 
    The $2$-norm on $\C^n$ is given by 
    \[ \|v\|_2 = (|v_1|^2 + |v_2|^2 + \cdots + |v_n|^2)^{1/2}. \] 
    More generally, for $1 \leq p < \infty$, the $p$-norm on $\C^n$ is 
    \[ \|v\|_p = (|v_1|^p + |v_2|^p + \cdots + |v_n|^p)^{1/p}. \] 
    The $\infty$-norm or supremum norm on $\C^n$ is 
    \[ \|v\|_\infty = \max\{|v_1|, |v_2|, \dots, |v_n|\}. \] 
    In fact, $\C^n$ equipped with each of these norms is a finite-dimensional 
    Banach space. 
\end{exmp}

We can also consider the continuous analogues for the space $C[a, b]$ 
of continuous functions on $[a, b]$.

\begin{exmp}{exmp:3.4}
    Let $f \in C[a, b]$. The $1$-norm on $C[a, b]$ is 
    \[ \|f\|_1 = \int_a^b |f(x)|\dd x. \] 
    The $2$-norm on $C[a, b]$ is given by 
    \[ \|f\|_2 = \left( \int_a^b |f(x)|^2\dd x \right)^{\!1/2}. \] 
    Finally, the $\infty$-norm on $C[a, b]$ is 
    \[ \|f\|_\infty = \sup_{x\in [a, b]} |f(x)|. \] 
    Note that $C[a, b]$ is a Banach space equipped with the supremum norm, 
    but it is not complete with respect to the $1$-norm and $2$-norm. 
\end{exmp}

Next, we consider inner product spaces and Hilbert spaces, which is
a special case of Banach spaces. 

\begin{defn}{defn:3.5} 
    An {\bf inner product space} is a pair $(H, \langle \cdot, \cdot \rangle)$ 
    where $H$ is a vector space over $\C$ and $\langle \cdot, \cdot \rangle
    : H \times H \to \C$ satisfies the following properties.
    \begin{enumerate}[(1)]
        \item $\langle x, y \rangle = \overline{\langle y, x \rangle}$
        for all $x, y \in H$. 
        \item For fixed $y \in H$, the map $x \mapsto \langle x, y \rangle$
        is a linear functional.
        \item {\bf Positive semidefiniteness:} $\langle x, x \rangle \geq 0$ 
        for all $x \in H$.
        \item {\bf Non-degeneracy:} $\langle x, x \rangle = 0$ if and only if 
        $x = 0$. 
    \end{enumerate} 
\end{defn}

Note that properties (1) and (2) imply that $\langle \cdot, \cdot \rangle$ 
is a {\bf sesquilinear form} on $H$. That is, $\langle \cdot, \cdot \rangle$ 
is linear in the left variable, and conjugately linear in the right 
variable. Moreover, we can define a norm on $H$ by 
\[ \|x\| = \langle x, x \rangle^{1/2}. \] 
We verify this fact in the following proposition. 

\begin{prop}{prop:3.6}
    Let $(H, \langle \cdot, \cdot \rangle)$ be an inner product space. 
    \begin{enumerate}[(1)]
        \item {\bf Cauchy-Schwarz inequality:} For all $x, y \in H$, 
        we have $|\langle x, y \rangle| \leq \|x\|\|y\|$. 
        \item $(H, \langle \cdot, \cdot \rangle)$ is a normed vector space. 
    \end{enumerate}
\end{prop}
\begin{pf}~
    \begin{enumerate}[(1)]

        \item Fix $x, y \in H$. The result is clear if $\|x\| = 0$ or 
        $\|y\| = 0$. Assume now that $x \neq 0$ and $y \neq 0$. For 
        $t \in \R$, define $p(t) = \|x - ty\|^2 \geq 0$. Observe that 
        \[ p(t) = \langle x - ty, x - ty \rangle = \|x\|^2 
        + t^2 \|y\|^2 + 2t\,\text{Re}\langle x, y \rangle. \] 
        By setting $a = \|y\|^2$, $b = 2\,\text{Re}\langle x, y \rangle$, 
        and $c = \|x\|^2$, we have $p(t) = at^2 + bt + c \geq 0$. 
        This means that $p(t)$ has at most one distinct real root, 
        and thus $b^2 - 4ac \leq 0$. In particular, we have 
        \[ 4(\text{Re}\langle x, y \rangle)^2 \leq 4\|x\|^2\|y\|^2, \] 
        or equivalently, $|\text{Re}\langle x, y \rangle| \leq \|x\|\|y\|$. 
        Choose $\alpha \in \C$ such that $|\alpha| = 1$ and 
        $\langle \alpha x, y \rangle = |\langle x, y \rangle|$. Then 
        we deduce that 
        \[ |\langle x, y \rangle| = \langle \alpha x, y \rangle
        = |\text{Re}\langle \alpha x, y \rangle| \leq \|\alpha x\|\|y\| 
        = \|x\|\|y\|. \] 

        \item We will show that $\|\cdot\|$ satisfies the triangle 
        inequality; the other properties are obvious. Given $x, y \in H$, 
        we have 
        \[ \|x + y\|^2 = \|x\|^2 + \|y\|^2 + 2\,\text{Re}\langle x, y \rangle 
        \leq \|x\|^2 + \|y\|^2 + 2\|x\|\|y\| = (\|x\| + \|y\|)^2. \qedhere \] 
    \end{enumerate}
\end{pf}

Since we have shown that $\|x\| = \langle x, x \rangle^{1/2}$ is indeed a 
norm on an inner product space $(H, \langle \cdot, \cdot \rangle)$, it 
makes sense to make the following definition. 

\begin{defn}{defn:3.7}
    A {\bf Hilbert space} is a complete inner product space. 
\end{defn}
