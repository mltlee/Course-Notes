\section{Motivation}\label{sec:1}
This course is a continuation of PMATH 351. It can be thought of 
as a gateway course to many areas of modern analysis, and has 
many applications such as partial differential equations or even
representation of theory of groups.

Even though this course is called ``Lebesgue Integration and Fourier Analysis'',
we will focus more on the latter, since there is a lot of overlap with 
PMATH 451 in terms of measure theory. First, we will begin by giving a 
hand-wavy derivation of the heat equation. We then try to solve the 
corresponding PDE, which will give us some motivation for studying 
Fourier analysis. 

\subsection{Deriving the Heat Equation}\label{subsec:1.1}
Take a ``nice'' region $D \subseteq \R^3$ with volume, such as a sphere, 
cylinder, or cube. Consider a solid body with shape $D$. At time $t = 0$, 
the body is heated to an initial temperature 
\[ u(x, y, z, t)|_{t=0} = u(x, y, z, 0). \] 
This is our initial condition. Moreover, for all $t > 0$, the temperature on 
the boundary $\partial D$ is specified; that is, we know the values of 
$u(x, y, z, t)$ for all $(x, y, z) \in \partial D$. These are the boundary 
conditions. Our goal is to find $u(x, y, z, t)$ for all $t > 0$ and 
$(x, y, z) \in D$. 

To begin, we will derive (using physics) the PDE governing $u$. The 
solid body with shape given by $D$ is assumed to have constant density
$\rho > 0$, and there is a specific heat constant $c > 0$. Then the 
heat content of $D$ is given by
\[ H(t) = \iiint_D c\rho u(x, y, z, t)\dd V. \] 
Behaving as a physicist would, we toss the derivative into the integral 
without question to obtain 
\begin{equation}\label{eq:1.1}
    H'(t) = \iiint_D c\rho u_t(x, y, z, t)\dd V. 
\end{equation}
Now, Fourier's Law states that heat flows from hotter to colder regions at 
a rate proportional to the temperature gradient 
\[ \nabla u(x, y, z) = (u_x, u_y, u_z). \] 
With our nice region $D$, heat only flows in and out through the surface 
$\partial D$. Then Fourier states that there exists $\kappa > 0$ such that 
$H'(t)$ is equal to $\kappa$ multiplied by the flux of $\nabla u$ through 
$\partial D$. That is, we have 
\[ H'(t) = \iint_{\partial D} \kappa(\nabla u) \cdot \dd\vec{S}, \] 
where ${\rm d}\vec{S}$ is the surface differential $\vec{n} \cdot {\rm d}S$. 
Recall that the divergence of a vector field $\vec{F} = (F_1, F_2, F_3)$ is 
defined by $\divergence(\vec{F}) := (F_1)_x + (F_2)_y + (F_3)_z$, and 
Gauss' Divergence Theorem states that 
\[ \iint_{\partial D} \vec F \cdot {\rm d}\vec S = \iiint_D 
\divergence(\vec{F})\dd V. \] 
In our case, we have $\vec F = \kappa\nabla u = (\kappa u_x, \kappa u_y, 
\kappa u_z)$, and hence 
\[ \divergence(\kappa\nabla u) = \kappa(u_{xx} + u_{yy} + u_{zz}) = \kappa\Delta u, \] 
where $\Delta u = u_{xx} + u_{yy} + u_{zz}$ is the Laplacian of $u$. From 
our above equation, this yields 
\begin{equation}\label{eq:1.2}
    H'(t) = \iiint_D \kappa\Delta u\dd V.
\end{equation}
Combining $\eqref{eq:1.1}$ and $\eqref{eq:1.2}$ and doing some rearranging, 
we end up with 
\[ \iiint_D (c\rho u_t - \kappa\Delta u)\dd V = 0. \] 
This holds for all ``nice'' regions, and implies that 
\[ c\rho u_t - \kappa\Delta u = 0. \] 
Setting $K = \kappa/(c\rho) > 0$, we obtain the heat equation 
\[ u_t = K\Delta u. \] 
We now want to solve this with our given initial and boundary conditions. 
As we would expect, this is very difficult! This is a PDE, and solving an
ODE is already a tall order. 

\subsection{Solving the Heat Equation}
For simplicity, we will instead consider the $1$-dimensional heat equation. 
Let us take a thin rod over the interval $[-\pi, \pi]$. Suppose that this 
rod is insulated so that heat only flows in the $x$-direction. In this 
case, the heat equation is given by 
\[ u_t = \kappa u_{xx}. \] 
Our initial condition is $f(x) = u(x, 0)$ where $f$ is piecewise continuous
or even just Riemann integrable, if we want to be more fancy. As for the 
boundary conditions, this really depends on the physical scenario. We give 
some examples of them below.
\begin{itemize}
    \item We may assert that the temperature is $0$ on the endpoints, so
    $u(-\pi, t) = u(\pi, t) = 0$ for all $t \geq 0$. These are called 
    Dirichlet boundary conditions.
    \item We can also assume that the endpoints are insulated, giving us 
    $u_x(-\pi, t) = u_x(\pi, t) = 0$ for all $t \geq 0$. These are called 
    Neumann boundary conditions.
\end{itemize}
For our purposes, we will consider a mixture of these and say that we have 
periodic boundary conditions. To be specific, we want it so that for all 
$t \geq 0$, we have 
\begin{align*}
    u(-\pi, t) &= u(\pi, t), \\ 
    u_x(-\pi, t) &= u_x(\pi, t).
\end{align*}
