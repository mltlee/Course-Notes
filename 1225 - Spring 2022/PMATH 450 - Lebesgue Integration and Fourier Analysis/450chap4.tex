\section{Fourier Analysis on the Circle} \label{sec:4}

\subsection{General Problem of Fourier Analysis} \label{subsec:4.1}
Recall that we defined $\T = \{z \in \C : |z| = 1\}$, which we can 
identify with the interval $[-\pi, \pi]$ via the map $\theta \mapsto 
e^{i\theta}$. We also identify the endpoints so that $\pi = -\pi$. 
We are interested in studying $(L^p(\T), \|\cdot\|_p)$, where
\[ \|f\|_p = \left( \frac{1}{2\pi} \int_{-\pi}^\pi |f(\theta)|^p \dd\theta 
\right)^{\!1/p} \]
for $1 \leq p < \infty$, and $(L^\infty(\T), \|\cdot\|_\infty)$ is defined 
as usual. Note that we have introduced a new normalization here, which is 
merely for cosmetic purposes to ensure that the identity function 
has norm $1$. 

\begin{exercise}{exercise:4.1}
    Show that for all $p \in (1, \infty)$, we have 
    \[ C(\T) \subseteq L^\infty(\T) \subseteq L^p(\T) \subseteq L^1(\T), \] 
    and that each of these inclusions are dense. Moreover, prove that 
    \[ \|f\|_\infty \geq \|f\|_p \geq \|f\|_1. \]  
\end{exercise}

Recall that at the beginning of the course in Section~\ref{subsec:1.3}, we 
discussed Fourier coefficients of continuous functions. We now write down a 
slightly more rigorous definition, allowing for Lebesgue integrable 
functions instead of only continuous ones.

\begin{defn}{defn:4.2}
    Let $f \in L^1(\T)$. For each $n \in \Z$, we define the {\bf $n$-th 
    Fourier coefficient} of $f$ by 
    \[ \hat f(n) = \frac1{2\pi} \int_{-\pi}^\pi f(\theta) e^{-in\theta} \dd\theta. \] 
    The {\bf Fourier series} of $f$ is the series $\sum_{n\in\Z} \hat f(n) 
    e^{in\theta}$, and we write 
    \[ f \sim \sum_{n\in\Z} \hat f(n) e^{in\theta}. \] 
\end{defn}

The general problem of Fourier analysis is to answer the following questions: 
\begin{itemize}
    \item Given $f \in L^p(\T)$ or $f \in C(\T)$, to what extent does the 
    sequence $(\hat f(n))_{n\in\Z}$ determine $f$? 
    \item Given a sequence $(a_n)_{n\in\Z}$, are there necessary or sufficient 
    conditions in which there exists a (possibly unique) function $f \in 
    L^p(\T)$ such that $a_n = \hat f(n)$ for all $n \in \Z$? 
    \item To what extent does the Fourier series represent $f$? Do we have 
    \[ f(\theta) = \sum_{n\in\Z} \hat f(n) e^{in\theta} \] 
    either pointwise, pointwise almost everywhere, or convergence with 
    respect to $\|\cdot\|_p$? 
\end{itemize}

\subsection{$L^2$-convergence of Fourier Series} \label{subsec:4.2}
It turns out that the theory for functions in $L^2(\T)$ is the most 
``beautiful'' and ``clean''. This is because $L^2(\T)$ is a Hilbert space! 
In fact, most of the theory for $L^2(\T)$ follows from general Hilbert 
space theory. 

\begin{defn}{defn:4.3}
    Let $(H, \langle \cdot, \cdot \rangle)$ be a Hilbert space. 
    \begin{itemize}
        \item Let $x, y \in H$. We say that $x$ and $y$ are {\bf orthogonal}
        if $\langle x, y \rangle = 0$, and we write $x \perp y$. 
        Note that this implies that $\|x + y\|^2 = \|x\|^2 + \|y\|^2$.

        Given a subset $S \subseteq H$, the {\bf orthogonal complement} of $S$ 
        is defined to be 
        \[ S^\perp = \{x \in H \mid \langle x, y \rangle = 0 \text{ for all }
        y \in S\}. \] 
        Note that $S^\perp$ is always a closed subspace. 

        \item An {\bf orthonormal system} in $H$ is a family $\{e_n\}_{n\in S} 
        \subseteq H$ such that for all $n, m \in S$, we have 
        \[ \langle e_n, e_m \rangle = \delta_{nm} = \begin{cases} 
            1, & \text{if } n = m, \\ 
            0, & \text{if } n \neq m. 
        \end{cases} \] 
        In particular, this means that $\|e_n\| = 1$ for all $n \in S$ 
        and $e_n \perp e_m$ whenever $n \neq m$. 

        \item An {\bf orthonormal basis} in $H$ is an orthonormal system 
        $\{e_n\}_{n\in S}$ such that $\Span\{e_n \mid n \in S\}$ is 
        dense in $H$ with respect to the norm $\|\cdot\|$ induced by the 
        inner product. 
    \end{itemize}
\end{defn}

\begin{exmp}{exmp:4.4}
    Let $S \neq \varnothing$ be a countable set. Then the space of 
    square-summable sequences, denoted 
    \[ \ell^2(S) = \left\{ a = (a_n)_{n\in S} : a_n \in \C,\; 
    \sum_{n\in S} |a_n|^2 < \infty \right\}, \] 
    is a Hilbert space with the inner product 
    \[ \langle a, b \rangle = \sum_{n\in S} a_n \overline{b_n}. \] 
    The norm induced by the inner product is given by 
    \[ \|a\| = \langle a, a \rangle^{1/2} = \left( \sum_{n\in S} |a_n|^2 
    \right)^{\!1/2}. \] 
    For each $n \in S$, let $\delta_n$ be the sequence with 
    $1$ in the $n$-th entry and $0$ in every other entry. Then 
    it is easy to see that $(\delta_n)_{n\in S}$ is an orthonormal 
    system in $\ell^2(S)$. In fact, its span is dense in $\ell^2(S)$, 
    so it forms an orthonormal basis. 
\end{exmp}

In this section, our interest is $(L^2(\T), \|\cdot\|_2)$. 
Note that the sequence of functions $\{e_n\}_{n\in \Z}$ defined by 
$e_n(\theta) = e^{in\theta}$ is an orthonormal system in $L^2(\T)$. We will 
show later that this is in fact an orthonormal basis for $L^2(\T)$. First, 
we will prove Parseval's Theorem. Recall that a topological space is 
called separable if it contains a countable dense subset. 

\begin{theo}[Parseval's Theorem]{theo:4.5}
    Let $(H, \langle \cdot, \cdot \rangle)$ be a separable Hilbert space. 
    Let $\{e_n\}_{n\in S}$ be an orthonormal system in $H$, and let 
    \[ K = \overline{\Span\{e_n \mid n \in S\}}^{\|\cdot\|} \subseteq H. \] 
    For $n \in S$ and $x \in H$, set $\hat x(n) = \langle x, e_n \rangle$. 
    \begin{enumerate}[(1)]
        \item For any $x \in H$, we have 
        \[ \sum_{n\in S} |\hat x(n)|^2 \leq \|x\|^2 < \infty. \] 
        \item The map $P : H \to H$ defined by 
        \[ Px = \sum_{n\in S} \hat x(n) e_n = \sum_{n\in S} 
        \langle x, e_n \rangle e_n \] 
        satisfies the following properties: 
        \begin{enumerate}[(a)]
            \item For any $P \in B(H, H)$, we have $\|P\| \leq 1$, 
            and $\|P\| = 1$ if $S \neq \varnothing$. 
            \item We have $PH = K$ and $P^2 = P$ (so $P$ is idempotent). 
            \item For all $x \in H$, we have $(x - Px) \perp K$. 
            \item We have $\ker(P) = K^\perp = \{x \in H \mid \langle x, k 
            \rangle = 0 \text{ for all } k \in K\}$. 
        \end{enumerate}
        Note that properties (b) and (c) imply that $P$ is the orthogonal 
        projection of $H$ onto $K$. 
        \item For all $x \in H$, $Px$ is the closest point in $K$ to $x$. 
        That is, we have 
        \[ \|x - Px\| = \inf_{k\in K} \|x - k\|. \] 
    \end{enumerate}
\end{theo}
\begin{pf}
    We first assume that $|S| < \infty$. In this case, we have 
    $K = \Span\{e_n \mid n \in S\}$ (meaning that there is no need to 
    take the closure with respect to $\|\cdot\|$), and for all $x \in H$, 
    the map $P : H \to H$ in the theorem is well-defined and linear. 
    To see that $P^2 = P$, note that 
    \[ P^2x = P(Px) = P\left( \sum_{n\in S} \hat x(n) e_n \right) 
    = \sum_{n\in S} \hat x(n) e_n = Px \] 
    by pushing the map $P$ to each $e_n$. Moreover, it is clear that $PH = K$. 

    Next, we see that 
    \[ \|Px\|^2 = \langle Px, Px \rangle = \sum_{n\in S} |\hat x(n)|^2 
    = \sum_{n\in S} \langle x, e_n \rangle \overline{\langle x, e_n \rangle} 
    = \left\langle x, \sum_{n\in S} \langle x, e_n \rangle e_n \right\rangle 
    = \langle x, Px \rangle \leq \|x\|\|Px\|, \] 
    where the final inequality follows from Cauchy-Schwarz. Rearranging 
    this yields $\|Px\|(\|x\| - \|Px\|) \geq 0$, and hence $\|Px\| \leq \|x\|$. 
    This means that $\|P\| \leq 1$, and $\|Pe_n\| = \|e_n\| = 1$ implies that 
    $\|P\| = 1$ when $S \neq \varnothing$. Our work above also shows that 
    \[ \sum_{n\in S} |\hat x(n)|^2 \leq \|x\|\|Px\| \leq \|x\|^2 < \infty. \] 
    For all $n_0 \in S$, notice that we have 
    \[ \langle x - Px, e_{n_0} \rangle = \hat x(n_0) - \hat x(n_0) = 0, \] 
    so $(x - Px) \perp \Span\{e_n \mid n \in S\} = K$ for all $x \in H$. 
    Note that $x \in \ker(P)$ if and only if $\hat x(n) = \langle x, e_n 
    \rangle = 0$ for all $n \in S$. This is equivalent to saying that 
    $x \perp K$; that is, $x \in K^\perp$. 

    Finally, for any $k \in K$, we have 
    \[ \|x - k\|^2 = \|(x - Px) + (Px - k)\|^2 = \|x - Px\|^2 + 
    \|Px - k\|^2 \geq \|x - Px\|^2, \] 
    which implies that $\|x - Px\| = \inf_{k\in K} \|x - k\|$. 

    Suppose now that $S$ is an infinite set. We leave it as an exercise 
    to show that the separability of $H$ implies that $|S| = |\N|$. 
    Without loss of generality, we will assume that $S = \N$, 
    so our orthonormal system is $\{e_n\}_{n=1}^\infty$. For each 
    $N \in \N$, let $K_N = \Span\{e_1, \dots, e_n\} \subseteq K$. 
    Let $P_N : H \to K_N$ be the corresponding projection 
    \[ P_Nx = \sum_{n=1}^N \hat x(n) e_n. \] 
    For all $N \in \N$, our above work shows that 
    \[ \sum_{n=1}^N |\hat x(n)|^2 \leq \|x\|^2 < \infty, \] 
    and taking $N \to \infty$ gives us 
    \[ \sum_{n=1}^\infty |\hat x(n)|^2 \leq \|x\|^2 < \infty. \] 
    For all $x \in H$, we claim that $(P_Nx)_{N=1}^\infty$ is a Cauchy 
    sequence. Taking $M < N$, we see that 
    \[ \|P_nx - P_mx\|^2 = \left\| \sum_{n=M+1}^N \hat x(n) e_n \right\|^{2}
    = \sum_{n=M+1}^N |\hat x(n)|^2, \] 
    which converges to $0$ as $M, N \to \infty$. Then we see that 
    \[ \lim_{N\to\infty} P_Nx = \sum_{n=1}^\infty \hat x(n) e_n \] 
    exists in $K$. We define the map $P : H \to K$ by 
    \[ Px = \lim_{N\to\infty} P_Nx = \sum_{n=1}^\infty \hat x(n) e_n, \] 
    which is linear. Moreover, we have $\|Px\| = \lim_{N\to\infty} 
    \|P_Nx\| \leq \|x\|$, so $P$ is bounded. 

    Note that $Px = x$ for all $x \in \Span\{e_1, \dots, e_N\}$ and $N \in \N$, 
    so taking $N \to \infty$ gives $Px = x$ for all $x \in K$. Also, we have 
    \[ \langle x - Px, e_n \rangle = \langle x, e_n \rangle - \langle x, 
    e_n \rangle = 0 \] 
    as before, so $(x - Px) \perp \overline{\Span\{e_n \mid n \in \N\}}^{\|\cdot\|} = K$. 
    Since $Px \in K$, this gives us $x \perp Px$, and thus 
    \[ \|x\|^2 = \|x - Px\|^2 + \|Px\|^2. \] 
    The rest of the proof is the same as the $|S| < \infty$ case. 
\end{pf}

\begin{cor}{cor:4.6}
    If $\{e_n\}_{n=1}^\infty$ is an orthonormal basis for a separable 
    Hilbert space $(H, \langle \cdot, \cdot \rangle)$, then for all $x \in H$, 
    we have 
    \[ \|x\|^2 = \sum_{n\in S} |\hat x(n)|^2. \] 
\end{cor}
\begin{pf}
    Note that $(x - Px) \perp K$ for all $x \in H$ where $P$ and $K$ 
    are as in Parseval's Theorem (Theorem~\ref{theo:4.5}). But 
    $\{e_n\}_{n=1}^\infty$ is an orthonormal basis, which means that 
    $H = K$. Then $x - Px = 0$ and hence $\|x - Px\| = 0$
    for all $x \in H$. From this, we obtain 
    \[ \|x\|^2 = \|Px\|^2 + \|x - Px\|^2 = \|Px\|^2 = \sum_{n\in S} |\hat x(n)|^2. 
    \qedhere \] 
\end{pf}

Now, we return to the case $(L^2(\T), \|\cdot\|_2)$ where 
\[ \|f\|_2^2 = \frac{1}{2\pi} \int_{-\pi}^\pi |f(\theta)|^2\dd \theta. \] 
We had an orthonormal system $\{e_n\}_{n\in\Z}$ in $L^2(\T)$ given by 
$e_n(\theta) = e^{in\theta}$. For $f \in L^2(\T)$, we have 
\[ \hat f(n) = \frac{1}{2\pi} \int_{-\pi}^\pi f(\theta) e^{-in\theta}\dd\theta 
= \langle f, e_n \rangle. \] 
We define the $N$-th partial sum of the Fourier series of $f$ as 
\[ S_N f(\theta) = \sum_{n=-N}^N \hat f(n) e^{in\theta}. \] 
In particular, denoting $\Pol(\T) = \Span\{e_n \mid n \in \Z\}$ as the 
``trigonometric functions on the circle'', we have $\Pol(\T) 
\subseteq C(\T) \subseteq L^2(\T)$ and $S_Nf \in \Pol(\T)$ for all 
$N \in \N$. From Parseval's Theorem, we see that 
\[ S_N : L^2(\T) \to \Span\{e_n \mid n \in \Z,\, |n| \leq N\} \subseteq 
\Pol(\T) \subseteq L^2(\T) \] 
is the orthogonal projection of $L^2(\T)$ onto $\Span\{e_n \mid n \in \Z,\, 
|n| \leq N\}$.

\begin{theo}{theo:4.7}
    The family $\{e_n\}_{n\in\Z}$ is an orthonormal basis for $L^2(\T)$. 
    In particular, if $f \in L^2(\T)$, then $f$ is the $\|\cdot\|_2$-limit 
    of its partial sums $(S_Nf)_{N=1}^\infty$. That is, with respect to 
    $\|\cdot\|_2$, we have 
    \[ f = \sum_{n\in\Z} \hat f(n) e_n = \lim_{N\to\infty} S_N f. \] 
\end{theo}
\begin{pf}
    We already know that $\{e_n\}_{n\in\Z}$ is an orthonormal system for 
    $L^2(\T)$. Consider the set 
    \[ K = \overline{\Span\{e_n \mid n \in \Z\}}^{\|\cdot\|_2} 
    = \overline{\Pol(\T)}^{\|\cdot\|_2} \subseteq L^2(\T). \] 
    To show that $\{e_n\}_{n\in\Z}$ is an orthonormal basis for $L^2(\T)$, 
    our aim is to show that $K = L^2(\T)$. 
    
    It suffices to prove that $\Pol(\T) \subseteq C(\T)$ is dense 
    in $C(\T)$ with respect to $\|\cdot\|_\infty$. Indeed, suppose that 
    $\Pol(\T)$ is dense in $C(\T)$ with respect to $\|\cdot\|_\infty$. 
    Let $\eps > 0$ and let $f \in L^2(\T)$. From Corollary~\ref{cor:3.21}, 
    we know that $C[-\pi, \pi]$ is dense in $L^2[-\pi, \pi]$ with respect 
    to $\|\cdot\|_2$. Hence, there exists $g \in C(\T)$ such that 
    $\|g - f\|_2 < \eps$. Choose $p \in \Pol(\T)$ such that $\|p - g\|_\infty 
    < \eps$, which exists by our above assumption. Then we obtain 
    \[ \|p - g\|_2 \leq \|p - g\|_\infty < \eps, \] 
    and the triangle inequality implies that $\|f - p\|_2 < 2\eps$. 

    To show that $\Pol(\T)$ is dense in $C(\T)$ with respect to 
    $\|\cdot\|_\infty$, we recall the Stone-Weierstrass Theorem. 
    It states that if $X$ is a compact metric space and ${\cal A} \subseteq C(X)$
    is an algebra which is self-adjoint, separates points, and contains a 
    non-zero constant function, then ${\cal A}$ is dense in $C(X)$ with respect 
    to $\|\cdot\|_\infty$. Recall that ${\cal A} \subseteq C(X)$ separates 
    points if for any distinct points $x, y \in X$, there exists a 
    function $f \in {\cal A}$ such that $f(x) \neq f(y)$. 

    In our case, we have $X = \T \subseteq \C$ and ${\cal A} = \Pol(\T)$. 
    It is clear that $\Pol(\T)$ is an algebra, and it is self-adjoint 
    since $\overline{e_n} = e_{-n}$. It also contains $1 = e_0 \in \Pol(\T)$. 
    Finally, we see that $\Pol(\T)$ separates points because if 
    $\theta_1, \theta_2 \in \T$ are distinct, then 
    \[ e_1(\theta_1) = e^{i\theta_1} \neq e^{i\theta_2} = e_1(\theta_2). \] 
    It follows from Stone-Weierstrass that $\Pol(\T)$ is dense in $C(\T)$ 
    with respect to $\|\cdot\|_\infty$, which shows that $\{e_n\}_{n\in\Z}$
    is an orthonormal basis for $L^2(\T)$. 

    Now, let $f \in L^2(\T)$. Our goal is to show that $\|f - S_Nf\|_2 \to 0$. 
    Let $\eps > 0$. From our work above, we can find $p \in \Pol(\T)$ 
    such that $\|f - p\|_2 < \eps$. By choosing $N \geq \deg(p)$, we have 
    \begin{align*}
        \|f - S_Nf\| &\leq \|f - p\|_2 + \|p - S_Nf\|_2 \\ 
        &= \|f - p\|_2 + \|S_Np - S_Nf\|_2 \\ 
        &\leq \|f - p\|_2 + \|S_N\| \|f - p\|_2 \\ 
        &< 2\eps 
    \end{align*}
    since $\|S_N\| \leq 1$. It follows that $f = \lim_{N\to\infty} S_Nf$ 
    with respect to $\|\cdot\|_2$. 
\end{pf}

\begin{cor}[Plancharel Identity]{cor:4.8}
    For any $f \in L^2(\T)$, we have 
    \[ \|f\|_2^2 = \sum_{n\in\Z} |\hat f(n)|^2. \] 
\end{cor}
\begin{pf}
    We know that $S_Nf \to f$ with respect to $\|\cdot\|_2$. Applying 
    Parseval's Theorem gives us 
    \[ \|f\|_2 = \lim_{N\to\infty} \|S_N f\|_2 
    = \lim_{N\to\infty} \sum_{n=-N}^N |\hat f(n)|^2 = 
    \sum_{n\in\Z} |\hat f(n)|^2. \qedhere \] 
\end{pf}

\begin{defn}{defn:4.9}
    Let $(H_1, \|\cdot\|_{H_1})$ and $(H_2, \|\cdot\|_{H_2})$ be Hilbert 
    spaces. A {\bf unitary isomorphism} from $H_1$ to $H_2$ is a 
    linear map $U : H_1 \to H_2$ which is surjective and isometric. That is, 
    for all $x \in H_1$, we have 
    \[ \|Ux\|_{H_2} = \|x\|_{H_1}. \] 
    Note that isometric implies injective, so $U$ is a bijection. 
\end{defn}

In particular, Plancharel's Theorem (Corollary~\ref{cor:4.8}) says that 
the function $U : L^2(\T) \to \ell^2(\Z)$ defined by $f \mapsto 
(\hat f(n))_{n\in\Z}$ is a unitary isomorphism. Then for 
$f, g \in L^2(\T)$, we have 
\[ \langle Uf, Ug \rangle = \sum_{n\in\Z} \hat f(n) \overline{\hat g(n)} 
= \lim_{N\to\infty} \sum_{n=-N}^N \hat f(n) \overline{\hat g(n)} 
= \lim_{N\to\infty} \langle S_n f, g \rangle 
= \langle f, g \rangle 
= \frac{1}{2\pi} \int_{-\pi}^\pi f(\theta) \overline{g(\theta)} \dd\theta, \] 
so $U$ preserves inner products. In fact, this property is equivalent to being 
surjective and isometric. 

\subsection{Pointwise and Uniform Convergence of Fourier Series} \label{subsec:4.3}
So far, we know that if $f \in L^2(\T)$, then the Fourier coefficients 
$(\hat f(n))_{n\in\Z}$ allow us to completely recover $f$ in terms of 
$L^2$-convergence. Namely, we have 
\[ S_Nf(\theta) = \sum_{n=-N}^N \hat f(n) e^{in\theta} \] 
where $\lim_{N\to\infty} \|f - S_Nf\|_2 = 0$. Moreover, by abstract 
measure theory, we know that for all $f \in L^2(\T)$, there exists a subsequence 
$(N_k)_{k=1}^\infty$ with $N_k < N_{k+1}$ for all $k \in \N$ such that 
\[ f(\theta) = \lim_{k\to\infty} S_{N_k}f(\theta) = \lim_{k\to\infty} 
\sum_{n=-N_k}^{N_k} \hat f(n) e^{in\theta} \] 
for almost every $\theta \in [-\pi, \pi]$. 

Our fundamental problem is the following: for all $f \in L^2(\T)$, do we have 
$S_Nf \to f$ almost everywhere (that is, without passing to a subsequence)? 
It turns out the answer is yes! However, this result is far beyond the 
scope of the course. We state it to give some context. 

\begin{theo}[Carleson's Theorem]{theo:4.10}
    Let $f \in L^p(\T)$ where $p \in (1, \infty]$. Then $S_n f \to f$ 
    almost everywhere, and $\|S_N f - f\|_p \to 0$. 
\end{theo}

In other words, there is no need to pass to a subsequence to get convergence 
almost everywhere. However, notice that in the above theorem, we had $p > 1$. 
The bad news is that everything falls apart when $p = 1$. 

\begin{theo}[Kolmogorov's Counterexample]{theo:4.11}
    There exists $f \in L^1(\T)$ such that $S_N f(\theta)$ diverges for 
    all $\theta \in [-\pi, \pi]$, and $S_n f \nrightarrow f$ in $\|\cdot\|_1$. 
\end{theo}

This result is again beyond the scope of the course, but it tells us that 
the topic of pointwise convergence is a very delicate and difficult one. 

Now, we will investigate the pointwise convergence of Fourier series for 
more ``regular'' functions. For example, if $f \in C(\T)$ or $f \in C^1(\T)$ 
(so that $f'$ exists and is continuous), is it true that $S_N f(\theta) \to 
f(\theta)$ for all $\theta \in [-\pi, \pi]$? 

\begin{lemma}{lemma:4.12}
    Let $f \in L^2(\T)$, and suppose that $(S_N f)_{N=1}^\infty 
    \subseteq \Pol(\T)$ is Cauchy with respect to $\|\cdot\|_\infty$. 
    Then there exists $g \in C(\T)$ such that $f = g$ almost everywhere 
    and $\|f - S_N f\|_\infty \to 0$. 
\end{lemma}
\begin{pf}
    Since $(S_N f)_{N=1}^\infty \subseteq \Pol(\T) \subseteq C(\T)$ is 
    Cauchy in $\|\cdot\|_\infty$, there exists a function $g \in C(\T)$ such 
    that $\|g - S_N f\|_\infty \to 0$. Moreover, by Theorem~\ref{theo:4.7}, 
    we know that $\|f - S_N f\|_2 \to 0$. To show that $f = g$
    almost everywhere, note that $f, g \in L^2(\T)$, so it is enough 
    to show that $\|g - f\|_2 \to 0$. Indeed, we see that 
    \[ \|g - f\|_2 \leq \|g - S_N f\|_2 + \|S_N f - f\|_2 
    \leq \|g - S_N f\|_\infty + \|S_N f - f\|_2 \to 0. \]
    This means that $f = g$ almost everywhere and $f \in L^\infty(\T)$, 
    with $\|f - S_N f\|_\infty = \|g - S_N f\|_\infty \to 0$. 
\end{pf}

\begin{cor}{cor:4.13}
    Let $f \in C(\T)$, and suppose that its Fourier coefficients satisfy 
    $\hat f \in \ell^1(\Z)$ (that is, $\sum_{n\in\Z} |\hat f(n)| < \infty$). 
    Then $S_N f \to f$ uniformly. 
\end{cor}
\begin{pf}
    Note that $|\hat f(n) e^{in\theta}| = |\hat f(n)|$, so the Fourier series 
    \[ \sum_{n\in\Z} \hat f(n) e^{in\theta} \] 
    converges absolutely with respect to $\|\cdot\|_\infty$. This 
    means that $(S_N f)_{N=1}^\infty$ is Cauchy in $\|\cdot\|_\infty$, 
    and applying Lemma~\ref{lemma:4.12} gives the result. 
\end{pf}

Now, we consider the convergence of a function under some differentiability 
and ``smoothness'' conditions. Here, we do not mean smooth in the sense 
that the function is infinitely differentiable, but more so how nicely 
it behaves. More specifically, we will discuss pointwise convergence 
when the function is continuously differentiable everywhere, 
and when the function is locally Lipschitz at a point. 

\begin{prop}{prop:4.14}
    Let $f \in C^1(\T)$. Then $S_N f \to f$ uniformly. 
\end{prop}
\begin{pf}
    By Corollary~\ref{cor:4.13}, it suffices to show that $\hat f \in 
    \ell^1(\Z)$. We will first compute the Fourier coefficients of 
    $f'$, and see how they relate with the Fourier coefficients of $f$. 
    Observe that 
    \begin{align*}
        \widehat{f'}(n) 
        &= \frac1{2\pi} \int_{-\pi}^\pi f'(\theta) e^{-in\theta}\dd\theta \\
        &= \frac1{2\pi} f(\theta) e^{-in\theta} \bigg|_{-\pi}^\pi 
        + \frac{in}{2\pi} \int_{-\pi}^\pi f(\theta) e^{-in\theta}\dd\theta \\ 
        &= 0 + in \hat f(n).
    \end{align*}
    Note that by assumption, we have $f' \in C(\T) \subseteq L^2(\T)$, which 
    means that $\widehat{f'} \in \ell^2(\Z)$. This tells us that 
    \[ \sum_{n\in\Z} |\widehat{f'}(n)|^2 = \sum_{n\in\Z} n^2 |\hat f(n)|^2 < \infty. \] 
    Applying the Cauchy-Schwarz inequality, we obtain 
    \[ \sum_{n\in\Z} |\hat f(n)| 
    = |\hat f(0)| + \sum_{n\neq 0} |n| |\hat f(n)| \left| \frac1n \right| 
    \leq |\hat f(0)| + \left( \sum_{n\neq 0} |n^2| |\hat f(n)|^2 \right)^{\!1/2}
    \left( \sum_{n\neq 0} \frac{1}{n^2} \right)^{\!1/2} < \infty, \] 
    and thus $\hat f \in \ell^1(\Z)$. The result follows. 
\end{pf}

What happens if $f'$ exists but is not continuous? What if $f$ is not even
differentiable everywhere? The previous arguments will no longer apply. 

\begin{theo}{theo:4.15}
    Let $f \in L^1(\T)$, and assume that $f$ is differentiable at a point 
    $\theta_0 \in [-\pi, \pi]$. Then $S_N f(\theta_0) \to f(\theta_0)$. 
\end{theo}
\begin{pf}
    We will prove something stronger: if $f \in L^1(\T)$ and 
    $f$ is locally Lipschitz at $\theta_0 \in [-\pi, \pi]$, then 
    \[ f(\theta_0) = \lim_{N\to\infty} S_N f(\theta_0). \] 
    Recall that $f$ is {\bf locally Lipschitz} at $\theta_0 \in [-\pi, \pi]$ 
    if there exists $L > 0$ and $\delta > 0$ such that $0 < 
    |\theta - \theta_0| < \delta$ implies $|f(\theta) - f(\theta_0)| 
    \leq L|\theta - \theta_0|$. Note that the existence of $f'(\theta_0)$ 
    implies that $f$ is locally Lipschitz at $\theta_0$, which then 
    implies that $f$ is continuous at $\theta_0$. 

    We will assume without loss of generality that $\theta_0 = 0$ 
    and $f(\theta_0) = 0$. This is because we can replace $f$ with 
    $\tilde f(\theta) = f(\theta + \theta_0) - f(\theta_0)$, and on 
    Assignment 4, we will show that the Fourier coefficients of 
    $\tilde f$ can be computed in terms of those of $f$. 

    Our goal now is to show that 
    \[ S_N f(0) = \sum_{n=-N}^N \hat f(n) \to f(0) = 0. \] 
    We will show this using a clever proof due to P. R. Chernoff. 
    Define the function 
    \[ g(\theta) = \begin{cases}
        f(\theta)/(e^{i\theta} - 1), & \text{if } \theta \in [-\pi, \pi] 
        \setminus \{0\}, \\ 
        0, & \text{if } \theta = 0. 
    \end{cases} \] 
    Note that $g$ is measurable. The value $g(0)$ is irrelevant, but 
    we note that if $f$ is differentiable, then it is natural to take 
    $g(0) = f'(0)/i$ since $g$ would then be continuous at $0$. 

    We claim that $g \in L^1(\T)$. The hard part here is the behaviour 
    of $g$ when $\theta$ is close to $0$. But since $f$ is locally Lipschitz 
    at $0$, there exists $L > 0$ and $\delta > 0$ such that 
    $0 < |\theta| < \delta$ implies $|f(\theta)| \leq L|\theta|$. 
    Then we get 
    \[ \|g\|_1 = \frac1{2\pi} \int_{|\theta|\leq\delta} |g(\theta)|\dd\theta 
    + \frac1{2\pi} \int_{|\theta|>\delta} |g(\theta)|\dd\theta. \] 
    The second integral is finite because $f \in L^1(\T)$ and 
    $1/|e^{i\theta}-1| \leq C$ for some constant $C$ when $|\theta| > \delta$. 
    We shall denote the value of the integral by $C'$. Now, we obtain 
    \begin{align*}
        \|g\|_1 &= \frac1{2\pi} \int_{|\theta|\leq\delta} |g(\theta)|\dd\theta + C' \\ 
        &\leq \frac{1}{2\pi} \int_{|\theta|\leq\delta} \frac{|f(\theta)|}{|e^{i\theta} - 1|}\dd\theta + C' \\ 
        &\leq \frac{1}{2\pi} \int_{|\theta|\leq\delta} \frac{L|\theta|}{|e^{i\theta} - 1|}\dd\theta + C' < \infty 
    \end{align*}
    since $L|\theta|/|e^{i\theta} - 1|$ is continuous at $\theta$, so we indeed 
    have $g \in L^1(\T)$. 

    Note that $f(\theta) = (e^{i\theta} - 1)g(\theta)$ for all $\theta \in 
    [-\pi, \pi]$, so we can write $f = (e_1 - 1)g$. Moreover, we can compute 
    \[ \hat f(n) = \widehat{e_1g}(n) - \hat g(n) = \hat g(n-1) - \hat g(n), \] 
    which implies that 
    \[ S_N f(0) = \sum_{n=-N}^N \hat f(n) = 
    \sum_{n=-N}^N (\hat g(n-1) - \hat g(n)) = \hat g(-N-1) - \hat g(N). \] 
    On Assignment 4, we will show that $g \in L^1(\T)$ implies that 
    $\hat g \in c_0(\Z)$ where $c_0(\Z) \subseteq \ell^\infty(\Z)$ is the 
    closed subspace of sequences $a = (a_n)_{n\in\Z}$ which vanish at infinity.
    That is, $a = (a_n)_{n\in\Z}$ satisfies $\|a\|_\infty = \sup_{n\in\Z} 
    |a_n| < \infty$ and 
    \[ \lim_{n\to\infty} a_n = \lim_{n\to\infty} a_{-n} = 0. \]  
    In particular, we get $\hat g(N) \to 0$ and $\hat g(-N) \to 0$ 
    as $N \to \infty$, and thus $S_N f \to 0$, as desired. 
\end{pf}

\begin{exmp}{exmp:4.16}
    Consider the function $f$ on $\T$ defined by 
    \[ f(\theta) = \begin{cases}
        1, & \text{if } \theta \in [0, \pi], \\ 
        0, & \text{if } \theta \in (-\pi, \theta). 
    \end{cases} \] 
    We will find its Fourier coefficients and investigate the pointwise 
    convergence of the Fourier series of $f$. First, we see that 
    \[ \hat f(0) = \frac{1}{2\pi} \int_{-\pi}^\pi f(\theta)\dd\theta 
    = \frac{1}{2\pi} \int_0^\pi 1\dd\theta = \frac12. \] 
    When $n \neq 0$, we obtain 
    \begin{align*}
        \hat f(n) 
        &= \frac{1}{2\pi} \int_{-\pi}^\pi f(\theta) e^{-in\theta}\dd\theta \\
        &= \frac{1}{2\pi} \int_0^\pi e^{-in\theta}\dd\theta \\
        &= -\frac{1}{2\pi in} e^{-in\theta} \bigg|_0^\pi \\
        &= \frac{1}{2\pi in} (1 - (-1)^n).
    \end{align*}
    In particular, notice that $\hat f(n) = -\hat f(-n)$ for $n \neq 0$. Then 
    the Fourier series of $f$ is given by 
    \begin{align*}
        \sum_{n\in\Z} \hat f(n) e^{in\theta} 
        &= \frac12 + \frac{1}{2\pi i} \sum_{n\in\Z\setminus\{0\}} \frac{1 - (-1)^n}{n} e^{in\theta} \\ 
        &= \frac12 + \frac1\pi \sum_{n=1}^\infty \frac{1 + (-1)^n}n \sin(n\theta) \\ 
        &= \frac12 + \frac1\pi \sum_{k=1}^\infty \frac{2}{2k-1} \sin((2k-1)\theta).  
    \end{align*}
    By Theorem~\ref{theo:4.15}, we know that $S_Nf(\theta) \to f(\theta)$ 
    whenever $\theta \notin \{0, \pi, -\pi\}$ because $f$ is differentiable 
    away from the above points. Taking $\theta = \pi/2$, we obtain 
    \[ 1 = f(\pi/2) = \lim_{n\to\infty} S_Nf(\pi/2) = 
    \frac12 + \frac1\pi \sum_{k=1}^\infty \frac{2}{2k-1} (-1)^{k+1}, \] 
    which yields the familiar formula 
    \[ \frac{\pi}{4} = 1 - \frac13 + \frac15 - \frac17 + \cdots 
    = \sum_{k=0}^\infty \frac{(-1)^{k+1}}{2k+1}. \] 
    However, our interest is the behaviour at $\theta_0 \in \{0, \pi, -\pi\}$, 
    where $f$ has jump discontinuities. At these points, we find that 
    \[ S_N f(\theta_0) = \frac12 = \frac{f(\theta_0^+) + f(\theta_0^-)}{2}, \] 
    where $f(\theta_0^+) = \lim_{\theta\to\theta_0^+} f(\theta)$ 
    and $f(\theta_0^-) = \lim_{\theta\to\theta_0^-} f(\theta)$. So the 
    Fourier series does not converge to $f$ at these values, but is 
    equal to the average of the left and right values! 
\end{exmp}

We will see in the following theorem that this sort of behaviour at jump 
discontinuities is generic. 

\begin{theo}{theo:4.17}
    Let $f \in L^1(\T)$ and $\theta_0 \in [-\pi, \pi]$. Assume that 
    \begin{enumerate}[(1)]
        \item $f(\theta_0^+) = \lim_{\theta\to\theta_0^+} f(\theta)$ 
        and $f(\theta_0^-) = \lim_{\theta\to\theta_0^-} f(\theta)$ exist, and 
        \item $f'(\theta_0^+) = \lim_{h\to0^+} (f(\theta_0+h) - f(\theta_0^+))/h$ 
        and $f'(\theta_0^-) = \lim_{h\to0^-} (f(\theta_0+h) - f(\theta_0^-))/h$ exist. 
    \end{enumerate} 
    Then we have 
    \[ \lim_{N\to\infty} S_N f(\theta_0) = \frac{f(\theta_0^+) + f(\theta_0^-)}{2}. \]
\end{theo}
\begin{pf}
    First, we define the step function 
    \[ h(\theta) = \begin{cases}
        f(\theta_0^-), & \text{if } \theta < \theta_0, \\ 
        f(\theta_0), & \text{if } \theta = \theta_0, \\ 
        f(\theta_0^+), & \text{if } \theta > \theta_0.
    \end{cases} \] 
    Let $\tilde h(\theta) = h(\theta + \theta_0) - f(\theta_0^-)$. 
    By our computations in Example~\ref{exmp:4.16}, we find that 
    \[ S_N \tilde h(\theta) \to \frac{f(\theta_0^+) - f(\theta_0^-)}{2}. \] 
    In particular, it follows that 
    \[ S_N h(\theta_0) = S_N \tilde h(0) + f(\theta_0^-) \to
    \frac{f(\theta_0^+) + f(\theta_0^-)}{2}. \] 
    Now, let $k = f - h$. We see that $k(\theta_0) = f(\theta_0) - h(\theta_0) = 0$ 
    and $k(\theta_0^+) = k(\theta_0^-) = 0$. Moreover, 
    $k'(\theta_0^+) = f'(\theta_0^+)$ and $k'(\theta_0^-) = f'(\theta_0^-)$ 
    both exist, implying that $k$ is locally Lipschitz at $\theta_0$. 
    By Theorem~\ref{theo:4.15}, we have $S_N k(\theta_0) \to k(\theta_0) = 0$ 
    and hence $S_N f(\theta_0) - S_N h(\theta_0) \to 0$. Since 
    $S_N h(\theta_0) \to (f(\theta_0^+) + f(\theta_0^-))/2$, we obtain 
    the desired result.  
\end{pf}

Next, we discuss term-by-term integration of Fourier series. 

\begin{theo}{theo:4.18}
    Let $f \in L^2(\T)$. For $\theta \in [-\pi, \pi]$, define 
    \[ g(\theta) = \int_{-\pi}^\theta f(t)\dd t. \] 
    Then $g \in C[-\pi, \pi]$, and we have 
    \begin{align*}
        g(\theta) = \sum_{n\in\Z} \hat f(n) \int_{-\pi}^\theta  e^{int}\dd t 
        = \hat f(0) (\theta + \pi) + \sum_{n\in\Z\setminus\{0\}} 
        \frac{\hat f(n)}{in} (e^{in\theta} - (-1)^n). 
    \end{align*}
    Moreover, the convergence is uniform. (Note that in general, 
    we do not have $g \in C(\T)$. This is only the case if 
    $g(\pi) = g(-\pi) = 0$, which is equivalent to saying that 
    $\hat f(0) = 0$. In the case that $\hat f(0) = 0$, it turns out 
    that the above is the Fourier series for $g$.)
\end{theo}
\begin{pf}
    Since $f \in L^2(\T)$, we have $S_N f \to f$ with respect to $\|\cdot\|_2$. 
    For $N \in \N$, let 
    \[ g_N(\theta) = \int_{-\pi}^\theta S_N f(t)\dd t = 
    \sum_{n=-N}^N \hat f(n) \int_{-\pi}^\theta e^{int}\dd t. \] 
    Then we obtain 
    \[ |g(\theta) - g_N(\theta)|^2 
    = \left| \int_{-\pi}^\theta (f(t) - S_Nf(t))\dd t \right|^2 
    \leq \left( \int_{-\pi}^\theta |f(t) - S_Nf(t)|\dd t \right)^{\!2} 
    \leq |\theta + \pi| \|f - S_N f\|_2^2  \to 0 \] 
    uniformly in $\theta$, where the last inequality was due 
    to Cauchy-Schwarz. 
\end{pf}

\subsection{Convolution and Summability Kernels} \label{subsec:4.4}
Recall that the general problem of Fourier analysis is to try 
and recover a function $f$ from $L^1(\T)$, $L^2(\T)$, or $C(\T)$ 
given its Fourier transform ${\cal F}f = (\hat f(n))_{n\in\Z}$. 
We already know the following: 
\begin{enumerate}[(1)]
    \item If $f \in L^2(\T)$, then we can recover $f$ as 
    $\lim_{N\to\infty} S_N f$ in the sense of $L^2$-convergence. 
    \item If $f$ is locally Lipschitz at a point $\theta$ (that is, 
    $f$ is ``regular''), then $f(\theta) = \lim_{N\to\infty} S_N f(\theta)$. 
    \item If $f \in C^1(\T)$, then $\|f - S_N f\|_\infty \to 0$. 
\end{enumerate}
But what if we merely assume that $f \in C(\T)$ or $f \in L^1(\T)$? 
In the case of $L^1(\T)$, it might not seem possible to recover 
$f$ from ${\cal F}f$ in light of Kolmogorov's counterexample 
(Theorem~\ref{theo:4.11}). We will see that it actually is possible, 
but we'll have to take an abstract approach to answer these questions. 
We first take a look at the convolution of two functions. 

\begin{prop}{prop:4.19}
    Let $f \in C(\T)$ and $g \in L^1(\T)$. Define a function 
    \[ h(\theta) = \frac{1}{2\pi} \int_{-\pi}^\pi f(\theta - t) g(t)\dd t. \] 
    Note that this is well-defined because the function 
    $t \mapsto f(\theta - t) g(t)$ is in $L^1(\T)$. We have 
    \begin{enumerate}[(1)]
        \item $h \in C(\T)$ and $\|h\|_\infty \leq \|f\|_\infty \|g\|_1$; 
        \item $\|h\|_1 \leq \|f\|_1 \|g\|_1$ and $\|h\|_2 \leq \|f\|_2 \|g\|_1$. 
    \end{enumerate}
    We write $h = f \star g$, and call it the {\bf convolution} of $f$ and $g$. 
\end{prop}
\begin{pf}~
    \begin{enumerate}[(1)]
        \item Since $f \in C(\T)$ and $\T$ is compact, we know that $f$ 
        is uniformly continuous. So for all $\eps > 0$, there exists 
        $\delta > 0$ such that $|s - t| < \delta$ implies 
        $|f(s) - f(t)| < \eps$. Suppose that $\theta, \theta_0 \in \T$ with 
        $|\theta - \theta_0| < \delta$. We get 
        \begin{align*}
            |h(\theta) - h(\theta_0)| 
            &= \frac{1}{2\pi} \left| \int_{-\pi}^\pi (f(\theta - t) - f(\theta_0 - t)) g(t)\dd t \right| \\ 
            &\leq \frac{1}{2\pi} \int_{-\pi}^\pi |f(\theta - t) - f(\theta_0 - t)| |g(t)|\dd t \\ 
            &< \frac{1}{2\pi} \int_{-\pi}^\pi \eps |g(t)|\dd t \leq \eps\|g\|_1, 
        \end{align*}
        where the second last inequality is because $|\theta - t - (\theta_0 - t)| < \delta$. 
        So $h \in C(\T)$, and it is clear that $\|h\|_\infty \leq 
        \|f\|_\infty \|g\|_1$ by a similar computation to above. 

        \item First suppose that $g \in C(\T)$. Then we have 
        \begin{align*}
            \|h\|_1 &= \frac{1}{2\pi} \int_{-\pi}^\pi |h(\theta)|\dd \theta \\ 
            &= \frac{1}{2\pi} \int_{-\pi}^\pi \left| \frac{1}{2\pi} \int_{-\pi}^\pi f(\theta - t) g(t)\dd t \right|\textrm{d}\theta \\ 
            &\leq \frac{1}{(2\pi)^2} \int_{-\pi}^\pi \int_{-\pi}^\pi |f(\theta - t)| |g(t)|\dd t \dd\theta \\ 
            &= \frac{1}{(2\pi)^2} \int_{-\pi}^\pi |g(t)| \left( \int_{-\pi}^\pi |f(\theta - t)|\dd\theta \right)\textrm{d}t
            = \|f\|_1 \|g\|_1,
        \end{align*}
        where the second last equality was obtained by applying Fubini's theorem
        to interchange $\textrm{d}t$ and $\textrm{d}\theta$, 
        noting that this is a Riemann integral since $f, g \in C(\T)$. 
        Similarly, we have 
        \begin{align*}
            \|h\|_2^2 
            &= \frac{1}{2\pi} \int_{-\pi}^\pi \frac{1}{2\pi} \left| \int_{-\pi}^\pi f(\theta - t) g(t)\dd t \right|^2 \textrm{d}\theta \\ 
            &\leq \frac{1}{2\pi} \int_{-\pi}^\pi \left( \frac{1}{2\pi} \int_{-\pi}^\pi |f(\theta - t)| |g(t)|\dd t \right)^{\!2} \textrm{d}\theta \\ 
            &\leq \frac{1}{2\pi} \int_{-\pi}^\pi \left( \frac{1}{2\pi} \int_{-\pi}^\pi |f(\theta - t)|^2 |g(t)|\dd t \right) \left( \frac{1}{2\pi} \int_{-\pi}^\pi |g(t)|\dd t \right)\textrm{d}\theta \\ 
            &= \|g\|_1 \cdot \frac{1}{2\pi} \int_{-\pi}^\pi |f(\theta - t)|^2 |g(t)|\dd t \dd \theta = \|g\|_1 \|f\|_2^2 \|g\|_1, 
        \end{align*}
        where the last equality was due to Fubini, and the second inequality 
        was from applying Cauchy-Schwarz to $|f(\theta - t)||g(t)| 
        = |f(\theta - t)||g(t)|^{1/2} |g(t)|^{1/2}$. 

        Suppose now that $g \in L^1(\T)$. Take a sequence $(g_n)_{n=1}^\infty 
        \subseteq C(\T)$ such that $g_n \to g$ in $\|\cdot\|_1$. 
        Then we know for $p \in \{1, 2\}$ that 
        $\|f \star g_n\|_p \leq \|f\|_p \|g_n\|_1$.
        We claim that $(f \star g_n)_{n=1}^\infty$ is Cauchy in $\|\cdot\|_p$ 
        for $p \in \{1, 2\}$. Indeed, we see that 
        \[ \|f \star g_n - f \star g_m\|_p = \|f \star (g_n - g_m)\|_p 
        \leq \|f\|_p \|g_n - g_m\|_1 \to 0 \] 
        since $g_n \to g$ in $\|\cdot\|_1$, and thus $\|g_n - g_m\|_1 \to 0$. 
        This means that $h = f \star g = \lim_{n\to\infty} f \star g_n$, 
        since the sequence $(f \star g_n)_{n=1}^\infty$ being Cauchy 
        implies that the limit exists, and 
        \[ \|f \star g - f \star g_n\|_p = \|f \star (g - g_n)\|_p 
        \leq \|f\|_p \|g - g_n\|_1 \to 0. \] 
        It follows that 
        $\|f \star g\|_p = \lim_{n\to\infty} \|f \star g_n\|_p 
        \leq \lim_{n\to\infty} \|f\|_p \|g_n\|_1 = \|f\|_p \|g\|_1$. \qedhere  
    \end{enumerate}
\end{pf}

\begin{remark}{remark:4.20}
    If $f, g \in C(\T)$, then $\star : C(\T) \times C(\T) \to C(\T)$ 
    turns $C(\T)$ into a {\bf commutative algebra} over $\C$. That is, 
    for all $f, g, h \in C(\T)$ and $\alpha \in \C$, we have 
    \begin{itemize}
        \item $f \star g = g \star f$; 
        \item $f \star (g + h) = f \star g + f \star h$; 
        \item $(f \star g) \star h = f \star (g \star h)$; and 
        \item $\alpha(f \star g) = (\alpha f) \star g$. 
    \end{itemize}
    It can be seen that $f \star g = g \star f$ by making a substitution 
    $s = \theta - t$. Note that $(C(\T), \star, +)$ does {\bf not} have a unit.
\end{remark}

\begin{theo}{theo:4.21}
    The convolution product $\star$ extends continuously to a multiplication 
    \begin{align*}
        \star : L^1(\T) \times L^1(\T) &\to L^1(\T) \\ 
        (f, g) &\mapsto f \star g = g \star f
    \end{align*}
    satisfying $\|f \star g\|_1 \leq \|f\|_1 \|g\|_1$. Moreover, for $f, g \in L^1(\T)$, 
    the function 
    \[ t \mapsto f(\theta - t) g(t) \] 
    is in $L^1(\T)$ for almost every $\theta \in [-\pi, \pi]$, and for such 
    $\theta$, we have 
    \[ (f \star g)(\theta) = \frac{1}{2\pi} \int_{-\pi}^\pi f(\theta - t)g(t)\dd t. \] 
\end{theo}
\begin{pf}
    It turns out that the integrability of the function 
    \[ t \mapsto f(\theta - t) g(t) \] 
    is highly non-trivial and requires a measure theoretic version of 
    Fubini, which is typically seen in PMATH 451. We will therefore 
    content ourselves to abstractly define $f \star g$ for 
    $f, g \in L^1(\T)$ via a limiting process. 

    Let $(f_n)_{n=1}^\infty, (g_n)_{n=1}^\infty \subseteq C(\T)$ 
    be sequences such that $f_n \to f$ and $g_n \to g$ in $\|\cdot\|_1$. 
    Note that $(f_n \star g_n)_{n=1}^\infty$ is Cauchy in $\|\cdot\|_1$. 
    Indeed, we see that  
    \begin{align*}
        \|f_n \star g_n - f_m \star g_m\|_1 
        &= \|(f_n - f_m) \star g_n + f_m \star (g_n - g_m)\|_1 \\ 
        &\leq \|(f_n - f_m) \star g_n\|_1 + \|f_m \star (g_n - g_m)\|_1 \\ 
        &\leq \|f_n - f_m\|_1 \|g_n\|_1 + \|f_m\|_1 \|g_n - g_m\|_1 \to 0 
    \end{align*}
    as $m, n \to \infty$ because $(f_n)_{n=1}^\infty$ and $(g_n)_{n=1}^\infty$ 
    are Cauchy. We define $f \star g = \lim_{n\to\infty} 
    f_n \star g_n \in L^1(\T)$. To see this is well-defined, we show that 
    $f \star g$ is independent of the choice of $(f_n)_{n=1}^\infty$ 
    and $(g_n)_{n=1}^\infty$. Suppose that $(\tilde f_n)_{n=1}^\infty$ 
    and $(\tilde g_n)_{n=1}^\infty$ also satisfy $\tilde f_n \to f$ and 
    $\tilde g_n \to g$ in $\|\cdot\|_1$. Then we obtain 
    \[ \|f_n \star g_n - \tilde f_n \star \tilde g_n\|_1 
    \leq \|f_n - \tilde f_n\|_1 \|g_n\|_1 + \|\tilde f_n\|_1 
    \|g_n - \tilde g_n\|_1 \to 0, \] 
    and thus $f \star g = \lim_{n\to\infty} f_n \star g_n = 
    \lim_{n\to\infty} \tilde f_n \star \tilde g_n$. 
\end{pf}
